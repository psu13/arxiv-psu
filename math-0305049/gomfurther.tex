
\chapter{Generalized Operads and Multicategories: Further Theory}
\lbl{ch:gom-further}

\chapterquote{%
The last paragraph plays on the postmodern fondness for
`multidimensionality' and `nonlinearity' by inventing a nonexistent field:
`multidimensional (nonlinear) logic'}{%
Sokal and Bricmont~\cite{SB}}


\noindent
This chapter is an assortment of topics in the theory of
$T$-multicategories.  Some are included because they answer natural
questions, some because they connect to established concepts for classical
operads, and some because we will need them later.  The reader who wants to
get on to geometrically interesting structures should skip this chapter and
come back later if necessary; there are no pictures here.

In~\ref{sec:more-monads} we `recall' some categorical language: maps
between monads, mates under adjunctions, and distributive laws.  This
language has nothing intrinsically to do with generalized multicategories,
but it will be efficient to use it in some parts of this chapter
(\ref{sec:alt-app},~\ref{sec:change}) and later chapters.  

The first three proper sections each recast one of the principal
definitions.  In~\ref{sec:alt-app} we find an alternative definition of
generalized multicategory, which amounts to characterizing a
$T$-multicategory $C$ by its free-algebra monad $T_C$ plus one small extra
piece of data.  Both~\ref{sec:alg-fibs} and~\ref{sec:endos} are
alternative ways of defining an algebra for a $T$-multicategory.  The
former generalizes the categorical fact that $\Set$-valued functors can be
described as discrete fibrations, and the latter the fact that a classical
operad-algebra is a map into an endomorphism operad (often taken as the
\emph{definition} of algebra by `working mathematicians' using operads).

The next two sections are also generalizations.  The free $T$-multicategory
on a $T$-graph is discussed in~\ref{sec:free-mti}.  Abstract as this may
seem, it is crucial to the way in which geometry arises spontaneously from
category theory: witness the planar trees of~\ref{sec:om-further} and the
opetopes of Chapter~\ref{ch:opetopic}.  Then we make a definition to
complete the phrase: `plain multicategories are to monoidal categories as
$T$-multicategories are to \ldots'; we call these things `$T$-structured
categories'~(\ref{sec:struc}).

So far all is generalization, but the final two sections are genuinely new.
A choice of cartesian monad $T$ specifies the type of input shape that the
operations in a $T$-multicategory will have, so changing $T$ amounts to a
change of shape.  In~\ref{sec:change} we show how to translate between
different shapes, in other words, how a relation between two monads $T$ and
$T'$ induces a relation between the classes of $T$- and
$T'$-multicategories.  Finally, in~\ref{sec:enr-mtis} we take a short look
at enrichment of generalized multicategories.  As mentioned in the
Introduction, there is much more to this than one might guess, and in
particular there is more than we have room for; this is just a taste.


\section{More on monads}
\lbl{sec:more-monads}%
%
\index{monad|(}
%

To compare approaches to higher categorical structures using different
shapes or of different dimensions we will need a notion of map $(\Eee, T)
\go (\Eee', T')$ between monads.  Actually, there are various such notions:
lax, colax, weak and strict.  Such comparisons lead to functors between
categories of structures, and to discuss adjunctions between such functors
it will be convenient to use the language of mates (an Australian
creation, of course).  It will also make certain later proofs
(\ref{thm:wk-2},~\ref{propn:free-enr}) easier if we know a little about
distributive laws, which are recipes for gluing together two monads $T$,
$T'$ on the same category to give a monad structure on the composite
functor $T'\of T$.

Nothing here is new.  I learned this material from Street~\cite{StrFTM}%
%
\index{Street, Ross}
%
and
Kelly%
%
\index{Kelly, Max}
%
and Street~\cite{KSRE2},%
%
\index{Street, Ross}
%
although I have changed some terminology.
Distributive laws were introduced by Beck~\cite{Beck}.%
%
\index{Beck, Jon}
%
 Where Street
discusses monads in an arbitrary 2-category $\cat{V}$, we stick to the case
$\cat{V} = \CAT$, since that is all we need.

Let $T = (T, \mu, \eta)$ be a monad on a category $\Eee$ and $T' = (T',
\mu', \eta')$ a monad on a category $\Eee'$.  A \demph{lax map of monads}%
%
\index{monad!map of|(}%
%
\lbl{p:lax-map-of-monads}
%
$(\Eee, T) \go (\Eee', T')$ is a functor $Q: \Eee \go \Eee'$ together with
a natural transformation
\[
\begin{diagram}
\Eee		&\rTo^T		&\Eee		\\
\dTo<Q		&\nent\tcs{\psi}&\dTo>Q		\\
\Eee'		&\rTo_{T'}	&\Eee'		\\
\end{diagram}
\]
making the diagrams 
\[
\begin{diagram}[height=2em]
T'^2 Q		&\rTo^{T'\psi}	&T'QT		&\rTo^{\psi T}	&QT^2	\\
\dTo<{\mu' Q}	&		&		&		&\dTo>{Q\mu}\\
T'Q		&		&\rTo_{\psi}	&		&QT	\\
\end{diagram}
%
\diagspace
%
\begin{diagram}[height=2em]
Q		&\rEquals	&Q		\\
\dTo<{\eta' Q}	&		&\dTo>{Q \eta}	\\
T'Q		&\rTo_{\psi}	&QT		\\
\end{diagram}
\]
commute.  If $(\Eee, T) \goby{(\twid{Q}, \twid{\psi})} (\Eee', T')$ is
another lax map of monads then a \demph{transformation}%
%
\index{monad!transformation of}%
%
\index{transformation!monads@of monads}
%
$(Q, \psi) \go
(\twid{Q}, \twid{\psi})$ is a natural transformation $Q \goby{\alpha}
\twid{Q}$ such that 
\[
\begin{diagram}[size=2em]
T'Q		&\rTo^\psi		&QT		\\
\dTo<{T'\alpha}	&			&\dTo>{\alpha T}\\
T'\twid{Q}	&\rTo_{\twid{\psi}}	&\twid{Q}T	\\
\end{diagram}
\]
commutes.  There is a strict 2-category%
%
\index{monad!two-category of@2-category of}
%
$\fcat{Mnd}_\mr{lax}$%
% 
\glo{Mndlax}
% 
whose 0-cells
are pairs \pr{\Eee}{T}, whose 1-cells are lax maps of monad, and whose
2-cells are transformations.

Dually, if $T$ and $T'$ are monads on categories $\Eee$ and $\Eee'$
respectively then a \demph{colax map of monads} $(\Eee, T) \go (\Eee', T')$
consists of a functor $P: \Eee \go \Eee'$ together with a natural
transformation 
\[
\begin{diagram}
\Eee		&\rTo^T		&\Eee		\\
\dTo<P		&\swnt\tcs{\phi}&\dTo>P		\\
\Eee'		&\rTo_{T'}	&\Eee'		\\
\end{diagram}
\]
satisfying axioms dual to those for lax maps.  With the accompanying notion
of \demph{transformation} between colax maps of monads, we obtain another
strict 2-category $\fcat{Mnd}_\mr{colax}$.

A \demph{weak map of monads} is a lax map $(Q, \psi)$ of monads in which
$\psi$ is an isomorphism (or equivalently, a colax map $(P, \phi)$ in which
$\phi$ is an isomorphism), and a \demph{strict map of monads}%
%
\index{monad!map of|)}
%
is a lax map
$(Q, \psi)$ in which $\psi$ is the identity (and so $T'Q = QT$).

Often $\Eee=\Eee'$ and the functor $\Eee \go \Eee'$ is the identity.  A
natural transformation $\psi: T' \go T$ \demph{commutes with the monad
structures}%
%
\index{commutes with monad structures}
%
if $(\id, \psi)$ is a lax map of monads $(\Eee,T) \go
(\Eee,T')$ or, equivalently, a colax map $(\Eee,T') \go (\Eee,T)$.

A crucial property of lax maps of monads is that they induce maps between
categories of algebras:%
%
\index{algebra!monad@for monad!induced functor}
%
$(Q, \psi): (\Eee, T) \go (\Eee', T')$ induces the
functor
\[
\begin{array}{rrcl}
Q_* = (Q, \psi)_*: 	&\Eee^T 	&\go 	&\Eee'^{T'},	\\%
% 
\glo{mndmapindftr}
% 
&&\\
			&\bktdvslob{TX}{h}{X}
			&\goesto&
\left(
\begin{diagram}[height=1.5em,scriptlabels]
T'QX		\\
\dTo>{\psi_X}	\\
QTX		\\
\dTo>{Qh}	\\
QX		\\
\end{diagram}
\right).
\end{array}
\]
(Dually, but less usefully for us, a colax map of monads $(P, \phi): (\Eee,
T) \go (\Eee', T')$ induces a functor $\Eee_T \go \Eee'_{T'}$ between
Kleisli categories.)  In fact we have:
%
\begin{lemma}	\lbl{lemma:lax-map-mnds-is-ftr}
Let $T = (T, \mu, \eta)$ and $T' = (T',\mu',\eta')$ be monads on categories
$\Eee$ and $\Eee'$, respectively.  Then there is a one-to-one
correspondence between lax maps of monads $(\Eee,T) \go (\Eee',T')$ and
pairs $(Q,R)$ of functors such that the square 
%
\begin{equation}	\label{diag:lax-map-alt}
\begin{diagram}[size=2em]
\Eee^T	&\rTo^R	&\Eee'^{T'}	\\
\dTo<{\mr{forgetful}}	&	&
\dTo>{\mr{forgetful}}	\\
\Eee	&\rTo_Q	&\Eee'		\\
\end{diagram}
\end{equation}
%
commutes, with a lax map $(Q,\psi)$ corresponding to the pair $(Q, Q_*)$.
\end{lemma}
% 
\begin{proof}
Given $(Q, \psi)$, the square~\bref{diag:lax-map-alt} with $R=Q_*$ plainly
commutes.  Conversely, take a pair $(Q, R)$ such
that~\bref{diag:lax-map-alt} commutes.  For each $X \in \Eee$, write $(T'QTX
\goby{\chi_X} QTX)$ for the image under $R$ of the free algebra $(T^2 X
\goby{\mu_X} TX)$, then put
\[
\psi_X = (T'QX \goby{T'Q\eta_X} T'QTX \goby{\chi_X} QTX).
\]
This defines a lax map of monads $(Q,\psi)$.  It is easily checked that the
two processes described are mutually inverse.
\done 
\end{proof}

Lax and colax maps can be related as `mates'.  Suppose we have an
adjunction
\[
\begin{diagram}[height=2em]
\cat{D}		\\
\uTo<P \ladj \dTo>Q	\\
\cat{D}'	\\
\end{diagram}
\]
and functors $\cat{D} \goby{T} \cat{D}$, $\cat{D}' \goby{T'} \cat{D}'$.
Then there is a one-to-one correspondence between natural transformations
$\phi$ and natural transformations $\psi$ with domains and codomains as
shown:
\[
\begin{diagram}
\cat{D} 	&\rTo^T		&\cat{D}	\\
\uTo<P		&\nwnt \tcs{\phi}&\uTo>P		\\
\cat{D}'	&\rTo_{T'}	&\cat{D}'	\\
\end{diagram}
% 
\diagspace
% 
\begin{diagram}
\cat{D} 	&\rTo^T			&\cat{D}	\\
\dTo<Q		&\nent \tcs{\psi}	&\dTo>Q		\\
\cat{D}'	&\rTo_{T'}		&\cat{D}'.	\\
\end{diagram}
\]
This is given by 
%
\begin{eqnarray*}
\psi	&=	&
\left(
T'Q \goby{\gamma T'Q} QPT'Q \goby{Q\phi Q} QTPQ \goby{QT\delta} QT
\right),	\\
\phi	&=	&
\left(
PT' \goby{PT'\gamma} PT'QP \goby{P\psi P} PQTP \goby{\delta TP} TP
\right)	
\end{eqnarray*}
%
where $\gamma$ and $\delta$ are the unit and counit of the adjunction.  We
call $\psi$ the \demph{mate}%
%
\index{mate}
%
of $\phi$ and write $\psi = \ovln{\phi}$;%
% 
\glo{mate}
% 
dually, we call $\phi$ the \demph{mate} of $\psi$ and write $\phi =
\ovln{\psi}$.  The world of mates is strictly monogamous: everybody has
exactly one mate, and your mate's mate is you ($\ovln{\ovln{\phi}} = \phi$,
$\ovln{\ovln{\psi}} = \psi$).  All imaginable statements about mates are
true.  In particular, if $T$ and $T'$ have the structure of monads then
$(P, \phi)$ is a colax%
%
\lbl{p:colax-lax-mate}
%
map of monads if and only if $(Q, \ovln{\phi})$ is a lax map of monads.

We finish by showing how to glue monads together.  Given two monads $(S,
\mu, \eta)$ and $(S', \mu', \eta')$ on the same category $\cat{C}$, how can
we give the composite functor $S' \of S$ the structure $(\widehat{\mu},
\widehat{\eta})$ of a monad on $\cat{C}$?  The unit is easy---
%
\begin{equation}	\label{eq:distrib-unit}
\widehat{\eta}
=
\left(
1
\goby{\eta' * \eta}
S' \of S
\right)
\end{equation}
% 
---but for the multiplication we need some extra data---
% 
\begin{equation}	\label{eq:distrib-mult}
\widehat{\mu}
=
\left(
S' \of S \of S' \of S
\goby{S' \sof ? \sof S}
S' \of S' \of S \of S
\goby{\mu' * \mu}
S' \of S
\right)
\end{equation}
% 
---and that is provided by a distributive law.  
%
\begin{defn}	\lbl{defn:distrib-law}
Let $S$ and $S'$ be monads on the same category.  A \demph{distributive
law}%
%
\index{distributive law}
%
$\lambda: S \of S' \go S' \of S$ is a natural transformation such that
$(S', \lambda)$ is a lax map of monads $S \go S$ and $(S, \lambda)$ is a
colax map of monads $S' \go S'$.  
\end{defn}

\begin{lemma}	\lbl{lemma:distrib-gives-monad}
Let $\lambda: S \of S' \go S' \of S$ be a distributive law between monads
$(S, \mu, \eta)$ and $(S', \mu', \eta')$ on a category $\cat{C}$.  Then the
formulas~\bref{eq:distrib-unit} and~\bref{eq:distrib-mult} (with $? =
\lambda$) define a monad structure $(\widehat{\mu}, \widehat{\eta})$ on the
functor $S' \of S$.  If the monads $S$ and $S'$ and the transformation
$\lambda$ are all cartesian then so too is the monad $S' \of S$.  \done
\end{lemma}

The distributive law $\lambda$ determines a lax map of monads $(S',
\lambda): S \go S$, hence a functor $\twid{S}: \cat{C}^S \go \cat{C}^S$.
More incisively, we have the following.
%
\begin{lemma}	\lbl{lemma:distrib-corr}
Let $S$ and $S'$ be monads on a category $\cat{C}$.  Then there is a
one-to-one correspondence between distributive laws $\lambda: S \of S' \go
S' \of S$ and monads $\twid{S}$ on $\cat{C}^S$ such that the forgetful
functor $U: \cat{C}^S \go \cat{C}$ is a strict map of monads $\twid{S} \go
S'$:
\[
\begin{diagram}[size=2em]
\cat{C}^S	&\rTo^{\twid{S}}	&\cat{C}^S	\\
\dTo<U		&			&\dTo>U		\\
\cat{C}		&\rTo_{S'}		&\cat{C}.	\\
\end{diagram}
\]
If the monad $S'$ is cartesian then so too is the monad $\twid{S}$.
\end{lemma}
%
\begin{proof}
Straightforward, using Lemma~\ref{lemma:lax-map-mnds-is-ftr}.  
\done
\end{proof}
%
A distributive law $S\of S' \go S'\of S$ therefore gives two new categories
of algebras, $\cat{C}^{S' \of S}$ and $(\cat{C}^S)^{\twid{S}}$; but they
are isomorphic by general principles of coherence, or more rigorously by
%
\begin{lemma}	\lbl{lemma:distrib-iso-algs}
Let $S$ and $S'$ be monads on a category $\cat{C}$, let $\lambda: S \of S'
\go S' \of S$ be a distributive law, and let $\twid{S}$ be the
corresponding monad on $\cat{C}^S$.  Then there is a canonical natural
transformation
\[
\begin{diagram}
\cat{C}^S	&\rTo^{\twid{S}}	&\cat{C}^S	\\
\dTo<U		&\nent\tcs{\psi}	&\dTo>U		\\
\cat{C}		&\rTo_{S'\of S}		&\cat{C}.	\\
\end{diagram}
\]
This makes $(U, \psi)$ into a lax map of monads $\twid{S} \go S' \of S$,
and the induced functor 
$
(U, \psi)_*: (\cat{C}^S)^{\twid{S}} \go \cat{C}^{S'\of S}
$
is an isomorphism of categories.  
\end{lemma}

\begin{proof}
The transformation $\psi$ is $S'U\epsln$, where $\epsln$ is the counit of
the free-forgetful adjunction $F\ladj U$ for $S$-algebras.  Algebras for
both $\twid{S}$ and $S' \of S$ can be described as triples $(X, h, h')$
where $X \in \cat{C}$, $h$ and $h'$ are respectively $S$-algebra and
$S'$-algebra structures on $X$, and the following diagram commutes:
\[
\begin{diagram}[size=1.5em]
		&		&SS'X		&\rTo^{Sh'}	&SX	\\
		&\ldTo<{\lambda_X}&		&		&	\\
S'SX		&		&		&		&\dTo>h	\\
\dTo<{S'h}	&		&		&		&	\\
S'X		&		&\rTo_{h'}	&		&X.	\\
\end{diagram}
\]
The details of the proof are, again, straightforward.
\done
\end{proof}%
%
\index{monad|)}
%





\section{Multicategories via monads}
\lbl{sec:alt-app}%
%
\index{generalized multicategory!equivalent definitions of|(}
%

Operads are meant to be regarded as algebraic%
%
\index{algebraic theory}
%
theories of a special kind.
Monads are meant to be regarded as algebraic theories of a general kind.
It is therefore natural to ask whether operads can be re-defined as monads
satisfying certain conditions.  

We show here that the answer is nearly `yes': for any cartesian monad $T$
on a cartesian category $\Eee$, a $T$-operad is the same thing as a
cartesian monad $S$ on $\Eee$ together with a cartesian natural
transformation $S \go T$ commuting with the monad structures.  (The
transformation really is necessary, by the results of
Appendix~\ref{app:special-cart}.)  An algebra for a $T$-operad is then just
an algebra for the corresponding monad $S$.  More generally, a version
holds for $T$-multicategories, to be regarded as \emph{many-sorted}
algebraic theories.

We will need to know a little about slice%
%
\index{slice!category}
%
categories.  If $E$ is an object
of a category $\Eee$ then the forgetful functor $U_E: \Eee/E \go \Eee$%
% 
\glo{fgtslice}
% 
creates pullbacks in the following strict sense: if
\[
\begin{diagram}[size=2em]
P	&\rTo	&X	\\
\dTo	&	&\dTo	\\
Y	&\rTo	&Z	\\
\end{diagram}
\]
is a pullback square in $\Eee$ and $(Z \go E)$ is a map in $\Eee$ then the
evident square
\[
\begin{diagram}
\bktdvslob{P}{}{E}	&\rTo	&\bktdvslob{X}{}{E}	\\
\dTo			&	&\dTo			\\
\bktdvslob{Y}{}{E}	&\rTo	&\bktdvslob{Z}{}{E}	\\
\end{diagram}
\]
in $\Eee/E$ is also a pullback.  Hence $U_E$ reflects pullbacks, and if
$\Eee$ is cartesian then the category $\Eee/E$ and the functor $U_E$ are
also cartesian.

\begin{propn}	\lbl{propn:ind-monad-cart}
Let $T$ be a cartesian monad on a cartesian category $\Eee$, and let $C$ be
a $T$-multicategory.  Then the induced monad $T_C$ on $\Eee/C_0$ is
cartesian, and there is a cartesian natural transformation
\[
\begin{diagram}
\Eee/C_0	&\rTo^{T_C}	&\Eee/C_0	\\
\dTo<{U_{C_0}}	&\swnt \tcs{\pi^C}	&\dTo>{U_{C_0}}	\\
\Eee		&\rTo_T		&\Eee		\\	
\end{diagram}
\]
such that $(U_{C_0}, \pi^C)$ is a colax map of monads $(\Eee/C_0, T_C) \go
(\Eee, T)$.  
\end{propn}
%
\begin{proof}
For $X = (X \goby{p} C_0) \in \Eee/C_0$, let $\pi^C_X$ be the map in the
diagram
\[
\begin{slopeydiag}
	&	&T_C X\Spbk&	&	&	&	\\
	&\ldTo<{\pi^C_X}&	&\rdTo	&	&	&	\\
TX	&	&	&	&C_1	&	&	\\
	&\rdTo<{Tp}&	&\ldTo>\dom&	&\rdTo>\cod&	\\
	&	&TC_0	&	&	&	&C_0	\\
\end{slopeydiag}
\]
defining $T_C X$.  It is easy to check that $\pi^C$ is natural and that
$(U_{C_0}, \pi^C)$ forms a colax map of monads.  Using the Pasting
Lemma~(\ref{lemma:pasting}) it is also easy to check that $\pi^C$ is a
cartesian natural transformation, and from this that $T_C$ is a cartesian
monad.  \done
\end{proof}

The proposition says that any $(\Eee,T)$-multicategory $C$ gives rise to a
triple $(E, S, \pi) = (C_0, T_C, \pi^C)$ where
%
\begin{enumerate}
\item 	\lbl{item:E}
$E$ is an object of $\Eee$
\item 
$S$ is a cartesian monad on $\Eee/E$
\item 	\lbl{item:pi}
$\pi$ is a cartesian natural transformation 
\[
\begin{diagram}
\Eee/E		&\rTo^{S}	&\Eee/E		\\
\dTo<{U_E}	&\swnt \tcs{\pi}&\dTo>{U_E}	\\
\Eee		&\rTo_T		&\Eee		\\
\end{diagram}
\]
such that $(U_E, \pi)$ is a colax map of monads $S \go T$.
\end{enumerate}
%
It turns out that this captures exactly what a $T$-multicategory is: every
triple $(E, S, \pi)$ satisfying these three conditions arises from a
$T$-multicategory, and the whole multicategory structure of $C$ can be
recovered from the associated triple $(C_0, T_C, \pi^C)$.  We will prove
this in a moment.  In the case of $T$-operads, assuming that $\Eee$ has a
terminal object $1$, this says that a $T$-operad is a pair $(S, \pi)$ where
$S$ is a cartesian monad on $\Eee$ and $\pi: S \go T$ is a cartesian
natural transformation commuting with the monad structures---just as
promised in the introduction to this section.

In particular, the monad $T_P$ on $\Set$ induced by a plain operad $P$ is
always cartesian, which gives a large class of cartesian monads on $\Set$.
(Not all cartesian monads on $\Set$ are of this type, but all strongly
regular theories are: see Appendix~\ref{app:special-cart}.)  The natural
transformation $\pi^P$ induces the obvious functor from $\fcat{Monoid}$ to
$\Alg(P)$.

We now give the alternative definition of $T$-multicategory.  Notation: if
$E \goby{e} \twid{E}$ is a map in a category $\Eee$ then $e_!$%
% 
\glo{compbang}
% 
is the
functor $\Eee/E \go \Eee/\twid{E}$ given by composition with $e$.

\begin{defn}	\lbl{defn:T-mti-colax}
Let $T$ be a cartesian monad on a cartesian category $\Eee$.  Define a
category $T\hyph\Multicat'$ as follows:
%
\begin{description}
\item[objects] are triples $(E, S, \pi)$ as
in~\bref{item:E}--\bref{item:pi} above
\item[maps] $(E, S, \pi) \go (\twid{E}, \twid{S}, \twid{\pi})$ are pairs
$(e, \phi)$ where $e: E \go \twid{E}$ is a map in $\Eee$ and 
\[
\begin{diagram}
\Eee/E		&\rTo^S			&\Eee/E		\\
\dTo<{e_!}	&\swnt \tcs{\phi}	&\dTo>{e_!}	\\
\Eee/\twid{E}	&\rTo_{\twid{S}}	&\Eee/\twid{E}	\\
\end{diagram}
\]
is a natural transformation such that $(e_!, \phi)$ is a colax map of
monads $S \go \twid{S}$ and 
%
\begin{equation}	\label{eq:colax-compat}
\begin{diagram}
\Eee/E			&\rTo^S			&\Eee/E		\\
\dTo<{e_!}		&\swnt \tcs{\phi}	&\dTo>{e_!}	\\
\Eee/\twid{E}		&\rTo~{\twid{S}}	&\Eee/\twid{E}	\\
\dTo<{U_{\twid{E}}}	&\swnt \tcs{\twid{\pi}}	&\dTo>{U_{\twid{E}}}\\
\Eee			&\rTo_T			&\Eee		\\
\end{diagram}
\diagspace
=
\diagspace
\begin{diagram}
\Eee/E			&\rTo^S			&\Eee/E		\\
\dTo<{U_E}		&\swnt \tcs{\pi}	&\dTo>{U_E}	\\
\Eee			&\rTo_T			&\Eee.		\\
\end{diagram}
\end{equation}
%
\end{description}
\end{defn}
%
The natural transformation $\phi$ in the definition of map is automatically
cartesian: this follows from equation~\bref{eq:colax-compat}, the fact that
$\pi$ and $\twid{\pi}$ are cartesian, the fact that $U_{\twid{E}}$ reflects
pullbacks, and the Pasting Lemma~(\ref{lemma:pasting}).  The functor $e_!$
is also cartesian.

\begin{propn}	\lbl{propn:T-mti-colax}
For any cartesian monad $T$ on a cartesian category $\Eee$, there is an
equivalence of categories
\[
T\hyph\Multicat \eqv T\hyph\Multicat'.
\]
\end{propn}

\begin{proof}
A $T$-multicategory $C$ gives rise to an object $(C_0, T_C, \pi^C)$ of
$T\hyph\Multicat'$, as in Proposition~\ref{propn:ind-monad-cart}.
Conversely, take an object $(E, S, \pi)$ of $T\hyph\Multicat'$, and define
%
\begin{eqnarray*}
C_0			&=	&E,					\\
(C_1 \goby{\cod} C_0)	&=	&S(C_0 \goby{1} C_0),			\\
\dom			&= 	&\pi_{1_{C_0}}: C_1 \go TC_0.
% \dom			&= 	&\pi_{(C_0 \goby{1} C_0)}: C_1 \go TC_0.
\end{eqnarray*}
%
This specifies a $T$-graph $C$, and there is a $T$-multicategory structure
on $C$ given by
\[
\comp = \mu_{1_{C_0}},
\diagspace
\ids = \eta_{1_{C_0}}.
\]
(It takes a little work to see that these make sense.)  The associativity
and identity axioms for the multicategory follow from the coherence axioms
for the colax map of monads $(U_E, \pi)$.  It is straightforward to check
that this extends to an equivalence of categories.
\done
\end{proof}

\begin{cor}	\lbl{cor:T-opd-colax}
Let $T$ be a cartesian monad on a category $\Eee$ with finite
limits.  Then $T\hyph\Operad$ is equivalent to the category
$T\hyph\Operad'$ in which
%
\begin{itemize}
\item an object is a cartesian monad $S$ on $\Eee$ together with a
cartesian natural transformation $S \go T$ commuting with the monad
structures
\item a map $(S, \pi) \go (\twid{S}, \twid{\pi})$ is a cartesian natural
transformation $\phi: S \go \twid{S}$ commuting with the monad structures
and satisfying $\twid{\pi} \of \phi = \pi$.
\end{itemize}
\end{cor}
%
\begin{proof}
Restrict the proof of~\ref{propn:T-mti-colax} to the case $C_0=E=1$.
\done
\end{proof}

Since the monad $T_C$ arising from a $T$-multicategory $C$ is cartesian, it
makes sense to ask what $T_C$-multicategories are.  The answer is simple:
%
\begin{cor}	\lbl{cor:TC-mti}
Let $T$ be a cartesian monad on a cartesian category $\Eee$, and let $C$ be
a $T$-multicategory.  Then there is an equivalence of categories
\[
T_C\hyph\Multicat \eqv T\hyph\Multicat/C,
\]
and if a $T_C$-multicategory $D$ corresponds to a $T$-multicategory
$\ovln{D}$ over $C$ then $\Alg(D) \iso \Alg(\ovln{D})$.
\end{cor}

\begin{proof}
For the first part, it is enough to show that 
\[
T_C\hyph\Multicat' 
\eqv
T\hyph\Multicat'/(C_0, T_C, \pi^C).
\]
An object of the right-hand side is an object $(E, S, \pi)$ of
$T\hyph\Multicat'$ together with a map $E \goby{e} C_0$ and a cartesian
natural transformation $\phi$ such that $(e_!, \phi)$ is a colax map of
monads and $\pi$ is the pasting of $\phi$ and $\pi^C$.  In other words, it
is just an object $(E \goby{e} C_0, S, \phi)$ of $T_C\hyph\Multicat'$.  The
first part follows; and for the second part, $\Alg(D) \iso (\Eee/E)^S \iso
\Alg(\ovln{D})$. \done
\end{proof}

This explains many of the examples in~\ref{sec:om} and~\ref{sec:algs}
(generalized multicategories and their algebras).  Take, for
instance,~\ref{eg:M-times-mti} and~\ref{eg:M-times-alg}, where we fixed a
monoid $M$ and considered $(M\times\dashbk)$-multicategories and their
algebras.  Write $T$ for the free monoid monad on $\Set$, and let $P$ be
the $T$-operad ($=$ plain operad) with $P(1)=M$ and $P(n)=\emptyset$
for $n\neq 1$.  Then $T_P = (M\times\dashbk)$, so by
Corollary~\ref{cor:TC-mti} an $(M\times\dashbk)$-multicategory is a
plain multicategory over $P$.  Evidently a plain multicategory over
$P$ can only have unary arrows, so in fact we have
\[
(M\times\dashbk)\hyph\Multicat \eqv \Cat/M.  
\]
Moreover, the second part of~\ref{cor:TC-mti} tells us that if an
$(M\times\dashbk)$-multicategory corresponds to an object $(C \goby{\phi}
M)$ of $\Cat/M$ then its category of algebras is just $\Alg(C) \eqv
\ftrcat{C}{\Set}$, as claimed in~\ref{eg:M-times-alg}.

This reformulation of $T$-multicategories in terms of monads and colax maps
between them has a dual, using \emph{lax} maps.  As we saw
in~\ref{sec:more-monads}, colax and lax maps can be related using mates.
The adjunctions involved are
%
\begin{equation}	\label{eq:two-adjns}
\begin{diagram}[height=2em]
\Eee		\\
\uTo<{U_E} \ladj \dTo>{(\dashbk\times E)}	\\
\Eee/E,		\\
\end{diagram}
% 
\diagspace \diagspace
% 
\begin{diagram}[height=2em]
\Eee/\twid{E}		\\
\uTo<{e_!} \ladj \dTo>{e^*}	\\
\Eee/E.			\\
\end{diagram}%
%
\index{slice!category}
%
% 
\end{equation}
%
In the first adjunction $E$ is an object of a category $\Eee$ with finite
limits, and the right adjoint to the forgetful functor $U_E$ sends $X \in
\Eee$ to $(X\times E \goby{\mr{pr}_2} E) \in \Eee/E$.  In the second $e: E
\go \twid{E}$ is a map in $\Eee$, and the right adjoint to $e_!$ is the
functor $e^*$ defined by pullback along $e$.  Actually, the first
is just the second in the case $\twid{E} = 1$.

\begin{defn}
Let $T$ be a cartesian monad on a category $\Eee$ with finite limits.  The
category $T\hyph\Multicat''$ is defined as follows:
%
\begin{description}
\item[objects] are triples $(E, S, \rho)$ where $E\in \Eee$, $S$ is a
cartesian monad on $\Eee/E$, and $\rho$ is a cartesian natural
transformation such that $(\dashbk\times E, \rho)$ is a lax map of monads
$T \go S$
\item[maps] $(E, S, \rho) \go (\twid{E}, \twid{S}, \twid{\rho})$ are pairs
$(e, \psi)$ where $e: E \go \twid{E}$ in $\Eee$ and $\psi$ is a cartesian
natural transformation such that $(e^*, \psi)$ is a lax map of monads
$\twid{S} \go S$ and an equation dual to~\bref{eq:colax-compat} in
Definition~\ref{defn:T-mti-colax} holds.
\end{description}
\end{defn}

\begin{propn}	%\lbl{propn:T-mti-lax}
For any cartesian monad $T$ on a category $\Eee$ with finite limits, there
is an isomorphism of categories
\[
T\hyph\Multicat'' \iso T\hyph\Multicat'.
\]
\end{propn}

\begin{proof}
Just take mates%
%
\index{mate}
%
throughout.  All the categories, functors and natural
transformations involved in the adjunctions~\bref{eq:two-adjns} are
cartesian, so under these adjunctions, the mate of a cartesian natural
transformation is also cartesian.  \done
\end{proof}

It follows that $T\hyph\Multicat'' \eqv T\hyph\Multicat$.  So given any
$T$-multicategory $C$, there is a corresponding lax map of monads
$(\dashbk\times C_0, \rho^C): T \go T_C$, and this induces a functor
$\Eee^T \go \Alg(C)$.  For instance, any monoid $M$ yields an algebra
for any plain multicategory $C$; concretely, this algebra $X$ is given
by $X(a)=M$ for all objects $a$ of $C$ and by $X(\theta) = $ ($n$-fold
multiplication) for all $n$-ary maps $\theta$ in $C$.  Similarly, any map
$C \go \twid{C}$ of $T$-multicategories corresponds to a lax map of monads
$T_{\twid{C}} \go T_C$, and this induces the functor $\Alg(\twid{C}) \go
\Alg(C)$ that we constructed directly at the end of
Chapter~\ref{ch:gom-basics}.%
%
\index{generalized multicategory!equivalent definitions of|)}
%



\section{Algebras via fibrations}
\lbl{sec:alg-fibs}%
%
\index{generalized multicategory!algebra for|(}%
%
\index{algebra!generalized multicategory@for generalized multicategory|(}
%


For a small category $C$, the functor category \ftrcat{C}{\Set} is
equivalent to the category of discrete opfibrations over $C$, as we saw
in~\ref{sec:cats}.  Here we extend this result from categories to
generalized multicategories.

By definition (p.~\pageref{p:defn-cl-d-opfib}), a functor $g: D \go C$
between ordinary categories is a discrete opfibration if for each object
$b$ of $D$ and arrow $g(b) \goby{\theta} a$ in $C$, there is a unique arrow
$b \goby{\chi} b'$ in $D$ such that $g(\chi) = \theta$.  Another way of
saying this is that in the diagram
\[
\begin{slopeydiag}
	&		&D_1		&		&	\\
	&\ldTo<{\dom}	&		&\rdTo>{\cod}	&	\\
D_0	&		&\dTo>{g_1}	&		&D_0	\\
\dTo<{g_0}&		&C_1		&		&\dTo>{g_0}\\
	&\ldTo<{\dom}	&		&\rdTo>{\cod}	&	\\
C_0	&		&		&		&C_0	\\
\end{slopeydiag}
\]
depicting $g$, the left-hand `square' is a pullback. 

Generalizing to any cartesian monad $T$ on any cartesian category $\Eee$,
let us call a map $D\goby{g}C$ of $T$-multicategories a \demph{discrete
opfibration}%
%
\index{fibration!discrete opfibration}
%
if the square
%
\begin{diagram}[size=2em,scriptlabels]
TD_{0}		&\lTo^{\dom}	&D_1		\\
\dTo<{Tg_0}	&		&\dTo>{g_1}	\\
TC_{0}		&\lTo^{\dom}	&C_1		\\
\end{diagram}
%
is a pullback.  We obtain, for any $T$-multicategory $C$, the category
$\fcat{DOpfib}(C)$%
% 
\glo{DOpfibgen}
% 
of discrete opfibrations over $C$: an object is a
discrete opfibration with codomain $C$, and a map from $(D \goby{g} C)$ to
$(D' \goby{g'} C)$ is a map $D \goby{f} D'$ of $T$-multicategories such
that $g' \of f = g$.  Such an $f$ is automatically a discrete opfibration too,
by the Pasting Lemma~(\ref{lemma:pasting}).  


\begin{thm}	\lbl{thm:alt-alg}
Let $T$ be a cartesian monad on a cartesian category $\Eee$, and let $C$ be
a $T$-multicategory. Then there is an equivalence of categories
\[
\fcat{DOpfib}(C) \eqv \Alg(C).
\]
\end{thm}


\begin{proof}
A $C$-algebra is an algebra for the monad $T_C$ on $\Eee/C_0$, which sends
an object $X = (X\goby{p} C_0)$ of $\Eee/C_0$ to the boxed composite in the
diagram 
\[
\setlength{\unitlength}{1em}
\begin{picture}(13,9.5)
\cell{0}{0}{bl}{%
\begin{slopeydiag}
	&	&T_C X\Spbk&	&	&	&	\\
	&\ldTo<{\pi_X}&	&\rdTo>{\nu_X}&	&	&	\\
TX	&	&	&	&C_1	&	&	\\
	&\rdTo<{Tp}&	&\ldTo>\dom&	&\rdTo>\cod&	\\
	&	&TC_0	&	&	&	&C_0,	\\
\end{slopeydiag}}
\put(10.9,-1){\line(1,1){2}}
\put(2.4,7.5){\line(1,1){2}}
\put(10.9,-1){\line(-1,1){8.5}}
\put(12.9,1){\line(-1,1){8.5}}
\end{picture}
\]
and therefore consists of an object $(X\goby{p} C_0)$ of $\Eee/C_0$
together with a map $h: T_C X \go X$ over $C_0$, satisfying axioms.

So, given a $C$-algebra $(X \goby{p} C_0, h)$ we obtain a commutative
diagram
\[
\begin{slopeydiag}
	&		&T_C X		&		&	\\
	&\ldTo<{\pi_X}	&		&\rdTo>{h}	&	\\
TX	&		&\dTo>{\nu_X}	&		&X	\\
\dTo<{Tp}&		&C_1		&		&\dTo>{p}\\
	&\ldTo<{\dom}	&		&\rdTo>{\cod}	&	\\
TC_0	&		&		&		&C_0,	\\
\end{slopeydiag}
\]
the left-hand half of which is a pullback square.  The top part of the
diagram defines a $T$-graph $D$, and there is a map $g: D \go C$ defined by
$g_0=p$ and $g_1=\nu_X$.  With some calculation we see that $D$ is
naturally a $T$-multicategory and $g$ a map of $T$-multicategories.
So we have constructed from the $C$-algebra $X$ a discrete
opfibration over $C$.

This defines a functor from $\Alg(C)$ to $\fcat{DOpfib}(C)$, which is
easily checked to be full, faithful and essentially surjective on objects.
\done
\end{proof}

Let us look more closely at the $T$-multicategory $D$ corresponding to a
$C$-algebra $h=(X \goby{p} C_0, h)$.  Generalizing the terminology for
ordinary categories (p.~\pageref{p:defn-caty-elts}), we call $D$ the
\demph{multicategory of elements}%
%
\index{multicategory!elements@of elements}%
%
\index{generalized multicategory!elements@of elements}%
%
\index{slice!generalized multicategory by algebra@of generalized multicategory by algebra}
%
of $h$ and write $D = C/h$.%
% 
\glo{genmtielts}
% 
%
\begin{propn}	\lbl{propn:slice-multicat-fib}
Let $T$ be a cartesian monad on a cartesian category $\Eee$, let $C$ be a
$T$-multicategory, and let $h$ be a $C$-algebra.  Then there is an
isomorphism of categories
\[
\fcat{DOpfib}(C/h) \iso \fcat{DOpfib}(C)/h.
\]
\end{propn}
%
\begin{proof}
Follows from the definition of $\fcat{DOpfib}$, using the observation above
that maps in $\fcat{DOpfib}(C)$ are automatically discrete opfibrations.
\done
\end{proof}
%
Hence $\Alg(C/h) \eqv \Alg(C)/h$, generalizing
Proposition~\ref{propn:pshf-slice}.  In fact, this equivalence is an
isomorphism.  To see this, recall from~\ref{eg:alg-to-multi} the process of
slicing a monad by an algebra: for any monad $S$ on a category \cat{F} and
any $S$-algebra $k$, there is a monad $S/k$ on \cat{D} with the property
that $\cat{F}^{S/k} \iso \cat{F}^S/k$.
%
\begin{propn}	\lbl{propn:slice-multicat}
Let $T$ be a cartesian monad on a cartesian category $\Eee$, let $C$ be a
$T$-multicategory, and let $h$ be a $C$-algebra.  Then there is an
isomorphism of monads $T_{C/h} \iso T_C /h$ and an isomorphism of
categories $\Alg(C/h) \iso \Alg(C)/h$. 
\end{propn}
%
\begin{proof}
The first assertion is easily verified, and the second follows immediately.
\done
\end{proof}

As an example, let $C$ be the terminal%
%
\index{generalized multicategory!terminal}
%
$T$-multicategory $1$.  We have $T_1
\iso T$ and so $\Alg(1) \iso \Alg(T)$~(\ref{eg:alg-terminal}).  Given a
$T$-algebra $h = (TX \goby{h} X)$, we therefore obtain a $T$-multicategory
$1/h$.  Plausibly enough, this is the same as the $T$-multicategory $h^+$
of~\ref{eg:multi-alg}, with graph
\[
TX \ogby{1} TX \goby{h} X.
\]
So by the results above, $T_{h^+} = T_{1/h} \iso T/h$ and $\Alg(h^+) =
\Alg(1/h) \iso \Alg(T)/h$; compare~\ref{eg:alg-to-multi}.

We could also define \demph{opalgebras}%
% 
\lbl{p:opalgebras}%
%
\index{opalgebra}
%
for a $T$-multicategory $C$ as discrete%
%
\index{fibration!discrete}
%
 brations over $C$: that is, as
maps $D \goby{g} C$ of $T$-multicategories such that the right-hand square
\[
\begin{diagram}[size=2em,scriptlabels]
D_1 		&\rTo^\cod	&D_0		\\
\dTo<{g_1}	&		&\dTo>{g_0}	\\
C_1		&\rTo^\cod	&C_0		\\
\end{diagram}
\]
of the diagram depicting $g$ is a pullback.  In the case of ordinary
categories $C$, an opalgebra for $C$ is a functor $C^\op \go \Set$.  In the
case of plain multicategories $C$, an opalgebra is a family $(X(a))_{a \in
C_0}$ of sets together with a function
\[
X(a) \go X(a_1) \times\cdots\times X(a_n)
\]
for each map $a_1, \ldots, a_n \go a$ in $C$, satisfying the obvious
axioms.  Note that these are different from the `coalgebras' or `right
modules' mentioned in~\ref{sec:om-further} and~\ref{sec:mmm}; we do not
discuss them any further.



\section{Algebras via endomorphisms}
\lbl{sec:endos}

An action of a monoid on a set is a homomorphism from the monoid to the
monoid of endomorphisms of the set.  A representation of a Lie algebra is a
homomorphism from it into the Lie algebra of endomorphisms of some vector
space.  An algebra for a plain operad is often defined as a map from it
into the operad of endomorphisms of some set~(\ref{eg:opd-End}).  Here we
show that algebras for generalized multicategories can be described in the
same way, assuming some mild properties of the base category $\Eee$.

First recall from~\ref{eg:alg-multi-End} what happens for plain
multicategories: given any family $(X(a))_{a\in E}$ of sets, there is an
associated plain multicategory $\END(X)$ with object-set $E$ and with
%
\begin{equation}	\label{eq:cl-endo}
(\END(X))(a_1, \ldots, a_n; a)
=
\Set (X(a_1) \times\cdots\times X(a_n), X(a) ),
\end{equation}
%
and if $C$ is a plain multicategory with object-set $C_0$ then a
$C$-algebra amounts to a family $(X(a))_{a\in C_0}$ of sets together with a
map $C \go \END(X)$ of multicategories leaving the objects unchanged.

To extend this to generalized multicategories we need to rephrase the
definition of $\END(X)$.  Let $T$ be the free monoid monad on $\Set$.
Recall that given a set $E$, a family $(X(a))_{a\in E}$ amounts to an
object $X \goby{p} E$ of $\Set/E$.  Then note that $X(a_1)
\times\cdots\times X(a_n)$ is the fibre over $(a_1, \ldots, a_n)$ in the
map $TX \goby{Tp} TE$, or equivalently that it is the fibre over $((a_1,
\ldots, a_n), a)$ in the map $TX \times E \goby{Tp \times 1} TE \times E$.
On the other hand, $X(a)$ is the fibre over $((a_1, \ldots, a_n), a)$ in
the map $TE \times X \goby{1 \times p} TE \times E$.  So if we define
objects
%
\begin{equation}	\label{eq:endo-graphs}
G_1(X) = \bktdvslob{TX \times E}{Tp \times 1}{TE \times E},
\diagspace
G_2(X) = \bktdvslob{TE \times X}{1 \times p}{TE \times E}
\end{equation}
%
of the category $\Set/(TE \times E)$, then~\bref{eq:cl-endo} says that the
underlying $T$-graph of the multicategory $\END(X)$ is the exponential
$G_2(X)^{G_1(X)}$.

It is now clear what the definition of endomorphism%
%
\index{endomorphism!generalized multicategory}%
%
\index{generalized multicategory!endomorphism}
%
multicategory in the
general case must be.  Let $T$ be a cartesian monad on a cartesian category
$\Eee$.  Assume further that $\Eee$ is \demph{locally%
%
\index{locally cartesian closed}
%
cartesian closed}:
for each object $D$ of $\Eee$, the slice category $\Eee/D$ is cartesian
closed (has exponentials).  This is true when $\Eee$ is a presheaf
category, as in the majority of our examples.  (For the definition of
cartesian closed, see, for instance, Mac Lane~\cite[IV.6]{MacCWM}.  For a
more full account, including a proof that presheaf categories are cartesian
closed, see Mac Lane and Moerdijk~\cite[I.6]{MM}; our
Proposition~\ref{propn:pshf-slice} then implies that each slice is
cartesian closed.)  Given $E\in\Eee$, define functors $G_1, G_2: \Eee/E \go
\Eee/(TE \times E)$ by the formulas of~\bref{eq:endo-graphs} above.  

The short story is that for any $X \in \Eee/E$, there is a natural
$T$-multicategory structure on the $T$-graph $\END(X) = G_2(X)^{G_1(X)}$,%
% 
\glo{Endgenmti}
% 
and that if $C$ is any $T$-multicategory then a $C$-algebra amounts to an
object $X$ of $\Eee/C_0$ together with a map $C \go
\END(X)$ of $T$-multicategories fixing the objects.

Here is the long story.  Given $T$ and $\Eee$ as above and $E
\in \Eee$, define a functor
\[
\begin{array}{rrcl}%
% 
\glo{Homgenmti}
% 
\HOM:	&(\Eee/E)^\op \times \Eee/E	&\go	&\Eee/(TE \times E)	\\
	&(X, Y)				&\goesto&G_2(Y)^{G_1(X)}.	
\end{array}
\]
% where the image is an exponential in the category $\Eee/(TE \times E)$.
Consider also the functor
\[
\Eee/(TE \times E) \times \Eee/E
\go
\Eee/E
\]
sending a pair
\[
C = 
\left(
\begin{diagram}[size=1.7em,noPS]
	&	&C_1	&	&	\\
	&\ldTo<\dom&	&\rdTo>\cod&	\\
TE	&	&	&	&E	\\
\end{diagram}
\right),
\diagspace
X =
\bktdvslob{X}{p}{E}
\]
to the boxed diagonal in the pullback diagram
\[
\setlength{\unitlength}{1em}
\begin{picture}(13,9.5)
\cell{0}{0}{bl}{%
\begin{slopeydiag}
	&	&C\of X \Spbk &	&	&	&	\\
	&\ldTo<{\pi_X}&	&\rdTo>{\nu_X}&	&	&	\\
TX	&	&	&	&C_1	&	&	\\
	&\rdTo<{Tp}&	&\ldTo>\dom&	&\rdTo>\cod&	\\
	&	&TE	&	&	&	&E.	\\
\end{slopeydiag}}
\put(10.9,-1){\line(1,1){2}}
\put(2.4,7.5){\line(1,1){2}}
\put(10.9,-1){\line(-1,1){8.5}}
\put(12.9,1){\line(-1,1){8.5}}
\end{picture}
\]
This functor is written more shortly as $(C, X) \goesto C\of X$; of course,
when $C$ has the structure of a $T$-multicategory, we usually write $C\of
\dashbk$ as $T_C$.

\begin{propn}	\lbl{propn:endo-hom-adjn}
Let $T$ be a cartesian monad on a cartesian, locally cartesian closed
category $\Eee$, and let $E$ be an object of $\Eee$.  Then there is an
isomorphism
% 
\begin{equation}	\label{eq:Hom-adjn}
\frac{\Eee}{TE \times E} (C, \HOM(X,Y))
\ \iso\ 
\frac{\Eee}{E} (C\of X, Y)
% (\Eee/(TE \times E)) (C, \HOM(X,Y))
% \iso
% (\Eee/E) (C\of X, Y)
\end{equation}
% 
natural in $C \in \Eee/(TE \times E)$ and $X, Y \in \Eee/E$.
\end{propn}


\begin{proof}
Write $X = (X \goby{p} E)$ and $Y = (Y \goby{q} E)$.  Product in $\Eee/(TE
\times E)$ is pullback over $TE \times E$ in $\Eee$, so the left-hand side
of~\bref{eq:Hom-adjn} is naturally isomorphic to
\[
\frac{\Eee}{TE \times E} (C \times_{TE \times E} G_1(X), G_2(Y)),
\]
where $C \times_{TE \times E} G_1(X) \go TE \times E$ is the diagonal of
the pullback square
\[
\begin{diagram}[size=2em]
\SEpbk C \times_{TE \times E} G_1(X) 	&\rTo	&C_1			\\
\dTo				&		&\dTo>{(\dom,\cod)}	\\
TX \times E			&\rTo_{Tp\times 1}	&TE \times E.	\\
\end{diagram}
\]
But by an easy calculation, we also have a pullback square
\[
\begin{diagram}[size=2em]
\SEpbk C\of X		&\rTo^{\nu_X}		&C_1			\\
\dTo<{(\pi_X, \cod\of \nu_X)}&			&\dTo>{(\dom,\cod)}	\\
TX \times E		&\rTo_{Tp\times 1}	&TE \times E,		\\
\end{diagram}
\]
so in fact an element of the left-hand side of~\bref{eq:Hom-adjn} is a map
$C \of X \go G_2(Y)$ in $\Eee/(TE \times E)$.  This is a map $C \of X \go
TE \times Y$ in $\Eee$ such that
\[
\begin{slopeydiag}
C\of X	&	&\rTo	&	&TE \times Y	\\
	&\rdTo<{(\dom\sof\nu_X, \cod\sof\nu_X)} 
		&	&\ldTo>{1\times q}&	\\
	&	&TE \times E&	&		\\
\end{slopeydiag}
\]
commutes, and this in turn is a map $C\of X \go Y$ in $\Eee/E$.
\done
\end{proof}

Next observe that $\Eee/(TE \times E)$%
%
\lbl{p:slice-monoidal}
%
is naturally a monoidal category: it is the full sub-bicategory of
$\Sp{\Eee}{T}$ whose only object is $E$.  Tensor product of objects
of $\Eee/(TE \times E)$ is composition $\of$ of 1-cells in $\Sp{\Eee}{T}$,
and a monoid in $\Eee/(TE \times E)$ is a $T$-multicategory $C$ with
$C_0=E$.  The functor
\[
\begin{array}{rcl}
\Eee/(TE \times E) \,\times\, \Eee/E 	&\go		&\Eee/E	\\
(C,X)					&\goesto	&C \of X\\
\end{array}
\]
then becomes an action of the monoidal category $\Eee/(TE \times E)$ on the
category $\Eee/E$, in the sense of~\ref{eg:mon-cat-action}: there are
coherent natural isomorphisms
\[
D \of (C \of X) \goiso (D\of C)\of X,
\diagspace
X \goiso 1_E \of X
\]
for $C, D \in \Eee/(TE \times E)$, $X \in \Eee/E$. 

\begin{propn}
Let $T$, $\Eee$ and $E$ be as in Proposition~\ref{propn:endo-hom-adjn}.
For each $X \in \Eee/E$, the $T$-graph $\END(X) = \HOM(X,X)$ naturally has
the structure of a $T$-multicategory.
\end{propn}

\begin{proof}
We have to define a composition map $\END(X) \of \END(X) \go \END(X)$.
First let $\mr{ev}_X: \END(X) \of X \go X$ be the map corresponding under
Proposition~\ref{propn:endo-hom-adjn} to the identity $\END(X) \go
\HOM(X,X)$.  Then define composition to be the map corresponding
under~\ref{propn:endo-hom-adjn} to the composite
\[
\END(X) \of \END(X) \of X
\goby{1 * \mr{ev}_X}
\END(X) \of X
\goby{\mr{ev}_X}
X.
\]
The definition of identities is similar but easier.  The associativity and
identity axioms follow from the axioms for an action of a monoidal
category~(\ref{eg:mon-cat-action}).  
\done
\end{proof}

We can now express the alternative definition of algebra.  Given $\Eee$ and
$T$ as above and a $T$-multicategory $C$, let $\Alg'(C)$ be the category in
which
%
\begin{description}
\item[objects] are pairs $(X, h)$ where $X \in \Eee/C_0$ and $h:
C \go \END(X)$ is a homomorphism of monoids in $\Eee/(TC_0 \times C_0)$
\item[maps] $(X, h) \go (Y, k)$ are maps $f: X \go Y$ in
$\Eee/C_0$ such that 
\[
\begin{diagram}[size=2em]
C		&\rTo^{h}		&\HOM(X,X)	\\
\dTo<{k}	&			&\dTo>{\HOM(1,f)}	\\
\HOM(Y,Y)       &\rTo_{\HOM(f,1)}       &\HOM(X,Y)		\\
\end{diagram}
\]
commutes.
\end{description}
%
(A homomorphism between monoids in $\Eee/(TC_0 \times C_0)$ is just a map
$f$ between the corresponding multicategories such that $f_0: C_0 \go C_0$
is the identity.)

% This definition is indeed the same as the usual definition of algebra:
%
\begin{thm}
Let $T$ be a cartesian monad on a cartesian, locally cartesian closed
category $\Eee$.  Let $C$ be a $T$-multicategory.  Then there is an
isomorphism of categories $\Alg'(C) \iso \Alg(C)$.
\end{thm}

\begin{proof}
Let $X \in \Eee/C_0$.  Proposition~\ref{propn:endo-hom-adjn} in the case
$Y=X$ gives a bijection between $T$-graph maps $h: C \go \END(X)$ and maps
$\ovln{h}: T_C(X) = C\of X \go X$ in $\Eee/C_0$.  Under this
correspondence, $h$ is a homomorphism of monoids if and only if $\ovln{h}$
is an algebra structure on $X$.  So we have a bijection between the objects
of $\Alg'(C)$ and those of $\Alg(C)$.  The remaining checks are
straightforward.  
\done
\end{proof}%
%
\index{generalized multicategory!algebra for|)}%
%
\index{algebra!generalized multicategory@for generalized multicategory|)}
%








\section{Free multicategories}
\lbl{sec:free-mti}%
%
\index{generalized multicategory!free|(}
%


Any directed graph freely generates a category: objects are vertices and
maps are chains of edges.  More generally, any `graph' in which each `edge'
has a finite sequence of inputs and a single output freely generates a
plain multicategory, as explained in~\ref{sec:om-further}: objects are
vertices and maps are trees of edges.  In this short section we extend this
to generalized multicategories.

The construction for plain multicategories involves an infinite recursive
process, so we cannot hope to generalize to arbitrary cartesian $\Eee$ and
$T$---after all, the category $\Eee$ being cartesian only means that it
admits certain finite limits.  There is, however, a class of cartesian
categories $\Eee$ and a class of cartesian monads $T$ for which the free
$\Cartpr$-multicategory construction is possible, the so-called
\demph{suitable}%
%
\index{category!suitable}%
%
\index{suitable}%
%
\index{monad!suitable}
%
categories and monads.  The definition of suitability is
quite complicated, but fortunately can be treated as a black box: all the
properties of suitable monads that we need are stated in this section, and
the details are confined to Appendix~\ref{app:free-mti}.

First, we have a good stock of suitable categories and monads:
%
\begin{thm}	\lbl{thm:free-gen}
Any presheaf category is suitable.  Any finitary cartesian monad on a
cartesian category is suitable.
\end{thm}
%
A functor is said to be \demph{finitary}%
%
\index{finitary}%
%
\index{functor!finitary}
%
%
if it preserves filtered colimits%
%
\index{colimit!filtered}
%
(themselves defined in Mac Lane~\cite[IX.1]{MacCWM}); a monad $(T, \mu,
\eta)$ is said to be \demph{finitary}%
%
\index{monad!finitary}
%
if the functor $T$ is finitary.
In almost all of the examples in this book, $T$ is a finitary monad on a
presheaf category.

Second, suitability is a sufficient condition for the existence of free
multicategories:
%
\begin{thm}	\lbl{thm:free-main}
Let $T$ be a suitable monad on a suitable category $\Eee$.  Then the
forgetful functor
\[
\Cartpr\hyph\Multicat \go \Eee^+ = \Cartpr\hyph\Graph%
% 
\glo{plusofcaty}
% 
\]
has a left adjoint, the adjunction is monadic, and if $T^+$%
% 
\glo{plusofmonad}
% 
is the induced
monad on $\Eee^+$ then both $T^+$ and $\Eee^+$ are suitable.
\end{thm}

\begin{example}	\lbl{eg:fc-cart}
The category of sets and the identity monad are
suitable~(\ref{thm:free-gen}).  In this case Theorem~\ref{thm:free-main}
tells us that there is a free category monad $\fc$ on the category of
directed graphs, and that it is suitable.  In particular it is cartesian, so
it makes sense to talk about $\fc$-multicategories,%
%
\index{fc-multicategory@$\fc$-multicategory}%
%
\index{category!free (fc)@free ($\fc$)}
%
as we did in
Chapter~\ref{ch:fcm}.
\end{example}

Taking the free category on a directed graph leaves the set of objects
(vertices) unchanged, and the corresponding fact for generalized
multicategories is expressed in a variant of the theorem.  Notation: if $E$
is an object of $\Eee$ then $\Cartpr\hyph\Multicat_E$%
% 
\glo{Multicatfixedobjs}
% 
is the subcategory of
$\Cartpr\hyph\Multicat$ whose objects $C$ satisfy $C_0=E$ and whose
morphisms $f$ satisfy $f_0=1_E$.  Observe that $\Eee/(TE \times E)$ is the
category of $T$-graphs with fixed object-of-objects $E$.
%
\begin{thm}	\lbl{thm:free-fixed}
Let $T$ be a suitable monad on a suitable category $\Eee$, and let $E \in
\Eee$.  Then the forgetful functor
\[
\Cartpr\hyph\Multicat_E \go \Eee^+_E = \Eee/(TE \times E)%
% 
\glo{pluscatyfo}
% 
\]
has a left adjoint, the adjunction is monadic, and if $T^+_E$%
% 
\glo{plusmonadfo}
% 
is the
induced monad on $\Eee^+_E$ then both $T^+_E$ and $\Eee^+_E$ are suitable.
\end{thm}

\begin{example}	\lbl{eg:free-cl-opd-cart}
The free monoid monad on the category of sets is suitable,
by~\ref{thm:free-gen}, hence the free%
%
\index{operad!free}
%
plain operad monad on $\Set/\nat$ is
also suitable, by~\ref{thm:free-fixed}.  In particular it is cartesian, as
claimed in~\ref{eg:mon-free-cl-opd}.  
\end{example}

In both examples the theorems were used to establish that $T^+$ and
$\Eee^+$, or $T^+_E$ and $\Eee^+_E$, were cartesian (rather than suitable).
We will use the full iterative strength in~\ref{sec:opetopes} to construct
the `opetopes'.

For technical purposes later on, we will need a refined version of these
results.  Wide pullbacks are defined on p.~\pageref{p:defn-wide-pb}.
%
\begin{propn}	\lbl{propn:free-refined}
If $\Eee$ is a presheaf category and the functor $T$ preserves wide
pullbacks then the same is true of $\Eee^+$ and $T^+$ in
Theorem~\ref{thm:free-main}, and of $\Eee^+_E$ and $T^+_E$ in
Theorem~\ref{thm:free-fixed}.  Moreover, if $T$ is finitary then so are
$T^+$ and $T^+_E$.
\end{propn}%
%
\index{generalized multicategory!free|)}
%




\section{Structured categories}
\lbl{sec:struc}%
%
\index{monoidal category!multicategory@\vs.\ multicategory|(}%
%
\index{structured category!generalized multicategory@\vs.\ generalized multicategory|(}%
%
\index{generalized multicategory!structured category@\vs.\ structured category|(}
%


Any monoidal category has an underlying plain multicategory.  Here we
meet `$T$-structured categories', for any cartesian monad $T$, which
bear the same relation to $T$-multicategories as strict monoidal categories
do to plain multicategories.  At the end we briefly consider the non-strict
case.

A strict monoidal category is a monoid in $\Cat$, or, equivalently, a
category in $\fcat{Monoid}$.  This makes sense because the category
$\fcat{Monoid}$ is cartesian.  More generally, if $T$ is a cartesian monad
on a cartesian category $\Eee$ then the category $\Eee^T$ of algebras is
also cartesian (since the forgetful functor $\Eee^T \go \Eee$ creates
limits), so the following definition makes sense:
%
\begin{defn}
Let $T$ be a cartesian monad on a cartesian category $\Eee$.  Then a
\demph{$T$-structured category}%
%
\index{structured category}
%
is a category in $\Eee^T$, and we write
$T\hyph\Struc$%
% 
\glo{Struc}
% 
or $\Cartpr\hyph\Struc$ for the category $\Cat(\Eee^T)$ of
$T$-structured categories.
\end{defn}
%
A $T$-structured category is, incidentally, a generalized multicategory:
\[
T\hyph\Struc \iso (\Eee^T, \id)\hyph\Multicat
\]
where $\id$ is the identity monad.

\begin{example}
For any cartesian category $\Eee$ we have 
\[
(\Eee,\id)\hyph\Struc \iso (\Eee,\id)\hyph\Multicat \iso \Cat(\Eee).
\]
\end{example}

\begin{example}
If $T$ is the free monoid monad on the category $\Eee$ of sets then
$T\hyph\Struc$ is the category $\fcat{StrMonCat}_\mr{str}$ of strict%
%
\index{monoidal category!strict}
%
monoidal categories and strict monoidal functors.
\end{example}

For an alternative definition, lift $T$ to a monad $\Cat(T)$ on
$\Cat(\Eee)$, then define a $T$-structured category as an algebra for
$\Cat(T)$.  This is equivalent; more precisely, there is an isomorphism of
categories
\[
\Cat(\Eee^T) \iso \Cat(\Eee)^{\Cat(T)}.
\]
In the plain case $\Cat(T)$ is the free strict monoidal category monad on
$\Cat(\Set) = \Cat$, so algebras for $\Cat(T)$ are certainly the same as
$T$-structured categories.

\begin{example}	\lbl{eg:struc-sr}
Let $T$ be the monad on $\Set$ corresponding to a strongly regular
algebraic theory, as in~\ref{eg:mon-CJ}.  It makes sense to take models of
such a theory in any category possessing finite products.  A $T$-structured
category is an algebra for $\Cat(T)$, which is merely a model of the theory
in $\Cat$.
\end{example}

\begin{example}	\lbl{eg:struc-pointed}
A specific instance is the monad $T = (1 + \dashbk)$ on $\Set$
corresponding to the theory of pointed sets~(\ref{eg:mon-exceptions}).
Then a $T$-structured category is a category $A$ together with a functor
from the terminal category $1$ into $A$; in other words, it is a category
$A$ with a distinguished object.
\end{example}

\begin{example}	\lbl{eg:struc-fin-lims}
The principle described in~\ref{eg:struc-sr} for finite product theories
holds equally for finite limit theories.  For instance, if $T$ is the free
plain operad monad on $\Eee = \Set^\nat$ (as in~\ref{eg:mon-free-cl-opd})
then a $T$-structured category is an operad in $\Cat$, that is, a
$\Cat$-operad (p.~\pageref{p:defn-V-Operad}).  So we now have three
descriptions of $\Cat$-operads:%
%
\index{Cat-operad@$\Cat$-operad!three definitions of}
%
as operads in $\Cat$, as $S$-operads where
$S$ is the free strict monoidal category monad on
$\Cat$~(\ref{eg:mti-Cat}), and as $T$-structured categories.
\end{example}

\begin{example}
When $T$ is the free strict $\omega$-category%
%
\index{omega-category@$\omega$-category!strict!free}
%
monad on the category of
globular sets~(\ref{eg:glob-mnd}), a $T$-structured category is what has
been called a `strict monoidal%
%
\index{monoidal globular category}
%
globular category' (Street~\cite[\S
1]{StrRMB} or Batanin~\cite[\S 2]{BatMGC}).%
%
\index{Batanin, Michael}
%
\end{example}

\begin{example}
Take the free category%
%
\index{category!free (fc)@free ($\fc$)}
%
monad $\fc$ on the category $\Eee$ of directed
graphs (Chapter~\ref{ch:fcm}).  Then an $\fc$-structured category is a
category in $\Eee^\fc \iso \Cat$, that is, a strict double%
%
\index{double category!strict}
%
category.
\end{example}

Every strict monoidal category $A$ has an underlying plain multicategory
$UA$, and every plain multicategory $C$ generates a free strict monoidal
category $FC$, giving an adjunction $F\ladj U$~(\ref{sec:om-further}).  The
same applies for generalized multicategories.  Any $T$-structured category
$A$ has an underlying $T$-multicategory $UA$, whose graph is given by
composing along the upper slopes of the pullback diagram
\[
\begin{slopeydiag}
	&		&(UA)_1\Spbk&	&	&	&	\\
	&\ldTo		&	&\rdTo	&	&	&	\\
TA_0	&		&	&	&A_1	&	&	\\
	&\rdTo<{h_0}	&	&\ldTo<\dom&	&\rdTo>\cod&	\\
	&		&A_0	&	&	&	&A_0	\\
\end{slopeydiag}
\]
in which $h_0: TA_0 \go A_0$ is the $T$-algebra structure on $A_0$.
Conversely, the free $T$-structured category $FC$ on a $T$-multicategory
$C$ has graph
\[
\begin{slopeydiag}
	&	&	&	&TC_1	&	&	&	&	\\
	&	&	&\ldTo<{T\dom}&	&\rdTo(4,4)>{T\cod}&&	&	\\
	&	&T^2 C_0&	&	&	&	&	&	\\
	&\ldTo<{\mu_{C_0}}&&	&	&	&	&	&	\\
TC_0	&	&	&	&	&	&	&	&TC_0	\\
\end{slopeydiag}
\]
and the $T$-algebra structure on $(FC)_i = TC_i$ is $\mu_{C_i}$ ($i=0,1$).
We then have the desired adjunction
%
\begin{equation}	\label{eq:struc-mti-adjn}
\begin{diagram}[height=2em]
T\hyph\Struc		\\
\uTo<F \ladj \dTo>U	\\
T\hyph\Multicat.	\\
\end{diagram}
\end{equation}
%

\begin{example}	\lbl{eg:free-struc-D}
Consider once more the free monoid monad $T$ on $\Eee=\Set$.  Take the
terminal plain multicategory $1$, which has graph
\[
\nat \ogby{1} \nat \goby{!} 1.
\]
Then $F1$ is a strict monoidal category with graph
\[
\nat \ogby{+} T\nat \goby{T!} \nat.
\]
The objects of $F1$ are the natural numbers and a map $m \go n$ in $F1$ is
a sequence \bftuple{m_1}{m_n} of natural numbers such that $m_1 +\cdots+
m_n = m$.  So $F1$ is the strict monoidal category $\scat{D}$%
%
\index{augmented simplex category $\scat{D}$}
%
of (possibly
empty) finite totally ordered sets, with addition as tensor and $0$ as
unit.  This is also suggested by diagram~\bref{diag:arrows-in-mon-cat}
(p.~\pageref{diag:arrows-in-mon-cat}).
\end{example}


\begin{example}	\lbl{eg:struc-disc}
Any object $K$ of a cartesian category $\cat{K}$ generates a category $DK$%
% 
\glo{discintcat}
% 
in $\cat{K}$, the \demph{discrete%
%
\index{category!discrete}
%
category} on $K$, uniquely determined by
its underlying graph
\[
K \ogby{1} K \goby{1} K.
\]
In particular, if $h = (TX \goby{h} X)$ is an algebra for some cartesian
monad $T$ then $Dh$ is a $T$-structured category and $UDh$ is a
$T$-multicategory with graph
\[
TX \ogby{1} TX \goby{h} X.
\]
So $UDh$ is the $T$-multicategory $h^+$%
%
\index{plus construction $\blank^+$}
%
discussed in
Example~\ref{eg:multi-alg}, and the triangle of functors
\[
\begin{diagram}[height=2em]
\Eee^T	&\rTo^{D}	&T\hyph\Struc	\\
	&\rdTo<{\blank^+}	&\dTo>{U}	\\
	&			&T\hyph\Multicat\\
\end{diagram}
\]
commutes up to natural isomorphism.  
\end{example}

I have tried to emphasize that multicategories are truly different from
monoidal categories (as well as providing a more natural language in many
situations).  This is borne out by the fact that the functor $U:
\fcat{StrMonCat}_\mr{str} \go \Multicat$ is far from an equivalence: it
is faithful, but neither full~(\ref{eg:map-mti-mon}) nor
essentially surjective on objects (p.~\pageref{p:not-ESO}).  

There is, however, a `representation%
%
\index{representation theorem}
%
theorem' saying that every
$T$-multicategory embeds fully in some $T$-structured category.  In
particular, every plain multicategory is a full sub-multicategory of the
underlying multicategory of some strict monoidal
category~(\ref{eg:multi-some-of-mon}).  I do not know of any use for the
theorem; it does not reduce the study of $T$-multicategories to the study
of $T$-structured categories any more than the Cayley%
%
\index{Cayley representation}
%
Representation
Theorem reduces the study of finite groups to the study of the symmetric
groups.  I therefore leave the proof as an exercise.  The precise statement
is as follows.
%
\begin{defn}
Let $T$ be a cartesian monad on a cartesian category $\Eee$.  A map $f: C
\go C'$ of $T$-multicategories is \demph{full%
%
\index{full and faithful}
%
and faithful} if the
square
\[
\begin{diagram}[size=2em]
C_1		&\rTo^{(\dom,\cod)}	&TC_0 \times C_0	\\
\dTo<{f_1}	&			&\dTo>{Tf_0 \times f_0}	\\
C'_1		&\rTo_{(\dom,\cod)}	&TC'_0 \times C'_0\\
\end{diagram}
\]
is a pullback.
\end{defn}
%
\begin{propn}	\lbl{propn:Cayley-multi}
Let $T$ be a cartesian monad on a cartesian category $\Eee$, and consider
the adjunction of~\bref{eq:struc-mti-adjn}
(p.~\pageref{eq:struc-mti-adjn}).  For each $T$-multicategory $C$, the unit
map $C \go UFC$ is full and faithful.  
\done
\end{propn}

We finish with two miscellaneous thoughts.

First, we have been considering generalizations of plain multicategories,
which are structures whose operations are `many in, one out'.  But
in~\ref{sec:om-further} we also considered PROs
(and their symmetric
cousins, PROPs), whose operations are `many%
%
\index{many in, many out}
%
in, many out'.  A PRO consists
of a set $S$ (the objects, or `colours', often taken to have only one
element) and a strict monoidal category whose underlying monoid of objects
is the free monoid on $S$.  The generalization to arbitrary cartesian
monads $T$ on cartesian categories $\Eee$ is clear: a \demph{$T$-PRO}%
%
\index{PRO!generalized}
%
should be defined as a pair $(S,A)$ where $S\in\Eee$ and $A$ is a
$T$-structured category whose underlying $T$-algebra of objects is the free
$T$-algebra on $S$.  We will not do anything with this definition, but
see~\ref{sec:many} for further discussion of `many in, many out'.

Second, if strict monoidal categories generalize to $T$-structured
categories, what do weak monoidal categories generalize to?  One answer
comes from realizing that the category $\Cat(\Eee)$ has the structure of a
strict 2-category and the monad $\Cat(T)$ the structure of a strict
2-monad.  We can then define a \demph{weak $T$-structured%
%
\index{structured category!weak}
%
category} to be a
weak algebra for this 2-monad, and indeed do the same in the lax case.  In
particular, if $T$ is the free plain operad monad on
$\Set^\nat$~(\ref{eg:struc-fin-lims}) then a weak $T$-structured category is
like a $\Cat$-operad,%
%
\index{Cat-operad@$\Cat$-operad!weak}
%
but with the operadic composition only obeying
associativity and unit laws up to coherent isomorphism.  For example, let
$P(n)$ be the category of Riemann%
%
\index{Riemann surface}%
%
\index{manifold!operad of}
%
surfaces whose boundaries are identified
with the disjoint union of $(n+1)$ copies of $S^1$, and define composition
by gluing: then $P$ forms a weak $T$-structured category.
Compare~\ref{eg:opd-Riemann} where, not having available the refined
language of generalized multicategories, we had to quotient out by
isomorphism and so lost (for instance) any information about automorphisms
of the objects.  We do not, however, pursue weak structured categories any
further in this book.%
%
\index{monoidal category!multicategory@\vs.\ multicategory|)}
%
\index{structured category!generalized multicategory@\vs.\ generalized multicategory|)}%
%
\index{generalized multicategory!structured category@\vs.\ structured category|)}
%








\section{Change of shape}
\lbl{sec:change}%
%
\index{change of shape|(}%
%
\index{generalized multicategory!change of shape|(}
%

To do higher-dimensional category theory we are going to want to move%
%
\index{n-category@$n$-category!definitions of!comparison}
%
between $n$-categories and $(n+1)$-categories and $\omega$-categories,
between globular and cubical and simplicial structures, and so on.  In
Chapter~\ref{ch:a-defn} we will see that a weak $n$-category can be defined
as an algebra for a certain $\gm{n}$-operad, where $\gm{n}$ is the free
strict $n$-category monad on the category of $n$-globular sets.  So if we
want to be able to relate $n$-categories to $(n+1)$-categories, for
instance, then we will need some way of relating $\gm{n}$-operads to
$\gm{n+1}$-operads and some way of relating their algebras.  In this
section we set up the supporting theory: in other words, we show what
happens to $T$-multicategories and their algebras as the monad $T$
varies.

Formally, we expect the assignment $\Cartpr \goesto \Cartpr\hyph\Multicat$
to be functorial in some way.  We saw in~\ref{sec:more-monads} that there
are notions of lax, colax, weak, and strict maps of monads, some of which
are special cases of others, and we will soon see that a map $\Cartpr \go 
(\Eee', T')$ of any one of these types induces a functor
\[
\Cartpr\hyph\Multicat \go (\Eee',T')\hyph\Multicat.
\]

First we need some terminology.  A lax map of monads $(Q, \psi): \Cartpr
\go (\Eee',T')$ is \demph{cartesian}%
%
\index{monad!map of!cartesian}%
%
\index{cartesian}
%
%
if the functor $Q$ is cartesian (but
note that the natural transformation $\psi$ need not be cartesian).
Cartesian monads, cartesian lax maps of monads, and transformations form a
sub-2-category $\fcat{CartMnd}_\mr{lax}$%
% 
\glo{CartMndlax}
% 
of $\fcat{Mnd}_\mr{lax}$.  Dually, a colax map of monads $(P, \phi)$ is
\demph{cartesian} if the functor $P$ \emph{and} the natural transformation
$\phi$ are cartesian.  Cartesian monads, cartesian colax maps of monads,
and transformations form a sub-2-category $\fcat{CartMnd}_\mr{colax}$ of
$\fcat{Mnd}_\mr{colax}$.

These definitions appear haphazard, with natural transformations required
to be cartesian, or not, at random.  I can justify this only pragmatically:
they are the conditions required to make the following constructions work.

The main constructions are as follows.  Let $(Q, \psi): \Cartpr \go
(\Eee',T')$ be a cartesian lax map of cartesian monads.  Then there is an
induced functor
\[
Q_* = (Q, \psi)_*: 
\Cartpr\hyph\Multicat \go (\Eee',T')\hyph\Multicat%
% 
\glo{laxmtiindftr}
% 
\]
sending an \Cartpr-multicategory $C$ to the $(\Eee',T')$-multicategory
$Q_* C$ whose underlying graph is given by composing along the upper
slopes of the pullback diagram
\[
\begin{diagram}[width=1.7em,height=1.7em,scriptlabels,noPS]
	&	&(Q_* C)_1\Spbk&	&	&	&	\\
	&\ldTo	&		&\rdTo	&	&	&	\\
T'Q C_0 &	&		&	&QC_1	&	&	\\
	&\rdTo<{\psi_{C_0}}&	&\ldTo<{Q\dom}&	&\rdTo>{Q\cod}&	\\
	&	&QT C_0		&	&	&	&QC_0=(Q_* C)_0\\
\end{diagram}
\]
and whose composition and identities are defined in an evident way.
Dually, let $(P, \phi): \Cartpr \go (\Eee',T')$ be a cartesian colax map of
cartesian monads.  Then there is an induced functor
\[
P_* = (P, \phi)_*: 
\Cartpr\hyph\Multicat \go (\Eee',T')\hyph\Multicat%
% 
\glo{colaxmtiindftr}
% 
\]
sending an \Cartpr-multicategory $C$ to the $(\Eee',T')$-multicategory
$P_* C$ with underlying graph
\[
\begin{diagram}[width=1.7em,height=1.7em,scriptlabels,noPS]
	&	&	&	&PC_1 = (P_* C)_1&&&	&	\\
	&	&	&\ldTo<{P\dom}&	&\rdTo(4,4)>{P\cod}&&	&	\\
	&	&PTC_0	&	&	&	&	&	&	\\
	&\ldTo<{\phi_{C_0}}&&	&	&	&	&	&	\\
T'PC_0	&	&	&	&	&	&	&	&
PC_0=(P_* C)_0.	\\
\end{diagram}
\]
Note that $(Q_* C)_0 = Q(C_0)$ and $(P_* C)_0 = P(C_0)$, so if $Q$ or $P$
preserves all finite limits then $Q_*$ or $P_*$ restricts to a functor
%
\label{p:change-operads}
%
\[
\Cartpr\hyph\Operad \go (\Eee',T')\hyph\Operad
\]
between categories of operads.

After filling in all the details we obtain two maps of strict 2-categories,
\[
\fcat{CartMnd_\mr{lax}} \go \fcat{CAT}, 
\diagspace
\fcat{CartMnd_\mr{colax}} \go \fcat{CAT},
\]
both defined on objects by $\Cartpr \goesto \Cartpr\hyph\Multicat$.  The
first is defined using pullbacks, so is only a weak functor; the second is
strict.  These dual functors agree where they intersect: the 2-categories
$\fcat{CartMnd}_\mr{lax}$ and $\fcat{CartMnd}_\mr{colax}$ `intersect' in
the 2-category $\fcat{CartMnd}_\mr{wk}$%
% 
\glo{CartMndwk}
% 
of cartesian monads, cartesian weak
maps of monads, and transformations, and the square
\[
\begin{diagram}[height=2em,width=5em]
\fcat{CartMnd}_\mr{wk}	&\rIncl		&\fcat{CartMnd}_\mr{colax}	\\
\dIncl			&		&\dTo				\\
\fcat{CartMnd}_\mr{lax}	&\rTo		&\CAT				\\
\end{diagram}
\]
commutes up to natural isomorphism.  (A \demph{cartesian%
%
\index{monad!map of!cartesian}
%
weak map} of
monads is a cartesian lax map $(Q, \psi)$ in which $\psi$ is a natural
isomorphism; then $\psi$ is automatically cartesian.)

\begin{example}	\lbl{eg:ind-ftr-int-cats}
Let $Q: \cat{D} \go \cat{D'}$ be a cartesian functor between cartesian
categories.  Then $(Q, \id): (\cat{D}, \id) \go (\cat{D'}, \id)$ is a
strict map of monads and so induces (unambiguously) a functor between the
categories of multicategories, 
\[
\left( 
(\cat{D},\id)\hyph\Multicat \goby{Q_*} (\cat{D'},\id)\hyph\Multicat
\right)
= 
\left( 
\Cat(\cat{D}) \goby{Q_*} \Cat(\cat{D'})
\right).
\]
This is the usual induced functor between categories of internal%
%
\index{category!internal}
%
categories.
\end{example}

There is another way in which the two processes are compatible, involving
adjunctions.  Suppose we have a diagram
%
\begin{equation}	\label{diag:adjt-change}
\begin{diagram}[height=2em]
\Cartpr			\\
\uTo<{\pr{P}{\phi}}	
\ladj			
\dTo>{\pr{Q}{\psi}}	\\
(\Eee',T')		\\
\end{diagram}
\end{equation}
%
in which $(P, \phi)$ is a cartesian colax map of cartesian monads, $(Q,
\psi)$ is a cartesian lax map, and there is an adjunction%
%
\index{monad!map of!adjunction between}
%
of functors
$P\ladj Q$ under which $\phi$ and $\psi$ are mates~(\ref{sec:more-monads}).
Then, as may be checked, there arises an adjunction between the functors
$P_*$ and $Q_*$ constructed above:
%
\begin{equation}   \label{diag:mti-adjn}
\begin{diagram}[height=2em]
\Cartpr\hyph\Multicat	\\
\uTo<{P_*}		
\ladj			
\dTo>{Q_*}		\\
(\Eee',T')\hyph\Multicat.\\
\end{diagram}
\end{equation}

\begin{example}	\lbl{eg:change-struc-adjn}
Let $T$ be a cartesian monad on a cartesian category $\Eee$.  Then there is
a diagram
\[
\begin{diagram}[height=2em]
(\Eee^T, \id)		\\
\uTo<{(F, \nu)}		
\ladj			
\dTo>{(U, \epsln)}	\\
(\Eee,T)		\\
\end{diagram}
\]
of the form~\bref{diag:adjt-change}, in which $F$ and $U$ are the free and
forgetful functors and $\nu$ and $\epsln$ are certain canonical natural
transformations.  This gives rise by the process just described to an
adjunction
\[
\begin{diagram}[height=2em]
(\Eee,T)\hyph\Struc	\\
\uTo<{F_*}		
\ladj			
\dTo>{U_*}		\\
(\Eee,T)\hyph\Multicat,	\\
\end{diagram}%
%
\index{structured category!generalized multicategory@\vs.\ generalized multicategory}%
%
\index{generalized multicategory!structured category@\vs.\ structured category}
%
\]
none other than the adjunction that was the subject of the previous
section.
\end{example}

\begin{example}   \lbl{eg:cart-adjts}
Let $(P, \phi): (\Eee, T) \go (\Eee', T')$ be a cartesian colax map of
cartesian monads, and suppose that the functor $P$ has a right adjoint $Q$.
Then $P_*$ has a right adjoint too: for taking the mate $\psi =
\ovln{\phi}$ gives $Q$ the structure of a cartesian lax map of monads,
% (p.~\pageref{p:colax-lax-mate}), 
leading to an adjunction $P_* \ladj Q_*$.
\end{example}

We have considered change of shape for $T$-multicategories; let us now do
the same for $T$-algebras%
%
\index{algebra!monad@for monad!induced functor}
%
and $T$-structured categories, concentrating on
lax rather than colax maps.  If $(Q, \psi): (\Eee, T) \go (\Eee', T')$ is a
cartesian lax map of cartesian monads then the induced functor $\Eee^T \go
\Eee'^{T'}$ is also cartesian, so by Example~\ref{eg:ind-ftr-int-cats}
induces in turn a functor $\Cat(\Eee^T) \go \Cat(\Eee'^{T'})$---in other
words, induces a functor on structured%
%
\index{structured category!change of shape}
%
categories,
\[
Q_*: \Cartpr\hyph\Struc \go (\Eee',T')\hyph\Struc.
\]
% 
The change-of-shape processes for algebras, structured categories and
multicategories are compatible: for any cartesian lax map $(Q, \psi):
\Cartpr \go (\Eee',T')$ of cartesian monads, the diagram
\[
\begin{diagram}[height=2em]
\Eee^T			&\rTo^{D}	&
(\Eee,T)\hyph\Struc	&\rTo^{U_*}		&
(\Eee,T)\hyph\Multicat	\\
\dTo<{Q_*}		&			&
\dTo>{Q_*}		&			&
\dTo>{Q_*}		\\
\Eee'^{T'}		&\rTo_{D}	&
(\Eee',T')\hyph\Struc	&\rTo_{U_*}		&
(\Eee',T')\hyph\Multicat\\
\end{diagram}
\]
commutes up to natural isomorphism.  Here $D$ is the discrete category
functor defined in~\ref{eg:struc-disc}, and the commutativity of the
left-hand square can be calculated directly.  The $U_*$'s are the forgetful
functors, which, as we saw in~\ref{eg:change-struc-adjn}, are induced by
lax maps $(U,\epsln)$.  To see that the right-hand square commutes, it is
enough to see that the square of lax maps
\[
\begin{diagram}[height=2em]
(\Eee^T, \id)		&\rTo^{(U,\epsln)}	&(\Eee,T)	\\
\dTo<{(Q_*,\id)}	&			&\dTo>{(Q,\psi)}\\
(\Eee'^{T'}, \id)	&\rTo_{(U,\epsln)}	&(\Eee',T')	\\
\end{diagram}
\]
commutes, and this is straightforward.

Finally, we answer the question posed in the introduction to the section:
how do algebras%
%
\index{algebra!generalized multicategory@for generalized multicategory!change of shape (induced functor)}
%
for multicategories behave under change of shape?  An
algebra for an \Cartpr-multicategory $C$ is an object $X$ over $C_0$ acted
on by $C$, so if $(Q, \psi): \Cartpr \go (\Eee',T')$ is a cartesian lax or
colax map then we might hope that $QX$, an object over $QC_0$, would be
acted on by $Q_* C$.  In other words, we might hope for a functor from
$\Alg(C)$ to $\Alg(Q_* C)$.  Such a functor does indeed exist, in both the
lax and colax cases.  

First take a cartesian lax map of cartesian monads, $(Q, \psi): \Cartpr \go
(\Eee',T')$.  For any $T$-multicategory $C$ there is an induced lax map of
monads
\[
(\Eee/C_0, T_C) \go (\Eee'/QC_0, T'_{Q_* C}), 
\]
comprised of the functor $\Eee/C_0 \go \Eee'/QC_0$ induced by $Q$ and a
natural transformation $\psi$ that may easily be determined.  This in turn
induces a functor on categories of algebras:
%
% \renewcommand{\arraystretch}{3}	% See Lamport p 207
%
\[
\begin{array}{rcl}
\Alg(C)		&\go	&\Alg(Q_* C),	\\
&&\\
\left(
\begin{diagram}[height=1.5em,scriptlabels]
T_C X	\\
\dTo>h	\\
X	\\
\end{diagram}
\right)						&
\goesto						&
\left(
\begin{diagram}[height=1.5em,scriptlabels]
T'_{Q_* C} Q X	\\
\dTo>{\psi^C_X}		\\
QT_C X			\\
\dTo>{Qh}		\\
QX			\\
\end{diagram}
\right)
.
\end{array}
\]
%
% \renewcommand{\arraystretch}{1}	% See Lamport p 207
%

Now take a cartesian colax map of cartesian monads, $(P, \phi): \Cartpr \go
(\Eee',T')$.  For any $T$-multicategory $C$ there is an induced \emph{weak}
map of monads 
\[
(\Eee/C_0, T_C) \go (\Eee'/PC_0, T'_{P_* C}), 
\]
which amounts to saying that if $X = (X \goby{p} C_0) \in \Eee/C_0$ then
$T'_{P_* C} P X \iso P T_C X$ canonically; and indeed, we have a diagram
\[
\begin{diagram}[width=1.7em,height=1.7em,scriptlabels,noPS]
	&	&	&	&\Spbk PT_C X&	&	&	&	\\
	&	&	&\ldTo	&	&\rdTo	&	&	&	\\
	&	&\Spbk PTX&	&	&	&PC_1	&	&	\\
	&\ldTo<{\phi_X}&&\rdTo>{PTp}&	&\ldTo>{P\dom}&	&\rdTo>{P\cod}&	\\
T'PX	&	&	&	&PTC_0	&	&	&	&PC_0	\\
	&\rdTo<{T'Pp}&	&\ldTo>{\phi_{C_0}}&&	&	&	&	\\
	&	&T'PC_0,&	&	&	&	&	&	\\
\end{diagram}
\]
giving the isomorphism required.  

\begin{example}	\lbl{eg:change-mon-multi}
In~\ref{sec:om-further} we considered the adjunction between plain
multicategories%
%
\index{monoidal category!multicategory@\vs.\ multicategory}
%
and strict monoidal categories, and
in~\ref{eg:change-struc-adjn} we saw that it is induced by certain maps of
monads: 
\[
\begin{diagram}[height=2em]
(\fcat{Monoid}, \id)			\\
\uTo<{(F,\nu)} \ladj \dTo>{(U,\epsln)}	\\
(\Set, \textrm{free monoid})		\\
\end{diagram}
%
\diagspace
\goesto
\diagspace
%
\begin{diagram}[height=2em]
\fcat{StrMonCat}_\mr{str}	\\
\uTo<{F_*} \ladj \dTo>{U_*}	\\
\Multicat.			\\
\end{diagram}
\]
The lax map $(U, \epsln)$ induces a functor $\Alg(A) \go \Alg(U_* A)$ for
each strict monoidal category $A$.  When $A$ is regarded as a
$(\fcat{Monoid}, \id)$-multicategory, an $A$-algebra is a lax monoidal
functor $A \go \Set$.  This is the same thing as an algebra for the
underlying multicategory $U_* A$~(\ref{eg:alg-mon}), and the induced
functor is in fact an isomorphism.

Conversely, the colax map $(F, \nu)$ induces a functor $\Alg(C) \go
\Alg(F_* C)$ for each multicategory $C$, whose explicit form is left as an
exercise.
\end{example}

It is no coincidence that the functors induced by $U$ in this example are
isomorphisms.  Roughly speaking, this is because $U$ is monadic:
%
\begin{propn}	\lbl{propn:shape-distrib}
Let $\cat{C}$ be a cartesian category, let $S$ and $S'$ be cartesian monads
on $\cat{C}$, and let $\lambda: S\of S' \go S'\of S$ be a cartesian
distributive law.  Write
\[
\begin{diagram}
\cat{C}^S	&\rTo^{\twid{S}}	&\cat{C}^S	\\
\dTo<U		&\nent\tcs{\psi}		&\dTo>U		\\
\cat{C}		&\rTo_{S'\of S}		&\cat{C}	\\
\end{diagram}
\]
for the induced lax map of monads, as in~\ref{lemma:distrib-iso-algs}.
Then for any $\twid{S}$-multicategory $C$, there is an isomorphism of
categories $\Alg(C) \iso \Alg((U, \psi)_* C)$.
\end{propn}
% 
The monads $\twid{S}$ and $S'\of S$ are cartesian
(\ref{lemma:distrib-gives-monad},~\ref{lemma:distrib-corr}), so it does
make sense to talk about multicategories for them.  
% 
\begin{proof}
We just prove this in the case where $C$ is an $\twid{S}$-operad $O$, since
that is all we will need later and the proof is a little easier.

As well as the lax map shown, we have a strict map of monads
\[
\begin{diagram}
\cat{C}^S	&\rTo^{\twid{S}}	&\cat{C}^S	\\
\dTo<U		&\neeq			&\dTo>U		\\
\cat{C}		&\rTo_{S'}		&\cat{C}.	\\
\end{diagram}
\]
So we have an $S'$-operad $(U, \id)_* O$, hence a monad $S'_{(U, \id)_*
O}$ on $\cat{C}$, of which $\twid{S}_O$ is a lift to $\cat{C}^S$.  By
Lemma~\ref{lemma:distrib-corr}, there is a corresponding distributive law
\[
S \of S'_{(U, \id)_* O} 
\go 
S'_{(U, \id)_* O} \of S. 
\]
This gives $S'_{(U, \id)_* O} \of S$ the structure of a monad on $\cat{C}$;
but it can be checked that this monad is exactly $(S' \of S)_{(U, \psi)_*
O}$, so by Lemma~\ref{lemma:distrib-iso-algs},
\[
\Alg(O)
=
(\cat{C}^S)^{\twid{S}_O}
\iso
\cat{C}^{S'_{(U, \id)_* O} \of S}
\iso
\cat{C}^{(S' \of S)_{(U, \psi)_* O}}
=
\Alg((U, \psi)_* O),
\]
as required.
\done
\end{proof}
% 
The forgetful functor in Example~\ref{eg:change-mon-multi} is the case
$\cat{C} = \Set$, $S = (\textrm{free monoid})$, $S' = \id$.  We will use
the Proposition when comparing definitions of weak 2-category
in~\ref{sec:wk-2}.%
%
\index{change of shape|)}%
%
\index{generalized multicategory!change of shape|)}
%




\section{Enrichment}	\lbl{sec:enr-mtis}

We finish with a brief look at a topic too large to fit in this book.  Its
slogan is `what can we enrich in?'  

Take, for example, $\Ab$-categories:%
%
\index{Ab-category@$\Ab$-category}
%
categories enriched in (or `over')
abelian groups.  The simplest definition of an $\Ab$-category is as a class
$C_0$ of objects together with an abelian group $C(a,b)$ for each $a, b \in
C_0$, a bilinear composition function
% 
\[
C(a,b), C(b,c) \go C(a,c)
\]
for each $a,b,c \in C_0$, and an identity $1_a \in C(a,a)$ for each $a \in
C_0$, satisfying associativity and identity axioms.  You \emph{could}
express composition as a linear map out of a tensor product, but this would
be an irrelevant%
%
\index{monoidal category!multicategory@\vs.\ multicategory}
%
elaboration; put another way, the first definition of
$\Ab$-category can be understood by someone who knows what a multilinear
map is but has not yet learned about tensor products.

More generally, if $V$ is a plain multicategory then there is an evident
definition of $V$-enriched%
%
\index{enrichment!category@of category!plain multicategory@in plain multicategory}
%
category.  Classically one enriches categories
in monoidal categories (as in~\ref{sec:cl-enr}), but this is an unnaturally
narrow setting; in the terminology of~\ref{sec:non-alg-notions},
representability%
%
\index{multicategory!representable}
%
of the multicategory is an irrelevance.

More generally still, suppose we have some type of categorical
structure---`widgets',
say.  Then the question is: for what types of
structure $V$ can we make a definition of `$V$-enriched%
%
\index{enrichment}
%
widget'?  In the
previous paragraphs widgets were categories, and we saw that $V$ could be a
plain multicategory.  (In fact, that is not all $V$ can be, as we will soon
discover.)  In my~\cite{GECM} paper, the question is answered when widgets
are $T$-multicategories for almost any cartesian monad $T$.  The general
definition of enriched $T$-multicategory is short, simple, and given below,
but unwinding its implications takes more space than we have; hence the
following sketch.

Let $T$ be a monad on a category $\Eee$, and suppose that both $T$ and
$\Eee$ are suitable~(\ref{sec:free-mti}), so that there is a cartesian free
$T$-multicategory monad $T^+$ on the category $\Eee^+$ of $T$-graphs.  For
any object $C_0$ of $\Eee$, there is a unique $T$-multicategory structure
on the $T$-graph
\[
\begin{slopeydiag}
	&	&TC_0 \times C_0	&		&	\\
	&\ldTo<{\mr{pr}_1}&		&\rdTo>{\mr{pr}_2}&	\\
TC_0	&	&			&		&C_0,	\\
\end{slopeydiag}
\]
and we write this $T$-multicategory as $IC_0$,%
% 
\glo{indiscgenmti}
% 
the \demph{indiscrete}%
%
\index{generalized multicategory!indiscrete}
%
$T$-multicategory on $C_0$.  Then $IC_0$ is a $T^+$-algebra, so
by~\ref{eg:multi-alg} gives rise to a $T^+$-multicategory $(IC_0)^+$ whose
domain map is the identity.
% 
\begin{defn}%
%
\index{enrichment!generalized multicategory@of generalized multicategory}%
%
\index{generalized multicategory!enriched}
%
%
Let $T$ be a suitable monad on a suitable category $\Eee$ and let $V$ be a
$T^+$-multicategory.  A \demph{$V$-enriched $T$-multicategory} is 
an object $C_0$ of $\Eee$ together with a map $(IC_0)^+ \go V$ of
$T^+$-multicategories. 
\end{defn}

\begin{example}
The most basic example is when $T$ is the identity monad on the category
$\Eee$ of sets.  Then $\Eee^+$ is the category of directed graphs, $T^+$ is
the free category monad $\fc$, and $T^+$-multicategories are the
$\fc$-multicategories of Chapter~\ref{ch:fcm}.  We therefore have a notion
of `$V$-enriched%
%
\index{enrichment!category@of category!fc-multicategory@in $\fc$-multicategory}
%
category' for any $\fc$-multicategory%
%
\index{fc-multicategory@$\fc$-multicategory!enrichment in}
%
$V$.  In the special
case that $V$ is a monoidal category~(\ref{eg:fcm-mon-cat}), we recover the
standard definition of enrichment.  More generally, if $V$ is a
bicategory~(\ref{eg:fcm-bicat}) then we recover the less well-known
definition of category enriched%
%
\index{enrichment!category@of category!bicategory@in bicategory}%
%
\index{bicategory!enrichment in}
%
in a bicategory (Walters~\cite{Wal}),
and if $V$ is a plain multicategory~(\ref{eg:fcm-cl-mti}) then we recover
the definition of category enriched in a plain multicategory.  For more on
enriched categories in this broad sense, see my~\cite{GEC}.
\end{example}

\begin{example}
The next most basic example is when $T$ is the free monoid monad on the
category $\Eee$ of sets.  Theorem~\ref{thm:sm-opetopic} tells us that any
symmetric monoidal category gives rise canonically to what is there called
a $T_2$-multicategory.  By definition, $T_2$ is the free $T$-operad monad
and $T^+$ the free $T$-multicategory monad, so a $T_2$-multicategory is a
special kind of $T^+$-multicategory.  Hence any symmetric monoidal category
gives rise canonically to a $T^+$-multicategory, giving us a definition of
plain multicategory%
%
\index{enrichment!plain multicategory@of plain multicategory!symmetric monoidal category@in symmetric monoidal category}
%
enriched in a symmetric monoidal category, and in
particular, of plain operad%
%
\index{operad!symmetric monoidal category@in symmetric monoidal category}
%
in a symmetric monoidal category.  These are
exactly the usual definitions (pp.~\pageref{p:sym-enr-mti},
\pageref{p:defn-V-Operad}).
\end{example}

\begin{example}	\lbl{eg:relaxed}
Borcherds~\cite{Borch}%
%
\index{Borcherds, Richard}
%
introduced certain structures called `relaxed%
%
\index{relaxed multicategory}%
%
\index{multicategory!relaxed}
%
multilinear categories' in his definition of vertex%
%
\index{vertex!algebra}
%
algebras over a vertex
group, and Soibelman~\cite{SoiMTC, SoiMBC}%
%
\index{Soibelman, Yan}
%
defined the same structures
independently in his work on quantum affine algebras.  As explained by
Borcherds, they can be regarded as categorical structures in which the maps
have singularities%
%
\index{singularity}
%
whose severity is measured by trees.  They also arise
completely naturally in the theory of enrichment: if $T$ is the free monoid
monad on the category $\Eee$ of sets, as in the previous example, then
there is a certain canonical $T^+$-multicategory $V$ such that
$V$-enriched%
%
\index{enrichment!plain multicategory@of plain multicategory!T-Zmulticategory@in $T^+$-multicategory}
%
$T$-multicategories are precisely relaxed multilinear categories.  See
Leinster~\cite[Ch.~4]{GECM} for details.
\end{example}

\begin{example}
In the next chapter we introduce the sequence $(T_n)_{n\in\nat}$ of
`opetopic'%
%
\index{opetopic!monad}
%
monads.  By definition, $T_n$ is the free $T_{n-1}$-operad
monad, and this means that there is a notion of $T_{n-1}$-multicategory
enriched in a $T_n$-multicategory.  More vaguely, a $T_n$-multicategory is
naturally regarded as an $(n+1)$-dimensional structure, so $n$-dimensional
structures can be enriched in $(n+1)$-dimensional structures.
\end{example}


\begin{notes}

Most parts of this chapter have appeared before \cite[\S 4]{GOM},
\cite[Ch.~3]{OHDCT}.  The thought that an operad is a cartesian monad
equipped with a cartesian natural transformation down to the free monoid
monad (\ref{sec:alt-app}) is closely related to Kelly's%
%
\index{Kelly, Max}
%
idea of a
`club'~\cite{KelCD, KelCDC}.%
%
\index{club}
%
 See Snydal~\cite{SnyEBG, SnyRMC}%
%
\index{Snydal, Craig}
%
for more on
the relaxed multicategory definition of vertex algebra.


\end{notes}
