
% abstd.tex
%
% This is meant to be the `standard' macros file, pretty minimal and stable,
% to be copied into every new directory and input into every document.  
%
% Other macros files:
%
% abmore.tex: 		To be created for each paper written, and to contain
% 			all new definitions and (marked clearly) all
% 			modifications of abstd.tex.




% 0.  Meta
% 1.  Packages
% 2.  Formatting	Characters, diagrams, proofs, references
% 3.  Typographic	Mathematical, text
% 4.  Single symbols	Set theory, category theory, misc
% 5.  Constants		Roman, italic, bold, blackboard bold
% 6.  Tuples		Plain tuples, homs  
% 7.  Arrows		Single, plural, slices
% 8.  PT diagrams	Arrows, environments, trees, misc
% 9.  Drafting



% 0. META

\newcommand{\mcm}[3]{\newcommand{#1}[#2]{{\ensuremath{#3}}}}


% 1. PACKAGES

\usepackage{latexsym}
\usepackage{amssymb}


% 2. FORMATTING

% Characters

\mcm{\emptybk}{0}{\:\:}
\mcm{\blank}{0}{(\emptybk)}
\mcm{\dashbk}{0}{-}
\mcm{\hyph}{0}{\mbox{-}}

% Diagrams

\mcm{\diagspace}{0}{\mbox{\hspace{2em}}}

% Proofs

\newcommand{\pf}{\noindent\textbf{Proof}}

% References

\newcommand{\bref}[1]{(\ref{#1})}
\newcommand{\ucontents}[2]{\addcontentsline{toc}{#1}{\numberline{}{#2}}}


% 3. TYPOGRAPHIC

% Mathematical 

\mcm{\mb}{1}{\mathbf{#1}}
\mcm{\mc}{1}{\mathcal{#1}}
\mcm{\mi}{1}{\mathit{#1}}
\mcm{\mr}{1}{\mathrm{#1}}
\mcm{\cat}{1}{\mc{#1}}
\mcm{\fcat}{1}{\mb{#1}}
\mcm{\ovln}{1}{\overline{#1}}
\mcm{\twid}{1}{\widetilde{#1}}

% Text

\newcommand{\slsh}{/\linebreak[0]}
\newcommand{\dblslsh}{//\linebreak[0]}
\newcommand{\dt}{.\linebreak[0]}


% 4. SINGLE SYMBOLS

% Set theory

\mcm{\sub}{0}{\,\subseteq\,}
\mcm{\such}{0}{\:|\:}
\mcm{\without}{0}{\setminus}

% Category theory

\mcm{\ladj}{0}{\,\dashv\,}
\mcm{\of}{0}{\raisebox{0.08ex}{\ensuremath{\scriptstyle\circ}}}
\mcm{\sof}{0}{\raisebox{0.08ex}{\ensuremath{\scriptscriptstyle\circ}}}

% Misc

\mcm{\bdry}{0}{\partial}
\mcm{\blob}{0}{\raisebox{.3ex}{\ensuremath{\scriptscriptstyle{\bullet}}}}
\newcommand{\epsln}{\varepsilon}
\mcm{\implies}{0}{\,\Rightarrow\,}


% 5. CONSTANTS

% Roman

\mcm{\Hom}{0}{\mr{Hom}}
\mcm{\ob}{0}{\mr{ob}\,}		% As binary operation
\mcm{\op}{0}{\mr{op}}

% Italic

\mcm{\comp}{0}{\mi{comp}}
\mcm{\id}{0}{\mi{id}}
\mcm{\ids}{0}{\mi{ids}}

% Bold

\mcm{\Ab}{0}{\fcat{Ab}}
\mcm{\Alg}{0}{\fcat{Alg}}
\mcm{\Bicat}{0}{\fcat{Bicat}}
\mcm{\Bim}{1}{\fcat{Bim}(#1)}
\mcm{\Cat}{0}{\fcat{Cat}}
\mcm{\fc}{0}{\fcat{fc}}
\mcm{\Gph}{0}{\fcat{Gph}}
\mcm{\Graph}{0}{\fcat{Graph}}
\mcm{\Multicat}{0}{\fcat{Multicat}}
\mcm{\One}{0}{\fcat{1}}
\mcm{\Set}{0}{\fcat{Set}}
\mcm{\Span}{0}{\fcat{Span}}
\mcm{\Struc}{0}{\fcat{Struc}}
\mcm{\Sym}{0}{\fcat{Sym}}
\mcm{\Top}{0}{\fcat{Top}}
\mcm{\UBicat}{0}{\fcat{UBicat}}

% Blackboard bold

\mcm{\integers}{0}{\mathbb{Z}}


% 6. TUPLES

% Plain tuples

\mcm{\range}{2}{#1,\,\ldots\,,#2}
\mcm{\tuplebts}{1}{(#1)}
\mcm{\bftuple}{2}{\tuplebts{\range{#1}{#2}}}
\mcm{\tuple}{3}{\tuplebts{\range{#1,#2}{#3}}}

% Homs

\mcm{\eend}{2}{#1[#2]}
\mcm{\ehom}{3}{#1[#2,#3]}
\mcm{\ftrcat}{2}{[#1,#2]}


% 7. ARROWS

% Single arrows

\mcm{\goesto}{0}{\,\longmapsto\,}
\mcm{\goiso}{0}{\goby{\diso}}
\mcm{\monic}{0}{\rMonic}
\mcm{\og}{0}{\lTo}
\mcm{\ogby}{1}{\lTo^{#1}}

% Plural arrows

\mcm{\oppair}{2}{\pile{\rTo^{\scriptstyle #1}\\ \lTo_{\scriptstyle #2}}}
\mcm{\parpair}{2}{\pile{\rTo^{\scriptstyle #1}\\ \rTo_{\scriptstyle #2}}}
\mcm{\parpairu}{0}{\pile{\rTo\\ \rTo}}

% Slice objects

\mcm{\vslob}{3}
	{\left.
	\begin{diagram}[height=1.5em]
	#1		\\
	\dTo>{\,#2}	\\
	#3		\\
	\end{diagram}
	\right.}


% 8. PT DIAGRAMS

% Arrows

\newarrow{Equals}=====
\newarrow{Get}....>
\newarrow{Goesto}|---{->}
\newarrow{Incl}C--->
\newarrow{Mod}--+->
\newarrow{Monic}{vee}---{vee}
\newarrow{NT}===={=>}

% Environments

\newenvironment{tree}
	{\begin{diagram}[height=1em,width=.75em,abut,noPS,tight]}	
	{\end{diagram}}

% Trees

\newcommand{\dn}{\dLine}
\newcommand{\lt}[1]{\ldLine(#1,2)}
\newcommand{\rt}[1]{\rdLine(#1,2)}
\mcm{\node}{0}{\bullet}
\mcm{\enode}{0}{\circ}
\mcm{\nl}{1}{\stackrel{\textstyle #1}{\node}}


% Miscellaneous

\mcm{\diso}{0}{\sim}
\mcm{\vdiso}{0}{\wr}
\newcommand{\pullshape}
	{\setlength{\unitlength}{1em}
	\begin{picture}(2,5)(-1,-5)
	\put(0,-5){\line(1,1){1}}
	\put(0,-5){\line(-1,1){1}}
	\end{picture}}
\newcommand{\Spbk}{\overprint{\raisebox{-2.5em}{\pullshape}}}


% 9. DRAFTING

% \newcommand{\mpar}[1]{\marginpar{\raggedleft \textsl{\textsf{#1}}}}
% \newenvironment{plan}{\begin{quote}\sffamily\slshape}{\end{quote}}










