
\chapter{Notions of Monoidal Category}
\lbl{ch:monoidal}

\chapterquote{%
In one sense all he ever wanted to be was someone with many nicknames}{%
Marcus~\cite{Marcus}}


\noindent
The concept of monoidal category is in such widespread use that one might
expect---or hope, at least---that its formalization would be thoroughly
understood.  Nevertheless, it is not.  Here we look at five different
possible definitions, plus one infinite family of definitions, of monoidal
category.  We prove equivalence results between almost all of them.

Apart from the wish to understand a common mathematical structure, there is
a reason for doing this motivated by higher-dimensional category theory.  A
monoidal category in the traditional sense is, as observed
in~\ref{sec:bicats}, 
% in~\ref{eg:bicat-mon-cat}, 
the same thing as a bicategory with only one object.  Similarly, any
proposed definition of weak $n$-category gives rise to a notion of monoidal
category, defined as a one-object weak 2-category.  So if we want to be
able to compare the (many) proposed definitions%
%
\index{n-category@$n$-category!definitions of!comparison}
%
of weak $n$-category then
we will certainly need a firm grip on the various notions of monoidal
category and how they are related.  (This is something like a physicist's
toy model: a manageably low-dimensional version of a higher-dimensional
system.)  Of course, if two definitions of weak $n$-category happen to
induce equivalent definitions of monoidal category then this does not imply
their equivalence in the general case, but the surprising variety of
different notions of monoidal category means that it is a surprisingly good
test.

In the classical definition of monoidal category, any pair $(X_1, X_2)$ of
objects has a specified tensor product $X_1 \otimes X_2$, and there is also
a specified unit object.  In~\ref{sec:unbiased} we consider a notion of
monoidal category in which any sequence $X_1, \ldots, X_n$ of objects has a
specified product; these are called `unbiased monoidal categories'.  More
generally, one might have a category equipped with various different tensor
products of various (finite) arities, but as long there are enough
isomorphisms between derived products, this should make essentially no
difference.  We formalize and prove this in~\ref{sec:alg-notions}.  

The definitions described so far are `algebraic' in that tensor is an
operation.  For instance, in the classical definition, two objects $X_1$
and $X_2$ give rise to an actual, specified, object, $X_1 \otimes X_2$, not
just an isomorphism class of objects.  This goes somewhat against
intuition: when using products of sets, for instance, it is not of the
slightest importance to remember the standard set-theoretic definition of
ordered pair ($(x_1, x_2) = \{ \{x_1\}, \{x_1, x_2\} \}$, as it happens);
all that matters is the universal property of the product.  So perhaps it
is better to use a notion of monoidal category in which the tensor product
of objects is only defined up to canonical isomorphism.  Three such
`non-algebraic' notions are considered in~\ref{sec:non-alg-notions}, one
using multicategories and the other two (closely related) using simplicial
objects.

Almost everything we do for monoidal categories could equally be done for
bicategories, as discussed in~\ref{sec:notions-bicat}.  The extension is
mostly routine---there are few new ideas---but the explanations are a
little easier in the special case of monoidal categories.  

Some generality is sacrificed in the theorems asserting equivalence of
different notions of monoidal category: these are equivalences of the
categories of monoidal categories, but we could have made them equivalences
of 2-categories (in other words, included monoidal transformations).  I
have not aimed to exhaust the subject, or the reader.



\section{Unbiased monoidal categories}
\lbl{sec:unbiased}

A classical monoidal category is a category $A$ equipped with a functor
$A^n \go A$ for each of $n=2$ and $n=0$ (the latter being the unit object
$I$), together with coherence data (Definition~\ref{defn:mon-cat}).  An
unbiased%
%
\index{monoidal category!unbiased}
%
monoidal category is the same, but with $n$ allowed to be any
natural number.  So in an unbiased monoidal category any four objects $a_1,
a_2, a_3, a_4$ have a specified tensor product
\[
(a_1 \otimes a_2 \otimes a_3 \otimes a_4),
\]
but in a classical monoidal category there are only derived products such
as
\[
(a_1 \otimes a_2) \otimes (a_3 \otimes a_4), 
\  
((a_1 \otimes a_2) \otimes a_3) \otimes a_4,
\  
a_1 \otimes ((I \otimes (a_2 \otimes I)) \otimes (a_3 \otimes a_4)).
\]
The classical definition is `biased' towards arities $2$ and $0$: it gives
them a special status.  Here we eliminate the bias.  

The coherence data for an unbiased monoidal category consists of
isomorphisms such as
\[
((a_1 \otimes a_2 \otimes a_3) \otimes (a_4 \otimes (a_5)) \otimes a_6)
\goiso
(a_1 \otimes (a_2 \otimes a_3 \otimes a_4 \otimes a_5) \otimes a_6).
\]
In fact, all such isomorphisms can be built up from two families of special
cases (the $\gamma$ and $\iota$ of~\ref{defn:lax-mon-cat} below).  We
require, of course, that the coherence isomorphisms satisfy all sensible
axioms.

We will show in the next section that the unbiased and classical
definitions are equivalent, in a strong sense.  The unbiased definition
seems much the more natural (having, for instance, no devious coherence
axioms), and much more useful for the purposes of theory.  When verifying
that a particular example of a category has a monoidal structure, it is
sometimes easier to use one definition, sometimes the other; but the
equivalence result means that we can take our pick.

Here is the definition of unbiased monoidal category.  It is no extra work
to define at the same time `lax monoidal categories', in which the
coherence maps need not be invertible. 

\begin{defn}	\lbl{defn:lax-mon-cat}
A \demph{lax%
%
\index{monoidal category!lax}
%
monoidal category} $A$ (or properly, $(A, \otimes,
\gamma, \iota)$) consists of 
%
\begin{itemize}
\item a category $A$
\item for each $n\in \nat$, a functor $\otimes_n: A^n \go A$,%
% 
\glo{otimesn}
%  
called \demph{$n$-fold%
%
\index{n-fold@$n$-fold!tensor}
%
tensor} and written
\[
(a_1, \ldots, a_n) \goesto (a_1 \otimes \cdots \otimes a_n)
% 
\glo{otimesninfix}
%
\]
% 
\item for each $n, k_1, \ldots, k_n \in \nat$ and double sequence $((a_1^1,
\ldots, a_1^{k_1}), \ldots, (a_n^1, \ldots, a_n^{k_n}))$ of objects of $A$,
a map
\[
\begin{array}{rl}
\gamma_{((a_1^1, \ldots, a_1^{k_1}), \ldots, (a_n^1, \ldots,
a_n^{k_n}))}:&
((a_1^1 \otimes \cdots \otimes a_1^{k_1}) \otimes \cdots \otimes
(a_n^1 \otimes \cdots \otimes a_n^{k_n}))\\
&\go
(a_1^1 \otimes\cdots\otimes a_1^{k_1} \otimes\cdots\otimes a_n^1
\otimes\cdots\otimes a_n^{k_n})
\end{array}
% 
\glo{gammalaxmoncat}
% 
\]
\item for each object $a$ of $A$, a map
\[
\iota_a: a \go (a),
% 
\glo{iotalaxmoncat}
% 
\]
\end{itemize}
%
with the following properties:
%
\begin{itemize}
\item
$\gamma_{((a_1^1, \ldots, a_1^{k_1}), \ldots, (a_n^1, \ldots,
a_n^{k_n}))}$ is natural in each of the $a_i^j$'s, and $\iota_a$ is natural
in $a$

\item
associativity: for any $n, m_p, k_p^q \in \nat$ and triple sequence 
$((( a_{p,q,r} )_{r=1}^{k_p^q} )_{q=1}^{m_p} )_{p=1}^{n}$ of objects, the
diagram
\[
\begin{diagram}[tight,width=4em,shortfall=2em,noPS] %,leftflush]
&&
\begin{scriptarray}
\scriptstyle 
  (
      ( 
          ( a_{1, 1, 1} \otimes\cdots\otimes a_{1, 1, k_1^1} )
          \otimes\cdots\otimes
          ( a_{1, m_1, 1} \otimes\cdots\otimes a_{1, m_1, k_1^{m_1}} )
      )
      \otimes\cdots
\\
\scriptstyle
\otimes
      ( 
          ( a_{n, 1, 1} \otimes\cdots\otimes a_{n, 1, k_n^1} )
          \otimes\cdots\otimes
          ( a_{n, m_n, 1} \otimes\cdots\otimes a_{n, m_n, k_n^{m_n}} ) 
      ) 
  )
\end{scriptarray}
& & \\
&\ldTo<{( \gamma_{D_1} \otimes\cdots\otimes \gamma_{D_n} )}
& &\rdTo>{\gamma_{D'}} & \\
\begin{scriptarray}
\scriptstyle
( 
    ( a_{1, 1, 1} \otimes\cdots\otimes a_{1, m_1, k_1^{m_1}} )
    \otimes\cdots
\\
\scriptstyle
\otimes
    ( a_{n, 1, 1} \otimes\cdots\otimes a_{n, m_n, k_n^{m_n}} )
)
\end{scriptarray}
& & & & 
\begin{scriptarray}
\scriptstyle
(
    ( a_{1, 1, 1} \otimes\cdots\otimes a_{1, 1, k_1^1} )
    \otimes\cdots
\\
\scriptstyle
\otimes
    ( a_{n, m_n, 1} \otimes\cdots\otimes a_{n, m_n, k_n^{m_n}} )
) 
\end{scriptarray}
\\
&\rdTo<{\gamma_D} & &\ldTo>{\gamma_{D''}} & \\
& &
\scriptstyle
(
   a_{1, 1, 1} \otimes\cdots\otimes a_{n, m_n, k_n^{m_n}} 
) 
& & \\
\end{diagram}
\]
commutes, where the double sequences $D_p, D, D', D''$ are
%
\begin{eqnarray*}
D_p	&=& 
\bftuple{\bftuple{a_{p, 1, 1}}{a_{p, 1, k_p^1}}}{\bftuple{a_{p, m_p,
1}}{a_{p, m_p, k_p^{m_p}}}},	\\
D	&=&	
\bftuple{\bftuple{a_{1, 1, 1}}{a_{1, m_1, k_1^{m_1}}}}{\bftuple{a_{n, 1,
1}}{a_{n, m_n, k_n^{m_n}}}},	\\
D'	&=&	
(
\bftuple{( a_{1, 1, 1} \otimes\cdots\otimes a_{1, 1, k_1^1} )}{(
a_{1, m_1, 1} \otimes\cdots\otimes a_{1, m_1, k_1^{m_1}} )}
, \ldots,	\\
&&
\bftuple{( a_{n, 1, 1} \otimes\cdots\otimes a_{n, 1, k_n^1} 
)}{( a_{n, m_n, 1} \otimes\cdots\otimes a_{n, m_n, k_n^{m_n}} )} ),	\\
D''	&=&
\bftuple{\bftuple{a_{1, 1, 1}}{a_{1, 1, k_1^1}}}{\bftuple{a_{n, m_n,
1}}{a_{n, m_n, k_n^{m_n}}}}
\end{eqnarray*}
%
\item
identity: for any $n\in\nat$ and sequence \bftuple{a_1}{a_n} of objects, the
diagrams
%
\begin{eqnarray*}
\begin{diagram}[scriptlabels,size=2em]
( a_1 \otimes\cdots\otimes a_n )	
&\rTo^{( \iota_{a_1} \otimes\cdots\otimes \iota_{a_n} )}
&( ( a_1 ) \otimes\cdots\otimes ( a_n ) )	\\
&\rdTo<{1}	
&\dTo>{\gamma_{\bftuple{(a_1)}{(a_n)}}}	\\
&& ( a_1 \otimes\cdots\otimes a_n )	\\
\end{diagram}
%
\\ 
%
\begin{diagram}[scriptlabels,size=2em]
( ( a_1 \otimes\cdots\otimes a_n ) )
&\lTo^{\iota_{( a_1 \otimes\cdots\otimes a_n )}}
&( a_1 \otimes\cdots\otimes a_n )	\\
\dTo<{\gamma_{(\bftuple{a_1}{a_n})}}
&\ldTo>{1}	& \\
( a_1 \otimes\cdots\otimes a_n ) & & \\
\end{diagram}
\end{eqnarray*}
%
commute.
\end{itemize}
%
An \demph{unbiased%
%
\index{monoidal category!unbiased}
%
monoidal category} is a lax monoidal category $(A,
\otimes, \gamma, \iota)$ in which each $\gamma_{((a_1^1, \ldots,
a_1^{k_1}), \ldots, (a_n^1, \ldots, a_n^{k_n}))}$ and each $\iota_a$ is
an isomorphism.  An \demph{unbiased strict monoidal category} is a lax
monoidal category $(A, \otimes, \gamma, \iota)$ in which each of the
$\gamma$'s and $\iota$'s is an identity map.
\end{defn}

\begin{remarks}{rmks:u-moncat}
\item
The bark of the associativity axiom is far worse than its bite.  All it
says is that any two ways of removing brackets are equal: for
instance, that the diagram
%
\begin{diagram}[width=4em,tight]%[leftflush=2.5em,noPS]
 & &
\begin{array}{l}
  \{[(a \otimes b) \otimes (c \otimes d)]
  \\
  \otimes [(e \otimes f) \otimes (g \otimes h)]\} 
\end{array}
& & \\ 
 &\ldTo<{\{\gamma \otimes \gamma\}} & &\rdTo>{\gamma} & \\ 
\begin{array}{l}
  \{[a \otimes b \otimes c \otimes d] 
  \\
  \otimes [e \otimes f \otimes g \otimes h]\} 
\end{array}
& & & &
\begin{array}{l}
  \{(a \otimes b) \otimes (c \otimes d)
  \\
  \otimes (e \otimes f) \otimes (g \otimes h)\} 
\end{array}
\\ 
 &\rdTo<{\gamma} & &\ldTo>{\gamma} & \\
 & &\{a \otimes b \otimes c \otimes d \otimes e \otimes f \otimes g \otimes
 h \} & & \\ 
\end{diagram}
%
commutes.  This is exactly the role of the associativity axiom for a monad
such as `free semigroup' on \Set, as observed in Mac Lane \cite[VI.4, after
Proposition 1]{MacCWM}.

\item	\lbl{rmk:axioms-obvious}
The coherence%
%
\index{coherence!axioms}
%
axioms for an unbiased monoidal category are `canonical' and
rather obvious, in contrast to those for classical monoidal categories;
they are the same shape as the diagrams expressing the associativity and
unit axioms for a monoid or monad~(\ref{defn:monad}). 

\item	\lbl{rmk:strict-unbiased}
In an unbiased \emph{strict} monoidal category, the coherence axioms
(naturality, associativity and identity) hold automatically.  Clearly,
unbiased strict monoidal categories are in one-to-one correspondence
with ordinary strict monoidal categories.

\end{remarks}

We have given a completely explicit definition of unbiased monoidal
category, but a more abstract version is possible.  First recall that if
\cat{C} is a strict 2-category then there is a notion of a \demph{strict
2-monad}%
% 
\lbl{p:defn-2-monad}\index{two-monad@2-monad} 
%
$(T, \mu, \eta)$ on \cat{C}, and there are notions of \demph{strict},
\demph{weak} and \demph{lax algebra}%
%
\index{algebra!two-monad@for 2-monad}
%
for such a 2-monad.  (See Kelly and
Street~\cite{KSRE2}, for instance; terminology varies between authors.)  In
particular, `free strict monoidal category' is a strict 2-monad $(\blank^*,
\mu, \eta)$ on $\Cat$; the functor $\blank^*$ is given on objects of \Cat\
by
\[
A^* = \coprod_{n\in\nat} A^n.
\]
A (small) unbiased monoidal category is precisely a weak algebra for
this 2-monad.  Explicitly, this says that an unbiased monoidal
category consists of a category $A$ together with a functor $\otimes: A^*
\go A$ and natural isomorphisms
%
\begin{equation}
\label{eq:coh-transfs}
\begin{diagram}
A^{**}			&\rTo^{\mu_A}	&A^*		\\
\dTo<{\otimes^*}	&\nent \gamma	&\dTo>{\otimes}	\\
A^*			&\rTo_{\otimes}	&A		\\
\end{diagram}
\diagspace
\begin{diagram}[size=1.5em]
A	&		&\rTo^{\eta_A}	&		&A^*		\\
	&\rdTo(4,4)<{1}	&		&\nent \iota	&		\\
	&		&		&		&\dTo>{\otimes}	\\
	& 		&		&		&		\\
	&		&		&		&A		\\
\end{diagram}
\end{equation}
%
satisfying associativity and identity axioms: the diagrams
\[
\begin{diagram}[size=2em]
\otimes\of\otimes^{*}\of\otimes^{**}	&\rTo^{\gamma*1}&\otimes\of\mu_{A}\of\otimes^{**}
&\rEquals	&\otimes\of\otimes^{*}\of\mu_{A^*}	\\
\dTo<{1*\gamma^*}	&	&	&	&\dTo>{\gamma*1}\\
\otimes\of\otimes^{*}\of\mu^*_{A}	&\rTo_{\gamma *1}
&\otimes\of\mu_{A}\of\mu^*_{A}	&\rEquals
&\otimes\of\mu_{A}\of\mu_{A^*}\\
\end{diagram}
\]
\[
\begin{diagram}[size=2em]
\otimes\of 1_{A}^*		&\rTo^{1*\iota^*}	&
\otimes\of\otimes^{*}\of\eta_{A}^*			\\
				&\rdTo<1		&
\dTo>{\gamma*1}						\\
				&			&
\otimes\of\mu_{A}\of\eta^*_{A}				\\
\end{diagram}
\diagspace
\begin{diagram}
\otimes\of\otimes^{*}\of\eta_{A^*}	&\rEquals	&
\otimes\of\eta_{A}\of\otimes	&\lTo^{\iota*1}	&1_A \of\otimes	\\
\dTo<{\gamma*1}				&		&
				&\ldTo(4,2)>1	&		\\
\otimes\of\mu_{A}\of\eta_{A^*}		&		&
				&		&		\\
\end{diagram}
\]
commute.  This may easily be verified.  Similarly, a lax monoidal category
is precisely a lax algebra for the 2-monad (for `lax' means that the
natural transformations $\gamma$ and $\iota$ are no longer required to be
isomorphisms) and an unbiased strict monoidal category is precisely a
strict algebra (for `strict' means that $\gamma$ and $\iota$ are required
to be identities, so that the diagrams~\bref{eq:coh-transfs} containing
them commute).

A different abstract way of defining unbiased monoidal category will be
explored in~\ref{sec:alg-notions}. 

The next step is to define maps between (lax and) unbiased monoidal
categories.  Again, we could use the language of 2-monads to do this, but
opt instead for an explicit definition.

\begin{defn}	\lbl{defn:u-lax-mon-ftr} 
Let $A$ and $A'$ be lax monoidal categories.  Write $\otimes$ for the
tensor and $\gamma$ and $\iota$ for the coherence maps in both categories.
A \demph{lax monoidal%
%
\index{monoidal functor!unbiased}
%
functor} $(P, \pi): A \go A'$ consists of
\begin{itemize}
\item
a functor $P: A \go A'$
\item
for each $n\in\nat$ and sequence $a_1, \ldots, a_n$ of objects of $A$, a
map 
\[
\pi_{a_1, \ldots, a_n}:
(Pa_1 \otimes\cdots\otimes Pa_n)
\go
P(a_1 \otimes\cdots\otimes a_n),
\]
\end{itemize}
%
such that
%
\begin{itemize}
\item
$\pi_{a_1, \ldots, a_n}$ is natural in each $a_i$
\item
for each $n, k_i \in \nat$ and double sequence
\bftuple{\bftuple{a_1^1}{a_1^{k_1}}}{\bftuple{a_n^1}{a_n^{k_n}}} of objects
of $A$, the diagram
\[
\begin{diagram}[scriptlabels,width=2em]
\begin{scriptarray}
  \scriptstyle
  (
     ( Pa_1^1 \otimes\cdots\otimes Pa_1^{k_1} )
     \otimes\cdots
  \\
  \scriptstyle
     \otimes 
     ( Pa_n^1 \otimes\cdots\otimes Pa_n^{k_n} )
  )
\end{scriptarray}
&\rTo^{\gamma_{\bftuple{\bftuple{Pa_1^1}{Pa_1^{k_1}}}{\bftuple{Pa_n^1}{Pa_n^{k_n}}}}} 
&
\scriptstyle
( Pa_1^1 \otimes\cdots\otimes Pa_n^{k_n} )	\\
\dTo>{( \pi_{\bftuple{a_1^1}{a_1^{k_1}}} \otimes\cdots\otimes
        \pi_{\bftuple{a_n^1}{a_n^{k_n}}}  )}
& & \\
\begin{scriptarray}
  \scriptstyle
  (
      P ( a_1^1 \otimes\cdots\otimes a_1^{k_1} )
      \otimes\cdots
  \\
  \scriptstyle
  \otimes
      P ( a_n^1 \otimes\cdots\otimes a_n^{k_n} )
  )
\end{scriptarray}
&&\dTo<{\pi_{\bftuple{a_1^1}{a_n^{k_n}}}}	\\
\dTo>{\pi_{\bftuple{( a_1^1 \otimes\cdots\otimes a_1^{k_1} )}{( a_n^1
\otimes\cdots\otimes a_n^{k_n} )}}} 
& & \\
\begin{scriptarray}
  \scriptstyle
  P (
       ( a_1^1 \otimes\cdots\otimes a_1^{k_1} )
       \otimes\cdots
  \\
  \scriptstyle
  \otimes
       ( a_n^1 \otimes\cdots\otimes a_n^{k_n} )
    )
\end{scriptarray}
&\rTo_{P\gamma_{\bftuple{\bftuple{a_1^1}{a_1^{k_1}}}{\bftuple{a_n^1}{a_n^{k_n}}}}}
&
\scriptstyle
P ( a_1^1 \otimes\cdots\otimes a_n^{k_n} ) \\
\end{diagram}
\]
commutes
\item
for each 1-cell $a$, the diagram
%
\begin{diagram}[size=2em]
Pa	&\rTo^{\iota_{Pa}}	&( Pa )	\\
\dEquals&			&\dTo>{\pi_a}	\\
Pa	&\rTo_{P\iota_a}	&P( a )	\\
\end{diagram}
%
commutes.
\end{itemize}
%
A \demph{weak monoidal functor} is a
lax monoidal functor $(P, \pi)$ for which each of the maps
$\pi_{a_1, \ldots, a_n}$ is an isomorphism.  A \demph{strict monoidal
functor} is a lax monoidal functor $(P, \pi)$ for which each of the maps
$\pi_{a_1, \ldots, a_n}$ is an identity map (in which case $P$ preserves
composites and identities strictly).
\end{defn}

We remarked in~\ref{rmks:u-moncat}\bref{rmk:axioms-obvious} that the
coherence%
%
\index{coherence!axioms}
%
axioms for an unbiased monoidal category were rather obvious,
having the shape of the axioms for a monoid or monad.  Perhaps the
coherence axioms for an unbiased lax monoidal functor are a little less
obvious; they are, however, the same shape as the axioms for a lax map of
monads ($=$ monad functor, p.~\pageref{p:lax-map-of-monads}), and in any
case seem `canonical'.

Lax monoidal functors can be composed in the evident way, and the weak and
strict versions are closed under this composition.  There are also the
evident identities.  So we obtain $3\times 3 = 9$ possible categories,
choosing one of `strict', `weak' or `lax' for both the objects and the
maps.  In notation explained in a moment, the inclusions of subcategories
are as follows:
\[
\begin{diagram}[height=1.2em]
\fcat{LaxMonCat}_\mr{str}	&\sub	&\fcat{LaxMonCat}_\mr{wk}	&\sub
&\fcat{LaxMonCat}_\mr{lax} \\
\rotsub	&	&\rotsub	&	&\rotsub	\\	
\UMCstr	&\sub	&\UMCwk		&\sub	&\UMClax	\\
\rotsub	&	&\rotsub	&	&\rotsub	\\	
\fcat{StrMonCat}_\mr{str}	&\sub	&\fcat{StrMonCat}_\mr{wk}	&\sub
&\fcat{StrMonCat}_\mr{lax}. \\
\end{diagram}%
% 
\glo{ninemoncat}%
%
\index{monoidal category!nine categories of}%
\]
For all three categories in the bottom (respectively, middle or top) row,
the objects are small strict (respectively, unbiased or lax) monoidal
categories.  For all three categories in the left-hand (respectively,
middle or right-hand) column, the maps are strict (respectively, weak or
lax) monoidal functors.  It is easy to check that the three categories in
the bottom row are isomorphic to the corresponding three categories in the
classical definition.

Of the nine categories, the three on the bottom-left to top-right diagonal
are the most conceptually natural: a level of strictness has been chosen
and stuck to.  In this chapter our focus is on the middle entry, $\UMCwk$,
where everything is weak.

Note that this $3\times 3$ picture does not appear in the classical,%
%
\index{monoidal category!unbiased vs. classical@unbiased \vs.\ classical}
%
`biased', approach to monoidal categories.  There the top row is obscured,
as there is no very satisfactory way to laxify%
%
\index{monoidal category!lax}
%
the classical definition of
monoidal category.  Admittedly it is possible to drop the condition that
the associativity maps $(a\otimes b) \otimes c \go a\otimes (b\otimes c)$
and unit maps $a\otimes 1 \go a \og 1\otimes a$ are isomorphisms (as
Borceux%
%
\index{Borceux, Francis}
%
does in his \cite{Borx1}, just after Definition 7.7.1), but somehow
this does not seem quite right.

To complete the picture, and to make possible the definition of equivalence
of unbiased monoidal categories, we define transformations.

\begin{defn}
Let $(P, \pi), (Q, \chi): A \go A'$ be lax monoidal functors between lax
monoidal categories.  A \demph{monoidal transformation}%
%
\index{monoidal transformation!unbiased}
%
$\sigma: (P, \pi)
\go (Q, \chi)$ is a natural transformation
\[
A \ctwomult{P}{Q}{\sigma} A'
\]
such that for all $a_1, \ldots, a_n \in A$, the diagram
%
\begin{diagram}[size=2em]
(Pa_1 \otimes \cdots \otimes Pa_n)	&
\rTo^{\pi_{a_1, \ldots, a_n}}		&
P(a_1 \otimes\cdots\otimes a_n)		\\
\dTo<{(\sigma_{a_1} \otimes\cdots\otimes \sigma_{a_n})}	&	&
\dTo>{\sigma_{(a_1 \otimes\cdots\otimes a_n)}}	\\
(Qa_1 \otimes \cdots \otimes Qa_n)	&
\rTo_{\chi_{a_1, \ldots, a_n}}		&
Q(a_1 \otimes\cdots\otimes a_n)		\\
\end{diagram}
% 
commutes.
\end{defn}
%
(This time, there is only one possible level of strictness.)

Monoidal transformations can be composed in the expected ways, so that the
nine categories above become strict 2-categories.  In particular, $\UMCwk$
is a 2-category, so~(\ref{propn:bicat-eqv-eqv}) there is a notion of
equivalence of unbiased monoidal categories.  Explicitly, $A$ and $A'$ are
\demph{equivalent}%
%
\index{equivalence!unbiased monoidal categories@of unbiased monoidal categories} 
%
if there exist weak monoidal functors and invertible
monoidal transformations
\[
\begin{diagram}
A &\pile{\rTo^{(P, \pi)} \\ \lTo_{(Q, \chi)}} &A',
\end{diagram}
\diagspace
A \ctwomult{1}{(Q, \chi) \of (P, \pi)}{\eta} A,
\diagspace
A' \ctwomult{(P, \pi) \of (Q, \chi)}{1}{\epsln} A',
\]
and it makes no difference if we insist that $((P, \pi), (Q, \chi), \eta,
\epsln)$ forms an adjunction in $\UMCwk$.  As we might expect from the case
of classical monoidal categories~(\ref{propn:mon-eqv-eqv}), there is the
following alternative formulation.
% 
\begin{propn}
Let $A$ and $A'$ be unbiased monoidal categories.  Then $A$ and $A'$ are
equivalent if and only if there exists a weak monoidal functor $(P, \pi): A
\go A'$ whose underlying functor $P$ is full, faithful and essentially
surjective on objects.
\end{propn}

\begin{proof} 
\latin{Mutatis mutandis}, this is the same as the proof
of~\ref{propn:mon-eqv-eqv}. 
% 
\done
\end{proof}

We can now state and prove a coherence%
%
\index{coherence!monoidal categories@for monoidal categories!unbiased} 
%
theorem for unbiased
monoidal categories.
% 
\begin{thm}	\lbl{thm:eqv-coh-umc}
Every unbiased monoidal category is equivalent to a strict monoidal category.
\end{thm}
% 
\begin{proof}
Let $A$ be an unbiased monoidal category.  We construct an (unbiased)
strict monoidal category $\st(A)$,%
% 
\glo{strictcover}
% 
the \demph{strict%
%
\index{cover, strict}
%
cover} of $A$, and a
weak monoidal functor $(P, \pi): \st(A) \go A$ whose underlying functor $P$
is full, faithful and essentially surjective on objects.  By the last
proposition, this is enough.

An object of $\st(A)$ is a finite sequence $(a_1, \ldots, a_n)$ of objects
of $A$ (with $n\in\nat$).  A map $(a_1, \ldots, a_n) \go (b_1, \ldots,
b_m)$ in $\st(A)$ is a map $(a_1 \otimes\cdots\otimes a_n) \go (b_1
\otimes\cdots\otimes b_m)$ in $A$, and composition and identities in
$\st(A)$ are as in $A$.  The tensor in $\st(A)$ is given on objects by
concatenation: 
\[
((a_1^1, \ldots, a_1^{k_1}) \otimes\cdots\otimes (a_n^1, \ldots,
a_n^{k_n}))
=
(a_1^1, \ldots, a_1^{k_1}, \ldots, a_n^1, \ldots, a_n^{k_n}).
\]
To define the tensor of maps in $\st(A)$, take maps 
%
\begin{eqnarray*}
(a_1^1 \otimes\cdots\otimes a_1^{k_1})	&
\goby{f_1}				&
(b_1^1 \otimes\cdots\otimes b_1^{l_1})	\\
\vdots	&\vdots	&\vdots	\\
(a_n^1 \otimes\cdots\otimes a_n^{k_n})	&
\goby{f_n}				&
(b_n^1 \otimes\cdots\otimes b_n^{l_n})
\end{eqnarray*}
%
in $A$; then their tensor product in $\st(A)$ is the composite map
%
\begin{eqnarray*}
(a_1^1 \otimes\cdots\otimes a_n^{k_n})	&
\goby{\gamma^{-1}}			&
((a_1^1 \otimes\cdots\otimes a_1^{k_1}) \otimes\cdots\otimes (a_n^1
\otimes\cdots\otimes a_n^{k_n}))	\\
					&
\goby{(f_1 \otimes\cdots\otimes f_n)}	&
((b_1^1 \otimes\cdots\otimes b_1^{l_1}) \otimes\cdots\otimes (b_n^1
\otimes\cdots\otimes b_n^{l_n}))	\\
					&
\goby{\gamma}				&
(b_1^1 \otimes\cdots\otimes b_n^{l_n})
\end{eqnarray*}
%
in $A$.  It is absolutely straightforward and not too arduous to check that
$\st(A)$ with this tensor forms a strict monoidal category.  

The functor $P: \st(A) \go A$ is defined by
$
(a_1, \ldots, a_n) \goesto (a_1 \otimes\cdots\otimes a_n)
$
on objects, and `is the identity' on maps---in other words, performs the
identification
\[
\st(A)((a_1, \ldots, a_n), (b_1, \ldots, b_m))
=
A((a_1 \otimes\cdots\otimes a_n), (b_1 \otimes\cdots\otimes b_m)).
\]
For each double sequence 
\[
D = ((a_1^1, \ldots, a_1^{k_1}), \ldots, (a_n^1, \ldots, a_n^{k_n}))
\]
of objects of $A$, the isomorphism
\[
\pi_D: 
(P(a_1^1, \ldots, a_1^{k_1}) \otimes\cdots\otimes 
P(a_n^1, \ldots, a_n^{k_n}))
\goiso
P((a_1^1 \otimes\cdots\otimes a_1^{k_1}) \otimes\cdots\otimes (a_n^1
\otimes\cdots\otimes a_n^{k_n}))
\]
is simply $\gamma_D$.  It is also quick and straightforward to check that
$(P,\pi)$ is a weak monoidal functor.

$P$ is certainly full and faithful.  Moreover, for each $a\in A$ we have
an isomorphism
\[
\iota_a: a \goiso (a) = Pa,
\]
and this proves that $P$ is essentially surjective on objects.
\done
\end{proof}

As discussed on p.~\pageref{p:coherence-discussion}, coherence%
%
\index{coherence|(}
%
theorems
take various forms, usually falling under one of two headings: `all
diagrams commute' or `every weak thing is equivalent to a strict thing'.
The one here is of the latter type.  In~\ref{sec:alg-notions} we will prove
a coherence theorem for unbiased monoidal categories that is essentially of
the former type.

The proof above was adapted from Joyal%
%
\index{Joyal, Andr\'e}
%
and Street~\cite[p.~29]{JSBTC}.%
%
\index{Street, Ross!coherence for monoidal categories@on coherence for monoidal categories}
%
There the $\st$ construction was done for \emph{classical}%
%
\index{monoidal category!unbiased vs. classical@unbiased \vs.\ classical}
%
monoidal
categories $A$, for which the situation is totally different.  What happens
is that the $n$-fold tensor product used in the definition of both $\st(A)$
and $P$ must be replaced by some derived, non-canonical, $n$-fold tensor
product, such as
\[
(a_1, \ldots, a_n) 
\goesto 
a_1 \otimes (a_2 \otimes (a_3 \otimes \cdots \otimes (a_{n-1} \otimes a_n)
\cdots ));
\]
and then in order to define both $\pi$ and the tensor product of maps in
$\st(A)$, it is necessary to use coherence isomorphisms such as
\[
\begin{array}{rl}
&
(a_1 \otimes (a_2 \otimes (a_3 \otimes a_4)))
\ \otimes\ 
(a_5 \otimes (a_6 \otimes a_7))
\\
\goiso &
a_1 \otimes (a_2 \otimes (a_3 \otimes (a_4 \otimes (a_5 \otimes 
(a_6 \otimes a_7))))) .
\end{array}
\]
It would be folly to attempt to define these coherence isomorphisms, and
prove that $\st(A)$ and $(P, \pi)$ have the requisite properties, without
the aid of a coherence theorem for monoidal categories.  That is, the work
required to do this would be of about the same volume and kind as the work
involved in the syntactic proof of the `all diagrams commute' coherence
theorem, so one might as well have proved that coherence theorem anyway.
In contrast, the proof that every \emph{unbiased} monoidal category is
equivalent to a strict one is easy, short, and needs no supporting results.

This does not, however, provide a short cut to proving any kind of
coherence result for classical monoidal categories.  In the next section we
will see that unbiased and classical monoidal categories are essentially
the same, and it then follows from Theorem~\ref{thm:eqv-coh-umc} that every
classical monoidal category is equivalent to a strict one.  However, as
with any serious undertaking involving classical monoidal categories, the
proof that they are the same as unbiased ones is close to impossible
without the use of a coherence theorem.%
%
\index{coherence|)}
%






\section{Algebraic notions of monoidal category}
\lbl{sec:alg-notions}


We have already seen two notions of monoidal category: classical and
unbiased.  In one of them there was an $n$-fold tensor product just for
$n\in\{0,2\}$, and in the other there was an $n$-fold tensor product for
all $n\in\nat$.  But what happens if we take a notion of monoidal category
in which there is an $n$-fold tensor product for each $n$ lying in some
other subset of $\nat$?  More generally, what if we allow any number of
different $n$-fold tensors (zero, one, or more) for each value of
$n\in\nat$?  

For instance, we might choose to take a notion of monoidal category in
which there are 6 unit objects, a single 3-fold tensor product, 8 11-fold
tensor products, and $\aleph_4$ 38-fold tensor products.  Just as long as
we add in enough coherence isomorphisms to ensure that any two $n$-fold
tensor products built up from the given ones are canonically isomorphic,
this new notion of monoidal category ought to be essentially the same as
the classical notion.  

This turns out to be the case.  Formally, we start with a `signature'%
%
\index{signature}
%
$\Sigma \in \Set^\nat$.  (In the example this was given by $\Sigma(0)=6$,
$\Sigma(1)=\Sigma(2)=0$, $\Sigma(3)=1$, and so on.)  From this we define
the category $\Sigma\hyph\MCwk$ of `$\Sigma$-monoidal%
%
\index{Sigma-monoidal category@$\Sigma$-monoidal category}%
%
\index{monoidal category!Sigma-@$\Sigma$-}
%
categories' and weak
monoidal functors between them.  We then show that, up to equivalence,
$\Sigma\hyph\MCwk$ is independent of the choice of $\Sigma$, assuming only
that $\Sigma$ is large enough that we can build at least one $n$-fold
tensor product for each $n\in\nat$.  We also show that the category $\MCwk$
of classical monoidal categories is isomorphic to $\Sigma\hyph\MCwk$ for a
certain value of $\Sigma$, and that the same goes for the unbiased version
$\UMCwk$ (for a different value of $\Sigma$); it follows that the
classical%
%
\index{monoidal category!unbiased vs. classical@unbiased \vs.\ classical}
%
and unbiased definitions are equivalent.

In the introduction to this chapter, I argued that comparing definitions of
monoidal category is important for understanding higher-dimensional
category theory.  Here is a further reason why it is important, specific to
this particular, `algebraic', family of definitions of monoidal category.

Consider the definition of monoid.  Usually a monoid is defined as a set
equipped with a binary operation and a nullary operation, satisfying
associativity and unit equations.  `Weakening'%
%
\index{weakening!presentation-sensitivity of}
%
or `categorifying'%
%
\index{categorification!presentation-sensitivity of}
%
this
definition, we obtain the classical definition of (weak) monoidal category.
However, this process of categorification is dependent on presentation.%
%
\index{presentation, sensitivity of categorification to}
%
That is, we could equally well have defined a monoid as a set equipped with
one $n$-ary operation for each $n\in\nat$, satisfying appropriate
equations, and categorifying \emph{this} gives a different notion of
monoidal category---the unbiased one.  So different presentations of
the same 0-dimensional theory (monoids) give, under this process of
categorification, different 1-dimensional theories (of monoidal category).

Thus, our purpose is to show that in this particular situation, the
presentation-sensitivity of categorification disappears when we work up to
equivalence.

More generally, a fully-developed theory of weak $n$-categories might
include a formal process of weakening, which would take as input a theory
of strict structures and give as output a theory of weak structures.  If
the weakening process depended on how the theory of strict structures was
presented, then we would have to ask whether different presentations always
gave equivalent theories of weak structures.  We might hope so; but who
knows?

Our first task is to define $\Sigma$-monoidal categories and maps between
them, for an arbitrary $\Sigma \in \Set^\nat$.  Here it is in outline.  A
$\Sigma$-monoidal category should be a category $A$ equipped with a tensor
product $\otimes_\sigma: A^n \go A$ for each $\sigma\in\Sigma(n)$.  There
should also be a coherence isomorphism between any pair of derived $n$-fold
tensors: for instance, there should be a specified isomorphism
\[
\otimes_{\sigma_1} (\otimes_{\sigma_2} (a_1, a_2, a_3), a_4, 
\otimes_{\sigma_3} (a_5) ) 
\goiso
\otimes_{\sigma_4} (a_1, a_2, \otimes_{\sigma_5} (a_3, a_4), a_5 )
\]
for any
\[
\sigma_1, \sigma_2 \in \Sigma(3), \sigma_3 \in \Sigma(1),
\sigma_4 \in \Sigma(4), \sigma_5 \in \Sigma(2)
\]
and any objects $a_1, \ldots, a_5$ of $A$.  This isomorphism is naturally
depicted as
%
\begin{equation}	\label{eq:Sigma-mon-cat-iso}
% \drmk{Picture with one tree $\goiso$ another.  Labelled with $\sigma_i$'s
% and $a_i$'s.}
\begin{centredpic}
\begin{picture}(10,6.2)(-0.5,0)
% bottom layer
\put(4.5,0){\line(0,1){1.5}}
% middle layer
\cell{4.5}{1.5}{c}{\vx}
\put(4.5,1.5){\line(-2,1){3}}
\put(4.5,1.5){\line(0,1){1.5}}
\put(4.5,1.5){\line(2,1){3}}
% top layer
\cell{1.5}{3}{c}{\vx}
\put(1.5,3){\line(-1,1){1.5}}
\put(1.5,3){\line(0,1){1.5}}
\put(1.5,3){\line(1,1){1.5}}
\cell{7.5}{3}{c}{\vx}
\put(7.5,3){\line(0,1){1.5}}
% sigma_i labels
\cell{4.8}{1.5}{tl}{\sigma_1}
\cell{1.2}{3}{r}{\sigma_2}
\cell{7.8}{3}{l}{\sigma_3}
% a_i labels
\cell{0}{4.7}{b}{a_1}
\cell{1.5}{4.7}{b}{a_2}
\cell{3}{4.7}{b}{a_3}
\cell{4.5}{3.2}{b}{a_4}
\cell{7.5}{4.7}{b}{a_5}
\end{picture}
\end{centredpic}
%
\diagspace
\goiso
\diagspace
%
\begin{centredpic}
\begin{picture}(9.2,6.2)(0,0)
% bottom layer
\put(4.5,0){\line(0,1){1.5}}
% middle layer
\cell{4.5}{1.5}{c}{\vx}
\put(4.5,1.5){\line(-3,1){4.5}}
\put(4.5,1.5){\line(-1,1){1.5}}
\put(4.5,1.5){\line(1,1){1.5}}
\put(4.5,1.5){\line(3,1){4.5}}
% top layer
\cell{6}{3}{c}{\vx}
\put(6,3){\line(-1,1){1.5}}
\put(6,3){\line(1,1){1.5}}
% sigma_i labels
\cell{4.8}{1.5}{tl}{\sigma_4}
\cell{6.3}{3}{l}{\sigma_5}
% a_i labels
\cell{0}{3.2}{b}{a_1}
\cell{3}{3.2}{b}{a_2}
\cell{4.5}{4.7}{b}{a_3}
\cell{7.5}{4.7}{b}{a_4}
\cell{9}{3.2}{b}{a_5}
\end{picture}
\end{centredpic}.
\end{equation}
%
Recall from~\ref{sec:om-further} that labelled trees arise as operations in
free operads: the two derived tensor products drawn above are elements of
$(F\Sigma)(5)$, where $F\Sigma$ is the free operad on $\Sigma$.  Hence
$F\Sigma$ is the operad of (derived) tensor operations.  To obtain the
coherence isomorphisms, replace the set $(F\Sigma)(n)$ by the indiscrete
category $I((F\Sigma)(n))$ whose objects are the elements of
$(F\Sigma)(n)$; then the picture above shows a typical map in
$I((F\Sigma)(5))$.

Also recall from p.~\pageref{p:defn-V-Operad} that there is a notion of
`$\cat{V}$-operad' for any symmetric monoidal category \cat{V}, and an
accompanying notion of an algebra (in \cat{V}) for any \cat{V}-operad.  The
categories $I((F\Sigma)(n))$ form a \Cat-operad, an algebra for which
consists of a category $A$, a functor $A^n \go A$ for each derived $n$-fold
tensor operation, and a natural isomorphism between any two functors $A^n
\go A$ so arising, all fitting together coherently.
% (The coherence is encoded in the uniqueness of maps in an indiscrete
% category.)  
This is exactly what we want a $\Sigma$-monoidal category to
be.

Precisely, define the functor
\[
\begin{array}{rrcl}
\blank\hyph\MCwk: 	&\Set^\nat	&\go	&\CAT^\op	\\
			&\Sigma		&\goesto&\Sigma\hyph\MCwk
\end{array}
\]%
% 
\glo{blankMonCatwk}%
% 
to be the composite of the functors
\[
\Set^\nat \goby{F} 
\Set\hyph\Operad \goby{I_*}
\Cat\hyph\Operad \goby{\Alg_\mr{wk}}
\CAT^\op,
\]%
% 
\glo{CatOperad}\glo{Algwk}%
% 
where the terms involved are now defined in turn.
%
\begin{itemize}
\item $\Set\hyph\Operad$ is the category of ordinary (\Set-)operads,
usually just called $\Operad$~(\ref{defn:plain-opd}).
\item $\Cat\hyph\Operad$ is the category of \Cat-operads
(p.~\pageref{p:defn-V-Operad}).
\item $F$ is the free operad functor (p.~\pageref{p:free-mti-ftr}).
\item $I: \Set \go \Cat$ is the functor assigning to each set the
indiscrete%
%
\index{category!indiscrete}
%
category on it (p.~\pageref{p:indiscrete}), and since $I$
preserves products, it induces a functor $I_*: \Set\hyph\Operad \go
\Cat\hyph\Operad$.
\item For a \Cat-operad%
%
\index{Cat-operad@$\Cat$-operad}
% 
$R$, the (large) category $\Alg_\mr{wk}(R)$
consists of $R$-algebras and \demph{weak
maps} between them.  So by
definition, an object of $\Alg_\mr{wk}(R)$ is a category $A$ together with
a sequence $(R(n) \times A^n \goby{\act{n}} A)_{n\in\nat}$ of functors,
compatible with the composition and identities of the operad $R$, and a map
$A \go A'$ in $\Alg_\mr{wk}(R)$ is a functor $P: A \go A'$ together with a
natural isomorphism
%
\begin{diagram}
R(n) \times A^n		&\rTo^{\act{n}}	&A	\\
\dTo<{1 \times P^n}	&\nent \pi_n	&\dTo>P	\\
R(n) \times A'^n	&\rTo_{\act{n}}	&A'	\\
\end{diagram}
%
for each $n\in\nat$, satisfying the coherence axioms in
Fig.~\ref{fig:wk-alg-coh}.  

\begin{figure}
\[
\begin{array}{c}
\begin{array}{rl}
&
\begin{diagram}[scriptlabels]
\begin{scriptarrayc}
  \scriptstyle
  R(n) \times \{ R(k_1) \times\cdots\times R(k_n) \} 
  \\
  \scriptstyle
  \times A^{k_1 + \cdots + k_n}	
\end{scriptarrayc}						&
\rTo^{1 \times \act{k_1} \times\cdots\times \act{k_n}}		&
\scriptstyle
R(n) \times A^n							&
\rTo^{\act{n}}							&
\scriptstyle
A								\\
\dTo<{1\times P^{k_1 + \cdots + k_n}}				&
\nent \scriptstyle
1\times \pi_{k_1} \times\cdots\times \pi_{k_n}		&
\dTo~{1 \times P^n}						&
\nent \scriptstyle
\pi_n							&
\dTo>P								\\
\begin{scriptarrayc}
  \scriptstyle
  R(n) \times \{ R(k_1) \times\cdots\times R(k_n) \} 
  \\
  \scriptstyle
  \times A'^{k_1 + \cdots + k_n}
\end{scriptarrayc}						&
\rTo_{1 \times \act{k_1} \times\cdots\times \act{k_n}}		&
\scriptstyle
R(n) \times A'^n							&
\rTo_{\act{n}}							&
\scriptstyle
A'								\\
\end{diagram}
\\
\\
\\
=
&
\begin{diagram}[scriptlabels]
\begin{scriptarrayc}
  \scriptstyle
  R(n) \times \{ R(k_1) \times\cdots\times R(k_n) \} 
  \\
  \scriptstyle
  \times A^{k_1 + \cdots + k_n}
\end{scriptarrayc}						&
\rTo^{\comp \times 1}						&
\scriptstyle
R(k_1 + \cdots + k_n) \times A^{k_1 + \cdots + k_n}			&
\rTo^{\act{k_1 + \cdots + k_n}}					&
\scriptstyle
A								\\
\dTo<{1 \times P^{k_1 + \cdots + k_n}}				&
\neeq								&
\dTo~{1 \times P^{k_1 + \cdots + k_n}}				&
\nent \scriptstyle
\pi_{k_1 + \cdots + k_n}					&
\dTo>P								\\
\begin{scriptarrayc}
  \scriptstyle
  R(n) \times \{ R(k_1) \times\cdots\times R(k_n) \} 
  \\
  \scriptstyle
  \times A'^{k_1 + \cdots + k_n}
\end{scriptarrayc}						&
\rTo_{\comp \times 1}						&
\scriptstyle
R(k_1 + \cdots + k_n) \times A'^{k_1 + \cdots + k_n}		&
\rTo_{\act{k_1 + \cdots + k_n}}					&
\scriptstyle
A'								\\
\end{diagram}
\end{array}
\\
\\
\\
\\
\begin{diagram}[scriptlabels]
\scriptstyle
1 \times A^1		&
\rTo^\diso	&
\scriptstyle
A	\\
\dTo<{1 \times P^1}	&
\neeq		&
\dTo>P	\\
\scriptstyle
1 \times A'^1		&
\rTo_\diso	&
\scriptstyle
A'	\\
\end{diagram}
\diagspace
=
\diagspace
\begin{diagram}[scriptlabels]
\scriptstyle
1 \times A^1		&
\rTo^{\ids \times 1}	&
\scriptstyle
R(1) \times A^1	&
\rTo^{\act{1}}		&
\scriptstyle
A			\\
\dTo<{1 \times P^1}	&
\neeq			&
\dTo~{1 \times P^1}	&
\nent \scriptstyle
\pi_1		&
\dTo>P			\\
\scriptstyle
1 \times A'^1		&
\rTo_{\ids \times 1}	&
\scriptstyle
R(1) \times A'^1	&
\rTo_{\act{1}}		&
\scriptstyle
A'			\\
\end{diagram}
\end{array}
\]
\caption{Coherence axioms for a weak map of $R$-algebras}
\label{fig:wk-alg-coh}
\end{figure}

\item For a map $H: R \go S$ of \Cat-operads, the induced functor $H^*:
\Alg_\mr{wk}(S) \go \Alg_\mr{wk}(R)$ is defined by composition with $H$: if
$A$ is an $S$-algebra then the resulting $R$-algebra has the same
underlying category and $R$-action given by
\[
R(n) \times A^n \goby{H_n \times 1} S(n) \times A^n \goby{\act{n}} A.
\]
\end{itemize}

In the construction above we used \emph{weak} maps between algebras for a
\Cat-operad, but we could just as well have used \demph{lax maps} (by
dropping the insistence that the natural transformations $\pi_n$ are
isomorphisms) or \demph{strict maps} (by insisting that the $\pi_n$'s are
identities).  The resulting functors $\Cat\hyph\Operad \go \CAT^\op$ are,
of course, called $\Alg_\mr{lax}$ and $\Alg_\mr{str}$,%
% 
\glo{Algstr}
% 
and so we have three functors
\[
\blank\hyph\MClax, \,
\blank\hyph\MCwk, \,
\blank\hyph\MCstr:
\Set^\nat \go \CAT^\op.
% 
\glo{blankMonCatlax}
% 
\]

\begin{defn}	\lbl{defn:Sigma-mon-cat}
Let $\Sigma\in\Set^\nat$.  A \demph{$\Sigma$-monoidal category}%
%
\index{Sigma-monoidal category@$\Sigma$-monoidal category}%
%
\index{monoidal category!Sigma-@$\Sigma$-}
%
is an
object of $\Sigma\hyph\MClax$ (or equivalently, of $\Sigma\hyph\MCwk$ or
$\Sigma\hyph\MCstr$).  A \demph{lax} (respectively, \demph{weak} or
\demph{strict}) \demph{monoidal functor}%
%
\index{monoidal functor!Sigma-@$\Sigma$-}
%
between $\Sigma$-monoidal
categories is a map in $\Sigma\hyph\MClax$ (respectively,
$\Sigma\hyph\MCwk$ or $\Sigma\hyph\MCstr$).
\end{defn}

Now we can state the results.  First, the notion of $\Sigma$-monoidal
category really does generalize the notions of unbiased and classical
monoidal category, as intended all along:

\begin{thm}%
%
\index{coherence!monoidal categories@for monoidal categories!unbiased} 
%
\lbl{thm:diag-coh-umc}
\thmname{Coherence for unbiased monoidal categories and functors}
Writing $1$ for the terminal object of $\Set^\nat$, there are
isomorphisms of categories
%
\begin{eqnarray*}
\UMClax		&\iso 	&1\hyph\MClax,	\\
\UMCwk		&\iso  	&1\hyph\MCwk , 	\\
\UMCstr		&\iso	&1\hyph\MCstr.   
\end{eqnarray*}
%
\end{thm}

\begin{proof}  
See Appendix~\ref{sec:app-UMC}.  It takes almost no calculation to see that
there is a canonical functor $1\hyph\MClax \go \UMClax$.  To see that it
is an isomorphism requires calculations using the coherence axioms for an
unbiased monoidal category.  Restricting to the weak and strict cases is
simple.  \done
\end{proof}

\begin{thm}
\lbl{thm:diag-coh-mc}
\thmname{Coherence%
%
\index{coherence!monoidal categories@for monoidal categories!classical} 
%
for classical monoidal categories and functors}
Write $\Sigma_\mr{c}$%
% 
\glo{Sigmac}
% 
for the object of $\Set^\nat$ given by
$\Sigma_\mr{c}(n)=1$ for $n\in\{0,2\}$ and $\Sigma_\mr{c}(n)=\emptyset$
otherwise.  Then there are isomorphisms of categories
%
\begin{eqnarray*}
\MClax		&\iso 	&\Sigma_\mr{c}\hyph\MClax,	\\
\MCwk		&\iso  	&\Sigma_\mr{c}\hyph\MCwk , 	\\
\MCstr		&\iso	&\Sigma_\mr{c}\hyph\MCstr.   
\end{eqnarray*}
%
\end{thm}

\begin{proof}  
See Appendix~\ref{sec:app-MC}.  The strategy is just the same as
in~\ref{thm:diag-coh-umc}, except that this time the calculations are in
principle much more tricky (because of the irregularity of the data and
axioms for a classical monoidal category) but in practice can be omitted
(by relying on the coherence theorems of others).  \done
\end{proof}

It took a little work to define `$\Sigma$-monoidal category', but I hope it
will be agreed that it is a completely natural definition, free from
\latin{ad hoc} coherence axioms and correctly embodying the idea of a
monoidal category with as many primitive tensor operations as are specified
by $\Sigma$.  So, for instance, the statement that the objects of $\MCwk$
correspond one-to-one with the objects of $\Sigma_\mr{c}\hyph\MCwk$ says
that the coherence data and axioms in the classical definition of monoidal
category are exactly right.  Were there no such isomorphism, it would be
the coherence data or axioms at fault, not the definition of
$\Sigma$-monoidal category.  This is why~\ref{thm:diag-coh-umc}
and~\ref{thm:diag-coh-mc} are called coherence theorems.

The unbiased coherence theorem~(\ref{thm:diag-coh-umc}) also tells us that
unbiased monoidal categories play a universal role: for any $\Sigma \in
\Set^\nat$, the unique map $\Sigma \go 1$ induces a canonical map from
unbiased monoidal categories to $\Sigma$-monoidal categories,
\[
\UMCwk \iso 1\hyph\MCwk \go \Sigma\hyph\MCwk.  
\]
Concretely, if we are given an unbiased monoidal category $A$ then we can
define a $\Sigma$-monoidal category by taking $\otimes_\sigma$ to be the
$n$-fold tensor $\otimes_n$, for each $\sigma\in \Sigma(n)$.

\begin{thm}	\lbl{thm:irrel-sig}%
%
\index{irrelevance of signature}\index{signature!irrelevance of}
%
\thmname{Irrelevance of signature for monoidal categories}
For any plausible $\Sigma, \Sigma' \in \Set^\nat$, there are equivalences
of categories
\[
\Sigma\hyph\MClax \eqv \Sigma'\hyph\MClax,
\diagspace
\Sigma\hyph\MCwk \eqv \Sigma'\hyph\MCwk.
\]
\end{thm}
% 
Here $\Sigma \in \Set^\nat$ is called \demph{plausible}%
%
\index{plausible}
%
if $\Sigma(0) \neq
\emptyset$ and for some $n\geq 2$, $\Sigma(n) \neq \emptyset$.  This means
that $n$-fold tensor products can be derived for all $n\in\nat$.  In
contrast, if $\Sigma(0) = \emptyset$ then $(F\Sigma)(0) = \emptyset$, so
$(IF\Sigma)(0) = \emptyset$, so in a $\Sigma$-monoidal category all the
tensor operations are of arity $n \geq 1$---there is no unit object.
Dually, if $\Sigma(n) = \emptyset$ for all $n\geq 2$ then there is no
derived binary tensor.  In these cases we would not expect
$\Sigma$-monoidal categories to be much like ordinary monoidal categories.
So plausibility is an obvious minimal requirement.

\begin{proof}
It is enough to prove the result in the case $\Sigma'=1$.  As can be seen
from the explicit description of free operads on
p.~\pageref{p:free-plain-clauses}, plausibility of $\Sigma$ says exactly
that $(F\Sigma)(n)\neq \emptyset$ for each $n\in\nat$, or equivalently that
there exists a map $1\go UF\Sigma$ in $\Set^\nat$, where $U:
\Set\hyph\Operad \go \Set^\nat$ is the forgetful functor.  By adjointness,
this says that there is a map $F1 \go F\Sigma$ of \Set-operads, giving a
map $I_*F1 \go I_*F\Sigma$ of \Cat-operads.  On the other hand, $1$ is the
terminal object of $\Set^\nat$, so we have maps $I_*F1 \oppairu
I_*F\Sigma$.

In brief, the rest of the proof runs as follows.  $\Cat\hyph\Operad$
naturally has the structure of a 2-category, because \Cat\ does; and if two
objects of $\Cat\hyph\Operad$ are equivalent then so are their images under
both $\Alg_\mr{lax}$ and $\Alg_\mr{wk}$.  By the nature of indiscrete
categories, the existence of maps $I_*F1 \oppairu I_*F\Sigma$ implies that
$I_*F1 \eqv I_*F\Sigma$.  The result follows.  However, the 2-categorical
details are rather tiresome to check and the reader may prefer to avoid
them.  The main reason for including them below is that they reveal why the
theorem holds at the lax and weak levels but not at the strict level.

So, for the conscientious, a \demph{transformation}
\[
R \ctwomult{H}{H'}{\alpha} S
\]
of $\Cat$-operads is a sequence
\[
\left(
R(n) \ctwomult{H_n}{H'_n}{\alpha_n} S(n)
\right)_{n\in\nat}
\]
of natural transformations, such that
\[
\begin{array}{rl}
&
\scriptstyle
R(n) \times R(k_1) \times\cdots\times R(k_n)
\cone{\comp}
R(k_1 + \cdots + k_n)
\ctwomult{H_{k_1 + \cdots + k_n}}%
{H'_{k_1 + \cdots + k_n}}%
{}
S(k_1 + \cdots + k_n)				\\
=	&
\scriptstyle
R(n) \times R(k_1) \times\cdots\times R(k_n)
\ctwomult{H_n \times H_{k_1} \times\cdots H_{k_n}}%
{H'_n \times H'_{k_1} \times\cdots H'_{k_n}}%
{}
S(n) \times S(k_1) \times\cdots\times S(k_n)
\cone{\comp}
S(k_1 + \cdots + k_n),
\end{array}
\]
where the unlabelled 2-cells are respectively $\alpha_{k_1 + \cdots + k_n}$
and $\alpha_n \times \alpha_{k_1} \times\cdots\times \alpha_{k_n}$, and
\[
1 \cone{\ids} R(1) \ctwomult{H_1}{H'_1}{\alpha_1} S(1)
\ =\ 
1 \ctwomult{\ids}{\ids}{1} S(1).
\]
With the evident compositions, $\Cat\hyph\Operad$%
% 
\glo{CatOperad2cat}
% 
becomes a strict
2-category.  The statement on indiscrete categories is easily proved; in
fact, if $R$ is any $\Cat$-operad and $Q$ any $\Set$-operad then there is a
unique transformation between any pair of maps $R \parpair{}{} I_*Q$.

The functor $\Alg_\mr{lax}$ becomes a strict map $\Cat\hyph\Operad \go
\CAT^{\mr{co}\,\op}$ of 2-categories, where the codomain is $\CAT$ with
both the 1-cells and the 2-cells reversed~(\ref{sec:bicats}).  To see this,
let $\alpha$ be a transformation of \Cat-operads as above.  We need to
produce a natural transformation
\[
\Alg_\mr{lax}(R) 
\ctwomultcoop{H^*}{H'^*}{}
\Alg_\mr{lax}(S),
\]
which in turn means producing for each $S$-algebra $A$ a lax map $H'^*(A)
\go H^*(A)$ of $R$-algebras.  The composite natural transformation
\[
R(n) \times A^n
\ctwomult{H_n}{H'_n}{\alpha_n} 
S(n) \times A^n
\cone{\mi{act}_n^A}
A
\]
can be re-drawn as a natural transformation
%
\begin{eqnarray}
\label{diag:induced-lax-map}
\begin{diagram}
R(n) \times A^n 	&\rTo^{\mi{act}_n^{H'^*(A)}}	&A	\\
\dTo<1			&\nent 				&\dTo>1	\\
R(n) \times A^n 	&\rTo_{\mi{act}_n^{H^*(A)}}	&A,	\\
\end{diagram}
\end{eqnarray}
%
and this gives the desired lax map.  (To make the necessary distinctions,
superscripts have been added to the $\act{n}$'s naming the algebra
concerned.)  We thus obtain a strict map
\[
\Alg_\mr{lax}: \Cat\hyph\Operad \go \CAT^{\mr{co}\,\op}
\]
of 2-categories.

It follows immediately that if $R$ and $S$ are equivalent \Cat-operads then
$\Alg_\mr{lax}(R)$ and $\Alg_\mr{lax}(S)$ are equivalent categories.  To
see that $\Alg_\mr{wk}$ also preserves equivalence of objects, note that if
$\alpha$ is an invertible transformation of \Cat-operads then the natural
transformation~\bref{diag:induced-lax-map} is also invertible, and so
defines a weak map of algebras.  Put another way, $\Alg_\mr{wk}$ is a map
from the 2-category (\Cat-operads $+$ maps $+$ invertible transformations)
into $\CAT^{\mr{co}\,\op}$.

(However,~\bref{diag:induced-lax-map} only induces a \emph{strict} map
$H'^*(A) \go H^*(A)$ if it is the identity transformation, which means that
if $R$ and $S$ are equivalent \Cat-operads then $\Alg_\mr{str}(R)$ and
$\Alg_\mr{str}(S)$ are not necessarily equivalent categories.)  \done
\end{proof}

\begin{cor}%
%
\index{monoidal category!unbiased vs. classical@unbiased \vs.\ classical}
%
There are equivalences of categories
\[
\UMClax \eqv \MClax, 
\diagspace
\UMCwk \eqv \MCwk.
\]
\end{cor}

\begin{proof}
\ref{thm:diag-coh-umc} $+$ \ref{thm:diag-coh-mc} $+$ \ref{thm:irrel-sig}.
\done
\end{proof}

We arrived at the equivalence of unbiased and classical monoidal categories
by a roundabout route,
\[
\UMClax \iso 1\hyph\MClax \eqv \Sigma_\mr{c}\hyph\MClax \iso \MClax,
\]
so it may be useful to consider a direct proof.  Given an unbiased monoidal
category $(A, \otimes, \gamma, \iota)$, we can canonically write down a
classical monoidal category: the underlying category is $A$, the tensor is
$\otimes_2$, the unit is $\otimes_0$, and the associativity and unit
coherence isomorphisms are formed from certain components of $\gamma$ and
$\iota$.  The converse process is non-canonical:
%
\lbl{p:c-to-u}
%
we have to choose for each $n\in\nat$ an $n$-fold tensor operation built up
from binary tensor operations and the unit object.  Put another way, we
have to choose for each $n\in\nat$ an $n$-leafed classical tree (where
`classical' means that each vertex has either $2$ or $0$ outgoing edges, as
in~\ref{eg:opd-of-cl-trees}); and since $F\Sigma_\mr{c}$ is exactly the
operad of classical trees, this corresponds to the step of the proof
of~\ref{thm:irrel-sig} where we chose a map $1 \go UF\Sigma$ in
$\Set^\nat$.

All of the results above can be repeated with monoidal transformations
brought into the picture.  Then $\Sigma$-monoidal categories form a strict
2-category, and all the isomorphisms and equivalences of categories become
isomorphisms and equivalences of 2-categories.  That we did not \emph{need}
to mention monoidal transformations in order to prove the equivalence of
the various notions of monoidal category says something about the strength
of our equivalence result.  For suppose we start with a $\Sigma$-monoidal
category, derive from it a $\Sigma'$-monoidal category, and derive from
that a second $\Sigma$-monoidal category.  Then, in our construction, the
two $\Sigma$-monoidal categories are not just equivalent in the 2-category
$\Sigma\hyph\MCwk$, but isomorphic.  So Theorem~\ref{thm:irrel-sig} is `one
level%
%
\lbl{p:one-level-better}
%
better' than might be expected.











\section{Non-algebraic notions of monoidal category}
\lbl{sec:non-alg-notions}


Coherence%
%
\index{coherence!axioms}
%
axioms have got a bad name for themselves: unmemorable,
unenlightening, and unwieldy, they are often regarded as bureaucracy to be
fought through grimly before getting on to the real business.  Some people
would, therefore, like to create a world where there are no coherence
axioms at all---or anyway, as few as possible.

Whatever the merits of this aspiration, it is a fact that it can be
achieved in some measure; that is, there exist approaches to various higher
categorical structures that involve almost no coherence axioms.  In this
chapter we look at two different such approaches for monoidal categories.
The first exploits the relation between monoidal categories and
multicategories.  I will describe it in some detail.  The second is based
on the idea of the nerve of a category, and has its historical roots in the
homotopy-algebraic structures known as $\Gamma$-spaces.  Since it has less
to do with the main themes of this book, I will explain it more sketchily.

\index{multicategory!underlying|(}%
%
So, we start by looking at monoidal categories \vs.\
multicategories.  One might argue that multicategories are conceptually
more primitive than monoidal categories: that an operation taking several
inputs and producing one output is a more basic idea than a set whose
elements are ordered tuples.  In any case, the first example of a `tensor%
%
\index{tensor!ring@over ring}
%
product' that many of us learned was implicitly introduced via
multicategories---the tensor product $V\otimes W$ of two vector spaces
being characterized as the codomain of a `universal bilinear map' out of
$V, W$.  Now, the monoidal category of vector spaces contains no more or
less information than the multicategory of vector spaces; given either one,
the other can be derived in its entirety.  So, with the thoughts in our
head that multicategories are basic and coherence axioms are bad, we might
hope to `define' a monoidal category as a multicategory in which there are
enough universal maps around.  And this is what we do.

Formally, we have a functor $V$%
% 
\glo{Vmonmti}
% 
assigning to each monoidal category its
underlying multicategory; we want to show that $V$ gives an equivalence
between (monoidal categories) and some subcategory \cat{R} of $\Multicat$;
and we want, moreover, to describe \cat{R}.  In the terms of the previous
paragraph, \cat{R} consists of those multicategories containing `enough
universal maps'.

Before starting we have to make precise something that has so far been left
vague. In Example~\ref{eg:multi-mon}, we defined the underlying
multicategory $C$ of a monoidal category $A$ to have the same objects as
$A$ and maps given by
\[
C(a_1, \ldots, a_n ; a) = A(a_1 \otimes\cdots\otimes a_n, a),
\]
but we deferred the question of what exactly the expression $a_1
\otimes\cdots\otimes a_n$ meant.  We answer it now, and so obtain a precise
definition of the functor $V$.

If $A$ is a strict monoidal category then the expression makes perfect
sense.  If $A$ is not strict then it still makes perfect sense as long as
we have chosen to use unbiased rather%
%
\index{monoidal category!unbiased vs. classical@unbiased \vs.\ classical}
%
than classical monoidal categories:
take $a_1 \otimes\cdots\otimes a_n$ to mean $\otimes_n(a_1, \ldots, a_n)$.
Bringing into play the coherence maps of $A$, we can also define
composition in $C$, and so obtain the entire multicategory structure of $C$
without trouble; we arrive at a functor
\[
V: \UMCwk \go \Multicat.
\]


What if we insist on starting from a classical monoidal category?  We can
certainly obtain a multicategory by passing first from classical to
unbiased and then applying the functor $V$ just mentioned.  This passage
amounts to a choice of an $n$-leafed classical%
%
\index{tree!classical}
%
tree $\tau_n\in\ctr(n)$ for
each $n\in\nat$ (see p.~\pageref{p:c-to-u}), and the resulting
multicategory $C$ has the same objects as $A$ and maps given by
\[
C(a_1, \ldots, a_n; a) = A(\otimes_{\tau_n}(a_1, \ldots, a_n), a)
\]
where $\otimes_{\tau_n}: A^n \go A$ is `tensor according to the shape of
$\tau_n$' (defined formally in~\ref{sec:app-MC}).  So, any such sequence
$\tau_\bullet = (\tau_n)_{n\in\nat}$ induces a functor
\[
V_{\tau_\bullet}:
\MCwk \go \Multicat.
\]
For instance, if we take one of the two most obvious choices of sequence
$\tau_\bullet$ and write $C = V_{\tau_\bullet}(A)$ as usual, then
\[
C(a_1, \ldots, a_n; a) = A(
a_1 \otimes (a_2 \otimes (a_3 \otimes \cdots \otimes (a_{n-1} \otimes a_n)
\cdots )), a ).
\]
But there is also a way of passing from classical monoidal categories to
multicategories without making any arbitrary choices.  Let $A$ be a
classical monoidal category.  For $n\in\nat$ and $\tau, \tau' \in \ctr(n)$,
let 
\[
A^n \ctwomult{\otimes_\tau}{\otimes_{\tau'}}{\delta_{\tau,\tau'}} A
\]
be the canonical isomorphism whose existence is asserted by the coherence
theorem~(\ref{sec:mon-cats}).  Now define a multicategory $C$ by taking an
object to be, as usual, just an object of $A$, and a map $a_1, \ldots, a_n
\go a$ to be a family $(f_\tau)_{\tau\in\ctr(n)}$ in which $f_\tau \in
A(\otimes_\tau (a_1, \ldots, a_n), a)$ for each $\tau\in\ctr(n)$ and
$f_{\tau'} \of \delta_{\tau,\tau'} = f_\tau$ for all $\tau, \tau' \in
\ctr(n)$.  This yields another functor
\[
V': \MCwk \go \Multicat.
\]
However, a family $(f_\tau)_{\tau\in\ctr(n)}$ as above is entirely
determined by any single $f_\tau$, so the multicategory $C = V'(A)$ just
constructed is isomorphic to the multicategory $V_{\tau_\bullet}(A)$
obtained by choosing a particular sequence $\tau_\bullet$ of trees.  So $V'
\iso V_{\tau_\bullet}$ for any $\tau_\bullet$, and henceforth we write $V'$
or $V_{\tau_\bullet}$ as just $V$.

We now have ways of obtaining a multicategory from either an unbiased or a
classical monoidal category, and it is a triviality to check that these are
compatible with the equivalence between unbiased and classical: the diagram
%
\begin{diagram}[size=2em]
\UMCwk 	&	&\rTo^{\eqv}	&	&\MCwk	\\
	&\rdTo<V&		&\ldTo>V&	\\
	&	&\Multicat	&	&	\\
\end{diagram}
%
commutes up to canonical isomorphism.  So we know unambiguously
what it means for a multicategory to be `the underlying multicategory of
some monoidal category', and similarly for maps.  

The results for which we hoped, exhibiting monoidal categories as special
multicategories, can be phrased in various different ways.
%
\begin{defn}	\lbl{defn:repn-multi}
A \demph{representation}%
%
\index{multicategory!representation of}
%
of a multicategory $C$ consists of an object
$\otimes(c_1, \ldots, c_n)$ and a map 
\[
u(c_1, \ldots, c_n):
c_1, \ldots, c_n 
\go
% \goby{u(c_1, \ldots, c_n)} 
\otimes(c_1, \ldots, c_n)
\]
for each $n\in\nat$ and $c_1, \ldots, c_n \in C$, with the following
factorization property (Fig.~\ref{fig:repn}): for any objects $c_1^1,
\ldots, c_1^{k_1}, \ldots, c_n^1, \ldots, c_n^{k_n}, c$ and any map $f:
c_1^1, \ldots, c_n^{k_n} \go c$, there is a unique map
\[
\ovln{f}: 
\otimes(c_1^1, \ldots, c_1^{k_1}), \ldots, \otimes(c_n^1, \ldots,
c_n^{k_n})
\go c
\]
such that 
\[
\ovln{f} \of 
(u(c_1^1, \ldots, c_1^{k_1}), \ldots, u(c_n^1, \ldots, c_n^{k_n}))
= f.
\]
A multicategory is \demph{representable}%
%
\index{multicategory!representable}
%
if it admits a
representation.
\end{defn}
%
\begin{figure}
\[
% DOMAIN
%
\begin{centredpic}
\begin{picture}(16,12)(0,-6)
% transistors
\cell{10}{0}{l}{\tusualdotty{\exists !\ovln{f}}}
\cell{2}{4}{l}{\tusual{}}
\cell{2}{-4}{l}{\tusual{}}
% labels on left-hand transistors
\cell{3.8}{4}{bl}{\tnelabel{u(c_1^1, \ldots, c_1^{k_1})}}
\cell{3.8}{-4}{tl}{\tselabel{u(c_n^1, \ldots, c_n^{k_n})}}
% short wires
\cell{14}{0}{l}{\toutputrgt{c}}
\cell{2}{4}{r}{\tinputslft{c_1^1}{c_1^{k_1}}}
\cell{2}{-4}{r}{\tinputslft{c_n^1}{c_n^{k_n}}}
% long wires
\qbezier(10,1.5)(8.5,1.5)(8,2.75)
\qbezier(6,4)(7.5,4)(8,2.75)
\cell{8}{3.2}{l}{\otimes(c_1^1, \ldots, c_1^{k_1})}
\qbezier(10,-1.5)(8.5,-1.5)(8,-2.75)
\qbezier(6,-4)(7.5,-4)(8,-2.75)
\cell{8}{-3.2}{l}{\otimes(c_n^1, \ldots, c_n^{k_n})}
% ellipses
\cell{9.2}{0.3}{c}{\vdots}
\cell{3}{0}{c}{\cdot}
\cell{3}{1}{c}{\cdot}
\cell{3}{-1}{c}{\cdot}
\end{picture}
\end{centredpic}
%
\mbox{\hspace{2em}}
=
\mbox{\hspace{2em}}
%
% CODOMAIN
%
\begin{centredpic}
\begin{picture}(8,12)(0,-6)
% transistor
\put(2,-6){\line(0,1){12}}
\put(6,0){\line(-2,-3){4}}
\put(6,0){\line(-2,3){4}}
\cell{3.5}{0}{c}{f}
% tags
\cell{2}{4}{r}{\tinputslft{c_1^1}{c_1^{k_1}}}
\cell{2}{-4}{r}{\tinputslft{c_n^1}{c_n^{k_n}}}
\cell{1.2}{0.3}{c}{\vdots}
\cell{6}{0}{l}{\toutputrgt{c}}
\end{picture}
\end{centredpic}
\]
% \hand{50}{1}
\caption{Representation of a multicategory}
\label{fig:repn}
\end{figure}

\begin{defn}	\lbl{defn:pre-univ-simple}
A map $c_1, \ldots, c_n \goby{u} c'$ in a multicategory is
\demph{pre-universal}%
%
\index{pre-universal!map in multicategory}
%
if (Fig.~\ref{fig:univ-simple}(a)) for any object
$c$ and map $c_1, \ldots, c_n \goby{f} c$, there is a unique map $c'
\goby{\ovln{f}} c$ such that $\ovln{f} \of u = f$.
\end{defn}
%
\begin{figure}
\centering
\setlength{\unitlength}{1em}
\begin{picture}(26.5,22)(0,0)
\cell{0}{20}{l}{%
\begin{picture}(14,4)(0,-2)
\cell{2}{0}{r}{\tinputslft{c_1}{c_n}}
\cell{2}{0}{l}{\tusual{u}}
\put(6,0){\line(1,0){2}}
\cell{7}{0.2}{b}{c'}
\cell{8}{0}{l}{\tusualdotty{\exists !\ovln{f}}}
\cell{12}{0}{l}{\toutputrgt{c}}
\end{picture}}
%
\cell{16}{20}{c}{=}
%
\cell{18}{20}{l}{%
\begin{picture}(8.5,4)(-0.5,-2)
\cell{2}{0}{r}{\tinputslft{c_1}{c_n}}
\cell{2}{0}{l}{\tusual{f}}
\cell{6}{0}{l}{\toutputrgt{c}}
\end{picture}}
% 
% 
\cell{13.25}{16}{b}{\textrm{(a)}}
%
%
\cell{0}{8}{l}{%
\begin{picture}(14,12)(0,-6)
% leftmost tags
\cell{2}{4}{r}{\tinputslft{a_1}{a_p}}
\cell{2}{0}{r}{\tinputslft{c_1}{c_n}}
\cell{2}{-4}{r}{\tinputslft{b_1}{b_q}}
% lefthand transistor
\cell{2}{0}{l}{\tusual{u}}
% joining wires
\qbezier(2,5.5)(3.875,5.5)(4.5,4.75)
\qbezier(7,4)(5.125,4)(4.5,4.75)
\qbezier(2,2.5)(3.875,2.5)(4.5,1.75)
\qbezier(7,1)(5.125,1)(4.5,1.75)
\put(6,0){\line(1,0){2}}
\cell{6.8}{0.1}{b}{c'}
\qbezier(2,-5.5)(3.875,-5.5)(4.5,-4.75)
\qbezier(7,-4)(5.125,-4)(4.5,-4.75)
\qbezier(2,-2.5)(3.875,-2.5)(4.5,-1.75)
\qbezier(7,-1)(5.125,-1)(4.5,-1.75)
% inputs to righthand transistor
\cell{8}{2.5}{r}{\tinputslft{}{}}
\cell{8}{-2.5}{r}{\tinputslft{}{}}
% righthand transistor
\qbezier[45](8,4.5)(8,0)(8,-4.5)
\qbezier[32](8,4.5)(10,2.25)(12,0)
\qbezier[32](8,-4.5)(10,-2.25)(12,0)
\cell{9.7}{0}{c}{\exists !\ovln{f}}
% output tag
\cell{12}{0}{l}{\toutputrgt{c}}
\end{picture}}
%
\cell{16}{8}{c}{=}
%
\cell{18}{8}{l}{%
\begin{picture}(8.5,12)(-0.5,-6)
% leftmost tags
\cell{2}{4}{r}{\tinputslft{a_1}{a_p}}
\cell{2}{0}{r}{\tinputslft{c_1}{c_n}}
\cell{2}{-4}{r}{\tinputslft{b_1}{b_q}}
% transistor
\put(2,6){\line(0,-1){12}}
\put(2,6){\line(2,-3){4}}
\put(2,-6){\line(2,3){4}}
% 
\cell{3.7}{0}{c}{f}
% output tag
\cell{6}{0}{l}{\toutputrgt{c}}
\end{picture}}
%
\cell{13.25}{0}{b}{\textrm{(b)}}
\end{picture}
% \hand{90}{2}
\caption{(a) Pre-universal map, and (b) universal map}
\label{fig:univ-simple}
\end{figure}
%
\begin{defn}	\lbl{defn:univ-simple}
A map $c_1, \ldots, c_n \goby{u} c'$ in a multicategory is 
\demph{universal}%
%
\index{universal!map in multicategory}
%
if (Fig.~\ref{fig:univ-simple}(b)) for any objects
$a_1, \ldots, a_p, b_1, \ldots, b_q, c$ and any map 
\[
a_1, \ldots, a_p, c_1, \ldots, c_n, b_1, \ldots, b_q \goby{f} c,
\]
there is a unique map $a_1, \ldots, a_p, c', b_1, \ldots, b_q
\goby{\ovln{f}} c$ such that $\ovln{f} \of_{p+1} u = f$.
(See~\ref{sec:om-further} for the $\of_{p+1}$ notation.)
\end{defn}

Here is the main result.
%
\begin{thm}	\lbl{thm:rep-multi}
\begin{enumerate}
\item	\lbl{part:rep-multi-objs}
The following conditions on a multicategory $C$ are equivalent:
%
\begin{itemize}
\item $C \iso V(A)$ for some monoidal category $A$
\item $C$ is representable
\item every sequence $c_1, \ldots, c_n$ of objects of $C$ is the domain of
some pre-universal map, and the composite of pre-universal maps is
pre-universal 
\item every sequence $c_1, \ldots, c_n$ of objects of $C$ is the domain of
some universal map.
\end{itemize}
Under these equivalent conditions, a map in $C$ is universal if and only if
it is pre-universal.
%
\item	\lbl{part:rep-multi-maps}
Let $A$ and $A'$ be monoidal categories.  The following conditions on
a map $H: V(A) \go V(A')$ of multicategories are equivalent:
%
\begin{itemize}
\item $H = V(P, \pi)$ for some weak monoidal functor $(P, \pi): A \go A'$
\item $H$ preserves%
%
\index{universal!preservation}
%
universal maps (that is, if $u$ is a universal
map in $V(A)$ then $Hu$ is a universal map in $V(A')$).
\end{itemize}
\end{enumerate}
\end{thm}
%
The functor $V$ is faithful, and therefore provides an equivalence between
$\UMCwk$ or $\MCwk$ (as you prefer) and the subcategory $\fcat{RepMulti}$
of $\Multicat$ consisting of the representable multicategories and the
universal-preserving maps.

The only subtle point here is that the existence of a pre-universal map for
every given domain is \emph{not} enough to ensure that the multicategory
comes from a monoidal category.  A specific example appears in
Leinster~\cite{FM}, but the point can be explained here in the familiar
context of vector spaces.  Suppose we are aware that for each pair $(X, Y)$
of vector spaces, there is an object $X\otimes Y$ and a bilinear map $X, Y
\goby{u_{X,Y}} X\otimes Y$ with the traditional universal property (which
we are calling pre-universality).  Then it does \emph{not} follow for
purely formal reasons that the tensor product is associative (up to
isomorphism): one has to use some actual properties of vector spaces.
Essentially, one either has to show that trilinear maps of the form
$u_{X\otimes Y, Z} \of (u_{X,Y}, 1_Z)$ and $u_{X, Y\otimes Z} \of (1_X,
u_{Y,Z})$ are pre-universal, or show that maps of the form $u_{X,Y}$ are in
fact universal.

The energetic reader with plenty of time on her hands will have no
difficulty in proving Theorem~\ref{thm:rep-multi}; the main ideas have been
explained and it is just a matter of settling the details.  So in a sense
that is an end to the matter: monoidal categories can be recognized as
multicategories with a certain property, and monoidal functors similarly,
all as hoped for originally.

We can, however, take things further.  With just a little more work than a
direct proof would involve, Theorem~\ref{thm:rep-multi} can be seen as a
special case of a result in the theory of fibrations%
%
\index{fibration!multicategories@of multicategories}
%
of multicategories.
This theory is a fairly predictable extension of the theory of fibrations
of ordinary categories, and the result of which~\ref{thm:rep-multi} is a
special case is the multicategorical analogue of a standard result on
categorical fibrations.

The basic theory of fibrations of multicategories is laid out in
Leinster~\cite{FM}, which culminates in the deduction of~\ref{thm:rep-multi}
and some related facts on, for instance, \emph{strict} monoidal categories
as multicategories.  Here is the short story.  For any category $D$, a
fibration (or really, opfibration)
% $C \go D$ 
over $D$ is essentially the same thing as a weak functor $D \go \Cat$.  (We
looked at the case of \emph{discrete} fibrations in~\ref{sec:cats}.)  With
appropriate definitions, a similar statement can be made for
multicategories.  Taking $D$ to be the terminal multicategory $1$, we find
that the unique map $C \go 1$ is a fibration exactly when $C$ is
representable, and that weak functors $1 \go \Cat$ are exactly unbiased
monoidal categories.  (Universal and pre-universal maps in $C$ correspond
to what are usually called cartesian and pre-cartesian maps.)  So a
representable multicategory is essentially the same thing as a monoidal
category.%
%
\index{multicategory!underlying|)}
%


\paragraph*{}


We now consider a different non-algebraic notion of monoidal category:
`homotopy monoidal categories'.  The idea can be explained as follows.

Recall that every small category has a nerve, and that this allows
categories to be described as simplicial sets satisfying certain
conditions.  Explicitly, if $n\in\nat$ then let $\upr{n} = \{ 0, 1, \ldots,
n \}$,%
% 
\glo{upr}
% 
and let $\Delta$%
% 
\glo{Delta}\index{simplex category $\Delta$}
% 
be the category whose objects are $\upr{0},
\upr{1}, \upr{2}, \ldots$ and whose morphisms are all order-preserving
functions; so $\Delta$ is equivalent to the category of nonempty finite
totally ordered sets.  A functor $\Delta^\op \go \Set$ is called a
\demph{simplicial%
%
\index{simplicial set}
%
set}.  (More generally, a functor $\Delta^\op \go \Eee$
is called a \demph{simplicial%
%
\index{simplicial object}
%
object in $\Eee$}.)  Any ordered set $(I,
\leq)$ can be regarded as a category with object-set $I$ and with exactly
one morphism $i \go j$ if $i\leq j$, and none otherwise.  This applies in
particular to the ordered sets \upr{n}, and so we may define the
\demph{nerve}% 
%
\lbl{p:defn-nerve}\index{nerve!category@of category} 
%
$NA$ of a small category $A$ as the simplicial set
\[
\begin{array}{rrcl}
NA: 	&\Delta^\op	&\go 		&\Set			\\
	&\upr{n}	&\goesto	&\Cat(\upr{n}, A).
\end{array}
\]
This gives a functor $N: \Cat \go \ftrcat{\Delta^\op}{\Set}$, which turns
out to be full and faithful.  Hence $\Cat$ is equivalent to the full
subcategory of $\ftrcat{\Delta^\op}{\Set}$ whose objects are those
simplicial sets $X$ isomorphic to $NA$ for some small category $A$.  There
are various intrinsic characterizations of such simplicial sets $X$.  We do
not need to think about the general case for now, only the special case of
one-object categories, that is, monoids.

So: let $k, n_1, \ldots, n_k \in \nat$, and for each $j\in \{ 1, \ldots, k
\}$, define a map $\iota_j$ in $\Delta$ by 
\[
\begin{array}{rrcl}
\iota_j: 	&\upr{n_j}	&\go		&\upr{n_1 + \cdots + n_k}\\
		&p		&\goesto	&n_1 + \cdots + n_{j-1} + p.
\end{array}
\]
Given also a simplicial set $X$, let%
%
\index{Segal, Graeme!map}
%
\[
\xi_{n_1, \ldots, n_k}: 
X\upr{n_1 + \cdots + n_k} 
\go 
X\upr{n_1} \times\cdots\times X\upr{n_k}
\]
be the map whose $j$th component is $X(\iota_j)$.  Write
\[
\xi^{(k)} = \xi_{1, \ldots, 1} : X\upr{k} \go X\upr{1}^k.
\]
%
\begin{propn}	\lbl{propn:simp-isos}
The following conditions on a simplicial set $X$ are equivalent:
%
\begin{enumerate}
\item	\lbl{item:simp-iso-general}
$\xi_{n_1, \ldots, n_k}: 
X\upr{n_1 + \cdots + n_k} 
\go 
X\upr{n_1} \times\cdots\times X\upr{n_k}$
is an isomorphism for all $k, n_1, \ldots, n_k \in \nat$
\item 	\lbl{item:simp-iso-classical}
$\xi_{m,n}: X\upr{m+n} \go X\upr{m} \times X\upr{n}$
is an isomorphism for all $m, n \in \nat$, and the unique map $X\upr{0} \go
1$ is an isomorphism
\item 	\lbl{item:simp-iso-powers}
$\xi^{(k)}: X\upr{k} \go X\upr{1}^k$
is an isomorphism for all $k\in\nat$
\item 	\lbl{item:simp-iso-nerve}
$X \iso NA$ for some monoid $A$.
\end{enumerate}
\end{propn}

\begin{proof} 
Straightforward.  Note that~\bref{item:simp-iso-classical} is
just~\bref{item:simp-iso-general} restricted to $k\in\{0,2\}$, and
similarly that~\bref{item:simp-iso-powers} is
just~\bref{item:simp-iso-general} in the case $n_1 = \cdots = n_k = 1$.
\done
\end{proof}

Monoids are, therefore, the same thing as simplicial sets satisfying any of
the conditions \bref{item:simp-iso-general}--\bref{item:simp-iso-powers}.
The proposition can be generalized: replace $\Set$ by any category $\Eee$
possessing finite products to give a description of monoids in $\Eee$ as
certain simplicial objects in $\Eee$.  In particular, if we take
$\Eee=\Cat$ then we obtain a description of strict monoidal categories as
certain simplicial objects in $\Cat$.  This suggests that a `righteous'
(weak)%
%
\index{weakening}
%
notion of monoidal category could be obtained by changing the
isomorphisms in Proposition~\ref{propn:simp-isos} to equivalences.

\begin{propn}	\lbl{propn:simp-eqs}
The following conditions on a functor $X: \Delta^\op \go \Cat$ are
equivalent: 
%
\begin{enumerate}
\item	\lbl{item:simp-eq-general}
$\xi_{n_1, \ldots, n_k}: 
X\upr{n_1 + \cdots + n_k} 
\go 
X\upr{n_1} \times\cdots\times X\upr{n_k}$
is an equivalence for all $k, n_1, \ldots, n_k \in \nat$
\item 	\lbl{item:simp-eq-classical}
$\xi_{m,n}: X\upr{m+n} \go X\upr{m} \times X\upr{n}$
is an equivalence for all $m, n \in \nat$, and the unique map $X\upr{0} \go
1$ is an equivalence
\item 	\lbl{item:simp-eq-powers}
$\xi^{(k)}: X\upr{k} \go X\upr{1}^k$
is an equivalence for all $k\in\nat$.
\end{enumerate}
\end{propn}

\begin{proof}
As noted above, both the implications
\bref{item:simp-eq-general}$\implies$\bref{item:simp-eq-classical} and 
\bref{item:simp-eq-general}$\implies$\bref{item:simp-eq-powers} are
trivial.  Their converses are straightforward inductions.
\done
\end{proof}

\begin{defn}%
%
\index{monoidal category!homotopy}
%
A \demph{homotopy monoidal category} is a functor $X: \Delta^\op \go \Cat$
satisfying the equivalent conditions
\ref{propn:simp-eqs}\bref{item:simp-eq-general}--\bref{item:simp-eq-powers}.
\fcat{HMonCat}%
% 
\glo{HMonCat}
% 
is the category of homotopy monoidal categories and natural
transformations between them.
\end{defn}

A similar definition can be made with \fcat{Top} replacing \Cat\ and
homotopy equivalences replacing categorical equivalences, to give a notion
of `topological monoid%
%
\index{monoid!topological}
%
up to homotopy' (Leinster~\cite[\S 4]{UTHM}).  It
can be shown that any loop%
%
\index{loop space}
%
space provides an example.  In fact, loop space
theory was where the idea first arose: there, topological monoids up to
homotopy were called `special $\Delta$-spaces'%
%
\index{Delta-space@$\Delta$-space}%
%
\index{special}%
%
\index{simplicial space}
%
or `special simplicial
spaces' (Segal~\cite{SegCCT},%
%
\index{Segal, Graeme}
%
Anderson~\cite{And},
Adams~\cite[p.~63]{Ad}). 

Before I say anything about the comparison with ordinary monoidal
categories, let me explain another route to the notion of homotopy monoidal
category.  

Let \scat{D} be the augmented simplex category~(\ref{eg:str-mon-D}), whose
objects are the (possibly empty) finite totally ordered sets $\lwr{n} =
\{1, \ldots, n\}$.  The fact that $\Delta$ is \scat{D} with the object
\lwr{0} removed is a red herring:%
%
\index{herring, red}%
%
\index{simplex category $\Delta$!augmented simplex category@\vs.\ augmented simplex category}% 
%
\index{augmented simplex category $\scat{D}$!simplex category@\vs.\ simplex category} 
%
we will not use \emph{this} connection
between $\Delta$ and \scat{D}.  

Now, $\scat{D}$ is the free monoidal category containing a monoid,%
%
\index{monoid!monoidal category@in monoidal category}
%
in the
sense that for any monoidal category $(\Eee, \otimes, I)$ (classical, say),
there is an equivalence
\[
\MCwk( (\scat{D},+,0), (\Eee,\otimes,I) )
\eqv
\Mon( \Eee, \otimes, I )
\]
between the category of weak monoidal functors $\scat{D} \go \Eee$ and the
category of monoids in $\Eee$.  For given a weak monoidal functor $\scat{D}
\go \Eee$, the image of any monoid in \scat{D} is a monoid in $\Eee$, and
in particular, the object \lwr{1} of \scat{D} has a unique monoid
structure, giving a monoid in \Eee.  Conversely, given a monoid $A$ in
$\Eee$, there arises a weak monoidal functor $\scat{D} \go \Eee$ sending
\lwr{n} to $A^n$.

Taking $\Eee = \Cat$ describes strict monoidal categories as weak monoidal
functors $(\scat{D},+,0) \go (\Cat,\times,1)$.  Such a weak monoidal
functor is an ordinary functor $W: \scat{D} \go \Cat$ together with
isomorphisms
% 
\begin{equation}	\label{eq:colax-coh-maps}
\omega_{m,n}: W(\lwr{m} + \lwr{n}) \go W\lwr{m} \times W\lwr{n},
\diagspace
\omega_\cdot: W\lwr{0} \go 1
\end{equation}
% 
($m, n \in\nat$) satisfying coherence axioms.  By changing `isomorphisms'
to `equivalences' in the previous sentence, we obtain another notion of
weak monoidal category.  Formally, given monoidal categories $\cat{D}$ and
$\cat{E}$, write $\fcat{MonCat}_\mr{colax}(\cat{D}, \cat{E})$ for the
category of colax monoidal functors $\cat{D} \go \cat{E}$ (as defined
in~\ref{defn:mon-ftr}) and monoidal transformations between them.  The new
`weak monoidal categories' are the objects of the category \fcat{HMonCat'}
defined as follows.
%
\begin{defn}	\lbl{defn:hmoncat-prime}
\fcat{HMonCat'} is the full subcategory of
\[
\fcat{MonCat}_\mr{colax}((\scat{D},+,0), (\Cat,\times,1))
\]
consisting of the colax monoidal functors $(W, \omega)$ for which each of
the functors~\bref{eq:colax-coh-maps} ($m, n \in \nat$) is an equivalence
of categories.
\end{defn}

You might object that this is useless as a definition of monoidal category,
depending as it does on pre-existing concepts of monoidal category and
colax monoidal functor.  There are several responses.  One is that we could
give an explicit description of what a colax monoidal functor $\scat{D} \go
\Cat$ is, along the lines of the traditional description of cosimplicial
objects by face and degeneracy maps, and this would eliminate the
dependence.  Another is that we will soon show that $\fcat{HMonCat'} \iso
\fcat{HMonCat}$, and $\fcat{HMonCat}$ is defined without mention of
monoidal categories.  A third is that while~\ref{defn:hmoncat-prime} might
not be good as a \emph{definition} of monoidal category, it is a useful
reformulation: for instance, if we change the monoidal category
$(\Cat,\times,1)$ to the monoidal category of chain
complexes and change
categorical equivalence to chain homotopy equivalence, then we obtain a
reasonable notion of homotopy%
%
\index{homotopy-algebraic structure}%
%
\index{algebra!differential graded}%
%
\index{chain complex!homotopy DGA}
%
differential%
%  
\lbl{p:hty-dgas}
% 
graded algebra.  This exhibits an advantage of the $\scat{D}$ approach over
the $\Delta$ approach: we can use it to discuss homotopy monoids in
monoidal categories where the tensor is not cartesian product.  Much more
on this can be found in my~\cite{UTHM} and~\cite{HAO}.

To show that $\fcat{HMonCat'} \iso \fcat{HMonCat}$, we first establish a
connection between $\Delta$ and $\scat{D}$.

\begin{propn}	\lbl{propn:D-Delta}
Let $\Eee$ be a category with finite products.  Then there is an
isomorphism of categories
\[
\fcat{MonCat}_\mr{colax}((\scat{D},+,0), (\Eee,\times,1))
\iso
\ftrcat{\Delta^\op}{\Eee}.
\]
\end{propn}

\begin{proof}  
This is a special case of a general result on Kleisli%
%
\index{Kleisli!category}
%
categories
(Leinster~\cite[3.1.6]{HAO}; beware the different notation).  It can also
be proved directly in the following way.  A functor $W: \scat{D} \go \Eee$
is conventionally depicted as a diagram
%
\begin{diagram}[width=2em,tight]
\cdot	&
\rTo	&
\cdot	&
\pile{\rTo\\ \lTo\\ \rTo}	&
\cdot	&
\pile{\rTo\\ \lTo\\ \rTo\\ \lTo\\ \rTo}	&
\cdot	&
\cdots
\end{diagram}
%
of objects and arrows in $\Eee$.  A colax monoidal structure $\omega$ on
$W$ amounts to a pair of maps
\[
W\lwr{m} \og W(\lwr{m} + \lwr{n}) \go W\lwr{n}
\]
for each $m,n\in\nat$, satisfying axioms implying, among other things,
that all of these maps can be built up from the special cases
\[
W\lwr{m} \og W(\lwr{m} + \lwr{1}), 
\diagspace
W(\lwr{1} + \lwr{n}) \go W\lwr{n}.
\]
So a colax monoidal functor $(W, \omega): \scat{D} \go \Eee$ looks like
%
\begin{diagram}[width=2em,tight]
\cdot	&
\pile{\lGet\\ \rTo\\ \lGet}	&
\cdot	&
\pile{\lGet\\ \rTo\\ \lTo\\ \rTo\\ \lGet}	&
\cdot	&
\pile{\lGet\\ \rTo\\ \lTo\\ \rTo\\ \lTo\\ \rTo\\ \lGet}	&
\cdot	&
\cdots,
\end{diagram}
%
and this is the conventional picture of a simplicial object in \Eee.
\done
\end{proof}

So if $\Eee$ is a category with ordinary, cartesian, products, simplicial
objects in $\Eee$ are the same as colax monoidal functors from $\scat{D}$
to $\Eee$.  This fails when $\Eee$ is a monoidal category whose tensor
product is not the cartesian product.  It could be argued that in this
situation, it would be better to define a simplicial object in $\Eee$ not
as a functor $\Delta^\op \go \Eee$, but rather as a colax monoidal functor
$\scat{D} \go \Eee$.  For example, it was the colax monoidal version that
made possible the definition of homotopy differential graded algebra
referred to above.

In Proposition~\ref{propn:D-Delta}, if the colax monoidal functor $(W,
\omega)$ corresponds to the simplicial object $X$ then we have
\[
W\lwr{n} = X\upr{n},
\diagspace
\omega_{m,n} = \xi_{m,n}, 
\diagspace
\omega_\cdot = \xi_\cdot
\]
(where $\xi_\cdot: X\upr{0} \go 1$ means $\xi_{n_1, \ldots, n_k}$ in the case
$k=0$).  So by using condition~\bref{item:simp-eq-classical} of
Proposition~\ref{propn:simp-eqs} we obtain:
%
\begin{cor}
The isomorphism of Proposition~\ref{propn:D-Delta} restricts to an
isomorphism $\fcat{HMonCat'} \iso \fcat{HMonCat}$.
\done
\end{cor}


Our two ways of approaching homotopy monoidal categories, via either nerves
in a finite product category or monoids in a monoidal category, are
therefore equivalent in a strong sense.  The next question is: are they
also equivalent to the standard notion of monoidal category?  I will stop
short of a precise equivalence result, and instead just indicate how to
pass back and forward between homotopy and `ordinary' monoidal categories.

So, let us start from an unbiased%
%
\index{monoidal category!unbiased}
%
monoidal category $(A, \otimes, \gamma,
\iota)$ and define from it a homotopy monoidal category $X \in
\fcat{HMonCat}$.  To define $X$ we will need to use unbiased bicategories
(defined in the next section).  For each $n\in\nat$ the ordered set \upr{n}
may be regarded as a category, and so as an unbiased bicategory in which
all 2-cells are identities.  Also, the unbiased monoidal category $A$ may
be regarded as an unbiased bicategory with only one object.  The homotopy
monoidal category $X$ is defined by taking $X\upr{n}$ to be the category
whose objects are all weak functors $\upr{n} \go A$ of unbiased
bicategories and whose maps are transformations of a suitably-chosen kind.
This is just a categorification%
%
\index{categorification}
%
of the usual nerve%
%
\index{nerve!categorified}
%
construction.

Conversely, let us start with $X \in \fcat{HMonCat}$ and derive from $X$ an
unbiased monoidal category $(A, \otimes, \gamma, \iota)$.  The category $A$
is $X\upr{1}$.  To obtain the rest of the data we first choose for each
$n\in\nat$ a functor $\psi^{(n)}: X\upr{1}^n \go X\upr{n}$ and natural
isomorphisms
\[
\eta^{(n)}: 1 \goiso \xi^{(n)} \of \psi^{(n)},
\diagspace
\epsln^{(n)}: \psi^{(n)} \of \xi^{(n)} \goiso 1
\]
such that $(\psi^{(n)},\xi^{(n)},\eta^{(n)},\epsln^{(n)})$ forms an adjoint
equivalence, which is possible since the functor $\xi^{(n)}$ is an
equivalence~(\ref{propn:eqv-eqv}).  Define $\delta^{(n)}: \upr{1} \go
\upr{n}$ by $\delta^{(n)}(0) = 0$ and $\delta^{(n)}(1) = n$.  Then define
$\otimes_n: A^n \go A$ as the composite
\[
X\upr{1}^n \goby{\psi^{(n)}} X\upr{n} \goby{X\delta^{(n)}} X\upr{1}.
\]
The coherence isomorphisms $\gamma$ and $\iota$ are defined from the
$\eta^{(n)}$'s and $\epsln^{(n)}$'s in a natural way.  The coherence axioms
follow from the fact that we chose \emph{adjoint} equivalences---that is,
they follow from the triangle identities.  So we arrive at an unbiased
monoidal category $(A, \otimes, \gamma, \iota)$.

The processes above determine functors $\UMCwk \oppairu \fcat{HMonCat}$.
It is fairly easy to see that composing one functor with the other does not
yield a functor isomorphic to the identity, either way round.  I believe,
however, that if \fcat{HMonCat} is made into a 2-category in a suitable way
then each composite functor is \emph{equivalent} to the identity.  This
would mean that the notion of homotopy monoidal category is essentially the
same as the other notions of monoidal category that we have discussed.


\paragraph*{}

To summarize the chapter so far: we have formalized the idea of non-strict
monoidal category in various ways, and, with the exception of homotopy
monoidal categories, shown that all the formalizations are equivalent.
Precisely, the following categories are equivalent:%
%
\index{monoidal category!definitions of}
%
%
\begin{quote}
\begin{tabular}{ll}
\MCwk		&(classical monoidal categories)	\\
\UMCwk		&(unbiased monoidal categories)		\\
$\Sigma\hyph\MCwk$&($\Sigma$-monoidal categories, for any plausible
		   $\Sigma$)\\
\fcat{RepMulti}	&(representable multicategories)	
\end{tabular}
\end{quote}
%
and these equivalences can presumably be extended to equivalences of
2-categories.  The categories
%
\begin{quote}
\begin{tabular}{ll}
\fcat{HMonCat}	&(homotopy monoidal categories, via simplicial objects)	\\
\fcat{HMonCat'}	&(homotopy monoidal categories, via monoidal functors)	
\end{tabular}
\end{quote}
%
are isomorphic, and there is reasonable hope that if they are made into
2-categories then they are equivalent to $\MCwk$.  There are still more
notions of monoidal category that we might contemplate---for instance, the
anamonoidal%
%
\index{anamonoidal category}
%
categories of Makkai~\cite{MakAAC}---but%
%
\index{Makkai, Michael}
%
we leave it at that.





\section{Notions of bicategory}
\lbl{sec:notions-bicat}


Everything that we have done for monoidal categories can also be done for
bicategories.  This is usually at the expense of setting up some slightly
more sophisticated language, which is why things so far have been done for
monoidal categories only.  Here we run through what we have done for
monoidal categories and generalize it to bicategories, noting any wrinkles.



\minihead{Unbiased bicategories}
% \paragraph*{}

\begin{defn}	\lbl{defn:lax-bicat}
A \demph{lax%
%
\index{bicategory!lax}
%
bicategory} $B$ (or properly, $(B, \of, \gamma, \iota)$)
consists of
%
\begin{itemize}
\item a class $B_0$ (often assumed to be a set), whose elements are called
\demph{objects} or \demph{0-cells}%
%
\index{cell!unbiased bicategory@of unbiased bicategory}
%
of $B$
\item for each $a,b\in B_0$, a category $B(a,b)$, whose objects are called
\demph{1-cells} and whose morphisms are called \demph{2-cells}
\item for each $n\in \nat$ and $a_0, \ldots, a_n \in B_0$, a functor 
\[
\ofdim{n}: B(a_{n-1}, a_n) \times \cdots \times B(a_0, a_1)
\go
B(a_0, a_n),
\]%
% 
\glo{nfoldcompbicat}%
% 
called \demph{$n$-fold%
%
\index{n-fold@$n$-fold!composition}
%
composition} and written
\[
\begin{array}{rcl}
(f_n, \ldots, f_1) 	&\goesto	&(f_n \of\cdots\of f_1)		\\
(\alpha_n, \ldots, \alpha_1)	
			&\goesto
					&(\alpha_n * \cdots * \alpha_1)
\end{array}
\]%
% 
\glo{nfoldcompbicatinfix}\glo{nfoldstar}%
% 
where the $f_i$'s are 1-cells and the $\alpha_i$'s are 2-cells
\item 
a 2-cell
\[
\begin{array}{rl}
\gamma_{((f_n^{k_n}, \ldots, f_n^1), \ldots, (f_1^{k_1}, \ldots,
f_1^1))}: &
((f_n^{k_n} \of \cdots \of f_n^1) \of \cdots \of
(f_1^{k_1} \of \cdots \of f_1^1))\\
&\go 
(f_n^{k_n} \of\cdots\of f_n^1 \of\cdots\of f_1^{k_1}
\of\cdots\of f_1^1)
\end{array}
\]%
% 
\glo{gammabicat}%
% 
for each $n, k_1, \ldots, k_n \in \nat$ and double sequence $((f_n^{k_n},
\ldots, f_n^1), \ldots, (f_1^{k_1}, \ldots, f_1^1))$ of 1-cells for which
the composites above make sense
\item a 2-cell
\[
\iota_f: f \go (f)
\]%
% 
\glo{iotabicat}%
% 
for each 1-cell $f$,
\end{itemize}
%
satisfying naturality and coherence axioms analogous to those for unbiased
monoidal categories (Definition~\ref{defn:lax-mon-cat}).

A lax bicategory $(B, \of, \gamma, \iota)$ is called an \demph{unbiased%
%
\index{bicategory!unbiased}
%
bicategory} (respectively, an \demph{unbiased strict 2-category}) if all
the components of $\gamma$ and $\iota$ are invertible 2-cells
(respectively, identity 2-cells).
\end{defn}

\begin{remarks}{rmks:u-bicat}
\item A lax bicategory with exactly one object is, of course, just a lax
monoidal category, and similarly for the weak and strict versions.
 
\item Unbiased strict 2-categories are in one-to-one correspondence with
ordinary strict 2-categories, easily.
\end{remarks}

As in the case of monoidal categories, there is an abstract version of the
definition of unbiased bicategory phrased in the language of 2-monads.
Previously we took the 2-monad `free strict monoidal category' on $\Cat$;
now we take the 2-monad `free strict 2-category' on $\Cat\hyph\Gph$, the
strict 2-category of $\Cat$-graphs.%
%
\index{Cat-graph@$\Cat$-graph}
%
 We met the ordinary category
$\Cat\hyph\Gph$ earlier~(\ref{defn:V-gph}).  The 2-category structure on
$\Cat$ induces a 2-category structure on $\Cat\hyph\Gph$ as follows: given
maps $P, Q: B \go B'$ of $\Cat$-graphs, there are only any 2-cells of the
form
\[
B \ctwomult{P}{Q}{} B'
\]
when $P_0 = Q_0: B_0 \go B'_0$, and in that case a 2-cell $\zeta$ is a
family of natural transformations
\[
\left(P_{a,b} \goby{\zeta_{a,b}} Q_{a,b} \right)_{a,b \in B_0}.
\]
Now, there is a forgetful map $\fcat{Str}\hyph 2\hyph\Cat \go
\Cat\hyph\Gph$ of 2-categories, and this has a left adjoint, so there is an
induced 2-monad $(\blank^*,\mu,\eta)$ on $\Cat\hyph\Gph$.  An unbiased
bicategory is exactly a weak algebra for this 2-monad, and the same applies
in the lax and strict cases.

\begin{defn}
Let $B$ and $B'$ be lax bicategories.  A \demph{lax functor}%
%
\index{bicategory!unbiased!functor of}%
%
\index{functor!unbiased bicategories@of unbiased bicategories}
%
%
$(P, \pi): B \go B'$ consists of
\begin{itemize}
\item
a function $P_0: B_0 \go B'_0$ (usually just written as $P$)
\item
for each $a,b\in B_0$, a functor $P_{a,b}: B(a,b) \go B'(Pa,Pb)$
\item
for each $n\in\nat$ and composable sequence $f_1, \ldots, f_n$ of 1-cells
of $B$, a 2-cell
\[
\pi_{f_n, \ldots, f_1}:
(Pf_n \of\cdots\of Pf_1)
\go
P(f_n \of\cdots\of f_1),
\]
\end{itemize}
%
satisfying axioms analogous to those in the definition of lax monoidal
functor~(\ref{defn:u-lax-mon-ftr}).  A \demph{weak functor} (respectively,
\demph{strict functor}) from $B$ to $B'$ is a lax functor $(P, \pi)$ for
which each component of $\pi$ is an invertible 2-cell (respectively,
an identity 2-cell).
\end{defn}

As an example of the benefits of the unbiased approach, consider
`\Hom-functors'.%
%
\index{Hom-functor}
%
 For any category $A$ there is a functor
\[
\Hom: A^\op \times A	\go \Set
\]
defined on objects by $(a,b) \goesto A(a,b)$ and on morphisms by
composition---that is, morphisms $a' \goby{f} a$ and $b \goby{g} b'$ in $A$
induce the function
\[
\begin{array}{rrcl}
\Hom(f,g):	&\Hom(a,b)	&\go	&\Hom(a',b'),	\\
		&p		&\goesto&g\of p\of f.
\end{array}
\]
Suppose we want to imitate this construction for bicategories, changing the
category $A$ to a bicategory $B$ and looking for a weak functor $\Hom:
B^\op \times B \go \Cat$.  If we use classical bicategories then we have a
problem: there is no such composite as $g\of p\of f$, and the best we can
do is to choose some substitute such as $(g\of p)\of f$ or $g\of (p\of f)$.
Although we could, say, consistently choose the first option and so arrive
at a weak functor $\Hom$, this is an arbitrary choice.  So there is no
canonical \Hom-functor in the classical world.  In the unbiased world,
however, we can simply take the ternary composite $(g\of p\of f)$, and
everything runs smoothly.

By choosing different strengths of bicategory and of maps between them, we
again obtain $9$ different categories:%
%
\index{bicategory!nine categories of}
%
\[
\begin{diagram}[height=1.2em]
\fcat{LaxBicat}_\mr{str}	&\sub	&\fcat{LaxBicat}_\mr{wk}	&\sub
&\fcat{LaxBicat}_\mr{lax} \\
\rotsub	&	&\rotsub	&	&\rotsub	\\	
\UBistr	&\sub	&\UBiwk		&\sub	&\UBilax	\\
\rotsub	&	&\rotsub	&	&\rotsub	\\	
\fcat{Str}\hyph 2\hyph\Cat_\mr{str}	&\sub	&\
\fcat{Str}\hyph 2\hyph\Cat_\mr{wk}	&\sub	&\
\fcat{Str}\hyph 2\hyph\Cat_\mr{lax}.			\\
\end{diagram}
\]%
% 
\glo{ninebicat}%
% 
For instance, $\UBiwk$ is the category of unbiased bicategories and weak
functors.  

Differences from the theory of unbiased monoidal categories emerge when we
try to define transformations between functors between unbiased
bicategories.  This should not come as a surprise given what we already
know in the classical case about transformations and modifications of
bicategories \vs.\ transformations of monoidal
categories~(\ref{eg:mon-cat-bicat-transf}).  More mysteriously, there seems
to be no satisfactorily unbiased way to formulate a definition of
transformation%
%
\index{transformation!unbiased bicategories@of unbiased bicategories}%
%
\index{bicategory!unbiased!transformation of}
%
or modification%
%
\index{modification!unbiased}%
%
\index{bicategory!unbiased!modification of}
% 
for unbiased bicategories; it seems that we are forced to grit our teeth
and write down biased-looking definitions.  This done, we obtain a notion
of biequivalence%
%
\index{biequivalence!unbiased}
%
of unbiased bicategories.  Just as in the classical
case~(\ref{propn:bieqv-eqv}), biequivalence amounts to the existence of a
weak functor that is essentially surjective on objects and locally an
equivalence.  The $\st$%
% 
\glo{strictcoverbicat}\index{cover, strict}
% 
construction for monoidal
categories~(\ref{thm:eqv-coh-umc}) generalizes without trouble to give

\begin{thm}	%\lbl{thm:eqv-coh-ubicat}%
%
\index{coherence!bicategories@for bicategories!unbiased}
%
Every unbiased bicategory is biequivalent to a strict 2-category.  
\done
\end{thm}

\begin{example}
Every topological space $X$ has a fundamental%
%
\index{fundamental!2-groupoid}
%
2-groupoid $\Pi_2 X$.%
% 
\glo{fundubicat}
% 
 We
saw how to define $\Pi_2 X$ as a classical bicategory
in~\ref{eg:bicat-Pi}.  Here we consider the unbiased version of $\Pi_2 X$,
in which $n$-fold composition is defined by choosing for each $n\in\nat$ a
reparametrization map $[0,1] \go [0,n]$ (the most obvious choice being
multiplication by $n$).  A 0-cell of the strict%
%
\index{cover, strict}
%
cover $\st(\Pi_2 X)$ is
a point of $X$, and a 1-cell is a pair $(n,\gamma)$ where $n\in\nat$ and
$\gamma: [0,n] \go X$ with $\gamma(0) = x$ and $\gamma(n) = y$.  This is
essentially the technique of Moore%
%
\index{Moore loop}
%
loops (Adams~\cite[p.~31]{Ad}), used to show that every loop space is
homotopy equivalent to a strict topological monoid.
\end{example}


\minihead{Algebraic notions of bicategory}
% \paragraph*{}

Here we take the very general family of algebraic notions of monoidal
category considered in~\ref{sec:alg-notions} and imitate it for
bicategories.  In other words, we set up a theory of $\Sigma$-bicategories,
where the `signature' $\Sigma$ is a sequence of sets.

Recall that we defined $\Sigma$-monoidal categories in three steps, the
functor $\blank\hyph\MCwk$ being the composite
\[
\Set^\nat \goby{F} 
\Set\hyph\Operad \goby{I_*}
\Cat\hyph\Operad \goby{\Alg_\mr{wk}}
\CAT^\op.
\]
The first two steps create an operad consisting of all the derived tensor
products arising from $\Sigma$ and all the coherence isomorphisms between
them.  The third takes algebras for this operad (or, if you prefer, models
for this theory), which in this case means forming the category of
`monoidal categories' with the kind of products described by the operad.
It is therefore only the third step that we need to change here.

So, given a \Cat-operad $R$, define a category $\fcat{CatAlg}_\mr{wk}(R)$%
% 
\glo{CatAlgwk}
% 
as follows.  An object is a \demph{categorical $R$-algebra},%
%
\index{categorical algebra for operad}%
%
\index{Cat-operad@$\Cat$-operad!category over}
%
%
 that is, a
\Cat-graph $B$ together with a functor
\[
\act{n}:
R(n) \times B(a_{n-1}, a_n) \times\cdots\times B(a_0, a_1)
\go
B(a_0, a_n)
\]
for each $n\in\nat$ and $a_0, \ldots, a_n \in B_0$, satisfying axioms very
similar to the usual axioms for an algebra for an operad.  (So if $B_0$ has
only one element then $B$ is just an $R$-algebra in the usual sense.)  A
\demph{weak map} $B \go B'$ of categorical $R$-algebras is a map $P: B \go
B'$ of \Cat-graphs together with a natural isomorphism
%
\begin{diagram}
R(n) \times B(a_{n-1}, a_n) \times\cdots\times B(a_0, a_1)	&
\rTo^{\act{n}}	&B(a_0, a_n)					\\
\dTo<{1 \times P_{a_{n-1}, a_n} \times\cdots\times P_{a_0, a_1}}&
\nent \pi_n	&\dTo>{P_{a_0,a_n}}				\\
R(n) \times B'(a_{n-1}, a_n) \times\cdots\times B'(a_0, a_1)	&
\rTo_{\act{n}}	&B'(Pa_0, Pa_n)					\\
\end{diagram}
%
for each $n\in\nat$ and $a_0, \ldots, a_n \in B_0$, satisfying axioms like
the ones in the monoidal case (Fig.~\ref{fig:wk-alg-coh},
p.~\pageref{fig:wk-alg-coh}).  This defines a category
$\fcat{CatAlg}_\mr{wk}(R)$.  Defining $\fcat{CatAlg}_\mr{wk}$ on maps in
the only sensible way, we obtain a functor
\[
\fcat{CatAlg}_\mr{wk}: \Cat\hyph\Operad \go \CAT^\op,
\]
and we then define the functor $\blank\hyph\Biwk$%
% 
\glo{blankBi}
% 
as the composite
\[
\Set^\nat \goby{F} 
\Set\hyph\Operad \goby{I_*}
\Cat\hyph\Operad \goby{\fcat{CatAlg}_\mr{wk}}
\CAT^\op.
\]
The lax and strict cases are, of course, done similarly.

\begin{defn}%
%
\index{Sigma-bicategory@$\Sigma$-bicategory}%
%
\index{bicategory!Sigma-@$\Sigma$-}
%
Let $\Sigma\in\Set^\nat$.  A \demph{$\Sigma$-bicategory} is an object of
$\Sigma\hyph\Bilax$ (or equivalently, of $\Sigma\hyph\Biwk$ or
$\Sigma\hyph\Bistr$).  A \demph{lax} (respectively, \demph{weak} or
\demph{strict}) \demph{functor}%
%
\index{functor!Sigma-bicategories@of $\Sigma$-bicategories}
%
between $\Sigma$-bicategories is a map in
$\Sigma\hyph\Bilax$ (respectively, $\Sigma\hyph\Biwk$ or
$\Sigma\hyph\Bistr$).
\end{defn}

All of the equivalence results in Section~\ref{sec:alg-notions} go
through.  Hence there are isomorphisms of categories%
%
\index{coherence!bicategories@for bicategories!unbiased}%
%
\index{coherence!bicategories@for bicategories!classical}
%
\[
\fcat{UBicat}_{-} \iso 1\hyph\Bicat_{-},
\diagspace
\Bicat_{-} \iso \Sigma_\mr{c}\hyph\Bicat_{-},
\]
where `$-$' represents any of `$\mr{lax}$', `$\mr{wk}$' or
`$\mr{str}$'.  The proofs are as in the monoidal
case~(\ref{thm:diag-coh-umc},~\ref{thm:diag-coh-mc}) with only cosmetic
changes.  Then, there is the irrelevance%
%
\index{irrelevance of signature}%
%
\index{signature!irrelevance of}
%
%
of signature theorem, analogous
to~\ref{thm:irrel-sig}:
\[
\Sigma\hyph\Bicat_{-} \eqv \Sigma'\hyph\Bicat_{-}
\]
for all plausible $\Sigma$ and $\Sigma'$, where `$-$' is either
`$\mr{lax}$' or `$\mr{wk}$'.  Again, the proof is essentially unchanged.
As a corollary, unbiased bicategories are the same as classical
bicategories:%
%
\index{bicategory!unbiased vs. classical@unbiased \vs.\ classical}
%
\[
\fcat{UBicat}_{-} \eqv \Bicat_{-}
\]
where `$-$' is either `$\mr{lax}$' or `$\mr{wk}$'.  

We saw on p.~\pageref{p:one-level-better} that the equivalence between
unbiased and classical monoidal categories is `one level better' than might
be expected, because it does not refer upwards to transformations.  In the
case of bicategories it is \emph{two} levels better, because modifications
are not needed either.




\minihead{Non-algebraic notions of bicategory}
% \paragraph*{}

I will say much less about these.  

Take representable%
%
\index{multicategory!representable}
%
multicategories first.  To formulate a notion of a
bicategory as a multicategory satisfying a representability condition, we
need to use a new kind of multicategory: instead of the arrows looking like
\[
\begin{centredpic}
\begin{picture}(4,8)(-2,0)
\cell{0}{6}{b}{\tinputsvert{a_1}{a_n}}
\cell{0}{6}{t}{\tusualvert{\alpha}}
\cell{0}{2}{t}{\toutputvert{a}}
\end{picture}
\end{centredpic},
\]
they should look like
\[
\setlength{\unitlength}{1mm}
\begin{picture}(36,15)(0,-2)
% Zero-cell marks
\cell{0}{0}{c}{\zmark}
\cell{6}{8}{c}{\zmark}
\cell{36}{0}{c}{\zmark}
% One-cell arrows
\put(0,0){\vector(3,4){6}}
\put(6,8){\vector(3,1){9}}
\put(30,8){\vector(3,-4){6}}
\put(0,0){\vector(1,0){36}}
% Two-cell arrow
\cell{18}{4.5}{c}{\Downarrow}
% \put(18,7){\vector(0,-1){5}}
% Dot-dot-dot
\cell{22}{9.5}{c}{\cdots}
% Labels
\cell{-2.5}{0}{c}{a_0}
\cell{4}{9}{c}{a_1}
\cell{39}{0}{c}{a_n.}
\cell{1}{5}{c}{f_1}
\cell{10}{11.5}{c}{f_2}
\cell{35.5}{5}{c}{f_n}
\cell{18}{-1.5}{c}{g}
\cell{20.5}{4.5}{c}{\alpha}
\end{picture}
\]
We will consider such multicategories later: they are a special kind of
`$\fc$-multicategory' (Example~\ref{eg:fcm-vdisc}), and they also belong to
the world of opetopic structures (Chapter~\ref{ch:opetopic}).  All the
results on monoidal categories as representable multicategories can be
extended unproblematically to bicategories, and the same goes for the
theory of fibrations%
%
\index{fibration!multicategories@of multicategories}
%
of multicategories (Leinster~\cite{FM}).

Consider, finally, homotopy monoidal categories.  A homotopy%
%
\index{bicategory!homotopy}
%
bicategory can
be defined as a functor $\Delta^\op \go \Cat$%
%
\index{simplicial object}
%
satisfying conditions similar
to (but, of course, looser than) those in Proposition~\ref{propn:simp-eqs}.
The new conditions involve pullbacks rather than products, and can be found
by considering nerves%
%
\index{nerve!category@of category}
%
of categories in general instead of just nerves of
monoids.  See~\ref{sec:non-alg-defns-n-cat} for further remarks.






\begin{notes}

The notion of unbiased monoidal category has been part of the collective
consciousness for a long while (Kelly~\cite{KelCD},
Hermida~\cite[9.1]{HerRM}).  Around 30 years ago Kelly%
%
\index{Kelly, Max}
%
and his
collaborators began investigating 2-monads and 2-dimensional%
%
\index{algebraic theory!two-dimensional@2-dimensional}
%
algebraic
theories (see Blackwell, Kelly and Power~\cite{BKP}, for instance), and
they surely knew that unbiased monoidal categories were equivalent to
classical monoidal categories in the way described above, although I have
not been able to find anywhere this is made explicit before my
own~\cite{OHDCT}.  If I am interpreting the (somewhat daunting) literature
correctly, one can also deduce from it some of the more general results
in~\ref{sec:alg-notions} on algebraic notions of monoidal category, but
there are considerable technical issues to be understood before one is in a
position to do so.  Here, in contrast, we have a quick and natural route to
the results.  Essentially, operads take the place of 2-monads as the means
of describing an algebraic theory on $\Cat$.  In order even to \emph{state}
the problem at hand, in any approach, the explicit or implicit use of
operads seems inevitable; an advantage of our approach is that we use
nothing more.

The idea that a monoidal category is a multicategory with enough universal
arrows goes back to Lambek's paper introducing
multicategories~\cite{LamDSCII} and is implicit in the definition of weak
$n$-category proposed by Baez and Dolan~\cite{BDHDA3}, but as far as I know
did not appear in print until Hermida~\cite{HerRM}.%
%
\index{Hermida, Claudio}
%

Homotopy monoidal categories were studied in my own~\cite{HAO} paper and
its introductory companion~\cite{UTHM}, in the following wider context:
given an operad $P$ and a monoidal category $\cat{A}$ with a distinguished
class of maps called `equivalences', there is a notion of `homotopy%
%
\index{homotopy-algebraic structure}%
%
\index{operad!homotopy-algebra for}
%
$P$-algebra in $\cat{A}$'.  The case $P = 1$, $\cat{A} = \Cat$ gives
homotopy monoidal categories; the case of homotopy differential graded
algebras was mentioned on p.~\pageref{p:hty-dgas}.  Homotopy monoidal
categories are related to the work of Simpson,%
%
\index{Simpson, Carlos}
%
Tamsamani,%
%
\index{Tamsamani, Zouhair}
%
To\"en,%
%
\index{To\"en, Bertrand}
%
and
Vezzosi%
%
\index{Vezzosi, Gabriele}
%
mentioned in the Notes to Chapter~\ref{ch:other-defns}.

\end{notes}
