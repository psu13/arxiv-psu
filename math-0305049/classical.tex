
\chapter{Classical Categorical Structures}
\lbl{ch:classical}

\chapterquote{%
We will need to use some very simple notions of category theory, an
esoteric subject noted for its difficulty and irrelevance}{%
Moore and Seiberg~\cite{MoSe}}


\noindent
You might imagine that you would need to be on top of the whole of ordinary
category theory before beginning to attempt the higher-dimensional version.
Happily, this is not the case.  The main prerequisite for this book is
basic categorical language, such as may be found in most introductory texts
on the subject.  Except in the appendices, we will need few actual
theorems.

The purpose of this chapter is to recall some familiar categorical ideas
and to explain some less familiar ones.  Where the boundary lies depends,
of course, on the reader, but very little here is genuinely new.
Section~\ref{sec:cats} is on ordinary, `1-dimensional', category theory,
and is a digest of the concepts that will be used later on.  Impatient
readers will want to skip immediately to~\ref{sec:mon-cats}, monoidal
categories.  This covers the basic concepts and two kinds of coherence
theorem.  \ref{sec:cl-enr} is a short section on categories enriched in
monoidal categories.  We need enrichment in the next
section,~\ref{sec:cl-strict}, on strict $n$-categories and strict
$\omega$-categories.  This sets the scene for later chapters, where we
consider the much more profound and interesting \emph{weak} $n$-categories.
Finally, in~\ref{sec:bicats}, we discuss bicategories, the best-known
notion of weak 2-category, including coherence and their (not completely
straightforward) relation to monoidal categories.

Examples of all these structures are given.  Topological spaces and chain
complexes are, as foreshadowed in the Motivation for Topologists, a
recurring theme.


\section{Categories}
\lbl{sec:cats}

This section is a sketch of the category theory on which the rest of the
text is built.  I have also taken the opportunity to state some notation
and some small results that will eventually be needed.

In later chapters I will assume familiarity with the language of
categories, functors, and natural transformations, and the basics of limits
and adjunctions.  I will also use the basic language of monads (sometimes
still called `triples').  As monads are less well known than the other
concepts, and as they will be central to this text, I have included a short
introduction to them below.  

Given objects $A$ and $B$ of a category $\cat{A}$, I write $\cat{A}(A,B)$%
% 
\glo{homset}\index{hom-set}
%
for the set of maps (or morphisms, or arrows) from $A$ to $B$, in
preference to the less informative $\Hom(A,B)$.  The opposite%
%
\index{opposite category}%
%
\index{category!opposite}
%
or dual%
%
\index{dual!category@of category}
%
of
$\cat{A}$ is written $\cat{A}^\op$.%
% 
\glo{catop}
%
 Isomorphism%
%
\index{isomorphism}
%
between objects in a
category is written $\iso$.%
% 
\glo{iso}
%
 For any two categories $\cat{A}$ and
$\cat{B}$, there is a category $\ftrcat{\cat{A}}{\cat{B}}$%
% 
\glo{ftrcat}\index{functor!category}
%
whose objects
are functors from $\cat{A}$ to $\cat{B}$ and whose maps are natural
transformations.

My set-theoretic%
%
\index{foundations}
%
morals are lax; I have avoided questions of `size'
whenever possible.  Where the issue is unavoidable, I have used the words
\demph{small}%
%
\index{small}
%
and \demph{large}%
%
\index{large}
%
to mean `forming a set' and `forming a
proper class' respectively.  Some readers may prefer to re-interpret this
using universes.  A category is \demph{small} if its collection of arrows
is small.

The category of sets is written as $\Set$%
% 
\glo{Set}
%
and the category of small
categories as $\Cat$;%
% 
\glo{Cat}
%
occasionally I refer to $\CAT$, the (huge)
category of all categories.  There are functors
\[
\begin{diagram}[width=3em,scriptlabels]
\Cat	&\pile{\lTo^D\\ \rTo~{\mr{ob}}\\ \lTo_I}	&\Set,	\\
\end{diagram}
\glo{disccat}\glo{obcat}\glo{indisccat}
\]
where
%
\begin{itemize}
\item $D$ sends a set $A$ to the \demph{discrete}%
%
\index{category!discrete}
%
category on $A$,
whose object-set is $A$ and all of whose maps are identities
\item $\mr{ob}$ sends a category to its set of objects%
%
\index{objects functor}
%
\item $I$ sends a set $A$ to the \demph{indiscrete}%
%
\lbl{p:indiscrete}\index{category!indiscrete}
%
category on $A$, whose object-set is $A$ and which has precisely one map $a
\go b$ for each $a, b \in A$; all maps are necessarily isomorphisms.
\end{itemize}

We use \demph{limits}%
%
\index{limit}
%
(`inverse limits', `projective limits') and
\demph{colimits}%
%
\index{colimit}
%
(`direct limits', `inductive limits').  Binary product%
%
\index{product}
%
is
written as $\times$%
% 
\glo{binprod}
%
and arbitrary product as $\prod$;%
% 
\glo{arbprod}
%
dually, binary
coproduct%
%
\index{coproduct}
%
(sum)%
%
\index{sum}
%
is written as $+$%
% 
\glo{bincoprod}
%
and arbitrary coproduct as $\coprod$.%
% 
\glo{arbcoprod}
%
Nullary products are terminal%
%
\index{terminal object}
%
(final)%
%
\index{final object}
%
objects, written as $1$;%
% 
\glo{terminal}
%
in
particular, $1$ often denotes a one-element set.  The unique map from an
object $A$ of a category to a terminal object $1$ of that category is
written $!: A \go 1$.%
% 
\glo{bang}
%

We will make particular use of \demph{pullbacks}%
%
\index{pullback}
%
(fibred%
%
\index{fibred product}
%
products).
Pullback squares are indicated by right-angle marks:%
% 
\glo{rightangle}
%
\[
\begin{diagram}[size=2em]
P\SEpbk		&\rTo	&B	\\
\dTo		&	&\dTo	\\
A		&\rTo	&C.	
\end{diagram}
\]
We also write $P = A \times_C B$.%
% 
\glo{pbtimes}
%
In later chapters dozens of elementary
manipulations of diagrams involving pullback squares are left to the
hypothetical conscientious reader; almost all are made easy by the
following invaluable lemma.
%
\begin{lemma}[Pasting Lemma] \lbl{lemma:pasting}\index{Pasting Lemma}
Take a commutative diagram of shape
\[
\begin{diagram}[size=2em]
\cdot	&\rTo	&\cdot	&\rTo	&\cdot	\\
\dTo	&	&\dTo	&	&\dTo	\\
\cdot	&\rTo	&\cdot	&\rTo	&\cdot	\\
\end{diagram}
\]
in some category, and suppose that the right-hand square is a pullback.
Then the left-hand square is a pullback if and only if the outer rectangle
is a pullback.
\done
\end{lemma}

An \demph{adjunction}%
%
\index{adjunction}
%
is a pair of functors $\cat{A} \oppair{F}{G} \cat{B}$
together with an isomorphism
%
\begin{equation}	\label{eq:adjn}
\cat{B}(FA, B) \goiso \cat{A}(A, GB)
\end{equation}
%
natural in $A \in \cat{A}$ and $B\in \cat{B}$.  Then $F$ is \demph{left
adjoint} to $G$, $G$ is \demph{right adjoint} to $F$, and I write $F\ladj
G$.%
% 
\glo{ladj}
%
In most of the examples that we meet, $G$ is a forgetful functor and
$F$ the corresponding free functor.  A typical example is that $\cat{A}$ is
the category of sets, $\cat{B}$ is the category of \demph{monoids}%
%
\index{monoid}
%
(sets
equipped with an associative binary multiplication and a two-sided unit),
$G$ forgets the monoid structure, and $F$ sends a set $A$ to the monoid%
%
\index{monoid!free}
%
\[
FA = \coprod_{n\in\nat} A^n
\]
of finite sequences of elements of $A$, whose multiplication is
concatenation of sequences. 

Take an adjunction as above, and let $A \in \cat{A}$: then applying
isomorphism~\bref{eq:adjn} to $1_{FA} \in \cat{B}(FA, FA)$ yields a map
$\eta_A: A \go GFA$.  The resulting natural transformation $\eta: 1_\cat{A}
\go GF$ is the \demph{unit}%
%
\index{unit!adjunction@of adjunction}
%
of the adjunction.  Dually, there is a
\demph{counit}%
%
\index{counit}
%
$\epsln: FG \go 1_\cat{B}$, and the unit and counit satisfy
the so-called triangle%
%
\index{triangle!identities}
%
identities (Mac Lane~\cite[IV.1(9)]{MacCWM}).  In fact, an adjunction can
 equivalently be defined as a quadruple $(F, G, \eta, \epsln)$ where
%
\begin{equation}	\label{eq:adjn-data}
\cat{A} \goby{F} \cat{B},
\diagspace
\cat{B} \goby{G} \cat{A},
\diagspace
1_\cat{A} \goby{\eta} GF, 
\diagspace
FG \goby{\epsln} 1_\cat{B}
\end{equation}
%
and $\eta$ and $\epsln$ satisfy the triangle identities.

Equivalence of categories can be formulated in several ways.  By
definition, an \demph{equivalence}%
%
\index{equivalence!categories@of categories}
%
of categories $\cat{A}$ and $\cat{B}$
consists of functors and natural transformations~\bref{eq:adjn-data} such
that $\eta$ and $\epsln$ are isomorphisms.  An \demph{adjoint%
%
\index{adjoint equivalence}
%
equivalence}
is an adjunction $(F, G, \eta, \epsln)$ that is also an equivalence.  A
functor $F: \cat{A} \go \cat{B}$ is \demph{essentially%
%
\index{essentially surjective on objects}
%
surjective on
objects} if for all $B \in \cat{B}$ there exists $A \in \cat{A}$ such that
$FA \iso B$.
%
\begin{propn}	\lbl{propn:eqv-eqv}
The following conditions on a functor $F: \cat{A} \go \cat{B}$ are
equivalent: 
%
\begin{enumerate}
\item \lbl{item:eqv-eqv-adjt-eqv}
there exist $G$, $\eta$ and $\epsln$ such that $(F, G, \eta, \epsln)$ is an
adjoint equivalence 
\item \lbl{item:eqv-eqv-eqv}
there exist $G$, $\eta$ and $\epsln$ such that $(F, G, \eta, \epsln)$ is an
equivalence 
\item \lbl{item:eqv-eqv-ffe}
$F$ is full, faithful, and essentially surjective on objects.
\end{enumerate}
\end{propn}
%
\begin{proof}
See Mac Lane~\cite[IV.4.1]{MacCWM}.  \done
\end{proof}
%
If the conditions of the proposition are satisfied then the functor $F$ is
called an \demph{equivalence}.%
%
\index{equivalence!categories@of categories}
%
 If the categories $\cat{A}$ and $\cat{B}$
are equivalent then we write $\cat{A} \eqv \cat{B}$%
% 
\glo{eqvcat}
%
(in contrast to
$\cat{A} \iso \cat{B}$, which denotes isomorphism).


\index{monad|(}
%
Monads are a remarkably economical formalization of the notion of
`algebraic%
%
\index{algebraic theory}
%
theory', traditionally formalized by universal algebraists in
various rather concrete and inflexible ways.  For example, there is a monad
corresponding to the theory of rings, another monad for the theory of
complex Lie algebras, another for the theory of topological groups, another
for the theory of strict $10$-categories, and so on, as we shall see.

A monad on a category $\cat{A}$ can be defined as a monoid in the monoidal
category $(\ftrcat{\cat{A}}{\cat{A}}, \of, 1_\cat{A})$ of endofunctors on
$\cat{A}$.
% category $(\End(\cat{A}), \of, 1_\cat{A})$ of endofunctors on $\cat{A}$.
Explicitly:
%
\begin{defn}	\lbl{defn:monad}
A \demph{monad} on a category $\cat{A}$ consists of a functor $T: \cat{A}
\go \cat{A}$ together with natural transformations
\[
\mu: T\of T \go T,
\diagspace
\eta: 1_\cat{A} \go T,
\]
called the \demph{multiplication}%
%
\index{multiplication of monad}
%
and \demph{unit}%
%
\index{unit!monad@of monad}
%
respectively, such that
the diagrams
\[
\begin{slopeydiag}
	&	&T\of T\of T	&	&	\\
	&\ldTo<{T \mu}&		&\rdTo>{\mu T}&	\\
T\of T	&	&		&	&T\of T	\\
	&\rdTo<\mu&		&\ldTo>\mu&	\\
	&	&T		&	&	\\
\end{slopeydiag}
\diagspace
\begin{slopeydiag}
	&		&T\of 1		\\
	&\ldTo<{T \eta}	&		\\
T\of T	&		&\dTo>\id	\\
	&\rdTo<\mu	&		\\
	&		&T		\\
\end{slopeydiag}
\diagspace
\begin{slopeydiag}
1\of T		&		&	\\
		&\rdTo>{\eta T}	&	\\
\dTo<\id	&		&T\of T	\\
		&\ldTo>\mu	&	\\
T		&		&	\\
\end{slopeydiag}
\]
commute (the \demph{associativity} and \demph{unit} laws).
\end{defn}

Any adjunction
\[
\begin{diagram}[height=2em]
\cat{B}	\\
\uTo<F \ladj \dTo>G	\\
\cat{A}	\\
\end{diagram}
\]
induces a monad $(T, \mu, \eta)$ on $\cat{A}$: take $T= G\of F$, $\eta$
to be the unit of the adjunction, and 
\[
\mu = G\epsln F: GFGF \go GF
\]
where $\epsln$ is the counit.  Often $\cat{A}$ is the category of sets and
$\cat{B}$ is a category of `algebras' of some kind, as in the following
examples. 

\begin{example}	\lbl{eg:monad-monoid}
Take the free-forgetful%
%
\index{monoid!free}
%
adjunction between the category of monoids and the
category of sets, as above.  Let $(T, \mu, \eta)$ be the induced monad.
Then $TA$ is $\coprod_{n\in\nat} A^n$, the set of finite sequences of
elements of $A$, for any set $A$.  The multiplication $\mu$ strips inner
brackets from double sequences:
\[
\begin{array}{rcl}
T(TA)		&\goby{\mu_A}		&TA,	\\

((a_1^1, \ldots, a_1^{k_1}), \ldots, (a_n^1, \ldots, a_n^{k_n}))	&
\goesto	&
(a_1^1, \ldots, a_1^{k_1}, \ldots, a_n^1, \ldots, a_n^{k_n})	
\end{array}
\]
($n, k_i \in \nat, a_i^j \in A$).  The unit $\eta$ forms sequences of
length 1:
\[
\begin{array}{rcl}
A	&\goby{\eta_A}	&TA,	\\
a	&\goesto	&(a).
\end{array}
\]
\end{example}

\begin{example}	\lbl{eg:monad-R-mod}
Fix a ring $R$.  One can form the free $R$-module%
%
\index{module!ring@over ring}
%
on any given set, and
conversely one can take the underlying set of any $R$-module, giving an
adjunction 
\[
\begin{diagram}[height=2em]
R\hyph\fcat{Mod}	\\
\uTo<F \ladj \dTo>G	\\
\Set	\\
\end{diagram}%
% 
\glo{RMod}
%
\]
hence a monad $(T, \mu, \eta)$ on $\Set$.  Explicitly, if $A$ is a set then
$TA$ is the set of formal $R$-linear combinations of elements of $A$.  The
multiplication $\mu$ realizes a formal linear combination of formal linear
combinations as a single formal linear combination, and the unit $\eta$
realizes an element of a set $A$ as a trivial linear combination of
elements of $A$.
\end{example}

\begin{example}	\lbl{eg:monad-alg-thy}
The same goes for all other `algebraic%
%
\index{algebraic theory}
%
theories': groups, Lie algebras,
Boolean algebras, \ldots.  The functor $T$ sends a set $A$ to the set of
formal words in the set $A$ (which in some cases, such as that of groups,
is cumbersome to describe).  There is no need for the ambient category
$\cat{A}$ to be $\Set$: the theory of topological groups,%
%
\index{group!topological}
%
for instance,
gives a monad on the category $\Top$%
% 
\glo{Topcat}
% 
of topological spaces.
\end{example}

A monad is meant to be an algebraic theory, so if we are handed a monad
then we ought to be able to say what its `models' are.  For instance, if we
are handed the monad of~\ref{eg:monad-R-mod} then its `models' should be
exactly $R$-modules.  Formally, if $T = (T, \mu, \eta)$ is a monad on a
category $\cat{A}$ then a \demph{$T$-algebra}%
%
\index{algebra!monad@for monad}
%
is an object $A \in \cat{A}$
together with a map $h: TA \go A$ compatible with the multiplication and
unit of the monad: see Mac Lane~\cite[VI.2]{MacCWM} for the axioms.  In the
case of~\ref{eg:monad-R-mod}, a $T$-algebra is a set $A$ equipped with a
function
\[
h:
\{ \textrm{formal }R \textrm{-linear combinations of elements of }A \}
\go 
A
\]
satisfying some axioms, and this does indeed amount exactly to an
$R$-module.%
%
\index{module!ring@over ring}
%

The category of algebras for a monad $T = (T, \mu, \eta)$ on a category
$\cat{A}$ is written $\cat{A}^T$.  There is an evident forgetful functor
$\cat{A}^T \go \cat{A}$, this has a left adjoint (forming `free%
%
\index{algebra!monad@for monad!free}
%
$T$-algebras'), and the monad on $\cat{A}$ induced by this adjunction is
just the original $T$.  So every monad arises from an adjunction, and
informally we have
\[
\{\textrm{monads on } \cat{A} \}
\subset
\{\textrm{adjunctions based on } \cat{A} \}.
\]
The inclusion is proper: not every adjunction is of the form $\cat{A}^T
\pile{\rTo_\top \\ \lTo} \cat{A}$ just described.  For instance, the
forgetful functor $\Top \go \Set$ has a left adjoint (forming discrete
spaces); the induced monad on $\Set$ is the identity, whose category of
algebras is merely $\Set$, and $\Set \not\eqv \Top$.  The adjunctions that
do arise from monads are called \demph{monadic}.%
%
\index{monadic adjunction}
%
 All of the adjunctions in
Examples \ref{eg:monad-monoid}--\ref{eg:monad-alg-thy} are monadic, and the
non-monadicity of the adjunction $\Top \pile{\rTo_\top \\ \lTo} \Set$
expresses the thought that topology%
%
\index{topology vs. algebra@topology \vs.\ algebra}
%
is not algebra.%
%
\index{monad|)}
%

\index{presheaf|(}
%
Presheaves will be important.  A \demph{presheaf} on a category $\cat{A}$
is a functor $\cat{A}^\op \go \Set$.  Any object $A \in \cat{A}$ gives rise
to a presheaf $\cat{A}(\dashbk, A)$ on $\cat{A}$, and this defines a
functor
\[
\begin{array}{rcl}
\cat{A}		&\go		&\ftrcat{\cat{A}^\op}{\Set}	\\
A		&\goesto	&\cat{A}(\dashbk, A),
\end{array}
\]
the \demph{Yoneda embedding}.%
%
\index{Yoneda!embedding}
%
 It is full and faithful.  This follows from
the Yoneda Lemma,%
%
\index{Yoneda!Lemma}
%
which states that if $A \in \cat{A}$ and $X$ is a
presheaf on $\cat{A}$ then natural transformations $\cat{A}(\dashbk, A) \go
X$ correspond one-to-one with elements of $XA$.

If $\Eee$ is a category and $S$ a set then there is a category $\Eee^S$,%
% 
\glo{catpower}
%
a
power%
%
\index{power of category}
%
of $\Eee$, whose objects are $S$-indexed families of objects of
$\Eee$.  On the other hand, if $\Eee$ is a category and $E$ an object of
$\Eee$ then there is a \demph{slice%
%
\index{slice!category}
%
category} $\cat{E}/E$,%
% 
\glo{slicecat}
%
whose objects
are maps $D \goby{p} E$ into $E$ and whose maps are commutative triangles.
If $S$ is a set then there is an equivalence of categories
%
\begin{equation}	\label{eq:Set-slice-power}
\Set^S \eqv \Set/S,
\end{equation}
%
given in one direction by taking the disjoint union of an $S$-indexed
family of sets, and in the other by taking fibres of a set over $S$.

There is an analogue of~\bref{eq:Set-slice-power} in which the set $S$ is
replaced by a category.  Fix a small category $\scat{A}$.  The replacement
for $\Set^S$ is $\ftrcat{\scat{A}^\op}{\Set}$, but what should replace the
slice category $\Set/S$?  First note that any presheaf $X$ on $\scat{A}$
gives rise to a category $\scat{A}/X$,%
% 
\glo{catelts}
%
the \demph{category of elements}%
% 
\lbl{p:defn-caty-elts}\index{category!elements@of elements}
% 
of $X$, whose objects are pairs $(A, x)$ with $A \in \scat{A}$
and $x \in XA$ and whose maps
\[
(A', x') \go (A, x)
\]
are maps $f: A' \go A$ in $\scat{A}$ such that $x' = (Xf)(x)$.  There is an
evident forgetful functor $\scat{A}/X \go \scat{A}$,
the \demph{Grothendieck fibration}%
%
\index{fibration!Grothendieck}
%
of $X$.  This is an example of a
\demph{discrete fibration},%
%
\index{fibration!discrete}
%
that is, a functor $G: \cat{D} \go \cat{C}$
such that
%
\begin{quote}
  for any object $D \in \cat{D}$ and map $C' \goby{p} GD$ in $\cat{C}$,
  there is a unique map $D' \goby{q} D$ in $\cat{D}$ such that $Gq = p$. 
\end{quote}
%
Discrete fibrations over $\cat{C}$ (that is, with codomain $\cat{C}$) can
be made into a category $\fcat{DFib}(\cat{C})$%
% 
\glo{DFibcat}
%
in a natural way, and this
is the desired generalization of slice category.  We then have an
equivalence
\[
\ftrcat{\scat{A}^\op}{\Set}
\eqv
\fcat{DFib}(\scat{A}),
\]
given in one direction by taking categories of elements, and in the other
by taking fibres in a suitable sense.  There is also a dual notion of
\demph{discrete opfibration},%
%
\lbl{p:defn-cl-d-opfib}\index{fibration!discrete opfibration} 
%
and an equivalence
\[
\ftrcat{\scat{A}}{\Set}
\eqv
\fcat{DOpfib}(\scat{A}).%
% 
\glo{DOpfibcat}
%
\]

A \demph{presheaf%
%
\index{presheaf!category}
%
category} is a category equivalent to
$\ftrcat{\scat{A}^\op}{\Set}$ for some small $\scat{A}$.  The class of
presheaf categories is closed under slicing:
%
\begin{propn}	\lbl{propn:pshf-slice}
Let $\scat{A}$ be a small category and $X$ a presheaf on $\scat{A}$.  Then
there is an equivalence of categories
\[
\ftrcat{\scat{A}^\op}{\Set} / X
\eqv
\ftrcat{(\scat{A}/X)^\op}{\Set}.
\]
\ \done
\end{propn}%
%
\index{presheaf|)}
%

\index{internal!algebraic structure|(}%
% 
Finally, we will need just a whisper of internal
category theory.  If
$\cat{A}$ is any category with finite products then an \demph{(internal)%
%
\index{group!internal}
%
group}
in $\cat{A}$ consists of an object $A \in \cat{A}$ together with
maps $m: A \times A \go A$ (multiplication), $e: 1 \go A$ (unit), and $i: A
\go A$ (inverses), such that certain diagrams expressing the group axioms
commute.  Thus, a group in $\Set$ is an ordinary group, a group in the
category of smooth manifolds is a Lie group, and so on.  A similar
definition pertains for algebraic structures other than groups.  Categories
themselves can be defined in this way: if $\cat{A}$ is any category with
pullbacks then an \demph{(internal)%
%
\index{category!internal}
%
category} $C$ in $\cat{A}$ is a diagram
\[
\begin{slopeydiag}
	&	&C_1	&	&	\\
	&\ldTo<\dom&	&\rdTo>\cod&	\\
C_0	&	&	&	&C_0	\\
\end{slopeydiag}%
% 
\glo{C0cat}\glo{C1cat}\glo{domcat}\glo{codcat}
%
\]
in $\cat{A}$ together with maps
\[
C_1 \times_{C_0} C_1 \goby{\comp} C_1,
\diagspace
C_0 \goby{\ids} C_1%
% 
\glo{compcat}\glo{idscat}
%
\]
in $\cat{A}$, satisfying certain axioms.  Here $C_1 \times_{C_0} C_1$
is the pullback%
%
\lbl{p:defn-caty-pb}
%
\[
\begin{slopeydiag}
	&	&C_1 \times_{C_0} C_1 \Spbk	&	&	\\
	&\ldTo	&				&\rdTo	&	\\
C_1	&	&				&	&C_1	\\
	&\rdTo<\cod&				&\ldTo>\dom&	\\
	&	&C_0.				&	&	\\
\end{slopeydiag}
\]
When $\cat{A} = \Set$ we recover the usual notion of small category: $C_0$
and $C_1$ are the sets of objects and of arrows, $\dom$ and $\cod$ are the
domain and codomain functions, $C_1 \times_{C_0} C_1$ is the set of
composable pairs of arrows, $\comp$ and $\ids$ are the functions
determining binary composition and identity maps, and the axioms specify
the domain and codomain of composites and identities and express
associativity and identity laws.  When $\cat{A} = \Top$ we obtain a notion
of `topological%
%
\index{topological category}\index{category!topological}
%
category',
in which both the set of objects and the set of
arrows carry a topology.  For instance, given any space $X$, there is a
topological category $C = \Pi_1 X$%
% 
\glo{Pi1top}\index{fundamental!1-groupoid}
%
in which $C_0 = X$ and $C_1$ is $X^{[0,1]}/\sim$, the space of all paths in
$X$ factored out by path homotopy relative to endpoints.%
%
\index{internal!algebraic structure|)}
%


\section{Monoidal categories}
\lbl{sec:mon-cats}


Monoidal categories come in a variety of flavours: strict, weak, plain,
braided, symmetric.  We look briefly at strict monoidal categories but
spend most time on the more important weak case and on the coherence
theorem: every weak monoidal category is equivalent to a strict one.

In the terminology of the previous section, a \demph{strict%
%
\index{monoidal category!strict}
%
monoidal
category} is an internal monoid in $\Cat$, that is, a category $\cat{A}$
equipped with a functor
\[
\begin{array}{rrcl}
\otimes:	&\cat{A} \times \cat{A}	&\go		&\cat{A},	\\
		&(A, B)			&\goesto	&A\otimes B
\end{array}%
% 
\glo{otimes}
%
\]
and an object $I\in \cat{A}$,%
% 
\glo{unitobj}
%
obeying strict associativity and unit laws:
\[
(A\otimes B)\otimes C 
=
A\otimes (B\otimes C),
\diagspace
I\otimes A
=
A,
\diagspace
A\otimes I
= 
A
\]
for all objects $A, B, C \in \cat{A}$, and similarly for morphisms.
Functoriality of $\otimes$ encodes the `interchange%
%
\index{interchange}
%
laws':
% 
\begin{equation}	\label{eq:mon-interchange}
(g' \of f') \otimes (g \of f)
=
(g' \otimes g) \of (f' \otimes f)
\end{equation}
% 
for all maps $g', f', f, f$ for which these composites make sense, and $1_A
\otimes 1_B = 1_{A \otimes B}$ for all objects $A$ and $B$.

Since one is usually not interested in equality of objects in a category,
only in isomorphism, strict monoidal categories are quite rare.

\begin{example}	\lbl{eg:str-mon-endo}
The category $\ftrcat{\cat{C}}{\cat{C}}$ of endofunctors on a given
category $\cat{C}$ has a strict monoidal structure given by composition (as
$\otimes$) and 
% the identity functor 
$1_{\cat{C}}$ (as $I$).
\end{example}

\begin{example}	\lbl{eg:str-mon-D}
Given a natural number $n$ (possibly $0$), let $\lwr{n}$%
% 
\glo{lwrn}
%
denote the
$n$-element set $\{1, \ldots, n\}$ with its usual total order.  Let
$\scat{D}$%
% 
\glo{augsimplexcat}\index{augmented simplex category $\scat{D}$}
%
be the category whose objects are the natural numbers and whose maps $m\go
n$ are the order-preserving functions $\lwr{m} \go \lwr{n}$.  This is the
`augmented simplex category', one object bigger than the standard
topologists' $\Delta$, and is equivalent to the category of (possibly
empty) finite totally ordered sets.  It has a strict monoidal structure
given by addition and $\lwr{0}$.
\end{example}

\begin{example}	\lbl{eg:str-mon-comm}
A category with only one object is just a monoid
(p.~\pageref{p:degen-cat-monoid}): if the category is called $\cat{A}$ and
its single object is called $\star$ then the monoid is the set
$M = \cat{A}(\star, \star)$ with composition $\of$ as multiplication and the
identity $1 = 1_\star$ as unit.  A one-object strict monoidal%
%
\index{monoidal category!degenerate}
%
category therefore
consists of a set $M$ with monoid structures $(\of, 1)$ and $(\otimes, 1)$
(the latter being tensor of arrows in the monoidal category), such
that~\bref{eq:mon-interchange} holds for all $g', f', g, f \in M$.
Lemma~\ref{lemma:EH} below tells us that this forces the binary operations
$\of$ and $\otimes$ to be equal and commutative.  So a one-object strict
monoidal category is just a commutative%
%
\index{monoid!commutative}
%
monoid.
\end{example}

\begin{lemma}[Eckmann--Hilton~\cite{EH}]	\lbl{lemma:EH}%
%
\index{Eckmann--Hilton argument}
%
Suppose that $\of$ and $\otimes$ are binary operations on a set $M$,
satisfying~\bref{eq:mon-interchange} for all $g', f', g, f \in M$, and
suppose that $\of$ and $\otimes$ share a two-sided unit.  Then $\of =
\otimes$ and $\of$ is commutative.
% and associative.  
\end{lemma}
%
\begin{proof}
Write $1$ for the unit.  Then for $g, f \in M$,
\[
g \of f
=
(g \otimes 1) \of (1 \otimes f)
=
(g \of 1) \otimes (1 \of f)
=
g \otimes f,
\]
so $\of = \otimes$, and 
\[
g \of f
=
(1 \otimes g) \of (f \otimes 1)
=
(1 \of f) \otimes (g \of 1)
=
f \otimes g,
\]
so $\of$ is commutative.  
\done
\end{proof}

Much more common are weak monoidal categories, usually just called
`monoidal categories'.
%
\begin{defn}	\lbl{defn:mon-cat}
A \demph{(weak) monoidal category}%
%
\index{monoidal category!classical}
%
is a category $\cat{A}$ together with a
functor 
% \[
$
\otimes: \cat{A} \times \cat{A} \go \cat{A},
$
% \]
% 
an object $I\in\cat{A}$ (the \demph{unit}), and isomorphisms
\[
(A\otimes B) \otimes C
\rTo^{\alpha_{A, B, C}}_{\diso}
A\otimes (B\otimes C),
\diagspace
I \otimes A
\rTo^{\lambda_A}_{\diso}
A,
\diagspace
A\otimes I
\rTo^{\rho_A}_{\diso}
A%
% 
\glo{monalpha}\glo{monlambda}\glo{monrho}
%
\]%
%
\index{associativity!isomorphism}%
%
\index{unit!isomorphism}%
%
natural in $A, B, C \in \cat{A}$ (\demph{coherence%
%
\index{coherence!isomorphism}
%
isomorphisms}), such
that the following diagrams commute for all $A, B, C, D \in \cat{A}$:
\[
\begin{array}{c}
\begin{diagram}[width=4em,height=2em,scriptlabels,tight,noPS]
	&	&	&
(A\otimes B) \otimes (C\otimes D)
				&	&	&	\\
	&	&
\ruTo(3,2)<{\alpha_{A\otimes B, C, D}}
			&	&	
\rdTo(3,2)>{\alpha_{A, B, C\otimes D}}
					&	&	\\
((A\otimes B) \otimes C)\otimes D
	&	&	&	&	&	&
A \otimes (B\otimes (C \otimes D))	\\
	&\rdTo(1,2)<{\alpha_{A, B, C} \otimes 1_D}
		&	&	&	&
\ruTo(1,2)>{1_A \otimes \alpha_{B, C, D}}
						&	\\
	&	
(A\otimes (B\otimes C))\otimes D
		&	&
\rTo_{\alpha_{A, B\otimes C, D}}
			&	&	
A\otimes ((B\otimes C)\otimes D)
					&	&	\\
\end{diagram}
% 
\\
\\
%
%
\index{pentagon}\index{triangle!coherence axiom}
%
%
\begin{diagram}[size=2em,scriptlabels,noPS]
(A\otimes I) \otimes B
	&	&
\rTo^{\alpha_{A, I, B}}
			&	&
A\otimes (I\otimes B)	
					\\
	&
\rdTo<{\rho_A \otimes 1_B}	
		&	&
\ldTo>{1_A \otimes \lambda_B}	
				&	\\
	&	&
A\otimes B.	
			&	&	\\
\end{diagram}
\end{array}
\]
\end{defn}
% 
The pentagon and triangle axioms ensure that `all diagrams' constructed out
of coherence isomorphisms commute.  This is one form of the coherence
theorem, discussed below.

\begin{example}
Strict monoidal categories can be identified with monoidal categories in
which all the components of $\alpha$, $\lambda$ and $\rho$ are identities.
\end{example}

\begin{example}
Let $\cat{A}$ be a category in which all finite%
%
\index{category!finite product}\index{monoidal category!cartesian}
%
products exist.
Choose a particular terminal object $1$, and for each $A, B \in
\cat{A}$ a particular product diagram $A \ogby{\mr{pr}_1} A\times B
\goby{\mr{pr}_2} B$.%
% 
\glo{pri}
%
 Then $\cat{A}$ acquires a monoidal structure
with $\otimes = \times$ and $I = 1$; the maps $\alpha$, $\lambda$ and
$\rho$ are the canonical ones.
\end{example}

\begin{example}
For any commutative ring $R$, the category of $R$-modules%
%
\index{module!ring@over ring}
%
is monoidal with respect to the usual tensor $\otimes_R$ and unit object
$R$.
\end{example}

\begin{example}	\lbl{eg:mon-cat-loops}
Take a topological space with basepoint.  There is a monoidal category
whose objects are loops%
%
\index{loop space}
%
on the basepoint and whose maps are homotopy classes of loop homotopies
(relative to the basepoint).  We have to take homotopy classes so that the
ordinary categorical composition obeys associativity and identity laws.
Tensor is concatenation of loops (on objects) and gluing of homotopies (on
maps).  The coherence isomorphisms are the evident reparametrizations.
\end{example}

Earlier we met the notion of (internal) algebraic structures, such as
groups, in a category with finite products.  There is no clear way to
extend this to arbitrary monoidal categories, since to express an axiom
such as $x \cdot x^{-1} = 1$ diagrammatically requires the
product-projections.  We can, however, define a \demph{monoid}%
%
\index{monoid!monoidal category@in monoidal category}
%
in a
monoidal category $\cat{A}$ as an object $A$ together with maps
\[
m: A \otimes A \go A,
\diagspace
e: I \go A
\]
such that associativity and unit diagrams similar to those in
Definition~\ref{defn:monad} commute.  With the obvious notion of map, this
gives a category $\Mon(\cat{A})$%
% 
\glo{Monofcat}
%
of monoids in $\cat{A}$.  When $\cat{A}$
is the category of sets, with product as monoidal structure, this is the
usual category of monoids.

There are various notions of map between monoidal categories.  In what
follows we use $\otimes$, $I$, $\alpha$, $\lambda$, and $\rho$ to denote
the monoidal structure of both the categories concerned.   
%
\begin{defn}	\lbl{defn:mon-ftr}
Let $\cat{A}$ and $\cat{A'}$ be monoidal categories.  A \demph{lax%
%
\index{monoidal functor!classical|(}
%
monoidal
functor} $F = (F, \phi): \cat{A} \go \cat{A'}$ is a functor $F: \cat{A} \go
\cat{A'}$ together with \demph{coherence%
%
\index{coherence!map}
%
maps}
\[
\phi_{A, B}: FA \otimes FB \go F(A\otimes B),
\diagspace
\phi_\cdot: I \go FI
\]
in $\cat{A'}$, the former natural in $A, B \in \cat{A}$, such that the
following diagrams commute for all $A, B, C \in \cat{A}$:
\[
\begin{array}{c}
\begin{diagram}[size=2em,scriptlabels]
(FA \otimes FB) \otimes FC	&\rTo^{\phi_{A,B} \otimes 1_{FC}}	&
F(A\otimes B) \otimes FC	&\rTo^{\phi_{A\otimes B, C}}		&
F((A\otimes B)\otimes C)	\\
\dTo<{\alpha_{FA, FB, FC}}	&					&
				&					&
\dTo>{F\alpha_{A, B, C}}	\\
FA \otimes (FB \otimes FC)	&\rTo_{1_{FA} \otimes \phi_{B,C}}	&
FA \otimes F(B \otimes C)	&\rTo_{\phi_{A, B\otimes C}}		&
F(A\otimes (B\otimes C))	\\
\end{diagram}
\\
\\
\begin{diagram}[size=2em,scriptlabels]
FA \otimes I	&\rTo^{1_{FA} \otimes \phi_\cdot}	&
% FA \otimes I	&\rTo^{1 \otimes \phi_\cdot}	&
FA \otimes FI	&\rTo^{\phi_{A, I}}			&
F(A\otimes I)	\\
\dTo<{\rho_{FA}}&					&
		&					&
\dTo>{F\rho_A}	\\
FA		&					&
\rEquals	&					&
FA		\\
\end{diagram}
% 
% \diagspace
\\
\\
% 
\begin{diagram}[size=2em,scriptlabels]
I\otimes FA	&\rTo^{\phi_\cdot \otimes 1_{FA}}	&
% I\otimes FA	&\rTo^{\phi_\cdot \otimes 1}	&
FI \otimes FA	&\rTo^{\phi_{I, A}}			&
F(I\otimes A)	\\
\dTo<{\lambda_{FA}}&					&
		&					&
\dTo>{F\lambda_A}\\
FA		&					&
\rEquals	&					&
FA.		\\
\end{diagram}
\end{array}
\]
A \demph{colax monoidal functor} $F = (F, \phi): \cat{A} \go \cat{A'}$ is a
functor $F: \cat{A} \go \cat{A'}$ together with maps
\[
\phi_{A, B}: F(A \otimes B) \go FA \otimes FB,
\diagspace
\phi_\cdot: FI \go I
\]
satisfying axioms dual to those above.  A \demph{weak} (respectively,
\demph{strict}) \demph{monoidal functor} is a lax monoidal functor $(F,
\phi)$ in which all the maps $\phi_{A, B}$ and $\phi_\cdot$ are
isomorphisms (respectively, identities).
\end{defn}
% 
We write $\MClax$%
% 
\glo{MClax}
%
for the category of monoidal categories and lax maps, and similarly
$\fcat{MonCat}_\mr{colax}$, $\MCwk$, and $\MCstr$.  There are various
alternative systems of terminology; in particular, what we call weak
monoidal functors are sometimes called `strong%
%
\index{monoidal functor!classical|)}
%
monoidal functors' or just
`monoidal functors'.
% 
\begin{example}	%\lbl{eg:lax-mon-Ab}
The forgetful functor $U: \Ab \go \Set$%
% 
\glo{Ab}
%
from abelian groups to sets has a
lax monoidal structure with respect to the usual monoidal structures on
$\Ab$ and $\Set$, given by the canonical maps
\[
UA \times UB \go U(A \otimes B)
\]
($A, B \in \Ab$) and the map $1 \go U\integers$ picking out $0\in\integers$.
\end{example}

\begin{example}	\lbl{eg:mon-cat-action}
Let $\cat{C}$ be a category and $\cat{A}$ a monoidal category.  A weak
monoidal functor from $\cat{A}$ to the monoidal category
$\ftrcat{\cat{C}}{\cat{C}}$ of endofunctors on
$\cat{C}$~(\ref{eg:str-mon-endo}) is called an \demph{action}%
%
\index{action!monoidal category@of monoidal category}
%
of $\cat{A}$
on $\cat{C}$, and amounts to a functor $\cat{A} \times \cat{C} \go \cat{C}$
together with coherence isomorphisms satisfying axioms.
\end{example}

To state one of the forms of the coherence theorem we will need a notion of
equivalence of monoidal categories, and for this we need in turn a notion
of transformation.
%
\begin{defn}
Let $(F, \phi), (G, \psi): \cat{A} \go \cat{A'}$ be lax monoidal functors.
A \demph{monoidal%
%
\index{monoidal transformation!classical}
%
transformation} $(F, \phi) \go (G, \psi)$ is a natural
transformation $\sigma: F \go G$ such that the following diagrams commute
for all $A, B \in \cat{A}$:
\[
\begin{diagram}[size=2em,scriptlabels]
FA \otimes FB	&\rTo^{\sigma_A \otimes \sigma_B}	&GA \otimes GB	\\
\dTo<{\phi_{A,B}}&					&
\dTo>{\psi_{A,B}}\\
F(A\otimes B)	&\rTo_{\sigma_{A\otimes B}}		&G(A\otimes B)	\\
\end{diagram}
% 
\diagspace
% 
\begin{diagram}[size=2em,scriptlabels]
I			&\rEquals	&I			\\
\dTo<{\phi_\cdot}	&		&\dTo>{\psi_\cdot}	\\
FI			&\rTo_{\sigma_I}&GI.			\\
\end{diagram}
\]
\end{defn}

A weak monoidal functor $(F,\phi)$ is called an \demph{equivalence}%
%
\index{equivalence!classical monoidal categories@of classical monoidal categories} 
%
of
monoidal categories if it satisfies the conditions of the following
proposition.  
% Condition~\bref{item:mon-eqv-long} is the more morally
% correct, and condition~\bref{item:mon-eqv-short} the more convenient.
% 
\begin{propn}	\lbl{propn:mon-eqv-eqv}
The following conditions on a weak monoidal functor $(F, \phi): \cat{A} \go
\cat{A'}$ are equivalent:
%
\begin{enumerate}
\item	\lbl{item:mon-eqv-long}
there exist a weak monoidal functor $(G, \psi): \cat{A'} \go \cat{A}$ and
invertible monoidal transformations
\[
\eta: 1_{\cat{A}} \go (G, \psi)\of (F, \phi),
\diagspace
\epsln: (F, \phi) \of (G, \psi) \go 1_{\cat{A'}}
\]
\item	\lbl{item:mon-eqv-short}
the functor $F$ is an equivalence of categories.
\end{enumerate}
\end{propn}
% 
\begin{proof}
If $F$ is an equivalence of categories then by
Proposition~\ref{propn:eqv-eqv}, there exist a functor $G$ and
transformations $\eta$ and $\epsln$ such that $(F, G, \eta, \epsln)$ is an
adjoint equivalence.  It is easy to verify that $G$ acquires a weak
monoidal structure and that $\eta$ and $\epsln$ are then invertible
monoidal transformations.
% The opposite implication is trivial.
\done
\end{proof}

\index{coherence!monoidal categories@for monoidal categories!classical|(} 
%
A coherence%
%
\index{coherence}
%
theorem is, roughly, a description of a structure that makes it
more manageable.%
%   
\lbl{p:coherence-discussion}
% 
For example, one coherence theorem for monoidal categories is that all
diagrams built out of the coherence isomorphisms commute.  Another is that
any weak monoidal category is equivalent to some strict monoidal category.
All non-trivial applications of monoidal categories rely on a coherence
theorem in some form; the axioms as they stand are just too unwieldy.
Indeed, one might argue that the `all diagrams commute' principle should be
an explicit part of the definition of monoidal category, and we will take
this approach when we come to define higher-dimensional categorical
structures.

`All diagrams commute' can be made precise in various ways.  A very direct
statement is in Mac%
%
\index{Mac Lane, Saunders}
%
Lane~\cite[VII.2]{MacCWM}, and a less direct (but
sharper) statement is~\ref{thm:diag-coh-mc} below.  A typical instance is
that the diagram
%
\begin{equation}	\label{diag:typical-coh}
\begin{diagram}[height=2em,width=2em,scriptlabels]
		&		&
(A\otimes (I\otimes B)) \otimes C
					&		&	\\
		&
\ldTo<{\alpha_{A, I, B}^{-1} \otimes 1}
				&	&
\rdTo(2,3)>{\alpha_{A, I\otimes B, C}}	
							&	\\
((A\otimes I) \otimes B)\otimes C	
		&		&	&		&	\\
\dTo<{(\rho_A \otimes 1)\otimes 1}	
		&		&	&		&
A \otimes ((I\otimes B) \otimes C)				\\
(A\otimes B)\otimes C		
		&		&	&
\ldTo(2,3)>{1\otimes (\lambda_B \otimes 1)}		
							&	\\
		&
\rdTo<{\alpha_{A, B, C}}	
				&	&		&	\\
		&		&
A\otimes (B\otimes C)	
					&		&	\\
\end{diagram}
\end{equation}
%
commutes for all objects $A, B, C$ of a monoidal category.  We will soon
see how this follows from the alternative form of the coherence theorem:
%
\begin{thm}[Coherence for monoidal categories]
\lbl{thm:coh-mon-wk-str}
Every monoidal category is equivalent to some strict monoidal category.
\end{thm}
%
Here is how \emph{not}%
%
\index{coherence!how not to prove}
%
to prove this: take a monoidal category $\cat{A}$, form a quotient strict
monoidal category $\cat{A'}$ by turning isomorphism into equality, and show
that the natural map $\cat{A} \go \cat{A'}$ is an equivalence.  To see why
this fails, first note that a monoidal category may have the property that
any two isomorphic objects are equal (and so, in particular, the tensor
product of objects is strictly associative and unital), but even so need
not be strict---the coherence isomorphisms need not be identities.  An
example can be found in Mac Lane~\cite[VII.1]{MacCWM}.  If $\cat{A}$ has
this property then identifying isomorphic objects of $\cat{A}$ has no
effect at all.  One might attempt to go further by identifying the
coherence isomorphisms with identities---for instance, identifying the two
maps
\[
(A\otimes B) \otimes C 
\parpair{\alpha_{A, B, C}}{1}
A \otimes (B\otimes C)
\]
---to make a strict monoidal category $\cat{A'}$; but then the quotient map
$\cat{A} \go \cat{A'}$ is not faithful, so not an equivalence.

\begin{prooflike}{Sketch proof of~\ref{thm:coh-mon-wk-str}}
This is a modification of Joyal%
%
\index{Joyal, Andr\'e|(}%
%
\index{Street, Ross!coherence for monoidal categories@on coherence for monoidal categories|(}
%
and Street's proof~\cite[1.4]{JSBTC}.  Let
$\cat{A}$ be a monoidal category.  We define a strict monoidal category
$\cat{A'}$ and a monoidal equivalence $\mathbf{y}: \cat{A} \go \cat{A'}$.
An object of $\cat{A'}$ is a pair $(E, \delta)$ where $E$ is an endofunctor
of the (unadorned) category $\cat{A}$ and $\delta$ is a family of
isomorphisms 
\[
\left(
\delta_{A, B}: (EA) \otimes B \goiso E(A\otimes B)
\right)_{A, B \in \cat{A}}
\]
natural in $A$ and $B$ and satisfying the evident coherence axioms.  Tensor
in $\cat{A'}$ is
\[
(E', \delta') \otimes (E, \delta) 
=
(E'\of E, \delta'')
\]
where $\delta''$ is defined in the only sensible way; then $\cat{A'}$ is a
strict monoidal category.  The functor $\mathbf{y}$ is given by
\[
\mathbf{y}(Z) = (Z\otimes\dashbk, \alpha_{Z, \dashbk, \dashbk}).
\]
It is weak monoidal and full, faithful and essentially surjective on
objects, so by~\ref{propn:mon-eqv-eqv} an equivalence of monoidal
categories.  \done
\end{prooflike}
%
Joyal and Street motivate their proof as a generalization of the Cayley%
%
\index{Cayley representation}%
%
\index{representation theorem}
%
%
Theorem representing any group as a group of permutations.  We find another
way of looking at it when we come to bicategories~(\ref{sec:bicats}).

Now let us deduce that `all diagrams commute', or at least, by way of
example, that diagram~\bref{diag:typical-coh} commutes.  For any objects
$A$, $B$ and $C$ of any monoidal category, write 
\[
(A \otimes (I\otimes B)) \otimes C
\parpair{\chi_{A, B, C}}{\omega_{A, B, C}}
A \otimes (B\otimes C)
\]
for the two composite coherence maps shown in~\bref{diag:typical-coh}.  Now
take a particular monoidal category $\cat{A}$ and objects $A, B, C \in
\cat{A}$; we want to conclude that $\chi_{A, B, C} = \omega_{A, B, C}$.  We
have a monoidal equivalence $(F, \phi)$ from $\cat{A}$ to a strict monoidal
category $\cat{A'}$, and a serially commutative diagram
\[
\begin{diagram}[size=2em,scriptlabels]
(FA \otimes (I\otimes FB)) \otimes FC	&
\pile{\rTo^{\chi_{FA, FB, FC}}\\ \rTo_{\omega_{FA, FB, FC}}}
					&
FA \otimes (FB \otimes FC)		\\
\dTo<{\phi}				&
					&
\dTo>{\phi}				\\
F( (A\otimes (I\otimes B)) \otimes C )	&
\pile{\rTo^{F \chi_{A, B, C}}\\ \rTo_{F \omega_{A, B, C}}}
		 			&
F( A \otimes (B\otimes C) )		\\
\end{diagram}
\]
where the maps labelled $\phi$ are built out of $\phi_\cdot$ and various
$\phi_{D, E}$'s.  (`Serially commutative' means that both the top and the
bottom square commute.)  Since $\cat{A'}$ is strict, $\chi_{FA, FB, FC} =
\omega_{FA, FB, FC}$; then since the $\phi$'s are isomorphisms, $F\chi_{A,
B, C} = F\omega_{A, B, C}$; then since $F$ is faithful, $\chi_{A, B, C} =
\omega_{A, B, C}$, as required.

There are similar diagrammatic coherence theorems for monoidal functors,
saying, for instance, that if $(F, \phi)$ is a lax monoidal functor then
any two maps
\[
((FA \otimes FB) \otimes I) \otimes (FC \otimes FD)
\parpairu
F(A \otimes ((B\otimes C) \otimes D))
\]
built out of copies of the coherence maps are equal.  The form of the
codomain is important, being $F$ applied to a product of objects; in
contrast, the coherence maps can be assembled to give two maps
\[
FI 
\parpairu 
FI \otimes FI
\]
that are in general not equal.  See Lewis~\cite{Lew}%
%
\index{Lewis, Geoffrey}
%
for both this
counterexample and a precise statement of coherence for monoidal functors.

Many everyday monoidal categories have a natural symmetric structure.
Formally, a symmetric monoidal category is a monoidal category $\cat{A}$
together with a specified isomorphism $\gamma_{A, B}: A \otimes B \go B
\otimes A$ for each pair $(A, B)$ of objects, satisfying coherence axioms.
There are various coherence theorems for symmetric monoidal categories
(Mac%
%
\index{Mac Lane, Saunders}
%
Lane~\cite{MacNAC}, Joyal and Street~\cite{JSBTC}).  Beware, however, that
the symmetry isomorphisms $\gamma_{A, B}$ cannot be turned into identities:
every symmetric monoidal category is equivalent to some symmetric strict
monoidal category, but not usually to any strict symmetric monoidal
category.  The latter structures---commutative monoids in $\Cat$---are
rare.%
%
\index{coherence!monoidal categories@for monoidal categories!classical|)} 
%

More general than symmetric monoidal categories are braided%
%
\index{monoidal category!braided}
%
monoidal
categories, mentioned in the Motivation for Topologists.  See Joyal%
%
\index{Joyal, Andr\'e|)}
% 
and
Street~\cite{JSBTC}%
%
\index{Street, Ross!coherence for monoidal categories@on coherence for monoidal categories|)}
% 
for the definitions and coherence theorems, and
Gordon,%
%
\index{Gordon, Robert}%
%
\index{tricategory}
%
Power and Street~\cite{GPS} for a 3-dimensional perspective.



\section{Enrichment}
\lbl{sec:cl-enr}

In many basic examples of categories $\cat{C}$, the hom-sets $\cat{C}(A,
B)$ are richer than mere sets.  For instance, if $\cat{C}$ is a category of
chain complexes then $\cat{C}(A, B)$ is an abelian group, and if $\cat{C}$
is a suitable category of topological spaces then $\cat{C}(A, B)$ is itself
a space.

This idea is called `enrichment' and can be formalized in various ways.
The best-known, enrichment in a monoidal category, is presented here.  We
will see later~(\ref{sec:enr-mtis}) that it is not the most natural or
general formalization, but it serves a purpose before we reach that point.

\begin{defn}	\lbl{defn:V-gph}
Let $\cat{V}$ be a category.  A \demph{$\cat{V}$-graph}%
%
\index{graph!enriched}
%
$X$ is a set $X_0$
together with a family $(X(x,x'))_{x,x' \in X_0}$ of objects of $\cat{V}$.
A \demph{map of $\cat{V}$-graphs} $f: X \go Y$ is a function $f_0: X_0 \go
Y_0$ together with a family of maps
\[
\left( X(x, x') \goby{f_{x, x'}} Y(f_0 x, f_0 x') \right)_{x, x' \in X_0}.
\]
We usually write both $f_0$ and $f_{x, x'}$ as just $f$.  The category of
$\cat{V}$-graphs is written $\cat{V}\hyph\Gph$.%
% 
\glo{VGph}
%
\end{defn}
% 
A $\Set$-graph is, then, an ordinary directed graph.  A category is a
 directed graph equipped with composition and identities, suggesting the
 following definition.
% 
\begin{defn}	\lbl{defn:cl-enr-cat}
Let $(\cat{V}, \otimes, I)$ be a monoidal category.  A \demph{category
enriched in $\cat{V}$},%
%
\index{enrichment!category@of category!monoidal category@in monoidal category} 
%
or \demph{\cat{V}-enriched
category}, is a
$\cat{V}$-graph $A$ together with families of maps
\[
\left(
A(b, c) \otimes A(a, b) \goby{\comp_{a,b,c}} A(a,c)
\right)_{a, b, c \in A_0},
\diagspace
\left(
I \goby{\ids_a} A(a,a)
\right)_{a \in A_0}%
% 
\glo{compenr}\glo{idsenr}
%
\]
in $\cat{V}$ satisfying associativity and identity axioms (expressed as
commutative diagrams).  A \demph{\cat{V}-enriched%
%
\index{enrichment!functor@of functor}%
%
\index{functor!enriched}
%
functor} $F: A \go B$ is
a map of the underlying $\cat{V}$-graphs commuting with the composition
maps $\comp_{a,b,c}$ and identity maps $\ids_a$.  This defines a category
$\cat{V}\hyph\Cat$.%
% 
\glo{VCat}
%
\end{defn}
% 
A $(\Set, \times, 1)$-enriched category is, of course, an ordinary (small)
category, and an $(\Ab, \otimes, \integers)$-enriched category is an
$\Ab$-category%
%
\index{Ab-category@$\Ab$-category}
%
in the sense of homological algebra.  A one-object
$\cat{V}$-enriched category is a monoid%
%
\index{monoid!monoidal category@in monoidal category}
%
in the monoidal category $\cat{V}$.
Compare and contrast internal%
%
\index{internal!enriched@\vs.\ enriched}%
%
\index{enrichment!internal@\vs.\ internal}
%
and enriched categories: in the case of
topological spaces, for instance, an internal category in $\Top$ is an
ordinary category equipped with a topology on the set of objects and a
topology on the set of all arrows, whereas a category enriched in $\Top$ is
an ordinary category equipped with a topology on each hom-set.

Any lax monoidal functor $Q = (Q, \phi): \cat{V} \go \cat{W}$ induces a
functor
\[
Q_*: \cat{V}\hyph\Cat \go \cat{W}\hyph\Cat.%
% 
\glo{starenr}
%
\]
In particular, if $\cat{V}$ is any monoidal category then the functor
$\cat{V}(I, \dashbk): \cat{V} \go \Set$ has a natural lax monoidal
structure, and the induced functor defines the \demph{underlying%
%
\index{enrichment!category@of category!underlying category}
%
category}
of a $\cat{V}$-enriched category.  This does exactly what we would expect
in the familiar cases of $\cat{V}$.

In the next section we will enrich in categories $\cat{V}$ whose monoidal
structure is ordinary (cartesian) product.  The following result will be
useful; its proof is straightforward.
%
\begin{propn}	\lbl{propn:fin-prod-enr}
\begin{enumerate}
\item	\lbl{item:fin-prod-enr-cat}
  If $\cat{V}$ is a category with finite products then the category
  $\cat{V}\hyph\Cat$ also has finite products.
\item	\lbl{item:fin-prod-enr-ftr}
  If $Q: \cat{V} \go \cat{W}$ is a finite-product-preserving functor
  between categories with finite products then the induced functor $Q_*:
  \cat{V}\hyph\Cat \go \cat{W}\hyph\Cat$ also preserves finite products.
\done
\end{enumerate}
\end{propn}

The theory of categories enriched in a monoidal category can be taken much
further: see Kelly~\cite{KelBCE}, for instance.  Under the assumption that
$\cat{V}$ is symmetric monoidal closed and has all limits and colimits,
very large parts of ordinary category theory can be extended to the
$\cat{V}$-enriched context.




\section{Strict $n$-categories}
\lbl{sec:cl-strict}


Strict $n$-categories are not encountered nearly as often as their weak
cousins, but there are nevertheless some significant examples.

We start with a very short definition of strict $n$-category, and some
examples.  This definition, being iterative, can seem opaque, so we then
formulate a much longer but equivalent definition providing a complementary
viewpoint.  We then look briefly at the infinite-dimensional case, strict
$\omega$-categories, and at the cubical analogue of strict $n$-categories.

The definition of strict $n$-category uses enrichment%
%
\index{enrichment!category@of category!finite product category@in finite product category}%
%
\index{enrichment!define n-category@to define $n$-category} 
%
in a category with
finite products.  
%
\begin{defn}	\lbl{defn:strict-n-cat-enr}
Let $\left( \strcat{n} \right)_{n\in\nat}$%
% 
\glo{strcatn}
%
be the sequence of categories
given inductively by
\[
\strcat{0} = \Set,
\diagspace
\strcat{(n+1)} = (\strcat{n})\hyph\Cat.
\]
A \demph{strict $n$-category}%
%
\index{n-category@$n$-category!strict}
%
is an object of $\strcat{n}$, and a
\demph{strict $n$-functor}%
%
\index{n-functor, strict@$n$-functor, strict}
%
is a map in $\strcat{n}$.
\end{defn}
%
This makes sense by
Proposition~\ref{propn:fin-prod-enr}\bref{item:fin-prod-enr-cat}.  

Strict $0$-categories are sets and strict $1$-categories are categories.  A
strict $2$-category%
%
\index{two-category@2-category!strict}
%
$A$ consists of a set $A_0$, a category $A(a,b)$ for
each $a, b \in A_0$, composition functors as in~\ref{defn:cl-enr-cat}, and
an identity object of $A(a,a)$ for each $a\in A_0$, all obeying
associativity and identity laws.  

\begin{example}
There is a (large) strict 2-category $\cat{A}$ in which $\cat{A}_0$ is the
class of topological%
%
\index{topological space!two-category of spaces@2-category of spaces}
%
spaces and, for spaces $X$ and $Y$, $\cat{A}(X, Y)$ is
the category whose objects are continuous maps $X \go Y$ and whose arrows
are homotopy
classes%
%
\index{homotopy!classes}
% 
of homotopies.  (We need to take homotopy classes so
that composition in $\cat{A}(X, Y)$ is associative and unital.)  The
composition functors
% 
\begin{equation}	\label{eq:2-cat-comp-ftr}
\cat{A}(Y, Z) \times \cat{A}(X, Y) \go \cat{A}(X, Z)
\end{equation}
%
and the identity objects $1_X \in \cat{A}(X, X)$ are the obvious ones.
\end{example}

\begin{example}
Similarly, there is a strict 2-category $\cat{A}$ in which $\cat{A}_0$ is
the class of chain%
%
\index{chain complex!two-category of complexes@2-category of complexes}
%
complexes and $\cat{A}(X, Y)$ is the category whose objects are chain maps
$X \go Y$ and whose arrows are homotopy classes%
%
\index{homotopy!classes}
%
of chain homotopies, in the
sense of p.~\pageref{p:ch-hty-hty}.  (This time we need to take homotopy
classes in order that the composition functors~\bref{eq:2-cat-comp-ftr}
really are functorial.)
\end{example}

\begin{example}	\lbl{eg:str-2-cat-Cat}
There is, self-referentially, a strict 2-category $\Cat$%
% 
\glo{Cat2cat}%
\index{two-category@2-category!categories@of categories}%
%
\index{category!two-category of@2-category of}
%
of categories.
Here $\Cat_0$ is the class of small categories and $\Cat(C, D)$ is the
functor category $\ftrcat{C}{D}$.  In fact, there is for each $n\in\nat$ a
strict $(n+1)$-category%
%
\index{n-category@$n$-category!n-ZZZcategory of@$(n+1)$-category of}
%
of strict $n$-categories: it can be proved by
induction that for each $n$ the category $\strcat{n}$ is cartesian closed,
which implies that it is naturally enriched in itself, in other words,
forms a strict $(n+1)$-category.  Sensitive readers may find this shocking:
the entire $(n+1)$-category of strict $n$-categories can be extracted from
the mere $1$-category.
\end{example}

We now build up to an alternative and more explicit definition of strict
$n$-category.  

A category can be regarded as a directed graph%
%
\index{graph!directed}
%
with structure.  The
most obvious $n$-dimensional analogue of a directed graph uses spherical or
`globular' shapes, as in the following definition.
%
\begin{defn}
Let $n\in\nat$.  An \demph{$n$-globular%
%
\index{n-globular set@$n$-globular set}
%
set} $X$ is a diagram 
\[
X(n)
\parpair{s}{t}
X(n-1)
\parpair{s}{t}
\diagspace 
\cdots
\diagspace 
\parpair{s}{t}
X(0)%
% 
\glo{globsce}\glo{globtgt}
%
\]
of sets and functions, such that
%
\begin{equation}	\label{eq:glob}
s(s(x)) = s(t(x)),
\diagspace
t(s(x)) = t(t(x))
\end{equation}
% 
for all $m\in \{2, \ldots, n \}$ and $x\in X(m)$.
\end{defn}

Alternatively, an $n$-globular set is a presheaf on the category
$\scat{G}_n$%
% 
\glo{Gn}
%
generated by objects and arrows
\[
n
\pile{\lTo^{\scriptstyle \sigma_n}\\ \lTo_{\scriptstyle \tau_n}}
n-1
\pile{\lTo^{\scriptstyle \sigma_{n-1}}\\ \lTo_{\scriptstyle \tau_{n-1}}}
\diagspace
\cdots
\diagspace
\pile{\lTo^{\scriptstyle \sigma_1}\\ \lTo_{\scriptstyle \tau_1}}
0
\]
subject to equations
\[
\sigma_m \of \sigma_{m-1} = \tau_m \of \sigma_{m-1},
\diagspace
\sigma_m \of \tau_{m-1} = \tau_m \of \tau_{m-1}
\]
($m \in \{ 2, \ldots, n \}$).  The category of $n$-globular sets can then
be defined as the presheaf category $\ftrcat{\scat{G}_n^\op}{\Set}$.

Let $X$ be an $n$-globular set.  Elements of $X(m)$ are called
\demph{$m$-cells}%
%
\index{cell!globular set@of globular set}
%
of $X$ and drawn as labels on an $m$-dimensional disk.
Thus, $a \in X(0)$ is drawn as
\[
\gzero{a}
\]
(and sometimes called an \demph{object}%
%
\index{object}
%
rather than a 0-cell), and $f \in
X(1)$ is drawn as
\[
\gfst{a} \gone{f} \glst{b}
\]
where $a = s(f)$ and $b = t(f)$.  We call $s(x)$ the \demph{source}%
%
\index{source}
%
of $x$,
and $t(x)$ the \demph{target};%
%
\index{target}
%
these are alternative names for `domain' and
`codomain'.  A 2-cell $\alpha \in X(2)$ is drawn as
\[
\gfst{a} \gtwo{f}{g}{\alpha} \glst{b}
\]
where
\[
f = s(\alpha),
\diagspace
g = t(\alpha),
\diagspace
a = s(f) = s(g),
\diagspace
b = t(f) = t(g).
\]
That $s(f) = s(g)$ and $t(f) = t(g)$ follows from the globularity
equations~\bref{eq:glob}.  A 3-cell $x \in X(3)$ is drawn as
\[
\gfst{a} \gthreecell{f}{g}{\alpha}{\beta}{x} \glst{b}
\]
where $\alpha = s(x)$, $\beta = t(x)$, and so on.  Sometimes, as in the
Motivation for Topologists, I have used double-shafted arrows%
%
\index{arrow!shafts of}\index{cell!depiction of}
%
for 2-cells,
triple-shafted arrows for 3-cells, and so on, and sometimes, as here, I
have stuck to single-shafted arrows for cells of all dimensions; this is
a purely visual choice.

\begin{example}	\lbl{eg:n-glob-set-Pi}
Let $n\in\nat$ and let $S$ be a topological space: then there is an
$n$-globular%
%
\index{n-globular set@$n$-globular set!space@from space}%
%
\index{topological space!n-globular set from@$n$-globular set from}
%
%
set in which an $m$-cell is a labelled $m$-dimensional disk in
$S$.  Formally, let $D^m$%
% 
\glo{disk}
%
be the closed $m$-dimensional Euclidean disk%
%
\index{disk}
%
(ball), and consider the diagram
\[
D^n
\pile{\lTo\\ \lTo}
D^{n-1}
\pile{\lTo\\ \lTo}
\diagspace
\cdots 
\diagspace 
\pile{\lTo\\ \lTo}
D^0 = 1
\]
formed by embedding $D^{m-1}$ as the upper or lower cap of $D^m$.  This is
a functor $\scat{G}_n \go \Top$ (an `$n$-coglobular space'), and so induces
a functor $\scat{G}_n^\op \go \Set$, the $n$-globular set
\[
\Top(D^n, S)
\pile{\rTo\\ \rTo}
\Top(D^{n-1}, S)
\pile{\rTo\\ \rTo}
\diagspace
\cdots 
\diagspace 
\pile{\rTo\\ \rTo}
\Top(D^0, S).
\]
\end{example}

\begin{example}	\lbl{eg:n-glob-set-ch-cx}
Analogously, any non-negatively graded chain%
%
\index{chain complex!n-globular set from@$n$-globular set from}
%
complex $C$ of abelian groups
give rise to an $n$-globular%
%
\index{n-globular set@$n$-globular set!chain complex@from chain complex}
%
set $X$ for each $n\in\nat$.  An $m$-cell of
$X$ is not quite just an element of $C_m$; for instance, we regard an
element $f$ of $C_1$ as a 1-cell $a \go b$ for any $a, b \in C_0$ such that
$d(f) = b - a$.  In general, an element of $X(m)$ is a $(2m+1)$-tuple
\[
\mathbf{c} =
(c_m, c_{m-1}^-, c_{m-1}^+, c_{m-2}^-, c_{m-2}^+, \ldots, c_0^-, c_0^+)
\]
where $c_m \in C_m$, $c_p^-, c_p^+ \in C_p$, and 
\[
d(c_m) = c_{m-1}^+ - c_{m-1}^-,
\diagspace
d(c_p^-) = d(c_p^+) = c_{p-1}^+ - c_{p-1}^-
\]
for all $p \in \{1, \ldots, m-1\}$.  We then put
\[
s(\mathbf{c}) = 
(c_{m-1}^-, c_{m-2}^-, c_{m-2}^+, \ldots, c_0^-, c_0^+)
\]
and dually the target.
\end{example}

In the following alternative definition, a strict $n$-category is an
$n$-globular set equipped with identities and various binary composition
operations, satisfying various axioms.  Identities are simple: every
$p$-cell $x$ has an identity $(p+1)$-cell $1_x$ on it ($0\leq p < n$), as
in
\[
\gzero{a} \ \goesto\  \gfst{a}\gone{1_a}\glst{a},
\diagspace
\gfst{a}\gone{f}\glst{b} \ \goesto\  \gfst{a}\gtwo{f}{f}{1_f}\glst{b}
\]
($p = 0, 1$).  There are $m$ different binary composition operations for
$m$-cells.  When $m=1$ this is ordinary categorical composition.  We saw
the 2 possibilities for composing 2-cells on p.~\pageref{p:2-cell-comps}.
The 3 ways of composing 3-cells are drawn as
\[
\gfstsu\gthreecellu\gzersu\gthreecellu\glstsu,
\diagspace
\gfstsu\gspecialthree\glstsu,
\diagspace
\gfstsu\gspecialtwo\glstsu.
\]

To express this formally we write, for any $n$-globular set $A$ and $0\leq
p\leq m\leq n$, 
\[
A(m) \times_{A(p)} A(m) 
=
\{
(x', x) \in A(m) \times A(m)
\such
t^{m-p}(x) = s^{m-p}(x') 
\},
\]
the set of pairs of $m$-cells with the potential to be joined along
$p$-cells.

\begin{defn}	\lbl{defn:strict-n-cat-glob}
Let $n\in\nat$.  A \demph{strict%
%
\index{n-category@$n$-category!strict}
%
$n$-category} is an $n$-globular set $A$
equipped with
%
\begin{itemize}
\item a function $\ofdim{p}: A(m) \times_{A(p)} A(m) \go A(m)$%
% 
\glo{ofdim}
%
for each $0\leq
  p < m \leq n$; we write $\ofdim{p}(x', x)$ as $x' \ofdim{p} x$ and call
  it a \demph{composite} of $x$ and $x'$
% 
\item a function $i: A(p) \go A(p+1)$%
% 
\glo{istrncat}
%
for each $0 \leq p < n$; we write $i(x)$ as $1_x$%
% 
\glo{onestrncat}
%
and call it the
\demph{identity} on $x$,
\end{itemize}
%
satisfying the following axioms:
%
\begin{enumerate}
\item	\lbl{item:s-t-comp}
(sources and targets of composites) if $0\leq p < m \leq n$ and $(x',
x) \in A(m) \times_{A(p)} A(m)$ then
\[
\begin{array}{llll}
s(x' \ofdim{p} x) = 
s(x) 			&	
\textrm{and}			&
t(x' \ofdim{p} x) = 
t(x') 			&
\textrm{if }
p = m-1	\\
s(x' \ofdim{p} x) = 
s(x') \ofdim{p} s(x)	&
\textrm{and}			&
t(x' \ofdim{p} x) = 
t(x') \ofdim{p} t(x)	&
\textrm{if }
p \leq m-2
\end{array}
\]
\item	%\lbl{item:s-t-ids}
(sources and targets of identities) if $0 \leq p < n$ and $x \in
  A(p)$ then $s(1_x) = x = t(1_x)$
\item (associativity) if $0\leq p < m \leq n$ and $x, x', x''
  \in A(m)$ with $(x'', x'), (x', x) \in A(m) \times_{A(p)} A(m)$ then
  \[
  (x'' \ofdim{p} x') \ofdim{p} x 
  = 
  x'' \ofdim{p} (x' \ofdim{p} x)
  \]
\item (identities) if $0\leq p < m \leq n$ and $x\in A(m)$ then 
  \[
  i^{m-p}(t^{m-p}(x)) \ofdim{p} x
  = 
  x
  =
  x \ofdim{p} i^{m-p}(s^{m-p}(x))
  \]
\item (binary interchange)%
%
\index{interchange}
%
if $0\leq q < p < m \leq n$ and $x, x', y, y'
  \in A(m)$ with
  \[
  (y', y), (x', x) \in A(m) \times_{A(p)} A(m),
  \ \ 
  (y', x'), (y, x) \in A(m) \times_{A(q)} A(m)
  \]
  then 
  \[
  (y' \ofdim{p} y) \ofdim{q} (x' \ofdim{p} x) 
  = 
  (y' \ofdim{q} x') \ofdim{p} (y \ofdim{q} x)
  \]
\item (nullary interchange)%
%
\index{interchange}
%
if $0\leq q < p < n$ and $(x', x) \in A(p)
  \times_{A(q)} A(p)$ then $1_{x'} \ofdim{q} 1_x = 1_{x' \sof_q x}$.
\end{enumerate}
%
If $A$ and $B$ are strict $n$-categories then a \demph{strict $n$-functor}%
%
\index{n-functor, strict@$n$-functor, strict}
%
is a map $f: A \go B$ of the underlying $n$-globular sets commuting with
composition and identities.  This defines a category $\strcat{n}$ of strict
$n$-categories. 
\end{defn}

\begin{propn}	\lbl{propn:str-n-cats-comparison}
The categories $\strcat{n}$ defined in~\ref{defn:strict-n-cat-enr}
and~\ref{defn:strict-n-cat-glob} are equivalent.
\end{propn}
%
\begin{prooflike}{Sketch proof}
We first compare the underlying graph structures.  Define for each
$n\in\nat$ the category $n\hyph\Gph$%
% 
\glo{nGph}
%
of \demph{$n$-graphs}%
%
\index{n-graph@$n$-graph}
%
by
\[
0\hyph\Gph = \Set,
\diagspace
(n+1)\hyph\Gph = (n\hyph\Gph)\hyph\Gph.
\]
An $(n+1)$-globular set amounts to a graph of $n$-globular sets: precisely,
an $(n+1)$-globular set $X$ corresponds to the graph $(X(a,b))_{a, b \in
X(0)}$ where $X(a,b)$ is the $n$-globular set defined by
\[
(X(a,b))(m) 
=
\{
x \in X(m+1)
\such
s^{m+1}(x) = a,\ 
t^{m+1}(x) = b
\}.
\]
So by induction, $n\hyph\Gph \eqv \ftrcat{\scat{G}_n^\op}{\Set}$.

Now we bring in the algebra.  Given a strict $(n+1)$-category $A$ in the
sense of~\ref{defn:strict-n-cat-glob}, the functions $\ofdim{p}$ and $i:
A(p) \go A(p+1)$ taken over $1\leq p < n$ give a strict $n$-category
structure on $A(a,b)$ for each $a, b \in A(0)$.  This determines a graph
$(A(a,b))_{a, b \in A(0)}$ of strict $n$-categories.  Moreover, the
functions $\ofdim{0}$ and $i: A(0) \go A(1)$ give this graph the structure
of a category enriched in $\strcat{n}$.  With a little work we find that a
strict $(n+1)$-category in the sense of~\ref{defn:strict-n-cat-glob} is, in
fact, \emph{exactly} a category enriched in $\strcat{n}$, and by induction
we are done.  \done
\end{prooflike}

Some examples of $n$-categories are more easily described using one
definition than the other.  It also helps to have the terminology of both
at hand.  We can now say that the strict 2-category $\Cat$
of~\ref{eg:str-2-cat-Cat} has categories as 0-cells, functors as 1-cells,
and natural transformations as 2-cells.  There are two kinds of composition
of natural transformations.  We usually write the (`vertical')%
%
\index{composition!vertical}
%
composite
of natural transformations
\[
C \cthree{F}{G}{H}{\alpha}{\beta} C'
\]
as $\beta\of\alpha$%
% 
\glo{vertcompnat}
%
(rather than the standard $\beta\ofdim{1}\alpha$) and
the (`horizontal')%
%
\index{composition!horizontal}
%
composite of natural transformations
\[
C \ctwo{F}{G}{\alpha} C' \ctwo{F'}{G'}{\alpha'} C''
\]
as $\alpha' * \alpha$%
% 
\glo{horizcompnat}
%
(rather than $\alpha' \ofdim{0} \alpha$).

\begin{example}	\lbl{eg:str-n-cat-ch-cx}
Let $n\in\nat$ and let $C$ be a chain%
%
\index{chain complex!n-category from@$n$-category from}
%
complex, as
in~\ref{eg:n-glob-set-ch-cx}.  Then the $n$-globular set $X$ arising from
$C$ has the structure of a strict $n$-category:%
%
\index{n-category@$n$-category!chain complex@from chain complex}
%
if $(\mathbf{d},
\mathbf{c}) \in X(m) \times_{X(p)} X(m)$ then $\mathbf{d} \ofdim{p}
\mathbf{c} = \mathbf{e}$ where, for $0\leq r \leq n$ and $\sigma \in \{ -,
+ \}$,
\[
e_r^\sigma
=
\left\{
\begin{array}{ll}
d_r^\sigma + c_r^\sigma	&\textrm{if } r > p	\\
c_p^-			&\textrm{if } r = p \textrm{ and } \sigma = -	\\
d_p^+			&\textrm{if } r = p \textrm{ and } \sigma = +	\\
c_r^\sigma = d_r^\sigma	&\textrm{if } r < p.
\end{array}
\right.
\]
\end{example}

Strict $\omega$-categories can also be defined in both styles.  The
globular or `global' definition is obvious.  Let $\scat{G}$%
% 
\glo{G}
%
be the category
generated by objects and arrows
\[
\cdots
\diagspace
\pile{\lTo^{\scriptstyle \sigma_{m+1}}\\ \lTo_{\scriptstyle \tau_{m+1}}}
m
\pile{\lTo^{\scriptstyle \sigma_m}\\ \lTo_{\scriptstyle \tau_m}}
m-1
\pile{\lTo^{\scriptstyle \sigma_{m-1}}\\ \lTo_{\scriptstyle \tau_{m-1}}}
\diagspace
\cdots
\diagspace
\pile{\lTo^{\scriptstyle \sigma_1}\\ \lTo_{\scriptstyle \tau_1}}
0
\]
subject to the usual globularity equations.  Then
$\ftrcat{\scat{G}^\op}{\Set}$ is the category of \demph{globular%
%
\index{globular set}
%
sets}, and
the category $\strcat{\omega}$%
% 
\glo{strcatomega}
%
of \demph{strict $\omega$-categories}%
%
\index{omega-category@$\omega$-category!strict}
%
is
defined just as in~\ref{defn:strict-n-cat-glob} but without the upper limit
of $n$.  

\begin{example}	\lbl{eg:str-omega-ch-cx}
Any chain%
%
\index{chain complex!omega-category from@$\omega$-category from}
%
complex $C$ gives rise to a strict $\omega$-category%
%
\index{omega-category@$\omega$-category!chain complex@from chain complex}
%
$X$, just as
in the previous example.  In fact, $X$ is an abelian group in
$\strcat{\omega}$, and in this way the category of abelian groups in
$\strcat{\omega}$ is equivalent to the category of non-negatively graded
chain complexes of abelian groups.
\end{example}

For the enriched or `local' definition, we first define a sequence
\[
\label{p:forgetful-strict-n}
\cdots 
\goby{S_{n+1}} 
\strcat{(n+1)}
\goby{S_n}
\strcat{n}
\goby{S_{n-1}}
\diagspace
\cdots
\diagspace
% \goby{S_1}
\Cat
\goby{S_0}
\Set%
% 
\glo{Sstrncat}
%
\]
of finite-product-preserving functors by $S_0 = \mr{ob}$ and $S_{n+1} =
(S_n)_*$, which is possible by
Proposition~\ref{propn:fin-prod-enr}\bref{item:fin-prod-enr-ftr}.  We then
take $\strcat{\omega}$ to be the limit of this diagram in $\CAT$.  It is
easy to prove
%
\begin{propn}
The two categories $\strcat{\omega}$ just defined are equivalent.
\done
\end{propn}


Strict $n$-categories use globular shapes; there is also a cubical
analogue.  Again, there is a short inductive definition and a longer
explicit version.  The short form uses internal rather than enriched
categories.  If $\cat{V}$ is a category with pullbacks then we write
$\Cat(\cat{V})$%
% 
\glo{Catinternal}
%
for the category of internal%
%
\index{category!internal}
%
categories in $\cat{V}$,
which, it can be shown, also has pullbacks.  
%
\begin{defn}
Let $(\strtuplecat{n})_{n\in\nat}$ be the sequence of categories given
inductively by
\[
\strtuplecat{0} = \Set,
\diagspace
\strtuplecat{(n+1)} = \Cat(\strtuplecat{n}).
\]
A \demph{strict $n$-tuple category}%
%
\index{n-tuple category@$n$-tuple category!strict}
%
(or `strict cubical
$n$-category') is an object of $\strtuplecat{n}$. 
\end{defn}

A strict single ($=$ 1-tuple) category is just a category.  A strict
double%
%
\index{double category!strict}
%
($=$ 2-tuple) category $D$ is a diagram  
\[
\begin{slopeydiag}
	&	&D_1	&	&	\\
	&\ldTo<\dom&	&\rdTo>\cod&	\\
D_0	&	&	&	&D_0	\\
\end{slopeydiag}
\]
of categories and functors, with extra structure.  The objects of $D_0$ are
called the \demph{0-cells}%
%
\index{cell!strict double category@of strict double category}
%
or \demph{objects} of $D$, the maps in $D_0$ are
the \demph{vertical 1-cells} of $D$, the objects of $D_1$ are the
\demph{horizontal 1-cells} of $D$, and the maps in $D_1$ are the
\demph{2-cells} of $D$, as in the picture
%
\begin{equation}	\label{diag:str-dbl-two-cell}
\begin{fcdiagram}
a	&\rTo^m			&a'	\\
\dTo<f	&\Downarrow\tcs{\theta}	&\dTo>{f'}\\
b	&\rTo_p			&b'	\\
\end{fcdiagram}
\end{equation}
%
where $a \goby{f} b$, $a' \goby{f'} b'$ are maps in $D_0$ and $m
\goby{\theta} p$ is a map in $D_1$, with $\dom(m) = a$, $\dom(\theta)=f$,
and so on.  The `extra structure' consists of various kinds of composition
and identities, so that vertical 1-cells can be composed vertically,
horizontal 1-cells can be composed horizontally, and 2-cells can be
composed both vertically and horizontally.  Thus, any $p \times q$ grid of
2-cells has a unique 2-cell composite, for any $p, q \in \nat$.

\begin{example}	\lbl{eg:2-cub-glob}
A strict double category in which all vertical 1-cells (or all horizontal
1-cells) are identities is just a strict 2-category.  
\end{example}

\begin{example}
A strict 2-category $A$ gives rise to a strict double category $D$ in two
other ways: take the 0-cells of $D$ to be those of $A$, both the
vertical and the horizontal 1-cells of $D$ to be the 1-cells of $A$, and
the 2-cells~\bref{diag:str-dbl-two-cell} of $D$ to be the 2-cells
\[
\begin{fcdiagram}
a	&\rTo^m			&a'	\\
\dTo<f	&\swnt\tcs{\theta}	&\dTo>{f'}\\
b	&\rTo_p			&b'	\\
\end{fcdiagram}
\]
in $A$, or the same with the arrow for $\theta$ reversed.
\end{example}

The longer definition of strict $n$-tuple category is omitted since we will
not need it, but goes roughly as follows.  As for the globular case, an
$n$-tuple category is a presheaf with extra algebraic structure.  

Let $\scat{H} = \scat{G}_1 = (1 \pile{\lTo^\sigma\\ \lTo_\tau} 0)$, so that
a presheaf on $\scat{H}$ is a directed graph.%
%
\index{graph!directed}
%
 If $n\in\nat$ then an
\demph{$n$-cubical%
%
\index{cubical!set}\index{n-cubical set@$n$-cubical set}
%
set} is a presheaf on $\scat{H}^n$.  An $n$-cubical set
consists of a set $X(M)$ for each $M \sub \{1, \ldots, n\}$, and a function
$X(\xi): X(M) \go X(P)$ for each $P \sub M$ and function $\xi: M\without P
\go \{-, +\}$, satisfying functoriality axioms.  For instance, a 2-cubical
set $X$ consists of sets
%
\begin{eqnarray*}
X(\emptyset)	&=	&\{ 0 \textrm{-cells} \}	\\
X(\{1\})	&=	&\{ \textrm{vertical } 1 \textrm{-cells} \}	\\
X(\{2\})	&=	&\{ \textrm{horizontal } 1 \textrm{-cells} \}	\\
X(\{1, 2\})	&=	&\{ 2 \textrm{-cells} \}
\end{eqnarray*}
%
with various source and target functions between them. 

A strict $n$-tuple category can then be defined as an $n$-cubical set $A$
together with a composition function
\[
A(P \cup \{ m \}) \times_{A(P)} A(P \cup \{ m \})
\go
A(P \cup \{ m \}) 
\]
and an identity function
\[
A(P)
\go
A(P \cup \{ m \}) 
\]
for each $P \sub \{1, \ldots, n\}$ and $m \in \{1, \ldots, n\} \without P$,
satisfying axioms.  Strict $n$-categories can be identified with strict
$n$-tuple categories whose underlying $n$-cubical set is degenerate%
%
\index{n-tuple category@$n$-tuple category!degenerate}
%
in a
certain way, generalizing Example~\ref{eg:2-cub-glob}.



\section{Bicategories}
\lbl{sec:bicats}

Bicategories are to strict 2-categories as weak monoidal categories are to
strict monoidal categories.  There are many other formalizations of the
idea of weak 2-category (see~\ref{sec:notions-bicat}), but bicategories are
the oldest and best-known.

We define a bicategory as a category `weakly enriched' in $\Cat$.  An
alternative definition as a 2-globular set with structure can be found in
B\'enabou~\cite{Ben},%
%
\index{Benabou, Jean@B\'enabou, Jean}
%
where bicategories were introduced.

\begin{defn}	\lbl{defn:cl-bicat}
A \demph{bicategory}%
%
\index{bicategory!classical}
%
$\cat{B}$ consists of
%
\begin{itemize}
\item a class $\cat{B}_0$, whose elements are called the \demph{objects} or
\demph{$0$-cells}%
%
\index{cell!classical bicategory@of classical bicategory}
%
of $\cat{B}$
\item for each $A, B \in \cat{B}_0$, a category $\cat{B}(A, B)$, whose
  objects $f$ are called the \demph{1-cells} of $\cat{B}$ and written 
  $
  \gfsts{A} \gones{f} \glsts{B}
  $
  and whose arrows $\gamma$ are called the \demph{2-cells} of $\cat{B}$ and
  written 
  \[
  \gfst{A} \gtwo{f}{g}{\gamma} \glst{B}
  \]
\item for each $A, B, C \in \cat{B}_0$, a functor
%
\[
\cat{B}(B, C) \times \cat{B}(A, B) \go \cat{B}(A, C)
\]
(\demph{composition}), written 
%
\begin{eqnarray*}
(g, f)			&\goesto	&g \of f,	\\
(\delta, \gamma)	&\goesto	&\delta * \gamma	
\end{eqnarray*}%
% 
\glo{bicatstar}%
% 
on 1-cells $f, g$ and 2-cells $\gamma, \delta$
\item for each $A \in \cat{B}_0$, an object $1_A \in \cat{B}(A, A)$ (the
  \demph{identity} on $A$)
\item for each triple 
  $
  \gfsts{A}\gones{f}\gblws{B}\gones{g}\gblws{C}\gones{h}\glsts{D}
  $
  of 1-cells, an isomorphism
  \[
  \gfst{A}
  \gtwo{(h\of g) \of f}{h \of (g\of f)}{\!\!\!\!\!\alpha_{h, g, f}}
  \glst{D}
  \]%
% 
\glo{bicatass}%
% 
in $\cat{B}(A, D)$ (the \demph{associativity
coherence isomorphism})%
%
\index{coherence!isomorphism}%
%
\index{associativity!isomorphism}
%
\item for each 1-cell $\gfsts{A}\gones{f}\glsts{B}$, isomorphisms
  \[
  \gfst{A}
  \gtwo{1_B \of f}{f}{\lambda_f}
  \glst{B},
  \diagspace
  \gfst{A}
  \gtwo{f \of 1_A}{f}{\rho_f}
  \glst{B}
  \]%
% 
\glo{bicatlambda}\glo{bicatrho}%
% 
in $\cat{B}(A, B)$ (the \demph{unit coherence isomorphisms})%
%
\index{unit!isomorphism}
%
\end{itemize}
%
such that the coherence isomorphisms $\alpha_{f, g, h}$, $\lambda_f$, and
$\rho_f$ are natural in $f$, $g$, and $h$ and satisfy pentagon and triangle
axioms like those in~\ref{defn:mon-cat} (replacing $((A\otimes B) \otimes
C) \otimes D$ with $((k\of h) \of g) \of f$, etc.).
\end{defn}

Functoriality of composition encodes `interchange%
%
\index{interchange}
%
laws', just as for
monoidal categories (\bref{eq:mon-interchange},
p.~\pageref{eq:mon-interchange}): the two evident derived composites of a
diagram of shape
\[
\gfstsu
\gthreesu 
\gzersu
\gthreesu 
\glstsu
\]
are equal, and similarly for 2-cells formed from diagrams
$
\gfstsu
\gonesu %{f}
\gzersu
\gonesu %{g}
\glstsu.
$
 

\begin{example}	\lbl{eg:bicat-mon-cat}
A bicategory%
%
\index{bicategory!degenerate}
%
with only one object is just a monoidal category.
\end{example}

\begin{example}
Strict 2-categories can be identified with bicategories in which all the
components of $\alpha$, $\lambda$ and $\rho$ are identities.
\end{example}

\begin{example}	\lbl{eg:bicat-Pi}
Any topological space $S$ gives rise to a bicategory $\Pi_2 S$,%
% 
\glo{Pitwobicat}
% 
the
\demph{fundamental%
%
\index{fundamental!2-groupoid}
%
2-groupoid} of $S$.  (`2-groupoid' means that all the
2-cells are isomorphisms and all the 1-cells are equivalences, in the sense
defined below.)  The objects of $\Pi_2 S$ are the points of $S$.  The
1-cells $a \go b$ are the paths%
%
\index{path}
%
from $a$ to $b$, that is, the maps $f: [0,1] \go S$ satisfying $f(0) = a$
and $f(1) = b$.  The 2-cells are the homotopy classes of path homotopies,
relative to endpoints.  Any particular point $s \in S$ determines a
one-object full sub-bicategory of $\Pi_2 S$ (that is, the sub-bicategory
whose only object is $s$ and with all possible 1- and 2-cells), and this is
the monoidal category of~\ref{eg:mon-cat-loops}.

Notice, incidentally, that a chain%
%
\index{chain complex!topological space@\vs.\ topological space}%
%
\index{topological space!chain complex@\vs.\ chain complex}
%
complex gives rise to a strict
$n$-category for each $n$
(\ref{eg:str-n-cat-ch-cx},~\ref{eg:str-omega-ch-cx}) but a space gives rise
to only \emph{weak}%
%
\index{n-category@$n$-category!weak vs. strict@weak \vs.\ strict}
%
structures.  This reflects the difference in difficulty
between homology%
%
\index{homology}
%
and homotopy.%
%
\index{homotopy!homology@\vs.\ homology}
%

The fundamental $\omega$-groupoid of a space is constructed
in~\ref{eg:wk-omega-cat-Pi}.  
\end{example}

\begin{example}	\lbl{eg:bicat-Ring}%
%
\index{bicategory!rings@of rings}
%
There is a bicategory $\cat{B}$ in which objects are rings, 1-cells $A \go
B$ are $(B, A)$-bimodules, and 2-cells are maps of bimodules.  (A
\demph{$(B, A)$-bimodule}%
%
\index{module!bimodule over rings}
%
is an abelian group $M$ equipped with a left
$B$-module structure and a right $A$-module structure satisfying 
$
(b \cdot m) \cdot a = b \cdot (m \cdot a)
$
for all $b\in B$, $m\in M$, $a\in A$.)  Composition is tensor:%
%
\index{tensor!ring@over ring}
%
if $M$ is a
$(B, A)$-bimodule and $N$ a $(C, B)$-bimodule then $N \otimes_B M$ is a
$(C, A)$-bimodule. 
\end{example}

\begin{example}
An $n$-category has $2^n$ duals,%
%
\index{dual!n-category@of $n$-category}%
%
\index{dual!bicategory@of bicategory}%
%
\index{bicategory!dual}
%
including the original article.  For a
bicategory $\cat{B}$, the dual obtained by reversing the 1-cells is
traditionally called $\cat{B}^\op$%
% 
\glo{bicatop}
% 
and that obtained by reversing the
2-cells is called $\cat{B}^{\mr{co}}$.%
% 
\glo{bicatco}
% 
 So the following four pictures show
corresponding 2-cells in $\cat{B}$, $\cat{B}^\op$, $\cat{B}^{\mr{co}}$ and
$\cat{B}^{\mr{co}\,\op} = \cat{B}^{\op\,\mr{co}}$:
\[
\gfsts{A}\gtwos{f}{g}{\gamma}\glsts{B},
\diagspace
\gfsts{A}\gtwoops{f}{g}{\gamma}\glsts{B},
\diagspace
\gfsts{A}\gtwocos{f}{g}{\gamma}\glsts{B},
\diagspace
\gfsts{A}\gtwocoops{f}{g}{\gamma}\glsts{B}.
\]
\end{example}

Since $\Cat$ is a bicategory~(\ref{eg:str-2-cat-Cat}), we may take
definitions from category theory and try to imitate them in an arbitrary
bicategory $\cat{B}$.  For example, a \demph{monad}%
%
\lbl{p:defn-monad-in-bicaty}\index{monad!bicategory@in bicategory}
%
in $\cat{B}$ is an object $A$ of $\cat{B}$ together with a 1-cell $A
\goby{t} A$ and 2-cells
\[
\mu: t \of t \go t, 
\diagspace
\eta: 1_A \go t,
\]
rendering commutative the diagrams of~\ref{defn:monad} (with $T$'s changed
to $t$'s, $T\mu$ changed to $1_t * \mu$, and so on).  An
\demph{adjunction}%
%
\index{adjunction!bicategory@in bicategory}
%
in $\cat{B}$ is a pair $(A, B)$ of objects together with 1- and 2-cells
% 
\begin{equation}	\label{eq:bicat-adjn-data}
A \goby{f} B,
\diagspace
B \goby{g} A,
\diagspace
1_A \goby{\eta} g \of f,
\diagspace
f\of g \goby{\epsln} 1_B
\end{equation}
% 
satisfying the triangle identities,%
%
\index{triangle!identities}
%
\[
(\epsln * 1_f) \of (1_f * \eta) = 1_f,
\diagspace
(1_g * \epsln) \of (1_g * \eta) = 1_g.
\]%
%
\index{equivalence!bicategory@in bicategory|(}%
%
An \demph{equivalence} between objects $A$ and $B$ of $\cat{B}$ is a
quadruple $(f, g, \eta, \epsln)$ of cells as in~\bref{eq:bicat-adjn-data}
such that $\eta$ and $\epsln$ are isomorphisms (in their respective
hom-categories).  An \demph{adjoint equivalence}%
%
\index{adjoint equivalence!bicategory@in bicategory}
%
is a quadruple that is
both an adjunction and an equivalence.  A 1-cell $f$ in a bicategory is
called an \demph{equivalence} if it satisfies the equivalent conditions of
the following result, which generalizes Proposition~\ref{propn:eqv-eqv}%
(\ref{item:eqv-eqv-adjt-eqv}--\ref{item:eqv-eqv-eqv}):
% 
\begin{propn}	\lbl{propn:bicat-eqv-eqv}
Let $\cat{B}$ be a bicategory.  The following conditions on a 1-cell $f: A
\go B$ in $\cat{B}$ are equivalent:
%
\begin{enumerate}
\item %\lbl{item:bicat-eqv-eqv-adjt-eqv}
there exist $g$, $\eta$ and $\epsln$ such that $(f, g, \eta, \epsln)$ is an
adjoint equivalence 
\item %\lbl{item:bicat-eqv-eqv-eqv}
there exist $g$, $\eta$ and $\epsln$ such that $(f, g, \eta, \epsln)$ is an
equivalence. 
\end{enumerate}
\end{propn}
%
\begin{proof}
More is true: given an equivalence $(f, g, \eta, \epsln)$, there is a
unique $\epsln'$ such that $(f, g, \eta, \epsln')$ is an adjoint
equivalence, namely
\[
f \of g
\goby{\epsln^{-1} * 1_{f\sof g}}
f \of g \of f \of g
\goby{1_f * \eta^{-1} * 1_g}
f \of g
\goby{\epsln}
1_B.
\]
Brackets have been omitted, as if $\cat{B}$ were a strict 2-category; the
conscientious reader can both fill in the missing coherence isomorphisms
and verify that $\eta$ and $\epsln'$ satisfy the triangle identities (a
long but elementary exercise).
\done
\end{proof}
% 
We write $A \eqv B$%
% 
\glo{eqvinbicat}
% 
if there exists an equivalence $A \go B$.%
%
\index{equivalence!bicategory@in bicategory|)}
%

Strict 2-categories form a strict 3-category~(\ref{eg:str-2-cat-Cat}),%
%
\index{n-category@$n$-category!n-ZZZcategory of@$(n+1)$-category of}
%
so
we would expect there to be notions of functor between bicategories and
transformation between functors, and then a further notion of map between
transformations.  On the other hand, one-object%
%
\index{bicategory!degenerate}
%
bicategories---monoidal
categories---only form a 2-category: there are various notions of functor
between monoidal categories, and a notion of transformation between
monoidal functors, but nothing more after that.  We will soon see how this
apparent paradox is resolved.

For functors everything goes smoothly. 
% 
\begin{defn}
Let $\cat{B}$ and $\cat{B'}$ be bicategories.  A \demph{lax functor}%
%
\index{functor!classical bicategories@of classical bicategories}%
%
\index{bicategory!classical!functor of}
% 
$F = (F, \phi): \cat{B} \go \cat{B'}$ consists of
%
\begin{itemize}
  \item a function $F_0: \cat{B}_0 \go \cat{B}'_0$, usually just written
  $F$
  \item for each $A, B \in \cat{B}_0$, a functor $F_{A, B}: \cat{B}(A, B)
  \go \cat{B'}(FA, FB)$, usually also written $F$
  \item for each composable pair $(f, g)$ of 1-cells in $\cat{B}$, a 2-cell
  $\phi_{g, f}: Fg \of Ff \go F(g\of f)$ 
  \item for each $A \in \cat{B}_0$, a 2-cell $\phi_A: 1_{FA} \go
  F1_A$ 
\end{itemize}
%
satisfying naturality and coherence axioms analogous to those of
Definition~\ref{defn:mon-ftr}.  \demph{Colax}, \demph{weak} and
\demph{strict functors} are also defined analogously. 
\end{defn}

\begin{example}	\lbl{eg:mon-cat-bicat-ftr}
Let $\cat{A}$ and $\cat{A'}$ be monoidal categories, and $\Sigma\cat{A}$%
% 
\glo{SigmaMCbi}%
\index{suspension!monoidal category@of monoidal category}
% 
and $\Sigma\cat{A'}$ the corresponding one-object bicategories.  Then lax
monoidal functors $\cat{A} \go \cat{A'}$ are exactly%
%
\index{bicategory!degenerate}
%
lax functors
$\Sigma\cat{A} \go \Sigma\cat{A'}$, and similarly for colax, weak and
strict functors. 
\end{example}

The evident composition of lax functors between bicategories is, perhaps
surprisingly, strictly associative and unital.  This means that we have a
category $\Bilax$%
% 
\glo{Bilax}
% 
of bicategories and lax functors, and subcategories $\Biwk$ and $\Bistr$.
The same goes for colax functors, but we concentrate on the lax (and above
all, the weak) case.

\begin{defn}
Let $(F, \phi), (G, \psi): \cat{B} \go \cat{B'}$ be lax functors between
bicategories.  A \demph{lax transformation}%
%
\index{transformation!classical bicategories@of classical bicategories}%
\index{bicategory!classical!transformation of}
$\sigma: F \go G$ consists of
%
\begin{itemize}
  \item for each $A \in \cat{B}_0$, a 1-cell $\sigma_A: FA \go GA$
  \item for each 1-cell $f: A \go B$ in $\cat{B}$, a 2-cell
  \[
  \begin{diagram}
  FA			&\rTo^{Ff}		&FB		\\
  \dTo<{\sigma_A}	&\nent \tcs{\sigma_f}	&\dTo>{\sigma_B}\\
  GA			&\rTo_{Gf}		&GB		\\
  \end{diagram}
  \]	
\end{itemize}
%
such that $\sigma_f$ is natural in $f$ and satisfies coherence axioms as in
Street~\cite[p.~568]{StrCS} or Leinster~\cite[1.2]{BB}.  A \demph{colax
transformation} is the same but with the direction of $\sigma_f$ reversed.
\demph{Weak} and \demph{strict transformations} are lax transformations in
which all the $\sigma_f$'s are, respectively, isomorphisms or identities.
\end{defn}

\begin{example}	\lbl{eg:mon-cat-bicat-transf}
Let $F, G: \cat{A} \go \cat{A'}$ be lax monoidal functors between monoidal
categories.  As in~\ref{eg:mon-cat-bicat-ftr}, these correspond%
%
\index{bicategory!degenerate}
%
to lax
functors $\Sigma F, \Sigma G: \Sigma\cat{A} \go \Sigma\cat{A'}$ between
bicategories.  A lax transformation $\Sigma F \go \Sigma G$ is an object
$S$ of $\cat{A'}$ together with a map
\[
\sigma_X: GX \otimes S \go S \otimes FX
\]
for each object $X$ of $\cat{A}$, satisfying coherence axioms.  (We are
writing $S = \sigma_\star$, where $\star$ is the unique object of
$\Sigma\cat{A}$.)

Transformations of one-object bicategories are, therefore, more general
than transformations of monoidal categories.  Monoidal transformations can
be identified with colax bicategorical transformations $\sigma$ for which
$\sigma_\star = I$.  Even putting aside the reversal of direction, we see
that what seems appropriate for bicategories in general seems inappropriate
in the one-object case.  We find out more in~\ref{cor:mon-eqv-bieqv}
and~\ref{eg:coh-mon-bi} below.
\end{example}

\begin{defn}
Let 
\[
\cat{B}
\ctwopar{F}{G}{\sigma}{\twid{\sigma}}
\cat{B'}
\]
be lax transformations between lax functors between bicategories.  A
\demph{modification}%
%
\index{modification!classical}\index{bicategory!classical!modification of}
%
$\Gamma: \sigma \go \twid{\sigma}$ consists of a 2-cell $\Gamma_A: \sigma_A
\go \twid{\sigma}_A$ for each $A \in \cat{B}_0$, such that $\Gamma_A$ is
natural in $A$ and satisfies a coherence axiom (Street~\cite[p.~569]{StrCS}
or Leinster~\cite[1.3]{BB}).
\end{defn}

Functors, transformations, and modifications can be composed in various
ways.  Let us consider just weak functors and transformations from now on.
It is straightforward to show that for any two bicategories $\cat{B}$ and
$\cat{B'}$, there is a functor bicategory $\ftrcat{\cat{B}}{\cat{B'}}$%
% 
\glo{ftrbicat}\index{functor!bicategory}
% 
whose objects are the weak functors from $\cat{B}$ to $\cat{B'}$, whose
1-cells are weak transformations, and whose 2-cells are modifications.
This is not usually a strict 2-category, because composing 1-cells in
$\ftrcat{\cat{B}}{\cat{B'}}$ involves composing 1-cells in $\cat{B'}$, but
it is a strict 2-category if $\cat{B'}$ is.  In particular, there is a
strict 2-category $\ftrcat{\cat{B}^\op}{\Cat}$ of `presheaves' on any
bicategory $\cat{B}$.

We might expect bicategories to form a weak 3-category, if we knew what one
was.  Gordon,%
%
\index{Gordon, Robert}
%
Power and Street gave an explicit definition of tricategory%
%
\index{tricategory}
%
(weak 3-category) in~\cite{GPS}, and showed that bicategories form a
tricategory.  It is worth noting, however, that there are two equally
sensible and symmetrically%
%
\index{handedness}
%
opposite ways of making a bicategory into a
tricategory, because there are two ways of horizontally composing weak
transformations, as is easily verified.  

\index{coherence!bicategories@for bicategories!classical|(}%
% 
There are coherence theorems for bicategories analogous to those for
monoidal categories.  `All diagrams commute' is handled in exactly the same
way.  For the other statement of coherence we need the right notion of
equivalence of bicategories.  

Let $(F, \phi): \cat{B} \go \cat{B'}$ be a weak functor between
bicategories.  We call $F$ a \demph{local%
%
\index{local equivalence}\index{equivalence!local}
%
equivalence} if for each $A, B
\in \cat{B}_0$, the functor $F_{A, B}: \cat{B}(A, B) \go \cat{B}'(FA, FB)$
is an equivalence of categories, and \demph{essentially%
%
\index{essentially surjective on objects}
%
surjective on
objects} if for each $A' \in \cat{B}'_0$, there exists $A \in \cat{B}_0$
such that $FA \eqv A'$.  We call $F$ a \demph{biequivalence}%
%
\index{biequivalence!classical}
%
if it
satisfies the equivalent conditions of the following result, analogous to
Proposition~\ref{propn:eqv-eqv}%
\bref{item:eqv-eqv-eqv}--\bref{item:eqv-eqv-ffe}:
% 
\begin{propn}	\lbl{propn:bieqv-eqv}
The following conditions on a weak functor $F: \cat{B} \go \cat{B'}$
between bicategories are equivalent:
%
\begin{enumerate}
\item	\lbl{item:bieqv-eqv-bieqv}
  there exists a weak functor $G: \cat{B'} \go \cat{B}$ such that
  $1_\cat{B} \eqv G \of F$ in $\ftrcat{\cat{B}}{\cat{B}}$ and 
  $F \of G \eqv 1_\cat{B'}$ in $\ftrcat{\cat{B'}}{\cat{B'}}$
\item	\lbl{item:bieqv-eqv-lee}
  $F$ is a local equivalence and essentially surjective on objects.
\end{enumerate}
\end{propn}
%
\begin{prooflike}{Sketch proof}
\bref{item:bieqv-eqv-bieqv} $\Rightarrow$ \bref{item:bieqv-eqv-lee} is
straightforward.  For the converse, choose for each $A' \in \cat{B'}$ an
object $GA' \in \cat{B}$ together with an \emph{adjoint} equivalence
between $FGA'$ and $A'$, which is possible by~\ref{propn:bicat-eqv-eqv}.
Then the remaining constructions and checks are straightforward, if
tedious.  \done
\end{prooflike}

Condition~\bref{item:bieqv-eqv-bieqv} of the Proposition says that there is
a system of functors, \emph{transformations and modifications} relating
$\cat{B}$ and $\cat{B'}$.  It therefore seems quite plausible that in the
one-object case, biequivalence is a looser relation than monoidal
equivalence.  Nevertheless,
%
\begin{cor}	\lbl{cor:mon-eqv-bieqv}%
%
\index{bicategory!degenerate}
%
Two monoidal categories are monoidally equivalent if and only if the
corresponding one-object bicategories are biequivalent. 
\end{cor}
%
\begin{proof}
The monoidal and bicategorical notions of weak functor are the
same~(\ref{eg:mon-cat-bicat-ftr}), so this follows from
conditions~\bref{item:bieqv-eqv-lee} of
Propositions~\ref{propn:mon-eqv-eqv} and~\ref{propn:bieqv-eqv}. 
\done
\end{proof}

\begin{thm}[Coherence for bicategories]	\lbl{thm:coh-bi-wk-str}
Every bicategory is biequivalent to some strict 2-category.
\end{thm}
%
\begin{prooflike}{Sketch proof}
Let $\cat{B}$ be a bicategory.  There is a weak functor
\[
\mathbf{y}: \cat{B} \go \ftrcat{\cat{B}^\op}{\Cat},
\]
the analogue of the Yoneda%
%
\index{Yoneda!embedding}
%
embedding for categories, sending an object $A$
of $\cat{B}$ to the weak functor
\[
\cat{B}(\dashbk, A): \cat{B}^\op \go \Cat
\]
and so on.  Just as the ordinary Yoneda embedding is full and faithful,
$\mathbf{y}$ is a local equivalence.  So if $\cat{B'}$ is the
sub-strict-2-category of $\ftrcat{\cat{B}^\op}{\Cat}$ consisting of all the
objects in the image of $\mathbf{y}$ and all the 1- and 2-cells between
them then $\mathbf{y}$ defines a biequivalence from $\cat{B}$ to
$\cat{B'}$.  
\done
\end{prooflike}

\begin{example}	\lbl{eg:coh-mon-bi}%
%
\index{bicategory!degenerate}
%
Corollary~\ref{cor:mon-eqv-bieqv} enables us to deduce coherence for
monoidal categories~(\ref{thm:coh-mon-wk-str}) from coherence for
bicategories.  In fact, the proof of~\ref{thm:coh-mon-wk-str} is exactly
the proof of~\ref{thm:coh-bi-wk-str} in the one-object case.  Let $\cat{B}$
be a bicategory with single object $\star$ and let $\cat{A}$ be the
corresponding monoidal category.  Then $\cat{B'}$ is the sub-2-category of
$\ftrcat{\cat{B}^\op}{\Cat}$ whose single object is the representable
functor $\cat{B}(\dashbk, \star)$, and a weak transformation from
$\cat{B}(\dashbk, \star)$ to itself is an object $(E, \delta)$ of
$\cat{A'}$ as in the proof of~\ref{thm:coh-mon-wk-str}, and so on.
\end{example}%
%
\index{coherence!bicategories@for bicategories!classical|)}%
%




\begin{notes}

Almost everything in this chapter is very well-known (in the sense of the
phrase to which mathematicians are accustomed).  I have used Mac
Lane~\cite{MacCWM} as my main reference for ordinary category theory, with
Borceux~\cite{Borx1, Borx2} as backup.  The standard text on enriched
category theory is Kelly~\cite{KelBCE}.  For higher category theory,
Street~\cite{StrCS} provides a useful survey and reference list.  Strict
$n$-categories were introduced by Ehresmann~\cite{Ehr}.%
%
\index{Ehresmann, Charles}
%

I thank Bill Fulton and Ross Street for an enlightening exchange on
equivalence of bicategories, and Nathalie Wahl for a useful conversation on
how not%
%
\index{coherence!how not to prove}
%
to prove coherence.

\end{notes}

