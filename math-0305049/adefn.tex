
\chapter{A Definition of Weak $n$-Category}
\lbl{ch:a-defn}

\chapterquote{%
Vico took it for granted that the first language of humanity was in the
form of hieroglyphics; that is, of metaphors and animated figures [\ldots]
He had intimated war with just `five real words': a frog, a mouse, a bird,
a ploughshare, and a bow}{%
Eco~\cite{Eco}}


\noindent
Algebraic structures are often defined in a way that suggests conflict:
generators \vs.\ relations, operations \vs.\ equations, composition%
%
\index{composition!coherence@\vs.\ coherence}%
%
\index{coherence!composition@\vs.\ composition}
%
%
\vs.\
coherence.  For example, in the definition of bicategory one equips a
2-globular set first with various composition operations, then with
coherence isomorphisms to ensure that some of the derived compositions are,
in fact, essentially the same.  One imagines the two sides pulling against
each other: more operations make the structure bigger and wilder, more
equations or coherence cells make it smaller and more tame.

With this picture in mind, the most obvious way to go about defining weak
$n$-category is to set up a family of higher-dimensional composition
operations subject to a family of higher-dimensional coherence constraints.
This is the strategy in Batanin's and Penon's proposed definitions, both of
which we discuss in the Chapter~\ref{ch:other-defns}.  But it is not our
strategy in this chapter.

In the definition proposed here, no distinction is made between composition
and coherence.  They are seen as two aspects of a single idea,
`contraction', not as opposing forces.  This unified approach is in many
ways more simple and graceful: one idea instead of two.

Contractions%
%
\index{contraction!map of globular sets@on map of globular sets}
%
are explained in~\ref{sec:contr}.  A map of globular sets may have the
property of being contractible, which viewed topologically means something
like being injective on homotopy; if so, it admits at least one
contraction, which is something like a homotopy%
%
\index{homotopy!lifting}
%
lifting.  These definitions
lead to definitions of contractibility of, and contraction on, a globular
operad.  By considering some low-dimensional situations, we see how
contraction alone generates a natural theory of weak $\omega$-categories. 

In~\ref{sec:defn-L} weak $\omega$-categories are defined formally as
algebras for the initial globular operad equipped with a contraction.  We
look at some examples, including the fundamental $\omega$-groupoid
of a topological space.

The finite-dimensional case, weak $n$-categories, is a shade less easy than
the infinite-dimensional case because we have to take care in the top
dimension.  In~\ref{sec:wk-n} we define weak $n$-categories and look at
various ways of constructing weak $n_1$-categories from weak
$n_2$-categories, for different (possibly infinite) values of $n_1$ and
$n_2$.  For instance, if $a$ and $b$ are 0-cells of an $n$-category $X$
then there is a `hom-$(n-1)$-category' $X(a,b)$.

The first test for a proposed definition of $n$-category is that it does
something sensible when $n\leq 2$.  We show in~\ref{sec:wk-2} that ours
passes: weak $0$-categories are sets, weak $1$-categories are categories,
and weak $2$-categories are unbiased bicategories.



\section{Contractions}
\lbl{sec:contr}

Given the language of globular operads, we need one more concept in order
to express our definition of weak $\omega$-category: contractions.  First
we define contraction on a map of globular sets, then we define contraction
on a globular operad, then we see what this has to do with the theory of
weak $\omega$-categories.

\begin{defn}
Let $X$ be a globular set and let $m\in\nat$.  Two cells $\alpha^-,
\alpha^+ \in X(m)$ are \demph{parallel}%
%
\index{parallel}
%
if $m=0$ or if $m\geq 1$,
$s(\alpha^-) = s(\alpha^+)$, and $t(\alpha^-) = t(\alpha^+)$.
\end{defn}

\begin{defn}	\lbl{defn:omega-map-contraction}
Let $q: X \go Y$ be a map of globular sets.  For $m\geq 1$ and $\phi\in
Y(m)$, write (Fig.~\ref{fig:contr-on-map})
%
\begin{eqnarray*}
\mr{Par}_q(\phi)	&
=	&
\{ (\theta^-, \theta^+) \in X(m-1) \times X(m-1) \such 
\theta^- \textrm{ and } \theta^+ \textrm{ are parallel,} 
\\ & &
q(\theta^-) = s(\phi), \textrm{ } q(\theta^+)= t(\phi)
\}.%
% 
\glo{Par}
% 
\end{eqnarray*}
%
A \demph{contraction}%
%
\index{contraction!map of globular sets@on map of globular sets}
%
$\kappa$ on $q$ is a family of functions
\[
\left(
\mr{Par}_q(\phi) \goby{\kappa_\phi} X(m) 
\right)_{m\geq 1, \phi\in Y(m)}
\]
such that for all $m\geq 1$, $\phi\in Y(m)$, and $(\theta^-, \theta^+) \in
\mr{Par}_q(\phi)$, 
\[
s(\kappa_\phi(\theta^-, \theta^+)) = \theta^-,
\ \ 
t(\kappa_\phi(\theta^-, \theta^+)) = \theta^+,
\ \ 
q(\kappa_\phi(\theta^-, \theta^+)) = \phi.
\]
\end{defn}
% 
\begin{figure}
\[
\gfst{\xi}\gtwodotty{\theta^-}{\theta^+}{}\glst{\xi'}
\mbox{\hspace{2.5em}}
\stackrel{q}{\goesto}
\mbox{\hspace{2.5em}}
\gfst{q(\xi)}\gtwo{q(\theta^-)}{q(\theta^+)}{\phi}\glst{q(\xi')}
\]
\caption{Effect of a contraction $\kappa$, shown for $m=2$.  The dotted
  arrow is $\kappa_\phi(\theta^-, \theta^+)$}
\label{fig:contr-on-map}
\end{figure}
% 
So a map admits a contraction when it is `injective%
%
\index{homotopy!injective on}
%
on homotopy' in some
oriented sense.

\begin{defn}	\lbl{defn:omega-opd-contraction}
A \demph{contraction}%
%
\index{contraction!collection@on collection}
%
on a collection $(P \goby{d} T1)$ is a contraction on
the map $d$.  A \demph{contraction}%
%
\index{contraction!globular operad@on globular operad}
%
on a globular operad is a contraction
on its underlying collection.  A map, collection or operad is
\demph{contractible}%
%
\index{contractible}
%
if it admits a contraction.
\end{defn}
% 
Explicitly, if $\pi$ is a pasting diagram then $\mr{Par}_d(\pi)$ is the set
of parallel pairs $(\theta^-, \theta^+)$ of elements of $P(\bdry\pi)$.  A
contraction assigns to each pasting diagram $\pi$ and parallel $\theta^-,
\theta^+ \in P(\bdry\pi)$ an element of $P(\pi)$ with source $\theta^-$ and
target $\theta^+$.  We usually write $\mr{Par}_P(\pi)$%
% 
\glo{Parcoll}
% 
instead of $\mr{Par}_d(\pi)$; then a contraction $\kappa$
on $P$ consists of a function 
\[
\kappa_\pi: \mr{Par}_P(\pi) \go P(\pi)
\]
for each pasting diagram $\pi$, satisfying the source and target axioms.

To understand what contractions have to do with weak $\omega$-categories,
let us forget these definitions for a while and ask: what should the operad
for weak $\omega$-categories be?%
%
\index{omega-category@$\omega$-category!weak!theories of}
%
 In other words, how should we pick a
globular operad $L$ in such a way that $L$-algebras might reasonably be
called weak $\omega$-categories?  The elements of $L(\pi)$ are all the
possible ways of composing a labelled diagram of shape $\pi$ in an
arbitrary weak $\omega$-category, so the question is: what such ways should
there be?

There are many sensible answers.  Even for weak 2-categories there is an
infinite family of definitions, all equivalent (Chapter~\ref{ch:monoidal}).
We saw, for instance, that we could choose to start with $100$ different
specified ways of composing diagrams of shape
\[
\gfstsu\gonesu\gzersu%\gone{}\gblws{}
\gonesu\glstsu,
\]
just as long as we made them all coherently isomorphic.  This particular
choice seems bizarre, and in choosing our theory $L$ we try to make it in
some sense canonical.

So let us decide what operations to put into the theory of weak
$\omega$-categories, starting at the bottom dimension and working our way
up.  

There should not be any operations in dimension $0$ except for the
identity: the only way of obtaining a $0$-cell in a weak $\omega$-category
is to start with that same $0$-cell and do nothing.  So if $\blob$ denotes
the unique $0$-pasting diagram then we want $L(\blob) = 1$.

In dimension $1$, let us be unbiased%
%
\index{omega-category@$\omega$-category!unbiased|(}
%
and specify for each $k\in\nat$ a
single way of composing a string
\[
\gfstsu\gonesu
\diagspace \cdots \diagspace 
\gonesu\glstsu
\]
of $k$ 1-cells.  These specified compositions can be built up to
make more complex ones: if we denote the specified $k$-fold composition of
1-cells in a weak $\omega$-category by
\[
\gfsts{a_0}\gones{f_1}
\diagspace
\cdots
\diagspace
\gones{f_k}\glsts{a_k}
\diagspace
\goesto
\diagspace
\gfsts{a_0}
\gones{(f_k \of\cdots\of f_1)}
\glsts{a_k}
\]
(as we did for unbiased bicategories) then an example of a built-up
operation is
\[
\gfsts{a_0}\gones{f_1}
\gblws{a_1}\gones{f_2}
\gblws{a_2}\gones{f_3}
\gblws{a_3}\gones{f_4}
\glsts{a_4}
\diagspace\goesto\diagspace
\gfsts{a_0}\gones{((f_4\of f_3\of f_2)\of f_1)}\glsts{a_4}.
\]
There should be no other ways of obtaining a 1-cell in a weak
$\omega$-category.  So if $\chi_k$%
% 
\glo{chi}
% 
is the 1-pasting diagram made up of $k$
1-cells then we want $L(\chi_k)$ to be the set $\tr(k)$ of $k$-leafed
trees~(\ref{eg:opd-of-trees}).

What operations in a weak $\omega$-category result in a 2-cell? 
First,
ordinary composition makes a 2-cell from diagrams of shapes such as
\[
\rho = \gfstsu\gfoursu\glstsu,
\diagspace
\sigma = \gfstsu\gtwosu\gzersu\gtwosu\glstsu.
\]
Choosing to be unbiased again, we specify an operation $\theta\in L(\rho)$
such that $s(\theta) = t(\theta) = \id$.  When acting in a weak
$\omega$-category, $\theta$ takes as input a diagram of 2-cells $\alpha_i$
and produces as output a single 2-cell as shown:
% $\phi$ as shown:
%
\begin{equation}	\label{diag:2-dim-high}
\gfsts{a}
\gfours{f}{g}{h}{k}{\alpha}{\beta}{\gamma}
\glsts{b}
\diagspace \goesto \diagspace
\gfsts{a}
\gtwos{f}{k}{} %{\phi}
\glsts{b}.
\end{equation}
%
Similarly, we specify one operation of arity $\sigma$ whose source and
target are both the specified element of $L(\chi_2)$ (which is ordinary
binary composition of 1-cells):
%
\begin{equation}	\label{diag:2-dim-wide}
\gfsts{a}
\gtwos{f}{g}{\alpha}
\gblws{b}
\gtwos{f'}{g'}{\alpha'}
\glsts{c}
\diagspace \goesto \diagspace
\gfsts{a}
\gtwos{(f' \of f)}{(g' \of g)}{} %{\phi}
\glsts{c}.
\end{equation}
%
Second, there are coherence 2-cells.  For instance, a string of three
1-cells gives rise to an associativity 2-cell:
%
\begin{equation}	\label{diag:2-dim-ass}
\gfstsu{}\gones{f_1}\gzersu\gones{f_2}\gzersu\gones{f_3}\gzersu
\diagspace
\goesto
\diagspace
\gfstsu
\gtwos{((f_3 \of f_2) \of f_1)}{(f_3 \of (f_2 \of f_1))}{}
\glstsu
\end{equation}
%
and similarly in more complex cases: 
% a string $(f_1, f_2, f_3, f_4)$ of 1-cells gives a 2-cell
%
\begin{equation}	\label{diag:2-dim-coh}
\gfstsu{}\gones{f_1}\gzersu\gones{f_2}\gzersu%
\gones{f_3}\gzersu\gones{f_4}\gzersu
\diagspace
\goesto
\diagspace
\gfstsu
\gtwos{(1\of (f_4 \of f_3 \of f_2) \of 1 \of f_1)}%
{(f_4 \of (f_3 \of (f_2 \of 1 \of f_1)))}%
{}
\glstsu.
\end{equation}
%
Being unbiased once more, we \emph{specify} operations giving 2-cells in
each of these two ways; we do not, for instance, insist that the coherence
cell on the right-hand side of~\bref{diag:2-dim-coh} should be equal to
some composite of associativity and identity coherence cells.

Here comes the crucial point.  It looks as if these two kinds of operations
resulting in 2-cells, composition%
%
\index{composition!coherence@\vs.\ coherence}%
%
\index{coherence!composition@\vs.\ composition}
%
and coherence, are quite
different---complementary, even.  But they can actually be regarded as two
instances of the same thought, and that makes matters much simpler.

First recall from~\ref{sec:free-strict} that any 1-dimensional picture of a
pasting diagram can also be taken to represent a (degenerate) element of
$\pd(2)$.  In particular, the left-hand sides of~\bref{diag:2-dim-ass}
and~\bref{diag:2-dim-coh} can be regarded as degenerate elements of
$\pd(2)$: they are $1_{\chi_3}$ and $1_{\chi_4}$ respectively
(see~\bref{eq:typical-id}).  Thus, each
of~\bref{diag:2-dim-high}--\bref{diag:2-dim-coh} portrays an element of
$L(\pi)$ for some $\pi\in\pd(2)$.

Now note that four times over, we have taken a 2-pasting diagram $\pi$ and
elements $\theta^-, \theta^+ \in L(\bdry\pi)$ and decreed that $L(\pi)$
should contain a specified element $\theta$ with source $\theta^-$ and
target $\theta^+$.  In the first two cases this is obvious; in the
third~(\ref{diag:2-dim-ass}) we took $\pi = 1_{\chi_3}$ (hence $\bdry\pi =
\chi_3$) and the evident two elements $\theta^-, \theta^+ \in L(\chi_3)$; the
last is similar.

Which pairs $(\theta^-, \theta^+)$ of 1-dimensional operations should we
use to generate the 2-dimensional operations?  The simplest possible
answer, and the properly unbiased%
%
\index{omega-category@$\omega$-category!unbiased|)}
%
one, is `all of them'.  So the principle
is:
%
\begin{quote}
  Let $\pi \in \pd(2)$ and $\theta^-, \theta^+ \in L(\bdry\pi)$.  Then
  there is a specified element $\theta\in L(\pi)$ satisfying $s(\theta) =
  \theta^-$ and $t(\theta) = \theta^+$.
\end{quote}
%
`Specified' means that we take these operations $\theta$ as primitive and
generate the new 2-dimensional operations in $L$ freely using its operadic
structure, just as one dimension down we took the $k$-fold composition
operations ($k\in\nat$) as primitive and generated from them derived
1-dimensional operations, indexed by trees.  In dimension 2 the
combinatorial situation is more difficult, and I will not attempt an
explicit description of $L(\pi)$ for $\pi\in\pd(2)$.  A categorical
description takes its place; in the next section, $L$ is defined as the
universal operad containing elements specified in this way.

Because our principle takes in both composition%
%
\index{composition!coherence@\vs.\ coherence}%
%
\index{coherence!composition@\vs.\ composition}
% 
and coherence at once, it
produces operations traditionally regarded as a hybrid of the two.  For
instance, there is a specified operation of the form
\[
\gfsts{a}
\gtwos{f}{g}{\alpha}
\gblws{b}
\gtwos{f'}{g'}{\alpha'}
\glsts{c}
\diagspace \goesto \diagspace
\gfsts{a}
\gtwos{((f' \of 1) \of f)}{(g' \of g)}{} %{\beta}
\glsts{c},
\]
where traditionally operations such as this would be built up from
horizontal composition~(\ref{diag:2-dim-wide}) and coherence cells.  The
spirit of our definition is that composition and coherence are \emph{not}
separate entities: they are two sides of the same coin.

What we have said for dimension 2 applies equally in all dimensions, with
just one small refinement: there can only be a $\theta$ satisfying
$s(\theta) = \theta^-$ and $t(\theta) = \theta^+$ if $\theta^-$ and
$\theta^+$ are parallel.  (This is trivial in dimension 2 because $L(0) =
1$.)  So the general principle is:
%
\begin{quote}
  Let $n\in\nat$, let $\pi \in \pd(n)$, and let $\theta^-, \theta^+$ be
  parallel elements of $L(\bdry\pi)$.  Then there is a specified element
  $\theta\in L(\pi)$ satisfying $s(\theta) = \theta^-$ and $t(\theta) =
  \theta^+$.
\end{quote}
%
In other words: 
%
\begin{quote}
  $L$ is equipped with a contraction.%
%
\index{contraction!globular operad@on globular operad}
%
\end{quote}

Everything that we did in choosing $L$ is encapsulated in this statement.
We started from the identity element of $L(0)$, part of the operad
structure of $L$.  Then we chose to specify $k$-fold compositions of
1-cells, which amounted to applying the contraction with $\pi = \chi_k$.
Then the operad structure of $L$ gave derived 1-dimensional operations such
as $(f_1, f_2, f_3) \goesto ((f_3 \of f_2) \of f_1)$.  Then we chose to
specify one operation of arity $\pi$ with any given source and target, for
each 2-pasting diagram $\pi$.  Then the operad structure of $L$ gave
derived 2-dimensional operations.  All in all, we chose $L$ to be the
universal operad equipped with a contraction.



\section{Weak $\omega$-categories}
\lbl{sec:defn-L}

We have decided that the operad for weak $\omega$-categories ought to come
with a specified contraction, and that it ought to be `universal',
`minimal', or `freely generated' as such.  Precisely, it ought to be
initial in the category of operads equipped with a contraction.  
%
\begin{defn}	\lbl{defn:OC}
The category $\fcat{OC}$%
% 
\glo{OC}
% 
of (globular) \demph{operads-with-contraction}%
%
\index{globular operad!contraction@with contraction}
%
has
as objects all pairs $(P,\kappa)$ where $P$ is a globular operad and
$\kappa$ a contraction on $P$, and as maps $(P, \kappa) \go (P', \kappa')$
all operad maps $f: P \go P'$ preserving contractions:
\[
f(\kappa_\pi(\theta^-, \theta^+)) 
=
\kappa'_\pi (f\theta^-, f\theta^+) 
\]
whenever $m\geq 1$, $\pi\in\pd(m)$, and $(\theta^-, \theta^+) \in
\mr{Par}_P(\pi)$.  
\end{defn}

\begin{propn}	\lbl{propn:OC-initial}
The category $\fcat{OC}$ has an initial object.
\end{propn}
%
\begin{proof}
Appendix~\ref{app:initial}. 
\done
\end{proof}
%
We write $(L, \lambda)$%
% 
\glo{initL}
% 
for the initial object of $\fcat{OC}$.  In the
previous section we described $L$ explicitly in low dimensions and saw
informally how to construct it in higher dimensions: if $L$ is known up to
and including dimension $n-1$, then $L(n)$ is obtained by first closing
under contraction then closing under $n$-dimensional operadic composition
and identities.

\begin{defn}	\lbl{defn:weak-omega-cat}%
%
\index{omega-category@$\omega$-category!weak}
%
A \demph{weak $\omega$-category} is an $L$-algebra.
\end{defn}
%
We write $\wkcat{\omega}$%
% 
\glo{wkomegaCat}
% 
for $\Alg(L)$.  Since the algebras construction
is functorial (p.~\pageref{p:Alg-functorial}), this category is determined
uniquely up to isomorphism.  

Observe that the maps in $\wkcat{\omega}$ preserve the operations from
$L$---that is, the weak $\omega$-category structure---\emph{strictly}.%
%
\index{omega-functor@$\omega$-functor, strict}%
%
\index{omega-category@$\omega$-category!weak!map of}
%
 In
this text we do not go as far as a definition of \emph{weak}
$\omega$-functor, nor do we reach a definition of (weak) equivalence of
weak $\omega$-categories.  These are serious omissions, and the reader may
feel cheated that we are stopping when we have barely begun, but that is
the state of the art.

Definition~\ref{defn:weak-omega-cat} is just one of many proposed
definitions of weak $\omega$-category.  In my~\cite{SDN}, it is
`Definition~\textbf{L1}'.  Its place among other definitions is discussed
in Chapter~\ref{ch:other-defns}.

 
\begin{example}	\lbl{eg:wk-omega-cat-contr-opd}
If $P$ is a contractible operad then any contraction $\kappa$ on $P$ gives
rise to a unique map $f: (L, \lambda) \go (P, \kappa)$ in $\fcat{OC}$,
whose underlying map $f: L \go P$ of operads induces a functor
%
\begin{equation}	\label{eq:P-omega-cat}
\Alg(P) \go \wkcat{\omega}.
\end{equation}
%
In other words, any algebra for an operad-with-contraction is canonically a
weak $\omega$-category.  All the examples below of weak $\omega$-categories
are constructed in this way.  

Functor~\bref{eq:P-omega-cat} is always faithful (being the identity on
underlying globular sets) and is full if for for each pasting diagram
$\pi$, the function $f_\pi: L(\pi) \go P(\pi)$ is surjective (exercise).
\end{example}

\begin{example}	\lbl{eg:wk-omega-cat-str}
The algebras for the terminal%
%
\index{globular operad!terminal}
% 
globular operad $1$ are the strict%
%
\index{omega-category@$\omega$-category!strict!operad for}
%
$\omega$-categories (by~\ref{eg:alg-terminal}), so the unique map $L \go 1$
induces a functor
%
\begin{equation}	\label{eq:str-to-wk}
\strcat{\omega} \go \wkcat{\omega}
\end{equation}
%
---a strict $\omega$-category is a special weak $\omega$-category.

The operad $1$ admits a unique contraction, and thus becomes the
\emph{terminal} object of $\fcat{OC}$.  Contractibility of $L$ implies that
$L(\pi)$ is nonempty for each pasting diagram $\pi$, so by the observations
of the previous example, functor~\bref{eq:str-to-wk} is full and faithful.
We would expect this: if $C$ and $D$ are strict $\omega$-categories then
there ought to be a single notion of `strict map $C \go D$', independent of
whether $C$ and $D$ are regarded as strict or as weak $\omega$-categories.
\end{example}

\begin{example}	\lbl{eg:wk-omega-cat-indisc}
A directed graph is \demph{indiscrete}%
%
\index{graph!directed!indiscrete}
%
if for all objects $x$ and $y$,
there is exactly one edge from $x$ to $y$.  Such a graph has a unique
category%
%
\index{category!indiscrete}
%
structure.  The $\omega$-categorical analogue of this observation
is that any \demph{contractible}%
%
\index{contractible!globular set}%
%
\index{globular set!contractible}
%
%
globular set $X$---one for which the
unique map $X \go 1$ is contractible---has a weak $\omega$-category
structure.  To see the analogy, note that contractibility of a globular set
says that any two parallel $n$-cells $x$ and $y$ have at least one
$(n+1)$-cell $f: x \go y$ between them, and indiscreteness of a graph says
that any two objects $x$ and $y$ have a map $f: x \go y$ between them and
any two parallel arrows have an equality between them.

A contractible globular set $X$ acquires the structure of a weak
$\omega$-category as follows.  Recall from~\ref{sec:endos} that there is an
endomorphism%
%
\index{globular operad!endomorphism}%
%
\index{endomorphism!globular operad}%
%
\index{globular operad!algebra for}%
%
\index{algebra!globular operad@for globular operad}
%
operad $\END(X)$, and that if $P$ is any operad then a
$P$-algebra structure on $X$ is just an operad map $P \go \END(X)$.  In
particular, $X$ is canonically an $\END(X)$-algebra.  So
by~\ref{eg:wk-omega-cat-contr-opd}, it is enough to prove that
contractibility of the globular set $X$ implies contractibility of the
operad $\END(X)$.  

It follows from the definition of $\END$ that for any globular set $X$ and
pasting diagram $\pi\in\pd(n)$, an element of $(\END(X))(\pi)$ is a
sequence of functions
\[
(f_n, f_{n-1}^-, f_{n-1}^+, f_{n-2}^-, f_{n-2}^+, \ldots, f_0^-, f_0^+)
\]
making the diagram
%
\begin{equation}	\label{eq:endo-element}
% \[
\begin{diagram}[scriptlabels,height=2.5em]
(TX)(\pi)	&\pile{\rTo^s\\ \rTo_t}	&
(TX)(\bdry\pi)	&\pile{\rTo^s\\ \rTo_t}	&
(TX)(\bdry^2\pi)&\pile{\rTo^s\\ \rTo_t} &
\ \ \cdots \ \ 	&\pile{\rTo^s\\ \rTo_t} &
(TX)(\bdry^n\pi)	\\
\dTo~{f_n}				&	&
\dTo<{f_{n-1}^-} \dTo>{f_{n-1}^+}	&	&
\dTo<{f_{n-2}^-} \dTo>{f_{n-2}^+}	&	&
					&	&
\dTo<{f_0^-} \dTo>{f_0^+}		\\
X(n)		&\pile{\rTo^s\\ \rTo_t}	&
X(n-1)		&\pile{\rTo^s\\ \rTo_t}	&
X(n-2)		&\pile{\rTo^s\\ \rTo_t} &
\ \ \cdots \ \ 	&\pile{\rTo^s\\ \rTo_t} &
X(0)		\\
\end{diagram}
% \]
\end{equation}
%
commute serially---that%
%
\index{serially commutative}
%
is,
\[
s \of f_i^- = f_{i-1}^- \of s, 
\diagspace
t \of f_i^+ = f_{i-1}^+ \of t
\]
for all $i \in \{ 1, \ldots, n \}$, interpreting both $f_n^-$ and $f_n^+$
as $f_n$.  (Compare Batanin~\cite[Prop.~7.2]{BatMGC}.)%
%
\index{Batanin, Michael!globular operads@on globular operads}
%
 Contractibility of
$\END(X)$ says that given all of such a serially commutative diagram except
for $f_n$, there exists a function $f_n$ completing it.  This holds if $X$
is contractible.

Different contractions on $X$ induce different contractions on $\END(X)$,
hence different weak $\omega$-category structures on $X$.  They should all
be `equivalent', and if $X$ is nonempty then the weak $\omega$-category $X$
should be equivalent to $1$, but we do not attempt to make this precise.
\end{example}

Our final example of a weak $\omega$-category is one of the principal
motivations for the subject. 
% 
\begin{example}	\lbl{eg:wk-omega-cat-Pi}
Any topological space $S$ gives rise to a weak $\omega$-category
$\Pi_\omega S$,%
% 
\glo{Piomega}
% 
its \demph{fundamental $\omega$-groupoid}.%
%
\index{fundamental!omega-groupoid@$\omega$-groupoid}
%
  Indeed, there
is a product-preserving functor
\[
\Pi_\omega: \Top \go \wkcat{\omega}.
\]

To show this, we first establish the relationship between spaces and
globular sets.  As in~\ref{eg:n-glob-set-Pi}, the Euclidean disks $D^n$
define a functor $\scat{G} \go \Top$, and by the mechanism described on
p.~\pageref{eq:induced-adjn}, this induces a pair of adjoint functors
\[
\begin{diagram}
\Top	&
\pile{\rTo^{\scriptstyle \Pi_\omega}_\top\\ 
	\lTo_{\scriptstyle | \cdot |}}	&
\ftrcat{\scat{G}^\op}{\Set}.
\end{diagram}
\]
The right adjoint is given on a space $S$ by
\[
(\Pi_\omega S)(n) = \Top(D^n, S),
\]
so $\Pi_\omega S$ is just like the singular simplicial set of $S$, but with
disks in place of simplices.  The left adjoint is \demph{geometric%
%
\index{geometric realization!globular set@of globular set}
%
realization}, given on a globular set $X$ by the coend%
%
\index{coend}
% 
formula
\[
|X| = \int^{n\in \scat{G}} X(n) \times D^n.
\]
For example, if $\pi$ is an $n$-pasting diagram then $|\rep{\pi}|$ is the
space resembling the usual picture of $\pi$; it is a finite CW-complex made
up of cells of dimension at most $n$, and is contractible, as can be proved
by induction.

To give the globular set $\Pi_\omega S$ the structure of a weak
$\omega$-category we define a contractible operad $P$ (independently of
$S$) and show that $\Pi_\omega S$ is naturally a $P$-algebra.
By~\ref{eg:wk-omega-cat-contr-opd}, this suffices.

Our operad $P$ is analogous to the universal%
%
\index{operad!universal for loop spaces}
%
operad for iterated loop
spaces~(\ref{eg:opd-univ-loop}).  If $\pi$ is an $n$-pasting diagram then
an element of $P(\pi)$ is a map from $D^n$ to $|\rep{\pi}|$ respecting the
boundaries; for instance, if $\pi$ is the $1$-pasting diagram $\chi_k$
consisting of $k$ arrows in a row then an element of $P(\pi)$ is an
endpoint-preserving map $[0,1] \go [0,k]$.  In general, an element of
$P(\pi)$ is a sequence of maps
\[
\theta =
(\theta_n, 
\theta_{n-1}^-, \theta_{n-1}^+, 
\theta_{n-2}^-, \theta_{n-2}^+, 
\ldots, 
\theta_0^-, \theta_0^+)
\]
making the diagram
\[
\begin{diagram}[scriptlabels,height=2.5em]
|\rep{\pi}|		&\pile{\lTo\\ \lTo}	&
|\rep{\bdry\pi}|	&\pile{\lTo\\ \lTo}	&
|\rep{\bdry^2\pi}|	&\pile{\lTo\\ \lTo} &
\ \ \cdots \ \ 	&\pile{\lTo\\ \lTo} &
|\rep{\bdry^n\pi}|=1	\\
\uTo~{\theta_n}					&	&
\uTo<{\theta_{n-1}^-} \uTo>{\theta_{n-1}^+}	&	&
\uTo<{\theta_{n-2}^-} \uTo>{\theta_{n-2}^+}	&	&
						&	&
\uTo<{\theta_0^-} \uTo>{\theta_0^+}		\\
D^n		&\pile{\lTo\\ \lTo}	&
D^{n-1}		&\pile{\lTo\\ \lTo}	&
D^{n-2}		&\pile{\lTo\\ \lTo} &
\ \ \cdots \ \ 	&\pile{\lTo\\ \lTo} &
D^0 = 1		\\
\end{diagram}
\]
commute serially (as in~\bref{eq:endo-element}).  The maps along the
top are induced by the source and target inclusions $\rep{\bdry^{i+1}\pi}
\parpairu \rep{\bdry^i\pi}$ described on p.~\pageref{p:rep-source-target};
they are all injective, so $\theta_n$ determines the whole of $\theta$.
The obvious restriction maps $P(\pi) \parpairu P(\bdry\pi)$ make $P$ into a
collection.

We have to show that the collection $P$ is contractible, has the structure
of an operad, and acts on the globular set $\Pi_\omega S$ for any space
$S$.  Contractibility amounts to the condition that if $\pi$ is an
$n$-pasting diagram then any continuous map from the unit $(n-1)$-sphere
into $|\rep{\pi}|$ extends to the whole unit $n$-ball $D^n$, which holds
because $|\rep{\pi}|$ is contractible.  An action of $P$ on $\Pi_\omega S$
consists of a function
\[
% P(\pi) \times 
\ovln{\theta}:
\ftrcat{\scat{G}^\op}{\Set}(\rep{\pi}, \Pi_\omega S)
\go
(\Pi_\omega S)(n)
\]
for each $n\in\nat$, $\pi\in\pd(n)$, and $\theta \in P(\pi)$, satisfying
axioms involving the operad structure of $P$ (yet to be defined).  By the
adjunction above, such a function can equally be written as
\[
\ovln{\theta}:
\Top(|\rep{\pi}|, S)
\go 
\Top(D^n, S),
\]
and we take $\ovln{\theta}$ to be composition with $\theta_n$.

All that remains is to endow the collection $P$ with the structure of an
operad and check some axioms.  As suggested by the similarity between the
diagrams in this example and the last, $P$ can be constructed as an
endomorphism operad.  By the Yoneda Lemma, an element of $P(\pi)$ is a
sequence of natural transformations
\[
(\alpha_n, 
\alpha_{n-1}^-, \alpha_{n-1}^+, 
\ldots, 
\alpha_0^-, \alpha_0^+)
\]
making the diagram
\[
\begin{diagram}[scriptlabels,height=2.5em]
\Top(|\rep{\pi}|, \dashbk)		&\pile{\rTo\\ \rTo}	&
\Top(|\rep{\bdry\pi}|, \dashbk)	&\pile{\rTo\\ \rTo}	&
\ \ \cdots \ \ 	&\pile{\rTo\\ \rTo} &
\Top(|\rep{\bdry^n\pi}|, \dashbk)	\\
\dTo~{\alpha_n}					&	&
\dTo<{\alpha_{n-1}^-} \dTo>{\alpha_{n-1}^+}	&	&
						&	&
\dTo<{\alpha_0^-} \dTo>{\alpha_0^+}			\\
\Top(D^n, \dashbk)			&\pile{\rTo\\ \rTo}	&
\Top(D^{n-1}, \dashbk)		&\pile{\rTo\\ \rTo}	&
\ \ \cdots \ \ 	&\pile{\rTo\\ \rTo} &
\Top(D^0, \dashbk)			\\
\end{diagram}
\]
commute serially.  But for any $m\in\nat$ and $\rho\in\pd(m)$ there are
isomorphisms
\[
\Top(|\rep{\rho}|, S) 
\iso
\ftrcat{\scat{G}^\op}{\Set}(\rep{\rho}, \Pi_\omega S)
\iso 
(T(\Pi_\omega S))(\rho)
\]
and
\[
\Top(D^m, S)
\iso
(\Pi_\omega S)(m),
\] 
both natural in $S\in\Top$, so an element of $P(\pi)$ is a sequence of
families of functions
% 
\begin{equation}	\label{eq:Pi-data}
\left(
(f_{S,n})_{S\in\Top}, 
(f_{S,n-1}^-)_{S\in\Top}, (f_{S,n-1}^+)_{S\in\Top}, 
\ldots, 
(f_{S,0}^-)_{S\in\Top}, (f_{S,0}^+)_{S\in\Top}
\right)
\end{equation}
% 
natural in $S$ and making the diagram 
\[
\begin{diagram}[scriptlabels,height=2.5em]
(T(\Pi_\omega S))(\pi)	&\pile{\rTo^s\\ \rTo_t}	&
(T(\Pi_\omega S))(\bdry\pi)	&\pile{\rTo^s\\ \rTo_t}	&
\ \ \cdots \ \ 	&\pile{\rTo^s\\ \rTo_t} &
(T(\Pi_\omega S))(\bdry^n\pi)		\\
\dTo~{f_{S,n}}				&	&
\dTo<{f_{S,n-1}^-} \dTo>{f_{S,n-1}^+}	&	&
					&	&
\dTo<{f_{S,0}^-} \dTo>{f_{S,0}^+}	\\
(\Pi_\omega S)(n)		&\pile{\rTo^s\\ \rTo_t}	&
(\Pi_\omega S)(n-1)		&\pile{\rTo^s\\ \rTo_t}	&
\ \ \cdots \ \ 	&\pile{\rTo^s\\ \rTo_t} &
(\Pi_\omega S)(0)		\\
\end{diagram}
\]
commute serially for each $S\in\Top$.  This now looks
like~\bref{eq:endo-element}, an operation in an endomorphism operad.
Indeed, composition with $T$ induces a cartesian monad $T_*$ on the
category
\[
\ftrcat{\Top}{\ftrcat{\scat{G}^\op}{\Set}},
\]
and if $R\in\Top$ then evaluation at $R$ induces a strict map of monads
\[
\mr{ev}_R: 
(\ftrcat{\Top}{\ftrcat{\scat{G}^\op}{\Set}}, T_*)
\go
(\ftrcat{\scat{G}^\op}{\Set}, T)
\]
hence a functor
\[
(\mr{ev}_R)_*: 
T_* \hyph\Operad
\go
T \hyph \Operad.
\]
We therefore have a $T_*$-operad $\END(\Pi_\omega)$ and a globular operad
\[
P' = (\mr{ev}_\emptyset)_* (\END(\Pi_\omega)), 
\]
and a few calculations reveal that an element of $P'(\pi)$ consists of data
as in~\bref{eq:Pi-data}.  So $P'(\pi) \iso P(\pi)$, giving $P$ the
structure of an operad.  Further checks show that the operad structure is
compatible with the action on $\Pi_\omega S$ described above.
\end{example}




\section{Weak $n$-categories}
\lbl{sec:wk-n}

We now imitate what we did for $\omega$-categories to obtain a
definition of weak $n$-category.  This is straightforward except for one
subtlety.  We then explore the relationship between weak $n$-categories for
different values of $n$, and in particular we see how a weak $n$-category
can be regarded as a weak $\omega$-category trivial above dimension $n$.

First recall from~\ref{sec:cl-strict} that an $n$-globular set is a
presheaf on the category $\scat{G}_n$ and that there is a forgetful functor
from $\strcat{n}$ to $\ftrcat{\scat{G}_n^\op}{\Set}$.
Theorem~\ref{thm:n-forgetful-properties} implies that this induces a
cartesian monad $\gm{n}$%
% 
\glo{gmn}%
%
\index{n-category@$n$-category!strict!free}
%
% 
on $\ftrcat{\scat{G}_n^\op}{\Set}$.  A
$\gm{n}$-graph whose object-of-objects is $1$ will be called an
\demph{$n$-collection},%
%
\index{collection!n-@$n$-}
%
a $\gm{n}$-operad will be called an
\demph{$n$-globular operad}%
%
\index{n-globular operad@$n$-globular operad}
%
or simply an \demph{$n$-operad}, and the
category of $n$-operads will be written $n\hyph\Operad$.%
% 
\glo{nOperad}
% 

The subtlety concerns operations in the top%
%
\index{top dimension}
%
dimension.  In a
bicategory, for example, there are various ways of composing 2-cell
diagrams of shape
\[
\gfstsu\gfoursu\gzersu\gtwosu\gzersu\gthreesu\glstsu,
\]
but each such way is uniquely determined once one has chosen how to compose
the string of three 1-cells along the top and the string of three 1-cells
along the bottom.  This is essentially the `all diagrams commute'
coherence%
%
\index{coherence!bicategories@for bicategories}
%
theorem.  In creating the theory of weak $\omega$-categories we never
declared any two operations to be equal: the contraction on the operad $L$
deferred the relation to the next dimension.  In the finite-dimensional
case we cannot always defer to the next dimension, and we need a notion of
contraction sensitive to that.

\begin{defn}
Let $n\in\nat$.  A map $q: X \go Y$ of $n$-globular sets is \demph{tame}%
%
\index{tame!map of n-globular sets@map of $n$-globular sets}
%
if
any two parallel $n$-cells $\alpha^-, \alpha^+$ of $X$ satisfying
$q(\alpha^-) = q(\alpha^+)$ are equal.
\end{defn}

\begin{defn}
Let $n\in\nat$.  A \demph{precontraction}%
%
\index{precontraction|(}
%
on a map $q: X \go Y$ of
$n$-globular sets is a family of functions
\[
\left(
\mr{Par}_q(\phi) \goby{\kappa_\phi} X(m) 
\right)_{1 \leq m \leq n, \phi\in Y(m)}
\]
satisfying the same conditions as those in
Definition~\ref{defn:omega-map-contraction}.  A \demph{contraction} is a
precontraction on a tame map.
\end{defn}
% 
Tameness means injectivity in the top dimension (relative to sources and
targets), so contractibility continues to mean something like `injectivity%
%
\index{homotopy!injective on}
%
on homotopy', as in the infinite-dimensional case.  

\begin{defn}
An $n$-collection $(P \goby{d} \gm{n}1)$ is \demph{tame}%
%
\index{tame!n-collection or n-operad@$n$-collection or $n$-operad}
%
if the map $d$ is
tame, and a \demph{(pre)contraction}%
%
\index{contraction!n-collection or n-operad@on $n$-collection or $n$-operad}%
%
\index{precontraction|)}
%
on the $n$-collection is a
(pre)contraction on the map $d$.
\end{defn}
% 
Tameness of an $n$-collection $P$ means that if for all $\pi\in\pd(n)$,
parallel elements of $P(\pi)$ are equal, or equivalently the function
\[
(s,t): P(\pi) \go \mr{Par}_P(\pi)
\]
is injective.  If $P$ is precontractible then $(s,t)$ is surjective, with
one-sided inverse $\kappa_\pi$.  Hence a precontraction is a contraction if
and only if $(s,t)$ is bijective for all $\pi\in\pd(n)$.  This implies that
a contractible $n$-operad is entirely%
%
\index{top dimension}
%
determined by its lower
$(n-1)$-dimensional part; if $\pi\in\pd(n)$ then $P(\pi)$ must be the set of
parallel pairs of elements of $P(\bdry\pi)$.  

\begin{example}	\lbl{eg:sesqui}
A \demph{sesquicategory}%
%
\index{sesquicategory}
%
is a category $X$ together with a functor
$\mr{HOM}: X^\op \times X \go \Cat$ such that
\[
\begin{diagram}[height=2em]
X^\op \times X	&\rTo^{\mr{HOM}}	&\Cat		\\
		&\rdTo<{\Hom}		&\dTo>{\mr{ob}}	\\
		&			&\Set		\\
\end{diagram}
\]
commutes.  The objects of $X$ are called the \demph{0-cells} of the
sesquicategory, and if $a$ and $b$ are 0-cells then the objects and arrows
of the category $\mr{HOM}(a,b)$ are called \demph{1-cells} and
\demph{2-cells} respectively.  Any strict 2-category has an underlying
sesquicategory.

Concretely, a sesquicategory consists of a 2-globular set together with
structure and axioms saying that any diagram of shape
\[
\gfstsu\gonesu
\diagspace 
\cdots
\diagspace 
\gonesu\glstsu
\]
has a unique composite 1-cell and any diagram of shape%
%
\index{whiskering}
%
\[
\gfstsu\gonesu\ \cdots\ \gonesu\gzersu
\gdotssu
\gzersu\gonesu\ \cdots\ \gonesu\glstsu
\]
has a unique composite 2-cell.  (The sequences of cells shown can have any
length, including zero.)  See Leinster~\cite[III.1]{SHDCT} or
Street~\cite[\S 2]{StrCS}%
%
\index{Street, Ross}
%
for a more detailed presentation.  What a sesquicategory does not have is a
canonical horizontal%
%
\index{composition!horizontal}
%
composition of
2-cells: given a diagram
%
\begin{equation}	\label{diag:ssq-horiz}
\gfsts{a}
\gtwos{f}{g}{\alpha}
\gfbws{a'}
\gtwos{f'}{g'}{\alpha'}
\glsts{a''},
\end{equation}
%
the only resulting 2-cells of the form
\[
\gfsts{a}
\gtwos{f'\sof f}{g'\of g}{}
\glsts{a''}
\]
are the derived composites $(\alpha' g) \of (f' \alpha)$ and $(g' \alpha)
\of (\alpha' f)$, which in general are not equal.

There is a 2-operad $\fcat{Ssq}$ whose algebras are sesquicategories.  If
$\pi$ is a 0- or 1-pasting diagram then $\fcat{Ssq}(\pi) = 1$.  A 2-pasting
diagram is a finite sequence $(k_1, \ldots, k_n)$ of natural numbers, as
observed in~\ref{sec:free-strict}; pictorially, $n$ is the width of the
diagram and $k_i$ the height of the $i$th column.  Writing $\mb{m} = \{ 1,
\ldots, m\}$, an element of $\fcat{Ssq}(k_1, \ldots, k_n)$ is a total order%
%
\index{order!non-canonical}
%
on the disjoint union $\mb{k_1} + \cdots + \mb{k_n}$ that restricts to the
standard order on each $\mb{k_i}$; hence
\[
|\fcat{Ssq}(k_1, \ldots, k_n)|
=
\frac{(k_1 + \cdots + k_n)!}{k_1! k_2! \cdots k_n!}.
\]

A precontraction on $\fcat{Ssq}$ amounts to a choice of element of
$\fcat{Ssq}(k_1, \ldots, k_n)$ for each $n, k_1, \ldots, k_n \in \nat$, and
since this set always has at least one element, $\fcat{Ssq}$ is
precontractible.  But it is not contractible, or equivalently not tame: for
since all elements of $\fcat{Ssq}(k_1, \ldots, k_n)$ are parallel, this
would say that $|\fcat{Ssq}(k_1, \ldots, k_n)| = 1$ for all $n, k_1,
\ldots, k_n$, which is false.
\end{example}

The categories $n\hyph\fcat{OP}$%
% 
\glo{nOP}
% 
of $n$-operads-with-precontraction and
$n\hyph\fcat{OC}$%
% 
\glo{nOC}
% 
of $n$-operads-with-contraction are defined analogously
to $\fcat{OC}$~(\ref{defn:OC}).  

\begin{propn}	\lbl{propn:n-OC-initial}%
%
\index{n-globular operad@$n$-globular operad!initial with contraction}
%
%
For each $n\in\nat$, the category $n\hyph\fcat{OC}$ has an initial object.
\end{propn}
% 
We write $(L_n, \lambda_n)$%
% 
\glo{initLn}
% 
for the initial object.  The proof comes
later~(\ref{cor:Ln-from-L}).

\begin{defn}%
%
\index{n-category@$n$-category!weak}
%
Let $n\in\nat$.  A \demph{weak $n$-category} is an $L_n$-algebra.
\end{defn}
% 
We write $\wkcat{n}$%
% 
\glo{wkncat}
% 
for $\Alg(L_n)$, the category of weak $n$-categories
and strict $n$-functors.  This contains $\strcat{n}$ as a full subcategory,
just as in the infinite-dimensional case~(\ref{eg:wk-omega-cat-str}).

%
\index{n-category@$n$-category!weak!varying n@for varying $n$|(}
%
We now embark on a comparison of the theories of $n$-categories for varying
values of $n$, including $n=\omega$.  Among other things we prove
Proposition~\ref{propn:n-OC-initial}, the strategy for which is as follows.
Imagine $L_n$ being built from the bottom dimension upwards, just as we did
for $L$: then it is clear that $L_n$ and $L$ should be the same up to and
including dimension $n-1$.  They are different in dimension%
%
\index{top dimension}
%
$n$, as
discussed above in the case $n=2$: if $\pi$ is an $n$-pasting diagram then
an element of $L_n(\pi)$ is a parallel pair of elements of $L_n(\bdry\pi)$.
So we define an $n$-operad from $L$ by discarding all the operations of
dimension $n$ and higher, then adding in parallel pairs as the operations
in dimension $n$, and we prove that this is the initial
$n$-operad-with-contraction.

Let $m, M \in \nat \cup \{\omega\}$ with $m\leq M$.  We compare $m$- and
$M$-dimensional structures, using the following natural conventions for the
$\omega$-dimensional case:
\[
\scat{G}_\omega = \scat{G},
\ \ 
\gm{\omega} = T,
\ \ 
\omega\hyph\Operad = T\hyph\Operad,
\ \ 
\omega\hyph\fcat{OP} = \omega\hyph\fcat{OC} = \fcat{OC}.%
% 
\glo{Gomega}\glo{omegaOperad}\glo{omegaOC}
% 
\]

We start with globular sets, and the adjunction 
\[
\begin{diagram}[height=2em]
\ftrcat{\scat{G}_M^\op}{\Set}	\\
\dTo<{R^M_m}	\ladj	\uTo>{I^M_m}	\\
\ftrcat{\scat{G}_m^\op}{\Set}.
\end{diagram}
\]
Here $R_M^m$%
% 
\glo{RMm}
% 
is \demph{restriction}:%
%
\index{restriction}%
%
\index{globular set!restriction of}
%
if $X$ is an $M$-globular set then
$(R^M_m X)(k) = X(k)$ for all $k\leq m$.  Its right adjoint $I^M_m$%
% 
\glo{IMm}
% 
forms
the \demph{indiscrete}%
%
\index{globular set!indiscrete}
%
$M$-globular set on an $m$-globular set $Y$, 
\[
\cdots 
\ 
\pile{\rTo^1\\ \rTo_1}
S
\pile{\rTo^1\\ \rTo_1}
S
\pile{\rTo^{\mr{pr}_1}\\ \rTo_{\mr{pr}_2}}
Y(m)
\pile{\rTo^s\\ \rTo_t}
Y(m-1)
\pile{\rTo^s\\ \rTo_t}
\ 
\cdots 
\ 
\pile{\rTo^s\\ \rTo_t}
Y(0)
\]
where $S$ is the set of parallel pairs of elements of $Y(m)$.  This
adjunction is familiar in the case $M=1$, $m=0$: $R^1_0$ assigns to a
directed graph its set of objects (vertices), and $I^1_0$ forms the
indiscrete graph on a set~(\ref{eg:wk-omega-cat-indisc}).  Formally,
$R^M_m$ is composition with, and $I^M_m$ right Kan extension along, the
obvious inclusion $\scat{G}_m \rIncl \scat{G}_M$.

Next we move to the level of operads.  The functor $R^M_m$ is naturally a
weak map of monads
\[
(\ftrcat{\scat{G}_M^\op}{\Set}, \gm{M}) 
\go 
(\ftrcat{\scat{G}_m^\op}{\Set}, \gm{m})
\]
(as shown in Appendix~\ref{app:free-strict}), and preserves limits.  Hence
$R^M_m$ is naturally a cartesian colax map of monads, and by
Example~\ref{eg:cart-adjts}, there is an induced adjunction
\[
\begin{diagram}[height=2em]
M\hyph\Operad	\\
\dTo<{(R^M_m)_*} \ladj	\uTo>{(I^M_m)_*}	\\
m\hyph\Operad.
\end{diagram}%
%
\index{globular operad!restriction of}%
%
\index{globular operad!indiscrete}
%
\]
The functor $(R^M_m)_*$ forgets the top $(M-m)$ dimensions of an
$M$-operad, and $(I^M_m)_*$ is defined on an $m$-operad $Q$ at a
$k$-pasting diagram $\pi$ by
\[
\left((I^M_m)_* Q \right)
(\pi)
=
\left\{
\begin{array}{ll}
\mr{Par}_Q(\bdry^{k-m-1}\pi)	&
			\textrm{if } k > m	\\
Q(\pi)			&\textrm{if } k \leq m.
\end{array}
\right.
\]

Now we bring in precontractions.  A precontraction on an $M$-operad $P$
restricts to a precontraction on the $m$-operad $(R^M_m)_* P$, and
similarly for $(I^M_m)_*$, so the previous adjunction lifts to an
adjunction
\[
\begin{diagram}[height=2em]
M\hyph\fcat{OP}	\\
\dTo<{(R^M_m)_*} \ladj	\uTo>{(I^M_m)_*}	\\
m\hyph\fcat{OP}.
\end{diagram}
\]

So far, we have proved:
%
\begin{propn}	\lbl{propn:change-of-dimension}
For any $m, M \in \nat \cup \{\omega\}$ with $m\leq M$, there are
adjunctions
\[
\begin{diagram}[height=2em]
\ftrcat{\scat{G}_M^\op}{\Set}	\\
\dTo<{R^M_m}	\ladj	\uTo>{I^M_m}	\\
\ftrcat{\scat{G}_m^\op}{\Set}
\end{diagram}
% 
\diagspace
% 
\begin{diagram}[height=2em]
M\hyph\Operad	\\
\dTo<{(R^M_m)_*} \ladj	\uTo>{(I^M_m)_*}	\\
m\hyph\Operad
\end{diagram}
% 
\diagspace
% 
\begin{diagram}[height=2em]
M\hyph\fcat{OP}	\\
\dTo<{(R^M_m)_*} \ladj	\uTo>{(I^M_m)_*}	\\
m\hyph\fcat{OP}
\end{diagram}
\]
defined by restriction $R^M_m$ and indiscrete extension $I^M_m$.  
\done
\end{propn}

Finally, we consider contractions themselves.  We have seen that an
$n$-operad-with-contraction is entirely determined by its underlying
$(n-1)$-operad-with-precontraction.  In fact, the two types of structure
are the same: 

\begin{propn}	\lbl{propn:OP-OC}
For any positive integer $n$, the third adjunction of
Proposition~\ref{propn:change-of-dimension} in the case $M=n$,
$m=n-1$ restricts to an equivalence
\[
\begin{diagram}[height=2em]
n\hyph\fcat{OC}	\\
\dTo<{(R^n_{n-1})_*} \eqv	\uTo>{(I^n_{n-1})_*}	\\
(n-1)\hyph\fcat{OP}.
\end{diagram}
\]
\end{propn}
%
\begin{proof}
Any adjunction $\cat{C} \pile{\rTo_\bot \\ \lTo} \cat{D}$ restricts to an
equivalence between the full subcategory of $\cat{C}$ consisting of those
objects at which the unit of the adjunction is an isomorphism and the full
subcategory of $\cat{D}$ defined dually.  In the case at hand we have
$(R^n_{n-1})_* \of (I^n_{n-1})_* = 1$, and the counit is the identity
transformation.  Given an $n$-operad $P$ with precontraction $\kappa$, the
unit map
\[
P \go (I^n_{n-1})_* (R^n_{n-1})_* P
\]
is the identity in dimensions less than $n$, and in dimension $n$ is made
up of the functions
\[
(s,t): P(\pi) \go \mr{Par}_P(\pi)
\]
($\pi\in\pd(n)$).  So $P$ belongs to the relevant subcategory of
$n\hyph\fcat{OP}$ if and only if this map is a bijection for all
$\pi\in\pd(n)$, that is, $\kappa$ is a contraction.
\done
\end{proof}

\begin{example}	\lbl{eg:Gray}
Let $\fcat{Ssq}$ be the 2-operad for sesquicategories~(\ref{eg:sesqui}).%
%
\index{sesquicategory}
%
The resulting 3-operad $\fcat{Gy} = (I^3_2)_* (\fcat{Ssq})$ is given on
$k$-pasting diagrams $\pi$ by
\[
\fcat{Gy}(\pi)
=
\left\{
\begin{array}{ll}
\fcat{Ssq}(\bdry\pi) \times \fcat{Ssq}(\bdry\pi)	&
\textrm{if } k=3,	\\
\fcat{Ssq}(\pi)		&
\textrm{if } k\leq 2.
\end{array}
\right.
\]
A $\fcat{Gy}$-algebra is what we will call a \demph{Gray-category}.%
%
\index{Gray-category}
%
 It
consists of a 3-globular set $X$ with a sesquicategory structure on its 0-,
1- and 2-cells and further structure in the top dimension: given a
3-pasting diagram labelled by cells of $X$, the ways of composing it to
yield a single 3-cell correspond one-to-one with the ways of composing both
the 2-dimensional diagram at its source and the 2-dimensional diagram at
its target.  

Let us explore the 3-dimensional operations.  By considering 3-pasting
diagrams $\pi$ for which $\bdry^2\pi$ is a single 1-cell, we see that for
any two 0-cells $a$ and $b$, the 2-globular set $X(a,b)$ has the structure
of a strict 2-category.  More subtly, take the 2-pasting diagram
\[
\rho 
=
\gfstsu\gtwosu\gzersu\gtwosu\glstsu,%
%
\index{composition!horizontal}
%
\]
and recall that $\bdry(1_\rho) = \rho$ (p.~\pageref{p:bdry-degen-pd}).  We
saw in~\ref{eg:sesqui} that $\fcat{Gy}(\rho) = \fcat{Ssq}(\rho)$
has exactly two elements $\theta_1$, $\theta_2$, sending the data
in~\bref{diag:ssq-horiz} to the derived composites
\[
\gamma_1 = (\alpha' g) \of (f' \alpha),
\diagspace
\gamma_2 = (g' \alpha) \of (\alpha' f) 
\]
respectively.  So $\fcat{Gy}(1_\rho) = \{ \theta_1, \theta_2 \}^2$.  The
element $(\theta_1, \theta_2)$ of $\fcat{Gy}(1_\rho)$ sends a cell
diagram~\bref{diag:ssq-horiz} in a Gray-category to a 3-cell of the form
\[
\gfst{a}
\gthreecell{f'\of f}{g'\of g}{\gamma_1}{\gamma_2}{}
\glst{a''}.
\]
Similarly, the data of~\bref{diag:ssq-horiz} is sent by $(\theta_2,
\theta_1)$ to a 3-cell $\gamma_2 \go \gamma_1$, and by $(\theta_i,
\theta_i)$ to the identity 3-cell on $\gamma_i$.  Since these are the only
four elements of $\fcat{Gy}(1_\rho)$, the two 2-cells $\gamma_2 \oppairu
\gamma_1$ are mutually inverse.  In other words, in a Gray-category there
is a specified isomorphism between the two derived horizontal
compositions
of 2-cells.

Here Gray-categories are treated as structures in their own right, but they
were introduced by Gordon,%
%
\index{Gordon, Robert}
%
Power and Street~\cite{GPS} as a special kind of tricategory%
%
\index{tricategory}
%
(their notion of weak 3-category).  The embedding of
$\fcat{Gy}$-algebras into the class of tricategories is non-canonical,
amounting to the choice of one of the two derived horizontal compositions
of 2-cells.  `Left-handed%
%
\index{handedness}
%
Gray-categories' and `right-handed
Gray-categories' therefore form (different) subclasses of the class of all
tricategories.  Gordon, Power and Street made the arbitrary decision to
consider the left-handed version (let us say), and proved the important
result that every tricategory is equivalent%
%
\index{coherence!tricategories@for tricategories}
%
to some left-handed
Gray-category.  By duality, the same result also holds for right-handed
Gray-categories.  

We do not set up a notion of weak equivalence of our weak $n$-categories,
so cannot attempt an analogous coherence theorem, but we can show how to
realize Gray-categories as weak 3-categories.  Choose a precontraction on
$\fcat{Ssq}$.  By the last proposition, this induces a contraction on
$(I^3_2)_*(\fcat{Ssq}) = \fcat{Gy}$, hence a map $L_3 \go \fcat{Gy}$ of
3-operads, hence a functor from Gray-categories to weak 3-categories (full
and faithful, in fact).  The final functor depends on the precontraction
chosen.  A precontraction is~(\ref{eg:sesqui}) a choice for each $n, k_1,
\ldots, k_n$ of a total order on the set $\mb{k_1} + \cdots + \mb{k_n}$
restricting to the standard order on each $\mb{k_i}$, and there are two
particularly obvious ones: order the set $\{1, \ldots, n\}$ either
backwards or forwards, then order $\mb{k_1} + \cdots + \mb{k_n}$
lexicographically.  The two embeddings induced are the analogues in our
unbiased world of the left- and right-handed embeddings in the biased world
of tricategories.
\end{example}

We can now read off results relating the theories of $n$- and
$\omega$-categories.
%
\begin{cor}	\lbl{cor:Ln-from-L}
$(I^n_{n-1})_* (R^\omega_{n-1})_* (L, \lambda)$ is an initial object of
$n\hyph\fcat{OC}$, for any positive integer $n$. 
\end{cor}
%
\begin{proof}
By~\ref{propn:change-of-dimension} and~\ref{propn:OP-OC}, we have a diagram
\[
\begin{diagram}[height=2.5em] %[height=2em,width=6em]
\omega\hyph\fcat{OC}	&	&	\\
\dTo<{(R^\omega_{n-1})_*} \ladj \uTo>{(I^\omega_{n-1})_*}	&
	&	\\
(n-1)\hyph\fcat{OP}	&
\pile{\lTo^{(R^n_{n-1})_*}_{\eqv}\\ \\ \rTo_{(I^n_{n-1})_*}}	&
n \hyph \fcat{OC},	\\
\end{diagram}
\]
and left adjoints and equivalences preserve initial objects.
\done
\end{proof}
%
This proves Proposition~\ref{propn:n-OC-initial}, the existence of an
initial $n$-operad-with-contraction, for $n\geq 1$, and tells us that
\[
(L_n, \lambda_n) 
\iso
(I^n_{n-1})_* (R^\omega_{n-1})_* (L, \lambda).
\]
(The case $n=0$ is done explicitly in~\ref{sec:wk-2}.)  So $L_n$ is
constructed from $L$ by first forgetting all of $L$ above dimension $n-1$,
then adjoining the only possible family of $n$-dimensional operations that
will make the resulting $n$-operad contractible.  In particular, $L_n$ and
$L$ agree up to and including dimension $n-1$, as do their associated
contractions: 
% 
\begin{cor}	\lbl{cor:lower-same}
For any positive integer $n$, there is an isomorphism 
\[
(R^n_{n-1})_* (L_n, \lambda_n) 
\iso
(R^\omega_{n-1})_* (L, \lambda)
\]
of $(n-1)$-operads-with-precontraction.
\done
\end{cor}

The ideas we have discussed suggest two alternative definitions of weak
$n$-category, which we now formulate and prove equivalent to the main one.

The first starts from the thought that the cells of dimension at most $n$
in a weak $\omega$-category do not usually form a weak $n$-category, but
they should do if the composition of $n$-cells is strict enough.  
% 
\begin{defn}%
%
\index{tame!algebra for globular operad}%
%
\index{algebra!globular operad@for globular operad!tame}%
%
\index{globular operad!algebra for!tame}
%
Let $n\in\nat$ and let $P$ be an $n$-operad.  A $P$-algebra $X$ is
\demph{tame} if 
\[
\ovln{\theta^-} = \ovln{\theta^+}: (TX)(\pi) \go X(n)
\]
for any $\pi\in\pd(n)$ and parallel $\theta^-, \theta^+ \in P(\pi)$.
\end{defn}
% 
Any algebra for a tame $n$-operad, and in particular any algebra for a
contractible $n$-operad, is tame.  A tame $((R^\omega_n)_* L)$-algebra
ought to be the same thing as a weak $n$-category.  

To state this precisely, note that the unit of the adjunction
$(R^n_{n-1})_* \ladj (I^n_{n-1})_*$ gives a map
% \[
% 
\begin{equation}	\label{eq:tame-unit}
\alpha:
(R^\omega_n)_* L 
\go 
(I^n_{n-1})_* (R^n_{n-1})_* (R^\omega_n)_* L
\iso
L_n
\end{equation}
% 
of operads, which induces a functor
%
\begin{equation}	\label{eq:tame-eqv}
\wkcat{n} \go \Alg((R^\omega_n)_* L).
\end{equation}
%
\begin{propn}	\lbl{propn:tame-n-cat}
The functor~\bref{eq:tame-eqv} restricts to an equivalence between
$\wkcat{n}$ and the full subcategory of $\Alg((R^\omega_n)_* L)$ consisting
of the tame algebras.
\end{propn}
%
The proof uses a rather technical lemma, whose own proof is
straightforward: 
%
\begin{lemma}
Let $S$ and $S'$ be monads on a category $\cat{C}$ and let $\psi: S \go
S'$ be a natural transformation commuting with the monad structures.  Write
$\cat{D}$ for the full subcategory of $\cat{C}^S$ consisting of the
$S$-algebras $(SX \goby{h} X)$ for which $h$ factors through $\psi_X$.  If
each component of $\psi$ is split epi then the induced functor
$\cat{C}^{S'} \go \cat{C}^S$ restricts to an equivalence $\cat{C}^{S'} \go
\cat{D}$.  
\done
\end{lemma}
% 

\begin{prooflike}{Proof of Proposition~\ref{propn:tame-n-cat}}
The $n$-operads $(R^\omega_n)_* L$ and $L_n$ induce respective monads
$(\gm{n})_{(R^\omega_n)_* L}$ and $(\gm{n})_{L_n}$ on
$\ftrcat{\scat{G}_n^\op}{\Set}$, and the map $\alpha$
of~\bref{eq:tame-unit} induces a transformation $\psi$ from the first
monad to the second, commuting with the monad structures.
If $X$ is any $n$-globular set then the map $\psi_{X}$ of $n$-globular
sets is split epi: in dimension $k$ it is the function
\[
\coprod_{\pi\in\pd(k)} L(\pi) \times (\gm{n} X)(\pi)
\goby{\coprod \alpha_\pi \times 1} 
\coprod_{\pi\in\pd(k)} L_n(\pi) \times (\gm{n} X)(\pi),
\]
and $\alpha_\pi$ is bijective when $k<n$ (Corollary~\ref{cor:lower-same}),
so it is enough to show that $\alpha_\pi$ is surjective when $k=n$; this in
turn is true because
\[
\alpha_\pi = (s,t): L(\pi) \go \mr{Par}_L(\pi)
\]
and $L$ is precontractible.  

The lemma now applies, and we have only to check that $\cat{D}$ is the
subcategory consisting of the tame $((R^\omega_n)_* L)$-algebras.  Since 
$\psi$ is the identity in dimensions less than $n$, an
$((R^\omega_n)_* L)$-algebra $X$ is in $\cat{D}$ if and only if there is a
factorization
\[
\begin{diagram}[height=2em]
L(\pi) \times (\gm{n} X)(\pi)	&
\rTo^{\alpha_\pi \times 1}	&
L_n(\pi) \times (\gm{n} X)(\pi)	\\
				&
\rdTo<{\mr{action}}		&
\dGet				\\
				&
				&
X(n)
\end{diagram}
\]
in the category of sets for each $\pi\in\pd(n)$.  We have already seen that
$\alpha_\pi$ identifies two elements of $L(\pi)$ just when they are
parallel, so this is indeed tameness.
\done 
\end{prooflike}

\begin{cor}	\lbl{cor:restriction}
Let $n\in\nat$ and $N\in \nat\cup \{\omega\}$, with $n\leq N$.  If $X$
is a weak $N$-category with the property that for all $\pi\in\pd(n)$ and
parallel $\theta^-, \theta^+ \in L_N(\pi)$, 
\[
\ovln{\theta^-} = \ovln{\theta^+}: (\gm{N} X)(\pi) \go X(n),
\]
then its $n$-dimensional restriction $R^N_n X$ inherits the structure of a
weak $n$-category.
\end{cor}
%
\begin{proof}
The $L_N$-algebra structure on $X$ induces an $((R^N_n)_* L_N)$-algebra
structure on $R^N_n X$ (see~\ref{sec:change}).  If $n=N$ the result is
trivial; otherwise $(R^N_n)_* L_N \iso (R^\omega_n)_* L$
by~\ref{cor:lower-same}, and the result follows
from~\ref{propn:tame-n-cat}.  \done
\end{proof}

The second alternative definition says that a weak $n$-category is a weak
$\omega$-category with only identity cells in dimensions higher than $n$.
We show that this is equivalent to the main definition of weak
$n$-category.  More generally, we show that if $n \leq N$ then a weak
$N$-category trivial above dimension $n$ is the same thing as a weak
$n$-category.  The most simple case is that a discrete category is the same
thing as a set.

\begin{defn}
Let $n\in\nat$ and $N\in \nat\cup \{\omega\}$, with $n\leq N$.  An
$N$-globular set $X$ is \demph{$n$-dimensional}%
%
\index{globular set!n-dimensional@$n$-dimensional}
%
if for all $n\leq k < N$,
the maps
\[
X(k+1) \parpair{s}{t} X(k)
\]
are equal and bijective.
\end{defn}
% 
Any $n$-dimensional $N$-globular set is isomorphic to a `strictly
$n$-dimensional' $N$-globular set, that is, one of the form
%
\begin{equation}	\label{diag:disc-glob-set}
\cdots
\parpair{1}{1}
X(n)
\parpair{1}{1}
X(n)
\parpair{s}{t}
X(n-1)
\parpair{s}{t}
\ 
\cdots
\ 
\parpair{s}{t}
X(0).
\end{equation}

\begin{thm}	\lbl{thm:n-dim}
Let $n\in\nat$ and $N\in \nat\cup \{\omega\}$, with $n\leq N$.  There
is an equivalence of categories
\[
\wkcat{n}
\eqv
(n \textup{-dimensional weak } N \textup{-categories})
\]
where the right-hand side is a full subcategory of $\wkcat{N}$.
\end{thm}
% 
To prove this we construct the discrete weak $N$-category on a weak
$n$-category, then show that the $N$-categories so arising are exactly the
$n$-dimensional ones.

We start with the discrete construction in the setting of \emph{strict}%
%
\index{n-category@$n$-category!weak vs. strict@weak \vs.\ strict}
%
higher categories, and derive from it the weak version.  Almost all of the
steps involved are thought-free applications of previously-established
theory.  Let
\[
D^N_n: 
\ftrcat{\scat{G}_n^\op}{\Set}
\go 
\ftrcat{\scat{G}_N^\op}{\Set}%
% 
\glo{DNn}
% 
\]
be the functor sending an $n$-globular set $X$ to the $N$-globular
set of~\bref{diag:disc-glob-set}.  This lifts to a functor
\[
\strcat{n} \rIncl \strcat{N},%
%
\index{n-category@$n$-category!strict!discrete}
%
\]
and by Lemma~\ref{lemma:lax-map-mnds-is-ftr}, there is a corresponding lax
map of monads
\[
(D^N_n, \psi): 
(\ftrcat{\scat{G}_n^\op}{\Set}, \gm{n})
\go 
(\ftrcat{\scat{G}_N^\op}{\Set}, \gm{N}), 
\]
which, since $D^N_n$ preserves finite limits, induces in turn a functor
\[
(D^N_n)_*:
n\hyph\Operad
\go
N\hyph\Operad.
\]
Explicitly, if $Q$ is an $n$-operad and $\pi$ a $k$-pasting diagram then
\[
((D^N_n)_* Q)(\pi)
=
\left\{
\begin{array}{ll}
Q(\bdry^{k-n} \pi)	&\textrm{if } k \geq n	\\
Q(\pi)			&\textrm{if } k\leq n,
\end{array}
\right.
\]
from which it follows that $(D^N_n)_*$ lifts naturally to a functor
\[
(D^N_n)_*: 
n\hyph\fcat{OC}
\go
N\hyph\fcat{OC}.
\]
Since $L_N$ is initial in $N\hyph\fcat{OC}$, there is a canonical map of
operads
\[
L_N \go (D^N_n)_* L_n,
\]
inducing a functor
\[
\Alg((D^N_n)_* L_n)
\go
\Alg(L_N) = \wkcat{N}.
\]
But we also have from~\ref{sec:change} a functor
\[
\wkcat{n} = \Alg(L_n)
\go
\Alg((D^N_n)_* L_n),
\]
and so obtain a composite functor
%
\begin{equation}	\label{eq:wk-disc-functor}
D^N_n: \wkcat{n} \go \wkcat{N},%
% 
\glo{DNnweak}
% 
\end{equation}
%
as required.  A weak $N$-category isomorphic to $D^N_n Y$ for some weak
$n$-category $Y$ will be called a \demph{discrete}%
%
\index{n-category@$n$-category!weak!discrete}
%
weak $N$-category on a weak $n$-category.

On underlying globular sets, the discrete functor~\bref{eq:wk-disc-functor}
is merely the original functor $D^N_n$, so~\bref{eq:wk-disc-functor}
restricts to a functor
%
\begin{equation}	\label{eq:n-dim-eqv}
D^N_n:
\wkcat{n}
\go
(n \textup{-dimensional weak } N \textup{-categories}).
\end{equation}
%
A sharper statement of Theorem~\ref{thm:n-dim} is that this is an
equivalence of categories.

\begin{prooflike}{Proof of~\ref{thm:n-dim}}
We show that restriction%
%
\index{restriction}
%
$R^N_n$ is inverse to the functor $D^N_n$
of~\bref{eq:n-dim-eqv}.  

First, if $X$ is an $n$-dimensional weak $N$-category then
Corollary~\ref{cor:restriction} applies to give $R^N_n X$ the structure of
a weak $n$-category.  For let $\pi\in\pd(n)$ and let $\theta^-$, $\theta^+$
be parallel elements of $L_N(\pi)$.  By contractibility of $L_N$, there
exists $\theta \in L_N(1_\pi)$ satisfying $s(\theta) = \theta^-$ and
$t(\theta) = \theta^+$.  Since any $(n+1)$-cell of $X$ has the same source
and target, we have
\[
\ovln{\theta^-} 
=
\ovln{s(\theta)}
=
s \of \ovln{\theta}
=
t \of \ovln{\theta}
=
\ovln{t(\theta)}
=
\ovln{\theta^+}.
\]
So by~\ref{cor:restriction}, $R^N_n$ induces a functor
\[
R^N_n:
(n \textup{-dimensional weak } N \textup{-categories})
\go
\wkcat{n}.
\]

The composite functor $R^N_n \of D^N_n$ on $\wkcat{n}$ is the identity,
ultimately because the same is true for strict $n$-categories.  Conversely,
let $X$ be an $n$-dimensional weak $N$-category.  We may assume that $X$ is
strictly $n$-dimensional~\bref{diag:disc-glob-set}, which means that there
is an equality $D^N_n R^N_n X = X$ of globular sets.  It remains only to
check that the two $L_N$-algebra structures on $X$ agree; but certainly
they agree in dimensions $n$ and lower, and $n$-dimensionality
of $X$ guarantees that they agree in dimensions higher than $n$ too.
\done
\end{prooflike}

We finish with a method for turning higher-dimensional categories into
lower-dimensional ones.  It is an analogue of the path space construction
in topology (or with slightly different analogies, the loop space
construction or desuspension): given an $n$-category, we forget the
$0$-cells and decrease all the dimensions by $1$ to produce an
$(n-1)$-category.

Again, we start by doing it in the strict%
%
\index{n-category@$n$-category!weak vs. strict@weak \vs.\ strict}
%
setting.  A strict $n$-category
is a category enriched in $\strcat{n}$, and a finite-limit-preserving
functor $\cat{V} \go \cat{W}$ between categories with finite limits induces
a finite-limit-preserving functor $\cat{V}\hyph\Cat \go \cat{W}\hyph\Cat$,
so the functor $\Cat \go \Set$ sending a category to its set of arrows
induces a finite-limit-preserving functor
\[
\strcat{n} \go \strcat{(n-1)}
\]
for each positive integer $n$.  This is a lift of the functor
\[
\begin{array}{rrcl}
J_n:	&
\ftrcat{\scat{G}_n^\op}{\Set}		&
\go	&
\ftrcat{\scat{G}_{n-1}^\op}{\Set},	\\%
% 
\glo{Jn}
% 
	&
X	&
\goesto	&
\left(
X(n)
\parpair{s}{t}
\ 
\cdots
\ 
\parpair{s}{t}
X(1)
\right).
\end{array}
\]
By~\ref{lemma:lax-map-mnds-is-ftr}, $J_n$ has the structure of a lax map of
monads
\[
\left(
\ftrcat{\scat{G}_n^\op}{\Set}, \gm{n}
\right)
\go 
\left(
\ftrcat{\scat{G}_{n-1}^\op}{\Set}, \gm{n-1}
\right),
\]
which, since $J_n$ preserves finite limits, induces a functor
\[
(J_n)_*: 
n\hyph\Operad
\go
(n-1)\hyph\Operad.
\]
To describe $(J_n)_*$ explicitly we use the \demph{suspension}%
%
\index{suspension!globular pasting diagram@of globular pasting diagram}
%
operator
$\Sigma: \pd(k) \go \pd(k+1)$,%
% 
\glo{suspglobpd}
% 
defined by $\Sigma\pi = (\pi)$.  Here
$\pd(k+1)$ is regarded as the free monoid on $\pd(k)$ and $(\pi)$ is a
sequence of length one.  An example explains the name:
\[
\Sigma
\left(
\gfst{}\gthree{}{}{}{}{}\gfbw{}\gone{}\glst{}
\right)
\ 
=
\ 
\gfst{}\gspecialone{}{}{}{}{}{}{}{}{}\glst{}.
\]
Now, if $P$ is an $n$-operad, $0\leq k\leq n-1$, and $\pi\in\pd(k)$, 
we have
\[
((J_n)_* P)(\pi) = P(\Sigma \pi),
\]
and using the equation $\bdry\Sigma\pi = \Sigma\bdry\pi$, we find that
$(J_n)_*$ lifts naturally to a functor
\[
(J_n)_*: 
n\hyph\fcat{OC} 
\go 
(n-1)\hyph\fcat{OC}.
\]
Just as for the discrete construction, this induces a functor
\[
J_n: \wkcat{n} \go \wkcat{(n-1)}
\]
whose effect on the underlying globular sets is the original `shift'
functor $J_n$.

The $(n-1)$-category $J_n X$ arising from an $n$-category $X$ is called the
\demph{localization}%
%
\index{localization}
%
of $X$.  Its underlying $(n-1)$-globular set is the
disjoint union over all $a, b \in X(0)$ of the $(n-1)$-globular sets
$X(a,b)$ defined in the proof of~\ref{propn:str-n-cats-comparison}.  Since
the functor $\gm{n-1}$ preserves
coproducts~(\ref{thm:n-forgetful-properties}),%
%
\index{coproduct!preserved by monad}%
%
\index{monad!coproduct-preserving}
%
it follows from the lemma
below that the $(n-1)$-category structure on $J_n X$ amounts to an
$(n-1)$-category structure on each $X(a,b)$, in both the strict and the
weak settings.  So localization defines functors
%
\begin{eqnarray*}
\strcat{n}      &\go        &(\strcat{(n-1)})\hyph\Gph,        \\
\wkcat{n}       &\go        &(\wkcat{(n-1)})\hyph\Gph.         
\end{eqnarray*}
%
In the strict case, an $n$-category is a graph of $(n-1)$-categories
together with composition functors obeying simple laws; this is just
ordinary enrichment.%
%
\index{enrichment!define n-category@to define $n$-category}
%
 In the weak case it is much more difficult to say
what extra structure is needed.

\begin{lemma}  \label{lemma:coprod-algs}
Let $S$ be a cartesian monad on a presheaf category $\cat{E}$, such that
the functor part of $S$ preserves coproducts.  Let $P$ be an $S$-operad
and let $(X_i)_{i\in I}$ be a family of objects of $\cat{E}$.  Then a
$P$-algebra structure on $\coprod X_i$ amounts to a $P$-algebra structure
on each $X_i$.  
\end{lemma}
%
\begin{proof}
It is enough to show that the functor $S_P$ preserves coproducts.  Since
$S_P$ is defined using $S$ and pullback, and pullbacks interact well with
coproducts in presheaf categories, this follows from the same property of
$S$.
\done
\end{proof}

Finally, localization%
%
\index{localization}
%
works just as well for $\omega$-categories.  The
localization functors $\strcat{n} \go \strcat{(n-1)}$ induce in the limit
an endofunctor of $\strcat{\omega}$, which on underlying globular sets is
the endofunctor $J$ of $\ftrcat{\scat{G}}{\Set}$ forgetting $0$-cells.  So
exactly as in the finite-dimensional case, a weak $\omega$-category $X$
gives rise to a family $(X(a,b))_{a, b \in X(0)}$ of weak
$\omega$-categories.%
%
\index{n-category@$n$-category!weak!varying n@for varying $n$|)}
%







\section{Weak $2$-categories}
\lbl{sec:wk-2}%
%
\index{bicategory!unbiased|(}
%


A polite person proposing a definition of weak $n$-category should explain
what happens when $n=2$.  With our definition, $\wkcat{2}$ turns out to be
equivalent to $\UBistr$, the category of small unbiased bicategories and
unbiased strict functors.

Observe that since the maps in $\wkcat{2}$ are \emph{strict} functors, we
obtain an equivalence with $\UBistr$, not $\UBiwk$ or $\UBilax$; and unlike
its weak and lax siblings, $\UBistr$ is not equivalent to the analogous
category of classical%
%
\index{bicategory!unbiased vs. classical@unbiased \vs.\ classical}
%
bicategories (or at least, the obvious functor is not
an equivalence).  So we cannot conclude that $\wkcat{2}$ is equivalent to
$\Bistr$.  Nevertheless, the results of~\ref{sec:notions-bicat} mean that
it is fair to regard classical bicategories as essentially the same as
unbiased bicategories, and therefore, by the results below, essentially the
same as weak 2-categories.  One would expect that if the definition of weak
functor between $n$-categories were in place, $\Biwk$ would be equivalent
to the category of weak 2-categories and weak functors.

\begin{thm}	\lbl{thm:wk-2}
There are equivalences of categories
%
\begin{eqnarray*}
\wkcat{0}	&\eqv	&\Set,	\\
\wkcat{1}	&\eqv	&\Cat,	\\
\wkcat{2}	&\eqv	&\UBistr.
\end{eqnarray*}
\end{thm}

So far we have ignored weak $0$-categories;%
%
\index{zero-category@0-category}
%
indeed, we have not even proved
that there is an initial 0-operad-with-contraction.  A $0$-globular set is
a set and the monad $\gm{0}$ is the identity, so a $0$-operad is a monoid
and an algebra for a $0$-operad is a set acted on by the corresponding
monoid.  There is a unique precontraction on every $0$-operad, which is a
contraction just when the corresponding monoid has cardinality $1$.  So
$0\hyph\fcat{OC}$ is the category of one-element monoids, any object $L_0$
of which is initial, and
\[
\wkcat{0} \eqv \Set.
\]

A weak $1$-category%
%
\index{one-category@1-category}
%
is a $1$-dimensional weak $2$-category
(Theorem~\ref{thm:n-dim}), so the middle equivalence of
Theorem~\ref{thm:wk-2} will follow from the last.  It is, however, easy
enough to prove directly.  We have just seen that a
$0$-operad-with-precontraction is a monoid, so the initial such is also the
terminal such.  The equivalence $1\hyph\fcat{OC} \eqv 0\hyph\fcat{OP}$
of~\ref{propn:OP-OC} then tells us that the initial
$1$-operad-with-contraction is the terminal $1$-operad.  But algebras for
the terminal $\gm{1}$-operad are just $\gm{1}$-algebras, so
\[
\wkcat{1} \eqv \Cat.
\]

It is not prohibitively difficult to prove the 2-dimensional equivalence
result explicitly, as in Leinster~\cite[4.8]{OHDCT}; 2-operads are just
about manageable.  Here we use an abstract method instead, taking advantage
of some earlier calculations.

Notation: we write 
% 
\begin{itemize}
\item $W$%
% 
\glo{Wfreemon}
% 
for both the free monoid monad on $\Set$ and the free strict
monoidal category monad on $\Cat$
\item $\cat{V}\hyph\Gph$ for the category of graphs%
%
\index{graph!enriched}
%
enriched in a given
finite product category $\cat{V}$~(\ref{defn:V-gph}), and $\Gph$%
% 
\glo{abbrevGph}
% 
for
$\Set\hyph\Gph$
\item $\Sigma: \cat{V} \go \cat{V}\hyph\Gph$%
% 
\glo{suspobjgraph}%
%
\index{suspension|(}
% 
for the functor sending an
object $V$ of $\cat{V}$ to the one-object $\cat{V}$-graph whose single
hom-set is $V$
\item $\fc_\cat{V}$%
% 
\glo{fcV}
% 
for the free%
%
\index{category!free enriched}
%
$\cat{V}$-enriched category monad on
$\cat{V}\hyph\Gph$ (when it exists).
\end{itemize}

A 1-globular set is a directed graph and $\gm{1}$ is the free category monad
$\fc$ of Chapter~\ref{ch:fcm}, so a $1$-operad is an $\fc$-operad,%
%
\index{fc-operad@$\fc$-operad}
%
which is
an $\fc$-multicategory with only one 0-cell and one horizontal 1-cell.  A
precontraction $\kappa$ on an $\fc$-operad assigns to each $r\in\nat$ and
pair $(f,f')$ of vertical 1-cells a 2-cell
\[
\begin{fcdiagram}
\bullet	&\rTo		&\bullet&\rTo		&\ 	&\cdots	
&\ 	&\rTo		&\bullet	\\
\dTo<{f}&		&	&		&\Downarrow\kappa_r(f,f')&
&	&		&\dTo>{f'}	\\
\bullet	&		&	&		&\rTo	&	
&	&		&\bullet	\\
\end{fcdiagram}
\]
with $r$ 1-cells along the top.  Recall from~\ref{eg:fcm-cl-opd} that there
is an embedding
\[
\Operad \rIncl \fc\hyph\Operad
\]
identifying plain operads with $\fc$-operads having only one 0-cell, one
vertical 1-cell and one horizontal 1-cell.  There we called the embedding
$\Sigma$; here we call it $\Sigma_*$, because it is induced by the weak map
of monads $\Sigma: (\Set, W) \go (\Gph, \fc)$.  If $P$ is a plain operad
then a precontraction on $\Sigma_* P$ consists of an element $\kappa_r \in
P(r)$ for each $r\in\nat$.  Take the plain operad $\tr$ of trees%
%
\index{tree!operad of|(}
%
and the
$r$-leafed corolla $\nu_r\in\tr(r)$ for each
$r\in\nat$~(\ref{eg:opd-of-trees}): then using the fact that $\tr$ is the
free operad containing an operation of each arity, it is easy to show that
the corresponding $1$-operad-with-precontraction $\Sigma_* \tr$ is initial.
So $L_2 = (I^2_1)_* \Sigma_* \tr$, and we have
%
\begin{equation}	\label{eq:wk-2-L2}
\wkcat{2} \iso \Alg((I^2_1)_* \Sigma_* \tr).
\end{equation}

On the other hand, we saw earlier that the theory of unbiased bicategories
is also generated by the operad of trees.  Specifically, we showed
in~\ref{sec:notions-bicat} that
\[
\UBistr 
\iso 
1\hyph\Bistr
=
\fcat{CatAlg}_\mr{str} I_* \tr.%
%
\index{Sigma-bicategory@$\Sigma$-bicategory}%
%
\index{bicategory!Sigma-@$\Sigma$-}
%
\]
The functors $I_*$ and $\fcat{CatAlg}_\mr{str}$ can be described as
follows.  We have maps of monads
\[
(\Set, W)
\goby{I}
(\Cat, W)
\goby{\Sigma}
(\Cat\hyph\Gph, \fc_\Cat)
\]
where $I$ is the indiscrete%
%
\index{category!indiscrete}
%
category functor (p.~\pageref{p:indiscrete});
$I$ is lax and $\Sigma$ is weak.  Recalling from~\ref{eg:mti-Cat} that a
$(\Cat, W)$-operad is a $\Cat$-operad, we obtain induced functors
\[
\Operad 
\goby{I_*} 
\Cat\hyph\Operad
\goby{\Sigma_*}
\fc_\Cat \hyph\Operad
\goby{\Alg}
\CAT^\op.
\]
The functor $I_*$ is the same as the one used in Chapter~\ref{ch:monoidal},
and the composite of the last two functors is $\fcat{CatAlg}_\mr{str}$, so
\[
\UBistr \iso \Alg(\Sigma_* I_* \tr).%
%
\index{tree!operad of|)}
%
\]
Comparing with~\bref{eq:wk-2-L2}, it is enough to prove
% 
\begin{lemma}
For any plain operad $P$, there is an isomorphism of categories
\[
\Alg((I^2_1)_* \Sigma_* P)
\iso
\Alg(\Sigma_* I_* P).
\]
\end{lemma}
% 
\begin{proof}
We have three monads on the category $\Gph\hyph\Gph$ of 2-globular sets:
first, $\fc\hyph\Gph$, the result of applying the 2-functor
$\blank\hyph\Gph: \CAT \go \CAT$ to the monad $\fc$ on $\Gph$; second,
$\fc_\Gph$, the free $\Gph$-enriched category monad; third, $\gm{2}$.  We
show in the proof of~\ref{propn:free-enr} that $\gm{2}$ is the result of
gluing $\fc\hyph\Gph$ to $\fc_\Gph$ by a distributive%
%
\index{distributive law}
%
law
\[
\lambda: 
(\fc\hyph\Gph) \of \fc_\Gph
\go
\fc_\Gph \of (\fc\hyph\Gph).
\]
We also saw in Lemma~\ref{lemma:distrib-corr} that a distributive law gives
rise to a monad `$\twid{S}$', which in this case is the monad $\fc_\Cat$ on
$\Cat\hyph\Gph$, and in Lemma~\ref{lemma:distrib-iso-algs} that it gives
rise to a lax map of monads, 
% `$(U, \psi)$', 
which in this case is of the form
\[
(U, \psi): 
(\Cat\hyph\Gph, \fc_\Cat) 
\go 
(\Gph\hyph\Gph, \gm{2})
\]
where $U$ is the forgetful functor.  

It is straightforward to check that there is an equality of natural
transformations
\[
\begin{diagram}[size=2em]
\Set		&\rTo^W		&\Set		\\
\dTo<\Sigma	&\neeq		&\dTo>\Sigma	\\
\Gph		&\rTo^{\gm{1}}	&\Gph		\\
\dTo<{I^2_1}	&\nent		&\dTo>{I^2_1}	\\
\Gph\hyph\Gph	&\rTo_{\gm{2}}	&\Gph\hyph\Gph	\\
\end{diagram}
% 
\diagspace
=
\diagspace
% 
\begin{diagram}[size=2em]
\Set		&\rTo^W		&\Set		\\
\dTo<I		&\nent		&\dTo>I		\\
\Cat		&\rTo^W		&\Cat		\\
\dTo<\Sigma	&\neeq		&\dTo>\Sigma	\\
\Cat\hyph\Gph	&\rTo^{\fc_\Cat}&\Cat\hyph\Gph	\\
\dTo<U		&\nent\psi	&\dTo>U		\\
\Gph\hyph\Gph	&\rTo_{\gm{2}}	&\Gph\hyph\Gph	\\
\end{diagram}
\]
where the unmarked transformations are the ones referred to above.  So
\[
(I^2_1)_* \Sigma_* = U_* \Sigma_* I_*:
\Operad \go \gm{2}\hyph\Operad,
\]%
%
\index{suspension|)}%
%
and it is enough to prove that for any $\fc_\Cat$-operad $Q$,
\[
\Alg(Q) \iso \Alg(U_* Q).
\]
This follows immediately from Proposition~\ref{propn:shape-distrib}.  
\done
\end{proof}
% 
That completes the proof of Theorem~\ref{thm:wk-2}.%
%
\index{bicategory!unbiased|)}
%
  

In~\ref{sec:contr} we said what we wanted the operad $L$ to look like
in low dimensions: if $\blob$ is the unique $0$-pasting diagram then
$L(\blob)$%
%
\lbl{p:L-blob}
%
should be a one-element set, and if $\chi_k$ is the $1$-pasting diagram made
up of $k$ arrows then $L(\chi_k)$ should be $\tr(k)$.  We now know that our
wishes were met: for by~\ref{propn:change-of-dimension}, the 2-dimensional
restriction $(R^\omega_2)_* L$ is the initial 2-operad-with-precontraction,
and we have shown this to be $\Sigma_* \tr$.

%
\index{three-category@3-category!definitions of|(}%
%
\index{tricategory!non-algebraic nature of|(}
%
What about $3$-categories?  It should be possible to write down an explicit
definition of unbiased tricategory (similar to that of unbiased
bicategory,~\ref{defn:lax-bicat}) and to prove that the category of
unbiased tricategories and strict maps is equivalent to $\wkcat{3}$.  This
would be a lot of work, and it is not clear that the result would have any
advantage over the abstract definition of weak $3$-category.  

We could also try to compare our weak $3$-categories with the tricategories 
of Gordon,%
%
\index{Gordon, Robert}
%
Power and Street~\cite{GPS}.  The analogous comparison one dimension down,
between weak $2$-categories and classical bicategories, is already
difficult because we do not have a notion of weak functor between
$n$-categories (see the beginning of this section).  A further difficulty
is that Gordon, Power and Street's definition is not quite algebraic;%
%
\index{algebraic theory!tricategories are not}
%
put
another way, the forgetful functor
\[
(\textrm{tricategories } + \textrm{ strict maps})
\go
\ftrcat{\scat{G}^\op}{\Set}
\]
seems highly unlikely to be monadic.  If true, this means that there can be
no 3-operad whose algebras are precisely tricategories (in apparent
contradiction to Batanin~\cite[p.~94]{BatMGC}).%
%
\index{Batanin, Michael!globular operads@on globular operads}
%
 

The reason why the theory of tricategories is not quite algebraic is as
follows.  Most of the definition of tricategory consists of some data
subject to some equations, but a small part does not: in items~(TD5)
and~(TD6), it is stipulated that certain transformations of bicategories
are equivalences.  This is not an algebraic axiom, as there are many
different choices of weak inverses and none has been specified.  Compare
the fact that the forgetful functor from non-empty sets to $\Set$ is not
monadic%
%
\index{monadic adjunction}
%
(indeed, has no left adjoint), in contrast to the forgetful functor
from pointed sets to $\Set$.  To make the definition algebraic we would
have to add in as data a weak inverse for each of these equivalences,
together with two invertible modifications witnessing that it is a weak
inverse, and then add more coherence axioms (saying, among other things,
that this data forms an \emph{adjoint} equivalence).  The result would be
an even more complicated, but conceptually pure, notion of tricategory.%
%
\index{three-category@3-category!definitions of|)}%
%
\index{tricategory!non-algebraic nature of|)}
%











\begin{notes}

Contractions were introduced in my~\cite{SHDCT}.  When I wrote and made
public the first version of that paper I believed that I was explaining
Batanin's notion of contraction,%
%
\index{contraction!notions of}
%
but in fact I was inventing a new one;
see~\ref{sec:alg-defns-n-cat} for an explanation of the difference.  

Some of the results here on $n$- and 2-categories appeared in
my~\cite{OHDCT}.  

See Crans~\cite[2.3]{CraTPG}%
%
\index{Crans, Sjoerd}
%
for a completely elementary definition of
Gray-category~(\ref{eg:Gray}).%
%
\index{Gray-category}
%
 I learned that the operad for
Gray-categories can be defined from the operad for sesquicategories from
Batanin~\cite[p.~94]{BatMGC}.%
%
\index{Batanin, Michael!globular operads@on globular operads}
%
 


\end{notes}
