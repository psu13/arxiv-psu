
\chapter{Free Multicategories}
\lbl{app:free-mti}%
%
\index{generalized multicategory!free|(}
%


\noindent
In this appendix we define what it means for a monad or a category to be
`suitable' and prove in outline the results on free multicategories
stated in~\ref{sec:free-mti}.

First we need some terminology.  Let $\Eee$ be a cartesian category,
$\scat{I}$ a small category, $D: \scat{I} \go \Eee$ a functor for which a
colimit exists, and $(D(I) \goby{p_I} L)_{I\in\scat{I}}$ a colimit cone.
Any map $L' \go L$ gives rise to a new functor $D': \scat{I} \go \Eee$ and
a new cone $(D'(I) \goby{p'_I} L')_{I\in\scat{I}}$ by pullback: if
$\Delta L$ denotes the functor $\scat{I} \go \Eee$ constant at $L$ then 
%
\begin{diagram}[size=2em]
D'		&\rTo	&D		\\
\dTo<{p'}	&	&\dTo>p		\\
\Delta L'	&\rTo	&\Delta L	\\
\end{diagram}
%
is a pullback square in the functor category $\ftrcat{\scat{I}}{\Eee}$.  We
say that the colimit $(D(I) \goby{p_I} L)_{I\in\scat{I}}$ is \demph{stable
under pullback}%
%
\index{stable under pullback}
%
if for any map $L' \go L$ in $\Eee$, the resulting cone
$(D'(I) \goby{p'_I} L')_{I\in\scat{I}}$ is also a colimit.  

The maps $p_I$ in a colimit cone $(D(I) \goby{p_I} L)_{I\in\scat{I}}$ are
called the \demph{coprojections}%
%
\index{coprojection}
%
of the colimit, so we say that the
colimit of $D$ `has monic coprojections' if each $p_I$ is monic.

A category is said to have \demph{disjoint finite coproducts}%
%
\index{disjoint coproduct}%
%
\index{coproduct!disjoint}
%
if it has
finite coproducts, these coproducts have monic coprojections, and for any
pair $A, B$ of objects, the square
%
\begin{diagram}[size=2em]
0	&\rTo		&B		\\
\dTo	&		&\dTo		\\
A	&\rTo		&A+B		\\
\end{diagram}
%
is a pullback.

Let $\omega$ be the natural numbers with their usual ordering. A \demph{nested
sequence}%
%
\index{nested sequence}
%
in a category \Eee\ is a functor $\omega\go\Eee$ in which the image
of every map in $\omega$ is monic; in other words, it is a diagram
\[
A_0 \monic A_1 \monic \cdots
\]
in \Eee, where $\monic$ indicates a monic.  A functor that
preserves pullbacks also preserves monics, so it makes sense for such a
functor to `preserve colimits of nested sequences'. 

Let $\scat{I}$ and $\scat{J}$ be small categories and $\Eee$ a category
with all limits of shape $\scat{I}$ and colimits of shape $\scat{J}$.  We
say that limits of shape $\scat{I}$ and colimits of shape $\scat{J}$
\demph{commute}%
%
\index{commuting of limit with colimit}
%
in $\Eee$ if for each functor $P: \scat{I} \times \scat{J}
\go \Eee$, the canonical map
%
\begin{equation}	\label{eq:lim-colim-map}
\lim_{\rightarrow\scat{J}} \lim_{\leftarrow\scat{I}} P
\go
\lim_{\leftarrow\scat{I}} \lim_{\rightarrow\scat{J}} P
\end{equation}
%
is an isomorphism.  In particular, let $\scat{I}$ be the 3-object category
such that limits over $\scat{I}$ are pullbacks, and let $\scat{J} =
\omega$; we say that pullbacks and colimits of nested sequences commute in
$\Eee$ if this canonical map is an isomorphism for all functors $P$ such
that $P(I,\dashbk): \omega \go \Eee$ is a nested sequence for each $I \in
\scat{I}$.

A category $\Eee$ is \demph{suitable}%
%
\index{suitable}
%
if
%
\begin{itemize}
\item $\Eee$ is cartesian
\item $\Eee$ has disjoint finite coproducts, and these are stable under
pullback
\item $\Eee$ has colimits of nested sequences; these commute with
pullbacks and have monic coprojections.
\end{itemize}
%
A monad $(T, \mu, \eta)$ is \demph{suitable}%
%
\index{suitable}
%
if
%
\begin{itemize}
\item $(T, \mu, \eta)$ is cartesian
\item $T$ preserves colimits of nested sequences.
\end{itemize}



\section{Proofs}
\lbl{sec:free-mti-proofs}

We sketch proofs of each of the results stated in~\ref{sec:free-mti}.

\begin{quotedthm}{Theorem~\ref{thm:free-gen}}
Any presheaf category is suitable.  Any finitary cartesian monad on a
cartesian category is suitable.
\end{quotedthm}
%
\begin{proof}
First note that the category $\omega$ is filtered.  The suitability of
$\Set$ then reduces to a collection of standard facts.  Presheaf categories
are also suitable, as limits and colimits in them are computed pointwise.  
The second sentence is trivial.  
\done
\end{proof}

Before we embark on the proofs of the main theorems, here is the main idea.
A $T$-multicategory with object-of-objects $E$ is a monoid in the monoidal
category $\Eee/(TE \times E)$ (p.~\pageref{p:slice-monoidal}), so a free
$T$-multicategory is a free%
%
\index{monoid!free}
%
monoid of sorts.  The usual formula for the
free monoid on an object $X$ of a monoidal category is $\coprod_{n\in\nat}
X^{\otimes n}$.  But this only works if the tensor product preserves
countable coproducts on each side, and this is only true in our context if
$T$ preserves countable coproducts, which is often not the case---consider
plain multicategories, for instance.  So we need a more subtle
construction.  What we actually do corresponds to taking the colimit of the
sequence
\[
\coprod_{k=0}^{0} X^{\otimes k}
\rMonic
\coprod_{k=0}^{1} X^{\otimes k}
\rMonic
\coprod_{k=0}^{2} X^{\otimes k}
\rMonic
\cdots
\]
in the case that the monoidal category \emph{does} have coproducts
preserved by the tensor; in the general case we replace $\coprod_{k=0}^{n}
X^{\otimes k}$ by $X^{(n)}$, defined recursively by
\[
X^{(0)} = I,
\diagspace 
X^{(n+1)} = I + (X \otimes X^{(n)}).
\]

\begin{quotedthm}{Theorem~\ref{thm:free-main}}
Let $T$ be a suitable monad on a suitable category $\Eee$.  Then the
forgetful functor
\[
\Cartpr\hyph\Multicat \go \Eee^+ = \Cartpr\hyph\Graph
\]
has a left adjoint, the adjunction is monadic, and if $T^+$ is the induced
monad on $\Eee^+$ then both $T^+$ and $\Eee^+$ are suitable.
\end{quotedthm}

\begin{proof}
We proceed in four steps:
\begin{enumerate}
\item 	\lbl{item:ftr}
construct a functor $F: \Eee^+ \go \Cartpr\hyph\Multicat$
\item 	%\lbl{item:adjn}
construct an adjunction between $F$ and the forgetful functor $U$
\item 	%\lbl{item:newsuit}
check that $\Eee^+$ and $T^+$ are suitable
\item 	%\lbl{item:monadic}
check that the adjunction is monadic.
\end{enumerate}
%
Each step goes as follows.
%
\begin{enumerate}
%
\item \emph{Construct a functor $F: \Eee^+ \go \Cartpr\hyph\Multicat$}.
Let $X$ be a $T$-graph.  For each $n\in\nat$, define a $T$-graph
\[
\begin{slopeydiag}
	&	&X_1^{(n)}&	&	\\
	&\ldTo<{d_n}&	&\rdTo>{c_n}&	\\
TX_0	&	&	&	&X_0	\\
\end{slopeydiag}
\]
by
%
\begin{itemize}
\item $X_1^{(0)}=X_0$, $d_0=\eta_{X_0}$, and $c_0=1$
\item $X_1^{(n+1)} = X_0 + X_1\of X_1^{(n)}$ (where $\of$ is 1-cell
composition in the bicategory $\Sp{\Eee}{T}$), with the obvious choices of
$d_{n+1}$ and $c_{n+1}$.
\end{itemize}
%
For each $n\in\nat$, define a map $i_n: X_1^{(n)} \go X_1^{(n+1)}$ by taking
%
\begin{itemize}
\item $i_0: X_0 \go X_0 + X_1 \of X_0$ to be first coprojection
\item $i_{n+1} = 1_{X_0} + (1_{X_1} * i_n)$.
\end{itemize}
%
Then the $i_n$'s are monic, and by taking $X_1^*$ to be the colimit of 
\[
X_1^{(0)} \rMonic^{i_0} X_1^{(1)} \rMonic^{i_1} \cdots
\]
we obtain a $T$-graph
\[
FX =
\left(
\begin{diagram}[width=1.7em,height=1.7em,scriptlabels,noPS]
	&	&X_1^*	&	&	\\
	&\ldTo<d&	&\rdTo>c&	\\
TX_0	&	&	&	&X_0	\\
\end{diagram}
\right).
\]
This $T$-graph $FX$ naturally has the structure of a $T$-multicategory.
The identities map $X_0 \go X_1^*$ is just the coprojection $X_1^{(0)}
\monic X_1^*$.  Composition comes from canonical maps $X_1^{(m)} \of
X_1^{(n)} \go X_1^{(m+n)}$ (defined by induction on $m$ for each fixed
$n$), which piece together to give a map $X_1^* \of X_1^* \go X_1^*$.  It
is the definition of composition that needs most of the suitability
axioms.

We have now described what $F$ does to objects, and extension to morphisms
is straightforward.

(The colimit of the nested sequence of $X_1^{(n)}$'s appears, in light
disguise, as the recursive description of the free plain%
%
\index{multicategory!free}
%
multicategory
monad in~\ref{sec:om-further}: $X_1^{(n)}$ is the set of formal expressions
that can be obtained from the first clause on
p.~\pageref{p:free-plain-clauses} and up to $n$ applications of the second
clause.)

\item \emph{Construct an adjunction between $F$ and $U$}.  We do this by
constructing unit and counit transformations and verifying the triangle
identities.  Both transformations are the identity on the object of objects
(`$X_0$'), so we only need define them on the object of arrows.  For the
unit, $\eta^+$, if $X\in\Eee^+$ is a $T$-graph then $\eta^+_X: X_1 \go
X_1^*$ is the composite
\[
X_1 \goiso X_1 \of X_0 \rMonic^{\mr{copr}_2} X_0 + X_1 \of X_0 = X_1^{(1)}
\rMonic X_1^*.
\]
For the counit, $\epsln^+$, let $C$ be a $T$-multicategory; write $X =
UC$ and use the notation $X_1^{(n)}$ and $X_1^*$ as in part~\bref{item:ftr}.
We need to define a map $\epsln^+_C: X_1^* \go C$.  To do this,
define for each $n\in\nat$ a map $\epsln^+_{C,n}: X_1^{(n)} \go C_1$ by 
%
\begin{itemize}
\item $\epsln^+_{C,0} = (C_0 \goby{\ids} C_1)$
\item $\epsln^+_{C,n+1} = 
(C_0 + C_1 \of X_1^{(n)} 
\goby{1+(1*\epsln^+_{C,n})} 
C_0 + C_1 \of C_1 
\goby{q} 
C_1)$, where $q$ is $\ids$ on the first summand and $\comp$ on the second,
\end{itemize}
%
and then take $\epsln^+_C$ to be the induced map on the colimit.
Verification of the triangle identities is straightforward.

\item \emph{Check that $\Eee^+$ and $T^+$ are suitable}.  The forgetful
functor
\[
\begin{array}{rcl}
\Eee^+ 	&\go		&\Eee \times \Eee	\\
X	&\goesto	&(X_0, X_1)		\\
\end{array}
\]
creates pullbacks and colimits.  This implies that $\Eee^+$ possesses
pullbacks, finite coproducts and colimits of nested sequences, and that
they behave as well as they do in $\Eee$.  So $\Eee^+$ is suitable, and it
is now straightforward to check that $T^+$ is suitable too.

\item \emph{Check that the adjunction is monadic}.  We apply the Monadicity
Theorem (Mac Lane~\cite[VI.7]{MacCWM}) by checking that $U$ creates
coequalizers for $U$-absolute-coequalizer pairs.  This is a completely
routine procedure and works for any cartesian (not necessarily suitable)
$\Eee$ and $T$.
\done
\end{enumerate}
%
\end{proof}

The fixed-object version of the theorem is now easy to deduce:
%
\begin{quotedthm}{Theorem~\ref{thm:free-fixed}}
Let $T$ be a suitable monad on a suitable category $\Eee$, and let $E \in
\Eee$.  Then the forgetful functor
\[
\Cartpr\hyph\Multicat_E \go \Eee^+_E = \Eee/(TE \times E)
\]
has a left adjoint, the adjunction is monadic, and if $T^+_E$ is the induced
monad on $\Eee^+_E$ then both $T^+_E$ and $\Eee^+_E$ are suitable.
\end{quotedthm}
%
\begin{proof}
In the adjunction $(F,U,\eta^+,\epsln^+)$ constructed in the proof
of~\ref{thm:free-main}, each of $F$, $U$, $\eta^+$ and $\epsln^+$ leaves
the object of objects unchanged.  The adjunction therefore restricts to an
adjunction between the subcategories $\Eee^+_E$ and
$\Cartpr\hyph\Multicat_E$, and the restricted adjunction is also monadic.

All we need to check, then, is that $\Eee^+_E$ and $T^+_E$ are suitable.
For $\Eee^+_E$, it is enough to know that the slice of a suitable category
is suitable, and to prove this we need only note that for any $E' \in
\Eee$, the forgetful functor $\Eee/E' \go \Eee$ creates both pullbacks and
colimits.  Suitability of $T^+_E$ follows from suitability of $T^+$ since
the inclusion $\Eee^+_E \rIncl \Eee^+$ preserves and reflects both
pullbacks and colimits of nested sequences.  \done
\end{proof}

Finally, we prove the result stating that if $\Eee$ and $T$ have
certain special properties beyond suitability then those properties are
inherited by $\Eee^+$ and $T^+$, or, in the fixed-object case, by
$\Eee^+_E$ and $T^+_E$.
%
\begin{quotedthm}{Proposition~\ref{propn:free-refined}}
If $\Eee$ is a presheaf category and the functor $T$ preserves wide
pullbacks then the same is true of $\Eee^+$ and $T^+$ in
Theorem~\ref{thm:free-main}, and of $\Eee^+_E$ and $T^+_E$ in
Theorem~\ref{thm:free-fixed}.  Moreover, if $T$ is finitary then so are
$T^+$ and $T^+_E$.
\end{quotedthm}
%
(Wide pullbacks were defined on p.~\pageref{p:defn-wide-pb}.)
%
\begin{proof}
First we show that $\Eee^+$ and $\Eee^+_E$ are presheaf categories.  Let
$G: \Eee \go \Eee$ be the functor defined by $G(E) = T(E) \times E$: then
in the terminology of p.~\pageref{p:Artin}, $\Eee^+$ is the Artin gluing
$\Eee\gluing G$.  Since $T$ preserves wide pullbacks, so too does $G$, and
Proposition~\ref{propn:Artin-gluing} then implies that $\Eee^+$ is a
presheaf category.  The fixed-object case is easier: the slice of a
presheaf category is a presheaf category~(\ref{propn:pshf-slice}).

Next we show that $T^+$ and $T^+_E$ preserve wide pullbacks.  Recall that
in the proof of~\ref{thm:free-main}, free $T$-multicategories were
constructed using coproducts, colimits of nested sequences, pullbacks, and
the functor $T$.  (The last two of these were hidden in the notation `$X
\of X_1^{(n)}$'.)  It is therefore enough to show that all four of these
entities commute with wide pullbacks.  The last two are immediate, and
Lemma~\ref{lemma:conn-lims} implies that coproducts commute with wide
pullbacks in $\Set$ (and so in any presheaf category); all that remains is
to prove that colimits of nested sequences commute with wide pullbacks in
$\Set$.  This is actually not true in general, but a slightly weaker
statement is true and suffices.  Specifically, let $\scat{I}$ be a category
of the form $\scat{P}_K$ (p.~\pageref{p:defn-wide-pb-shape}), so that a
limit over $\scat{I}$ is a wide pullback; let $\scat{J} = \omega$; and let
$P: \scat{I} \times \scat{J} \go \Set$ be a functor such that
\[
\begin{diagram}[height=2em]
P(I,J)	&\rTo	&P(I,J')	\\
\dTo	&	&\dTo		\\
P(I',J)	&\rTo	&P(I',J')	\\
\end{diagram}
\]
is a pullback square for each pair of maps $(I \go I', J \go J')$.  Then
the canonical map~\bref{eq:lim-colim-map} (p.~\pageref{eq:lim-colim-map})
is an isomorphism.  In the case at hand these squares are of the form
\[
\begin{diagram}[height=2em]
X_1^{(n)}		&\rTo^{i_n}	&X_1^{(n+1)}		\\
\dTo<{f_1^{(n)}}	&		&\dTo>{f_1^{(n+1)}}	\\
Y_1^{(n)}		&\rTo_{i_n}	&Y_1^{(n+1)}		\\
\end{diagram}
\]
where $f: X \go Y$ is some map of $T$-graphs, and it is easily checked that
such squares are pullbacks.

For `moreover' we have to show that $T^+$ and $T^+_E$ preserve filtered
colimits if $T$ does.  Just as in the previous paragraph, this reduces to
the statement that filtered colimits commute with pullbacks in $\Set$,
whose truth is well-known (Mac Lane~\cite[IX.2]{MacCWM}).  \done
\end{proof}



\begin{notes}

These proofs first appeared in my~\cite{GECM}.  They have much in common
with the free monoid construction of Baues,%
%
\index{Baues, Hans-Joachim}
%
Jibladze%
%
\index{Jibladze, Mamuka}
%
and Tonks~\cite{BJT}.%
%
\index{Tonks, Andy}
%
\index{generalized multicategory!free|)}
% 





\end{notes}
