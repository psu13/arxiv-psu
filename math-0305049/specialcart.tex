
\chapter{Special Cartesian Monads}
\lbl{app:special-cart}

\chapterquote{%
Pictures can't say `ain't'}{%
Worth~\cite{Wor}}%
%
\index{monad!cartesian|(}
%


\noindent
We have met many monads, most of them cartesian.  Some had special
properties beyond being cartesian---for instance, some were the monads
arising from operads, and, as will be explained, some admitted a certain
explicit representation.  Here we look at these special kinds of cartesian
monad and prove some results supporting the theory in the main text.

First~(\ref{sec:opds-alg-thys}) we look at the monads arising from
plain operads.  Monads are algebraic theories, so we can ask which
algebraic theories come from operads.  The answer turns out to be the
strongly regular theories~(\ref{eg:opd-sr}).

In~\ref{sec:alt-app} we saw that the monad arising from an
operad is cartesian.
% ~(\ref{propn:ind-monad-cart}).  
It now follows that the monad corresponding to a strongly regular theory is
cartesian, a fact we used in many of the examples in
Chapter~\ref{ch:gom-basics}.  More precisely, we saw
in~\ref{cor:T-opd-colax} that a monad on $\Set$ arises from a plain operad
if and only if it is cartesian and `augmented over the free monoid monad',
meaning that there exists a cartesian natural transformation from it into
the free monoid functor, commuting with the monad structures.  On the other
hand, not every cartesian monad on $\Set$ possesses such an augmentation,
as we shall see.

One possible drawback of the generalized operad approach to
higher-dimensional category theory is that it can involve monads that are
rather hard to describe explicitly.  For instance, the sequence $T_n$ of
`opetopic' monads (Chapter~\ref{ch:opetopic}) was generated recursively
using nothing more than the existence of free operads, and to describe
$T_n$ explicitly beyond low values of $n$ is difficult.  The second and
third sections of this appendix ease this difficulty.

The basic result is that if a $\Set$-valued functor preserves infinitary or
`wide' pullbacks---which is not asking much more than it be
cartesian---then it is the coproduct of a family of representable functors.
This goes some way towards providing the explicit form desired in the
previous paragraph.  In~\ref{sec:fam-rep-Set} we look at monads on $\Set$
whose functor parts are `familially representable' in this sense.
(Finitary familially representable functors can also be viewed as a
non-symmetric version of the analytic%
%
\index{functor!analytic}
%
functors of Joyal~\cite{JoyFAE}.)%
%
\index{Joyal, Andr\'e}
%
In~\ref{sec:fam-rep-pshf} we examine monads $(T,\mu,\eta)$ on presheaf
categories whose functor parts $T$ satisfy an analogous condition.  All the
opetopic%
%
\index{opetopic!monad}
%
monads $T_n$ are of this form, and this enables us to prove
in~\ref{sec:cart-sym} that every symmetric multicategory gives rise
naturally to a $T_n$-multicategory for each $n\in\nat$.

The situation for the various types of finitary cartesian monad on $\Set$ is
depicted in Fig.~\ref{fig:cart-monads}.
%
\begin{figure}
\begin{tabular}{c}
\fbox{\parbox{.9\textwidth}{operadic monads\\
$=$ strongly regular finitary algebraic theories\\
$=$ cartesian monads augmented over the free monoid monad}}\\
$\subset$\\
\fbox{\parbox{.9\textwidth}{finitary wide-pullback-preserving monads\\
$=$ finitary connected-limit-preserving monads\\
$=$ finitary familially representable monads}}\\
$\subseteq$\\
\fbox{\parbox{.9\textwidth}{finitary cartesian monads}}
\end{tabular}
\caption{Classes of finitary cartesian monads on $\Set$}
\label{fig:cart-monads}
\end{figure}
%
The terminology is defined and the proofs are given below.
Example~\ref{eg:fam-rep-not-opdc} shows that the first inclusion is proper;
I do not know whether the second is too.







\section{Operads and algebraic theories}
\lbl{sec:opds-alg-thys}%
%
\index{algebraic theory!plain operad as|(}
%
\index{operad!algebraic theory@as algebraic theory|(}
%


Some algebraic theories can be described by plain operads, and some cannot.
In this section we will prove that the theories that can are precisely
those that admit strongly regular presentations, in the sense of
Example~\ref{eg:opd-sr}.

What exactly this means is as follows.  Any plain operad $P$ gives rise
to a monad $(T_P, \mu^P, \eta^P)$ on $\Set$, as we saw in~\ref{sec:algs}.
Say that a monad on $\Set$ is \demph{operadic}%
%
\index{monad!operadic}
%
if it is isomorphic to the
monad $(T_P, \mu^P, \eta^P)$ arising from some plain operad $P$.  On
the other hand, any algebraic theory gives rise to a monad on $\Set$, as we
will discuss in a moment.  Say that a monad on $\Set$ is \demph{strongly
regular}%
%
\index{monad!strongly regular}%
%
\index{strongly regular theory}
%
if it is isomorphic to the monad arising from some strongly
regular finitary theory.  Our result is:
%
\begin{thm}	\lbl{thm:sr-operadic}
A monad on $\Set$ is operadic if and only if it is strongly regular. 
\end{thm}
%
This means that if we have before us an algebraic theory presented by
operations and strongly regular equations then we may deduce immediately
that it can be described by an operad.  Moreover, since any operadic monad
is cartesian (Proposition~\ref{propn:ind-monad-cart}), we obtain the useful
corollary:
%
\begin{cor}
Any strongly regular monad on $\Set$ is cartesian.  
\done
\end{cor}
%
This is used throughout Section~\ref{sec:cart-monads} to generate examples
of cartesian monads on $\Set$.  It is also proved in Carboni and
Johnstone~\cite{CJ}, Proposition~3.2 (part of which is wrong, as discussed
on p.~\pageref{p:CJ-error}, but the part we are using here is right).  The
other half of Theorem~\ref{thm:sr-operadic} says that any operad can be
presented by a system of operations and strongly regular equations.  This
may not be so useful, but gives the story a tidy ending.

The definition of an algebraic theory and the way that one gives rise to a
monad on $\Set$ are well known, and a detailed account can be found in, for
instance, Manes~\cite{Manes} or Borceux~\cite{Borx2}.  Things are easier
when the theory is strongly regular, and this case is all we will need, so
I will describe it in full.

Let $\Sigma \in \Set^\nat$.  Write $F\Sigma$ for the free plain operad on
$\Sigma$ (or for the underlying object of $\Set^\nat$), as constructed
explicitly in~\ref{sec:om-further}.  We think of $\Sigma$ as a
`signature':%
%
\index{signature}
%
$\Sigma(n)$ is the set of primitive $n$-ary operations, and $(F\Sigma)(n)$
is the set of $n$-ary operations derived from the primitive ones.  A
\demph{strongly regular presentation of an algebraic theory}%
%
\index{strongly regular theory}
%
is a pair
$(\Sigma, E)$ where $\Sigma \in \Set^\nat$ and $E$ is a family
$(E_n)_{n\in\nat}$ with $E_n \sub (F\Sigma)(n) \times (F\Sigma)(n)$.  We
think of $E$ as the system of equations.  For example, the usual
presentation of the theory of semigroups is
\[
\Sigma(n) = 
\left\{
\begin{array}{ll}
\{\sigma \}	&\textrm{if } n=2	\\
\emptyset	&\textrm{otherwise,}
\end{array}
\right.
%
\diagspace
%
E_n = 
\left\{
\begin{array}{ll}
\{ (\sigma\of(\sigma, 1), \sigma\of(1, \sigma)) \}&\textrm{if } n = 3	\\
\emptyset					&\textrm{otherwise.}	\\
\end{array}
\right.
\]
(Implicitly, we are using `algebraic theory'%
%
\index{algebraic theory}
%
to mean `finitary,
single-sorted algebraic theory'.)

The operad $F\Sigma$ induces a monad $(T_{F\Sigma}, \mu^{F\Sigma},
\eta^{F\Sigma})$ on $\Set$.  We have 
\[
T_{F\Sigma} A 
= 
\coprod_{n\in\nat} (F\Sigma)(n) \times A^n
\]
for any set $A$; this is usually called the set of `$\Sigma$-terms in $A$'.
Bringing in equations, if $(\Sigma, E)$ is a strongly regular
presentation of an algebraic theory then the induced monad
$(T_{(\Sigma,E)}, \mu^{(\Sigma,E)}, \eta^{(\Sigma,E)})$ on $\Set$ is given
as follows.  For any set $A$,
\[
T_{(\Sigma,E)} A = (T_{F\Sigma} A)/\sim_A%
% 
\glo{simA}
% 
\]
where $\sim_A$ is the smallest equivalence relation on $T_{F\Sigma}A$
such that
%
\begin{itemize}
\item $\sim_A$ is a congruence:%
%
\index{congruence}
%
if $\sigma\in\Sigma(n)$ and we have $\tau_i
\in (F\Sigma)(k_i)$, $\hat{\tau}_i \in (F\Sigma)(\hat{k}_i)$, $a_i^j,
\hat{a}_i^j \in A$ satisfying
\[
(\tau_i, a_i^1, \ldots, a_i^{k_i}) \sim_A 
(\hat{\tau}_i, \hat{a}_i^1, \ldots, \hat{a}_i^{\hat{k}_i})
\]
for each $i = 1, \ldots, n$, then 
\[
(\sigma\of(\tau_1, \ldots, \tau_n), a_1^1, \ldots, a_n^{k_n})
\sim_A
(\sigma\of(\hat{\tau}_1, \ldots, \hat{\tau}_n), 
\hat{a}_1^1, \ldots, \hat{a}_n^{\hat{k}_n})
\]
\item the equations are satisfied: if $(\tau, \hat{\tau}) \in E_n$ and
$\tau_1 \in (F\Sigma)(k_1), \ldots, \tau_n \in (F\Sigma)(k_n)$, $a_i^j \in
A$, then
\[
(\tau\of(\tau_1, \ldots, \tau_n), a_1^1, \ldots, a_n^{k_n})
\sim_A
(\hat{\tau}\of(\tau_1, \ldots, \tau_n), a_1^1, \ldots, a_n^{k_n}).
\]
\end{itemize}
%
This defines an endofunctor $T_{(\Sigma,E)}$ of $\Set$ and a natural
transformation $\epsln: T_{F\Sigma} \go T_{(\Sigma,E)}$ (the quotient map).
The multiplication $\mu^{(\Sigma,E)}_A$ and unit $\eta^{(\Sigma,E)}_A$ of
the monad are the unique maps making
%
\begin{equation}	\label{eq:qt-mult-unit}
\begin{diagram}[size=2em]
T_{F\Sigma}^2 A		&\rTo^{\mu^{F\Sigma}_A}	&T_{F\Sigma} A	&
			\lTo^{\eta^{F\Sigma}_A}	&A		\\
\dTo<{(\epsln*\epsln)_A}&			&\dTo~{\epsln_A}&
						&\dEquals	\\
T_{(\Sigma,E)}^2 A	&\rGet_{\mu^{(\Sigma,E)}_A}&T_{(\Sigma,E)} A&
			\lGet_{\eta^{(\Sigma,E)}_A}&A,		\\
\end{diagram}
\end{equation}
%
commute.  By definition, a monad on $\Set$ is strongly regular if and only
if it is isomorphic to a monad of the form $(T_{(\Sigma,E)},
\mu^{(\Sigma,E)}, \eta^{(\Sigma,E)})$.

We have now given sense to the terms in the theorem and can start to prove
it.  Fix a strongly regular presentation $(\Sigma, E)$ of an algebraic
theory.  The first task is to simplify the clauses above defining $\sim_A$,
which we do by reducing to the case $A=1$.  Note that $\sim_1$ is an
equivalence relation on $\coprod_{n\in\nat} (F\Sigma)(n)$.


%
\begin{lemma}
Let $A$ be a set, $n, \hat{n} \in \nat$, $\tau \in (F\Sigma)(n)$,
$\hat{\tau} \in (F\Sigma)(\hat{n})$.
%
\begin{enumerate}
\item 	\lbl{item:one-to-A}
If $\tau \sim_1 \hat{\tau}$ then $n = \hat{n}$ and $(\tau, a_1, \ldots,
a_n) \sim_A (\hat{\tau}, a_1, \ldots, a_n)$ for all $a_1, \ldots, a_n \in
A$. 
\item 	\lbl{item:A-to-one}
If $a_1, \ldots, a_n, \hat{a}_1, \ldots, \hat{a}_{\hat{n}} \in A$ and $(\tau,
a_1, \ldots, a_n) \sim_A (\hat{\tau}, \hat{a}_1, \ldots, \hat{a}_{\hat{n}})$
then $n=\hat{n}$, $a_1=\hat{a}_1, \ldots, a_n=\hat{a}_{\hat{n}}$, and $\tau
\sim_1 \hat{\tau}$.
\end{enumerate}
\end{lemma}

\begin{proof}
To prove~\bref{item:one-to-A}, define a relation $\approx$ on
$\coprod_{m\in\nat} (F\Sigma)(m)$ as follows: for $\phi\in(F\Sigma)(m)$ and
$\hat{\phi} \in (F\Sigma)(\hat{m})$,
\[
\phi \approx \hat{\phi}
\iff
m = \hat{m} 
\textrm{ and }
(\phi, a_1, \ldots, a_m) \sim_A (\hat{\phi}, a_1, \ldots, a_m)
\textrm{ for all }
a_1, \ldots, a_m \in A.
\]
Then $\approx$ is an equivalence relation since $\sim_A$ is.  If we can
prove that $\approx$ also satisfies the two conditions for which $\sim_1$
is minimal then we will have $\sim_1 \sub \approx$,
proving~\bref{item:one-to-A}.  So, for the first condition ($\approx$ is a
congruence): suppose that $\sigma\in\Sigma(m)$ and that for each $i=1,
\ldots, m$ we have $\phi_i, \hat{\phi}_i \in (F\Sigma)(k_i)$ with $\phi_i
\approx \hat{\phi}_i$.  Then for all $a_1^1, \ldots, a_m^{k_m} \in A$, the
fact that $\phi_i \approx \hat{\phi}_i$ implies that
\[
(\phi_i, a_i^1, \ldots, a_i^{k_i}) 
\sim_A
(\hat{\phi}_i, a_i^1, \ldots, a_i^{k_i}),
\]
and so the fact that $\sim_A$ is a congruence implies that
\[
(\sigma\of(\phi_1, \ldots, \phi_m), a_1^1, \ldots, a_m^{k_m})
\sim_A
(\sigma\of(\hat{\phi}_1, \ldots, \hat{\phi}_m), a_1^1, \ldots, a_m^{k_m}).
\]
Hence $\sigma\of(\phi_1, \ldots, \phi_m) \approx \sigma\of(\hat{\phi}_1,
\ldots, \hat{\phi}_m)$, as required.  For the second condition (the
equations are satisfied): if $(\phi, \hat{\phi}) \in E_m$ then
$\phi\approx\hat{\phi}$, by taking $\tau_1, \ldots, \tau_n$ all to be the
identity $1 \in (F\Sigma)(1)$ in the second condition for $\sim_A$.  

Several of the proofs here are of this type: we have an equivalence
relation $\sim$ (in this case $\sim_1$) defined to be minimal such that
certain conditions are satisfied, and we want to prove that if $x \sim y$
then some conclusion involving $x$ and $y$ holds.  To do this we define a
new equivalence relation $\approx$ by `$x \approx y$ if and only if the
conclusion holds'; we then show that $\approx$ satisfies the conditions for
which $\sim$ was minimal, and the result follows.  These proofs, like
diagram chases in homological algebra, are more illuminating to write than
to read, so the remaining similar ones (including~\bref{item:A-to-one}) are
omitted.  None of them is significantly more complicated than the one
just done.  
\done
\end{proof}

\begin{cor}	\lbl{cor:functors-same}
For any set $A$, the quotient map $\epsln_A: T_{F\Sigma}A \go T_{(\Sigma,
E)}A$ induces a bijection 
\[
\coprod_{n\in\nat} ((F\Sigma)(n)/\sim_1) \times A^n
\goiso
T_{(\Sigma,E)}A.
\]
\ \done
\end{cor}

Having expressed the equivalence relation $\sim_A$ for an arbitrary set $A$
in terms of the single equivalence relation $\sim_1$, we now show that
$\sim_1$ has a congruence property of an operadic kind.
%
\begin{lemma}	\lbl{lemma:operadic-congruence}
If $\tau, \hat{\tau} \in (F\Sigma)(n)$ with $\tau \sim_1 \hat{\tau}$, and
$\tau_i, \hat{\tau}_i \in (F\Sigma)(k_i)$ with $\tau_i \sim_1 \hat{\tau}_i$
for each $i = 1, \ldots, n$, then
\[
\tau \of (\tau_1, \ldots, \tau_n) \sim_1 
\hat{\tau} \of (\hat{\tau}_1, \ldots, \hat{\tau}_n).
\]
\end{lemma}

\begin{proof}
Define a relation $\approx$ on $\coprod_{m\in\nat}(F\Sigma)(m)$ as follows:
if $\phi\in (F\Sigma)(m)$ and $\hat{\phi}\in (F\Sigma)(\hat{m})$ then $\phi
\approx \hat{\phi}$ if and only if $m=\hat{m}$ and
\[
\begin{array}{ll}
\textrm{for all }  	&
\phi_i, \hat{\phi}_i \in (F\Sigma)(k_i) 
\textrm{ with }
\phi_i \sim_1 \hat{\phi}_i
\ (i = 1, \ldots, n),	\\
\textrm{we have }	&
\phi \of (\phi_1, \ldots, \phi_m)
\sim_1
\hat{\phi} \of (\hat{\phi}_1, \ldots, \hat{\phi}_m).
\end{array}
\]
Then show that $\sim_1 \sub \approx$ in the usual way. 
\done
\end{proof}
%
Define $P_{(\Sigma,E)}(n) = (F\Sigma)(n)/\sim_1$ for each $n$: then
Lemma~\ref{lemma:operadic-congruence} tells us that there is a unique
operad structure on $P_{(\Sigma,E)}$ such that the quotient map $F\Sigma
\go P_{(\Sigma,E)}$ is a map of operads.  So starting from a strongly
regular presentation of a theory we have constructed an operad, and this
operad induces the same monad as the theory:
%
\begin{cor}	\lbl{cor:sr-implies-operadic}
There is an isomorphism of monads $T_{P_{(\Sigma,E)}} \iso
T_{(\Sigma,E)}$.  Hence any strongly regular monad is operadic.
\end{cor}
%
\begin{proof}
We have only to prove the first sentence; the second follows immediately.
By Corollary~\ref{cor:functors-same}, there is an isomorphism of functors
$T_{P_{(\Sigma,E)}} \iso T_{(\Sigma,E)}$ making the diagram
\[
\begin{slopeydiag}
		&	&T_{F\Sigma}	&		&	\\
		&\ldTo	&		&\rdTo>{\epsln}	&	\\
T_{P_{(\Sigma,E)}}&	&\rTo^{\diso}	&		&T_{(\Sigma,E)}\\
\end{slopeydiag}
\]
commute, where the left-hand arrow is the quotient map.  The multiplication
$\mu^{P_{(\Sigma,E)}}$ of the monad $T_{P_{(\Sigma,E)}}$ comes from
composition in the operad $P_{(\Sigma,E)}$, and this in turn comes via the
quotient map from composition in the operad $F\Sigma$, so it follows from
diagram~\bref{eq:qt-mult-unit} (p.~\pageref{eq:qt-mult-unit}) that
$\mu^{P_{(\Sigma,E)}}$ corresponds under the isomorphism to
$\mu^{(\Sigma,E)}$.  The same goes for units; hence the result.  \done
\end{proof}

This concludes the proof of the more `useful' half of
Theorem~\ref{thm:sr-operadic}.  To prove the converse, we first establish
one more fact about strongly regular presentations in general.
%
\begin{lemma}	\lbl{lemma:univ-sr}
Let $(\Sigma,E)$ be a strongly regular presentation of an algebraic
theory.  Then the quotient map $\epsln: F\Sigma \go P_{(\Sigma,E)}$ is the
universal operad map out of $F\Sigma$ with the property that $\epsln(\tau)
= \epsln(\hat{\tau})$ for all $(\tau,\hat{\tau}) \in E_n$.
\end{lemma}
%
`Universal' means that if $\zeta: F\Sigma \go Q$ is a map of operads
satisfying $\zeta(\tau) = \zeta(\hat{\tau})$ for all $(\tau,\hat{\tau}) \in
E_n$ then there is a unique operad map $\ovln{\zeta}: P_{(\Sigma,E)} \go Q$
such that $\ovln{\zeta} \of \epsln = \zeta$.
%
\begin{proof}
Universality can be re-formulated as the following condition: $\sim_1$ is
the smallest equivalence relation on $\coprod_{n\in\nat}$ that is an
operadic congruence (in other words, such that
Lemma~\ref{lemma:operadic-congruence} holds) and satisfies $\tau \sim_1
\hat{\tau}$ for all $(\tau,\hat{\tau}) \in E_n$.  The proof is by the usual
method.  \done
\end{proof}

Now fix an operad $P$.  Define $\Sigma_P \in \Set^\nat$ by
\[
\Sigma(n) = \{\sigma_\theta \such \theta\in P(n) \}
\]
where $\sigma_\theta$ is a formal symbol.  Let $E_P$ be the system of
equations with elements
\[
(\sigma_\theta \of (\sigma_{\theta_1}, \ldots, \sigma_{\theta_n}),
\sigma_{\theta\sof (\theta_1, \ldots, \theta_n)})
\in 
(E_P)_{k_1 + \cdots + k_n}
\]
for each $\theta\in P(n)$, $\theta_i \in P(k_i)$, and 
\[
(1, \sigma_1) \in (E_P)_1.
\]
Then $(\Sigma_P, E_P)$ is a strongly regular presentation of an algebraic
theory. 

There is a map $\epsln: F\Sigma_P \go P$ of operads, a component of the
counit of the adjunction between $\Operad$ and $\Set^\nat$.
Concretely~(\ref{sec:om-further}), $\epsln$ is described by the inductive
clauses
%
\begin{eqnarray*}
\epsln(1)	&=	&1,	\\
\epsln(\sigma_\theta \of (\tau_1, \ldots, \tau_n))	&
=	&
\theta \of (\epsln(\tau_1), \ldots, \epsln(\tau_n))	
\end{eqnarray*}
%
for $\theta\in P(n), \tau_i\in (F\Sigma_P)(k_i)$.
%
\begin{lemma}	\lbl{lemma:univ-operadic}
$\epsln: F\Sigma_P \go P$ is the universal operad map out of $F\Sigma_P$
with the property that $\epsln(\tau) = \epsln(\hat{\tau})$ for all
$(\tau,\hat{\tau}) \in (E_P)_n$.
\end{lemma}
%
\begin{proof}
That $\epsln$ does have the property is easily verified.  For universality,
take a map $\zeta: F\Sigma \go Q$ of operads such that $\zeta(\tau) =
\zeta(\hat{\tau})$ for all $(\tau,\hat{\tau}) \in (E_P)_n$; we want there
to be a unique operad map $\ovln{\zeta}: P \go Q$ such that $\ovln{\zeta}
\of \epsln = \zeta$.  Since $\epsln(\sigma_\theta) = \theta$ for all
$\theta \in P(n)$, the only possibility is $\ovln{\zeta}(\theta) =
\zeta(\sigma_\theta)$, and it is easy to check that this map $\ovln{\zeta}$
does satisfy the conditions required.  \done
\end{proof}

\begin{cor}	\lbl{cor:operadic-implies-sr}
There is an isomorphism of monads $T_{(\Sigma_P,E_P)} \iso T_P$.  Hence any
operadic monad is strongly regular.
\end{cor}
%
\begin{proof}
Lemmas~\ref{lemma:univ-sr} and~\ref{lemma:univ-operadic} together imply
that there is an isomorphism of operads $P_{(\Sigma_P,E_P)} \iso P$; so
there is an isomorphism of monads $T_{P_{(\Sigma_P,E_P)}} \iso T_P$.
Corollary~\ref{cor:sr-implies-operadic} implies that
$T_{P_{(\Sigma_P,E_P)}} \iso T_{(\Sigma_P,E_P)}$.  The result follows.
\done
\end{proof}

The proof of Theorem~\ref{thm:sr-operadic} is now complete.

Similar results can be envisaged for other kinds of operad.  For example,
any symmetric%
%
\index{operad!symmetric}
%
operad induces a monad on $\Set$ whose algebras are exactly
the algebras for the operad; an obvious conjecture is that the monads
arising in this way are those that can be presented by finitary operations
and equations in which the same variables appear, without repetition, on
each side (but not necessarily in the same order).  Another possibility is
to produce results of this kind for operads in a symmetric%
%
\index{operad!symmetric monoidal category@in symmetric monoidal category}
%
monoidal
category; compare Example~\ref{eg:opd-sr-enr}.  We would like, for
instance, a general principle telling us that the theory of Lie%
%
\index{Lie algebra}
%
algebras
can be described by a symmetric operad of vector spaces on the grounds that
its governing equations
%
\begin{eqnarray*}
{[}x,y] + [y,x] 				&=	&0,	\\
{[}x,[y,z]] + [y,[z,x]] + [z,[x,y]] 		&=	&0
\end{eqnarray*}
%
are both `good': the same variables are involved, without repetition, in
each summand.  This principle is well-known informally, but as far as I
know has not been proved.

A more difficult generalization would be to algebraic%
%
\index{algebraic theory!presheaf category@on presheaf category}
%
theories on presheaf
categories.  For example, take the theory of strict $\omega$-categories%
%
\index{omega-category@$\omega$-category!strict!theory of}
%
presented in the algebraic way~(\ref{defn:strict-n-cat-glob}).  This is not
a `strongly regular' presentation in the simple-minded sense, because the
interchange%
%
\index{interchange}
%
law
\[
(\beta' \ofdim{p} \beta) \ofdim{q} (\alpha' \ofdim{p} \alpha)
=
(\beta' \ofdim{q} \alpha') \ofdim{p} (\beta \ofdim{q} \alpha)
\]
does permute the order of the variables.  Yet somehow this equation should
be regarded as `good', since when the appropriate picture is drawn---e.g.
%
\lbl{p:sr-presheaf}%
%
\[
\left(
\begin{array}{c}
\gfstsu \gtwos{}{}{\alpha} \glstsu \\
\of \\
\gfstsu \gtwos{}{}{\alpha'} \glstsu
\end{array}
\right)
\ \of\ 
\left(
\begin{array}{c}
\gfstsu \gtwos{}{}{\beta} \glstsu \\
\of \\
\gfstsu \gtwos{}{}{\beta'} \glstsu
\end{array}
\right)
=
\begin{array}{c}
\left(
\gfstsu \gtwos{}{}{\alpha} \glstsu
\ \of\ 
\gfstsu \gtwos{}{}{\beta} \glstsu
\right)					\\
\of					\\
\left(
\gfstsu \gtwos{}{}{\alpha'} \glstsu
\ \of\ 
\gfstsu \gtwos{}{}{\beta'} \glstsu
\right)					
\end{array}
\]
% 
for $p=1$ and $q=0$---there is no movement in the positions of the cells.
So we hope for some notion of a strongly regular theory on a presheaf
category, and a result saying that the monad induced by such a theory is at
least cartesian, and perhaps operadic in some sense.  This would render
unnecessary the \latin{ad hoc} calculations of
Appendix~\ref{app:free-strict} showing that the free strict
$\omega$-category monad is cartesian.  However, such a general result has
yet to be found.%
%
\index{algebraic theory!plain operad as|)}
%
\index{operad!algebraic theory@as algebraic theory|)}
%





\section{Familially representable monads on $\Set$}
\lbl{sec:fam-rep-Set}


%
\index{representable functor|(}
%
Not every set-valued functor is representable, but every set-valued functor
is a colimit of representables.  It turns out that an intermediate
condition is relevant to the theory of operads: that of being a
\emph{coproduct} of representables.  Such a functor is said to be
`familially representable'.  

Familial representability has been studied in, among other places, an
important paper of Carboni%
%
\index{Carboni, Aurelio}
%
and Johnstone~\cite{CJ}.%
%
\index{Johnstone, Peter}
%
 This section is not
much more than an account of some of their results.  Our goal is to
describe monads on $\Set$ whose functor parts are familially representable
and whose natural transformation parts are cartesian; we lead up to this by
considering, more generally, familially representable functors into $\Set$.

But first, as a warm-up, consider ordinary representability.  Here follow
some very well-known facts, presented in a way that foreshadows the
material on familial representability.

Let us say that a category $\cat{A}$ has the \demph{adjoint functor
property} if any limit-preserving functor from $\cat{A}$ to a locally small
category has a left adjoint.  (`Limit-preserving' really means
`small-limit-preserving'.)  For example, the Special Adjoint Functor
Theorem states that any complete, well-powered, locally small category with
a cogenerating set has the adjoint functor property.  In particular, $\Set$
has the adjoint functor property.

\begin{propn}	\lbl{propn:eqv-condns-rep}
Let $\cat{A}$ be a category with the adjoint functor property.  The
following conditions on a functor $T: \cat{A} \go \Set$ are equivalent:
%
\begin{enumerate}
\item	\lbl{item:rep-lim}
$T$ preserves limits
\item 	\lbl{item:rep-adj}
$T$ has a left adjoint
\item 	\lbl{item:rep-rep}
$T$ is representable.
\end{enumerate}
\end{propn}
%
\begin{proof}
\begin{description}
\item[\bref{item:rep-lim}\implies\bref{item:rep-adj}] Adjoint functor
property.
\item[\bref{item:rep-adj}\implies\bref{item:rep-rep}] Take a left
adjoint $S$ to $T$: then
$T$ is represented by $S1$.
\item[\bref{item:rep-adj}\implies\bref{item:rep-rep}] Standard.
\done
\end{description}
\end{proof}

Suppose we want to define the subcategory of $\ftrcat{\cat{A}}{\Set}$
consisting of only those things that are representable.  We know what it
means for a functor $\cat{A} \go \Set$ to be representable.  By rights, a
natural transformation
%
\begin{equation}	\label{eq:rep-transf}
\cat{A}(X, \dashbk) \go \cat{A}(X', \dashbk)
\end{equation}
%
should qualify as a map in our subcategory just when it is of the form
$\cat{A}(f, \dashbk)$ for some map $f:X' \go X$ in $\cat{A}$.  But the
Yoneda Lemma implies that \emph{all} natural
transformations~\bref{eq:rep-transf} are of this form, so in fact the
appropriate subcategory is just the full subcategory of representable
functors.

Representable functors are certainly cartesian, so we might reasonably ask
when a natural transformation between representables is cartesian.%
%
\index{transformation!cartesian}
%
 The
answer is `seldom': the natural transformation~\bref{eq:rep-transf}
induced by a map $f: X' \go X$ is cartesian just when $f$ is an
isomorphism.%
%
\index{representable functor|)}
%

The basic result on familial representability is like
Proposition~\ref{propn:eqv-condns-rep}, but involves a weaker set of
equivalent conditions.  To express them we need some terminology.

Let $\cat{A}$ be a category, $I$ a set, and $(X_i)_{i\in I}$
a family of objects of $\cat{A}$: then there is a functor
\[
\coprod_{i\in I} \cat{A}(X_i, \dashbk): \cat{A} \go \Set.
\]
Such a functor, or one isomorphic to it, is said to be \demph{familially
representable};%
%
\index{familial representability!functor into Set@of functor into $\Set$}
%
$(X_i)_{i\in I}$ is the \demph{representing family}.  Note
that we can recover $I$ as the value of the functor at the terminal object
of $\cat{A}$, if it has one.

A category is \demph{connected}%
%
\index{connected}
%
if it is nonempty and any functor from it
into a discrete category is constant.  A \demph{connected limit}%
%
\index{limit!connected}
%
is a limit
of a functor whose domain is a connected category.  The crucial fact about
connected limits is:
%
\begin{lemma}	\lbl{lemma:conn-lims}
Connected limits commute with small coproducts in $\Set$.
\done
\end{lemma}
%

\begin{cor}	\lbl{cor:conn-lims}
For any set $I$, the forgetful functor $\Set/I \go \Set$ preserves and
reflects connected limits.
\end{cor}
%
\begin{proof}
$\Set/I$ is equivalent to $\Set^I$, and under this equivalence the
forgetful functor becomes the functor $\Sigma: \Set^I \go \Set$ defined by
$\Sigma((X_i)_{i\in I}) = \coprod_{i\in I} X_i$.
Lemma~\ref{lemma:conn-lims} tells us that $\Sigma$ preserves connected
limits.  Moreover, $\Set^I$ has all connected limits and $\Sigma$ reflects
isomorphisms, so preservation implies reflection.  \done
\end{proof}

One type of connected limit is of particular importance to us.  A
\demph{wide%
%
\lbl{p:defn-wide-pb}
%
pullback}%
%
\index{wide pullback}%
%
\index{pullback}
%
is a limit of shape
%
\begin{diagram}[height=2em]
\gzersu	&\gzersu&\gzersu	&\ldots	&	&\gzersu	&\ldots\\
	&\rdTo(4,2)&\rdTo(3,2)	&\rdTo(2,2)&\ldTo(1,2)&		&	\\
	&	&		&	&\gzeros{}&		&	\\
\end{diagram}
%
where the top row indicates a set of any cardinality.  Formally, for any
set $K$ let $\scat{P}_K$%
%
\lbl{p:defn-wide-pb-shape}\glo{widepbshape}
%
be the (connected) category whose objects are the elements of
$K$ together with a further object $\infty$ and whose only non-identity
maps are a single map $k \go \infty$ for each $k\in K$; a wide pullback is
a limit of a functor with domain $\scat{P}_K$ for some set $K$.  If $K$ has
cardinality two then this is just an ordinary pullback.

If $T: \cat{A} \go \cat{B}$ is a functor and $\cat{A}$ has a terminal
object then we write $\slind{T}: \cat{A} \go \cat{B}/T1$ for the functor
\[
X \goesto \bktdvslob{TX}{T!}{T1}.
\]
We recover $T$ from $\slind{T}$ as the composite
%
\begin{equation}	\label{eq:slind-comp}
T = (\cat{A} \goby{\slind{T}} \cat{B}/T1 \goby{\mr{forgetful}} \cat{B}).
\end{equation}


\begin{propn}	\lbl{propn:eqv-condns-fam-rep}
Let \cat{A} be a category with the adjoint functor property and a terminal
object.  The following conditions on a functor $T: \cat{A} \go \Set$ are
equivalent:
%
\begin{enumerate}
\item	\lbl{item:fam-rep-conn}
$T$ preserves connected limits
\item	\lbl{item:fam-rep-wide}
$T$ preserves wide pullbacks
\item	\lbl{item:fam-rep-lim}
$\slind{T}$ preserves limits
\item	\lbl{item:fam-rep-adj}
$\slind{T}$ has a left adjoint
\item	\lbl{item:fam-rep-fam-rep}
$T$ is familially representable.
\end{enumerate}
\end{propn}
%
\begin{proof}
\begin{description}
\item[\bref{item:fam-rep-conn}\implies\bref{item:fam-rep-wide}]
A wide pullback is a connected limit.
\item[\bref{item:fam-rep-wide}\implies\bref{item:fam-rep-lim}] The
forgetful functor $\Set/T1 \go \Set$ reflects wide pullbacks
(Corollary~\ref{cor:conn-lims}), and $T$ preserves them, so by
equation~\bref{eq:slind-comp}, $\slind{T}$ preserves them too.  $\slind{T}$
also preserves terminal objects, trivially.  So $\slind{T}$ preserves all
limits.
\item[\bref{item:fam-rep-lim}\implies\bref{item:fam-rep-adj}]
Adjoint functor property.
\item[\bref{item:fam-rep-adj}\implies\bref{item:fam-rep-fam-rep}]
Take a left adjoint $S$ to $\slind{T}$.  For each $i\in T1$
we have adjunctions
\[
\begin{diagram}[scriptlabels]
\cat{A}	&
\pile{\rTo^{\slind{T}}_\top\\  \\ \lTo_{S}}	&
\Set/T1	&
\pile{\rTo^{\blank_i}_\top \\ \\ \lTo_{i_!}}	&
\Set,	\\
\end{diagram}
\]
where $\blank_i$ takes the fibre over $i$ and $i_! K$ is the function $K
\go T1$ constant at $i$.  So
\[
TX	\iso	\coprod_{i\in T1} (\slind{T}X)_i	
	\iso	\coprod_{i\in T1} \Set(1, (\slind{T}X)_i)	
	\iso	\coprod_{i\in T1} \cat{A}(S i_! 1, X)
\]
naturally in $X \in \cat{A}$.

\item[\bref{item:fam-rep-fam-rep}\implies\bref{item:fam-rep-conn}]
Representables preserve limits~(\ref{propn:eqv-condns-rep}) and coproducts
commute with connected limits in $\Set$~(\ref{lemma:conn-lims}), so any
coproduct of representables preserves connected limits.
\done
\end{description}
\end{proof}

Let $(X_i)_{i\in I}$ and $(X'_{i'})_{i'\in I'}$ be families of objects in a
category $\cat{A}$.  A function $\phi: I \go I'$ together with a family of
maps $(X'_{\phi(i)} \goby{f_i} X_i)_{i\in I}$ induces%
\lbl{p:transf-fam-rep}
%
a natural transformation%
%
\index{transformation!familially representable functors@of familially representable functors}
%
\begin{equation}	\label{eq:fam-rep-transf}
\coprod_{i \in I} \cat{A}(X_i, \dashbk)
\go
\coprod_{i' \in I'} \cat{A}(X'_{i'}, \dashbk),
\end{equation}
%
and the Yoneda Lemma implies that, in fact, all natural
transformations~\bref{eq:fam-rep-transf} arise uniquely in this way.  Let
$\Fam(\cat{A})$%
% 
\glo{Fam}
% 
be the category whose objects are families $(X_i)_{i\in I}$
of objects of $\cat{A}$ and whose maps $(X_i)_{i\in I} \go (X'_{i'})_{i'\in
I'}$ are pairs $(\phi, f)$ as above.  (This is the free coproduct
cocompletion of $\cat{A}^\op$.)  There is a functor
%
\begin{equation}	\label{eq:fam-rep-Yoneda}
\begin{array}{rrcl}
\fcat{y}:	&\Fam(\cat{A})	&\go	&\ftrcat{\cat{A}}{\Set}	\\
		&(X_i)_{i\in I}	&\goesto&
\coprod_{i\in I} \cat{A} (X_i, \dashbk),%
% 
\glo{FamYon}
% 
\end{array}
\end{equation}
%
and we have just seen that it is full and faithful.  It therefore defines
an equivalence between $\Fam(\cat{A})$ and the full subcategory of
$\ftrcat{\cat{A}}{\Set}$ formed by the familially representable functors.

A transformation $(\phi, f)$ as above is cartesian%
%
\lbl{p:cart-transf-fam-rep}%
%
\index{transformation!cartesian}
%
just when each of its pieces
\[
\cat{A}(f_i, \dashbk):
\cat{A}(X_i, \dashbk)
\go
\cat{A}(X'_{\phi(i)}, \dashbk)
\]
($i\in I$) is cartesian, which, from our earlier result, happens just when
each $f_i: X'_{\phi(i)} \go X_i$ is an isomorphism.


We have already encountered many familially representable endofunctors of
$\Set$:
%
\begin{example}
For any plain operad $P$, the functor part of the induced monad $T_P$
on $\Set$ is familially representable.  Indeed,
\[
T_P(X) 
\iso \coprod_{n\in\nat} P(n) \times X^n
\iso \coprod_{i\in I} [X_i, X]
\]
where $I=\coprod_{n\in\nat} P(n)$, $X_i$ is an $n$-element set for $i\in
P(n)$, and $[X_i,X]$%
% 
\glo{homsetset}
% 
denotes the hom-set $\Set(X_i,X)$.
\end{example}
%
In this example all of the sets $X_i$ are finite and, as we saw
in~\ref{sec:opds-alg-thys}, the monad $T_P$ corresponds to a finitary
algebraic theory.  Here is the general result on finiteness.%
%
\index{finitary}%
%
\index{algebraic theory!finitary}
%
\begin{propn}	\lbl{propn:fam-rep-Set-finitary}
The following are equivalent conditions on a family $(X_i)_{i\in I}$ of
sets:
%
\begin{enumerate}
\item	\lbl{item:fam-rep-finitary}
the functor $\coprod_{i\in I} [X_i, \dashbk]: \Set \go \Set$ is finitary
\item	\lbl{item:fam-rep-finite}
the set $X_i$ is finite for each $i\in I$
\item	\lbl{item:fam-rep-fin-operadic}
there is a sequence $(P(n))_{n\in\nat}$ of sets such that 
\[
\coprod_{i\in I} [X_i, \dashbk] 
\iso 
\coprod_{n\in\nat} P(n) \times (\dashbk)^n.
\]
\end{enumerate}
\end{propn}
%
\begin{proof}
\begin{description}
\item[\bref{item:fam-rep-finitary}\implies\bref{item:fam-rep-finite}] 
For each $i\in I$ we have a pullback square
\[
\begin{diagram}[height=2.5em,width=4em]
[X_i, \dashbk]	&\rIncl^{\mr{copr}_i}	&\coprod_{i\in I} [X_i, \dashbk]\\
\dTo		&			&\dTo>{\mr{pr}}			\\
\Delta 1	&\rIncl_{\mr{copr}_i}	&\Delta I			\\
\end{diagram}
\]
in $\ftrcat{\Set}{\Set}$, where $\Delta K: \Set \go \Set$ denotes the
constant functor with value $K$.  We know that all four functors except
perhaps $[X_i, \dashbk]$ are finitary; since pullbacks commute with
filtered colimits in $\Set$ it follows that $[X_i, \dashbk]$ is finitary
too.  This in turn implies that $X_i$ is finite, as is well-known (Ad\'amek
and Rosick\'y, \cite[p.~9]{AR}).
\item[\bref{item:fam-rep-finite}\implies\bref{item:fam-rep-fin-operadic}]
For each $n\in\nat$, put
\[
P(n) = \{i \in I \such \textrm{cardinality}(X_i) = n \}.
\]
By choosing for each $i\in I$ a bijection $X_i \goiso \{1, \ldots, n\}$, we
obtain isomorphisms
\[
\coprod_{i\in I} [X_i, X] 
\iso
\coprod_{n\in\nat} \coprod_{i \in P(n)} [X_i, X]
\iso 
\coprod_{n\in\nat} \coprod_{i \in P(n)} X^n
\iso
\coprod_{n\in\nat} P(n) \times X^n
\]
natural in $X \in \Set$.
\item[\bref{item:fam-rep-fin-operadic}\implies\bref{item:fam-rep-finitary}]
If~\bref{item:fam-rep-fin-operadic} holds then $\coprod_{i\in I} [X_i,
\dashbk]$ is a colimit of endofunctors of $\Set$ of the form $(\dashbk)^n$
($n\in\nat$); each of these is finitary, so the whole functor is finitary.
\done
\end{description}
\end{proof}
%
So an endofunctor of $\Set$ is finitary and familially representable if and
only if it is, in the sense of~\bref{item:fam-rep-fin-operadic}, operadic.
The story for \emph{monads} is quite different, as we see
below~(\ref{eg:fam-rep-not-opdc}).  A hint of the subtlety is that the
isomorphism constructed in the proof of
\bref{item:fam-rep-finite}\implies\bref{item:fam-rep-fin-operadic} is not
canonical: it involves an arbitrary choice of a total ordering of $X_i$,
for each $i\in I$.

Let us turn, then, to monads on $\Set$.  A \demph{familially representable
monad}%
%
\index{familial representability!monad on Set@of monad on $\Set$}%
%
\index{monad!familially representable}
%
$(T, \mu, \eta)$ on $\Set$ is one whose functor part $T$ is
familially representable and whose natural transformation parts $\mu$ and
$\eta$ are cartesian.  (For example, any operadic monad on $\Set$ is
familially representable.)  To see what $\mu$ and $\eta$ look like in terms
of the representing family of $T$, we need to consider composition of
familially representable functors.

So, let 
\[
S \iso \coprod_{h\in H} [W_h, \dashbk],
\diagspace
T \iso \coprod_{i\in I} [X_i, \dashbk]
\]
be familially representable endofunctors of $\Set$.
Conditions~\bref{item:fam-rep-conn} and~\bref{item:fam-rep-wide} of
Proposition~\ref{propn:eqv-condns-fam-rep} make it clear that $T\of S$ must
be familially representable, but what is the representing family?  For any
set $X$ we have
%
\begin{eqnarray*}
TSX	&\iso	&\coprod_{i\in I} 
		\left[X_i, \coprod_{h\in H} [W_h, X] \right]	\\
	&\iso	&\coprod_{i\in I} \coprod_{g\in [X_i, H]}
		\left[\coprod_{x\in X_i} W_{g(x)}, X \right].	
\end{eqnarray*}
%
Hence
\[
T\of S \iso \coprod_{j\in J} [Y_j, \dashbk]
\]
where
\[
J = \coprod_{i\in I} [X_i, H] = TH,
\diagspace
Y_{i,g} = \coprod_{x\in X_i} W_{g(x)} 
\]
for $i\in I$ and $g\in [X_i, H]$ (that is, $(i,g) \in J$).  In particular, the
case $S=T$ gives
\[
T^2 \iso \coprod_{j\in J} [Y_j, \dashbk]
\]
where
\[
J = \coprod_{i\in I} [X_i, I] = TI,
\diagspace
Y_{i,g} = \coprod_{x\in X_i} X_{g(x)}. 
\]
The familial representation of the identity functor is
\[
\id \iso \coprod_{i\in 1} [1, \dashbk].
\]

A familially representable monad on $\Set$ therefore consists of
%
\begin{itemize}
\item a family $(X_i)_{i\in I}$ of sets, inducing the functor $T =
\coprod_{i\in I} [X_i, \dashbk]$
\item a function
\[
m: \{ (i, g) \such i \in I, g \in [X_i, I] \}  \go  I
\]
and for each $i\in I$ and $g\in [X_i, I]$, a bijection
\[
X_{m(i,g)} \goiso \coprod_{x\in X_i} X_{g(x)},
\]
inducing the natural transformation $\mu: T^2 \go T$ whose component
$\mu_X$ at a set $X$ is the composite
% 
\begin{eqnarray*}
\coprod_{i\in I, g\in [X_i, I]} 
\left[ \coprod_{x\in X_i} X_{g(x)}, X \right] 
&
\goiso
&
\coprod_{i\in I, g\in [X_i, I]} [X_{m(i,g)}, X]
\\
&
\goby{\mr{canonical}}
&
\coprod_{k\in I} [X_k, X]
\end{eqnarray*}
% 
\item an element $e\in I$
such that $X_e$ is a one-element set, inducing the
natural transformation $\eta: 1 \go T$ whose component $\eta_X$ at a set
$X$ is the composite
\[
X \goiso [X_e,X] \rIncl \coprod_{i\in I} [X_i, X],
\]
\end{itemize}
%
such that $\mu$ and $\eta$ obey associativity and unit laws.  The monad
is finitary just when all the $X_i$'s are finite.

\begin{example}
Let $(T, \mu, \eta)$ be the free monoid monad on $\Set$.  Then
\[
TX = \coprod_{i\in\nat} [i,X],
\]
so in this case $I=\nat$ and $X_i = i$.  (We use $i$ to denote an
$i$-element set, say $\{1, \ldots, i\}$.)  The multiplication function
\[
m: \{ (i, g) \such i \in \nat, g\in [i, \nat] \}  \go  \nat
\]
is given by
\[
m(i, g) = g(1) + \cdots + g(i),
\]
and there is an obvious choice of bijection
\[
X_{g(1) + \cdots + g(i)} \goiso X_{g(1)} + \cdots + X_{g(i)}.
\]
The unit element $e$ is $1\in\nat$.
\end{example}

This explicit form for familially representable monads on $\Set$ will
enable us to prove, for instance, that a commutative monoid is naturally an
algebra for any finitary familially representable monad on $\Set$
(Example~\ref{eg:cm-ftr-Set}).

We have now proved almost all of the equalities and inclusions in
Fig.~\ref{fig:cart-monads}.  All that remains is to prove that the first
inclusion is proper: not every finitary familially representable monad on
$\Set$ is operadic (strongly%
%
\index{strongly regular theory!familially representable monad@\vs.\ familially representable monad}%
%
\index{familial representability!strong regularity@\vs.\ strong regularity}
%
regular).  At this point we come to an error in Carboni and
Johnstone~\cite{CJ}.  In their Proposition~3.2, they prove that every
strongly regular monad on $\Set$ is familially representable; but they also
claim the converse, which is false.%
%
\lbl{p:CJ-error}  
%
The error is in the final sentence of the proof.  Having started with a
familially representable monad $(T, \mu, \eta)$, they construct a strongly
regular monad---let us call it $(T',\mu',\eta')$---and claim that the
monads $(T, \mu, \eta)$ and $(T', \mu', \eta')$ are isomorphic; and while
there is indeed a (non-canonical) isomorphism between the functors $T$ and
$T'$, there is not in general an isomorphism that commutes with the monad
structures.  The following counterexample is from Carboni and Johnstone's
corrigenda~\cite{CJcorr}.

\begin{example}		\lbl{eg:fam-rep-not-opdc}
The monad $(T, \mu, \eta)$ on $\Set$ corresponding to the theory of
monoids%
%
\index{monoid!anti-involution@with anti-involution}
%
with an anti-involution
(\ref{eg:mon-mon-with-anti-inv},~\ref{eg:mti-anti-inv}) is familially
representable and finitary, but not operadic.

The free monoid with anti-involution on a set $X$ is the set of expressions
$x_1^{\sigma_1} x_2^{\sigma_2} \cdots x_n^{\sigma_n}$ where $n\geq 0$, $x_i
\in X$, and $\sigma_i \in \{+1, -1\}$: so 
\[
TX \iso \coprod_{n\in\nat} (\{+1, -1\} \times X)^n
\iso \coprod_{i\in I} [X_i, X]
\]
where $I = \coprod_{n\in\nat} \{+1, -1\}^n$ and $X_{(\sigma_1, \ldots,
\sigma_n)} = n$.  The unit map at $X$ is 
\[
\begin{array}{rrcl}
\eta_X:	&X	&\go		&TX	\\
	&x	&\goesto	&(+1, x) \in (\{+1,-1\} \times X)^1.
\end{array}
\]
So far this is the same as for monoids with \emph{involution}
(\ref{eg:mon-mon-with-inv},~\ref{eg:mti-inv}), but the multiplication is
different: its component $\mu_X$ at $X$ is the map
\[
\coprod_{n\in\nat} \left(
\{+1,-1\} \times 
\coprod_{k\in\nat} (\{+1,-1\} \times X)^k 
\right)^n
\go
\coprod_{m\in\nat} (\{+1,-1\} \times X)^m
\]
given by
\[
\begin{array}{r}
% 
\mu_X\left(
(\sigma_1, (\sigma_1^1, x_1^1, \ldots, \sigma_1^{k_1}, x_1^{k_1})),
\ldots,
(\sigma_n, (\sigma_n^1, x_n^1, \ldots, \sigma_n^{k_n}, x_n^{k_n}))
\right)	\\
= 
(\hat{\sigma}_1^1, \hat{x}_1^1, \ldots, 
	\hat{\sigma}_1^{k_1}, \hat{x}_1^{k_1},
\ldots,
\hat{\sigma}_n^1, \hat{x}_n^1, \ldots, 
	\hat{\sigma}_n^{k_n}, \hat{x}_n^{k_n})
% 
\end{array}
\]
where for $i\in \{1, \ldots, n\}$,
\[
(\hat{\sigma}_i^1, \hat{x}_i^1, \ldots, 
	\hat{\sigma}_i^{k_i}, \hat{x}_i^{k_i})
=
\left\{
\begin{array}{ll}
(\sigma_i^1, x_i^1, \ldots, \sigma_i^{k_i}, x_i^{k_i})	&
\textrm{if } \sigma_i = +1	\\
(-\sigma_i^{k_i}, x_i^{k_i}, \ldots, -\sigma_i^1, x_i^1)&
\textrm{if } \sigma_i = -1.	\\
\end{array}
\right.
\]
The functor $T$ is plainly familially representable, and finitary
by~\ref{propn:fam-rep-Set-finitary}.  The unit and multiplication are
cartesian by the observations on p.~\pageref{p:cart-transf-fam-rep}.  So
the monad is familially representable and finitary, as claimed.

On the other hand, the following argument---reminiscent of a standard proof
of Brouwer's%
%
\index{Brouwer's fixed point theorem}
%
fixed point theorem---shows that it is not operadic.  For
suppose it were.  Then by Proposition~\ref{propn:ind-monad-cart}, there is
a cartesian natural transformation $\psi$ from $T$ to the free monoid monad
$S$, commuting with the monad structures.  There is also a cartesian
natural transformation $\theta: S \go T$ commuting with the monad
structures: in the notation of p.~\pageref{p:transf-fam-rep}, it is given
by the function
\[
\begin{array}{rrcl}
\phi:	&\nat	&\go		&\{+1, -1\}^n		\\
	&n	&\goesto	&(+1, \ldots, +1)
\end{array}
\]
together with the identity map $f_n: n \go n$ for each $n\in\nat$; on
categories of algebras, it induces the functor $\Set^T \go \Set^S$
forgetting anti-involutions.  So we have a commutative triangle
\[
\begin{slopeydiag}
	&		&T		&		&	\\
	&\ruTo<\theta	&		&\rdTo>\psi	&	\\
S	&		&\rTo_{\psi\theta}&		&S	\\
\end{slopeydiag}
\]
of cartesian natural transformations, each commuting with the monad
structures.  The members of the representing family of $S$ all have
different cardinalities, so any cartesian natural endomorphism of $S$ is
actually an automorphism.  Hence in the induced triangle of functors on
categories of algebras,
\[
\begin{diagram}[width=2em,height=2em,scriptlabels,noPS]
	&	&\textrm{(monoids with anti-involution)}&	&	\\
	&\ldTo<{\mr{forgetful}}&			&\luTo	&	\\
\fcat{Monoid}&	&\lTo					&	&
\fcat{Monoid},								\\
\end{diagram}
\]
the functor along the bottom is an isomorphism.  This implies that the
forgetful functor is surjective on objects.  So every monoid admits an
anti-involution, and in particular, every monoid is isomorphic to its
opposite (the same monoid with the order of multiplication reversed); but,
for instance, the monoid of endomorphisms of a two-element set does not
have this property.  This is the desired contradiction.
\end{example}






\section{Familially representable monads on presheaf categories}
\lbl{sec:fam-rep-pshf}


The cartesian monads arising in this book are typically monads on presheaf
categories.  Very often they are familially representable (in a sense
defined shortly), and this section provides some of the theory of such
monads.  After some general preliminaries, we concentrate on the special
case of presheaf categories $\ftrcat{\scat{B}^\op}{\Set}$ where $\scat{B}$
is discrete, arriving eventually at an explicit description of finitary
familially representable monads on such categories.  This will be used
in the next section to provide a link between symmetric and generalized
multicategories. 

Let $\cat{A}$ be a category and $\scat{B}$ a small category.  What should
it mean for a functor $T: \cat{A} \go \ftrcat{\scat{B}^\op}{\Set}$ to be
familially representable?  If $\scat{B}$ is discrete then the answer is
clear: for each $b\in\scat{B}$ the $b$-component 
\[
T_b = T(\dashbk)(b): \cat{A} \go \Set
\]
should be familially representable.  If $\scat{B}$ is not discrete then the
answer is not quite so obvious: we certainly do want each $T_b$ to be
familially representable, but we also want the representing family to vary
functorially as $b$ varies in $\scat{B}$.  Thus:
%
\begin{defn}	\lbl{defn:fam-rep-pshf}
Let $\cat{A}$ be a category and $\scat{B}$ a small category.  A functor $T:
\cat{A} \go \ftrcat{\scat{B}^\op}{\Set}$ is \demph{familially
representable}%
%
\index{familial representability!functor into presheaf category@of functor into presheaf category}
%
if there exists a functor $R$ making the diagram
\[
\begin{diagram}[height=2em]
\scat{B}^\op	&\rGet^R		&\Fam(\cat{A})		\\
		&\rdTo<{\ovln{T}}	&\dTo>{\fcat{y}}	\\
		&			&\ftrcat{\cat{A}}{\Set}	\\
\end{diagram}
\]
commute up to natural isomorphism.  Here $\ovln{T}$ is the transpose of $T$
and $\fcat{y}$ is the `Yoneda' functor of~\bref{eq:fam-rep-Yoneda}
(p.~\pageref{eq:fam-rep-Yoneda}).
\end{defn}
%
Note that when $\scat{B}$ is the terminal category, this is compatible with
the definition of familial representability for set-valued functors.  Note
also that since $\fcat{y}$ is full and faithful, the functor $R$ is
determined uniquely up to isomorphism, if it exists.

This definition is morally the correct one but in practical terms
needlessly elaborate: if each $T_b$ is familially representable then the
representing family \emph{automatically} varies functorially in $b$, as the
equivalence~\bref{item:fam-rep-pshf-fr-pw}
$\Leftrightarrow$~\bref{item:fam-rep-pshf-fr} in the following result
shows.
%
\begin{propn}	\lbl{propn:eqv-condns-fam-rep-pshf}
Let $\cat{A}$ be a category with the adjoint functor property and a
terminal object.  Let $\scat{B}$ be a small category.  The following
conditions on a functor $T: \cat{A} \go \ftrcat{\scat{B}^\op}{\Set}$ are
equivalent: 
%
\begin{enumerate}
\item	\lbl{item:fam-rep-pshf-conn}
$T$ preserves connected limits
\item	\lbl{item:fam-rep-pshf-wide}
$T$ preserves wide pullbacks
\item	\lbl{item:fam-rep-pshf-fr-pw}
for each $b\in \scat{B}$, the functor $T_b: \cat{A} \go \Set$ is familially
representable 
\item 	\lbl{item:fam-rep-pshf-fr}
$T$ is familially representable.
\end{enumerate}
\end{propn}
%
In our applications of this result, $\cat{A}$ will be a presheaf category.
Such an $\cat{A}$ does have the adjoint functor property: for by
Borceux~\cite[4.7.2]{Borx1}, it satisfies the hypotheses of the Special
Adjoint Functor Theorem.
%
\begin{proof}
%
\begin{description}
\item[\bref{item:fam-rep-pshf-conn} $\Leftrightarrow$
\bref{item:fam-rep-pshf-wide} $\Leftrightarrow$
\bref{item:fam-rep-pshf-fr-pw}] 
Since limits in a presheaf category are computed pointwise, $T$ preserves
limits of a given shape if and only if $T_b$ preserves them for each
$b\in\scat{B}$.  So these implications follow from
Proposition~\ref{propn:eqv-condns-fam-rep}.
\item[\bref{item:fam-rep-pshf-fr-pw} $\Rightarrow$ \bref{item:fam-rep-pshf-fr}]
Choose for each $b\in \scat{B}$ a representing family $(X_{b,i})_{i\in
I(b)}$ and an isomorphism
\[
\alpha_b: T_b 
\goiso 
\fcat{y}((X_{b,i})_{i\in I(b)}) = 
\coprod_{i\in I(b)} \cat{A}(X_{b,i}, \dashbk).
\]
Since $\fcat{y}$ is full and faithful, the assignment $b \goesto R(b) = 
(X_{b,i})_{i\in I(b)}$ extends uniquely to a functor $R$ such that
$\alpha$ is a natural isomorphism $\ovln{T} \goiso \fcat{y} \of R$.  
\item[\bref{item:fam-rep-pshf-fr} $\Rightarrow$
\bref{item:fam-rep-pshf-fr-pw}] 
Trivial.
\done
\end{description}
\end{proof}

A familially representable functor $\cat{A} \go
\ftrcat{\scat{B}^\op}{\Set}$ is determined by a functor $R: \scat{B}^\op
\go \Fam(\cat{A})$.  Explicitly, such a functor $R$ consists of
%
\begin{itemize}
\item a functor $I: \scat{B}^\op \go \Set$ 
\item for each $b \in \scat{B}$, a family $(X_{b,i})_{i\in I(b)}$ of objects
of $\cat{A}$
\item for each $b' \goby{g} b$ in $\scat{B}$ and each $i\in I(b)$, a map
\[
X_{g,i}: X_{b', (Ig)(i)} \go X_{b,i}
\]
in $\cat{A}$
\end{itemize}
%
such that
%
\begin{itemize}
\item if $b'' \goby{g'} b' \goby{g} b$ in $\scat{B}$ and $i\in I(b)$ then
$X_{g'\sof g, i} = X_{g, i} \of X_{g', (Ig)(i)}$
\item if $b\in \scat{B}$ then $X_{1_b,i} = 1_{X_{b,i}}$.
\end{itemize}
%
The resulting functor $T: \cat{A} \go \ftrcat{\scat{B}^\op}{\Set}$ is given
at $X \in \cat{A}$ and $b \in \scat{B}$ by
\[
(TX)(b) = \coprod_{i\in I(b)} \cat{A} (X_{b,i}, X).
\]

A good example of a familially representable functor into a presheaf
category is the free strict $\omega$-category%
%
\index{familial representability!free strict omega-category functor@of free strict $\omega$-category functor}
%
functor $T$
discussed in Chapter~\ref{ch:globular} and Appendix~\ref{app:free-strict}.
Here is the 1-dimensional version.

\begin{example}
Let $\scat{B} = (0 \parpair{\sigma}{\tau} 1)$, so that
$\ftrcat{\scat{B}^\op}{\Set}$ is the category \fcat{DGph} of directed
graphs; let $\cat{A} = \fcat{DGph}$ too.  Then the free%
%
\index{category!free (fc)@free ($\fc$)}
%
category functor
$T: \fcat{DGph} \go \fcat{DGph}$ is familially representable.  The directed
graph $I$ is defined by $I(0)=1$ and $I(1)=\nat$.  The families
$(X_{b,i})_{i\in I(b)}$ ($b\in \scat{B}$) are defined as follows.  For
$b=1$ and $i\in\nat$, let $X_{1,i}$ be the graph
\[
\gfsts{0} \gones{} \gfbws{1} \gones{} 
\diagspace \cdots \diagspace 
\gones{} \glsts{i}
\]
with $(i+1)$ vertices and $i$ edges.  For $b=0$, writing $1=\{j\}$, let
\[
X_{0,j} = \gzeros{} = X_{1,0}.
\]
The maps $0 \parpair{\sigma}{\tau} 1$ in $\scat{B}$ induce, for each
$i\in\nat$, the graph maps
\[
X_{0,j} \parpairu X_{1,i}
\]
sending $X_{0,j}$ to the first and last vertex of $X_{1,i}$, respectively.
This data does represent $T$: if $X$ is a directed graph then
\[
\coprod_{i\in I(1)} \fcat{DGph}(X_{1,i}, X)	
=
\coprod_{i\in \nat} 
\{ \textrm{strings of } i \textrm{ arrows in } X \}
= 
(TX)_1
\]
and
\[
\coprod_{j\in I(0)} \fcat{DGph}(X_{0,j}, X)	
=
X_0 
=
(TX)_0,
\]
as required.
\end{example}


A short digression: the \demph{Artin%
%
\lbl{p:Artin}
%
gluing}%
%
\index{Artin gluing}%
%
\index{gluing}
%
of a functor $T: \cat{A} \go \cat{B}$ is the category
$\cat{B}\gluing T$ in which an object is a triple $(X, Y, \pi)$ with
$X\in\cat{A}$, $Y\in\cat{B}$, and $\pi: Y \go TX$ in $\cat{B}$, and a map
$(X, Y, \pi) \go (X', Y', \pi')$ is a pair of maps $(X\go X', Y\go Y')$
making the evident square commute.
%
\begin{propn}[Carboni--Johnstone]	\lbl{propn:Artin-gluing}%
%
\index{Carboni, Aurelio}%
%
\index{Johnstone, Peter}
%
Let $\cat{A}$ and $\cat{B}$ be presheaf categories and $T: \cat{A} \go
\cat{B}$ a functor.  Then $\cat{B}\gluing T$ is a presheaf category if
and only if $T$ preserves wide pullbacks.  
\end{propn}
%
\begin{proof}
See Carboni and Johnstone~\cite[4.4(v)]{CJ}.  Their proof of `if' is a
little roundabout, so the following sketch of a direct version may be of
interest.

By Proposition~\ref{propn:eqv-condns-fam-rep-pshf}, $T$ is familially
representable.  Suppose that $\cat{A} = \ftrcat{\scat{A}^\op}{\Set}$ and
$\cat{B} = \ftrcat{\scat{B}^\op}{\Set}$, and represent $T$ by the families
$(X_{b,i})_{i\in I(b)}$, as above.  An object of $\cat{B}\gluing T$
consists of functors $X: \scat{A}^\op \go \Set$ and $Y: \scat{B}^\op \go
\Set$ and a map
\[
\pi_b: Yb \go \coprod_{i\in I(b)} 
\ftrcat{\scat{A}^\op}{\Set} (X_{b,i}, X)
\]
for each $b\in \scat{B}$, satisfying naturality axioms.  Equivalently, it
consists of
%
\begin{enumerate}
\item a functor $X: \scat{A}^\op \go \Set$,
\item a family $(Y(b,i))_{b\in\scat{B}, i\in I(b)}$ of sets, functorial in
$b$, and
\item	\lbl{item:gluing-object-three}
for each $b\in\scat{B}$, $i\in I(b)$ and $y\in Y(b,i)$, a natural
transformation $X_{b,i} \go X$, 
\end{enumerate}
%
satisfying axioms; indeed,~\bref{item:gluing-object-three} can equivalently
be replaced by
%
\begin{enumerate}
\newcommand{\temptheenumi}{\theenumi}
\renewcommand{\theenumi}{\ref{item:gluing-object-three}$'$}
\item
\renewcommand{\theenumi}{\temptheenumi} 
%
for each $b\in\scat{B}$, $i\in I(b)$, $a\in\scat{A}$ and $x\in X_{b,i}(a)$,
a function $Y(b,i) \go X(a)$.
\end{enumerate}
%
Equivalently, it is a functor $\scat{C}^\op \go \Set$, where $\scat{C}$ is
the category whose object-set $\scat{C}_0$ is the disjoint union
\[
\scat{C}_0 = \scat{A}_0 + \{(b, i) \such b\in\scat{B}_0, i\in I(b)\},
\]
whose maps are given by
%
\begin{eqnarray*}
\scat{C}(a', a)		&=	&\scat{A}(a', a),			\\
\scat{C}((b,i), (b',i'))&=	&\{g\in\scat{B}(b,b') \such (Ig)(i')=i\},\\
\scat{C}(a, (b,i))	&=	&X_{b,i}(a),				\\
\scat{C}((b,i), a)	&=	&\emptyset
\end{eqnarray*}
%
($a, a' \in \scat{A}_0$, $b, b'\in \scat{B}_0$, $i\in I(b)$, $i'\in
I(b')$), and whose composition and identities are the evident ones.
\done  
\end{proof}

%
\index{power of Set@power of $\Set$|(}%
%
\index{Set, power of@$\Set$, power of|(}
%
Returning to the main story of familially representable functors into
presheaf categories, let us restrict our attention to functors of the form
\[
T: \Set^C \go \Set^B
\]
where $C$ and $B$ are sets.  This makes the calculations much easier but
still provides a wide enough context for our applications. 

A familially representable functor $T: \Set^C \go \Set^B$ consists of 
%
\begin{itemize}
\item a family $(I(b))_{b\in B}$ of sets
\item for each $b\in B$, a family $(X_{b,i})_{i\in I(b)}$ of objects of
$\Set^C$, 
\end{itemize}
%
and the actual functor $T$ is then given by 
\[
(TX)(b) 
\iso
\coprod_{i\in I(b)} \Set^C(X_{b,i}, X) 
\iso
\coprod_{i\in I(b)} \prod_{c \in C} [X_{b,i}(c), X(c)]
\]
for each $X\in \Set^C$ and $b\in B$.

Let $T': \Set^C \go \Set^B$ be another such functor, with representing
families $(X_{b,i'})_{i'\in I'(b)}$ ($b\in B$).  A natural transformation%
%
\index{transformation!familially representable functors@of familially representable functors}
%
\[
\Set^C \ctwomult{T'}{T}{} \Set^B
\]
consists merely of a natural transformation
\[
\Set^C \ctwomult{T'_b}{T_b}{} \Set
\]
for each $b\in B$.  So by the results of the previous section, a
transformation $T' \go T$ is described by
%
\begin{itemize}
\item for each $b\in B$, a function $\phi_b: I'(b) \go I(b)$
\item for each $b\in B$ and $i' \in I'(b)$, a map 
$
f_{b,i'}: X_{b, \phi_b(i')} \go X'_{b,i'}
$
in $\Set^C$,
\end{itemize}
%
and the induced map $(T'X)(b) \go (TX)(b)$ is the composite
\[
\begin{array}{rl}
						&
\coprod_{i'\in I'(b)} \Set^C (X'_{b,i'}, X)	\\
\goby{\coprod_{i'\in I'(b)} f_{b,i'}^*}		&
\coprod_{i'\in I'(b)} \Set^C (X_{b, \phi_b(i')}, X)	\\
\goby{\mr{canonical}}				&
\coprod_{i\in I(b)} \Set^C (X_{b,i}, X)
\end{array}
\]
($X\in \Set^C$, $b\in B$).  The transformation is cartesian%
%
\index{transformation!cartesian}
%
if and only if
each map $f_{b,i'}$ is an isomorphism.

Just as for functors $\Set \go \Set$, the theory tells us that the class of
familially representable functors between presheaf categories is closed
under composition.  Working out the representing family of a composite
seems extremely complicated for presheaf categories in general, but is
manageable in our restricted context of direct powers of $\Set$.  So, take
sets $D$, $C$ and $B$ and familially representable functors
\[
\Set^D \goby{S} \Set^C \goby{T} \Set^B
\]
given by
\[
(SW)(c) \iso \coprod_{h\in H(c)} \Set^D (W_{c,h}, W),
\diagspace
(TX)(b) \iso \coprod_{i\in I(b)} \Set^C (X_{b,i}, X)
\]
($W\in \Set^D$, $X\in \Set^C$, $c\in C$, $b\in B$).  Then for $W\in \Set^D$
and $b\in B$,
%
\begin{eqnarray*}
(TSW)(b)	&\iso	&
\coprod_{i\in I(b)} \Set^C (X_{b,i}, SW)			\\
	&\iso	&
\coprod_{i\in I(b)} \prod_{c\in C} 
[X_{b,i}(c), \coprod_{h\in H(c)} \Set^D (W_{c,h}, W)]		\\
	&\iso	&
\coprod_{i\in I(b)} \prod_{c\in C} 
\coprod_{g\in [X_{b,i}(c), H(c)]} \prod_{x\in X_{b,i}(c)}
\Set^D (W_{c, g(x)}, W)						\\
	&\iso	&
\coprod_{i\in I(b)} \coprod_{\gamma\in \Set^C(X_{b,i}, H)} 
\prod_{c\in C} \prod_{x\in X_{b,i}(c)}
\Set^D (W_{c, \gamma_c(x)}, W)						\\
	&\iso	&
\coprod_{(i,\gamma)\in J(b)} \Set^D (Y_{b,i,\gamma}, W),
\end{eqnarray*}
%
where 
%
\begin{eqnarray*}
J(b)		&=	&
\coprod_{i\in I(b)} \Set^C (X_{b,i}, H) \iso (TH)(b),	\\
Y_{b,i,\gamma}	&=	&
\coprod_{c\in C, x\in X_{b,i}(c)} W_{c,\gamma_c(x)}.
\end{eqnarray*}
%
The identity functor on $\Set^B$ is also familially representable: for
$X\in \Set^B$, 
\[
X(b) \iso \coprod_{i\in 1} \Set^B (\delta_b, X)
\]
where, treating $B$ as a discrete category, $\delta_b = B(\dashbk,b) \in
\Set^B$.

Assembling these descriptions gives a description of monads on $\Set^B$
whose functor parts are familially representable and whose natural
transformation parts are cartesian---\demph{familially representable
monads}%
%
\index{familial representability!monad on power of Set@of monad on power of $\Set$}%
%
\index{monad!familially representable}
%
on $\Set^B$, as we call them.  Such a monad consists of:
%
\begin{itemize}
\item for each $b\in B$, a set $I(b)$ and a family $(X_{b,i})_{i\in I(b)}$
of objects of $\Set^B$, inducing the functor
\[
\begin{array}{rl}
T: 		&\Set^B \go \Set^B,		\\
(TX)(b) = 	&\coprod_{i\in I(b)} \Set^B (X_{b,i}, X)
\end{array}
\]
\item for each $b\in B$, a function
\[
m_b: 
\{(i,\gamma) \such i\in I(b), \gamma\in \Set^B (X_{b,i}, X) \}
\go
I(b),
\]
and for each $b\in B$, $i\in I(b)$ and $\gamma\in\Set^B (X_{b,i}, X)$, an
isomorphism 
\[
X_{b, m_b(i,\gamma)} 
\goiso
\coprod_{c\in B, x\in X_{b,i}(c)} X_{c,\gamma_c(x)},
\]
inducing the natural transformation $\mu: T^2 \go T$ whose component
\[
\mu_{X,b}: (T^2 X)(b) \go (TX)(b)
\]
is the composite
\[
\begin{array}{rl}
	&
\coprod_{i\in I(b), \gamma\in \Set^B (X_{b,i}, I)}
\Set^B (\coprod_{c\in B, x\in X_{b,i}(c)} X_{c,\gamma_c(x)}, X)	\\
\goiso	&
\coprod_{i\in I(b), \gamma\in \Set^B (X_{b,i}, I)}
\Set^B (X_{b, m_b(i,\gamma)}, X)				\\
\goby{\mr{canonical}}	&
\coprod_{k\in I(b)} \Set^B (X_{b,k}, X)
\end{array}
\]
\item for each $b\in B$, an element $e_b \in I(b)$ such that $X_{b, e_b}
\iso \delta_b$, inducing the natural transformation $\eta: 1 \go T$ whose
component
\[
\eta_{X,b}: X(b) \go (TX)(b)
\]
is the composite
\[
X(b) 
\goiso 
\Set^B (X_{b,e_b}, X)
\rIncl
\coprod_{i\in I(b)} \Set^B (X_{b,i}, X),
\]
\end{itemize}
%
such that $\mu$ and $\eta$ obey associativity and unit laws.  This is
complicated, but rest assured that it gets no worse.

The aim of this section was to describe finitary familially representable
monads on categories of the form $\Set^B$, and we are nearly there.  All
that remains is `finitary'.%
%
\index{finitary}
%
\begin{propn}	\lbl{propn:fr-pshf-finitary}
Let $C$ and $B$ be sets.  Let $T: \Set^C \go \Set^B$ be a familially
representable functor with representing families $(X_{b,i})_{i\in I(b)}$
($b\in B$).  The following
% conditions on $T$ 
are equivalent:
%
\begin{enumerate}
\item	\lbl{item:fin-fam-rep-pshf-finitary}
$T$ is a finitary functor
\item	\lbl{item:fin-fam-rep-pshf-finite-power}
$\coprod_{c\in C} X_{b,i} (c)$ is a finite set for each $b\in B$ and $i\in
I(b)$. 
\end{enumerate}
\end{propn}
%
\begin{proof}
$T$ is finitary if and only if the functor
\[
T_b = \coprod_{i\in I(b)} \Set^C (X_{b,i}, \dashbk): 	
\Set^C	\go \Set
\]
is finitary for each $b\in B$.  By the arguments in the proof of
Proposition~\ref{propn:fam-rep-Set-finitary}, $T_b$ is finitary if and only
if the functor
\[
\Set^C (X_{b,i}, \dashbk): \Set^C \go \Set
\]
is finitary for each $i\in I(b)$.  But $\Set^C (X_{b,i}, \dashbk)$ is
finitary if and only if $\coprod_{c\in C} X_{b,i}(c)$ is a finite set
(Ad\'amek%
%
\index{Adamek, Jiri@Ad\'amek, Ji\v{r}\'\i}
%
and Rosick\'y~\cite[p.~9]{AR}).%
%
\index{Rosicky, Jiri@Rosick\'y, Ji\v{r}\'\i}
%
 \done
\end{proof}




\section{Cartesian structures from symmetric structures}
\lbl{sec:cart-sym}%
%
\index{monoid!commutative|(}
%

A commutative monoid is a structure in which every finite family of
elements has a well-defined sum.  So if $T = \coprod_{i\in I} [X_i,
\dashbk]$ is any finitary familially representable monad on $\Set$ then any
commutative monoid $A$ is naturally a $T$-algebra via the map
\[
TA = \coprod_{i\in I} [X_i, A] \go A
\]
whose $i$-component sends a family $(a_x)_{x\in X_i}$ to $\sum_{x\in X_i}
a_x$.  

This is the simplest case of the theme of this section: how symmetric
structures give rise to cartesian structures.  We prove two main results.
The first is that if $B$ is any set, $T$ any finitary familially
representable monad on $\Set^B$, and $A$ any commutative monoid, then the
constant family $(A)_{b\in B}$ is naturally a $T$-algebra.  (Just now we
looked at the case $B=1$.)  It follows that any commutative monoid is
naturally a $T_n$-operad, where $T_n$ is the $n$th opetopic
monad~(\ref{sec:opetopes}).  We can also extract a rigorous definition of
the set-with-multiplicities of $n$-opetopes underlying a given
$n$-dimensional opetopic pasting diagram.

The second main result is that any symmetric multicategory $A$ gives rise
to a $T$-multicategory.  Here $T$ is, again, any finitary familially
representable monad on $\Set^B$ for some set $B$, and the object-of-objects
of the induced $T$-multicategory is $(A_0)_{b\in B}$.  So this is like the
result for commutative monoids but one level up; it states that symmetric
multicategories play some kind of universal role for generalized
multicategories, despite (apparently) not being generalized multicategories
themselves.  A corollary is that any symmetric multicategory is naturally a
$T_n$-multicategory for each $n$, previously stated as
Theorem~\ref{thm:sm-opetopic} and pictorially very plausible---see
p.~\pageref{eq:Tn-sym-arrow}.

We start with commutative monoids.  The first main result mentioned above
is reasonably clear informally.  I shall, however, prove it with some care
in preparation for the proof of the second main result, which involves many
of the same thoughts in a more complex setting.  

Notation: for any set $B$ there are adjoint functors
%
\[
\begin{diagram}[scriptlabels]
\Set^B	&\pile{\rTo^\Sigma_\bot\\ \lTo_\Delta}	&\Set,\\
\end{diagram}%
% 
\glo{Sigmacoproduct}
% 
\]
where $\Sigma X = \coprod_{b\in B} Xb$ and $(\Delta A)(b) = A$.  

\begin{thm}	\lbl{thm:cm-main}
Let $B$ be a set and $T$ a finitary familially representable monad on
$\Set^B$.  Then there is a canonical functor
\[
\fcat{CommMon} \go (\Set^B)^T
\]
making the diagram
\[
\begin{diagram}[size=2em]
\fcat{CommMon}		&\rTo		&(\Set^B)^T		\\
\dTo<{\mr{forgetful}}	&
					&
\dTo>{\mr{forgetful}}	\\
\Set			&\rTo_\Delta	&\Set^B			\\
\end{diagram}
\]
commute.
\end{thm}
%
(The standard notation is awkward here: $(\Set^B)^T$ is the category of
algebras for the monad $T$ on the category $\Set^B$.)
%
\begin{proof}
To make the link between commutative monoids and familial representability
we use the fat%
%
\index{monoid!commutative!fat}
%
commutative monoids of~\ref{sec:comm-mons}, which come
equipped with explicit operations for summing arbitrary finite families
(not just finite sequences) of elements.  By Theorem~\ref{thm:fat-cm-eqv},
it suffices to prove the present theorem with `\fcat{CommMon}' replaced by
`\fcat{FatCommMon}'.

Represent the functor $T$ by families $(X_{b,i})_{i\in I(b)}$ ($b\in B$),
the unit $\eta$ of the monad by $e$, and the multiplication $\mu$ by $m$
and a nameless isomorphism, as in the description at the end
of~\ref{sec:fam-rep-pshf}.

For any set $A$ and any $b\in B$ we have
\[
(T\Delta A)(b) 
\iso 
\coprod_{i\in I(b)} (X_{b,i}, \Delta A) 
\iso
\coprod_{i\in I(b)} [\Sigma X_{b,i}, A],
\]
and by Proposition~\ref{propn:fr-pshf-finitary}, the set $\Sigma X_{b,i}$ is
finite.  So given a fat commutative monoid $A$, we have a map
\[
\theta^A_B: (T\Delta A)(b) \go (\Delta A)(b) = A
\]
whose component at $i\in I(b)$ is summation
\[
\sum_{\Sigma X_{b,i}}: [\Sigma X_{b,i}, A] \go A. 
\]
If we can show that $\theta^A$ is a $T$-algebra structure on $\Delta A$
then we are done: the functoriality is trivial.

For the multiplication axiom we have to show that the square
\[
\begin{diagram}[height=2em]
(T^2 \Delta A)(b)	&\rTo^{\mu_{\Delta A, b}}	&(T\Delta A)(b)	\\
\dTo<{(T\theta^A)_b}	&			&\dTo>{\theta^A_b}	\\
(T\Delta A)(b)		&\rTo_{\theta^A_b}		&A		\\
\end{diagram}
\]
commutes, for each $b\in B$.  We have
\[
(T^2 \Delta A)(b) 
\iso 
\coprod_{(i,\gamma)\in J(b)} 
[\Sigma Y_{b,i,\gamma}, A]
\]
where
%
\begin{eqnarray*}
J(b)	&=	&
\{ (i,\gamma) \such i\in I(b), \gamma\in \Set^B(X_{b,i}, I) \},	\\
Y_{b,i,\gamma}	&=	&
\coprod_{c\in B, x\in X_{b,i}(c)} X_{c,\gamma_c(x)}.
\end{eqnarray*}
%
Let $i\in I(b)$ and $\gamma\in \Set^B(X_{b,i}, I)$.  Write $k = m_b(i,\gamma)
\in I$; then part of the data for the monad is an isomorphism $X_{b,k} \goiso
Y_{b,i,\gamma}$.  The clockwise route around the square has
$(i,\gamma)$-component
\[
\begin{diagram}[height=2em]
[\Sigma Y_{b,i,\gamma}, A]	&\rTo^\diso	&[\Sigma X_{b,k}, A]	\\
				&		&
\dTo>{\sum_{\Sigma X_{b,k}}}						\\
				&		&A,			\\
\end{diagram}
\]
and by Lemma~\ref{lemma:fat-cm}, this is just $\sum_{\Sigma
Y_{b,i,\gamma}}$.  For the anticlockwise route, an element of $[\Sigma
Y_{b,i,\gamma}, A]$ is a family $(a_{c,x,d,w})_{c,x,d,w}$ indexed over
\[
c\in B, x\in X_{b,i}(c), d\in B, w\in X_{c, \gamma_c(x)} (d).
\]
For each $c\in B$ and $x\in X_{b,i}(c)$ we therefore have a family
\[
(a_{c,x,d,w})_{d,w} \in [\Sigma X_{x, \gamma_c(x)}, A],
\]
and 
%
\[
\theta^A_c ((a_{c,x,d,w})_{d,w}) = 
\sum_{d,w} a_{c,x,d,w} \in A,
\]
so
\[
(T\theta^A)_b (a_{c,x,d,w})_{c,x,d,w} =
\left( \sum_{d,w} a_{c,x,d,w} \right)_{c,x}
\in
[\Sigma X_{b,i}, A].
\]
So the anticlockwise route sends $(a_{c,x,d,w})_{c,x,d,w}$ to $\sum_{c,x}
\sum_{d,w} a_{c,x,d,w}$, and the clockwise route sends it to
$\sum_{c,x,d,w} a_{c,x,d,w}$; the two are equal.

For the unit axiom we have to show that
\[
\begin{diagram}[height=2em]
(\Delta A)(b)	&\rTo^{\eta_{\Delta A, b}}	&(T\Delta A)(b)		\\
		&\rdTo<1			&\dTo>{\theta^A_b}	\\
		&				&A			\\
\end{diagram}
\]
commutes, for each $b\in B$.  Write $k = e_b\in I$; we know that $X_{b,k}
\iso \delta_b$, and so $\Sigma X_{b,k}$ is a one-element set.  So if $a\in A$
then 
\[
\theta^A_b (\eta_{\Delta A, b} (a))
=
\theta^A_b ((a)_{x\in\Sigma X_{b,k}})
=
\sum_{x\in\Sigma X_{b,k}} a
=
a,
\]
as required.
\done
\end{proof}

\begin{example}	\lbl{eg:cm-ftr-opetopic}
Let $n\in\nat$; take $B$ to be the set $O_{n+1}$ of $(n+1)$-opetopes and
$T$ to be the $(n+1)$th opetopic%
%
\index{opetopic!monad}
%
monad $T_{n+1}$ on $\Set/O_{n+1} \eqv
\Set^{O_{n+1}}$.  By Proposition~\ref{propn:free-refined} and induction,
$T_{n+1}$ is finitary and preserves wide pullbacks, so is familially
representable by Proposition~\ref{propn:eqv-condns-fam-rep-pshf}.
Theorem~\ref{thm:cm-main} then gives a canonical functor
\[
\fcat{CommMon} 
\go 
(\Set/O_{n+1})^{T_{n+1}}
\eqv
T_n \hyph\Operad,
\]
proving Theorem~\ref{thm:cm-ftr-opetopic}.
\end{example}

\begin{example}	\lbl{eg:cm-ftr-Set}
For any finitary familially representable monad $T$ on $\Set$, there is a
canonical functor $\fcat{CommMon} \go \Set^T$ preserving underlying sets.
Most such $T$'s that we have met have been operadic,%
%
\index{monad!operadic}
%
in which case
something stronger is true: there is a canonical functor from the category
of not-necessarily-commutative monoids to $\Set^T$, as noted after
Proposition~\ref{propn:ind-monad-cart}.  But, for instance, if $T$ is the
monad corresponding to the theory of monoids%
%
\index{monoid!anti-involution@with anti-involution}
%
with an
anti-involution (Example~\ref{eg:fam-rep-not-opdc}) then the commutativity is
necessary.  Concretely, Theorem~\ref{thm:cm-main} produces the functor
\[
\begin{array}{rcl}
\fcat{CommMon} 	&\go 	&(\textrm{monoids with an anti-involution}) 	\\
(A,\cdot,1)	&\goesto&(A, \cdot, 1, \blank^\circ)			\\
\end{array}
\]
defined by taking the anti-involution $\blank^\circ$ to be the identity;
that this does define an anti-involution is exactly commutativity.
\end{example}

Write $M$%
% 
\glo{Mfreecommmon}
% 
for the free commutative monoid monad on $\Set$.
Lemma~\ref{lemma:lax-map-mnds-is-ftr} tells us that
Theorem~\ref{thm:cm-main} is equivalent to:
%
\begin{cor}	\lbl{cor:cm-lax-map}
Let $B$ be a set and $T$ a finitary familially representable monad on
$\Set^B$.  Then there is a canonical natural transformation
\[
\begin{diagram}
\Set^B 		&\rTo^T		&\Set^B		\\
\uTo<\Delta	&\sent \psi^T	&\uTo>\Delta	\\
\Set		&\rTo_M		&\Set		\\
\end{diagram}
\]
with the property that $(\Delta, \psi^T)$ is a lax map of monads $(\Set,M)
\go (\Set^B,T)$.  
\done
\end{cor}

When $B=1$, this corollary provides a natural transformation $\psi^T: T \go
M$ commuting with the monad structures.  It is straightforward to check
that if $\alpha: T' \go T$ is a cartesian natural transformation commuting
with the monad structures then $\psi^T \of \alpha = \psi^{T'}$.  So $M$ is
the vertex (codomain) of a cone on the inclusion functor
\[
\begin{array}{rl}
	&(\textrm{finitary familially representable monads on } \Set	\\
	&+ \textrm{ cartesian transformations commuting with the monad
	structures}) 							\\
\rIncl	&(\textrm{monads on } \Set					\\
	&+ \textrm{ transformations commuting with the monad structures}), 
\end{array}
\]
in which the coprojections are the transformations $\psi^T$.  In fact, it
is a colimit cone: the theory of commutative monoids plays a universal role
for finitary familially representable monads on $\Set$, despite not being
familially representable itself.  Since this fact will not be used, I will
not prove it; the main tactic is to consider the free algebraic theory on a
single $n$-ary operation.

As explained on p.~\pageref{p:colax-lax-mate}, we can translate between lax
and colax maps of monads using mates.  Applying this to
Corollary~\ref{cor:cm-lax-map} gives:
%
\begin{cor}	\lbl{cor:cm-colax-map}
Let $B$ be a set and $T$ a finitary familially representable monad on
$\Set^B$.  Then there is a canonical natural transformation
\[
\begin{diagram}
\Set^B 		&\rTo^T		&\Set^B		\\
\dTo<\Sigma	&\swnt \phi^T	&\dTo>\Sigma	\\
\Set		&\rTo_M		&\Set		\\
\end{diagram}
\]
with the property that $(\Sigma, \phi^T)$ is a colax map of monads
$(\Set^B,T) \go (\Set,M)$.
\end{cor}
%
\begin{proof}
Take $\phi^T$ to be the mate of $\psi^T$ under the adjunction $\Sigma \ladj
\Delta$. 
\done
\end{proof}

\begin{example}	\lbl{eg:constituents}
Let $n\in\nat$ and take $B=O_n$ and $T=T_n$,%
%
\index{opetopic!monad}
%
as in
Example~\ref{eg:cm-ftr-opetopic} but with the indexing shifted.  An object
$X$ of $\Set^{O_n} \eqv \Set/O_n$ is thought of as a set of labelled
$n$-opetopes.  An element of $T_n X$ (or rather, of its underlying set
$\Sigma T_n X$) is then an $X$-labelled $n$-pasting diagram;%
%
\index{pasting diagram!opetopic}
%
on the other
hand, an element of $M \Sigma X$ is a finite set-with-multiplicities of
labels (disregarding shapes completely).  So there ought to be a forgetful
function $\Sigma T_n X \go M \Sigma X$, and there is: $\phi^{T_n}_X$.

When $X$ is the terminal object of $\Set^{O_n}$ we have $\Sigma X = O_n$
and $\Sigma T_n X = O_{n+1}$, so $\phi^{T_n}_X$ is a map $O_{n+1} \go M
O_n$.  This sends an $n$-pasting diagram to the set-with-multiplicities%
%
\index{pasting diagram!opetopic!ordering of}
%
of
its constituent $n$-opetopes: for example, the $2$-pasting
diagram~\bref{diag:bigger-two-pd} on p.~\pageref{diag:bigger-two-pd}
(stripped of its labels) is sent to the set-with-multiplicities
\[
\left[ \ 
\topez{}{\Downarrow},\ 
\topea{}{}{\Downarrow},\ 
\topec{}{}{}{}{\Downarrow},\ 
\topec{}{}{}{}{\Downarrow},\ 
\topeeu{\Downarrow}\ 
\right]
\]
of 2-opetopes.
\end{example}%
%
\index{monoid!commutative|)}


\index{multicategory!symmetric vs. generalized@symmetric \vs.\ generalized|(}
%
We now consider symmetric multicategories.  Much of what follows is similar
to what we did for commutative monoids but at a higher level of complexity.

\begin{thm}	\lbl{thm:sm-main}
Let $B$ be a set and $T$ a finitary familially representable monad on
$\Set^B$.  Then there is a canonical functor 
\[
\fcat{FatSymMulticat} \go T\hyph\Multicat%
%
\index{multicategory!symmetric!fat}
%
\]
making the diagram
\[
\begin{diagram}[size=2em]
\fcat{FatSymMulticat}	&\rTo		&T\hyph\Multicat	\\
\dTo<{\blank_0}		&		&\dTo>{\blank_0}	\\
\Set			&\rTo_\Delta	&\Set^B			\\
\end{diagram}
\]
commute, where in both cases $\blank_0$ is the functor assigning to a
multicategory its object of objects.
\end{thm}
%
Before we prove the Theorem let us gather a corollary and some examples.
%
\begin{cor}
Theorem~\ref{thm:sm-main} holds with `\fcat{FatSymMulticat}' replaced by
`\fcat{SymMulticat}' and `canonical' by `canonical up to isomorphism'.  
\end{cor}
%
\begin{proof}
Follows from Theorem~\ref{thm:fat-sm-eqv}.
\done
\end{proof}

\begin{example}	\lbl{eg:sm-opetopic}
For any $n\in\nat$, let $B$ be the set $O_n$ of $n$-opetopes and let $T_n$
be the $n$th opetopic%
%
\index{opetopic!monad}
%
monad.  Then, as observed
in~\ref{eg:cm-ftr-opetopic}, $T_n$ is finitary and familially
representable.  So the Corollary produces a functor
\[
\fcat{SymMulticat} \go T_n\hyph\Multicat,
\]
proving Theorem~\ref{thm:sm-opetopic}. 
\end{example}

\begin{example}
Taking $B=1$, any symmetric multicategory is naturally a $T$-multicategory
for any finitary familially representable monad $T$ on $\Set$.  If $T$ is
operadic%
%
\index{monad!operadic}
%
then we do not need the symmetries: the canonical natural
transformation from $T$ to the free monoid monad induces a functor
\[
\Multicat \go T\hyph\Multicat.
\]
This is analogous to the situation for commutative monoids described
in~\ref{eg:cm-ftr-Set}.
\end{example}

We finish with a proof of Theorem~\ref{thm:sm-main}.  Undeniably it is
complicated, but there is no great conceptual difficulty; the main struggle
is against drowning in notation.  It may help to keep
Example~\ref{eg:sm-opetopic} in mind.

\begin{prooflike}{Proof of Theorem~\ref{thm:sm-main}}
Let $T$ be a finitary familially representable monad on $\Set^B$.  As in
the description just before Proposition~\ref{propn:fr-pshf-finitary},
represent $T$ by families $(X_{b,i})_{i\in I(b)}$ ($b\in B$), the unit
$\eta$ by $e$, and the multiplication $\mu$ by $m$ and a bijection
\[
s_{b,i,\gamma}: 
X_{b, m_b(i, \gamma)}
\goiso
\coprod_{c\in B, x\in X_{b,i}(c)} X_{c, \gamma_c(x)}
\]
for each $b\in B$, $i\in I$, and $\gamma\in \Set^B(X_{b,i}, I)$.  Given
such $b$, $i$, and $\gamma$, and given $d\in B$, write
\[
s_{b,i,\gamma,d}(v) = (c(v), x(v), w(v))
\] 
for any $v\in X_{b, m_b(i, \gamma)}(d)$; so here $c(v) \in B$, $x(v) \in
X_{b,i}(c)$, and $w \in X_{c, \gamma_c(x)}(d)$.

Since $T$ is finitary, Proposition~\ref{propn:fr-pshf-finitary} tells us
that the set $\Sigma X_{b,i}$ is finite for each $b\in B$ and $i\in I(b)$.

Let $P$ be a fat symmetric multicategory.  Our main task is to define from
$P$ a $T$-multicategory $C$.  

Define $C_0 = \Delta P_0$ (as we must).  Then for each $b\in B$ we have
%
\begin{itemize}
\item $C_0(b) = P_0$
\item $(TC_0)(b) = \coprod_{i\in I(b)} [\Sigma X_{b,i}, P_0]$, so an
element of $(TC_0)(b)$ consists of an element $i\in I(b)$ together with a
family $(q_{c,x})_{c,x}$ of objects of $P$ indexed over $c\in B$ and $x\in
X_{b,i}(c)$.
\end{itemize}

Define the $T$-graph
\[
\begin{slopeydiag}
	&		&C_1	&		&	\\
	&\ldTo<\dom	&	&\rdTo>\cod	&	\\
TC_0	&		&	&		&C_0	\\
\end{slopeydiag}
\]
by declaring an element of $C_1(b)$ with domain $(i, (q_{c,x})_{c,x}) \in
(TC_0)(b)$ and codomain $r\in (C_0)(b)$ to be a map
\[
(q_{c,x})_{c,x} \goby{\theta} r
\]
in $P$.

To define $\comp: C_1\of C_1 \go C_1$ we first have to compute $C_1 \of
C_1$.  With a little effort we find that an element of $(C_1 \of C_1)(b)$
consists of the following data (Fig.~\ref{fig:sm-comp-data}):
%
\begin{figure}
\[
\begin{array}{r}
\begin{diagram}[width=2em,height=1.5em,scriptlabels,tight]
   &       &   &       &   &       &C_1\of C_1\Spbk&& &       &   \\
   &       &   &       &   &\ldTo  &      &\rdTo  &   &       &   \\
   &       &   &       &TC_1&      &      &       &C_1&       &   \\
   &       &   &\ldTo  &   &\rdTo  &      &\ldTo  &   &\rdTo  &   \\
   &       &T^2 C_0&   &   &       &TC_0  &       &   &       &C_0\\
   &\ldTo<{\mu_{C_0}}&&&   &       &      &       &   &       &   \\
TC_0&      &   &       &   &       &      &       &   &       &   \\
\end{diagram}
\\
\begin{diagram}[width=2em,height=1.5em,scriptlabels,tight]
   &       &   &       &   &       &((\phi_{c,x})_{c,x},\theta)&&&&\\
   &       &   &       &   &\ldGoesto&    &\rdGoesto& &       &   \\
   &       &   &       &(\phi_{c,x})_{c,x}&&&     &\theta&    &   \\
   &       &   &\ldGoesto& &\rdGoesto&    &\ldGoesto& &\rdGoesto& \\
   &       &(p_{c,x,d,w})_{c,x,d,w}&&&&(q_{c,x})_{c,x}&&&     &r  \\
   &\ldGoesto& &       &   &       &      &       &   &       &   \\
(p_{c(v),x(v),d,w(v)})_{d,v}&&&&&  &      &       &   &       &   \\
\end{diagram}
\end{array}
\]
\caption{Data for composition in $C$.  Each entry in the lower diagram is
an element of the $b$-component of the corresponding object in the upper
diagram.}
\label{fig:sm-comp-data}
\end{figure}
%
\begin{itemize}
\item an element $i\in I(b)$ and a function $\gamma: X_{b,i} \go I$
\item a map $(q_{c,x})_{c,x} \goby{\theta} r$ in $P$
\item a family of maps $((p_{c,x,d,w})_{d,w} \goby{\phi_{c,x}}
q_{c,x})_{c,x}$ in $P$, where the inner indexing is over $d\in B$ and $w\in
X_{c, \gamma_c(x)}(d)$.
\end{itemize}
%
Going down the left-hand slope of the diagram in
Fig.~\ref{fig:sm-comp-data}, the image of this element of $(C_1\of C_1)(b)$
in $(TC_0)(b)$ is the element $m_b(i,\gamma)$ of $I(b)$ together with the
family $(p_{c(v),x(v),d,w(v)})_{d,v}$ indexed over $d\in B$ and $v\in
X_{b,m_b(i,\gamma)}(d)$.  So to define composition, we must derive from our
data a map
%
\begin{equation}	\label{eq:comp-form}
(p_{c(v),x(v),d,w(v)})_{d,v} \go r
\end{equation}
%
in $P$.  But we have the composite
\[
\theta\of (\phi_{c,x})_{c,x}: (p_{c,x,d,w})_{c,x,d,w} \go r
\]
in $P$ and the bijection
% 
\begin{eqnarray*}
\Sigma s_{b,i,\gamma}: %&
\Sigma X_{b, m_b(i, \gamma)}	&
\goiso	&
\Sigma \left(
\coprod_{c\in B, x\in X_{b,i}(c)} X_{c, \gamma_c(x)}	
\right) \\
	%&
(d,v)	&
\goesto	&
(c(v), x(v), d, w(v)),
\end{eqnarray*}
% 
so $(\theta\of (\phi_{c,x})_{c,x}) \cdot (\Sigma s_{b,i,\gamma})$ is a map
of the form~\bref{eq:comp-form}; and this is what we define the composite
to be.  

To define $\ids: C_0 \go C_1$, first note that if $p\in C_0(b) = P_0$ then
$\eta_{C_0,b}(p) \in (TC_0)(b)$ consists of the element $e_b\in I(b)$
together with the family $(p)_{c,u}$ indexed over $c\in B$ and $u\in
U = X_{b,e_b}(c)$, as in the diagrams
\[
\begin{diagram}[width=2em,height=1.5em,scriptlabels,tight]
	&		&C_0	&	&	\\
	&\ldTo<{\eta_{C_0}}&	&\rdTo>1&	\\
TC_0	&		&	&	&C_0	\\
\end{diagram}
%
\diagspace
%
\begin{diagram}[width=2em,height=1.5em,scriptlabels,tight]
	&		&p	&	&	\\
	&\ldGoesto	&	&\rdGoesto&	\\
(p)_{c,u}&		&	&	&p.	\\
\end{diagram}
\]
Since $U$ is a one-element set, we may define $\ids(p)\in C_1(b)$ to be the
identity map
\[
1_p^U: (p)_{c,u} \go p
\]
in $P$, and this has the correct domain and codomain.

We have now defined all the data for the $T$-multicategory $C$.  It only
remains to check the associativity and identity axioms, and these follow
from the associativity and identity axioms on the fat symmetric
multicategory $P$.  So we have defined the desired functor
\[
\fcat{FatSymMulticat} \go T\hyph\Multicat
\]
on objects.

The rest is trivial.  If $f: P \go P'$ is a map of fat symmetric
multicategories and $C$ and $C'$ are the corresponding $T$-multicategories
then we have to define a map $h: C \go C'$.  We take $h_0 = \Delta f_0$ (as
we must) and define $h$ to act on maps as $f$ does.  Since $f$ preserves
all the structure, so does $h$.  Functoriality is immediate.  
\done
\end{prooflike}%
%
\index{multicategory!symmetric vs. generalized@symmetric \vs.\ generalized|)}
%
\index{power of Set@power of $\Set$|)}%
%
\index{Set, power of@$\Set$, power of|)}
%
\index{monad!cartesian|)}
%




\begin{notes}

The main result of Section~\ref{sec:opds-alg-thys}---that the theories
described by plain operads are exactly the strongly regular ones---must
exist as a subconscious principle, at least, in the mind of anyone who has
worked with operads.  Nevertheless, this is as far as I know the first
proof.  A brief sketch proof was given in my~\cite[4.6]{GOM}.

The theory of familially representable monads on presheaf categories
presented here is clearly unsatisfactory.  The proof of
Theorem~\ref{thm:sm-main} is so complicated that it is at the limits of
tolerability, for this author at least.  Moreover, we have not even
attempted to describe explicitly a cartesian monad structure on a
familially representable endofunctor on a general presheaf category
$\ftrcat{\scat{B}^\op}{\Set}$ (where $\scat{B}$ need not be discrete), nor
to describe explicitly algebras for such monads.  So the theory works, but
only just.

Other ideas on the relation between cartesian and symmetric structures can
be found in Weber~\cite{Web}%
%
\index{Weber, Mark}
%
and Batanin~\cite{BatCSO}.%
%
\index{Batanin, Michael}
%



\end{notes}
