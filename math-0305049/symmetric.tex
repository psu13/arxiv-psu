
% \addtocontents{toc}{\contentsline {chapter}{\numberline {}}{}}
\ucontents{part}{Appendices}


\chapter{Symmetric Structures}
\lbl{app:sym}

\chapterquote{%
For my birthday I got a humidifier and a de-humidifier \ldots\  I put them
in the same room and let them fight it out}{%
Steven Wright}


\noindent
Here we meet an alternative definition of symmetric multicategory and
prove it equivalent to the usual one.  This has two purposes.  First,
the alternative definition is in some ways nicer and more natural than the
usual one, avoiding as it does the delicate matter of formulating the
symmetry axioms (\ref{eg:opd-Sym},~\ref{defn:sym-mti}).  Second, it will be
used in Appendix~\ref{app:special-cart} to show that every symmetric
multicategory gives rise to a $T$-multicategory for each $T$ belonging to a
certain large class of cartesian monads.

Actually, we start~(\ref{sec:comm-mons}) with an alternative definition of
commutative monoid.  Although this could hardly be shorter than the
standard definition, it acts as a warm-up to the alternative definition of
symmetric multicategory~(\ref{sec:sym-multis}).



\section{Commutative monoids}
\lbl{sec:comm-mons}%
%
\index{monoid!commutative|(}
%

Here we define `fat commutative monoids' and prove that they are
essentially the same as the ordinary kind.  The idea is to have some direct
way of summing arbitrary finite families $(a_x)_{x\in X}$ of elements of a
commutative monoid $A$, not just ordered sequences $a_1, \ldots, a_n$.
%
\begin{defn}	\lbl{defn:fat-cm}
A \demph{fat commutative monoid}%
%
\index{monoid!commutative!fat}
%
is a set $A$ equipped with a function
$\sum_X: A^X \go A$%
% 
\glo{fatsum}
% 
for each finite set $X$, satisfying the axioms below.
We write elements of $A^X$ as families $(a_x)_{x\in X}$ (where $a_x \in
A$) and $\sum_{X} (a_x)_{x\in X}$ as $\sum_{x\in X} a_x$.  The axioms are:
%
\begin{itemize}
\item for any map $s: X \go Y$ of finite sets and any family $(a_x)_{x\in
X}$ of elements of $A$, 
\[
\sum_{y\in Y} \sum_{x \in s^{-1} \{y\}} a_x
=
\sum_{x\in X} a_x
\]
\item for any one-element set $X$ and any $a\in A$, 
\[
a = \sum_{x\in X} a.
\]
\end{itemize}
%
A \demph{map} $A \go A'$ of fat commutative monoids is a function $f: A \go
A'$ such that for all finite sets $X$, the square
\[
\begin{diagram}[size=2em]
A^X		&\rTo^{f^X}	&A'^X		\\
\dTo<{{\textstyle\Sigma}_X}	&		&\dTo>{{\textstyle\Sigma}_X}\\
A		&\rTo_{f}	&A'		\\
\end{diagram}
\]
commutes.  
\end{defn}
%
Observe the following crucial property immediately:
%
\begin{lemma}	\lbl{lemma:fat-cm}
Let $A$ be a fat commutative monoid, $s: X \go Y$ a bijection between
finite sets, and $(b_y)_{y\in Y}$ an indexed family of elements of $A$.
Then 
\[
\sum_{x\in X} b_{s(x)} = \sum_{y\in Y} b_y.
\]
\end{lemma}
%
\begin{proof}
Define a family $(a_x)_{x\in X}$ by $a_x = b_{s(x)}$.  Then
\[
\sum_{x\in X} a_x 
=
\sum_{y\in Y} \sum_{x\in s^{-1}\{y\}} a_x 
=
\sum_{y\in Y} b_y,
\]
by the first and second axioms respectively.
\done
\end{proof}

This allows us to take the expected liberties with notation.  If, for
example, we have a family of elements $a_{v,w,x}$ indexed over finite sets
$V$, $W$ and $X$, then we may write $\sum_{v\in V, w\in W, x\in X}
a_{v,w,x}$ without ambiguity; the sum could `officially' be interpreted as
either of
\[
\sum_{((v,w),x) \in (V\times W)\times X} a_{v,w,x}
\diagspace
\textrm{or}
\diagspace
\sum_{(v,(w,x)) \in V\times (W\times X)} a_{v,w,x}
\]
(or some further possibility), but these expressions are equal.

Those concerned with foundations might feel uneasy about the idea of
specifying a function $\sum_X: A^X \go A$ `for each finite%
%
\lbl{p:quantification-qualms}
%
set $X$'.  The remedy is to choose a small full subcategory $\scat{F}$ of
the category of finite sets and functions, such that $\scat{F}$ contains at
least one object of each finite cardinality, and interpret `finite set' as
`object of $\scat{F}\,$'; everything works just as well.  In particular,
you might choose to replace the category of finite sets with its skeleton
whose objects are the natural numbers $\mb{n} = \{1, \ldots, n\}$, and this
might seem like a simplifying move, but it can actually make fat
commutative monoids confusing to work with: for instance, whereas
bijections
\[
X \goby{s} Y \goby{t} Z
\]
can be composed in only one possible order, permutations $s, t \in S_n$ can
be composed in two.  I will stick with `all finite sets'.

Write $\fcat{FatCommMon}$%
% 
\glo{FatCommMon}
% 
and $\fcat{CommMon}$ for the categories of fat
and ordinary commutative monoids, respectively.

\begin{thm}	\lbl{thm:fat-cm-eqv}
There is an isomorphism of categories 
\[
\fcat{FatCommMon} \iso
\fcat{CommMon}.
\]
\end{thm}
%
\begin{proof}
We show that both sides are isomorphic to the category $\fcat{UCommMon}$ of
unbiased commutative monoids.  By definition, an \demph{unbiased
commutative monoid}%
%
\index{monoid!commutative!unbiased}
%
is a set $A$ equipped with an $n$-ary addition
operation
\[
\begin{array}{rcl}
A^n			&\go		&A			\\
(a_1, \ldots, a_n)	&\goesto	&(a_1 + \cdots + a_n)
\end{array}
\]
for each $n\in\nat$, satisfying the three axioms displayed in
Example~\ref{eg:opd-terminal} (written there with $\cdot$ instead of $+$
and $x$'s instead of $a$'s).  We have $\fcat{UCommMon} \iso
\fcat{CommMon}$, easily.

Given a fat commutative monoid $(A, \sum)$, define an unbiased
commutative monoid structure $+$ on $A$ by
\[
(a_1 + \cdots + a_n) 
= 
\sum_{x\in \{1, \ldots, n\} } a_x.
\]
The first two axioms for an unbiased commutative monoid follow from the two
axioms for a fat commutative monoid, and the third follows from
Lemma~\ref{lemma:fat-cm}.

Conversely, take an unbiased commutative monoid $(A,+)$ and define a fat
commutative monoid structure $\sum$ on $A$ as follows.  For any finite set
$X$, let $n_X\in\nat$ be the cardinality of $X$ and choose a bijection $t_X:
\{1, \ldots, n_X\} \goiso X$; then define $\sum_X: A^X \go A$ by
\[
\sum_{x\in X} a_x = (a_{t_X(1)} + \cdots + a_{t_X(n_X)}).
\]
By commutativity, this definition is independent of the choice of $t_X$.
Clearly the axioms for a fat commutative monoid are satisfied.

It is straightforward to check that these two processes are mutually
inverse and extend to an isomorphism of categories.
% Lemma~\ref{lemma:fat-cm} is again useful here.  
\done
\end{proof}%
%
\index{monoid!commutative|)}
%


\section{Symmetric multicategories}
\lbl{sec:sym-multis}%
%
\index{multicategory!symmetric|(}
%

As in the previous section, we reformulate a notion of symmetric structure
by moving from finite sequences to finite families: so in a `fat
symmetric multicategory', maps look like
\[
(a_x)_{x\in X} \goby{\theta} b
\]
rather than
\[
a_1, \ldots, a_n \goby{\theta} b.
\]

\begin{defn}	\lbl{defn:fat-sm}
A \demph{fat symmetric multicategory}%
%
\index{multicategory!symmetric!fat}
%
$A$ consists of
%
\begin{itemize}
\item a set $A_0$,%
% 
\glo{A0fat}
% 
whose elements are called the \demph{objects} of $A$
\item for each finite set $X$, family $(a_x)_{x\in X}$ of objects, and
object $b$, a set $C((a_x)_{x\in X}; b)$,%
% 
\glo{fathomset}
% 
whose elements $\theta$ are called
\demph{maps} in $A$ and written
\[
(a_x)_{x\in X} \goby{\theta} b
\]
\item for each function $s: X \go Y$ between finite sets, family
$(a_x)_{x\in X}$ of objects, family $(b_y)_{y\in Y}$ of objects, and object
$c$, a function
\[
A((b_y)_{y\in Y}; c)
\times
\prod_{y\in Y} A((a_x)_{x\in s^{-1}\{y\}}; b_y)
\go
A((a_x)_{x\in X}; c),
\]
called \demph{composition} and written
\[
(\phi, (\theta_y)_{y\in Y}) \goesto \phi \of (\theta_y)_{y\in Y}%
% 
\glo{fatcomp}
% 
\]
\item for each one-element set $X$ and object $a$, an \demph{identity} map
\[
1_a^X \in A((a)_{x\in X}; a),%
% 
\glo{fatids}
% 
\]
\end{itemize}
%
satisfying
%
\begin{itemize}
\item associativity: if $X \goby{s} Y \goby{t} Z$ are functions between
finite sets and 
\[
(a_x)_{x\in s^{-1}\{y\}} \goby{\theta_y} b_y,
\diagspace
(b_y)_{y\in t^{-1}\{z\}} \goby{\phi_z} c_z,
\diagspace
(c_z)_{z\in Z} \goby{\psi} d
\]
are maps in $A$ (for $y\in Y$, $z\in Z$), then
\[
(\psi \of (\phi_z)_{z\in Z}) \of (\theta_y)_{y\in Y}
=
\psi \of (\phi_z \of (\theta_y)_{y\in t^{-1}\{z\}})_{z\in Z}
\]
\item left identity axiom: if $X$ is a finite set, $Y$ a one-element set,
and $\theta: (a_x)_{x\in X} \go b$ a map in $A$, then
\[
1_b^Y \of (\theta)_{y\in Y} = \theta
\]
\item right identity axiom: if $X$ is a finite set and $\theta: (a_x)_{x\in
X} \go b$ a map in $A$, then 
\[
\theta \of (1_{a_x}^{\{x\}})_{x\in X} = \theta.
\]
\end{itemize}
%
A \demph{map} $f: A \go A'$ of fat symmetric multicategories consists of
%
\begin{itemize}
\item a function $f: A_0 \go A'_0$ 
\item for each finite set $X$, family $(a_x)_{x\in X}$ of objects of $A$,
and object $b$ of $A$, a function
\[
A( (a_x)_{x\in X}; b )
\go
A'( (fa_x)_{x\in X}; fb ),
\]
also written as $f$,
\end{itemize}
%
such that
%
\begin{itemize}
\item $f(\phi \of (\theta_y)_{y\in Y}) = f(\phi) \of (f(\theta_y))_{y\in Y}$
whenever these composites make sense
\item $f(1_a^X) = 1_{fa}^X$ whenever $a$ is an object of $A$ and $X$ is a
one-element set.   
\end{itemize}
%
This defines a category $\fcat{FatSymMulticat}$.%
% 
\glo{FatSymMulticat}
% 
\end{defn}


As promised, this definition avoids the delicate symmetry axioms present in
the traditional version.  The following lemma, analogous to
Lemma~\ref{lemma:fat-cm}, shows that the symmetric group actions really are
hiding in there.
%
\begin{lemma}	\lbl{lemma:fat-sm}
Let $A$ be a fat symmetric multicategory.  Then any bijection $s: X \go Y$
between finite sets and map $\phi: (b_y)_{y\in Y} \go c$ in $A$ give rise
to a map $\phi\cdot s:%
% 
\glo{fatcdot}
% 
(b_{s(x)})_{x\in X} \go c$ in $A$.  This
construction satisfies
\[
\phi\cdot (s\of r) = (\phi\cdot s)\cdot r,
\diagspace
\theta \cdot 1_Y = \theta,
\]
where $W \goby{r} X \goby{s} Y$ in the first equation.  Moreover, if $f$ is a
map of fat symmetric multicategories then $f(\phi\cdot s) = f(\phi)\cdot s$
whenever these expressions make sense.  
\end{lemma}
%
\begin{proof}
Take $s$ and $\phi$ as in the statement.  Define a family $(a_x)_{x\in X}$
by $a_x = b_{s(x)}$.  For each $y\in Y$ we have the map
\[
1_{b_y}^{s^{-1}\{y\}}: 
(b_y)_{x\in s^{-1}\{y\}} \go b_y;
\]
but $b_y = a_{s^{-1}(y)}$, so the domain of this map is $(a_x)_{x\in
s^{-1}\{y\}}$.  We may therefore define
\[
\phi\cdot s = 
\phi\of \left( 1_{b_y}^{s^{-1}\{y\}} \right)_{y\in Y}:
(a_x)_{x\in X} \go c.
\]
The two equations follow from the associativity and identity axioms
for a fat symmetric multicategory, respectively.  That maps of fat
symmetric multicategories preserve $\cdot$ is immediate from the
definitions. 
\done
\end{proof}
%
This lemma is very useful in the proof below that fat and ordinary
symmetric multicategories are essentially the same.  Like
Lemma~\ref{lemma:fat-cm}, it also allows us to take notational liberties.
If, for example, we have a family of objects $a_{v,w,x}$ indexed over
finite sets $V$, $W$ and $X$, then we may safely speak of `maps
\[
(a_{v,w,x})_{v\in V, w\in W, x\in X} \go b
\]
in $A$'; it does not matter whether the indexing set in the domain is
meant to be $(V\times W)\times X$ or $V\times (W\times X)$ (or some other
3-fold product), as the canonical isomorphism between them induces a
canonical isomorphism of hom-sets,
\[
A((a_{v,w,x})_{((v,w),x) \in (V\times W) \times X} ; b)
\goiso
A((a_{v,w,x})_{(v,(w,x)) \in V\times (W \times X)} ; b).
\]

\begin{example}
A \demph{fat symmetric operad}%
%
\index{operad!symmetric!fat}
%
$P$ is, of course, a fat symmetric
multicategory with only one object; it consists of a set $P(X)$ for each
finite set $X$, together with composition and identity operations.  

Many well-known examples of symmetric operads are naturally regarded as fat
symmetric operads.  For instance, there is a fat symmetric operad $\ldisks$
where an element of $\ldisks(X)$ is an $X$-indexed family $(\alpha_x)_{x\in
X}$ of disjoint little%
%
\index{operad!little disks}
%
disks inside the unit disk
(compare~\ref{eg:opd-little-disks}).  Or, for any set $S$ there is a fat
symmetric operad $\END(S)$%
%
\index{operad!endomorphism}%
%
\index{endomorphism!symmetric operad}
%
defined by $ (\END(S))(X) = \Set(S^X, S)$
(compare~\ref{eg:opd-End}).  Or, there is a fat symmetric operad $O$ in
which $O(X)$ is the set of total orders%
%
\index{order!operad of orders}\index{operad!orders@of orders}
%
 on $X$, with composition done
lexicographically; under the equivalence we are about to establish, it
corresponds to the ordinary operad $\SymOpd$ of
symmetries.%
%
\index{operad!symmetries@of symmetries}
%
(This last example appeared in Beilinson%
%
\index{Beilinson, Alexander}
%
and Drinfeld%
%
\index{Drinfeld, Vladimir}
%
\cite[1.1.4]{BeDr}
and~\ref{eg:opd-Sym} above.)
\end{example}

\begin{thm}	\lbl{thm:fat-sm-eqv}
There is a canonical equivalence of categories
\[
\ovln{\blank}: \fcat{FatSymMulticat} \goby{\eqv} \fcat{SymMulticat}.
\]
\end{thm}

\begin{proof}
We define the functor $\ovln{\blank}$ and show that it is full, faithful
and (genuinely) surjective on objects.  Details are omitted.

To define $\ovln{\blank}$ on objects, take a fat symmetric multicategory
$A$.  The symmetric multicategory $\ovln{A}$ has the same objects as $A$
and hom-sets 
\[
\ovln{A} (a_1, \ldots, a_n; b) 
=
A((a_x)_{x\in [1,n]}; b),
\]
where for $m,n\in \nat$ we write 
\[
[m,n] = \{ l \in \nat \such m \leq l \leq n \}.
\]
For composition, take maps
\[
\begin{array}{c}
a_1^1, \ldots, a_1^{k_1} \goby{\theta_1} b_1,
\diagspace \ldots, \diagspace 
a_n^1, \ldots, a_n^{k_n} \goby{\theta_n} b_n,	\\
b_1, \ldots, b_n \goby{\phi} c
\end{array}
\]
in $\ovln{A}$.  Define objects $a_1, \ldots, a_{k_1 + \cdots + k_n}$ by the
equation of formal sequences
\[
(a_1, \ldots, a_{k_1 + \cdots + k_n}) 
=
(a_1^1, \ldots, a_1^{k_1}, \ldots, a_n^1, \ldots, a_n^{k_n}).
\]
For each $x\in [1,n]$ there is an obvious bijection
\[
t_x: 
[k_1 + \cdots + k_{x-1} + 1, k_1 + \cdots + k_{x-1} + k_x]
\goiso
[1, k_x]
\]
defined by subtraction: so the map 
\[
(a_{k_1 + \cdots + k_{x-1} + y})_{y\in [1,k_x]}
=
(a_x^y)_{y\in [1,k_n]}
\goby{\theta_x}
b_x
\]
in $A$ gives rise to a map
\[
(a_z)_{z\in [k_1 + \cdots + k_{x-1} + 1, k_1 + \cdots + k_{x-1} + k_x]}
\goby{\theta_x \cdot t_x}
b_x
\]
in $A$, by Lemma~\ref{lemma:fat-sm}.  It now makes sense to define
composition in $\ovln{A}$ by
\[
\phi \of (\theta_1, \ldots, \theta_n)
=
\phi \of (\theta_x \cdot t_x)_{x\in [1,n]},
\]
since the domain of this map is $(a_z)_{z\in [1, k_1 + \cdots + k_n]}$.
Identities in $\ovln{A}$ are easier: for $a\in A$, put $1_a =
1_a^{[1,1]}$.  The structure $\ovln{A}$ just defined really is a symmetric
multicategory, as is straightforward to prove with the aid of
Lemma~\ref{lemma:fat-sm}.  

The definition of the functor $\ovln{\blank}$ on morphisms and the proof
of functoriality are also straightforward.

Now we show that $\ovln{\blank}$ is surjective on objects.  For each finite
set $X$, choose a bijection $s_X: [1,n_X] \go X$, where $n_X =
\mr{card}(X)$.  In the case that $X=[m+1,m+n]$ for some $m,n\in\nat$,
choose $s_X$ to be the obvious bijection (add $m$).  Let $C$ be a symmetric
multicategory; our task is to define a fat symmetric multicategory $A$ such
that $\ovln{A} = C$.  (This is the uphill direction and is bound to require
more work.)  We define the objects of $A$ to be the objects of $C$.  If
$(a_x)_{x\in X}$ is a finite family of objects then we define
\[
A((a_x)_{x\in X}; b)
=
C(a_{s_X(1)}, \ldots, a_{s_X(n_X)}; b).
\]
If $X$ is a one-element set and $a$ an object then we put
\[
1_a^X = 1_a \in C(a;a) = A((a)_{x\in X}; a).
\]
The definition of composition is clear in principle but fiddly in practice,
hence omitted.  The idea is that a composite in $A$ can almost be defined
as a composite in $C$, but because the bijections $s_X$ were chosen at
random, we have to apply a symmetry after composing in $C$---the unique
symmetry that makes the domain come out right.  The axioms for the fat
symmetric multicategory $A$ then follow, with some effort, from the
ordinary symmetric multicategory axioms on $C$.

We also have to show that $\ovln{A} = C$.  Certainly their object-sets are
equal, and their hom-sets are equal because
\[
\ovln{A}(a_1, \ldots, a_n; b)
=
A((a_x)_{x\in [1,n]}; b)
=
C(a_{s_{[1,n]}(1)}, \ldots, a_{s_{[1,n]}(n)}; b)
\]
and we chose $s_{[1,n]}$ to be the identity.  Composition in $\ovln{A}$ was
defined using the obvious bijections
\[
[m+1, m+k] \goiso [1,k]
\]
for certain values of $m$ and $k$, and to show that it coincides with
composition in $A$ we use the fact that $s_{[m+1,m+k]}$ was also chosen to
be the obvious bijection.

Next, $\ovln{\blank}$ is full.  Let $A$ and $A'$ be fat symmetric
multicategories and $h: \ovln{A} \go \ovln{A'}$ a map of ordinary symmetric
multicategories; we define a map $f: A \go A'$ such that $\ovln{f} = h$.
On objects, $f(a) = h(a)$.  Given a map 
$
\theta: (a_x)_{x\in X} \go b
$
in $A$, choose a bijection $s: [1,n] \go X$, where $n=\mr{card}(X)$.  Then
we have the map
\[
\theta\cdot s: (a_{s(y)})_{y\in [1,n]} \go b
\]
in $A$, that is, we have 
\[
\theta\cdot s: a_{s(1)}, \ldots, a_{s(n)} \go b
\]
in $\ovln{A}$; so we obtain the map
\[
h(\theta\cdot s): fa_{s(1)}, \ldots, fa_{s(n)} \go fb
\]
in $\ovln{A'}$, that is, 
\[
h(\theta\cdot s): (fa_{s(y)})_{y\in [1,n]} \go fb
\]
in $A'$.  It therefore makes sense to define
\[
f(\theta) = h(\theta\cdot s) \cdot s^{-1}: (fa_x)_{x\in X} \go fb.
\]
This definition is independent of the choice of $s$: note that any other
choice is of the form $s\of \sigma$ for some $\sigma\in S_n$, then use
Lemma~\ref{lemma:fat-sm} and the fact that $h$ preserves symmetric group
actions.  It is straightforward to check that $f$ really is a map of fat
symmetric multicategories and that $\ovln{f} = h$.  

Finally, $\ovln{\blank}$ is faithful.  Let $A \parpair{f}{g} A'$ be a pair
of maps of fat symmetric multicategories satisfying $\ovln{f} = \ovln{g}$.
Certainly $f$ and $g$ agree on objects.  Given $\theta: (a_x)_{x\in X} \go
b$ in $A$, choose a bijection $s: [1,n] \go X$, where $n=\mr{card}(X)$;
then we have a map 
\[
\theta\cdot s: a_{s(1)}, \ldots, a_{s(n)} \go b
\]
in $\ovln{A}$.  Using Lemma~\ref{lemma:fat-sm},
\[
f(\theta)
=
f(\theta\cdot s \cdot s^{-1})
=
f(\theta\cdot s) \cdot s^{-1}
=
\ovln{f}(\theta\cdot s) \cdot s^{-1},
\]
and similarly $g(\theta) = \ovln{g}(\theta\cdot s) \cdot s^{-1}$, so
$f(\theta) = g(\theta)$. 
\done
\end{proof}%
%
\index{multicategory!symmetric|)}
%


\begin{notes}

Something very close to the notion of fat symmetric multicategories
appeared in Beilinson%
%
\index{Beilinson, Alexander}
%
and Drinfeld%
%
\index{Drinfeld, Vladimir}
%
\cite[\S 1.1]{BeDr}, under the name of
`pseudo-tensor%
%
\index{pseudo-tensor category}%
%
\index{category!pseudo-tensor}
%
categories'.  The idea has probably appeared elsewhere too.
Beilinson and Drinfeld insisted that the domain $(a_x)_{x\in X}$ of a map
should be a \emph{non-empty} finite family of objects, and correspondingly
that the functions called $s: X\go Y$ in Definition~\ref{defn:fat-sm}
should be surjective.  This amounts to excluding the possibility of
nullary%
%
\index{nullary!arrow}
%
maps, which we have no reason to do.  They also handled one-member families
slightly differently.







\end{notes}
