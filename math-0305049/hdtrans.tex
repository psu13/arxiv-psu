
% hdtrans.tex
% 
% This is one of the files in the new, modular, hdpics system.
% Created on 18/6/2002.
%
% Files:
% hdpics.tex		Just the kernel.  Always necessary.
% hdglob.tex		Globular pics, large and small
% hdglobrare.tex	Less commonly used globular pics, including
% 			the ones with multiply-shafted arrows
% hdcell.tex		Cellular pics
% hdcellrare.tex	Less commonly used globular pics, including
% 			the ones with multiply-shafted arrows
% hdope.tex		Pics of opetopes
% hdtrans.tex		Transistors 




%%%%%%%%%%%%%%%%%%%%%%%%%%%%%%%%%%%%%%%%%%%%%%%%%%%%%%%%%%%%%%%%%%%%%%%%%%%
% 
% LITTLE COMPONENTS
% 

\newcommand{\tinputabv}[1]{%
\begin{picture}(1,0.4)(0,0)
% label
\cell{0.6}{0.4}{br}{#1}%
% dash
\put(0,0){\line(1,0){1}}
\end{picture}}

\newcommand{\tinputblw}[1]{%
\begin{picture}(1,0.4)(0,-0.4)
% label
\cell{0.6}{-0.4}{tr}{#1}%
% dash
\put(0,0){\line(1,0){1}}
\end{picture}}

\newcommand{\tinputlft}[1]{%
\begin{picture}(1,0)(0,0)
% label
\cell{-0.4}{0}{r}{#1}%
% dash
\put(0,0){\line(1,0){1}}
\end{picture}}

\newcommand{\tinputslft}[2]{%
\begin{picture}(1.4,3.8)(-0.4,-1.9)
% inputs
\cell{0}{1.5}{l}{\tinputlft{#1}}%
\cell{0}{-1.5}{l}{\tinputlft{#2}}%
% ellipsis
\cell{0.2}{0.3}{c}{\vdots}%
\end{picture}}

\newcommand{\tinputsslft}[3]{%
\begin{picture}(1.4,3.8)(-0.4,-1.9)
% inputs
\cell{0}{1.5}{l}{\tinputlft{#1}}%
\cell{0}{0.7}{l}{\tinputlft{#2}}%
\cell{0}{-1.5}{l}{\tinputlft{#3}}%
% ellipsis
\cell{0.2}{-0.2}{c}{\vdots}%
\end{picture}}

\newcommand{\tinputs}[2]{%
\begin{picture}(1,3.8)(0,-1.9)
% inputs
\cell{0}{1.5}{l}{\tinputabv{#1}}%
\cell{0}{-1.5}{l}{\tinputblw{#2}}%
% ellipsis
\cell{0.2}{0.2}{c}{\vdots}%
\end{picture}}

\newcommand{\toutputabv}[1]{%
\begin{picture}(1,0.4)(-1,0)
% label
\cell{-0.6}{0.4}{bl}{#1}%
% dash
\put(0,0){\line(-1,0){1}}%
\end{picture}}

\newcommand{\toutputrgt}[1]{%
\begin{picture}(1,0)(-1,0)
% label
\cell{0.4}{0}{l}{#1}%
% dash
\put(0,0){\line(-1,0){1}}%
\end{picture}}

\newcommand{\toutputwide}[1]{%
\begin{picture}(4,0.4)(-4,0)
% label
\cell{-3.6}{0.4}{bl}{#1}%
% dash
\put(0,0){\line(-1,0){4}}%
\end{picture}}

\newcommand{\tinputslftsmall}[2]{%
\begin{picture}(1.4,3.2)(-0.4,-1.6)
% inputs
\cell{0}{1.1}{l}{\tinputlft{#1}}%
\cell{0}{-1.1}{l}{\tinputlft{#2}}%
% ellipsis
\cell{0.2}{0.3}{c}{\vdots}%
\end{picture}}

\newcommand{\tinputvert}[1]{%
\begin{picture}(0,1)(0,-1)
% label
\cell{0}{0.4}{b}{#1}%
% dash
\put(0,0){\line(0,-1){1}}
\end{picture}}

\newcommand{\tinputsvert}[2]{%
\begin{picture}(3.8,1)(-1.9,0)
% inputs
\cell{-1.5}{0}{b}{\tinputvert{#1}}
\cell{1.5}{0}{b}{\tinputvert{#2}}
% ellipsis
\cell{0}{0.8}{c}{\cdots}
\end{picture}}

\newcommand{\tinputssmallvert}[2]{%
\begin{picture}(3.2,1)(-1.6,0)
% inputs
\cell{-1.1}{0}{b}{\tinputvert{#1}}
\cell{1.1}{0}{b}{\tinputvert{#2}}
% ellipsis
\cell{0}{0.8}{c}{\cdots}
\end{picture}}

\newcommand{\tinputssvert}[3]{%
\begin{picture}(3.8,1)(-1.9,0)
% inputs
\cell{-1.5}{0}{b}{\tinputvert{#1}}
\cell{-0.6}{0}{b}{\tinputvert{#2}}
\cell{1.5}{0}{b}{\tinputvert{#3}}
% ellipsis
\cell{0.45}{0.6}{c}{\cdots}
\end{picture}}

\newcommand{\toutputvert}[1]{%
\begin{picture}(0,1)(0,0)
% label
\cell{0}{-0.4}{t}{#1}%
% dash
\put(0,0){\line(0,1){1}}
\end{picture}}



%%%%%%%%%%%%%%%%%%%%%%%%%%%%%%%%%%%%%%%%%%%%%%%%%%%%%%%%%%%%%%%%%%%%%%%%%%%
% 
% TRANSISTORS	
% 

% Warning!  I seem to have abandoned this chunk in favour of
% something `more modular' (below)

% \newcommand{\tusualmid}[4]{%
% \begin{picture}(6,4)(-3,-2)
% % centre label
% \cell{-0.2}{0}{c}{#1}%
% % inputs and output
% \cell{-2}{0}{r}{\tinputs{#2}{#3}}%
% \cell{2}{0}{bl}{\toutputabv{#4}}%
% % triangle
% \put(-2,-2){\line(0,1){4}}%
% \put(-2,2){\line(2,-1){4}}%
% \put(2,0){\line(-2,-1){4}}%
% \end{picture}}
% 
% 
% \newcommand{\tusualfst}[4]{%
% \begin{picture}(6,4)(-3,-2)
% % centre label
% \cell{-0.2}{0}{c}{#1}%
% % inputs and output
% \cell{-2}{0}{r}{\tinputslft{#2}{#3}}%
% \cell{2}{0}{bl}{\toutputabv{#4}}%
% % triangle
% \put(-2,-2){\line(0,1){4}}%
% \put(-2,2){\line(2,-1){4}}%
% \put(2,0){\line(-2,-1){4}}%
% \end{picture}}
% 
% 
% \newcommand{\tusuallst}[4]{%
% \begin{picture}(6,4)(-3,-2)
% % centre label
% \cell{-0.2}{0}{c}{#1}%
% % inputs and output
% \cell{-2}{0}{r}{\tinputs{#2}{#3}}%
% \cell{2}{0}{l}{\toutputrgt{#4}}%
% % triangle
% \put(-2,-2){\line(0,1){4}}%
% \put(-2,2){\line(2,-1){4}}%
% \put(2,0){\line(-2,-1){4}}%
% \end{picture}}
% 
% 
% \newcommand{\tusualdot}[4]{%
% \begin{picture}(6,4)(-3,-2)
% % centre label
% \cell{-0.2}{0}{c}{#1}%
% % inputs and output
% \cell{-2}{0}{r}{\tinputs{#2}{#3}}%
% \cell{2}{0}{l}{\toutputrgt{#4}}%
% % dotted triangle
% \qbezier[20](-2,-2)(-2,0)(-2,2)%
% \qbezier[22](-2,2)(0,1)(2,0)%
% \qbezier[22](2,0)(0,-1)(-2,-2)%
% \end{picture}}
% 
% 
% \newcommand{\tusualaln}[4]{%
% \begin{picture}(6,4)(-3,-2)
% % centre label
% \cell{-0.2}{0}{c}{#1}%
% % inputs and output
% \cell{-2}{0}{r}{\tinputslft{#2}{#3}}%
% \cell{2}{0}{l}{\toutputrgt{#4}}%
% % triangle
% \put(-2,-2){\line(0,1){4}}%
% \put(-2,2){\line(2,-1){4}}%
% \put(2,0){\line(-2,-1){4}}%
% \end{picture}}
% 
% 
% \newcommand{\tunarymid}[3]{%
% \begin{picture}(6,4)(-3,-2)
% % centre label
% \cell{-0.2}{0}{c}{#1}%
% % inputs and output
% \cell{-2}{0}{br}{\tinputabv{#2}}%
% \cell{2}{0}{bl}{\toutputabv{#3}}%
% % triangle
% \put(-2,-2){\line(0,1){4}}%
% \put(-2,2){\line(2,-1){4}}%
% \put(2,0){\line(-2,-1){4}}%
% \end{picture}}
% 
% 
% \newcommand{\tunaryfst}[3]{%
% \begin{picture}(6,4)(-3,-2)
% % centre label
% \cell{-0.2}{0}{c}{#1}%
% % inputs and output
% \cell{-2}{0}{r}{\tinputlft{#2}}%
% \cell{2}{0}{bl}{\toutputabv{#3}}%
% % triangle
% \put(-2,-2){\line(0,1){4}}%
% \put(-2,2){\line(2,-1){4}}%
% \put(2,0){\line(-2,-1){4}}%
% \end{picture}}
% 
% 
% \newcommand{\tunarylst}[3]{%
% \begin{picture}(6,4)(-3,-2)
% % centre label
% \cell{-0.2}{0}{c}{#1}%
% % inputs and output
% \cell{-2}{0}{br}{\tinputabv{#2}}%
% \cell{2}{0}{l}{\toutputrgt{#3}}%
% % triangle
% \put(-2,-2){\line(0,1){4}}%
% \put(-2,2){\line(2,-1){4}}%
% \put(2,0){\line(-2,-1){4}}%
% \end{picture}}
% 
% 
% \newcommand{\tunarydot}[3]{%
% \begin{picture}(6,4)(-3,-2)
% % centre label
% \cell{-0.2}{0}{c}{#1}%
% % inputs and output
% \cell{-2}{0}{br}{\tinputabv{#2}}%
% \cell{2}{0}{l}{\toutputrgt{#3}}%
% % dotted triangle
% \qbezier[20](-2,-2)(-2,0)(-2,2)%
% \qbezier[22](-2,2)(0,1)(2,0)%
% \qbezier[22](2,0)(0,-1)(-2,-2)%
% \end{picture}}
% 
% 
% \newcommand{\tunaryaln}[3]{%
% \begin{picture}(6,4)(-3,-2)
% % centre label
% \cell{-0.2}{0}{c}{#1}%
% % inputs and output
% \cell{-2}{0}{r}{\tinputlft{#2}}%
% \cell{2}{0}{l}{\toutputrgt{#3}}%
% % triangle
% \put(-2,-2){\line(0,1){4}}%
% \put(-2,2){\line(2,-1){4}}%
% \put(2,0){\line(-2,-1){4}}%
% \end{picture}}


% More modular way of doing it


\newcommand{\tusual}[1]{%
\begin{picture}(4,4)(-2,-2)
% centre label
\cell{-0.2}{0}{c}{#1}%
% triangle
\put(-2,-2){\line(0,1){4}}%
\put(-2,2){\line(2,-1){4}}%
\put(2,0){\line(-2,-1){4}}%
\end{picture}}


\newcommand{\tsmall}[1]{%
\begin{picture}(3,3)(-1.5,-1.5)
% centre label
\cell{-0.2}{0}{c}{#1}%
% triangle
\put(-1.5,-1.5){\line(0,1){3}}%
\put(-1.5,1.5){\line(2,-1){3}}%
\put(1.5,0){\line(-2,-1){3}}%
\end{picture}}


\newcommand{\twide}[1]{%
\begin{picture}(8,4)(-4,-2)
% centre label
\cell{-0.4}{0}{c}{#1}%
% triangle
\put(-4,-2){\line(0,1){4}}%
\put(-4,2){\line(4,-1){8}}%
\put(4,0){\line(-4,-1){8}}%
\end{picture}}


\newcommand{\tbig}[1]{%
\begin{picture}(8,8)(-4,-4)
% centre label
\cell{-0.2}{0}{c}{#1}%
% triangle
\put(-4,-4){\line(0,1){8}}%
\put(-4,4){\line(2,-1){8}}%
\put(4,0){\line(-2,-1){8}}%
\end{picture}}


\newcommand{\tmid}[1]{%
\begin{picture}(6,6)(-3,-3)
% centre label
\cell{-0.2}{0}{c}{#1}%
% triangle
\put(-3,-3){\line(0,1){6}}%
\put(-3,3){\line(2,-1){6}}%
\put(3,0){\line(-2,-1){6}}%
\end{picture}}

\newcommand{\tusualvert}[1]{%
\begin{picture}(4,4)(-2,-2)
% centre label
\cell{0}{0.75}{c}{#1}%
% triangle
\put(-2,2){\line(1,0){4}}%
\put(-2,2){\line(1,-2){2}}%
\put(2,2){\line(-1,-2){2}}%
\end{picture}}

\newcommand{\tsmallvert}[1]{%
\begin{picture}(3,3)(-1.5,-1.5)
% centre label
\cell{0}{0.5}{c}{#1}%
% triangle
\put(-1.5,1.5){\line(1,0){3}}%
\put(-1.5,1.5){\line(1,-2){1.5}}%
\put(1.5,1.5){\line(-1,-2){1.5}}%
\end{picture}}

\newcommand{\twiggleleft}{%
\begin{picture}(0.5,0.8)(-0.5,0)
\qbezier(0,0)(-0.5,0.4)(0,0.8)
\end{picture}}

\newcommand{\twiggleright}{%
\begin{picture}(0.5,0.8)(0,0)
\qbezier(0,0)(0.5,0.4)(0,0.8)
\end{picture}}

\newcommand{\twiggly}[1]{%
\begin{picture}(4.5,4)(0,-2)
% transistor
\put(4.5,0){\line(-2,-1){4}}
\put(4.5,0){\line(-2,1){4}}
\cell{0.5}{2}{tr}{\twiggleleft}
\cell{0.5}{1.2}{tl}{\twiggleright}
\cell{0.5}{0.4}{tr}{\twiggleleft}
\cell{0.5}{-0.4}{tl}{\twiggleright}
\cell{0.5}{-1.2}{tr}{\twiggleleft}
\cell{2.5}{0}{c}{#1}
\end{picture}}


\newcommand{\tusualdotty}[1]{%
\begin{picture}(4,4)(-2,-2)
% centre label
\cell{-0.2}{0}{c}{#1}%
% triangle
\qbezier[20](-2,-2)(-2,0)(-2,2)
\qbezier[24](-2,2)(0,1)(2,0)
\qbezier[24](2,0)(0,-1)(-2,-2)
\end{picture}}


\newcommand{\tsmalldotty}[1]{%
\begin{picture}(3,3)(-1.5,-1.5)
% centre label
\cell{-0.2}{0}{c}{#1}%
% triangle
\qbezier[15](-1.5,-1.5)(-1.5,0)(-1.5,1.5)
\qbezier[18](-1.5,1.5)(0,0.75)(1.5,0)
\qbezier[18](1.5,0)(0,-0.75)(-1.5,-1.5)
\end{picture}}


% % Joining curve
% 
% \newlength{\hdtemptwo}
% \newlength{\hdtempthree}
% 
% \newcommand{\sweep}[4]{%
% \setlength{\hdtemp}{0.5
% 
% Argh!  There doesn't seem to be a way of doing non-integer arithmetic in
% Latex! 
% 

% Labelling curves

\newcommand{\tnelabel}[1]{%
\begin{picture}(1.2,1.5)(0,0)%
\cell{0}{0}{bl}{\qbezier(0,0)(0.1,1.3)(1.0,1.4)}
\cell{0}{0}{c}{\scriptstyle\bullet}
\cell{1.2}{1.5}{l}{#1}
\end{picture}}


\newcommand{\tselabel}[1]{%
\begin{picture}(1.2,1.5)(0,-1.5)%
\cell{0}{0}{tl}{\qbezier(0,0)(0.1,-1.3)(1.0,-1.4)}
\cell{0}{0}{c}{\scriptstyle\bullet}
\cell{1.2}{-1.5}{l}{#1}
\end{picture}}

% \cell{3.8}{-4}{tl}{\qbezier(0,0)(0.1,-1.3)(1.0,-1.4)}
% \cell{3.8}{-4}{c}{\scriptstyle\bullet}
