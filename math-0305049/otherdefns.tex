
\chapter{Other Definitions of Weak $n$-Category}
\lbl{ch:other-defns}

\chapterquote{%
Zounds!  I was never so bethump'd with words!}{%
Shakespeare~\cite{Sha}}

%
\index{n-category@$n$-category!definitions of|(}
%
\noindent
The definition of weak $n$-category studied in the previous chapter is, of
course, just one of a host of proposed definitions.  Ten of them were
described in my~\cite{SDN} survey, all except one in formal, precise terms.
However, the format of that paper did not allow for serious discussion of
the interrelationships, and one might get the impression from it that the
ten definitions embodied eight or so completely different approaches to the
subject.

I hope to correct that impression here.  Fundamentally, there seem to be
only two approaches.

In the first, a weak $n$-category is regarded as a presheaf%
%
\index{presheaf!structure@with structure}
%
with
\emph{structure}.%
%
\index{structure vs. properties@structure \vs.\ properties}%
%
\index{properties vs. structure@properties \vs.\ structure}
%
Usually `presheaf' means $n$-globular set,
and `structure' means $S$-algebra structure for some monad $S$, often
coming from a globular operad.  The definition studied in the previous
chapter is of this type.  

In the second approach, a weak $n$-category is regarded as a presheaf with
\emph{properties}.%
%
\index{presheaf!properties@with properties}
%  
 There is no hope that a weak $n$-category could be
defined as an $n$-globular set with properties, so the category
$\scat{C}_n$ on which we are taking presheaves must be larger than
$\scat{G}_n$; presheaves on $\scat{C}_n$ must somehow have composition
built in.  The case $n=1$ makes this clear.  In the first approach, a
category is defined as a directed graph (presheaf on $\scat{G}_1$) with
structure.  In the second, a category cannot be defined a category as a
presheaf-with-properties on $\scat{G}_1$, but it can be defined as a
presheaf-with-properties on the larger category $\Delta$:%
%
\index{simplex category $\Delta$}
%
this is the
standard characterization of a category by its
nerve~(p.~\pageref{p:defn-nerve}).%
%
\index{nerve!category@of category}
%
 

There are other descriptions of the difference between the two approaches.
In the sense of the introduction to Chapter~\ref{ch:monoidal}, the first is
algebraic%
%
\index{algebraic theory!n-categories@of $n$-categories}
%
(the various types of composition in a weak $n$-category are
\emph{bona fide} operations) and the second is non-algebraic (composites
are not determined uniquely, only up to equivalence).  Or, the first
approach can be summarized as `take the theory of strict $n$-categories,
weaken%
%
\index{weakening!theory of n-categories@for theory of $n$-categories}
%
it, then take models for the weakened theory', and the second as
`take weak models for the theory of strict $n$-categories'.%
%
\index{n-category@$n$-category!definitions of|)}
%

The definitions following the two approaches are discussed
in~\ref{sec:alg-defns-n-cat} and~\ref{sec:non-alg-defns-n-cat}
respectively.  What I have chosen to say about each definition (and which
definitions I have chosen to say anything about at all) is dictated by how
much I feel capable of saying in a simple and not too technical way; the
emphasis is therefore rather uneven.  In particular, there is more on the
definitions close to that of the previous chapter than on those further
away.  





\section{Algebraic definitions}
\lbl{sec:alg-defns-n-cat}%
%
\index{algebraic theory!n-categories@of $n$-categories|(}
%



Here we discuss the definitions proposed by Batanin, Penon, Trimble, and
May.  We continue to use the notation of the previous chapter: $T$ is the
free strict $\omega$-category monad, $\pd = T1$, and so on.


\minihead{Batanin's definition}%
%
\index{Batanin, Michael!definition of n-category@definition of $n$-category|(}%
%
\index{n-category@$n$-category!definitions of!Batanin's|(}
%
%

The definition of weak $\omega$-category studied in the previous chapter is
a simplification of Batanin's~\cite{BatMGC} definition.  There are two main
differences.  The less significant is bias: where our definition treats
composition of all shapes equally, Batanin's gives special status to binary
and nullary compositions.  For instance, our weak $2$-categories are
unbiased bicategories, but his are classical, biased, bicategories.
The more significant difference is conceptual.  We integrated composition%
%
\index{composition!coherence@\vs.\ coherence}%
%
\index{coherence!composition@\vs.\ composition}
%
and coherence into the single notion of contraction; Batanin keeps the two
separate.  This makes his definition more complicated to state, but more
obvious from the traditional point of view.

Composition is handled as follows.  The unit map $\eta_1: 1 \go T1 = \pd$
picks out, for each $m\in\nat$, the $m$-pasting diagram $\iota_m$ looking
like a single $m$-cell.  Define 
a collection $\fcat{binpd} \rIncl \pd$%
%
\index{pasting diagram!globular!binary}
%
by
\[
\fcat{binpd}(m)
=
\{ \iota_m \of_0 \iota_m,\,
\iota_m \of_1 \iota_m,\,
\ldots,\,
\iota_m \of_m \iota_m \}
\subseteq \pd(m)
\]
where $\of_p$ is composition in the strict $\omega$-category $\pd$ and
$\iota_m \of_m \iota_m$ means $\iota_m$; then $\fcat{binpd}$
consists of binary and unary diagrams such as
\[
\begin{array}{rclcrcl}
\iota_1 \of_0 \iota_1 &= &
\gfstsu\gonesu\gzersu\gonesu\glstsu,	&&
\iota_1 &=&
\gfstsu\gonesu\glstsu,	\\
\iota_2 \of_0 \iota_2 &= & 
\gfstsu\gtwosu\gzersu\gtwosu\glstsu,	&&	
\iota_2 \of_1 \iota_2 &= &
\gfstsu\gthreesu\glstsu.
\end{array}
\]
% 
\begin{defn}
Let $P$ be a globular operad.  A \demph{system%
%
\index{system of compositions}%
%
\index{compositions, system of}
%
of (binary) compositions}
in $P$ is a map of collections $\fcat{binpd} \go P$, written
\[
\left(\iota_m \of_p \iota_m	\in \fcat{binpd}(m) \right)
\ \goesto\ 
\left(\delta^m_p \in P(\iota_m \of_p \iota_m)\right),%
% 
\glo{deltasys}
% 
\]
such that $\delta^m_m$ is the identity operation $1_m \in P(\iota_m)$ for
each $m\in\nat$.  
\end{defn}
%
So, for instance, $s(\delta^2_0) = t(\delta^2_0) = \delta^1_0$ and
$s(\delta^2_1) = t(\delta^2_1) = 1_1$.  (If we wanted to do an unbiased
version of Batanin's definition, we could replace $\fcat{binpd}$ by
$\pd$.)  

\begin{example}	\lbl{eg:contr-gives-sys}
A contraction $\kappa$ on an operad $P$ canonically determines a system
$\delta^\blob_\blob$ of compositions, defined inductively by
\[
\delta^m_p =
\left\{
\begin{array}{ll}
\kappa_{\iota_m \sof_p \iota_m} (\delta^{m-1}_p, \delta^{m-1}_p)	&
\textrm{if }
p < m,	\\
1_m	&
\textrm{if }
p = m.
\end{array}
\right.
\]
\end{example}

To describe coherence we use the notion of a \demph{reflexive%
%
\index{reflexive globular set}%
%
\index{globular set!reflexive}
%
globular
set}, that is, a globular set $Y$ together with functions
\[
\cdots \ 
\lTo^i	Y(k+1) 
\lTo^i	Y(k)
\lTo^i
\ \cdots \ 
\lTo^i	Y(0),
\]%
% 
\glo{irefl}%
% 
written $i(\psi) = 1_\psi$, such that $s(1_\psi) = t(1_\psi) = \psi$ for
each $k\geq 0$ and $\psi\in Y(k)$.  Reflexive globular sets form a presheaf
category $\ftrcat{\scat{R}^\op}{\Set}$.%
% 
\glo{Rrefl}
% 
 The underlying globular set of a
strict $\omega$-category is canonically reflexive, taking the identity
cells $1_\psi$.  This applies in particular to $T1 = \pd$; if $\pi \in
\pd(k)$ then $1_\pi \in \pd(k+1)$ is the degenerate $(k+1)$-pasting
diagram represented by the same picture as $\pi$.
%
\begin{defn}
Let $X$ be a globular set, $Y$ a reflexive globular set, and $q: X \go Y$ a
map of globular sets.  For $k\geq 0$ and $\psi\in Y(k)$, write
(Fig.~\ref{fig:coh-on-map}) 
% 
\begin{eqnarray*}
\mr{Par}'_q(k)	&
=	&
\{ (\theta^-, \theta^+) \in X(k) \times X(k) \such	
\theta^- \textrm{ and } \theta^+ \textrm{ are parallel, }
\\  &	&
q(\theta^-) = q(\theta^+) \}.
\end{eqnarray*}
% 
A \demph{coherence}%
%
\index{coherence!map of globular sets@on map of globular sets}
%
$\zeta$ on $q$ is a family of functions
\[
\left(
\mr{Par}'_q(k) \goby{\zeta_k} X(k+1)
\right)_{k\geq 0}
\]
such that for all $k\geq 0$ and $(\theta^-, \theta^+) \in \mr{Par}'_q(k)$,
writing $\zeta_k$ as $\zeta$,
\[
s(\zeta(\theta^-, \theta^+)) = \theta^-,
\ \ 
t(\zeta(\theta^-, \theta^+)) = \theta^+,
\ \ 
q(\zeta(\theta^-, \theta^+)) = 1_{q(\theta^-)} (= 1_{q(\theta^+)}).
\]
\end{defn}
% 
\begin{figure}
\[
\gfst{\xi}\gtwodotty{\theta^-}{\theta^+}{}%
\glst{\xi'}
\mbox{\hspace{2.5em}}
\stackrel{q}{\goesto}
\mbox{\hspace{2.5em}}
\gfst{q(\xi)}\gtwo{\psi}{\psi}{1_\psi}\glst{q(\xi')}
\]
\caption{Effect of a coherence $\zeta$, shown for $k=1$.  The dotted arrow
  is $\zeta(\theta^-, \theta^+)$, and $\psi = q(\theta^-) = q(\theta^+)$}
\label{fig:coh-on-map}
\end{figure}

\begin{example}
Any contraction on a map canonically determines a coherence, as is clear
from a comparison of Figs.~\ref{fig:contr-on-map}
(p.~\pageref{fig:contr-on-map}) and~\ref{fig:coh-on-map}.  Formally,
$\mr{Par}'_q(k) = \coprod_{\psi\in Y(k)} \mr{Par}_q(1_\psi)$ and a
contraction $\kappa$ determines the coherence $\zeta$ given by
$\zeta(\theta^-, \theta^+) = \kappa_{1_\psi}(\theta^-, \theta^+)$ where
$\psi = q(\theta^-) = q(\theta^+)$.  The class of contractible maps is
closed under composition, but the class of maps admitting a coherence is
not.
\end{example}

\begin{defn}	\lbl{defn:coh-coll}
A \demph{coherence}%
%
\index{coherence!collection@on collection}
%
on a collection $(P \goby{d} T1)$ is a coherence on the
map $d$, and a \demph{coherence}%
%
\index{coherence!globular operad@on globular operad}
%
on an operad is a coherence on its
underlying collection.  A map, collection, or operad is \demph{coherent}%
%
\index{coherent!map}%
%
\index{coherent!globular operad}
%
if
it admits a coherence.  
\end{defn}
%
Explicitly, a coherence on an operad $P$ assigns to each $\pi\in\pd(k)$ and
parallel pair $\theta^-, \theta^+ \in P(\pi)$ an element $\zeta(\theta^-,
\theta^+)$ of $P(1_\pi)$ with source $\theta^-$ and target $\theta^+$.  Let
$X$ be a $P$-algebra; then since $\rep{\pi} = \rep{1_\pi}$
(p.~\pageref{p:degen-rep}), we have $(TX)(\pi) = (TX)(1_\pi)$, so if
$\mathbf{x} \in (TX)(\pi)$ then there is a $(k+1)$-cell
\[
\ovln{\zeta(\theta^-, \theta^+)} (\mathbf{x}):
\ovln{\theta^-}(\mathbf{x})
\go 
\ovln{\theta^+}(\mathbf{x})
\]
in $X$ connecting the two `composites' $\ovln{\theta^-}(\mathbf{x})$ and
$\ovln{\theta^+}(\mathbf{x})$ of $\mathbf{x}$.  Taking $\pi = \iota_k$ and
$\theta^- = \theta^+ = 1_k$, this provides in particular a reflexive
structure on the underlying globular set of $X$.
%
\begin{example}	\lbl{eg:coh-vs-contr}
Any contractible operad is coherent (by the previous example), but not
conversely.  For instance, there is an operad $R$ whose algebras are
reflexive globular sets; it is uniquely determined by
\[
R(\pi) =
\left\{
\begin{array}{ll}
1		&
\textrm{if } \pi \in 
\{ \iota_k, 1_{\iota_{k-1}}, 1_{1_{\iota_{k-2}}}, \ldots \}
\\
\emptyset	&\textrm{otherwise}
\end{array}
\right.
\]
($\pi\in\pd(k)$).  This operad is coherent (trivially) but not contractible
(since some of the sets $R(\pi)$ are empty).  So a given globular set $X$
is an algebra for some coherent operad if and only if it admits a reflexive
structure; on the other hand, by~\ref{eg:wk-omega-cat-contr-opd}, it is an
algebra for some contractible operad if and only if it admits a weak
$\omega$-category structure.
\end{example}

%
\index{contraction!notions of|(}%
%
A coherence is what Batanin%
%
\index{contraction!Batanin's sense@in Batanin's sense}
%
calls a contraction.  As we have seen, our
contractions are more powerful, providing both a coherence and a system of
compositions.  

Rather confusingly, at several points in Batanin~\cite{BatMGC} the word
`contractible' is used as an abbreviation for `contractible [coherent] and
admitting a system of compositions'.  In particular, weak $\omega$- and
$n$-categories are often referred to as algebras for a `universal
contractible operad'.  This is not meant literally: `universal' means
weakly%
%
\index{weakly initial}
%
initial (in other words, there is at least one map from it to any
other contractible operad), and the operad $R$ of~\ref{eg:coh-vs-contr} is
in Batanin's terminology contractible, so any genuine `universal
contractible operad' $P$ satisfies $P(\pi) = \emptyset$ for almost all
pasting diagrams $\pi$.  A $P$-algebra is then nothing like an
$\omega$-category.  Indeed, $R$ is in Batanin's terminology the initial
operad equipped with a contraction, and its algebras are mere reflexive
globular sets.  The system of compositions is a vital ingredient; left out,
the situation degenerates almost entirely.

I believe it is the case that given a system of compositions and a
coherence on an operad, a contraction can be built.  This is the
non-canonical converse to the canonical process in the other direction; the
situation is like that of biased \vs.\ unbiased bicategories.  So it
appears that despite an abuse of terminology and two different definitions
of contractibility, the term `contractible operad' means exactly the same
in Batanin's work as here.%
%
\index{contraction!notions of|)}
%

The two ingredients---composition and coherence---can be combined to make a
definition of weak $\omega$-category in several possible ways:
%
\begin{itemize}
\item
Imitate the definition of the previous chapter.  In other words, take the
category of globular operads equipped with both a system of compositions
and a coherence, prove that it has an initial object $(B,%
% 
\glo{BBatopd}
% 
\delta^\blob_\blob, \kappa)$, and define a weak $\omega$-category as a
$B$-algebra.  This is Definition \textbf{B1} in my~\cite{SDN} survey.
% 
\item
Use Batanin's operad $K$, constructed in his Theorem~8.1.  He proves that
$K$ is weakly initial in the full subcategory of $T\hyph\Operad$ consisting
of the coherent operads admitting a system of compositions.  Weak
initiality does not characterize $K$ up to isomorphism, so one needs some
further information about $K$ in order to use this definition.  It seems to
be claimed that $K$ is initial in the category of operads equipped with a
system of compositions and a coherence (Remark~2 after the proof of
Theorem~8.1), but it does not seem obvious that this claim is true,
essentially because of the set-theoretic complement taken in the proof of
Lemma~8.1.
%
\item 
Define a weak $\omega$-category as a pair $(P, X)$ where $P$ is a coherent
operad admitting a system of compositions and $X$ is a $P$-algebra.  Given
such a pair $(P, X)$, we can choose a system of compositions and a
coherence on $P$, and this turns $X$ into a $B$-algebra---that is, a weak
$\omega$-category in the sense of the first method.  The present method has
some variants: we might insist that $P(\blob) = 1$, where $\blob$ is the
unique $0$-pasting diagram, or we might drop the condition that $P$ admits
a system of compositions and replace it with the more relaxed requirement
that $P(\pi) \neq \emptyset$ for all pasting diagrams $\pi$ (giving
Batanin's Definition~8.6 of `weak $\omega$-categorical object').
\end{itemize}

Batanin's weak $\omega$-categories can be compared with the weak
$\omega$-categories of the previous chapter.  We have already shown that a
contraction on a globular operad gives rise canonically to a system of
compositions and a coherence
(\ref{eg:contr-gives-sys},~\ref{eg:coh-vs-contr}).  This is true in
particular of the operad $L$, so there is a canonical map $B \go L$ of
operads, inducing in turn a canonical functor from $L$-algebras to
$B$-algebras.  Conversely, $B$ is non-canonically contractible, so there is
a non-canonical functor in the other direction.  In the case $n=2$, this
is the comparison of biased and unbiased bicategories.%
%
\index{Batanin, Michael!definition of n-category@definition of $n$-category|)}%
%
\index{n-category@$n$-category!definitions of!Batanin's|)}
%


\minihead{Penon's definition}%
%
\index{Penon, Jacques!definition of n-category@definition of $n$-category|(}%
%
\index{n-category@$n$-category!definitions of!Penon's|(}
%


The definition of weak $\omega$-category proposed by Penon~\cite{Pen} does
not use the language of operads, but is nevertheless close in spirit to the
definition of Batanin.  It can be stated very quickly.

\begin{defn}
Let $q: X \go Y$ be a map of reflexive globular sets.  A coherence $\zeta$
on $q$ is \demph{normal}%
%
\index{normal coherence}%
%
\index{coherence!normal}
%
if $\zeta(\theta, \theta) = 1_\theta$ for all
$k\geq 0$ and $\theta\in X(k)$.
\end{defn}
%
Penon calls a normal coherence an \'etirement,%
%
\index{etirement@\'etirement}
%
or stretching.%
%
\index{stretching}
%
This might
seem to conflict with the contraction terminology, but it is only a matter
of viewpoint: $X$ is being shrunk, $Y$ stretched.

\begin{defn}
An \demph{$\omega$-magma}%
%
\index{magma}%
%
\index{omega-magma@$\omega$-magma}
%
is a reflexive globular set $X$ equipped with a
binary composition function $\of_p: X(m) \times_p X(m) \go X(m)$ for each
$m > p \geq 0$, satisfying the source and target axioms
of~\ref{defn:strict-n-cat-glob}\bref{item:s-t-comp}.
\end{defn}
%
An $\omega$-magma is a very wild structure, and a strict $\omega$-category
very tame; weak $\omega$-categories are somewhere in between.  

Let $\cat{Q}$ be the category whose objects are quadruples $(X, Y, q,
\zeta)$ with $X$ an $\omega$-magma, $Y$ a strict $\omega$-category, $q: X
\go Y$ a map of $\omega$-magmas, and $\zeta$ a normal coherence on $q$.
Maps
\[
(X, Y, q, \zeta) \go (X', Y', q', \zeta')
\]
in $\cat{Q}$ are pairs $(X \go X', Y \go Y')$ of maps commuting with all
the structure present.  There is a forgetful functor $U$ from $\cat{Q}$ to
the category $\ftrcat{\scat{R}^\op}{\Set}$ of reflexive globular sets,
sending $(X, Y, q, \zeta)$ to the underlying reflexive globular set of $X$.
This has a left adjoint $F$, and a weak $\omega$-category is
defined as an algebra for the induced monad $U\of F$ on
$\ftrcat{\scat{R}^\op}{\Set}$.

Observe that $U$ takes the underlying reflexive globular set of $X$, not of
$Y$.  The object $(X, Y, q, \zeta)$ of $\cat{Q}$ should therefore be
regarded as $X$ (not $Y$) equipped with extra structure, making it perhaps
more apt to think of an object of $\cat{Q}$ as a shrinking rather than a
stretching.

Recent work of Cheng%
%
\index{Cheng, Eugenia}
%
relates Penon's definition to the definition of the
previous chapter through a series of intermediate definitions; older work
of Batanin~\cite{BatPMW}%
%
\index{Batanin, Michael}%
%
\index{Batanin, Michael!definition of n-category@definition of $n$-category}
%
relates Penon's definition to his own.  One point
can be explained immediately.  Take the operad $B$ equipped with its system
of compositions $\delta^\blob_\blob$ and its coherence $\kappa$.  There is
a unique reflexive structure on the underlying globular set of $B$ for
which $\kappa$ is normal, namely $1_\theta = \zeta(\theta, \theta)$.  Also,
the system of compositions in the operad $B$ makes its underlying globular
set into an $\omega$-magma: given $0\leq p < m$ and $(\theta_1, \theta_2)
\in B(m) \times_{B(p)} B(m)$, put
\[
\theta_1 \ofdim{p} \theta_2 
=
\delta^m_p \of (\theta_1, \theta_2)
\]
where the $\of$ on the right-hand side is operadic composition.  This gives
$B \go T1$ the structure of an object of $\cat{Q}$.  Now, if $X$ is any
$B$-algebra, we have a pullback square
\[
\begin{diagram}[size=1.7em]
	&	&T_B X\Spbk&	&	\\
	&\ldTo  &	&\rdTo	&	\\
TX	&	&	&	&B	\\
	&\rdTo<{T!}&	&\ldTo  &	\\
	&	&T1,	&	&	\\
\end{diagram}
\]
and the $\cat{Q}$-object structure on $B \go T1$ induces a $\cat{Q}$-object
structure on $T_B X \go TX$.  So any $B$-algebra gives rise to an object
of $\cat{Q}$ and hence, applying the comparison functor for the monad $U\of
F$, a Penon weak $\omega$-category.%
%
\index{Penon, Jacques!definition of n-category@definition of $n$-category|)}%
%
\index{n-category@$n$-category!definitions of!Penon's|)}
%




\minihead{Trimble's and May's definitions}%
%
\index{Trimble, Todd|(}%
%
\index{n-category@$n$-category!definitions of!Trimble's|(}%
%
\index{n-category@$n$-category!flabby|(}%
%
\index{flabby n-category@flabby $n$-category|(}
%
\index{fundamental!n-groupoid@$n$-groupoid|(}
%


Trimble has also proposed a simple definition of $n$-category, unpublished
but written up as Definition \textbf{Tr} in my~\cite{SDN} survey.  He was
not quite so ambitious as to attempt a fully weak notion of $n$-category;
rather, he sought a notion just general enough to capture the fundamental
$n$-groupoids of topological spaces.  He called his structures `flabby
$n$-categories'.

Trimble's definition uses simple operad language.  Let $\cat{A}$ be a
category with finite products and let $P$ be a (non-symmetric) operad in
$\cat{A}$.  Extending slightly the terminology of
p.~\pageref{p:P-category}, a \demph{categorical $P$-algebra}%
%
\index{categorical algebra for operad}
%
is an
$\cat{A}$-graph $X$ together with a map
%
\begin{equation}	\label{eq:cat-alg-action}
P(k) \times X(x_0, x_1) \times \cdots \times X(x_{k-1}, x_k)
\go 
X(x_0, x_k)
\end{equation}
%
for each $k\in\nat$ and $x_0, \ldots, x_k \in X_0$, satisfying the evident
axioms.  The category of categorical $P$-algebras is written
$\fcat{CatAlg}(P)$,%
% 
\glo{TrimCatAlg}
% 
and itself has finite products.  If $F: \cat{A} \go
\cat{A'}$ is a finite-product-preserving functor then there is an induced
operad $F_* P$ in $\cat{A'}$ and a finite-product-preserving functor
\[
\widehat{F}: 
\fcat{CatAlg}(P) \go \fcat{CatAlg}(F_* P).
\]
  
The topological content consists of two observations: first
(Example~\ref{eg:opd-Trimble}), that if $E$ is the operad of path%
%
\index{operad!path reparametrizations@of path reparametrizations}
%
reparametrizations then there is a canonical functor
\[
\Xi: \Top \go \fcat{CatAlg}(E),
\]
and second, that taking path-components defines a finite-product-preserving
functor $\Pi_0: \Top \go \Set$.

Applying the operadic constructions recursively to the topological data, we
define for each $n\in\nat$ a category $\fcat{Flabby}\hyph n\hyph\Cat$ with
finite products and a functor
\[
\Pi_n: \Top \go \fcat{Flabby}\hyph n\hyph\Cat 
\]%
% 
\glo{Pin}%
% 
preserving finite products: $\fcat{Flabby}\hyph 0\hyph\Cat = \Set$, and
$\Pi_{n+1}$ is the composite functor
\[
% \Pi_{n+1}
% =
% \left(
\Top 
\goby{\Xi} 
\fcat{CatAlg}(E) 
\goby{\widehat{\Pi_n}} 
\fcat{CatAlg}((\Pi_n)_* E)
=
\fcat{Flabby}\hyph (n+1)\hyph\Cat.
% \right).
\]
That completes the definition.

Operads in $\Top$ can be regarded as $T$-operads, where $T$ is the free
topological%
%
\index{monoid!topological!free}
%
monoid monad (\ref{eg:mon-free-topo-monoid},~\ref{eg:mti-Top}).
The operadic techniques used in Trimble's definition are then expressible
in the language of generalized operads, and I make the following
%
\begin{claim}
  For each $n\in\nat$, there is a contractible globular $n$-operad whose
  category of algebras is equivalent to $\fcat{Flabby}\hyph n\hyph\Cat$.
\end{claim}
%
I hope to prove this elsewhere.  It implies that every flabby $n$-category
is a weak $n$-category in the sense of the previous chapter.  The
$n$-operad concerned is something like an $n$-dimensional version of the
operad $P$ of Example~\ref{eg:wk-omega-cat-Pi} (fundamental weak
$\omega$-groupoids), but a little smaller.

\index{May, Peter!definition of n-category@definition of $n$-category|(}%
%
\index{n-category@$n$-category!definitions of!May's|(}
%
Alternatively, we can generalize Trimble's definition by considering
operads in an arbitrary symmetric monoidal category.  This leads us to the
definition of enriched $n$-category proposed by May~\cite{MayOCA} (although
that is not what led May there).%
%
\index{Trimble, Todd|)}%
%
\index{n-category@$n$-category!definitions of!Trimble's|)}%
%
\index{n-category@$n$-category!flabby|)}%
%
\index{flabby n-category@flabby $n$-category|)}
%
\index{fundamental!n-groupoid@$n$-groupoid|)}
%

Let $\cat{B}$ and $\cat{G}$ be symmetric monoidal categories and suppose
that $\cat{B}$ acts on $\cat{G}$ in a way compatible with both monoidal
structures.  (Trivial example: $\cat{B} = \cat{G}$ with tensor as action.)
Let $P$ be an operad in $\cat{B}$.  Then we can define a
\demph{categorical%
%
\index{categorical algebra for operad}
%
$P$-algebra in $\cat{G}$} as a $\cat{G}$-graph $X$ together with maps as
in~\bref{eq:cat-alg-action}, with the first $\times$ replaced by the action
$\odot$ and the others $\times$'s by $\otimes$'s, subject to the inevitable
axioms.  For example, if $\cat{B} = \Top$ and $\cat{G} = \fcat{Flabby}\hyph
n\hyph\Cat$ with monoidal structures given by products then the functor
$\Pi_n: \cat{B} \go \cat{G}$ induces an action $\odot$ of $\cat{B}$ on
$\cat{G}$ by $B \odot G = \Pi_n(B) \times G$, and a categorical $P$-algebra
in $\cat{G}$ is what we previously called a categorical $((\Pi_n)_*
P)$-algebra.

The idea now is that if we can find some substitute for the functor $\Xi:
\Top \go \fcat{CatAlg}(E)$ then we can imitate Trimble's recursive
definition in this more general setting.  So, May starts with a `base'
symmetric monoidal category $\cat{B}$ and an operad $P$ in $\cat{B}$, each
carrying certain extra structure and satisfying certain extra properties,
the details of which need not concern us here.  He then considers symmetric
monoidal categories $\cat{G}$, restricting his attention to just those that
are `good' in the sense that they too have certain extra structure and
properties, including that they are acted on by $\cat{B}$.  Then the point
is that
%
\begin{itemize}
\item the trivial example $\cat{G} = \cat{B}$ is good
\item if $\cat{G}$ is good then so is the category of categorical
  $P$-algebras in $\cat{G}$.
\end{itemize}
%
So given a category $\cat{B}$ and operad $P$ in $\cat{B}$ as above, we can
define for each $n\in\nat$ the category $\cat{B}(n; P)$ of
\demph{$n$-$P$-categories enriched in $\cat{B}$}%
%
\index{enrichment!of n-categories@of $n$-categories}%
%
\index{n-category@$n$-category!enriched}
%
as follows: 
%
\begin{itemize}
\item $\cat{B}(0; P) = \cat{B}$
\item $\cat{B}(n+1; P) = (\textrm{categorical } P \textrm{-algebras in } 
\cat{B}(n; P) )$.
\end{itemize}
% 
An operad $P$ in $\cat{B}$ is called an \demph{$A_\infty$-operad}%
%
\index{A-@$A_\infty$-!operad}
%
if for each $k\in\nat$, the object $P(k)$ of $\cat{B}$ is weakly equivalent
to the unit object; here `weakly equivalent' refers to a Quillen model%
%
\index{model category}
%
category structure on $\cat{B}$, which is part of the assumed structure.
May proposes that when $P$ is an $A_\infty$-operad, $n$-$P$-categories
enriched in $\cat{B}$ should be called weak $n$-categories enriched in
$\cat{B}$.  His definition, like Trimble's, aims for a slightly different
target from most of the other definitions.  It has enrichment built in; he
writes
%
\begin{quote}
  In all of the earlier approaches, $0$-categories are understood to be
  sets, whereas we prefer a context in which $0$-categories come with their
  own homotopy%
%
\index{homotopy-algebraic structure}%
%
\index{homotopy!zero-categories@of 0-categories}
%
%
theory.
\end{quote}
%
So, for instance, we might take the category $\cat{B}$ of $0$-categories to
be a convenient category of topological spaces, or the category of
simplicial sets, or a category of chain complexes.%
%
\index{algebraic theory!n-categories@of $n$-categories|)}
%
\index{May, Peter!definition of n-category@definition of $n$-category|)}%
%
\index{n-category@$n$-category!definitions of!May's|)}
 



\section{Non-algebraic definitions}
\lbl{sec:non-alg-defns-n-cat}

Most of the proposed non-algebraic definitions of weak $n$-category can be
expressed neatly using a generalization of the standard nerve construction,
which describes a category as a simplicial set with properties.  We discuss
nerves in general, then the definitions of Joyal, Tamsamani, Simpson, Baez
and Dolan (and others), Street, and Leinster (Definition $\mathbf{L'}$
of~\cite{SDN}).  We finish by looking at some structures approximating to
the idea of a weak $\omega$-category in which all cells of dimension $2$
and higher are invertible: $A_\infty$-categories, Segal categories, and
quasi-categories.  These have been found especially useful in geometry.


\minihead{Nerves}%
%
\index{nerve|(}
%

The nerve idea allows us to define species of mathematical structures by
saying on the one hand what the structures look like locally, and on the
other how the local pieces are allowed to be fitted together.  We consider
it in some generality.

The setting is a category $\cat{C}$ of `mathematical structures' with a
small subcategory $\scat{C}$ of `local pieces' or `building blocks'.  The
condition that every object of $\cat{C}$ is put together from objects of
$\scat{C}$ is called density, defined in a moment.

\begin{propn}
Let $\scat{D}$ be a small category and $\cat{D}$ a category with small
colimits.  The following conditions on a functor $I: \scat{D} \go \cat{D}$
are equivalent:
%
\begin{enumerate}
\item	\lbl{item:dense-coend}
for each $Y\in\cat{D}$, the canonical map
\[
\int^{D\in\scat{D}} \cat{D}(ID, Y) \times ID
\go
Y
\]
is an isomorphism
\item	\lbl{item:dense-ff}
the functor
\[
\begin{array}{rcl}
\cat{D}	&\go		&\ftrcat{\scat{D}^\op}{\Set},	\\
Y	&\goesto	&\cat{D}(I\dashbk, Y)
\end{array}
\]
is full and faithful.
\end{enumerate}
\end{propn}
%
\begin{proof}
See Mac Lane~\cite[X.6]{MacCWM}.
\done
\end{proof}
% 
A functor $I: \scat{D} \go \cat{D}$ is \demph{dense}%
%
\index{dense}
%
if it satisfies the
equivalent conditions of the Proposition, and a subcategory $\scat{C}$ of a
category $\cat{C}$ is \demph{dense} if the inclusion functor $\scat{C}
\rIncl \cat{C}$ is dense.  Condition~\bref{item:dense-coend} formalizes the
idea that objects of $\cat{C}$ are pasted-together objects of $\scat{C}$.
Condition~\bref{item:dense-ff} is what we use in examples to prove density.

\begin{example}
Let $n\in\nat$, let $n\hyph\fcat{Mfd}$ be the category of smooth
$n$-manifolds%
%
\index{manifold}
%
and smooth maps, and let $\scat{U}_n$ be the subcategory
whose objects are all open subsets of $\mathbb{R}^n$ and whose maps $f: U
\go U'$ are diffeomorphisms from $U$ to an open subset of $U'$.  Then
$\scat{U}_n$ is dense in $n\hyph\fcat{Mfd}$: every manifold is a pasting of
Euclidean open sets.
\end{example}

\begin{example}
Let $k$ be a field, let $\fcat{Vect}_k$ be the category of all vector%
%
\index{vector space}
%
spaces over $k$, and let $\fcat{Mat}_k$%
% 
\glo{Matk}
% 
be the category whose objects are
the natural numbers and whose maps $m\go n$ are $n\times m$ matrices%
%
\index{matrix}
%
over
$k$.  There is a natural inclusion $\fcat{Mat}_k \rIncl \fcat{Vect}_k$
(sending $n$ to $k^n$), and $\fcat{Mat}_k$ is then dense in
$\fcat{Vect}_k$.  This reflects the fact that any vector space is the
colimit (pasting-together) of its finite-dimensional subspaces, which in
turn is true because the theory of vector spaces is finitary (its
operations take only a finite number of arguments).  
\end{example}

\begin{example}
The inclusion $\Delta \rIncl \Cat$%
%
\index{simplex category $\Delta$}
%
(p.~\pageref{p:defn-nerve}) is also
dense.  This says informally that a category is built out of objects,
arrows, commutative triangles, commutative tetrahedra, \ldots, and source,
target and identity functions between them.  We would not expect the
commutative tetrahedra and higher-dimensional simplices to be necessary,
and indeed, if $\Delta_2$ denotes the full subcategory of $\Delta$
consisting of the objects $\upr{0}$, $\upr{1}$ and $\upr{2}$ then the
inclusion $\Delta_2 \rIncl \Cat$ is also dense.
\end{example}
%
Generalizing the terminology of this example, if $\scat{C}$ is a dense
subcategory of $\cat{C}$ and $X$ is an object of $\cat{C}$ then the
\demph{nerve}%
%
\index{nerve!general notion}
%
(over $\scat{C}$) of $X$ is the presheaf $\cat{C}(\dashbk,
X)$ on $\scat{C}$.

For us, the crucial point about density is that it allows the mathematical
structures (objects of $\cat{C}$) to be viewed as
presheaves-with-properties%
%
\index{presheaf!properties@with properties}
%
on the category $\scat{C}$ of local pieces.
That is, if $\scat{C}$ is dense in $\cat{C}$ then $\cat{C}$ is equivalent
to the full subcategory of $\ftrcat{\scat{C}^\op}{\Set}$ consisting of
those presheaves isomorphic to the nerve of some object of $\cat{C}$.
Sometimes the presheaves arising as nerves can be characterized
intrinsically, yielding an alternative definition of $\cat{C}$ by
presheaves on $\scat{C}$.

\begin{example}%
%
\index{vector space}
%
A vector space over a field $k$ can be \emph{defined} as a presheaf $V:
\fcat{Mat}_k^\op \go \Set$ preserving finite limits.
\end{example}

\begin{example}%
%
\index{manifold!alternative definition of}
%
A smooth $n$-manifold can be \emph{defined} as a presheaf on $\scat{U}_n$
with certain properties.
\end{example}

\begin{example}	\lbl{eg:nerve-cat-chars}
A category can be \emph{defined} as a presheaf on $\Delta$%
%
\index{simplex category $\Delta$}
%
(or indeed on $\Delta_2$) with certain properties.  There are various ways
to express those properties: for instance, categories are functors
$\Delta^\op \go \Set$ preserving finite limits, or preserving certain
pullbacks, or they are simplicial sets in which every inner horn%
%
\index{horn!filler for}
%
has a
unique filler.  (A horn is \demph{inner}%
%
\index{horn!inner}
%
%
if the missing face is not the first or the last one.)
\end{example}

%
\index{nerve!n-category@of $n$-category|(}
%
The structures we want to define are weak $n$-categories, and the strategy
is:
%
\begin{itemize}
\item find a small dense subcategory $\scat{C}$ of the category of
  \emph{strict} $n$-categories
\item find conditions on a presheaf on $\scat{C}$ equivalent to it being a
  nerve of a strict $n$-category
\item relax%
%
\index{weakening!theory of n-categories@for theory of $n$-categories}
%
those conditions to obtain a definition of \emph{weak}
  $n$-category.
\end{itemize}
%
Most of the definitions of weak $n$-category described below can be
regarded as implementations of this strategy.  A different way to put it is
that we seek an intrinsic characterization of presheaves on $\scat{C}$ of
the form
\[
C
\ \goesto\ 
\{ \textrm{weak functors } C \go Y \}
\]
for some weak $n$-category $Y$.  Of course, we start from a position of not
knowing what a weak $n$-category or functor is, but we choose the
conditions on presheaves to fit the usual intuitions.%
%
\index{nerve|)}
%



\minihead{Joyal's definition}%
%
\index{Joyal, Andr\'e!definition of n-category@definition of $n$-category|(}%
%
\index{n-category@$n$-category!definitions of!Joyal's|(}%


Perhaps the most obvious implementation is to take the local pieces to be
all globular%
%
\index{pasting diagram!globular}
%
pasting diagrams.  So, let $\Delta_\omega$%
% 
\glo{Deltaomega}%
%
\index{simplex category $\Delta$!omega-dimensional analogue of@$\omega$-dimensional analogue of}
% 
be the category
with object-set $\coprod_{m\in\nat} \pd(m)$ and hom-sets
\[
\Delta_\omega (\sigma, \pi)
=
\strcat{\omega}(F\rep{\sigma}, F\rep{\pi})
\iso
\ftrcat{\scat{G}^\op}{\Set} (\rep{\sigma}, T\rep{\pi})
\]
where $F: \ftrcat{\scat{G}^\op}{\Set} \go \strcat{\omega}$ is the free
strict $\omega$-category functor and, as in Chapter~\ref{ch:a-defn}, $T$ is
the corresponding monad.  There is an inclusion functor $\Delta_\omega
\rIncl \strcat{\omega}$, and a typical object of the corresponding
subcategory of $\strcat{\omega}$ is the $\omega$-category naturally
depicted as
\[
\gfstsu\gfoursu\gzersu\gonesu\gzersu\gthreesu\glstsu
\]
---that is, freely generated by the 0-, 1- and 2-cells shown, and with only
identity cells in dimensions 3 and above.  

The subcategory $\Delta_\omega$ of $\strcat{\omega}$ is dense.  That the
induced functor $\strcat{\omega} \go \ftrcat{\Delta_\omega^\op}{\Set}$ is
faithful follows from $\Delta_\omega$ containing the trivial pasting
diagrams
\[
\gzersu\,,
\diagspace
\gfstsu\gonesu\glstsu,
\diagspace
\gfstsu\gtwosu\glstsu,
\diagspace
\ldots
\]
representing single $m$-cells.  That it is full follows from
$\Delta_\omega$ containing maps such as 
% 
\begin{equation}	\label{eq:comp-inducers}
\gfstsu\gtwosu\glstsu
\ \goby{f}\ 
\gfstsu\gonesu\glstsu
\diagspace
\textrm{and}
\diagspace
\gfstsu\gtwosu\glstsu
\ \goby{g}\ 
\gfstsu\gthreesu\glstsu
\end{equation}
% 
which, for appropriately chosen $f$ and $g$, induce 2-cell identities and
vertical 2-cell composition respectively.

A strict $\omega$-category can therefore be defined as a presheaf on
$\Delta_\omega$ with properties.  Joyal has proposed~\cite{JoyDDT} a way of
describing and then relaxing those properties: a strict $\omega$-category
is a presheaf on $\Delta_\omega$ for which `every inner horn%
%
\index{horn!filler for}
%
has a unique
filler',
and a weak $\omega$-category is defined by simply dropping the
uniqueness.

We can do the same with $\omega$ replaced by any finite $n$ (taking care in
the top%
%
\index{top dimension}
%
dimension).  Recall from p.~\pageref{p:degen-rep} that if $1_\pi$
denotes the $(m+1)$-pasting diagram resembling an $m$-pasting diagram $\pi$
then $\rep{\pi} \iso \rep{1_\pi}$: so $\Delta_n$ is equivalent to its full
subcategory consisting of just the $n$-pasting diagrams.  For instance,
$\Delta_1$ is equivalent to the usual category $\Delta$ of $1$-pasting
diagrams
\[
\gzersu\,,
\diagspace
\gfstsu\gonesu\glstsu,
\diagspace
\gfstsu\gonesu\gzersu\gonesu\glstsu, 
\diagspace
\ldots,
\]
and we recover the standard nerve construction for categories.  

Joyal also noted a duality.%
%
\index{duality!intervals vs. ordered sets@intervals \vs.\ ordered sets}
%
 The category $\Delta$ is equivalent to
the opposite of the category $\scat{I}$ of \demph{finite strict intervals},
that is, finite totally ordered sets with distinct least and greatest
elements (to be preserved by the maps).  Generalizing this, he defined a
category $\scat{I}_\omega$ of `finite disks',%
%
\index{disk!Joyal's sense@in Joyal's sense}
%
equivalent to the opposite of
$\Delta_\omega$.  
% (See the Notes at the end of the chapter for further
% references.)  
So his weak $\omega$-categories are functors $\scat{I}_\omega \go \Set$
satisfying horn-filling conditions.%
%
\index{Joyal, Andr\'e!definition of n-category@definition of $n$-category|)}%
%
\index{n-category@$n$-category!definitions of!Joyal's|)}%
%




\minihead{Tamsamani's and Simpson's definitions}%
%
\index{Simpson, Carlos!definition of n-category@definition of $n$-category|(}%
%
\index{n-category@$n$-category!definitions of!Simpson's|(}%
%
\index{Tamsamani, Zouhair!definition of n-category@definition of $n$-category|(}%
%
\index{n-category@$n$-category!definitions of!Tamsamani's|(}%
%

A very similar story can be told for the definitions proposed by
Tamsamani~\cite{TamSNN} and Simpson~\cite{SimCMS}.  Observe that in the
proof of the density of $\Delta_\omega$ in $\strcat{\omega}$, we did not
use many of the pasting diagrams, so we can replace $\Delta_\omega$ by a
smaller category.  Tamsamani and Simpson consider just `cuboidal'%
%
\index{pasting diagram!cubical}
%
pasting
diagrams such as
\[
\gfstsu\gfoursu\gzersu\gfoursu\gzersu\gfoursu\gzersu\gfoursu\glstsu
\]
and similarly `cuboidal' maps between the strict $\omega$-categories that
they generate.

Let us restrict ourselves to the $n$-dimensional case, since that is a
little easier.  There is a functor
\[
I: \Delta^n \go \strcat{n}
\]%
%
\index{simplex category $\Delta$}%
%
which, for instance, when $n=2$, sends $(\upr{4}, \upr{3})$ to the free
strict 2-category on the diagram above.  In general, each $(r_1, \ldots,
r_n) \in \nat^n$ determines an $n$-pasting diagram $\pi_{r_1, \ldots,
r_n}$, given inductively by
\[
\pi_{r_1, \ldots, r_n}
=
\left(\pi_{r_2, \ldots, r_n}, \ldots, \pi_{r_2, \ldots, r_n}\right)
\]
with $r_1$ terms on the right-hand side, and then
\[
I(\upr{r_1}, \ldots, \upr{r_n})
=
F\rep{\pi_{r_1, \ldots, r_n}}.
\]
To describe $I$ on maps, take, for instance, $n=2$ and the map $(\id,
\delta): (\upr{1}, \upr{1}) \go (\upr{1}, \upr{2})$ in which $\delta$ is
the injection omitting $1 \in \upr{2}$ from its image; then $I(\id,
\delta)$ is the map $g$ of~\bref{eq:comp-inducers}.

By exactly the same argument as for Joyal's definition, the functor
$\Delta^n \go \strcat{n}$ is dense.  A strict $n$-category is therefore the
same thing as a presheaf on $\Delta^n$ (a `multisimplicial%
%
\index{multisimplicial set}%
%
\index{simplicial set!multi-}
%
%
set') with
properties, and relaxing those properties gives a definition of weak
$n$-category.

Nerves of strict $n$-categories are characterized among functors
$(\Delta^n)^\op \go \Set$ by the properties that the functor is degenerate
in certain ways (to give us $n$-categories rather%
%
\index{n-tuple category@$n$-tuple category!degenerate}
%
than $n$-tuple
categories) and, more significantly, that certain pullbacks are preserved.
Tamsamani sets up a notion of equivalence,%
%
\index{equivalence!n-categories@of $n$-categories}
%
and defines weak $n$-category by
asking only that the pullbacks are preserved up to equivalence.  Simpson
does the same, but with a more stringent notion of equivalence that he
calls `easy%
%
\index{equivalence!easy}
%
equivalence'.  It is indeed easier, and is nearly the same as
the notion of contractibility%
%
\index{contractible!map of globular sets}
%
of a map of globular sets: see my
survey~\cite{SDN} for details.  In the special case of one-object
2-categories, Tamsamani's definition gives the homotopy monoidal categories
of Section~\ref{sec:non-alg-notions}, and Simpson's gives the same but with
the extra condition that the functors $\xi^{(k)}$ of
Proposition~\ref{propn:simp-eqs}, which for homotopy%
%
\index{monoidal category!homotopy}
%
monoidal categories
are required to be equivalences, are \emph{genuinely} surjective on
objects.%
%
\index{Simpson, Carlos!definition of n-category@definition of $n$-category|)}%
%
\index{n-category@$n$-category!definitions of!Simpson's|)}%
%
\index{Tamsamani, Zouhair!definition of n-category@definition of $n$-category|)}%
%
\index{n-category@$n$-category!definitions of!Tamsamani's|)}%
%








\minihead{Opetopic definitions}%
%
\index{Baez, John!definition of n-category@definition of $n$-category|(}%
%
\index{Dolan, James!definition of n-category@definition of $n$-category|(}%
%
\index{n-category@$n$-category!definitions of!opetopic|(}%



We have already~(\ref{sec:ope-n}) looked at the opetopic definitions of
weak $n$-category: that of Baez and Dolan and subsequent variants.  They
are all of the form `a weak $n$-category is an opetopic set with
properties', for varying meanings of `opetopic set' and varying lists of
properties.  Here we see how this fits in with the nerve idea.  The
situation is, as we shall see, slightly different from that in the
definitions of Joyal, Tamsamani, and Simpson.

We start with the category $\scat{O}$ of opetopes.  As mentioned
on p.~\pageref{p:ope-to-n-cat}, there is an embedding $\scat{O} \rIncl
\strcat{\omega}$, and this induces a functor 
\[
U: \strcat{\omega} \go \ftrcat{\scat{O}^\op}{\Set}
\]
sending a strict $\omega$-category to its underlying opetopic set.

This functor is faithful, because there is for each $m\in\nat$ an
$m$-opetope resembling a single globular $m$-cell.  It is not, however,
full.  To see this, note that if $F: A \go B$ is a strict map of strict
$\omega$-categories then the induced map $U(F): U(A) \go U(B)$ preserves
universality%
%
\index{universal!preservation}
%
of cells: for instance, if $f$ and $g$ are abutting 1-cells in
$A$ then $U(F)$ sends the canonical 2-cell
\[
\topeb{f}{g}{g\of f}{\Downarrow}
\]
in $U(A)$ to the canonical 2-cell 
\[
\topeb{Ff}{Fg}{(Fg)\of (Ff)}{\Downarrow}
\]
in $U(B)$.  But not every map $U(A) \go U(B)$ of opetopic sets preserves
universality; indeed, any \emph{lax} map $A \go B$ of strict
$\omega$-categories ought (in principle, at least) to induce a map $U(A)
\go U(B)$, and this will preserve universality if and only if the lax map
is weak.  Compare the relationship between monoidal categories and plain
multicategories (\ref{eg:map-mti-mon},~\ref{sec:non-alg-notions}).

So $\scat{O}$ is not dense in $\strcat{\omega}$, and correspondingly $U$
does not define an equivalence between $\strcat{\omega}$ and a full
subcategory of $\ftrcat{\scat{O}^\op}{\Set}$.  But with a slight
modification, the nerve idea can still be made to work.  For the above
arguments suggest that $U$ defines an equivalence between
\[
(\textrm{strict } \omega \textrm{-categories } + \textrm{ weak maps})
\]
and a full subcategory of 
\[
(\textrm{opetopic sets } + \textrm{universality-preserving maps}),
\]
and it is then, as usual, a matter of identifying the characteristic
properties of those opetopic sets arising from strict $\omega$-categories,
then relaxing the properties to obtain a definition of weak
$\omega$-category.  So a weak $\omega$-category is defined as an opetopic
set with properties, and a weak $\omega$-functor as a map of opetopic sets
preserving universality.%
%
\index{Baez, John!definition of n-category@definition of $n$-category|)}%
%
\index{Dolan, James!definition of n-category@definition of $n$-category|)}%
%
\index{n-category@$n$-category!definitions of!opetopic|)}%




\minihead{Street's definition}%
%
\index{Street, Ross!definition of n-category@definition of $n$-category|(}%
%
\index{n-category@$n$-category!definitions of!Street's|(}%
%
\index{simplicial set!n-category as@$n$-category as|(}


The definition of weak $\omega$-category proposed by Street has the
distinctions of being the first and probably the most tentatively phrased;
it hides in the last paragraph of his paper of~\cite{StrAOS}.  It was part
of the inspiration for Baez and Dolan's definition, and has much in common
with it, but uses simplicial rather than opetopic sets.

Street follows the nerve idea explicitly.  He first constructs an embedding
\[
I: \Delta \rIncl \strcat{\omega},
\]
where $I(m)$ is the $m$th `oriental',%
%
\index{oriental}
%
the free strict%
%
\index{omega-category@$\omega$-category!strict!free}
%
$\omega$-category on
an $m$-simplex.  For example, $I(3)$ is the strict $\omega$-category
generated freely by 0-, 1-, 2- and 3-cells
\[
\setlength{\unitlength}{1em}
\begin{picture}(21,4.6)
% 
\cell{0}{0}{bl}{%
\begin{picture}(7,4.6)(-0.5,-0.8)
% 0-dim labels
\cell{0}{0}{c}{a_0}
\cell{1}{3}{c}{a_1}
\cell{5}{3}{c}{a_2}
\cell{6}{0}{c}{a_3}
% arrows
\put(0.05,0.4){\vector(1,3){0.75}}
\put(1.4,3){\vector(1,0){3}}
\put(5.2,2.65){\vector(1,-3){0.75}}
\put(0.5,0){\vector(1,0){4.9}}
\qbezier(0.4,0.3)(2.5,1.5)(4.6,2.7)
\put(4.6,2.7){\vector(3,2){0}}
% 1-dim labels
\cell{0.3}{1.5}{r}{\scriptstyle f_{01}}
\cell{3}{3.2}{b}{\scriptstyle f_{12}}
\cell{5.7}{1.5}{l}{\scriptstyle f_{23}}
\cell{3}{-0.2}{t}{\scriptstyle f_{03}}
\cell{2.2}{0.7}{c}{\scriptstyle f_{02}}
% 2-dim arrows
\cell{1.5}{2}{c}{\rotatebox{35}{$\Uparrow$}}
\cell{3.7}{1}{c}{\Uparrow}
% 2-dim labels
\cell{1.8}{2.2}{l}{\scriptstyle \alpha_{012}}
\cell{4}{0.9}{l}{\scriptstyle \alpha_{023}}
\end{picture}}
% 
\cell{10.5}{2.7}{t}{\Rrightarrow}
\cell{10.5}{3}{b}{\Gamma}
%
\cell{14}{0}{bl}{%
\begin{picture}(7,4.6)(-0.5,-0.8)
% 0-dim labels
\cell{0}{0}{c}{a_0}
\cell{1}{3}{c}{a_1}
\cell{5}{3}{c}{a_2}
\cell{6}{0}{c}{a_3\makebox[0em][l]{.}}
% arrows
\put(0.05,0.4){\vector(1,3){0.75}}
\put(1.4,3){\vector(1,0){3}}
\put(5.2,2.65){\vector(1,-3){0.75}}
\put(0.5,0){\vector(1,0){4.9}}
\qbezier(5.6,0.3)(3.5,1.5)(1.4,2.7)
\put(5.6,0.3){\vector(3,-2){0}}
% 1-dim labels
\cell{0.3}{1.5}{r}{\scriptstyle f_{01}}
\cell{3}{3.2}{b}{\scriptstyle f_{12}}
\cell{5.7}{1.5}{l}{\scriptstyle f_{23}}
\cell{3}{-0.2}{t}{\scriptstyle f_{03}}
\cell{3.8}{0.7}{c}{\scriptstyle f_{13}}
% 2-dim arrows
\cell{4.5}{2}{c}{\rotatebox{-35}{$\Uparrow$}}
\cell{2.3}{1}{c}{\Uparrow}
% 2-dim labels
\cell{4.2}{2.2}{r}{\scriptstyle \alpha_{123}}
\cell{2}{0.9}{r}{\scriptstyle \alpha_{013}}
\end{picture}}
% 
\end{picture}
% \hand{30}{66}.
\]
(Orientation needs care.)  This induces a functor 
\[
U: \strcat{\omega} \go \ftrcat{\Delta^\op}{\Set}
\]
and, roughly speaking, a weak $\omega$-category is defined as a simplicial
set with horn-filling%
%
\index{horn!filler for}
%
properties.  

This is, however, a slightly inaccurate account.  For similar reasons to
those in the opetopic case, $\Delta$ is not dense in $\strcat{\omega}$; the
functor $U$ is again faithful but not full.  Street's original solution was
to replace $\ftrcat{\Delta^\op}{\Set}$ by the category $\fcat{Sss}$ of
\demph{stratified%
%
\index{simplicial set!stratified}
%
simplicial sets}, that is, simplicial sets equipped with
a class of distinguished cells in each dimension (to be thought of as
`universal',%
%
\index{universal!cell of n-category@cell of $n$-category}
%
`hollow',%
%
\index{hollow}
%
or `thin').%
%
\index{thin}
%
 The underlying simplicial set of a
strict $\omega$-category has a canonical stratification, so $U$ lifts to a
functor
\[
U': \strcat{\omega} \go \fcat{Sss},
\]
and $U'$ \emph{is} full and faithful.  Detailed work by Street
\cite{StrAOS,StrFN} and Verity%
%
\index{Verity, Dominic}
%
(unpublished) gives precise conditions for
an object of $\fcat{Sss}$ to be in the image of $U'$.  So a strict
$\omega$-category is the same thing as a simplicial set equipped with a
class of distinguished cells satisfying some conditions.  One of the
conditions is that certain horns have unique fillers, and dropping the
uniqueness gives Street's proposed definition of weak $\omega$-category.

The most vexing aspect of this proposal is that extra structure is required
on the simplicial set.  It would seem more satisfactory if, as in the
opetopic approach, the universal cells could be recognized intrinsically.
A recent paper of Street~\cite{StrWOC} aims to repair this apparent defect,
proposing a similar definition in which a weak $\omega$-category is
genuinely a simplicial set with properties.%
%
\index{Street, Ross!definition of n-category@definition of $n$-category|)}%
%
\index{n-category@$n$-category!definitions of!Street's|)}%
%
\index{simplicial set!n-category as@$n$-category as|)}
%
\index{nerve!n-category@of $n$-category|)}
%





\minihead{Contractible multicategories}%
%
\index{multicategory!contractible|(}%
%
\index{contractible!multicategory|(}%
%
\index{globular multicategory|(}
%
%
\index{n-category@$n$-category!definitions of!contractible multicategory|(}
%

The last definition of weak $n$-category that we consider was introduced as
Definition $\mathbf{L'}$ in my~\cite{SDN} survey.  Like the other
definitions in this section, it is non-algebraic and can be described in
terms of nerves.  The nerve description seems, however, to be rather
complicated (the shapes involved being a combination of globular and
opetopic) and not especially helpful, so we approach it from another
angle instead.  

The idea is that a weak $\omega$-category is meant to be a `weak%
%
\index{algebra!monad@for monad!weak}%
%
\index{weakening!theory of n-categories@for theory of $n$-categories}
%
algebra'
for the free strict $\omega$-category monad on globular sets.  We saw
in~\ref{eg:multi-alg} that for any cartesian monad $T$, a \emph{strict}
$T$-algebra is the same thing as a $T$-multicategory whose domain map is
the identity---in other words, with underlying graph of the form
\[
\begin{diagram}[size=1.7em]
	&	&TX	&	&	\\
	&\ldTo<1&	&\rdTo>h&	\\
TX	&	&	&	&X.	\\  
\end{diagram}
\]
To define `weak $T$-algebra' we relax the condition that the domain map is
the identity, asking only that it be an equivalence in some sense.
Contractibility together with surjectivity on 0-cells is a reasonable
notion of equivalence: contractible means something like `injective%
%
\index{homotopy!injective on}
%
on
homotopy', and any contractible map surjective on $0$-cells is surjective
on $m$-cells for all $m\in\nat$.  For $1$-dimensional structures, it means
full, faithful and surjective on objects.  We also ask that the domain map
is injective on $0$-cells, expressing the thought that $0$-cells in an
$\omega$-category should not be composable.

Definition $\mathbf{L'}$ says, then, that a weak $\omega$-category is a
globular multicategory $C$ whose domain map $C_1 \go TC_0$ is bijective on
$0$-cells and contractible. 

Any weak $\omega$-category in the sense of the previous chapter gives rise
canonically to one in the sense of $\mathbf{L'}$.  This is the
`multicategory%
%
\index{generalized multicategory!elements@of elements}
%
of elements' construction of~\ref{sec:alg-fibs}: if $(T_L X
\goby{h} X)$ is an $L$-algebra then there is a commutative diagram
\[
% \begin{diagram}[size=1.7em]
\begin{slopeydiag}
	&		&T_L X		&		&	\\
	&\ldTo		&		&\rdTo>h	&	\\
TX	&		&\dTo		&		&X	\\
\dTo<{T!}&		&L		&		&\dTo>!	\\
	&\ldTo		&		&\rdTo		&	\\
T1	&		&		&		&1	\\
\end{slopeydiag}
% \end{diagram}
\]
the left-hand half of which is a pullback square and the top part of which
forms a multicategory $C^X$ with $C^X_0 = X$ and $C^X_1 = T_L X$.  The map
$L \go T1$ is contractible (by definition of $L$) and bijective on
$0$-cells (because, as we saw on p.~\pageref{p:L-blob}, if $\blob$ denotes
the $0$-pasting diagram then $L(\blob) = 1$).  Contractibility and
bijectivity on 0-cells are stable under pullback, so the multicategory
$C^X$ is a weak $\omega$-category in the sense of $\mathbf{L'}$.

Unpicking this construction explains further the idea behind $\mathbf{L'}$.
An $m$-cell of $T_L X$ is a pair 
\[
(\theta, \mathbf{x}) 
\in
\coprod_{\pi\in\pd(m)}
L(\pi) 
\times
\ftrcat{\scat{G}^\op}{\Set}(\rep{\pi}, X),
\]
which lies over $\mathbf{x} \in (TX)(m)$ and $\ovln{\theta}(\mathbf{x}) \in
X(m)$.  It is usefully regarded as a `way of composing' the labelled
pasting diagram $\mathbf{x}$.  (Contractibility guarantees that there are
plenty of ways of composing.)  Among all weak $\omega$-categories in the
sense of $\mathbf{L'}$, those of the form $C^X$ have the special feature
that the set of ways of composing a labelled pasting diagram $\mathbf{x}
\in (TX)(\pi)$ depends only on the pasting diagram $\pi$, not on the
labels: it is just $L(\pi)$.

So in the definition of the previous chapter, the ways of composition
available in a weak $\omega$-category are prescribed once and for all; in
the present definition, they are allowed to vary from $\omega$-category to
$\omega$-category.  This is precisely analogous to the difference between
the loop%
%
\index{loop space!machine}
%
space machinery of Boardman--Vogt%
%
\index{Boardman, Michael}%
%
\index{Vogt, Rainer}
%
and May%
%
\index{May, Peter}
%
(with fixed parameter
spaces forming an operad) and that of Segal%
%
\index{Segal, Graeme}
%
(with a variable, flabby,%
%
\index{flab}
%
structure): see Adams~\cite[p.~60]{Ad}.%
%
\index{multicategory!contractible|)}%
%
\index{contractible!multicategory|)}%
%
\index{globular multicategory|)}
%
\index{n-category@$n$-category!definitions of!contractible multicategory|)}
%



\minihead{Locally groupoidal structures}%
%
\index{n-category@$n$-category!locally groupoidal|(}%
%
\index{locally groupoidal structures|(}
%

Many of the weak $\omega$-categories of interest in geometry have the
property that all cells of dimension $2$ and higher are equivalences
(weakly invertible).%
%
\index{invertibility}
%
 Some examples were given in `Motivation for
Topologists'.  There are several structures in use that, roughly speaking,
aim to formalize the idea of such a weak $\omega$-category.  I will
describe three of them here.

A weak $\omega$-category in which all cells (of dimension $1$ and higher)
are equivalences is called a \demph{weak $\omega$-groupoid}.%
%
\index{omega-groupoid@$\omega$-groupoid}
%
  This is, of
course, subject to precise definitions of weak $\omega$-category and
equivalence.  From the topological viewpoint one of the main purposes of
$\omega$-groupoids is to model homotopy%
%
\index{homotopy!type}%
%
\index{fundamental!omega-groupoid@$\omega$-groupoid}
%
types of spaces (see Grothendieck's
letter of~\cite{GroPS}, for instance), so it is reasonable to replace
$\omega$-groupoids by spaces, or perhaps simplicial%
%
\index{simplicial set!omega-category from@$\omega$-category from}
%
sets or chain
complexes.%
%
\index{chain complex!omega-category from@$\omega$-category from}
%
  The structures we seek are, therefore, graphs $(X(x,x'))_{x, x'
\in X_0}$ of spaces, simplicial sets, or chain complexes, together with
extra data determining some kind of weak composition.

We have already seen one version of this: an $A_\infty$-category%
%
\index{A-@$A_\infty$-!category}
%
(p.~\pageref{p:A-infty-category}) is a graph $X$ of chain complexes together
with various composition maps
\[
X(x_{k-1}, x_k) \otimes \cdots \otimes X(x_0, x_1)
\go
X(x_0, x_k)
\]
parametrized by elements of the operad $A_\infty$.  There is a similar
notion with spaces in place of complexes.  These are algebraic definitions
(so properly belong in the previous section).

A similar but non-algebraic notion is that of a Segal
category (sometimes
called by other names: see the Notes below).  Take a bisimplicial%
%
\index{bisimplicial set}%
%
\index{simplicial set!bi-}
%
set,
expressed as a functor
\[
X: \Delta^\op \go \ftrcat{\Delta^\op}{\Set},
\]
and suppose that the simplicial set $X\upr{0}$ is discrete (that is, the
functor $X\upr{0}$ is constant).  Write $X_0$ for the set of points
(constant value) of $X\upr{0}$ .  Then for each $k\in\nat$, the simplicial
set $X\upr{k}$ decomposes naturally as a coproduct
\[
 X\upr{k} 
\iso 
\coprod_{x_0, \ldots, x_k \in X_0} 
X(x_0, \ldots, x_k),
\]
and for each $k\in\nat$ and $x_0, \ldots, x_k \in X_0$, there is a natural
map
\[
X(x_0, \ldots, x_k)
\go
X(x_{k-1}, x_k) \times \cdots \times X(x_0, x_1).
\]
A bisimplicial set $X$ is called a \demph{Segal%
%
\index{Segal, Graeme!category}
%
 category} if $X\upr{0}$ is
discrete and each of these canonical maps is a weak equivalence of
simplicial sets, in the homotopy-theoretic sense.  This definition is very
closely related to the definitions of $n$-category proposed by Simpson%
%
\index{Simpson, Carlos!definition of n-category@definition of $n$-category}%
%
\index{n-category@$n$-category!definitions of!Simpson's}%
and
Tamsamani,%
%
\index{Tamsamani, Zouhair!definition of n-category@definition of $n$-category}%
%
\index{n-category@$n$-category!definitions of!Tamsamani's}%
%
 and in particular to the definition of homotopy%
%
\index{bicategory!homotopy}
%
bicategory in
Chapter~\ref{ch:monoidal}.

Finally, there are the quasi-categories of Joyal, Boardman, and Vogt.  We
have a diagram
\[
\begin{diagram}[size=1.7em]
	&	&\{\textrm{quasi-categories}\}	&	&	\\
	&\ruIncl&			&\luIncl&	\\
\{\textrm{categories}\}&&		&	&
\{\textrm{Kan complexes}\}\\
	&\luIncl&			&\ruIncl&	\\
	&	&\{\textrm{groupoids}\}&	&	\\
\end{diagram}%
%
\index{groupoid}
%
\]
of classes of simplicial sets, in which
%
\begin{itemize}
\item categories are identified with simplicial sets in which every inner
  horn%
%
\index{horn!filler for}
%
has a unique filler~(\ref{eg:nerve-cat-chars})
\item \demph{Kan%
%
\index{Kan, Daniel!complex}
%
complexes} are simplicial sets in which every horn
  has at least one filler (the principal example being the underlying
  simplicial set of a space)
\item at the intersection, groupoids are simplicial sets in which every
  horn has a unique filler
\item at the union, \demph{quasi-categories}%
%
\index{quasi-category}%
%
\index{category!quasi-}
%
%
are simplicial sets in which
  every inner horn has at least one filler.
\end{itemize}
%
Large amounts of the theory of ordinary categories can be reproduced for
quasi-categories, although requiring much longer proofs.  Given a
simplicial set $X$ and elements $x, x' \in X\upr{0}$, there is a simplicial
set $X(x, x')$, the analogue of the space of paths from $x$ to $x'$ in
topology, and it can be shown that if $X$ is a quasi-category then each
$X(x, x')$ is a Kan complex.  So quasi-categories do indeed approximate the
idea of a weak $\omega$-category in which all cells of dimension at least
$2$ are invertible.%
%
\index{n-category@$n$-category!locally groupoidal|)}%
%
\index{locally groupoidal structures|)}
%





\begin{notes}

An extensive bibliography and historical discussion of proposed
definitions of $n$-category is in my survey paper~\cite{SDN}. 

Various people have confused the contractions%
%
\index{contraction!notions of}
%
of Chapter~\ref{ch:a-defn}
with the contractions of Batanin, principally me (see the Notes to
Chapter~\ref{ch:a-defn}) but also Berger~\cite[1.20]{Ber}.%
%
\index{Berger, Clemens}
%
 I apologize to
Batanin for stealing his word, using it for something else, then
renaming his original concept~(\ref{defn:coh-coll}).

Berger~\cite{Ber}%
%
\index{Berger, Clemens}
%
has investigated nerves of $\omega$-categories in detail,
making connections to various proposed definitions of weak
$\omega$-categories, especially Joyal's.  Other work on higher-dimensional
nerves has been done by Street%
%
\index{Street, Ross}
%
(\cite[\S 10]{StrCS} and references therein)
and Duskin%
%
\index{Duskin, John}
%
\cite{DusSM1, DusSM2}.  Joyal's definition has also been
illuminated by Makkai%
%
\index{Makkai, Michael}
%
and Zawadowski~\cite{MZ}%
%
\index{Zawadowski, Marek}
%
and Batanin%
%
\index{Batanin, Michael}
%
and
Street~\cite{BSUPM}.%
%
\index{Street, Ross}
%

The observation that most geometrically interesting $\omega$-categories are
locally groupoidal was made to me by Bertrand To\"en.%
%
\index{To\"en, Bertrand}
%
 His~\cite{TV} paper
with Vezzosi%
%
\index{Vezzosi, Gabriele}
%
gives an introduction to Segal categories, as well as further
references.  In particular, they point to Dwyer,%
%
\index{Dwyer, William}
%
Kan%
%
\index{Kan, Daniel}
%
and Smith~\cite{DKS},%
%
\index{Smith, Jeffrey}
%
where Segal categories were called special%
%
\index{bisimplicial set!special}%
%
\index{simplicial set!bi-!special}%
%
\index{special}
%
bisimplicial sets, and
Schw\"anzl%
%
\index{Schwanzl@Schw\"anzl, Roland}
%
and Vogt~\cite{SV},%
%
\index{Vogt, Rainer}
%
where they were called $\Delta$-categories.%
%
\index{Delta-category@$\Delta$-category}%
%
\index{category!Delta-@$\Delta$-}
%
%

Joyal's quasi-categories are a renaming of Boardman%
%
\index{Boardman, Michael}
%
and Vogt's%
%
\index{Vogt, Rainer}
%
`restricted%
%
\index{Kan, Daniel!complex}
%
Kan complexes'~\cite{BV}.  His work on quasi-categories
remains unpublished, but has been presented in seminars since 1997 or
earlier.

I thank Sjoerd Crans for useful conversations on the details of Batanin's
definition, Jacques Penon for the observation that coherent maps do not
compose, and Michael Batanin and Ross Street for useful comments on their
respective definitions.


\end{notes}
