\section{Thermal correlators in 2d CFTs}
\label{appendix2dthermal}
\subsection{Modes in a BTZ black hole}
In our discussion above we frequently referred to modes in the black
hole background. In this appendix, we explore these modes in the
background of a BTZ black hole. This will allow us to check several
claims including their behaviour near the horizon and their growth for
large spacelike momenta that we made above. 

Let us start with the usual BTZ coordinates
\[
ds^2 = -(r^2-r_h^2)dt^2 + (r^2-r_h^2)^{-1}dr^2 + r^2 dx^2.
\]
The horizon is at $r_h$ and $x$ runs from $-\infty$ to $+\infty$ i.e. we are looking at a planar BTZ black hole. The temperature of the black hole is
\[
\beta = {2\pi \over r_h}
\]
We consider a massive scalar whose dual operator has dimension $\Delta$, obeying the equation
\[
(\Box -m^2)\phi=0
\]
Solving the KG equation with an ansatz of the form $e^{-i\omega t}e^{i k x}\psi(r)$ we find two linearly independent solutions
\[
\psi_1(r)=\left({r^2\over r_h^2}\right)^a\left({r^2\over r_h^2}-1\right)^b \, _2F_1\left(1+a+b-{\Delta \over 2},a+b+{\Delta \over 2},1+2b,1-{r^2\over r_h^2}\right)
\]
\[
\psi_2(r) =\left({r^2\over r_h^2}\right)^{a}\left({r^2\over r_h^2}-1\right)^{-b}\, _2F_1\left(1+a-b-{\Delta
\over 2},a-b+{\Delta \over 2},1-2b,1-{r^2 \over r_h^2} \right)
\]
where 
\[
a = {i k \over 2 r_h}\qquad b= {i \omega \over 2r_h}
\]
For any given $\omega,k$ only a specific linear combination of these modes is normalizable at infinity. Using hypergeometric identities we find that the normalizable combination is
\[
\psi_1(r) -\left[{\Gamma(1+2b)\Gamma(a-b+{\Delta\over 2})
\Gamma(-a-b+{\Delta\over 2})\over \Gamma(1-2b) \Gamma(-a+b+{\Delta \over 2})
\Gamma(a+b+{\Delta \over 2})}\right]\psi_2(r)
\]
Using some hypergeometric identities and fixing the overall normalization we can rewrite the normalizable mode as
\[
\hat{\psi}_{\omega,k}(r) = {1\over \Gamma(\Delta)\sqrt{r_h}}\sqrt{\Gamma(a+b+\Delta/2)\Gamma(a-b+\Delta/2)
\Gamma(-a+b+\Delta/2)\Gamma(-a-b+\Delta/2)\over \Gamma(2b)\Gamma(-2b)}
\]
\[
\times \left({r^2\over r_h^2}\right)^a \left({r^2\over r_h^2}-1 \right)^{-a-{\Delta\over 2}}
{ \, _2 F_1\left(a-b+\Delta/2,a+b+\Delta/2,\Delta,{r_h^2 \over r_h^2-r^2}\right)}
\]
The overall normalization was fixed so that the modes are canonically normalized with respect to the Klein Gordon norm, which is why we use the hatted notation --- as discussed in section
\ref{modesbrane}. In particular, near the horizon we have the expansion\footnote{The overall factor of $r_h^{-1/2}$ in the near horizon normalization, is automatically fixed if we 
require the modes defined by \eqref{psinearhapp}, \eqref{btzmodes} to have canonical Klein-Gordon norm. }
\be
\label{psinearhapp}                          
\hat{\psi}_{\omega,k}(r) = r_h^{-1/2} \left(e^{i\delta_{\omega,k}}e^{i\omega r_*} + e^{-i \delta_{\omega,k}} e^{-i\omega r_*}\right)
\ee
where $r_*$ is the tortoise coordinate (in which horizon is at $r_*\rightarrow -\infty$) 
\[
 r_* = {1\over 2 r_h}\log\left(r-r_h\over r+r_h\right)
\]
and the ``phase shift'' is
\[
e^{i \delta_{\omega,k}} = 4^{b} \sqrt{\Gamma(-2b) \,\Gamma(a+b+\Delta/2)\,\Gamma(-a+b+\Delta/2)\over\Gamma(2b)\,
\Gamma(-a-b+\Delta/2)\,\Gamma(a-b+\Delta/2)}
\]
To be more precise the complete bulk mode is
\be
\label{btzmodes}
\hat{f}_{\omega,k}(t,x,r) = \hat{\psi}_{\omega,k}(r) e^{-i\omega t} e^{i k x}
\ee
here both the frequency $\omega$ and the momentum $k$ are continuous.

For the bulk operator field we have
\be
\label{bulkfieldbtz}
\phi(t,x,r) = 
\int_{\omega>0} {d\omega\,dk  \over (2 \pi)^2} {1\over \sqrt{2\omega}}
\left(a_{\omega,k}\hat{f}_{\omega,k}(t,x,r) + {\rm h.c.}\right)
\ee
with
\[
[a_{\omega,k},a^\dagger_{\omega',k'}] = \delta(\omega-\omega')\delta(k-k')
\]
In the AdS Hartle-Hawking state we have
\[
\begin{split}
&\langle a_{\omega,k} \, a^\dagger_{\omega',k'}\rangle_{\rm HH} = {e^{\beta \omega}\over e^{\beta \omega}-1} \delta(\omega-\omega')\delta(k-k')\\
&\langle a_{\omega,k}^\dagger \,a_{\omega',k'}\rangle_{\rm HH} = {1\over e^{\beta \omega}-1} \delta(\omega-\omega')\delta(k-k')
\end{split}
\]
Using these formulas we can compute the bulk 2-point function and then by taking the bulk points to the boundary, we can recover the boundary 2-point function
\[
G_\beta(t,x;t',x') = \lim_{r,r'\rightarrow \infty}  \left[\normfact^2 r^{\Delta}(r')^{\Delta} \langle \phi(t,x,r) \phi(t',x',r')\rangle_{\rm HH}\right],
\]
where $\normfact$ is given in \eqref{normfactdef}. 
In momentum space, and using time- and space-translational invariance we have
\[
G_\beta(\omega,k) = \int dt\,dx\, e^{i\omega t - i k x} G_\beta(t,x;0,0)
\]
From the previous results we find
\be
\label{btzfourp}
G_\beta(\omega,k) = 
\normfact^2 {1\over 2 \pi (2\omega)}{ e^{\beta \omega} \over e^{\beta \omega}-1}\left({2\pi \over \beta}\right)^{2\Delta-1}\left|{\Gamma\left(i{\beta(\omega+k)\over 4\pi}+{\Delta\over 2}\right)\Gamma\left(i{\beta(\omega-k)\over 4\pi}+{\Delta\over 2}\right)
\over  \Gamma(\Delta)\Gamma\left(i{\beta\omega\over 2\pi}\right)}\right|^2\qquad,\qquad \omega>0
\ee
and
\be
\label{btzfourn}
G_\beta(-\omega,k) = 
\normfact^2 {1\over 2 \pi (2\omega)}{ 1 \over e^{\beta \omega}-1}\left({2\pi \over \beta}\right)^{2\Delta-1}\left|{\Gamma\left(i{\beta(\omega+k)\over 4\pi}+{\Delta\over 2}\right)\Gamma\left(i{\beta(\omega-k)\over 4\pi}+{\Delta\over 2}\right)
\over  \Gamma(\Delta)\Gamma\left(i{\beta\omega\over 2\pi}\right)}\right|^2\quad,\qquad \omega>0
\ee
Actually although the two forms above are useful to see the qualitative
properties of the solution we can rewrite them in a simpler form using 
\be
\left| \Gamma\left(i{\beta\omega\over 2\pi}\right) \right|^2 = {2 \pi^2 \over \beta |\omega|} {1 \over e^{\beta |\omega| \over 2} - e^{-\beta |\omega| \over 2}}
\ee
This leads to the expression
\be
G_{\beta}(\omega, k) = {\normfact^2 \over (2 \pi)^2}  e^{\beta \omega \over 2} 
\left(2 \pi \over \beta \right)^{2 \Delta - 2} \left|{\Gamma\left(i{\beta(\omega+k)\over 4\pi}+{\Delta\over 2}\right)\Gamma\left(i{\beta(\omega-k)\over 4\pi}+{\Delta\over 2}\right) \over  \Gamma(\Delta)}\right|^2,
\ee
and this expression is valid for $\omega$ both positive and negative.

In the next subsection we will rederive these expressions from the boundary CFT, using the constraints from 2d conformal invariance. Here we notice that these results manifestly 
satisfy the general properties we mentioned in section \ref{sec:outside}:

i) It is obvious from the expressions above that the KMS condition is satisfied 
\[
 G(-\omega,k) = e^{-\beta \omega} G(\omega,k)
\]

ii) It can be checked that the 2-point function $G(\omega,k)$ is exponentially suppressed for large spacelike momenta, that is for fixed $\omega$ and large $k$ we have
\[ 
 G(\omega,k) \underset{|k| \rightarrow \infty}{ \lessapprox} e^{-{\beta k \over 2}}
\]
as expected from our general arguments in section \ref{subsec:thermalfourier}.

iii) It can be checked that in the limit of low temperature ($\beta\rightarrow \infty$) the thermal Wightman function $G_\beta(\omega,k)$ reduces to the zero-temperature one 
that we found in section \ref{sec:emptyads}, that is
$$
G_\beta(\omega,k) \approx N_\Delta \,\theta(\omega)\,\theta(\omega^2-k^2)\, (\omega^2 - k^2)^{\Delta-1} ,\qquad {\rm for}\quad \beta\rightarrow \infty
$$
\subsection{Boundary correlators in 2 dimensions}
The two
dimensional CFT correlator at finite temperature, can be completely fixed using conformal invariance and we will rederive the results \eqref{btzfourp}, \eqref{btzfourn} directly from the boundary.
We start with the Euclidean correlator. We put the CFT on ${\mathbb R}^1\times {\mathbb S}^1$ where the perimeter of the circle is $\beta$. We then have
\be
\label{exact2dcorrelator}
\langle {\cal O}(\tau,x) {\cal O}(0,0)\rangle_{\beta} = \left({2\pi \over \beta} \right)^{2\Delta} \left[2 \cosh\left({2\pi x \over \beta}\right)- 2 \cos\left({2\pi \tau \over \beta}\right)\right]^{-\Delta}
 \ee
By taking the short distance expansion we can check that the normalization is correct.
However, we can now use modular invariance and understand this to be
the thermal correlator of a CFT on flat space at an inverse
temperature $\beta$.  

Now, let us continue to Lorentzian space, following the same logic as in section \ref{sec:emptyads}. (See also \cite{Skenderis:2008dg}.)
We find that
the time-ordered correlator is given by
\[
\langle T\left\{{\cal O}(t,x), {\cal O}(0,0) \right\} \rangle_{\beta} = \left({2 \pi \over \beta} \right)^{2 \Delta}  \left[2 \cosh \left({2 \pi x \over \beta} \right) - 
2 \cosh \left( {2 \pi (1 - i \epsilon) t \over \beta} \right) \right]^{-\Delta}
 \]
and for the Wightman correlator, we have
\[
\begin{split}
&\langle {\cal O}(t,x), {\cal O}(0,0) \rangle_{\beta} = \left({2 \pi \over \beta} \right)^{2 \Delta}  \left[2 \cosh \left({2 \pi x \over \beta} \right) - 
2 \cosh \left( {2 \pi (t - i \epsilon) \over \beta} \right)  \right]^{-\Delta} \\
&= \left({2 \pi \over \beta} \right)^{2 \Delta} 2^{-2 \Delta} \left[ \sinh {\pi (u - i \epsilon) \over \beta} \sinh{\pi (v - i \epsilon) \over \beta} \right]^{-\Delta}
\end{split}
 \]
where we defined $u=t-x,\,v=t+x$. So, the basic integral that we are interested in is
\be
\label{ibetadef}
I_{\beta}(k_{+}) = \left({2 \pi \over \beta} \right)^{ \Delta} 2^{- \Delta} \int_{-\infty}^{\infty} e^{i k_{+} u}  \left(\sinh{\pi (u - i \epsilon) \over \beta} \right)^{-\Delta} d u
\ee

Let us understand the qualitative properties of this integral. We see that the integral has branch cuts running along the segments $(-\infty + i n \beta + i \epsilon, i n \beta + i \epsilon)$ for integer $n$. Now, if $k_{+} > 0$, then we can move the $u$ contour down to the first branch cut, which is at ${\rm Im}~u = - i \beta$. So, for negative $k_{+}$, we get a contribution proportional to
$e^{-k_+ \beta}$ i.e. the integral is exponentially damped for large negative $k_{+}$.  On the other hand, for positive $k_{+}$, there is no such damping. 

To evaluate the integral precisely, we write it as
\be
\begin{split}
I_{\beta}(k_{+}) &= \left({2 \pi \over \beta} \right)^{\Delta} 2^{-\Delta} \left[e^{i \pi \Delta} \int_{-\infty}^{0} d u \, e^{i k_{+} u} \left| \sinh {\pi u \over \beta} \right|^{-\Delta} + \int_0^{\infty} d u \, e^{i k_{+} u} \left| \sinh {\pi u \over \beta} \right|^{-\Delta} \right],\\
\end{split}
\ee
where the phase factors come from the $i \epsilon$ prescription explained in section \ref{sec:emptyads}.
This integral can be performed analytically to obtain
\be
\begin{split}
I_{\beta}(k_{+})
&=  \left({2 \pi \over \beta} \right)^{ \Delta} {\pi^2 \beta \over 4} {\csc (\pi \Delta )  \over  \Gamma (\Delta )} \left[ e^{i \pi \Delta} \frac{ \Gamma \left(\frac{1}{2} \left(\frac{i k_{+} \beta}{\pi }+\Delta \right)\right)}{\Gamma \left(\frac{i k_{+}
   \beta }{2 \pi }-\frac{\Delta }{2}+1\right) } 
+ \frac{  \Gamma \left(\frac{1}{2} \left(\Delta -\frac{i k_{+} \beta }{\pi }\right)\right)}{\Gamma \left(-\frac{i k_{+} \beta }{2 \pi }-\frac{\Delta }{2}+1\right)} \right] \\
&= \left({2 \pi \over \beta} \right)^{ \Delta} {\pi^3 \beta \over 4} \frac{ \left(\coth \left(\frac{1}{2} (k_{+} \beta +i \pi  \Delta )\right)+1\right) \text{csch}\left(\frac{1}{2} (k_{+} \beta -i   \pi  \Delta )\right)}{\Gamma \left(-\frac{i k_{+} \beta }{2 \pi }-\frac{\Delta }{2}+1\right) \Gamma \left(\frac{i k_{+}\beta }{2 \pi }-\frac{\Delta }{2}+1\right) \Gamma (\Delta )} \\
&= \left({2 \pi \over \beta} \right)^{ \Delta - 1} {\pi^2 \over 2} \frac{ e^{i \pi \Delta + k_{+} \beta \over 2} \Gamma \left(\frac{\Delta }{2} + \frac{i k_{+} \beta }{2 \pi } \right) \Gamma \left(\frac{\Delta }{2} - \frac{i k_{+}\beta }{2 \pi }\right)}{ \Gamma (\Delta )}.
\end{split}
\ee

Since the full answer for the Green function is given by
\be
G_{\beta}(\omega, k) =  I_{\beta}({k + \omega \over 2}) I_{\beta}({k - \omega \over 2}),
\ee
and since, in this case $\normfact^2 = 2 \pi (2 \pi)^2$, 
we see that our boundary calculation matches precisely with the answer
from the bulk up to a momentum and temperature independent pre-factor.
% that we did not keep track of above.
%were not careful about above.
%is independent of both the momentum and the temperature.
%numerical pre-factor that is independent of both the momentum and temperature. 

%%% Local Variables: 
%%% mode: latex
%%% TeX-master: "infalling_paper"
%%% End: 






