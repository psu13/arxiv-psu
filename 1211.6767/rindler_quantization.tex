\section{Quantization in Rindler space \label{rindlerq}}

  
\subsection{Expansion in Rindler modes}
We start with $d+1$ dimensional Minkowski space
\[
ds^2 = -dt^2 + dz^2 + d\vect{x}^2
\]
and consider a massless scalar field obeying
\[
 \Box \phi = 0
\]

\subsubsection{Region I}

We first expand the field in modes in region I. In that region the Rindler coordinates $(\tau,\sigma,\vect{x})$ are defined by $t= \sigma \sinh \tau,\, z = \sigma \cosh t$. The metric looks like
\[
ds^2 = -\sigma^2 d\tau^2 + d\sigma^2 + d\vect{x}^2
\]
The field in region I has the expansion
\[
\phi(\tau,\sigma,\vect{x}) = \int_{\omega>0}{d\omega d^{d-1}\vect{k} \over (2 \pi)^d} \left[{1\over \sqrt{2\omega}} a_{\omega,\vect{k}} e^{-i\omega \tau+i \vect{k} \vect{x}}{2 K_{i\omega}(|\vect{k}| \sigma)
\over |\Gamma(i\omega)|} + {\rm h.c.}
\right]
\]
Notice that the Bessel function $K_{i\omega}(|\vect{k}| \sigma)$ is real. 

We define the lightcone coordinates in Minkowski space
\[
U = t-z
\]
\[
V= t+z
\]
In region I we have $U<0,V>0$. In terms of the Rindler coordinates in region I we have
\[
U = - \sigma e^{-\tau}\]
\[
V = \sigma e^{\tau}
\]
Considering the field expansion near the horizon between regions I and II we find
\[
\phi  \underset{U \rightarrow 0}{\approx} \int_{\omega>0}{d\omega d^{d-1}\vect{k} \over (2 \pi)^d}{1\over \sqrt{2\omega}} a_{\omega\vect{k}} e^{i \vect{k} \vect{x}} \left[V^{-i\omega} e^{-i\delta} + (-U)^{i\omega}e^{i\delta}\right]
\]
\[
+ {1\over \sqrt{2\omega}} a_{\omega,\vect{k}}^\dagger e^{-i \vect{k} \vect{x}} \left[V^{i\omega} e^{i\delta} + (-U)^{-i\omega}e^{-i\delta}\right]
\]
where the phase shift is
\begin{equation}
\label{phase}
e^{i\delta} = \left({|\vect{k}|\over 2}\right)^{i\omega} {\Gamma(-i\omega) \over |\Gamma(i\omega)|}
\end{equation}
Due to the Riemann-Lebesgue lemma, when $U\rightarrow 0$ terms like $U^{-i \omega}$ can be discarded. So what is important is to keep the terms that depend on $V$ only, and we have
\begin{equation}
\label{boundaryIandII}
\phi  \underset{U \rightarrow 0}{\approx}\int_{\omega>0}{d\omega d^{d-1}\vect{k} \over (2 \pi)^d} {1\over \sqrt{2\omega}} \left[a_{\omega,\vect{k}} e^{i \vect{k} \vect{x}} V^{-i\omega} e^{-i\delta} + 
 a_{\omega,\vect{k}}^\dagger e^{-i \vect{k} \vect{x}} V^{i\omega} e^{i\delta} \right]
\end{equation}

\subsubsection{Region III}

Here the coordinates are $t= -\sigma \sinh \tau,\,\,z=-\sigma \cosh \tau$. We notice that the Rindler time $\tau$ runs opposite of the Minkowski time $t$. Also,
notice that due to our parametrization we still have $\sigma>0$ even though $z<0$ in region III. We expand the field as
\[
\phi(\tau,\sigma,\vect{x}) = \int_{\omega>0}{d\omega d^{d-1}\vect{k} \over (2 \pi)^d} \left[{1\over \sqrt{2\omega}} \wa_{\omega,\vect{k}} e^{i\omega \tau-i \vect{k} \vect{x}}{2 K_{i\omega}(|\vect{k}| \sigma)
\over |\Gamma(i\omega)|} + {\rm h.c.} \right]
\]
Notice that we have defined the modes $\wa_{\omega,\vect{k}}$ to multiply the function which is the conjugate of the one before.
We want to expand the field near the horizon between regions III and II. Again in Minkowski lightcone coordinates we have
\[
U = \sigma e^{-\tau}
\]
\[
V = -\sigma e^\tau
\]
The horizon is now at $V\rightarrow 0$. Following the same steps as for region I we find that near the horizon the non-vanishing terms are
\begin{equation}
\label{boundaryIIandIII}
\phi \underset{V \rightarrow 0}{\approx} \int_{\omega>0}{d\omega d^{d-1}\vect{k} \over (2 \pi)^d} {1\over \sqrt{2\omega}}\left[ \wa_{\omega,\vect{k}} e^{-i \vect{k} \vect{x}} U^{-i\omega} e^{-i\delta}
+\wa^\dagger_{\omega,\vect{k}} e^{i\vect{k}\vect{x}} U^{i\omega}e^{i\delta}   \right]
\end{equation}
the phase factor is again given by \eqref{phase}.

\subsubsection{Region II}

Here we choose the Rindler coordinates as $t =\sigma  \cosh \tau,\,\, z = \sigma \sinh \tau$. The horizon between I and II is at $\tau\rightarrow +\infty$. In terms of the lightcone coordinates
we have
\[
U = \sigma e^{-\tau}
\]
\[
V = \sigma e^{\tau}
\]
We write the general expansion 
as
\[
\phi = \int_{\omega>0}{d\omega d^{d-1}\vect{k} \over (2 \pi)^d} {1\over \sqrt{2\omega}}\left[A_{\omega,\vect{k}}e^{-i \omega \tau + i \vect{k}\vect{x}} J_{i\omega}(|\vect{k}|\sigma) +
B_{\omega,\vect{k}} e^{-i \omega t + i \vect{k} \vect{x}} J_{-i \omega}(|\vect{k}|\sigma) + {\rm h.c.} \right]
\]
For the $J$ Bessel functions as $\sigma\rightarrow 0$ we have
\[
J_{i\omega}(|\vect{k}|\sigma) \approx {1\over \Gamma(1+i\omega)} \left({|\vect{k}|\sigma \over 2}\right)^{i\omega} +\ldots
\]
Now we are looking at the expansion closed to the horizon between I and II. There we have $U\rightarrow 0$, 
so we keep only the terms which depend on $V$ and we find
\[
\phi \underset{U \rightarrow 0}{\approx} \int_{\omega>0}{d\omega d^{d-1}\vect{k} \over (2 \pi)^d} {1\over \sqrt{2\omega}}
\left[
{B_{\omega,\vect{k}}\over \Gamma(1-i\omega)} e^{i\vect{k} \vect{x}}\left({|\vect{k}|\over 2}\right)^{-i\omega} V^{-i\omega}+{\rm h.c.}
\right]
\]
Comparing with the expansion \eqref{boundaryIandII} we find that
\[
B_{\omega,\vect{k}} = {\Gamma(1-i\omega) \Gamma(i\omega) \over |\Gamma(i\omega)|}a_{\omega,\vect{k}}= -i\sqrt{\pi \omega \over \sinh(\pi \omega)} a_{\omega,\vect{k}}
\]
While, looking at the horizon between II and III we find
\[
\phi \underset{V \rightarrow 0}{\approx} \int_{\omega>0} {d\omega d^{d-1} \vect{k}  \over (2 \pi)^d} {1\over \sqrt{2\omega}} \left[{A_{\omega,\vect{k}}\over \Gamma(1+i\omega)}e^{i\vect{k}\vect{x}}
\left({|\vect{k}|\over 2}\right)^{i\omega}U^{i\omega} + {\rm h.c.}
\right]
\]
Comparing with \eqref{boundaryIIandIII} we find
\[
A_{\omega,\vect{k}} = {\Gamma(1+i\omega)\Gamma(-i\omega) \over |\Gamma(i\omega)|}\widetilde{a}_{\omega,\vect{k}}^\dagger
=i\sqrt{\pi \omega \over \sinh(\pi \omega)} \widetilde{a}_{\omega,\vect{k}}^\dagger
\]
Putting everything together we find the expansion in region II
\be
\label{regionIIrindler}
\phi(\tau,\sigma,\vect{x}) = \int_{\omega>0}{d\omega d^{d-1}\vect{k}  \over (2 \pi)^d} {1\over \sqrt{2\omega}}\sqrt{\pi \omega\over \sinh(\pi \omega)} \left[i\,\widetilde{a}^\dagger_{\omega,\vect{k}}
\,e^{-i\omega \tau+ i \vect{k} \vect{x}}\, J_{i\omega}(|\vect{k}|\sigma) 
- i a_{\omega,\vect{k}} e^{-i\omega \tau + i \vect{k} \vect{x}} J_{-i\omega}(|\vect{k}|\sigma)+ {\rm h.c.}\right]
\ee 	
\subsubsection{Bogoliubov transformation}

Now we express the Rindler modes in terms of the Unruh modes $d^{1,2}_{\omega,\vect{k}}$. We have
\be
\label{unruh}
d^1_{\omega,\vect{k}} = {a_{\omega,\vect{k}} - e^{-\pi \omega}\, \widetilde{a}_{\omega,\vect{k}}^\dagger \over \sqrt{1-e^{-2\pi \omega}}}\qquad,\qquad 
d^2_{\omega,\vect{k}} = {\widetilde{a}_{\omega,-\vect{k}} - e^{-\pi \omega}\, a_{\omega,-\vect{k}}^\dagger \over \sqrt{1-e^{-2\pi \omega}}}
\ee
or inverting
\be
\label{unruhinv}
a_{\omega,\vect{k}} = {d^1_{\omega,\vect{k}} + e^{-\pi \omega}(d^2_{\omega,-\vect{k}})^\dagger\over \sqrt{1-e^{-2\pi \omega}}}\qquad,\qquad
\widetilde{a}_{\omega,\vect{k}} = {d^2_{\omega,-\vect{k}}+ e^{-\pi \omega} (d^1_{\omega,\vect{k}})^\dagger \over  \sqrt{1-e^{-2\pi \omega}}}
\ee
The Minkowski vacuum $|0\rangle$ is defined by $d^{1,2}_{\omega,\vect{k}}|0\rangle =0$. The Rindler mode occupation levels are
\[
 \langle 0| a_{\omega,\vect{k}} a^\dagger_{\omega',\vect{k}'}|0\rangle ={e^{2\pi \omega} \over e^{2\pi \omega}-1}\delta(\omega-\omega')\delta^{d-1}(\vect{k}-\vect{k}')\,\,,\,\,
  \langle 0| a_{\omega,\vect{k}}^\dagger a_{\omega',\vect{k}'}|0\rangle ={1 \over e^{2\pi \omega}-1}\delta(\omega-\omega')\delta^{d-1}(\vect{k}-\vect{k}')
\]
and similarly for the $\widetilde{a}_{\omega,\vect{k}}$ modes.


\subsection{2-point function in terms of Rindler modes}


\subsubsection{Region I}

We can now write the usual Wightman 2-point function of a scalar field in terms of the Rindler modes. For  points in region I we have
\[
 \langle 0| \phi(\tau_1,\sigma_1,\vect{x}_1) \phi(\tau_2,\sigma_2,\vect{x}_2)|0\rangle = (2\pi)^d \int_{\omega>0}{d\omega d^{d-1}\vect{k}  \over (2 \pi)^d}\]
 \[{1\over 2\omega}
 \Bigg[{e^{2\pi \omega}\over e^{2\pi \omega}-1} {4K_{i\omega}(|\vect{k}|\sigma_1)K_{i\omega}(|\vect{k}|\sigma_2)
\over |\Gamma(i\omega)|^2}e^{-i\omega \tau_{12}+i\vect{k}\,\vect{x}_{12}}+{1\over e^{2\pi \omega}-1} {4K_{i\omega}(|\vect{k}|\sigma_1)K_{i\omega}(|\vect{k}|\sigma_2)
\over |\Gamma(i\omega)|^2}e^{i\omega \tau_{12}-i\vect{k}\,\vect{x}_{12}}\Bigg]\]
We are interested in the convergence of this integral in the region $\omega=0$. We have the explicit factor of ${1\over 2\omega}$ in front, the thermal occupation factors 
give another factor of ${1\over \omega}$. The Bessel function $K_{i\omega}(z)$ goes to a non-zero constant as $\omega$ goes to zero, for fixed $z$. Finally the factor ${1\over |\Gamma(i\omega)|^2}$
goes like $\omega^2$ for small $\omega$. All in all, the integrand goes like $\omega^0$ for small $\omega$ and hence the integral converges when $\omega\rightarrow 0$.

\subsubsection{Region II}

Now let us consider two points behind the Rindler horizon i.e. in region II, using the expansion \eqref{regionIIrindler}. We have contributions of several bilinears made out of $a_{\omega,\vect{k}}$ and $\widetilde{a}_{\omega,\vect{k}}$. If focus on only the contributions from the bilinears $\langle 0|a_{\omega,\vect{k}}
a_{\omega',\vect{k}'}^\dagger|0\rangle$ and $\langle 0|a_{\omega,\vect{k}}^\dagger
a_{\omega',\vect{k}'}|0\rangle$ of the non-tilde operators, we find the terms
\[
\begin{split}
 (2\pi)^d \int_{\omega>0}{d\omega d^{d-1}\vect{k}  \over (2 \pi)^d} {\pi \over \sinh(\pi \omega)}\Bigg[ &{e^{2\pi \omega} \over e^{2\pi \omega}-1} J_{i\omega}(|\vect{k}|\sigma_1)
 J_{-i\omega}(|\vect{k}|\sigma_2)e^{-i\omega \tau_{12}+i \vect{k}\,\vect{x}_{12}}\\
& + {1 \over e^{2\pi \omega}-1} J_{-i\omega}(|\vect{k}|\sigma_1)
 J_{i\omega}(|\vect{k}|\sigma_2)e^{i\omega \tau_{12}-i \vect{k}\,\vect{x}_{12}}\Bigg]
\end{split}
\]
For small $\omega$ (and fixed $\sigma_1,\sigma_2$) the Bessel functions go to nonzero constants, hence the integrand goes like ${1\over \omega^2}$ and the integral
diverges as
\[
 \int_{\omega>0} {d\omega \over \omega^2}
 \]
This seems to suggest that the small $\omega$ region has a very large contribution to the 2-point function, however we know that this cannot be the correct answer.

After all, the 2-point function that we are considering the the standard Wightman function of a massless scalar field, which is obviously finite for two points in region II. Hence it must be that, while the contributions from these two terms mentioned above are formally divergent, the total contribution from all terms --- that is from the non-tildes and from cross terms--- must be finite. There must be cancellations between the terms that we have considered and the remaining terms. 

In other words, if we regroup the terms correctly {\it before} doing the $\omega$ integral, the resulting expression must be manifestly finite as we integrate down to $\omega=0$. It turns out that regrouping the $a_{\omega,\vect{k}}$ and $\widetilde{a}_{\omega,\vect{k}}$ into the ``Unruh combinations'' \eqref{unruh} makes the integral manifestly convergent. Indeed, substituting from \eqref{unruhinv} into \eqref{regionIIrindler} we find that the field in region II can be written as
\[
\phi(\tau,\sigma,\vect{x})= \int_{\omega>0} {d\omega d^{d-1}\vect{k}  \over (2 \pi)^d} {1\over \sqrt{2\omega}}\sqrt{\pi \omega\over \sinh(\pi \omega)}\left[e^{-i\omega \tau + i \vect{k} \, \vect{x}}\,i\, \left({
e^{-\pi \omega}J_{i\omega}(|\vect{k}|z) - J_{-i\omega}(|\vect{k}|\sigma)\over \sqrt{1-e^{-2\pi \omega}}}\right)d^1_{\omega,\vect{k}} + {\rm h.c.} \right]
 \]
\[
+{\rm terms\,\,involving\,\,}d^2_{\omega,\vect{k}}
\]
We have that
\[
e^{-\pi \omega}J_{i\omega}(z) - J_{-i\omega}(z) = -\sinh (\pi \omega)H^2_{i\omega}(z)
\]
Hence we find
\[
 \phi(\tau,\sigma,\vect{x}) = \int_{\omega>0} {d\omega d^{d-1}\vect{k} \over (2 \pi)^d} {\sqrt{\pi}\over 2}e^{\pi \omega\over 2}\left[-\,i \,e^{-i\omega \tau + i \vect{k} \,\vect{x}} 
 H^2_{i\omega}(|\vect{k}|\sigma) d^1_{\omega,\vect{k}}
 + {\rm h.c.} \right]
\]
\[
+{\rm terms\,\,involving\,\,}d^2_{\omega,\vect{k}}
\]
On the Minkowski vacuum we have $d^{1,2}_{\omega,k} |0\rangle = 0$. So the 2-point function for points in region II becomes
\[
\langle 0| \phi(\tau_1,\sigma_1,\vect{x}_1) \phi(\tau_2,\sigma_2,\vect{x}_2) |0\rangle = (2\pi)\int_{\omega>0}{d\omega d^{d-1} \vect{k} \over (2 \pi)^d} {\pi \over 4} e^{\pi \omega}
e^{-i\omega t_{12}+i \vect{k}\, \vect{x}_{12}} H^2_{i \omega}(|\vect{k}|\sigma_1)\,H^2_{i\omega}(|\vect{k}|\sigma_2)^* +\]
\[
+{\rm terms\,\,involving\,\,}d^2_{\omega,\vect{k}}
\]
In this form we notice that the contribution from
$d^1_{\omega,\vect{k}}$ is manifestly finite when integrating all the
way down to $\omega=0$. The same is true for the contribution from $d^2_{\omega,\vect{k}}$.

%%% Local Variables: 
%%% mode: latex
%%% TeX-master: "infalling_paper"
%%% End: 


