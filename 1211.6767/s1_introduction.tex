
\section{Introduction}
Even though quantum gravity has attracted theoretical interest for decades, several aspects of the theory continue to be actively debated. This includes, among other questions, the issue of how to define ``local operators'' in a theory of quantum gravity. Where black holes are involved, the situation seems to be even more puzzling. What is the nature of spacetime behind the horizon of the black hole? What about the horizon itself? Even though the principle of equivalence suggests that there is nothing special about the horizon of a large black hole, there have been several speculations that quantum gravity effects cause the interior to be modified into a fuzzball \cite{Mathur:2012np,Mathur:2009hf,Mathur:2008nj,Lunin:2002bj,Lunin:2001fv}, and more recently, that the horizon of an ``old black hole'' 
is replaced by a firewall \cite{Almheiri:2012rt}. (See also \cite{Bousso:2012as, Nomura:2012sw, Mathur:2012jk, Susskind:2012rm, Bena:2012zi, Giveon:2012kp, Banks:2012nn, Ori:2012jx, Brustein:2012jn, Susskind:2012uw, Marolf:2012xe,  Hossenfelder:2012mr, Nomura:2012cx, Hwang:2012nn, Larjo:2012jt,braunstein2009v1, Avery:2012tf, Chowdhury:2012vd}.)  These latter proposals originate, not from direct calculations in quantum gravity, but rather in arguments that the information 
paradox (in various incarnations) cannot be solved without modifying the geometry at, or behind the horizon.

While quantum gravity is a mysterious subject, the AdS/CFT correspondence \cite{Maldacena:1997re} provides us with a setting where we can examine these ideas within a perfectly well defined theory. To this end, in this paper we would like to examine the following conceptual questions:
\begin{itemize}
\item
Is it possible to describe the results of local experiments in AdS, at least within perturbation theory in ${1 \over N}$, using the boundary field theory?
\item
In the presence of a black hole, can we also describe local experiments {\em behind} the black hole horizon in the boundary theory?
\item
Does our construction of the degrees of freedom behind the horizon shed any light on the information puzzle?
\end{itemize}
More precisely, we will imagine that within anti-de Sitter space, we start with some matter that then collapses to form a large black hole. We will then try and reconstruct local operators outside this black hole, and behind the horizon. 

In fact, if we imagine an observer who lives outside this black hole for a while and then dives in, then we would require answers to all the questions above to describe his experience; this is the reason for the title of our paper.

We will consider some strongly coupled CFT in $d$ dimensions, which has the properties that would allow it to have a bulk dual, without restricting ourselves to any specific example of the AdS/CFT correspondence. We will place the CFT in a pure initial state that thermalizes after a while i.e. it evolves to a state that is almost indistinguishable from a thermal state.  In this state, we will then show how to reorganize all the operators that are accessible in the CFT to a low-energy observer, into fields
 that are labeled by points in the semi-classical geometry of a big black hole in \ads[d+1]. We emphasize that these are still CFT operators, although rather than being labeled by a boundary point, they are labeled by a bulk point. We will further show, to lowest order in the ${1 \over N}$ expansion and argue to higher orders, that the correlators of the operators that we have constructed are the same as the correlators of perturbative fields on this geometry. Second, we will 
push this construction {\em past the horizon} and show how to construct perturbative fields behind the horizon in terms of CFT operators. This will give us important clues about how to resolve the firewall paradox, as we explore in section \ref{sec:applications}. 

Our construction follows important work by several other authors \cite{Banks:1998dd, Balasubramanian:1999ri, Bena:1999jv, Hamilton:2006fh,Hamilton:2006az,Hamilton:2005ju,Hamilton:2007wj,VanRaamsdonk:2009ar,VanRaamsdonk:2010pw,VanRaamsdonk:2011zz,Czech:2012bh,Heemskerk:2012mn}. However, as we describe in more detail below, we have been able to make some technical improvements on the construction of holographic operators outside the horizon. Our construction of the black hole interior in terms of CFT data is new, and to our knowledge has not been explored before.  

We now quickly summarize our results. Before considering the black hole, we start by considering empty AdS. To construct local operators in this space, we consider a generalized free-field \cite{ElShowk:2011ag} ${\cal O}(t, \vect{x})$ when the CFT is in the vacuum. Working with the modes ${\cal O}_{\omega,\vect{k}}$ of ${\cal O}(t, \vect{x})$ in  momentum space, we are able to write a CFT operator that is labeled by a point in the bulk of the AdS Poincare patch
\be
\label{cftpure}
\phi_{\text{CFT}} (t,\vect{x}, z) = \int_{\omega>0} {d \omega d^{d-1} \vect{k} \over (2 \pi)^d} \,\left[ {\cal O}_{\omega, \vect{k}} \xi_{\omega, \vect{k}}(t, \vect{x}, z) + \text{h.c}.\right]
\ee
When the mode functions $\xi$ are appropriately chosen, the operator on the left has the same correlators as a free-field propagating in \ads[d+1]: for example, its commutator at two points that are spacelike separated in \ads[d+1] vanishes. We show how these operators can be continued beyond the Poincare patch onto all of global AdS. This serves as a warm up for our next task of looking beyond the black hole horizon. 

We then consider generalized free-fields ${\cal O}(t,\vect{x})$, but in a CFT state that, although pure, is ``close'' to the thermal state. We will refer to this state as $\state{\Psi}$ below. Now, we find that we have to write
\be
\label{cftthermal}
\phi_{\text{CFT}} (t,\vect{x}, z) = \int_{\omega>0} {d \omega  d^{d-1} \vect{k} \over (2 \pi)^d}\,\left[ {\cal O}_{\omega, \vect{k}} f_{\omega, \vect{k}}(t, \vect{x}, z) + \text{h.c}.\right]
\ee
Although \eqref{cftthermal} looks deceptively similar to \eqref{cftpure}, there are several differences. First, the mode functions $f$ are different from the ones that we encountered above. Another important difference is that, while in \eqref{cftpure}, we can set the mode functions for all cases where $\omega^2 < \vect{k}^2$ to zero, we cannot do so in \eqref{cftthermal}. Nevertheless, with $f$ chosen appropriately, \eqref{cftthermal} gives a good  and local description of fields in front of the black hole horizon.

Next, we point out that in this pure state $\state{\Psi}$, after it has settled down, for each such operator ${\cal O}$, there must necessarily exist operators $\widetilde{\cal O}$ that have the properties that they (a) commute with ${\cal O}$ and (b) that, in the state $\state{\Psi}$, measurements of $\widetilde{\cal O}$ are completely (anti)-correlated with measurements of ${\cal O}$. (We make this more precise in section \ref{sec:behind}.) For us, these operators $\widetilde{\cal O}$ play the role that operators in the ``second copy'' of the CFT would have played, had we been dealing with an eternal black hole.  Using these operators $\widetilde{\cal O}$ we now construct operators behind the horizon: 
\[ 
\phi_{\text{CFT}}(t,\vect{x},z) =
\int_{\omega>0} {d\omega d^{d-1}\vect{k}  \over (2 \pi)^d}\, \left[ {\cal O}_{\omega,\vect{k}} g_{\omega,\vect{k}}^{(1)}(t,\vect{x},z) + \widetilde{\cal O}_{\omega,\vect{k}} g_{\omega,\vect{k}}^{(2)}(t,\vect{x},z)+ \text{h.c.}
\right]
 \]
where $g^{(1)}$ and $g^{(2)}$ are again functions that are chosen to make 
this operator local.  

Our construction is perfectly regular as we cross the horizon. This appears to be in contradiction with both the fuzzball and the firewall proposals.  In section \ref{sec:applications} we first show that it is not possible to pinpoint the microstate of the CFT by measuring correlators of light operators to any given fixed order in the ${1 \over N}$ expansion. Our construction then implies that by doing experiments that are limited to some finite order in ${1 \over N}$, either at low or high energies, the bulk observer cannot distinguish the microstates of the black hole. Since the fuzzball proposal solves the information paradox by postulating that the observer can detect the microstate by doing ``low energy experiments'', our proposal appears to be inconsistent with this resolution.

Turning to the firewall paradox, in this paper we provide only indirect evidence for the absence of firewalls at the horizon. This is because our construction works in detail for a big black hole that does not evaporate (except over the Poincare recurrence time), and so this leaves us with the theoretical possibility that small black holes in AdS could have firewalls near the horizon. However, our description of the degrees of freedom in the interior of the black hole also provides us with several lessons that we can use to understand the information paradox that appear to make firewalls superfluous. 

In particular, our construction can, roughly, be interpreted as showing how, if we expand our space of observables to include operators that give us finer and finer information about the CFT microstate, then eventually we reach a stage where the operators behind the horizon are no longer independent of 
the operators in front of the horizon. Such a description provides a natural realization of black-hole ``complementarity'' \cite{'tHooft:1990fr,Susskind:1993if} and, as we explore below, removes the necessity of firewalls. 


We should also point out that while this might naively imply a violation of causality, this causality violation is visible only when we measure very high point correlators of operators (where the number of insertions scales with $N$) or equivalently measure a low point correlator to exponential accuracy in $N$. At this level of accuracy, we do not believe that a semi-classical spacetime description makes sense at all and so this putative causality-violation is not a cause for concern.

There have been many earlier attempts to study the interior of the black hole using AdS/CFT and the literature on the information paradox is vast. Besides the papers  mentioned above, a very incomplete list of references which were relevant to our work include  \cite{Kiem:1995iy,Balasubramanian:1999zv,Giddings:2001pt,Maldacena:2001kr,Hubeny:2002dg,Kraus:2002iv,Levi:2003cx,Fidkowski:2003nf,Barbon:2003aq,Kaplan:2004qe,Balasubramanian:2004zu,Festuccia:2005pi,Balasubramanian:2005mg,
Balasubramanian:2005qu,Festuccia:2006sa,Balasubramanian:2007qv,Marolf:2008mf,Balasubramanian:2008da,deBoer:2009un,Horowitz:2009wm,Balasubramanian:2011dm,Avery:2011nb,Simon:2011zza}. 

A brief overview of this paper is as follows.  In section \ref{sec:emptyads}, we describe the construction of local operators in empty AdS using boundary operators in a flat space CFT in a pure state.  We work in the Poincare patch and then show how our construction can be continued past the Poincare horizon into global AdS.  In section \ref{sec:outside}, we repeat this construction in front of the horizon of a big black hole in AdS. In section \ref{sec:behind}, we push this construction past the black-hole horizon and write down local bulk operators behind the horizon in terms of CFT operators.  In section \ref{sec:applications}, we discuss the implications of these results, with a particular view to the information paradox and various unconventional proposals for resolving it --- including the ``firewall'' paradox and the fuzzball proposal. We also discuss a qubit toy model that is surprisingly effective at realizing many of these ideas. In section \ref{sec:subtleties}, we address some common worries 
and 
show that our construction is not destabilized by tiny effects like 
Poincare recurrence on the boundary. We also describe schematically how it may be extended to higher orders in the ${1 \over N}$ expansion, and discuss some other subtleties. We conclude in section \ref{sec:conclusion}. The appendices provide some technical details and also examine the specific case of two-dimensional thermal correlators and the BTZ-black-hole background.



%%% Local Variables: 
%%% mode: latex
%%% TeX-master: "infalling_paper"
%%% End: 
