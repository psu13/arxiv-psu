
\section{Conclusions and further directions \label{sec:conclusion}}

Let us summarize the implications of our construction. Since the paper is already quite long we will be as brief as possible. 

In what follows it is important to keep in mind
that our precise results apply to a black brane or a big black hole\footnote{Although, for notational reasons, we presented the results for a black brane it should be clear that similar local bulk observables
can be defined for a big black hole in global AdS, by replacing the integrals over $\vect{k}$ with sums over spherical harmonics.} in AdS, neither of which evaporate. We leave it to the reader to decide how relevant the following conclusions are for  black holes in asymptotically flat space.

\vskip5pt
\noindent {\bf Reconstructing local bulk observables outside the black hole:} 
We revisited the construction of local operators near a big AdS black hole. At large $N$ it is straightforward to write CFT operators which reconstruct local bulk fields. We pointed out that a divergence which appears when attempting to write a transfer function in position space, is a harmless artifact --- effectively ``regularized'' by the general behavior of boundary thermal correlators --- and can be evaded by working with the operators in momentum space. In particular, there is no need for an analytic continuation along the spacelike directions of the boundary. The construction of local bulk observables outside the black hole seems to be robust under the inclusion of $1/N$ corrections.

\vskip10pt

\noindent {\bf Reconstructing the region behind the horizon:} We argued that in certain theories, including large $N$ gauge theories, there is a natural splitting of the Hilbert space into coarse- and fine-grained components. Typical pure states have the property that they ``self-thermalize'', i.e. the reduced density matrix of the coarse-grained Hilbert space is automatically driven by the dynamics very close to the thermal one. In such a situation, for every operator acting on the coarse-grained Hilbert space, it is natural to identify a ``partner'' operator acting on the fine-grained Hilbert space. These partner operators obey an identical algebra as the 
coarse-grained operators, and can be identified with the  ``tilde''  operators in the ``thermofield doubled Hilbert space'' formalism. Using these partner operators we can reconstruct local bulk observables behind the horizon of the black hole. We argue that, despite what one might naively expect,  local bulk observables behind the horizon are 
essentially supported over finite time scales on the boundary and hence --- in the large $N$ limit --- they are not too sensitive on the specific microstate. This implies that the ``interior geometry'' (as probed by low-energy experiments) looks the same for all pure states, in contradiction with (some versions of) the fuzzball proposal.

\vskip10pt

\noindent {\bf Fate of the infalling observer:} From these results it follows that a semi-classical observer falling cross the horizon of a big black hole,  will not measure anything special. In particular he will not see a firewall or a fuzzball. Our precise prediction is the following. If we first fix: i) the size of the black hole in AdS units,  ii) the trajectory of the infalling observer and the points at which he will measure the local fields, iii) the number of measurements he can make and the accuracy he has iv) the type (mass) of fields that he can measure {\it and then} send $N$ to infinity (i.e. we do not scale any of the previous quantities with $N$), then the observer will not notice any difference from semi-classical GR. Whether this is consistent or not with the fuzzball proposal depends on its precise definition. However, to the extent that the fuzzball proposal posits a departure from
semi-classical GR at energy scales of the order of the temperature of the black hole, our construction seems to contradict it.

\vskip10pt

\noindent {\bf Lessons about the information paradox:}  We argued that, contrary to the claims of \cite{Mathur:2012np} and \cite{Almheiri:2012rt}, {\it small corrections to the Hawking computation can restore
unitarity}. The corrections suggested by our construction are small in the sense that low point correlators of perturbative fields are almost exactly the same as those computed by semi-classical quantum gravity.  The intuitive reason that this is consistent with the unitarity of the Hawking process is that the number of emitted Hawking particles is large, so even small correlations between them can carry away the information. 
The works of \cite{Mathur:2012np} and \cite{Almheiri:2012rt}, claimed that this is not possible by invoking the strong subadditivity theorem applied to particles located {\it both inside and outside}  the black hole.
Our construction of local bulk observables, explicitly demonstrates that the semi-classical Hilbert spaces corresponding to the interior and exterior of the black hole {\it are not independent}
and hence it is not permissible to use the strong subadditivity theorem.
We discussed this in the context of a simple qubit toy model. Our construction suggests a form of ``black hole complementarity'' from the point of view of the boundary conformal field theory, which would be interesting to understand in more detail.

\vskip10pt

\noindent {\bf Future work:} Clearly there are many questions that need to be explored in more detail. An obvious one is to understand how to extend the reconstruction of the bulk in perturbation theory in $1/N$. We would like to further investigate the properties of the tilde operators, the splitting of the Hilbert space into coarse- and fine-grained components and the conditions under which an isolated quantum system can undergo thermalization. We hope to revisit this question in future work.
It would be nice to find a more realistic dynamical toy-model that would capture the essential features of complementarity. Finally, given that we have constructed local bulk observables behind the horizon, it would obviously be interesting to study what happens when the bulk point approaches the black hole singularity.


%%% Local Variables: 
%%% mode: latex
%%% TeX-master: "infalling_paper"
%%% End: 
