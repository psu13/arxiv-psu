\section{Looking beyond the black hole horizon \label{sec:behind}}

In the previous section we reviewed the boundary construction of local bulk observables which lie outside the horizon of the black hole. What about points behind the horizon? 

For the reconstruction of local bulk observables in region $\front$ it was important to identify the modes of the scalar field in region $\front$ with the Fourier modes of the boundary operator.  In section \ref{semiblack} we explained that in order to write operators in region $\black$ (inside the black hole) we need modes {\it both} from region $\front$ and region $\other$. This raises the question: what should play the role of the modes of region $\other$ in the boundary field theory?

In fact the reader can easily persuade herself that, without these modes, it is not possible to write down a local field operator behind the horizon. Heuristically, the reason for this is as follows. If we try and extend the modes that we have constructed for region I, past the horizon, it is only the ``left moving'' modes that can be smoothly analytically continued past the horizon. To obtain a local field theory, we need independent right-movers behind the horizon, and these must come from some analogue of region $\other$.\footnote{The very perceptive reader might note that our condition of imposing normalizability at the boundary actually led to a constraint between left and right movers. This is true, but our condition was imposed on a time-like boundary. Behind the horizon, if we were to impose such a constraint, it would set constraints on a space-like boundary leading to a loss of locality.}


Now, as we have pointed out above, at late times, the geometry of a collapsing black hole is well approximated by the geometry of an eternal black hole. It is also known that an eternal black hole in anti-de Sitter space can be described by {\em two} CFTs --- one on each boundary of this geometry --- that are placed in a particular entangled state. This construction was first explored in \cite{Maldacena:2001kr}. In this picture, it is clear what plays the role of the required right-moving modes in the eternal black-hole. These are simply the right-moving modes that are constructed in region $\other$ and come from the second copy of the CFT. This story is consistent with the fact that, at finite temperature, the CFT is well described in the thermofield doubled formalism of Takahashi and Umezawa \cite{takahashi1996thermo}. 

In this thermofield doubled formalism, we double the Hilbert space of the theory and consider the pure state
\be
\label{thermostate}
|\Psi_{\text{tfd}}\rangle = {1\over \sqrt{Z_\beta}}\sum e^{-{\beta E \over 2}} |E\rangle \otimes |E\rangle,
\ee
where the states of the original theory are labeled by the energy $E$ (and perhaps other quantum numbers).  Corresponding to this ``doubling'' of the Hilbert space, every operator ${\cal O}$ in the original Hilbert space can be given a partner $\widetilde{\cal O}$ that acts on the second copy of the space. The point of this construction is that thermal
expectation values of operators in the original (single-copy) Hilbert space, can equivalently be computed as standard expectation values {\it on the  pure state} $|\Psi_{\text{tfd}}\rangle$ in the doubled Hilbert space
\[
 Z_{\beta}^{-1}{\rm Tr}\left[e^{-\beta H} {\cal O}(t_1,\vect{x}_1)...{\cal O}(t_n,\vect{x}_n) \right] = \langle \Psi_{\text{tfd}}| {\cal O}(t_1,\vect{x}_1)...{\cal O}(t_n,\vect{x}_n)|\Psi_{\text{tfd}}\rangle
\]
Moreover, correlators involving both ${\cal O}$ and $\widetilde{\cal O}$ in the doubled Hilbert space can be related to correlators
in the thermal, single-copy Hilbert space, with the replacement 
\[\widetilde{\cal O}\left(t,\vect{x})\rightarrow {\cal O}(t+i\beta/2,\vect{x}\right)\]
where the analytic continuation in complex time is defined  along the ``principal sheet''  as described in section \ref{therman}. More precisely, we have
\[
 Z_{\beta}^{-1}{\rm Tr}\left[e^{-\beta H} {\cal O}(t_1,\vect{x}_1)...{\cal O}\left(t_k+i\beta /2,\vect{x}_k\right)...\right] =
 \langle \Psi_{\text{tfd}}| {\cal O}(t_1,\vect{x}_1)...\widetilde{\cal O}(t_k,\vect{x}_k)...|\Psi_{\text{tfd}}\rangle
\]
It is an easy exercise to verify that this identification of $\widetilde{\cal O}(t,\vect{x})$ in the thermofield doubled Hilbert space, with ${\cal O}\left(t+i\beta/2,\vect{x}\right)$ inside the thermal correlators of the single Hilbert space, is consistent with the commutator $[{\cal O},\widetilde{\cal O}]=0$ in the doubled Hilbert space.

\vskip10pt

To summarize, it is well known that if we have a single theory {\it placed in a thermal density matrix}, we can map the thermal correlators to correlators in a {\it doubled theory} placed in the specific entangled pure state \eqref{thermostate}. And moreover, operators acting on the ``second copy'' of the doubled theory correspond to analytically continued versions of the thermal correlators of the original theory.  

However, besides this story about {\it thermal correlators}, we know that in the original single theory there is a large number of {\it pure states} which approximate the thermal density matrix to an excellent approximation. For these pure states, what is the meaning of the thermofield doubled formalism and of the second copy of operators $\widetilde{\cal O}$? We answer this question in the next subsection.

\subsection{Approximating a  pure state as a thermofield doubled state}
While the thermofield formalism is used quite frequently, and quite possibly the answer to our question above is clear to experts, we were surprised not to find it stated clearly in the literature.  We believe that the correct answer is as follows. 

Consider a pure-state $|\Psi\rangle$ in a theory with many degrees of freedom that thermalizes. We try to ``coarse-grain'' this state and identify some degrees of freedom that are easily accessible to measurement, and others that are not. The fact that the observables of the theory admit a division into ``easily accessible to measurements'' (coarse-grained observables) and those that are not (fine-grained observables), is a {\it prerequisite} to be able to talk about  ``thermalization of a pure state''. If we start with a closed quantum system, we can talk about ``thermalization of a pure state'' if there is some sense that we can divide the full system into two parts, one of which plays the role of the 
``environment/heat-bath'' and the other of the ``system in thermal equilibrium'' and if moreover the dynamics is such that under time evolution the pure state $|\Psi\rangle$, {\it when projected on the small system}, eventually looks approximately like a thermal density matrix. Clearly this is an approximate notion, which can only be made precise if there is some parameter which controls the validity of these approximations. In the case relevant to us, the full quantum system is the order $O(N^2)$ degrees of freedom of the large $N$ gauge theory, which may be in some specific pure state $|\Psi\rangle$, while the role of the ``sub-system'' that reaches thermal equilibrium is the $O(1)$ degrees of freedom corresponding to light gauge invariant operators. The parameter which controls the validity of the splitting into coarse- and fine-grained spaces is $N$.


For example, in the case that we have been considering above, the observer may be able to measure the eigenvalue of the ``occupation level'' $n_{\omega,\vect{k}} = \hat{\cal O}^{\dagger}_{\omega, \vect{k}} \hat{\cal O}_{\omega, \vect{k}}$ of the modes \eqref{rescaled} but this does not completely specify the state of the system; for that, the observer would need to measure several other operators including those that correspond to ``stringy'' and trans-Planckian degrees of freedom in the bulk. Using this intuition, we divide the Hilbert space of the CFT into a direct-product of a ``coarse-grained'' part and a ``fine-grained'' part
\be
\label{hcoarsefine}
{\cal H} = {\cal H}_{\text{coarse}} \otimes {\cal H}_{\text{fine}}.
 \ee

There are many such possible divisions of the gauge theory into coarse- and fine-grained degrees of freedom. However, as long as the coarse-grained degrees of freedom do not probe too much information --- we quantify this phrase below ---  about the state we will see that our interpretation below holds. 

We now consider some operator in the theory. For any generalized field ${\cal O}(t, \vect{x})$, what our low energy observer really measures is
\be
\label{coarseprojector}
{\cal O}_{c}(t,\vect{x}) = {\cal P}_{\text{coarse}} {\cal O}(t,\vect{x}),
\ee
where ${\cal P}_{\text{coarse}}$ is the ``projector'' onto the coarse-grained Hilbert space which traces out the fine grained degrees of freedom. We should mention that a division of the Hilbert space into ``coarse'' and ``fine'' degrees of freedom is often
performed in studies of ``dissipation'' in quantum systems. As far as we know, this was first attempted by von Neumann \cite{von2010proof} and our ``coarse'' operators are precisely the ``macroscopic'' operators of that paper. 

Now, let us turn to the pure state that we are considering, which is close to a thermal state. Corresponding to the Hilbert space decomposition above,  this state can be written as
\be
\label{entangledcoarsefine}
| \Psi \rangle = \sum_{i,j} \alpha_{ij} | \Psi^{c}_i \rangle \otimes | \Psi^{f}_j \rangle,
\ee
i.e. as an entangled state between the coarse- and fine-grained parts of the Hilbert space. Here the states $|\Psi^{c}_i\rangle$ run over an orthonormal basis of states in the coarse-grained Hilbert space, and $|\Psi^{f}_j\rangle$ run over
an orthonormal basis in the fine-grained Hilbert space.

In fact, what the low-energy observer sees directly is not
this entangled state, but rather a density matrix. Since the pure state is close to being thermal, we even know the eigenvalues of this density matrix. In some basis, the coarse-grained density matrix can be written
\be
\label{thermaldensity}
\rho_{c} = {1 \over Z^c_{\beta}}\text{diag}\left(e^{-\beta E_i} \right)
\ee
where $E_i$ are the eigenvalues of the coarse-grained Hamiltonian and $Z^c_{\beta} = \sum_i e^{-\beta E_i}$ is the coarse-grained partition function.\footnote{
 In in any given pure state $|\Psi \rangle$, the exact coarse-grained density matrix may be different from the exactly thermal one above, but these differences are unimportant in what follows, so we ignore them.}


What is the form of the density matrix in the fine-grained Hilbert space? In fact, as is well known, we can find a basis for the {\em fine-grained} Hilbert space so that the density matrix looks exactly like \eqref{thermaldensity} with zeroes in the other places. 

The argument is very simple, and we repeat it here for the reader's convenience. We need to ``diagonalize'' the matrix $\alpha_{i j}$ in \eqref{entangledcoarsefine}. This process, called a  ``singular value decomposition'', involves writing $\alpha_{i j}$ as
\[
\alpha_{i j} = \sum_m U_{i m} D_{m m} V_{m j}
 \]
where $U$ is a unitary matrix that acts in ${\cal H}_{\text{coarse}}$ and $V$ is a unitary matrix in  ${\cal H}_{\text{fine}}$, while $D$ is a rectangular-diagonal matrix i.e. a matrix of the form
\be
\label{rectangularmat}
D = \begin{pmatrix}
D_{11} & 0 & 0 & 0 &\ldots \\
0 & D_{22} & 0 & 0 & \ldots \\
 0 & 0 & D_{33} & 0 & \ldots
\end{pmatrix}
 \ee
This means that we can re-write \eqref{entangledcoarsefine} as
\be
\label{maximalentanglement}
| \Psi \rangle = \sum_i D_{i i} | \hat{\Psi}_i^{c} \rangle \otimes | \hat{\Psi}_i^{f} \rangle
\ee
where now, both $|\hat{\Psi}_i^{c}\rangle$ and $|\hat{\Psi}_i^f\rangle$ are orthonormal. In fact from \eqref{thermaldensity}, we already know that $D_{i i} = {1 \over \sqrt{Z^c_{\beta}}} e^{-{\beta E_i \over 2}}$.

Now, \eqref{maximalentanglement} immediately tells us that the fine-grained density matrix must also look like
\[
\rho_f = \sum_i D_{i i} |\hat{\Psi}_i^{f} \rangle \langle \hat{\Psi}_i^f |.
 \]
Let us return to the generalized free field ${\cal O}(t,\vect{x})$ that we have been discussing. As we mentioned, the low energy observer can only measure ${\cal O}_{c}(t,x)$ defined in \eqref{coarseprojector}. However, in the coarse-grained Hilbert space, we should be able
to write\footnote{Note, that we have kept the time evolution of the coarse operator arbitrary. In particular, the division \eqref{hcoarsefine} does not correspond to an analogous separation of the Hamiltonian, which may well couple the coarse and fine degrees of freedom. In studies of quantum dissipation, this coupling is sometimes quantified by means of an ``influence functional.'' It would be interesting
to understand this in detail for the boundary theory.}
\[
{\cal O}_{c}(t,\vect{x}) = \sum_{i_1, i_2} {\omega}_{i_1 i_2}(t,\vect{x})|\hat{\Psi}^c_{i_1} \rangle \langle \hat{\Psi}_{i_2}^c|.
 \]
Now, we {\em define} an operator that acts in the same way on the fine-grained Hilbert space
\be
\label{tildeodef}
\widetilde{\cal O}(t,\vect{x}) = \sum_{i_1, i_2} \omega^*_{i_1 i_2}(t,\vect{x})|\hat{\Psi}^f_{i_1} \rangle \langle \hat{\Psi}_{i_2}^f|. 
 \ee
The complex conjugation on $\omega$ is consistent with the original conventions of \cite{takahashi1996thermo}, and implies that the map between ${\cal O}$ and $\widetilde{\cal O}$ is {\em anti-linear.} This is necessary to ensure that the map is invariant under a change of basis that leaves \eqref{maximalentanglement} invariant.


What we have shown is that, even in a pure state for the gauge theory, for each generalized free-field ${\cal O}(t,\vect{x})$ and its corresponding modes ${\cal O}_{\omega, \vect{k}}$, we spontaneously obtain doubled operators $\widetilde{\cal O}(t,\vect{x})$, with corresponding modes $\widetilde{\cal O}_{\omega, \vect{k}}$! We explore this emergence further in section \ref{sec:applications} in a concrete example.\footnote{The coarse-fine splitting of the Hilbert space described in this section holds to an excellent approximation in the CFT at large-N. The construction described here, also
provides very important intuition including the ``state dependence''
of the $\widetilde{\cal O}(t,\vect{x})$ operators. However, we refer
the reader to \cite{Papadodimas:2013,Papadodimas:2013b} for a somewhat more
refined construction.}

These doubled operators are precisely what we need to construct local operators in region \black. However, before we attack that question we first want to mention a relation between the correlators of these operators and analytically continued correlators of ordinary operators.

From this point in this paper, we will drop the subscript ${\cal O}_c(t, \vect{x})$; it is understood that correlators now refer to the coarse grained operators only.



\subsection{Relation to analytically continued thermal correlators}
In this subsection, we will point out below that  $\widetilde{\cal O}$ operators  can---in a sense that we make precise ---be thought of as analytically continued versions of the ordinary operators. 

Let us say that we wish to compute a two-point correlator involving both ${\cal O}$ and $\widetilde{\cal O}$ in the state $| \Psi \rangle$ above
\be
\label{o1o2rel}
\begin{split}
&\langle \Psi | {\cal O}(t_1, \vect{x}_1) \widetilde{\cal O}(t_2, \vect{x}_2) | \Psi \rangle =  (Z^c_{\beta})^{-1}\sum_{i_1, i_2} e^{-{\beta (E_{i_1} + E_{i_2}) \over 2}} \langle \hat{\Psi}^{c}_{i_1} | {\cal O}(t_1, \vect{x}_1) | \hat{\Psi}^{c}_{i_2} \rangle  \langle \hat{\Psi}^{f}_{i_1} | \widetilde{\cal O}(t_2, \vect{x}_2) | \hat{\Psi}^{f}_{i_2} \rangle \\
&= (Z^c_{\beta})^{-1} \sum_{i_1, i_2} e^{-{\beta (E_{i_1} + E_{i_2}) \over 2}} \langle \hat{\Psi}^{c}_{i_1} | {\cal O}(t_1, \vect{x}_1) | \hat{\Psi}^{c}_{i_2} \rangle  \langle \hat{\Psi}^{c}_{i_1} | {\cal O}(t_2, \vect{x}_2) | \hat{\Psi}^{c}_{i_2} \rangle^* \\
&=  (Z^c_{\beta})^{-1} \sum_{i_1, i_2} e^{-{\beta (E_{i_1} + E_{i_2}) \over 2}} \langle \hat{\Psi}^{c}_{i_1} | {\cal O}(t_1, \vect{x}_1) | \hat{\Psi}^{c}_{i_2} \rangle  \langle \hat{\Psi}^{c}_{i_2} | {\cal O}(t_2, \vect{x}_2) | \hat{\Psi}^{c}_{i_1} \rangle \\
&= (Z^c_{\beta})^{-1} \sum_{i_1, i_2} e^{-\beta E_{i_1} } \langle \hat{\Psi}^{c}_{i_1} | {\cal O}(t_1, \vect{x}_1) | \hat{\Psi}^{c}_{i_2} \rangle  \langle \hat{\Psi}^{c}_{i_2} | {\cal O}(t_2 + {i \beta \over 2}, \vect{x}_2) | \hat{\Psi}^{c}_{i_1} \rangle \\
&= (Z^c_{\beta})^{-1}{\rm Tr}_c\left[e^{-\beta H} {\cal O}(t_1, \vect{x}_1) {\cal O}(t_2 + {i \beta \over 2}, \vect{x}_2) \right],
\end{split}
\ee
where the last trace is over the coarse grained Hilbert space.
Note that the operator ordering is important. The analytically continued operator always appears on the right. As we discussed above, when we discussed thermal correlators,  if this operator had appeared on the left, the correlator above would naively have diverged and we would have had to define it by giving a prescription for how to cross the branch cut. In fact, the ordering in \eqref{o1o2rel} corresponds to always remaining in the {\em principal branch} of the function ${\cal F}$ described in section \ref{therman}. 










\subsection{Reconstructing the full eternal black hole from a single CFT}
We now describe how to reconstruct, from the CFT, local bulk operators behind the horizon. We emphasize that we are working in a {\em single CFT} in a {\em pure state.} 
As explained in the previous subsection, for every local operator ${\cal O}(t,\vect{x})$ we define the operator
\[
{\cal O}(t, \vect{x}) \rightarrow \widetilde{\cal O} (t,\vect{x}).
\]
and its Fourier modes
\[
\widetilde{\cal O}_{\omega,\vect{k}} = \int dt d^{d-1}\vect{x} \,e^{i\omega t-i\vect{k} \vect{x}}\, \widetilde{\cal O}(t,\vect{x})
\]
The complex conjugation in \eqref{tildeodef} implies that the  modes $\widetilde{\cal O}_{\omega,\vect{k}}$ generate an algebra that is identical to
that of the modes ${\cal O}^{\dagger}_{\omega,\vect{k}}$, while at the same time these two sets of modes commute with each other. In the previous 
section we identified the gauge theory modes ${\cal O}_{\omega,\vect{k}}$ with the Schwarzschild modes $a_{\omega,\vect{k}}$ in region I of the eternal black hole. It is clear that the modes $\widetilde{\cal O}_{\omega,\vect{k}}$ and $\widetilde{\cal O}_{-\omega,\vect{k}}$ --- which we emphasize act on the same Hilbert space, of the single CFT --- play the role of the bulk modes $\widetilde{a}^{\dagger}_{\omega,\vect{k}}$ and $\widetilde{a}_{\omega,\vect{k}}$  in region III of the eternal black hole. This means that we identify
\[
\begin{split}
\hatbb{\widetilde{\cal O}}_{\omega,\vect{k}}\qquad&\Leftrightarrow \qquad a^{\dagger}_{\omega,\vect{k}},  \\
\hatbb{\widetilde{\cal O}}_{-\omega,\vect{k}}\qquad&\Leftrightarrow \qquad a_{\omega,\vect{k}},  
\end{split}
\]
where both identifications assume $\omega > 0$ and $\hatbb{\widetilde{\cal O}}$ is defined in analogy to \eqref{rescaled}.


Having established this identification, it is now straightforward to write down local bulk operators of all four regions of the eternal black hole. For region I, we already did that in the previous section and we had
\[
\phi_{\text{CFT}}^{\rm I}(t,\vect{x},z) = \int_{\omega>0}{d\omega d^{d-1}\vect{k} \over (2 \pi)^d}\,\left[{\cal O}_{\omega,\vect{k}} f_{\omega,\vect{k}}(t,\vect{x},z) + {\cal O}_{\omega,\vect{k}}^\dagger f^*_{\omega,\vect{k}}(t,\vect{x},z)\right]
\] 
For region III, we have a similar expansion. We define the analogue of the AdS-Schwarzschild coordinates $t,\vect{x},z$ for region III, in terms of the Kruskal coordinates $U,V$ by the relations
\be
\label{kruskalcb}
\begin{split}
& u = t-z_*\qquad,\qquad v=t+z_* \\ &
U = e^{-{d u \over 2 \zhor}}\qquad,\qquad V = -e^{{d v \over 2 \zhor}}
\end{split}
\ee
with $z_*$ defined in terms of $z$ exactly as above equation \eqref{lightconec}. Then the CFT operator 
\be
\label{finalregionother}
\phi_{\text{CFT}}^{\other}(t,\vect{x},z) = \int_{\omega>0}{d\omega d^{d-1}\vect{k} \over (2 \pi)^d}\,\left[\widetilde{\cal O}_{\omega,\vect{k}} f_{\omega,\vect{k}}(t,\vect{x},z) + \widetilde{\cal O}_{\omega,\vect{k}}^\dagger f^*_{\omega,\vect{k}}(t,\vect{x},z)\right]
\ee
plays the role of the local bulk field in region $\other$. Notice that while we parameterize the points in region III by the same set of coordinates $t,\vect{x},z$, they are obviously distinct points from those in region I, since their $U,V$ values are different. 

Notice that the relations \eqref{kruskalcb} for region $\other$ are similar, but not quite the same, as the relations \eqref{kruskalc} for region $\front$. In particular, the different signs in the second line imply that as Kruskal time increases, $t$ decreases in region $\other$. Related to this, we have the fact that although \eqref{finalregionother} and \eqref{uplifta} look similar, $\widetilde{O}_{\omega, k}$ is actually a ``creation operator.'' (See appendix \ref{rindlerq} for a similar situation in the simpler case of Rindler space quantization.)

It is now straightforward to use the expansions in regions I and III to define a local field in regions II and IV. We focus on region II which is our main interest.
We can parametrize region $\black$ by $t,\vect{x},z$, with $\zhor<z<\infty$ and $-\infty<t<\infty$, which is now a spacelike coordinate. Remember that in these coordinates the singularity is at $z\rightarrow \infty$. The time $t$ in this region increases as we approach the horizon between $\front$ and $\black$. 
We look for solutions of the Klein Gordon equation of the form 
\[
g_{\omega,\vect{k}}(t,\vect{x},z) = e^{-i\omega t+i\vect{k}\vect{x}} \chi_{\omega,\vect{k}}(z)
\]
We get a 2nd order ODE for $\chi_{\omega,\vect{k}}(z)$. What is important is that in region II there are no boundary conditions that we need to impose (unlike what happened in region $\front$ or $\other$ where we imposed the ``normalizable'' boundary conditions at infinity). So it seems that in region $\black$ we have twice the number of modes as in region $\front$ (or $\other$). Of course this was to be expected since region $\black$ lies in the causal future of both regions $\front$ and $\other$ and the information of both regions $\front$ and $\other$ eventually enters region $\black$. So it is normal to have twice as many modes in $\black$.


Hence for any choice of $\omega,\vect{k}$ we get two linearly independent solutions $\chi_{\omega,\vect{k}}^{(1,2)}(z)$. We can choose the linear combination of these solutions so that as $z\rightarrow \zhor$ (from larger values, i.e. from inside the black hole) we have
\[
\chi_{\omega,\vect{k}}^{(1)}(z) \sim c(\omega, \vect{k}) (z-\zhor)^{-i \omega},\qquad \chi_{\omega,\vect{k}}^{(2)}(z) \sim c(-\omega, \vect{k})(z-\zhor)^{i \omega}\quad {\rm for}\quad z\rightarrow \zhor^+,
\]
where $c(\omega, \vect{k})$ also appeared in \eqref{nearhpsi}. 
Now imposing the continuity of the field
at the horizon between $\front$ and $\black$ and also between $\other$ and $\black$ we find that the field in region $\black$ has the expansion
\begin{equation}
\boxed{\label{finalbehind}
\phi^{\rm II}_{\text{CFT}}(t,\vect{x},z) =
\int_{\omega>0} {d\omega d^{d-1}\vect{k}  \over (2 \pi)^d}\left[ {\cal O}_{\omega,\vect{k}}\, g_{\omega,\vect{k}}^{(1)}(t,\vect{x},z) + \widetilde{\cal O}_{\omega,\vect{k}} \,g_{\omega,\vect{k}}^{(2)}(t,\vect{x},z)+ {\rm h.c.}
\right]}
\end{equation}
where $g^{(1,2)}_{\omega,\vect{k}}(t,\vect{x},z) = e^{-i\omega t+i \vect{k}\vect{x}}\chi^{(1,2)}(z)$. A similar construction can be performed for the field in region $\white$.

To summarize, using the boundary modes ${\cal O}_{\omega,\vect{k}}$ and $\widetilde{\cal O}_{\omega,\vect{k}}$ which were described above, we can reconstruct the local quantum field $\phi$ everywhere in the Kruskal extension and that at large $N$ the correlators of these operators satisfy the analogue of \eqref{bulkrecon}, but now operators can be inside or outside the horizon
\begin{equation}
\label{bulkreconb}
Z_{\beta}^{-1}{\rm Tr}\left(e^{-\beta H} \phi_{\text{CFT}}(t_1,\vect{x}_1,z_1) \ldots \phi_{\text{CFT}}(t_n,\vect{x}_n,z_n)\right)_{\text{CFT}}  = \langle \phi_{\text{bulk}}(t_1,\vect{x}_1,z_1)...\phi_{\text{bulk}}(t_n,\vect{x}_n,z_n)\rangle_{\rm HH}
\end{equation}
The expansions \eqref{finalbehind} can be written more explicitly for the case of the BTZ black hole, where the mode wave functions $g_{\omega,\vect{k}}^{(1,2)}$ can be found analytically.

In particular, for any two points $P_1,P_2$ in any of the two regions I,II we have
\be
\label{localityb}
 [\phi_{\rm CFT}(P_1)\,,\,\phi_{\rm CFT}(P_2)] = 0
\ee
when $P_1,P_2$ are spacelike separated with respect to the causal structure of the AdS-Kruskal diagram. Here, to write the operator $\phi_{\rm CFT}(P)$, we use either \eqref{uplifta} or \eqref{finalbehind} depending
on whether $P$ is in region I or II. Also, as before, this equation holds as an operator equation inside the thermal trace/or on a heavy
typical pure state, to leading order at large $N$ and modulo the caveats in section \ref{factorcav}.

Notice, that while it naively seems so, the claim \eqref{localityb} is not in conflict with the idea from black hole complementarity, that the Hilbert space of the interior of the black hole is not completely
independent from the Hilbert space outside. The point is that \eqref{localityb} holds only in the 
large $N$ limit and up to the aforementioned caveats. So while the commutator of {\it two simple} measurements of the scalar field, inside and outside the horizon, is zero in 
the large $N$ limit, if we consider the commutator of an operator inside the horizon with a very complicated measurement outside (which effectively corresponds to measuring $N^{a}\,\,\text{with}~ a>0$ instances of $\phi$), we have no reason to expect the commutator to vanish --- {\it not even in the large $N$ limit}. Hence the Hilbert
spaces inside and outside are not completely independent. These issues will be discussed in more detail in section \ref{sec:applications}.

%%% Local Variables: 
%%% mode: latex
%%% TeX-master: "infalling_paper"
%%% End: 

 


