\section{Applications \label{sec:applications}}
Our construction above tells us that bulk correlators can be written in terms of boundary correlators. This has a significant advantage: it translates questions about quantum gravity in the bulk, which are mysterious, to questions about conformal field theory correlators that are well defined. 

There has been significant recent discussion of the information paradox. In an important paper, Mathur \cite{Mathur:2009hf} sharpened the contradiction between semi-classical evolution and unitarity in quantum gravity, by using the strong subadditivity of entropy. This argument was recently re-emphasized in \cite{Almheiri:2012rt}. Both papers pointed out that one way to resolve the contradiction would be to abandon the standard assumption that the horizon of the black hole is featureless. We can use our construction to obtain hints about the nature of the horizon, and also to the resolution of the information paradox. We do this in turn below. 

\subsection{Nature of the horizon \label{subsechorizonnat}}

First, let us address the issue of what happens to the infalling observer when he crosses the horizon. This is the same as the question of what happens to correlators of $\phi_{\text{CFT}}$ for points near the horizon in a typical heavy pure state. If the theory has a good large $N$ expansion in the thermal state then, as we show below, these correlators are the same (to leading order in ${1 \over N}$) to those in the thermal ensemble. In the thermal ensemble, it is clear that correlators of $\phi_{\text{CFT}}$ are the same as those predicted by 
semi-classical GR. 

The reader should not be concerned about the fact that on one side of the horizon we have both the left and right moving modes of a field constructed just from ${\cal O}$,  while on the other side the modes of ${\widetilde{\cal{O}}}$ appear. 
This is analogous to the fact that when we quantize a field in Minkowski space using Rindler coordinates (see Appendix \ref{rindlerq} for notation and a review), in region I only the modes $a_{\omega,k}$ appear, while in region II the modes $\widetilde{a}_{\omega,k}$ also appear.  Nevertheless, in the state corresponding to the Minkowski vacuum correlation functions are perfectly regular across the horizon. 
In
precisely the same fashion, the definition \eqref{tildeodef} ensures that correlators of $\widetilde{\cal{O}}$, in a typical pure state close to the thermal state,  have the same properties as those of ${\cal{O}}$, which we discussed in detail in section \ref{subsec:thermalfourier}. From here, it is easy to show that, in our construction, correlation functions are regular across the horizon.

So, we will conclude below that
\begin{quote}
For AdS duals of conformal field theories with a good ${1 \over N}$ expansion at finite temperature, our construction within AdS/CFT predicts that an observer falling through the horizon of a big black hole will measure the correlators predicted by semi-classical GR. In particular, he will not observe anything special about the horizon by doing either high energy\footnote{But not too high. Here we are talking about experiments at energy scale $E$, measured in the local frame of the infalling observer, where $E$ can be anything, as long as it does not scale with $N$ (or $\lambda$ --- in the case of the ${\cal N}=4$ SYM).} or low energy experiments. 
\end{quote}

This conclusion should be contrasted with the fuzzball proposal, which makes a different claim about the nature of the horizon.  We are aware that there are different perspectives on the fuzzball proposal. So, for the sake of precision, we will consider the following statement:
\begin{quote}
{\it (Extrapolated) Fuzzball Proposal:} The geometry is modified near the region where one expects to find the horizon, and it is this modification which contains information about the microstate that resolves the information paradox. It is possible to see these geometrical differences by doing experiments at the scale $T$--- the temperature of the black hole.\footnote{Regardless of the ability of the infalling observer to actually conduct experiments at this energy scale in his lifetime, this is a statement about correlation functions that we can examine.}
\end{quote}
It appears to us that this interpretation of the proposal is necessary if we are to resolve the information paradox using this proposal: the fuzzball proposal resolves the Hawking paradox by finding a mechanism to encode information about the black hole microstate in the Hawking radiation. This is also the statement that we seem to find in the recent literature \cite{Mathur:2009hf,Mathur:2012np, Mathur:2011uj, Avery:2012tf,Chowdhury:2012vd}. 

Since the geometry can be measured by correlators of the metric, and of other {\em light fields}, the fuzzball proposal must imply the correlators of light operators differ between the different states of the ensemble, so that if we do our construction in one state, we get one geometry whereas in another state we get another geometry. More precisely, we take a typical pure state $|\Psi\rangle$ in the CFT and evaluate the correlator  
\begin{equation}
\label{finalcorrelator}
\langle \Psi| \phi_{\text{CFT}}(P_1)...\phi_{\text{CFT}}(P_n) |\Psi\rangle
\end{equation}
where the points $P_1,...P_n$ correspond to bulk points along the trajectory of an infalling observer, then the proposal seems to be that this correlator gives us information about the microstate $|\Psi \rangle$.

Our construction tells us that (a)  $\phi_{\text{CFT}}$ can be written in terms of the boundary operator ${\cal O}$ and (b) to accurately evaluate
correlators involving insertions of $\phi_{\text{CFT}}$ with finite momentum and frequency, we do not need a parametric enhancement of this frequency 
on the boundary. The latter property is discussed in more detail in section \ref{sec:subtleties}.  So, our construction allows us to translate the extrapolated fuzzball proposal into the  following statement in the CFT:
\begin{quote}
{\it Implication of the (Extrapolated) Fuzzball Proposal}: By measuring low point correlators of light operators, we can distinguish between the various microstates of the CFT that constitute a black hole. 
\end{quote}
We will now show that this is impossible if the CFT admits a large $N$ expansion for thermal correlators. If such a large $N$ expansion exists--- as we believe it does for, say, ${\cn = 4}$ Super Yang-Mills at strong coupling--- then our construction, or the idea that bulk correlators can be written as non-singular transforms of CFT correlators is in contradiction with the fuzzball proposal.

Let us now consider a 2-point function for simplicity (higher-point functions can be considered similarly)
\[
\langle \Psi | {\cal O}(x_1) {\cal O}(x_2) | \Psi \rangle.
 \]
 The question is whether we can use this 2-point information to extract some information about the state $|\Psi\rangle$.  
Let us say that we can measure this correlator to order ${1 \over N^{\alpha}}$, where $\alpha$ is some fixed power (that does not scale like $N$). Then we have
\begin{align}
\label{corrfourpt} \langle \Psi | {\cal O}(x_1) {\cal O}(x_2) | \Psi \rangle &= \langle \Psi(\infty) {\cal O}(x_1) {\cal O}(x_2)  
\Psi(0) \rangle \\ \label{corrope} &= \sum_Q C_{{\cal O }{\cal O}}^Q \langle \Psi(\infty) {Q(x_2) \over |x_1 - x_2|^{2 \Delta_{\cal O} - \Delta_Q}} \Psi(0) \rangle \\ \label{corrthreept1}
&= \sum_Q C_{{\cal O}{\cal O}}^Q C_{\Psi \Psi Q}  {1 \over |x_1 - x_2|^{2 \Delta_{\cal O} - \Delta_Q} |x_2|^{\Delta_Q}} \\ \label{corrsumq}
&\equiv \sum_Q G_Q(x_1, x_2).
\end{align}
Here in \eqref{corrfourpt} we have written the 2-point correlator in the state $|\Psi\rangle$ as a 4-point correlator in the vacuum. In \eqref{corrope} we have used the OPE expansion to write the operator product of the two ${\cal O}$ operators as a sum over all other operators $Q$ in the theory, with coefficients that are fixed by conformal invariance up to 3-point coefficients, which are pure numbers, denoted by $C_{{\cal O}{\cal O}}^Q$ above.\footnote{More precisely, this is true only for scalar operators. We have suppressed tensor operators only to lighten the notation. The inclusion of these does not alter the argument in any way.} We have then used conformal invariance again to evaluate the remaining 3-point function, and finally $G_Q$ is merely a short-hand for the contribution of the operator $Q$ to the initial correlator.

Now the key point is that at any given value of $x_1, x_2$, if this correlator has a large $N$ expansion, and if we are measuring it to an accuracy ${1 \over N^{\alpha}}$, then the number of 
operators $Q$ that can contribute to this order must itself be bounded by $N^{\alpha}$. One might have hoped to disentangle the contribution of different operators $Q$ by separating the points $x_1$ and $x_2$ by some order $N$ distance, but as we saw in our construction above we do not need to do so to reconstruct local operators near the horizon if we are measuring bulk correlators that are separated by $O(1)$ near the horizon.

So the correlator really can, at most, contain information about the product $C_{{\cal O}{\cal O}}^Q C_{\Psi \Psi Q}$ for some $\kappa N^{\alpha}$ operators, where $\kappa$ is some  order $1$ number. However, the  black hole consists of $e^{S}$ states. Since $S \propto N$ (we remind the reader that, in our notation, $N$ is a measure of the central charge and not the rank of the boundary gauge group) and the temperature and the leading constants do not scale with $N$ in our setup, and so we can loosely say that we need $e^{N}$ pieces of information to identify the black hole microstate. Clearly this is impossible with the information we can glean from the correlator. To distinguish the different microstates of the black hole, we would need to measure correlators to an accuracy $e^{-N}$. At this level it is not clear (and probably not true) that the correlators have a geometric interpretation.


 It is useful to consider the example of 2-charge solutions \cite{Lunin:2001fv,Lunin:2002bj} which can be identified with the Ramond ground states
 of the  $D1-D5$ CFT. In that case also, most of the classical solutions are string scale. In the AdS$_5$ case, to obtain information about string scale objects, we would need to measure correlators to an accuracy $e^{-\lambda}$. However, what our construction here tells us is that to recover information about a {\it big} black hole, we will need to measure correlators to an accuracy that scales exponentially with $N$, not just with $\lambda$. While string scale geometries might possibly make sense, by switching duality frames, the degeneracy of big black holes looks like it might come from Planck scale geometries.  We do not understand how to discern geometries at the Planck scale.



\subsection{Recovering Information, small corrections, firewalls and complementarity}
We now turn to the issue of the information paradox itself. Indeed, this is the key issue in all the recent discussions on whether the nature of the horizon needs to be modified. For example, several regular solutions, with the same charges as the black hole, have been found in several theories (see \cite{Skenderis:2008qn} for a review and references to the very extended literature). These solutions are extremely interesting. However, by themselves, they are not enough to justify the extrapolated fuzzball proposal that we considered above. Instead, Mathur \cite{Mathur:2009hf} provided an indirect argument for why the nature of the horizon had to be 
modified to resolve the information paradox. 

A very similar argument has recently been made in \cite{Almheiri:2012rt}. However, starting with the same argument as Mathur,  Almheiri et al. stopped at  the (weaker) conclusion that the nature of the horizon must be modified. Since any such modification would generically cause the infalling observer to burn up, these authors spoke of ``firewalls'' rather than fuzzballs. 
 Since our conclusion suggests the opposite --- that the nature of the horizon is {\em not} modified --- we must explain how this is consistent with 
the preservation of information. We briefly review the information loss arguments below. We then describe why we disagree with these arguments and furthermore how our proposal suggests a natural resolution to the information paradox.  
\subsubsection{Strong Subadditivity and the Information Paradox}
\paragraph{Review of the ``Hawking Theorem''}
The argument that the information paradox cannot be resolved if we retain
the hypothesis that the horizon is featureless was termed the ``Hawking theorem'' in \cite{Mathur:2009hf}. 

Consider a toy model of the Hawking process where a pair of entangled qubits are produced at the horizon at each step; one of them is emitted into the radiation outside and the other falls into the black hole. So, at each 
stage we produce the pair
\be
\label{pairproduction}
{1 \over \sqrt{2}} \left( |0 \rangle |0 \rangle + | 1 \rangle | 1 \rangle \right).
\ee
Of these, the second bit goes out of the black hole, whereas the first bit falls inside. We will call the bit that goes outside $B$, while the bit that falls inside is $C$. Then, from \eqref{pairproduction}, we can abstract the lesson that $B$ and $C$ are maximally entangled up to small corrections
\be
\label{bcpure}
S_{BC} \approx 0.
\ee
This more general conclusion also follows from the semi-classical 
observation that, in a certain frame, the state of the black-hole is merely the vacuum, and Hawking radiation is obtained by performing a Bogoliubov transformation on this state. 

If we consider the density matrix of the bit outside, in this case, it is given by
\be
\label{rhosingle}
\rho_B =  {1 \over 2} \begin{pmatrix}1&0\\0&1 \end{pmatrix} \equiv {1 \over 2} I.
\ee
The entanglement entropy of the bit with the black hole is given by
\[
S_B = -\tr \left( \rho_B \log \rho_B \right) = \ln 2.
 \]
However, in general we can abstract away the rule that
\be
\label{bthermal}
S_{B} > 0, \quad S_{C} > 0.
\ee
Since $S_{BC} \approx 0$, we have $S_{C} \approx S_{B}$. 

Apart from the systems $B$ and $C$ designated above, let us also consider the system $A$, which comprises all the radiation that has been emitted by the black hole up to the step under consideration. If we assume that the black hole started in a pure state, then {\em eventually} the entanglement entropy of this radiation with the black hole should start decreasing. 
So, for a very old black hole, it must be true that
\be
\label{BpurifiesA}
S_{AB} < S_A.
\ee
This is the statement that when the bit $B$ goes and joins the radiation $A$, it decreases its entropy. 

Mathur \cite{Mathur:2009hf} pointed out that strong subadditivity tells us that \eqref{BpurifiesA}, \eqref{bcpure} and \eqref{bthermal} appear to be in contradiction. For three {\em separate} systems $A, B, C$, it is possible to show that
\be
\label{strongsubadditivitym}
S_{A} + S_{C} \leq S_{AB} + S_{BC}
\ee
Since we have $S_{A} > S_{AB}$ and $S_{C} > S_{BC}$, we seem to have a contradiction. 

The same argument was made in a slightly different way by \cite{Almheiri:2012rt}. They instead used the identity
\be
\label{strongsubadditivityp}
S_{ABC} + S_{B} \leq S_{AB} + S_{BC},
\ee
which is equivalent to \eqref{strongsubadditivitym}. Since $S_{BC} = 0$ (see \eqref{bcpure}), we have $S_{ABC} = S_{A}$. Then from \eqref{BpurifiesA}, we get
\be
S_{A} + S_{B} < S_{A}  \Rightarrow S_{B} < 0 \, ?
\ee
This clearly contradicts \eqref{bthermal}. 


\paragraph{The Resolution\\}
The resolution to this paradox is automatically provided by our construction of the black hole interior in section \ref{sec:behind}:
\begin{quote}
Our construction makes it clear that for the black hole it is incorrect to apply \eqref{strongsubadditivitym} or \eqref{strongsubadditivityp} since $A, B, C$ {\em cannot} be treated as separate subsystems at the level of
accuracy where \eqref{BpurifiesA} is true. 
\end{quote}
Note that bit $C$ that falls in, is an excitation of the $\widetilde{\cal O}$ operators. However, as we explained in detail above, the $\widetilde{\cal O}$ operators arise because we coarse-grained our initial Hilbert space. 


Now, at the start of black hole evaporation this is a perfectly good description since we can easily accommodate the
outgoing radiation $A$ in our coarse grained Hilbert space.\footnote{We should mention that, in a strict sense, our construction works for a big black hole in AdS. With standard reflecting boundary conditions, this black hole never evaporates. To make it evaporate, we need to couple the boundary 
theory to some other system which collects the ``glueballs'' as they form; such a system could mimic a boundary, which absorbs the radiation that reaches it. However, the reader who does not want to think of such a construction should note that the moral --- obtaining more and more fine details
of the system causes the semi-classical description in terms of a spacetime to break down --- is very robust and  should carry over to flat space black holes.}
As more and more bits emerge, we need to enlarge our coarse-grained Hilbert space to
describe them. 

As we do this, beyond a point, our construction of the $\widetilde{\cal O}$ operators breaks down. In a discrete Hilbert space, this happens when the number of rows in the matrix in \eqref{rectangularmat} (the dimension of ${\cal H}_{\text{coarse}}$) become larger than the number of columns (the dimension of ${\cal H}_{\text{fine}}).$ 

Indeed, it is precisely at this point --- when the density matrix of the radiation outside is of the same dimensionality as the rest of the system --- that we expect \eqref{BpurifiesA} to start being true. 

So, our construction points out that if we wish to make our description of particles outside the black hole so precise that it can keep track of the quantum state of all the particles that have been emitted by an old black hole then, at this level of precision, the semi-classical picture of spacetime is invalid. 

 This {\em does not} imply a breakdown of effective field theory. In fact an infalling observer using effective field theory would measure the operator that we constructed in \eqref{finalbehind} and its correlators. We can rephrase the ``strong subadditivity'' paradox in terms of correlation functions. However, it is clear that the number of operators in these correlation functions would have to scale with $N$. It is for such correlators, with $O(N)$ insertions that we cannot --- and should not expect to be able to ---  use effective field theory. 

We should mention that a variant of our ideas was discussed in \cite{Bousso:2012as,Susskind:2012uw}, where it was written in the form $A = R_B$ indicating that the interior of the black-hole was, in some sense, a ``scrambled'' version of the exterior. The authors of \cite{Bousso:2012as,Susskind:2012uw} decided that this led to problems involving ``quantum cloning'' (quantum information appears to be duplicated), and with ``chronology protection''. Our idea is subtly different: when the CFT is obsered at a coarse-grained level, it is possible to write down a semi-classical spacetime that reproduces these observations, with a very specific combination of the fine-grained degrees of freedom playing the part of the interior of the black hole. Both these regions are rewritings of parts of the same CFT Hilbert space. If we insist on a higher level of accuracy, then it is not quantum mechanics that breaks down but rather the {\em interpretation of CFT correlators} in terms of a semi-classical and local spacetime. 

\subsubsection{Can small correlations unitarize Hawking radiation?}
Once we have resolved the strong subadditivity paradox, there is still
apparently an information puzzle that makes no reference to the ingoing bit (i.e. to system $C$) : how can the large number of apparently
thermal bits that are emitted by the black hole lead to a pure state. In fact the resolution to this has
been understood for a long time \cite{Page:1993df}. We review this puzzle here and its resolution also.  The reader who is already familiar with this
can skip straight to the example in \ref{sec:toy}.
\paragraph{The Paradox\\}
Naively if we assume that the {\em same process} of Hawking radiation repeats $K$ times (where $K$ is the number of bits emitted), then the density matrix of the radiation outside will look like
\[
\rho_K = (\rho_B)^K,
 \]
where, by $\rho_B^K$ we really mean
\[
\rho_0^K \equiv \rho \otimes \rho \otimes \rho \ldots {\text{K~times}},
 \]
i.e. it is a $2^K \times 2^K$ dimensional matrix.

The entropy of this density matrix is
\be
\label{entangleK}
S_{\text{hawk}} = -\tr(\rho_K \ln \rho_K) = K \ln 2.
\ee

If we modify the density matrix \eqref{rhosingle} by a small amount --- such corrections would be expected through quantum and stringy effects --- but continue to assume that each Hawking emission is {\em exactly independent} then the entanglement entropy computed above does not change appreciably.

More precisely, if we take each individual density matrix to be
\[
\rho_{\text{str}} = \rho_1 + \epsilon \rho_{\text{corr}},
 \]
then
\[
S_{\text{hawk}} - \left[- \tr(\rho_{\text{str}}) \ln (\rho_{\text{str}})\right] \sim O(\epsilon).
 \]
This conclusion continues to hold even if we allow $\rho_{\text{corr}}$ at each step to be different but {\em uncorrelated}.

The paper \cite{Mathur:2012np} explains that ordinary objects like coal or burning paper avoid this paradox since in those cases successive emissions are not uncorrelated. So, for example, if the computer on which we are typing this were to go up in flames then successive portions of the computer would emit distinct and identifiable thermal emissions. First, the keyboard would burn, and by collecting that radiation, an observer could recover some information about the keyboard, even if the keyboard were only a negligible fraction of the mass of the full computer. 

However, this is simply the statement that small subsystems of everyday objects are {\em not} maximally entangled with the rest of the object. As we will show in a toy example below, by introducing very small off-diagonal elements (which, however, may link the first bit emitted to the last bit emitted), we can unitarize Hawking radiation. 

\paragraph{The Resolution\\}

Just to be specific, let us consider a 4-dimensional asymptotically flat black hole of Schwarzschild radius $R_{bh}$ and take $l_{\rm p}$ to be the Planck scale. In the lifetime of this black hole, it will emit about $N \approx \left({ R_{\text{bh}} \over l_{\text{p}}}\right)^2$ quanta.  Clearly, for the first $K << N$ of these quanta, the correction is unimportant and the density matrix is well approximated by the Hawking matrix. 

Let us imagine that the density matrix that is produced at each step is actually given by
\be
\label{unitarizingcorr}
\rho_{\text{exact}} = \rho_{\text{hawk}} + 2^{-N} \rho_{\text{corr}},
\ee
where $\rho_{\text{corr}}$ is a density matrix that, in the natural basis of observables, has elements that are $O(1)$. This correction is exponentially suppressed and can have several origins. Moreover, since it is exponentially suppressed in ${R_{\text{bh}} \over l_{\text{p}}}$, it cannot be detected at any order in perturbation theory: this is an inherently non-perturbative correction. 

This correction is reminiscent of the ``second saddle point'' discussed in \cite{Maldacena:2001kr}. It would be nice to see if such a semi-classical correction can be identified directly in Lorentzian space. However, even if such a semi-classical perspective does not exist, and this correction is visible in the gauge theory but cannot be interpreted in terms of a simple geometric process, that would not be a contradiction. 

Now, it is clear from \eqref{unitarizingcorr} that for the first $K$ emissions, where $K << N$, the density matrix of the radiation is given by the Hawking matrix to an excellent approximation. However, as $K$ grows large and becomes comparable to $N$, we see that the individual elements of the Hawking density matrix become so small, that they are comparable to the size of the corrections. At this point, the corrections can no longer be neglected.

In fact, with the numerical pre-factors that we have inserted in front, the corrections are precisely of the correct order to unitarize the process. Note that after $N$ steps, the Hawking density matrix looks like
\[
\rho_{\text{hawk}} = {1 \over 2^N} I_{2^N \times 2^N},
 \]
where $I_{2^N \times 2^N}$ is the identity matrix in $2^N$ dimensions. 

For the full density matrix to be unitary, we must have, {\em in some basis}, 
\[
\rho_{\text{exact}} = \begin{pmatrix}1&0&0&\ldots&0\\
0&0&0&\ldots&0\\
\ldots&\ldots&\ldots&\ldots&\ldots \\
0&0&0&\ldots&0\\
\end{pmatrix}.
 \]
So, in this basis---which may be quite unnatural from the point of observables accessible to an observer at low energies---the correction matrix must look like
\be
\label{rhocorrunusualbasis}
\rho_{\text{corr}} 
%= \rho_{\text{exact}} 
= \begin{pmatrix}2^N-1&0&0&\ldots&0\\
0&-1&0&\ldots&0\\
\ldots&\ldots&\ldots&\ldots&\ldots \\
0&0&0&\ldots&-1\\
\end{pmatrix}.
\ee
We now show that this is not inconsistent with the statement below \eqref{unitarizingcorr}--- that $\rho_{\text{corr}}$ has all elements of $O(1)$ in a basis of natural observables. 

For \eqref{rhocorrunusualbasis} to hold, we must have
\[
\tr(\rho_{\text{corr}}^2) = 2^{2 N}.
 \]
However, even if all the elements of $\rho_{\text{corr}}$ in an ordinary basis are $O(1)$, we have
\[
\tr(\rho_{\text{corr}}^2)= \sum_{i j} \rho_{\text{corr}}^{i j} \rho_{\text{corr}}^{i j} = \Or[2^{2N}],
 \]
since the sum over $i,j$ runs over $2^{2N}$ elements. 

\subsubsection{An Example \label{sec:toy}}
Let us now give an example where all the ideas are realized. In particular:
\begin{enumerate}
\item
We will construct a toy model that describes the emission of ``spins''; each emitted spin has an almost exactly thermal density matrix. 
\item
We will show how small corrections can unitarize this process so that after more than half the system has ``evaporated'', the density matrix that describes all the emitted spins together starts becoming pure.
\item
We will show how, for each spin that is emitted, we can also identify (within an effective description), a corresponding ``spin'' that remains within the system and is perfectly anti-correlated with the emitted spin. This is the analogue of the bit ``C'' that falls inside the black hole. However, we will show that 
this effective description breaks down at {\em precisely} in time to avoid any contradiction with the strong subadditivity of entropy.
\end{enumerate}

Consider a system of $N$ spins, each of which can be in two states--- $|0\rangle$, or $|1 \rangle$. The Hilbert space of the system is spanned by $2^N$ basis vectors, each of which can be represented by a single binary number between $0$ and $2^{N-1}$. We can use this as a convenient representation of our basis
\[
|0 \rangle_{b} \equiv |000\ldots00 \rangle, |1 \rangle_{b} \equiv |000\ldots01\rangle, |2 \rangle_{b} \equiv |000\ldots10\rangle, |3 \rangle_{b} \equiv |000\ldots11\rangle, \ldots,
 \]
where we have placed a $b$ in the subscript of the new basis to distinguish it from the old one.
Now, consider the state
\be
\label{typicalstate}
|\Psi\rangle = {1 \over 2^{N/2}} \sum_{j=0}^{2^N-1} p_j | j \rangle_b,
\ee
where $p_j$ is a number that can be either $1$ or $-1$. There are $2^{2^N-1}$ such states (since there are $2^N$ coefficients, but states that are the same up to an overall minus sign are equivalent) but let us consider a ``typical state'' in this set --- where we pick the $p_j$ coefficients with an equal probability to be either $1$ or $-1$.

In doing so, we are not considering the ensemble of states formed by the various choices of the $p_j$. We make some choice, and then consider the pure state $|\Psi\rangle$ corresponding to this choice. In speaking of a ``random'' choice above, we are merely pointing out that for our purposes the precise choice of the sequence $p_j$ is not important, and most choices will work.

Now, if we consider the reduced density matrix corresponding to the first spin, we find that we have
\begin{align}
\rho_1 &= {1 \over 2^{N}} \left(\sum_{j=0}^{2^{N-1}-1} p_{2 j}^2 |0 \rangle \langle 0| + p_{2 j+1}^2 |1 \rangle \langle 1| + p_{2 j} p_{2 j +1}\Big(
|0 \rangle \langle 1| +|1 \rangle \langle 0| \Big)\right) \\
&= {1 \over 2} \Bigg(|0 \rangle \langle 0| + |1 \rangle \langle 1| + \Or[2^{-{N\over2}}] \Big(|0 \rangle \langle 1|+|1 \rangle \langle 0| \Big) \Bigg).
\end{align}
Here, in the last step we have used the fact that successive $p_j$ coefficients are uncorrelated. Depending on the precise choice of the $p_j$ coefficients, that we started out with, we will get different values for ${1 \over 2^N} \sum p_{2 j} p_{2 j + 1}$. However, for a typical state, we expect this value to be of the order $2^{-{N \over 2}}$.\footnote{For a very few states, this value can be close to $1$, but these states are not ``typical'' and do not meet our purpose.}

Thus we see that for ``typical states'', the density matrix of the first spin is very close to the density matrix predicted by Hawking. The full density matrix is also of the form \eqref{unitarizingcorr}. However, the corrections become important once we start considering (in this example) ${N \over 2}$ spins. 

Thus if the dynamics of this spin chain naturally leads it to one of these typical states, then its evaporation will appear much like Hawking radiation. By a basis change, we can transform $\rho_{\text{corr}}$ into precisely the form \eqref{rhocorrunusualbasis}.

However, this is not sufficient. Indeed, as we pointed out above, a key feature of Hawking radiation is that for each emitted photon, there is also an ``ingoing photon'' that can perhaps be observed by an observer inside the black hole. The two photons are maximally entangled; each of them individually looks thermal but taken together they form a pure state. We referred to this feature in \eqref{pairproduction} above.

From our construction of the $\widetilde{O}$ operators above, we know how to construct the ingoing bit. To make this more precise, let us assume that in a ``coarse-grained'' description we can observe the first $p$ bits of the $N$ qubits in the toy model above. We assume that the other $N-p$ bits are much harder to observe and enter only in the ``fine grained'' description of the system.  In the same notation as above, we can write down a basis for both of these spaces
\[
\begin{split}
\text{coarse}: \quad |i\rangle_{\text{c}}, ~ \text{with}~ 0 \leq i \leq 2^p - 1\\
\text{fine}:\quad |j\rangle_{\text{f}}, ~\text{with}~ 0 \leq j \leq 2^{N-p} - 1.
\end{split}
 \]
The subscripts $c$ and $f$ emphasize that these are in a different vector space, from the space we used to describe the full system above. 
Now the state $|\Psi\rangle$ can clearly be written as
\[
| \Psi \rangle = \sum_{i, j} a_{i j} |i, j \rangle = \sum_i |i \rangle_{\text{c}} \otimes | \phi_i \rangle_{\text{f}},
 \]
where $|i, j\rangle \equiv |i\rangle_{\text{c}} \otimes |j\rangle_{\text{f}}$ and 
\[
|\phi_i \rangle_{\text{f}} \equiv \sum_j a_{i j} | j \rangle_{\text{f}}.
 \]
Let us call $S_1$ the operator acting on the first spin as the Pauli matrix $\sigma_3 = \begin{pmatrix} & 1 & 0 \\ & 0 & -1\end{pmatrix}$.
With this notation, the operator $\widetilde{S}_1$ in this basis can be written as
\be
\label{tildes1}
\widetilde{S}_1 = {\cal I}_{\text{c}} \otimes \left( \sum_{i=2^{p-1}}^{2^p - 1} | \phi_i \rangle  \langle \phi_i | - \sum_{i = 0}^{2^{p-1} - 1} | \phi_i \rangle \langle \phi_i | \right),
\ee
where ${\cal I}_{\text{c}}$ is the identity on the coarse-grained space. 

Note that the definition of $\widetilde{S}_1$ depends not only on the state $| \Psi \rangle$ but also on our division of the space into a coarse-grained and a fine-grained part. This operator $\widetilde{S}_1$ commutes with all operators in the coarse-grained space as is evident from \eqref{tildes1}
\[
[\widetilde{S}_1, S_m] = 0, ~\text{for}~1 \leq m \leq p.
 \]
Moreover, in the state $|\Psi \rangle$ measurements of $S_1$ are precisely anti-correlated with measurements of $\widetilde{S}_1$.  Similarly, we can define operators $\widetilde{S}_2, \ldots \widetilde{S}_p$ that are anti-correlated with $S_2 \ldots S_p$ respectively, and also commute with all the ordinary operators.

It is also important to recognize where this process breaks down. Once we expand our description of the coarse-grained system to $p = {N \over 2}$, then the construction above is no longer possible; there is simply not ``enough space'' in the fine-grained space. 
%Nevertheless, we could still define effective operators $\widetilde{S}_m$ that behave approximately like the operators above. For example, even for $p > {N \over 2}$, we could retain the definition of $\widetilde{S}_1$ from the case where $p = {N \over 2}$. This $\widetilde{S}_1$ will not commute with, say, $S_{{N \over 2} + 1}$. However, to detect this lack of commutation in the state $|\Psi\rangle$, we will have to make measurements that are precise up to $2^{-{N \over 2}}$.

Before we conclude this section, we would like to emphasize that this toy model holds important lessons for what happens in the real black hole. First it shows us how exponentially suppressed corrections can conspire, in an equally large Hilbert space, to restore unitarity to a seemingly thermal density matrix. Second, this system also gives us a toy model of black hole complementarity.  As long as less than half the spins have been measured, it is possible to pair each emitted spin with a perfectly anti-correlated and independent degree of freedom in the remaining spins. This is analogous to the statement that as long as we are measuring a suitably limited number of observables, there is no difficulty in describing the interior of the black hole as a separate space where semi-classical dynamics holds. However, once we start probing the system very finely by going either to very high energies, or with very heavy operators, the semi-classical spacetime description starts to break down. This is precisely what 
happens 
in the spin model above.  

%%% Local Variables: 
%%% mode: latex
%%% TeX-master: "infalling_paper"
%%% End: 

