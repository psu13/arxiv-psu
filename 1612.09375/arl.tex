% Basic Category Theory
% Tom Leinster <Tom.Leinster@ed.ac.uk>
% 
% Copyright (c) Tom Leinster 2014-2016
% 
% Chapter 6: Adjoints, representables and limits
% 

\chapter{Adjoints, representables and limits}
\label{ch:arl}


We have approached the idea of universal property from three different
angles, producing three different formalisms: adjointness,
representability, and limits.  In this final chapter, we work out the
connections between them.

In principle, anything that can be described in one of the three formalisms
can also be described in the others.  The situation is similar to that of
cartesian and polar coordinates: anything that can be done in polar
coordinates can in principle be done in cartesian coordinates, and vice
versa, but some things are more gracefully done in one system than the
other.

In comparing the three approaches, we will discover many of the fundamental
results of category theory.  Here are some highlights.
% 
\begin{itemize}
\item 
Limits and colimits in functor categories work in the simplest possible
way.

\item 
The embedding of a category $\scat{A}$ into its presheaf category
$\pshf{\scat{A}}$ preserves limits (but not colimits).

\item 
The representables are the prime numbers of presheaves: every presheaf can
be expressed canonically as a colimit of representables.

\item 
A functor with a left adjoint preserves limits.  Under suitable hypotheses,
the converse holds too.

\item 
Categories of presheaves $\pshf{\scat{A}}$ behave very much like the
category of sets, the beginning of an incredible story that brings together
the subjects of logic and geometry.
\end{itemize}



\section{Limits in terms of representables and adjoints}
\label{sec:lra}


There is more than one way to present the definition of limit.  In
Chapter~\ref{ch:lims}, we used an explicit form of the definition that is
particularly convenient for examples.  But we will soon be developing the
\emph{theory} of limits and colimits, and for that, a rephrased form of the
definition is useful.  In fact, we rephrase it in two different ways: once
in terms of representability, and once in terms of adjoints.

We begin by showing that cones are simply natural transformations of a
special kind.  To do this, we need some notation.  Given categories
$\scat{I}$ and $\cat{A}$ and an object $A \in \cat{A}$, there is a functor
$\Delta A\from \scat{I} \to \cat{A}$ with constant value $A$ on objects and
$1_A$ on maps.  This defines, for each $\scat{I}$ and $\cat{A}$, the
\demph{diagonal%
%
\index{functor!diagonal}
%
functor}
\[
\Delta\from \cat{A} \to \ftrcat{\scat{I}}{\cat{A}}.%
%
\ntn{diag-gen}
%
\]
The name can be understood by considering the case in which $\scat{I}$ is the
discrete category with two objects; then $\ftrcat{\scat{I}}{\cat{A}} = \cat{A}
\times \cat{A}$ and $\Delta(A) = (A, A)$.

Now, given a diagram $D\from \scat{I} \to \cat{A}$ and an object $A \in
\cat{A}$, a cone on $D$ with vertex $A$ is simply%
%
\index{cone!natural transformation@as natural transformation}
%
a natural transformation
\[
\xymatrix{
\scat{I} \rtwocell^{\Delta A}_{D} &\cat{A}.
}
\]
Writing $\Cone(A, D)$%
%
\ntn{Cone}
%
for the set of cones on $D$ with vertex $A$, we therefore have
% 
\begin{equation}        
\label{eq:cones-as-transfs}
\Cone(A, D)
=
\ftrcat{\scat{I}}{\cat{A}} (\Delta A, D).
\end{equation}
% 
Thus, $\Cone(A, D)$ is functorial in $A$ (contravariantly) and $D$
(covariantly).

Here is our first rephrasing of the definition of limit.

\begin{propn}   
\label{propn:lim-rep}
%
\index{limit!representation of cone functor@as representation of cone
  functor} 
%
Let $\scat{I}$ be a small category, $\cat{A}$ a category, and $D\from \scat{I}
\to \cat{A}$ a diagram.  Then there is a one-to-one correspondence between
limit cones on $D$ and representations of the functor
\[
\Cone(\dashbk, D) \from \cat{A}^\op \to \Set,
\]
with the representing objects of \hspace{.05em}$\Cone(\dashbk, D)$ being the
limit objects (that is, the vertices of the limit cones) of $D$.
\end{propn}

Briefly put: a limit of $D$ is a representation of
$\ftrcat{\scat{I}}{\cat{A}}(\Delta\dashbk, D)$. 

\begin{pf}
By Corollary~\ref{cor:rep-univ}, a representation of $\Cone(\dashbk, D)$
consists of a cone on $D$ with a certain universal property.  This is exactly
the universal property in the definition of limit cone.
\end{pf}

The proposition formalizes the thought that cones on a diagram $D$
correspond one-to-one with maps into $\lt{\scat{I}} D$.  It implies that if
$D$ has a limit then
% 
\begin{equation}        
\label{eq:cone-is-map-to-lim}
\Cone(A, D) \iso \cat{A}\biggl(A, \lt{\scat{I}} D\biggr)
\end{equation}
% 
naturally in $A$.  The correspondence is given from left to right by 
\[
(f_I)_{I \in \scat{I}} \mapsto \bar{f}
\]
(in the notation of Definition~\ref{defn:lim}), and from right to left by 
\[
(p_I \of g)_{I \in \scat{I}} \mapsfrom g
\]  
where $p_I\from \lt{\scat{I}} D \to D(I)$ are the projections.

From Proposition~\ref{propn:lim-rep} and Corollary~\ref{cor:reps-unique} we
deduce: 
% 
\begin{cor}     
\label{cor:lims-unique}
%
\index{limit!uniqueness of}
% 
Limits are unique up to isomorphism.
\qed
\end{cor}

The characterization~\eqref{eq:cones-as-transfs} of cones suggests that we
might consider varying the diagram $D$ as well as the vertex $A$.  We are
naturally led to ask questions such as: given a map $D \to D'$ between
diagrams, is there an induced map between the limits of $D$ and $D'$?  The
answer is yes (Figure~\ref{fig:induced-maps}):

\begin{figure}
\[
\xymatrix@R=2ex{
        &       &\,      \\
A \ar[r] \ar[dddd]_s 
\ar[rru] \ar[rrd]       
\ar@{}[rdddd]|-{\text{\normalsize(b)}}&
\lim D \ar@{.>}[dddd]^{\lim \alpha}       
\ar[ru] \ar[rd]
\ar@{}[rdddd]|-{\text{\normalsize(a)}}&
\save*+<10ex>[o][F]{\,}\restore \hspace*{5.5em}D 
        \\
        &       & \ar[dd]^\alpha        \\
        &       &        \\
        &       &\,      \\
A' \ar[r]   
\ar[rru] \ar[rrd]       &
\lim D' \ar[ru] \ar[rd]         &       
\save*+<10ex>[o][F]{\,}\restore \hspace*{5.8em}D'
\\
        &       &\,       
}
\]
\caption{Illustration of Lemma~\ref{lemma:lim-functorial}.}
\label{fig:induced-maps}
\end{figure}

\begin{lemma}   
\label{lemma:lim-functorial}
Let $\scat{I}$ be a small category and $\xymatrix@1{\scat{I}
  \rtwocell^D_{D'}{\alpha} &\cat{A}}$ a natural transformation.  Let
\[
\biggl(\lt{\scat{I}} D \toby{p_I} D(I)\biggr)_{I \in \scat{I}}
\qquad
\text{and}
\qquad
\biggl(\lt{\scat{I}} D' \toby{p'_I} D'(I)\biggr)_{I \in \scat{I}}
\]
be limit cones.  Then:
% 
\begin{enumerate}[(b)]
\item   
\label{lemma:lim-functorial:induced}
there is a unique map%
%
\index{limit!map between limits}
%
$\lt{\scat{I}} \alpha \from \lt{\scat{I}} D \to \lt{\scat{I}} D'$ such that
for all $I \in \scat{I}$, the square
\[
\xymatrix{
\lt{\scat{I}} D \ar[r]^-{p_I} \ar[d]_{\lt{\scat{I}} \alpha}     &
D(I) \ar[d]^{\alpha_I}    \\
\lt{\scat{I}} D' \ar[r]_-{p'_I}         &
D'(I)
}
\]
commutes;

\item   
\label{lemma:lim-functorial:commutes}
given cones $\Bigl(A \toby{f_I} D(I)\Bigr)_{I \in \scat{I}}$ and $\Bigl(A'
\toby{f'_I} D'(I)\Bigr)_{I \in \scat{I}}$ and a map $s\from A \to A'$
such that
\[
\xymatrix{
A \ar[r]^-{f_I} \ar[d]_s        &
D(I) \ar[d]^{\alpha_I}  \\
A' \ar[r]_-{f'_I}       &
D'(I)
}
\]
commutes for all $I \in \scat{I}$, the square
\[
\xymatrix{
A \ar[r]^-{\bar{f}} \ar[d]_s        &
\lt{\scat{I}} D \ar[d]^{\lt{\scat{I}} \alpha}  \\
A' \ar[r]_-{\ovln{f'}}       &
\lt{\scat{I}} D'
}
\]
also commutes.
\end{enumerate}
\end{lemma}

\begin{pf}
Part~\bref{lemma:lim-functorial:induced} follows immediately from the fact
that $\biggl(\lt{\scat{I}} D \toby{\alpha_I p_I} D'(I)\biggr)_{I \in
  \scat{I}}$ is a cone on $D'$.  To
prove~\bref{lemma:lim-functorial:commutes}, note that for each $I \in
\scat{I}$, we have
\[
p'_I \of \biggl(\lt{\scat{I}} \alpha\biggr) \of \bar{f}
=
\alpha_I \of p_I \of \bar{f}
=
\alpha_I \of f_I
=
f'_I \of s
=
p'_I \of \ovln{f'} \of s.
\]
So by Exercise~\ref{ex:jointly-monic}\bref{part:j-m-main},
$\biggl(\lt{\scat{I}} \alpha \biggr) \of \bar{f} = \ovln{f'} \of s$.
\end{pf}

We can now give the second rephrasing of the definition of limit.  It only
applies when the category has \emph{all} limits of the shape concerned.

\begin{propn}   
\label{propn:lim-const-adjn}
%
\index{limit!adjoint@as adjoint}
%
Let $\scat{I}$ be a small category and $\cat{A}$ a category with all limits of
shape $\scat{I}$.  Then $\lt{\scat{I}}$ defines a functor
$\ftrcat{\scat{I}}{\cat{A}} \to \cat{A}$, and this functor is right adjoint
to the diagonal functor.  
\end{propn}

\begin{pf}
Choose for each $D \in \ftrcat{\scat{I}}{\cat{A}}$ a limit cone on $D$, and
call its vertex $\lt{\scat{I}} D$.  For each map $\alpha\from D \to D'$ in
$\ftrcat{\scat{I}}{\cat{A}}$, we have a canonical map
$\lt{\scat{I}} \alpha \from \lt{\scat{I}} D \to \lt{\scat{I}} D'$,
defined as in
Lemma~\ref{lemma:lim-functorial}\bref{lemma:lim-functorial:induced}.  This
makes $\lt{\scat{I}}$ into a functor.  Proposition~\ref{propn:lim-rep}
implies that
\[
\ftrcat{\scat{I}}{\cat{A}}(\Delta A, D)
=
\Cone(A, D)
\iso
\cat{A}\biggl(A, \lt{\scat{I}} D\biggr)
\]
naturally in $A$, and taking $s = 1_A$ in
Lemma~\ref{lemma:lim-functorial}\bref{lemma:lim-functorial:commutes} tells
us that the isomorphism is also natural in $D$.
\end{pf}

To define the functor $\lt{\scat{I}}$, we had to \emph{choose}%
%
\index{limit!uniqueness of}
%
for each $D$ a limit cone on $D$.  This is a non-canonical choice.
Nevertheless, different choices only affect the functor $\lt{\scat{I}}$ up
to natural isomorphism, by uniqueness of adjoints.


\exs


\begin{question}
Interpret all the theory of this section in the special case where
$\scat{I}$ is the discrete category with two objects.
\end{question}


\begin{question}
What is the content of Proposition~\ref{propn:lim-const-adjn} when
$\scat{I}$ is a group and $\cat{A} = \Set$?  What about the dual of
Proposition~\ref{propn:lim-const-adjn}? 
\end{question}



\section{Limits and colimits of presheaves}
\label{sec:lim-pshf}


What do limits and colimits look like in functor categories
$\ftrcat{\cat{A}}{\cat{B}}$?  In particular, what do they look like in
presheaf categories $\ftrcat{\cat{A}^\op}{\Set}$?  More particularly still,
what about limits and colimits of representables?  Are they, too,
representable?

We will answer all these questions.  In order to do so, we first prove that
representables preserve limits.  


\minihead{Representables preserve limits}
%
\index{functor!representable!preserves limits|(}
%

Let us begin by recalling that, by definition of product, a map $A \to X
\times Y$ amounts to a pair of maps $(A \to X,\, A \to Y)$.  Here $A$, $X$ and
$Y$ are objects of a category $\cat{A}$ with binary products.  There is,
therefore, a bijection
% 
\begin{equation}
\label{eq:prod-rep}
\cat{A}(A, X \times Y)
\iso 
\cat{A}(A, X) \times \cat{A}(A, Y)
%
\index{product!map into}
%
\end{equation}
% 
natural in $A, X, Y \in \cat{A}$.  

Is this a special feature of products, or does some analogous statement
hold for every kind of limit?  Let us try equalizers.  Suppose that
$\cat{A}$ has equalizers, and write
$\Eq\biggl(\parpairi{X}{Y}{s}{t}\biggr)$ for the equalizer of maps $s$ and
$t$.  By definition of equalizer, maps
% 
\begin{equation}
\label{eq:map-to-eq}
A \: \to \: \Eq\biggl(\parpairi{X}{Y}{s}{t}\biggr)
\end{equation}
% 
correspond one-to-one with maps $f\from A \to X$ such that $s \of f = t \of
f$.  Now recall that $s$ induces a map 
\[
s_* = \cat{A}(A, s)\from \cat{A}(A, X) \to \cat{A}(A, Y),
\]
and similarly for $t$.  In this notation, what we have just said is that
maps~\eqref{eq:map-to-eq} correspond one-to-one with elements $f \in
\cat{A}(A, X)$ such that
\[
\bigl(\cat{A}(A, s)\bigr)(f) = \bigl(\cat{A}(A, t)\bigr)(f).
\]
By the explicit formula for equalizers in $\Set$
(Example~\ref{eg:equalizers-Set}), such an $f$ is exactly an element of the
equalizer of $\cat{A}(A, s)$ and $\cat{A}(A, t)$.  So, we have a canonical
bijection
% 
\begin{equation}
\label{eq:equalizer-rep}
\cat{A}
\biggl( 
A,\, \Eq\Bigl(\!\parpair{X}{Y}{s}{t}\!\Bigr)
\biggr)
\iso 
\Eq\biggl(\!
\parpair{\cat{A}(A, X)}{\cat{A}(A, Y)}{\cat{A}(A, s)}{\cat{A}(A, t)}
\!\biggr).
%
\index{equalizer!map into}
%
\end{equation}
% 
This looks something like our isomorphism~\eqref{eq:prod-rep} for products.

The isomorphisms~\eqref{eq:prod-rep} and~\eqref{eq:equalizer-rep} suggest
that, more generally, we might have
% 
\begin{equation}        
\label{eq:rep-pres-lims}
\cat{A}\biggl(A, \lt{\scat{I}} D\biggr)
\iso
\lt{\scat{I}} \cat{A}(A, D)
\end{equation}
% 
naturally in $A \in \cat{A}$ and $D \in \ftrcat{\scat{I}}{\cat{A}}$, whenever 
$\cat{A}$ is a category with limits of shape $\scat{I}$.  Here $\cat{A}(A,
D)$%
%
\ntn{AAD}
%
is the functor
\[
\begin{array}{cccc}
\cat{A}(A, D)\from      &\scat{I}       &\to        &\Set   \\
                        &I              &\mapsto    &\cat{A}(A, D(I)).
\end{array}
\]
This functor could also be written as $\cat{A}(A, D(\dashbk))$, and is the
composite 
\[
\xymatrix@1@C+1em{
\scat{I} \ar[r]^D       &
\cat{A} \ar[r]^-{\cat{A}(A, \dashbk)}    &
\Set.
}
\]
The conjectured isomorphism~\eqref{eq:rep-pres-lims} states, essentially,
that representables preserve limits.  We now set about proving this.  

\begin{lemma}   
\label{lemma:cone-rep}
Let $\scat{I}$ be a small category, $\cat{A}$ a locally small category, $D\from
\scat{I} \to \cat{A}$ a diagram, and $A \in \cat{A}$.  Then
\[
\Cone(A, D) 
\iso 
\lt{\scat{I}} \cat{A}(A, D)
%
\index{cone!set of cones as limit}
%
\]
naturally in $A$ and $D$.
\end{lemma}

\begin{pf}
Like all functors from a small category into $\Set$, the functor
$\cat{A}(A, D)$ does have a limit, given by the explicit
formula~\eqref{eq:Set-lim}.  According to this formula, $\lt{\scat{I}}
\cat{A}(A, D)$ is the set of all families $(f_I)_{I \in \scat{I}}$ such
that $f_I \in \cat{A}(A, D(I))$ for all $I \in \scat{I}$ and
% 
\begin{equation}        
\label{eq:cone-from-lim}
(\cat{A}(A, Du))(f_I) 
= 
f_J
\end{equation}
% 
for all $I \toby{u} J$ in $\scat{I}$.  But equation~\eqref{eq:cone-from-lim}
just says that $(Du) \of f_I = f_J$, so an element of
$\lt{\scat{I}} \cat{A}(A, D)$ is nothing but a cone on $D$ with vertex $A$.  
\end{pf}

\begin{propn}[Representables preserve limits] 
\label{propn:reps-cts}
\hspace*{-5pt}%
Let $\cat{A}$ be a locally small category and $A \in \cat{A}$.  Then
$\cat{A}(A, \dashbk) \from \cat{A} \to \Set$ preserves limits.
\end{propn}

\begin{pf}
Let $\scat{I}$ be a small category and let $D\from \scat{I} \to \cat{A}$ be a
diagram that has a limit.  Then
\[
\cat{A}\biggl(A, \lt{\scat{I}} D\biggr)
\iso
\Cone(A, D)
\iso
\lt{\scat{I}} \cat{A}(A, D)
\]
naturally in $A$.  Here the first isomorphism is
Proposition~\ref{propn:lim-rep} (or more particularly, the
isomorphism~\eqref{eq:cone-is-map-to-lim} that follows it), and the second is
Lemma~\ref{lemma:cone-rep}.
\end{pf}

\begin{remark}  
\label{rmk:rep-pres}
Proposition~\ref{propn:reps-cts} tells us that
% 
\begin{equation}        
\label{eq:lim-out}
\cat{A} \biggl(A, \lt{\scat{I}} D\biggr)
\iso
\lt{\scat{I}} \cat{A}(A, D).
%
\index{limit!map into}
%
\end{equation}
% 
To dualize Proposition~\ref{propn:reps-cts}, we replace $\cat{A}$ by
$\cat{A}^\op$.  Thus, $\cat{A}(\dashbk, A)\from \cat{A}^\op \to \Set$
preserves limits.  A limit in $\cat{A}^\op$ is a colimit in $\cat{A}$, so
$\cat{A}(\dashbk, A)$ transforms colimits in $\cat{A}$ into limits in
$\Set$:
% 
\begin{equation}
\label{eq:colim-out}
\cat{A}\biggl(\colt{\scat{I}} D, A\biggr)
\iso
\lt{\scat{I}^{\op}} \cat{A}(D, A).
%
\index{colimit!map out of}
%
\end{equation}
% 
The right-hand side is a \emph{limit}, not%
%
\index{limit!colimit@vs.\ colimit}
%
a colimit!  So even though~\eqref{eq:lim-out} and~\eqref{eq:colim-out} are
dual statements, there are, in total, more limits than colimits involved.
Somehow, limits have the upper hand.

For example, let $X$, $Y$ and $A$ be objects of a category $\cat{A}$, and
suppose that the sum $X + Y$ exists.  By definition of sum, a map $X + Y
\to A$%
%
\index{sum!map out of}
%
amounts to a pair of maps $(X \to A,\, Y \to A)$.  In other words, there is
a canonical isomorphism
\[
\cat{A}(X + Y, A) 
\iso
\cat{A}(X, A) \times \cat{A}(Y, A).
\]
This is the isomorphism~\eqref{eq:colim-out} in the case where $\scat{I}$ is
the discrete category with two objects.%
%
\index{functor!representable!preserves limits|)}
%
\end{remark}


\minihead{Limits in functor categories}
%
\index{functor!category!limits in|(}%
\index{limit!functor category@in functor category|(}%
%

Earlier, we learned that it is sometimes useful to view functors as objects
in their own right, rather than as maps of categories.  For instance, when
$G$ is a group, functors $G \to \Set$ are $G$-sets
(Example~\ref{eg:functor-action}), which one would usually regard as
`things' rather than `maps'.  This point of view leads to the concept of
functor category.

We now begin an analysis of limits and colimits in functor categories
$\ftrcat{\scat{A}}{\cat{S}}$.  Here $\scat{A}$ is small and $\cat{S}$ is
locally small; these conditions together guarantee that
$\ftrcat{\scat{A}}{\cat{S}}$ is locally small.  The most important cases
for us will be $\cat{S} = \Set$ and $\cat{S} = \Set^\op$.  For that reason,
we will assume whenever necessary that $\cat{S}$ has all limits and
colimits.

We show that limits and colimits in $\ftrcat{\scat{A}}{\cat{S}}$ work in
the simplest way imaginable.  For instance, if $\cat{S}$ has binary products
then so does $\ftrcat{\scat{A}}{\cat{S}}$, and the product%
%
\index{functor!product of functors}
%
of two functors $X, Y\from \scat{A} \to \cat{S}$ is the functor $X \times
Y\from \scat{A} \to \cat{S}$ given by
\[
(X \times Y)(A) = X(A) \times Y(A)
\]
for all $A \in \scat{A}$.  

\begin{notn}
Let $\scat{A}$ and $\cat{S}$ be categories.  For each $A \in \scat{A}$, there
is a functor
\[
\begin{array}{cccc}
\ev_A\from      &\ftrcat{\scat{A}}{\cat{S}}     &
\to             &\cat{S}        \\
                &X                              &
\mapsto         &X(A),
\end{array}%
%
\ntn{ev}
%
\]
called \demph{evaluation}%
%
\index{evaluation}
%
at $A$.  We will be working with diagrams in $\ftrcat{\scat{A}}{\cat{S}}$,
and given such a diagram $D\from \scat{I} \to \ftrcat{\scat{A}}{\cat{S}}$,
we have for each $A \in \scat{A}$ a functor
\[
\begin{array}{cccc}
\ev_A \of D\from&\scat{I}       &\to            &\cat{S}        \\
                &I              &\mapsto        &D(I)(A).        
\end{array}
\]
We write $\ev_A \of D$ as $D(\dashbk)(A)$.%
%
\ntn{DblankA}
%
\end{notn}

\begin{thm}[Limits in functor categories]
\label{thm:pw}
Let $\scat{A}$ and $\scat{I}$ be small categories and $\cat{S}$ a locally
small category.  Let $D\from \scat{I} \to \ftrcat{\scat{A}}{\cat{S}}$ be a
diagram, and suppose that for each $A \in \scat{A}$, the diagram
$D(\dashbk)(A)\from \scat{I} \to \cat{S}$ has a limit.  Then there is a
cone on $D$ whose image under $\ev_A$ is a limit cone on $D(\dashbk)(A)$
for each $A \in \scat{A}$.  Moreover, any such cone on $D$ is a limit cone.
\end{thm}

Theorem~\ref{thm:pw} is often expressed as a slogan:
% 
\begin{slogan}
Limits in a functor category are computed pointwise.%
%
\index{limit!computed pointwise}%
\index{pointwise}
%
\end{slogan}
% 
The `points' in the word `pointwise' are the objects of $\scat{A}$.  The
slogan means, for example, that given two functors $X, Y \in
\ftrcat{\scat{A}}{\cat{S}}$, their product can be computed by first taking
the product $X(A) \times Y(A)$ in $\cat{S}$ for each `point' $A$, then
assembling them to form a functor $X \times Y$.

Of course, Theorem~\ref{thm:pw} has a dual, stating that colimits in a
functor category are also computed pointwise.

\begin{pfof}{Theorem~\ref{thm:pw}}
Take for each $A \in \scat{A}$ a limit cone 
% 
\begin{equation}
\label{eq:pw-cone}
\Bigl( L(A) \toby{p_{I, A}} D(I)(A) \Bigr)_{I \in \scat{I}}
\end{equation}
% 
on the diagram $D(\dashbk)(A)\from \scat{I} \to \cat{S}$.  We prove two
statements:
% 
\begin{enumerate}[(b)]
\item 
\label{item:pw-lift} 
there is exactly one way of extending $L$ to a functor on $\scat{A}$ with
the property that $\Bigl(L \toby{p_I} D(I)\Bigr)_{I \in \scat{I}}$ is a
cone on $D$;  

\item   
\label{item:pw-lim}
this cone $\Bigl(L \toby{p_I} D(I)\Bigr)_{I \in \scat{I}}$ is a limit cone.
\end{enumerate}
% 
The theorem will follow immediately.

For~\bref{item:pw-lift}, take a map $f\from A \to A'$ in $\scat{A}$.
Lemma~\ref{lemma:lim-functorial}\bref{lemma:lim-functorial:induced} applied
to the natural transformation
\[
\xymatrix@C+6em{
\scat{I} \rtwocell<5>^{D(\dashbk)(A)}_{D(\dashbk)(A')}%
{\hspace{2.5em}D(\dashbk)(f)}      &
\cat{S}
}
\]
implies that there is a unique map $L(f)\from L(A) \to L(A')$ such that for
all $I \in \scat{I}$, the square
% 
\begin{equation}
\label{eq:param-lim}
\begin{array}{c}
\xymatrix{
L(A) \ar[r]^-{p_{I, A}} \ar[d]_{L(f)}   &
D(I)(A) \ar[d]^{D(I)(f)}        \\
L(A') \ar[r]_-{p_{I, A'}}       &
D(I)(A')
}
\end{array}
\end{equation}
% 
commutes.  (This is our \emph{definition} of $L(f)$.)  We have now defined $L$
on objects and maps of $\scat{A}$.  It is easy to check that $L$ preserves
composition and identities, and is therefore a functor $L\from \scat{A}
\to \cat{S}$.  Moreover, the commutativity of diagram~\eqref{eq:param-lim}
says exactly that for each $I \in \scat{I}$, the family $\Bigl(L(A)
\toby{p_{I, A}} D(I)(A)\Bigr)_{A \in \scat{A}}$ is a natural
transformation
\[
\xymatrix@C+1em{
\scat{A} \rtwocell^{L}_{D(I)}{\hspace{.3em}p_I} &\cat{S}.
}
\]
So we have a family $\Bigl(L \toby{p_I} D(I)\Bigr)_{I \in \scat{I}}$ of
maps in $\ftrcat{\scat{A}}{\cat{S}}$, and from the fact
that~\eqref{eq:pw-cone} is a cone on $D(\dashbk)(A)$ for each $A \in
\scat{A}$, it follows immediately that $\Bigl(L \toby{p_I} D(I)\Bigr)_{I
  \in \scat{I}}$ is a cone on $D$.

For~\bref{item:pw-lim}, let $X \in \ftrcat{\scat{A}}{\cat{S}}$ and let
$\Bigl(X \toby{q_I} D(I)\Bigr)_{I \in \scat{I}}$ be a cone on $D$ in
$\ftrcat{\scat{A}}{\cat{S}}$.  For each $A \in \scat{A}$, we have a cone
\[
\Bigl(X(A) \toby{q_{I, A}} D(I)(A)\Bigr)_{I \in \scat{I}}
\]
on
$D(\dashbk)(A)$ in $\cat{S}$, so there is a unique map $\bar{q}_A\from X(A)
\to L(A)$ such that $p_{I, A} \of \bar{q}_A = q_{I, A}$ for all $I \in
\scat{I}$.  It only remains to prove that $\bar{q}_A$ is natural in $A$, and
that follows from
Lemma~\ref{lemma:lim-functorial}\bref{lemma:lim-functorial:commutes}.  
\end{pfof}

Theorem~\ref{thm:pw} has many important consequences.  We begin by recording
a cruder form of the theorem (and its dual), which we will use repeatedly.

\begin{cor}     
\label{cor:pw-main}
Let $\scat{I}$ and $\scat{A}$ be small categories, and $\cat{S}$ a locally
small category.  If $\cat{S}$ has all limits (respectively, colimits) of
shape $\scat{I}$ then so does $\ftrcat{\scat{A}}{\cat{S}}$, and for each $A
\in \scat{A}$, the evaluation functor $\ev_A\from
\ftrcat{\scat{A}}{\cat{S}} \to \cat{S}$ preserves them.
\qed
\end{cor}

\begin{warning}
If $\cat{S}$ does \emph{not} have all limits of shape $\scat{I}$ then
$\ftrcat{\scat{A}}{\cat{S}}$ may contain limits of shape $\scat{I}$
that are not%
%
\index{limit!non-pointwise}
%
computed pointwise, that is, are not preserved by all the evaluation
functors.  Examples can be constructed, as in Section~3.3 of \citeKel.
\end{warning}

Theorem~\ref{thm:pw} will also help us to prove that limits commute with
limits, in the following sense.  Take categories $\scat{I}$, $\scat{J}$ and
$\cat{S}$.  There are isomorphisms of categories
\[
\ftrcat{\scat{I}}{\ftrcat{\scat{J}}{\cat{S}}}
\iso 
\ftrcat{\scat{I} \times \scat{J}}{\cat{S}}
\iso
\ftrcat{\scat{J}}{\ftrcat{\scat{I}}{\cat{S}}}.
\]
(See Remark~\ref{rmks:global-hom}\bref{rmks:global-hom:cc} and
Exercise~\ref{ex:ftr-on-product}.)  Under these isomorphisms, a functor
$D\from \scat{I} \times \scat{J} \to \cat{S}$ corresponds to the functors
\[
\begin{array}[t]{cccc}
D^\bl\from  &\scat{I}       &\to    &\ftrcat{\scat{J}}{\cat{S}}    \\
        &I              &\mapsto&D(I, \dashbk)
\end{array}
% 
\qquad \text{and} \qquad
% 
\begin{array}[t]{cccc}
D_\bl\from  &\scat{J}       &\to    &\ftrcat{\scat{I}}{\cat{S}}    \\
        &J              &\mapsto&D(\dashbk, J).
\end{array}%
%
\ntn{D-upper-lower}
%
\]
Supposing that $\cat{S}$ has all limits, so do the various functor
categories, by Corollary~\ref{cor:pw-main}.  In particular, there is an
object $\lt{\scat{I}} D^\bl$ of $\ftrcat{\scat{J}}{\cat{S}}$.  This is
itself a diagram in $\cat{S}$, so we obtain in turn an object
$\lt{\scat{J}} \lt{\scat{I}} D^\bl$ of $\cat{S}$.  Alternatively, we can
take limits in the other order, producing an object $\lt{\scat{I}}
\lt{\scat{J}} D_\bl$ of $\cat{S}$.  And there is a third possibility:
taking the limit of $D$ itself, we obtain another object $\lt{\scat{I}
  \times \scat{J}} D$ of $\cat{S}$.  The next result states that these
three objects are the same.  That is, it makes no difference what order we
take limits in.

\begin{propn}[Limits commute with limits] 
\label{propn:lims-lims}
%
\index{limit!commutativity with limits}
% 
Let $\scat{I}$ and $\scat{J}$ be small categories.  Let $\cat{S}$ be a
locally small category with limits of shape $\scat{I}$ and of shape
$\scat{J}$.  Then for all $D\from \scat{I} \times \scat{J} \to \cat{S}$, we
have 
\[
\lt{\scat{J}} \lt{\scat{I}} D^\bl
\iso
\lt{\scat{I} \times \scat{J}} D
\iso 
\lt{\scat{I}} \lt{\scat{J}} D_\bl,
\]
and all these limits exist.  In particular, $\cat{S}$ has limits of shape
$\scat{I} \times \scat{J}$.  
\end{propn}
% 
This is sometimes half-jokingly called Fubini's%
%
\index{Fubini's theorem}
%
theorem, as it is something like changing the order of integration in a
double integral.  The analogy is more appealing with \emph{co}limits,%
%
\index{colimit!integration@and integration}
%
since, like integrals, colimits can be thought of as a context-sensitive
version of sums.

\begin{pf}
By symmetry, it is enough to prove the first isomorphism.  Since $\cat{S}$
has limits of shape $\scat{I}$, so does $\ftrcat{\scat{J}}{\cat{S}}$ (by
Corollary~\ref{cor:pw-main}).  So $\lt{\scat{I}} D^\bl$ exists; it is an
object of $\ftrcat{\scat{J}}{\cat{S}}$.  Since $\cat{S}$ has limits of
shape $\scat{J}$, $\lt{\scat{J}} \lt{\scat{I}} D^\bl$ exists; it is an
object of $\cat{S}$.  Then for $S \in \cat{S}$,
% 
\begin{align*}
\cat{S}\biggl( S, \lt{\scat{J}} \lt{\scat{I}} D^\bl \biggr)     &
\iso    
\ftrcat{\scat{J}}{\cat{S}}
\biggl( \Delta S, \lt{\scat{I}} D^\bl \biggr)                   \\
        &
\iso   
\ftrcat{\scat{I}}{\ftrcat{\scat{J}}{\cat{S}}} 
(\Delta (\Delta S), D^\bl)                                      \\
        &
\iso   
\ftrcat{\scat{I} \times \scat{J}}{\cat{S}} (\Delta S, D)
\end{align*}
% 
naturally in $S$.  The first two steps each follow from
Proposition~\ref{propn:lim-rep}.  The third uses the isomorphism
$\ftrcat{\scat{I}}{\ftrcat{\scat{J}}{\cat{S}}} \iso \ftrcat{\scat{I} \times
\scat{J}}{\cat{S}}$, under which $\Delta(\Delta S)$ corresponds to $\Delta S$
and $D^\bl$ corresponds to $D$.

Hence $\lt{\scat{J}} \lt{\scat{I}} D^\bl$ is
a representing object for the functor $\ftrcat{\scat{I} \times
\scat{J}}{\cat{S}}(\Delta\dashbk, D)$.  By Proposition~\ref{propn:lim-rep}
again, this says that $\lt{\scat{I} \times \scat{J}} D$ exists and is
isomorphic to $\lt{\scat{J}} \lt{\scat{I}} D^\bl$.
\end{pf}

\begin{example}        
\label{eg:lims-lims}
When $\scat{I} = \scat{J} = \fbox{$\bullet \hspace*{2.5em} \bullet$}\ $,
Proposition~\ref{propn:lims-lims} says that binary products commute with
binary products: if $\cat{S}$ has binary products and $S_{11}, S_{12},
S_{21}, S_{22} \in \cat{S}$ then the 4-fold product $\prod_{i, j \in \{1,
  2\}} S_{ij}$ exists and satisfies
\[
(S_{11} \times S_{21}) \times (S_{12} \times S_{22})
\iso
\prod_{i, j \in \{1, 2\}} S_{ij}
\iso
(S_{11} \times S_{12}) \times (S_{21} \times S_{22}).
\]
More generally, it makes no difference what order we write products in or
where we put the brackets: there are canonical isomorphisms
% 
\begin{align*}
S \times T              &
\iso   
T \times S,             
%
\index{product!commutativity of}
%
\\
(S \times T) \times U   &
\iso   
S \times (T \times U)
%
\index{product!associativity of}%
\index{associativity}
% 
\end{align*}
% 
in any category with binary products.   If there is also a terminal object
$1$, there are further canonical isomorphisms
\[
S \times 1 
\iso 
S 
\iso 
1 \times S.
\]
\end{example}

\begin{warning}
The dual of Proposition~\ref{propn:lims-lims} states that colimits commute
with colimits.  For instance,
\[
(S_{11} + S_{21}) + (S_{12} + S_{22})
\iso
(S_{11} + S_{12}) + (S_{21} + S_{22})
\]
in any category $\cat{S}$ with binary sums.  But limits do \emph{not}%
%
\index{limit!commutativity with colimits@non-commutativity with colimits}
%
in general commute with colimits.  For instance, in general,
\[
(S_{11} + S_{21}) \times (S_{12} + S_{22})
\not\iso
(S_{11} \times S_{12}) + (S_{21} \times S_{22}).
\]
A counterexample is given by taking $\cat{S} = \Set$ and each $S_{ij}$ to be a
one-element set.  Then the left-hand side has $(1 + 1) \times (1 + 1) = 4$
elements, whereas the right-hand side has $(1 \times 1) + (1 \times 1) = 2$
elements.
\end{warning}

Here are two further consequences of Theorem~\ref{thm:pw}.

\begin{cor}
%
\index{presheaf!category of presheaves!limits in}
% 
Let $\scat{A}$ be a small category.  Then $\pshf{\scat{A}}$ has all limits and
colimits, and for each $A \in \scat{A}$, the evaluation functor $\ev_A\from
\pshf{\scat{A}} \to \Set$ preserves them.
\end{cor}

\begin{pf}
Since $\Set$ has all limits and colimits, this is immediate from
Corollary~\ref{cor:pw-main}.
\end{pf}

\begin{cor}     
\label{cor:yoneda-cts}
%
\index{Yoneda embedding!preserves limits}%
\index{functor!representable!limit of representables|(}
%
The Yoneda embedding $\h_\bl\from \scat{A} \to \pshf{\scat{A}}$ preserves
limits, for any small category $\scat{A}$.
\end{cor}

\begin{pf}
Let $D\from \scat{I} \to \scat{A}$ be a diagram in $\scat{A}$, and let
$\biggl( \lt{\scat{I}} D \toby{p_I} D(I) \biggr)_{I \in \scat{I}}$ be a
limit cone.  For each $A \in \scat{A}$, the composite functor
\[
\scat{A} \toby{\h_\bl} \pshf{\scat{A}} \toby{\ev_A} \Set
\]
is $\h^A$, which preserves limits (Proposition~\ref{propn:reps-cts}).  So for
each $A \in \scat{A}$, 
\[
\biggl( 
\ev_A \h_\bl \biggl( \lt{\scat{I}} D \biggr) \,
\xymatrix@1@C+2em{\mbox{} \ar[r]^{\ev_A \h_\bl (p_I)}&\mbox{}}
\ev_A \h_\bl (D(I)) 
\biggr)_{I \in \scat{I}}
\]
is a limit cone.  But then, by the `moreover' part of Theorem~\ref{thm:pw}
applied to the diagram $\h_\bl \of D$ in $\pshf{\scat{A}}$, the cone
\[
\biggl( 
\h_\bl \biggl(\lt{\scat{I}} D\biggr)
\toby{\h_\bl (p_I)} 
\h_\bl (D(I)) 
\biggr)_{I \in \scat{I}}
\]
is also a limit, as required.
\end{pf}

\begin{example}
Let $\scat{A}$ be a category with binary products.
Corollary~\ref{cor:yoneda-cts} implies that for all $X, Y \in \scat{A}$,
% 
\begin{equation}        
\label{eq:yon-pres-prods}
\h_{X \times Y} \iso \h_X \times \h_Y
\end{equation}
% 
in $\ftrcat{\scat{A}^\op}{\Set}$.  When evaluated at a particular object
$A$, this says that
\[
\scat{A}(A, X \times Y) \iso \scat{A}(A, X) \times \scat{A}(A, Y)
%
\index{product!map into}
%
\]
(using the fact that products are computed pointwise).  This is the
isomorphism~\eqref{eq:prod-rep} that we met at the beginning of this
section.

Suppose that we view $\scat{A}$ as a subcategory of $\pshf{\scat{A}}$,
identifying $A \in \scat{A}$ with the representable $\h_A \in
\pshf{\scat{A}}$ as in Figure~\ref{fig:yoneda-embedding}.  Then the
isomorphism~\eqref{eq:yon-pres-prods} means that given two objects of
$\scat{A}$ whose product we want to form, it makes no difference whether we
think of the product as taking place in $\scat{A}$ or $\pshf{\scat{A}}$.
Similarly, if $\scat{A}$ has all limits, taking limits does not help us to
escape from $\scat{A}$ into the rest of $\pshf{\scat{A}}$: any limit of
representable presheaves is again representable.%
%
\index{functor!representable!limit of representables|)}
%
\end{example}

\begin{warning} 
\label{warning:yon-colims}
%
\index{Yoneda embedding!preserves colimits@does not preserve colimits}
% 
The Yoneda embedding does \emph{not} preserve colimits.  For example, if
$\scat{A}$ has an initial object $0$ then $\h_0$ is not initial, since
$\h_0(0) = \scat{A}(0, 0)$ is a one-element set, whereas the initial object
of $\pshf{\scat{A}}$ is the presheaf with constant value $\emptyset$.  We
investigate colimits of representables next.
%
\index{functor!category!limits in|)}%
\index{limit!functor category@in functor category|)}%
%
\end{warning}


\minihead{Every presheaf is a colimit of representables}
%
\index{presheaf!colimit of representables@as colimit of representables|(}%
\index{functor!representable!colimit of representables|(}%
%

We now know that the Yoneda embedding preserves limits but not colimits.
In fact, the situation for colimits is at the opposite extreme from the
situation for limits: by taking colimits of representable presheaves, we
can obtain any presheaf we like!  This is the last main result of this
section.

Every positive integer can be expressed as a product of primes%
%
\index{prime numbers}
%
in an essentially unique way.  Somewhat similarly, every presheaf can be
expressed as a colimit of representables in a canonical (though not unique)
way.  The representables are the building blocks of presheaves.

For a different analogy, recall that any complex function holomorphic%
%
\index{holomorphic function}
%
in a neighbourhood of $0$ has a power%
%
\index{power!series}
%
series expansion, such as
\[
e^z 
=
1 + z + \frac{z^2}{2!} + \frac{z^3}{3!} + \cdots.
\]
In this sense, the power functions $z \mapsto z^n$ are the building blocks
of holomorphic functions.  We could even take the analogy further:
$\blank^n$ is like a representable $\Hom(n, \dashbk)$, and in the
categorical context, quotients and sums are types of colimit.

Before we state and prove the theorem, let us look at an easy special case.

\begin{example} 
\label{eg:discrete-dense}
Let $\scat{A}$ be the discrete category with two objects, $K$ and $L$.  A
presheaf $X$ on $\scat{A}$ is just a pair $(X(K), X(L))$ of sets, and
$\pshf{\scat{A}} \iso \Set \times \Set$.  There are two representables,
$\h_K$ and $\h_L$, given by
\[
\h_A(B) 
= 
\scat{A}(B, A)
\iso
\begin{cases}
1               &\text{if } A = B,    \\
\emptyset       &\text{if } A \neq B  
\end{cases}
\]
($A, B \in \{ K, L \}$).  Identifying $\pshf{\scat{A}}$ with $\Set \times
\Set$, we have $\h_K \iso (1, \emptyset)$ and $\h_L \iso (\emptyset, 1)$.
Every object of $\Set \times \Set$ is a sum of copies of $(1, \emptyset)$
and $(\emptyset, 1)$.  Suppose, for instance, that $X(K)$ has three elements
and $X(L)$ has two elements.  Then
\[
(X(K), X(L))
\iso
(1, \emptyset) + (1, \emptyset) + (1, \emptyset) + 
(\emptyset, 1) + (\emptyset, 1)
\]
in $\Set \times \Set$.  Equivalently,
\[
X 
\iso 
\h_K + \h_K + \h_K + \h_L + \h_L 
\]
in $\pshf{\scat{A}}$, exhibiting $X$ as a sum of representables.  
\end{example}

In this example, $X$ is expressed as a sum of five representables, that is,
a sum indexed by the set $X(K) + X(L)$ of `elements' of $X$.  A sum is a
colimit over a discrete category.  In the general case, a presheaf $X$ on a
category $\scat{A}$ is expressed as a colimit over a category whose objects
can be thought of as the `elements' of $X$.  This is made precise by the
following definition.

\begin{defn}    
\label{defn:cat-elts}
Let $\scat{A}$ be a category and $X$ a presheaf on $\scat{A}$.  The
\demph{category%
%
\index{element!category of elements}%
\index{category!elements@of elements}
%
of elements} $\elt{X}$%
%
\ntn{cat-elts}
%
of $X$ is the category in which:
% 
\begin{itemize}
\item 
objects are pairs $(A, x)$ with $A \in \scat{A}$ and $x \in X(A)$;

\item 
maps $(A', x') \to (A, x)$ are maps $f\from A' \to A$ in $\scat{A}$ such
that $(Xf)(x) = x'$.
\end{itemize}
\end{defn}
% 
There is a projection functor $P\from \elt{X} \to \scat{A}$ defined by $P(A, x)
= A$ and $P(f) = f$.

The following `density theorem' states that every presheaf is a colimit of
representables in a canonical way.  It is secretly dual to the Yoneda
lemma.  This becomes apparent if one expresses both in suitably lofty
categorical language (that of ends, or that of bimodules); but that is
beyond the scope of this book.

\begin{thm}[Density]
\label{thm:density}
%
\index{density}
% 
Let $\scat{A}$ be a small category and $X$ a presheaf on $\scat{A}$.  Then $X$
is the colimit of the diagram
\[
\elt{X} \toby{P} \scat{A} \toby{\h_\bl} \pshf{\scat{A}}
\] 
in $\pshf{\scat{A}}$; that is, $X \iso \colt{\elt{X}} (\h_\bl \of P)$.
\end{thm}

\begin{pf}
First note that since $\scat{A}$ is small, so too is $\elt{X}$.  Hence $\h_\bl
\of P$ really is a diagram in our customary sense
(Definition~\ref{defn:diagram}). 

Now let $Y \in \pshf{\scat{A}}$.  A cocone on $\h_\bl \of P$ with vertex $Y$ is
a family
\[
\Bigl(\h_A \toby{\alpha_{A, x}} Y\Bigr)_{A \in \scat{A}, x \in X(A)}
\]
of natural transformations with the property that for all maps $A'
\toby{f} A$ in $\scat{A}$ and all $x \in X(A)$, the diagram
\[
\xymatrix@R=1ex{
\h_{A'} \ar[rd]^-{\alpha_{A', (Xf)(x)}} \ar[dd]_{\h_f}   &       \\
                                                        &Y      \\
\h_A \ar[ru]_-{\alpha_{A, x}}
}
\]
commutes.  

Equivalently (by the Yoneda lemma), a cocone on $\h_\bl \of P$ with vertex $Y$
is a family 
\[
(y_{A, x})_{A \in \scat{A}, x \in X(A)},
\]
with $y_{A, x} \in Y(A)$, such that for all maps $A' \toby{f} A$ in
$\scat{A}$ and all $x \in X(A)$,
\[
(Yf)(y_{A, x}) = y_{A', (Xf)(x)}.
\]
To see this, note that if $\alpha_{A, x} \in \pshf{\scat{A}}(\h_A, Y)$
corresponds to $y_{A, x} \in Y(A)$, then $\alpha_{A, x} \of \h_f \in
\pshf{\scat{A}}(\h_{A'}, Y)$ corresponds to $(Yf)(y_{A, x}) \in Y(A')$.

Equivalently (writing $y_{A, x}$ as $\bar{\alpha}_A(x)$), it is a family
\[
\Bigl(X(A) \toby{\bar{\alpha}_A} Y(A)\Bigr)_{A \in \scat{A}}
\] 
of functions with the property that for all maps $A' \toby{f} A$
in $\scat{A}$ and all $x \in X(A)$,
\[
(Yf)\bigl(\bar{\alpha}_A(x)\bigr) 
=
\bar{\alpha}_{A'}\bigl((Xf)(x)\bigr).
\]
But this is simply a natural transformation $\bar{\alpha}\from X \to Y$.
So we have, for each $Y \in \pshf{\scat{A}}$, a canonical bijection
\[
\ftrcat{\elt{X}}{\pshf{\scat{A}}} (\h_\bl \of P, \, \Delta Y)
\iso 
\pshf{\scat{A}} (X, Y).
\]
Hence $X$ is the colimit of $\h_\bl \of P$.
\end{pf}

\begin{example}
In Example~\ref{eg:discrete-dense}, we expressed a particular presheaf $X$
as a sum of representables.  Let us check that the way we did this is a
special case of the general construction in the density theorem.

Since $\scat{A}$ is discrete, the category of elements $\elt{X}$ is also
discrete; it is the set $X(K) + X(L)$ with five elements.  The projection
$P\from \elt{X} \to \scat{A}$ sends three of the elements to $K$ and the
other two to $L$, so the diagram $\h_\bl \of P\from \elt{X} \to
\pshf{\scat{A}}$ sends three of the elements to $\h_K$ and two to $\h_L$.
The colimit of $\h_\bl \of P$ is the sum of these five representables,
which is $X$, just as in Example~\ref{eg:discrete-dense}.
\end{example}

\begin{remarks}
\begin{enumerate}[(b)]
\item 
The term `category of elements'%
%
\index{element!category of elements}%
\index{category!elements@of elements}
%
is compatible with the generalized%
%
\index{element!generalized}
%
element terminology introduced in Definition~\ref{defn:gen-elt}.  A
generalized element of an object $X$ is just a map into $X$, say $Z \to X$;
but, as explained after that definition, we often focus on certain special
shapes $Z$.  Now suppose that we are working in a presheaf category
$\pshf{\scat{A}}$.  Among all presheaves, the representables have a special
status, so we might be especially interested in generalized elements of
representable shape.  The Yoneda lemma implies that for a presheaf $X$, the
generalized elements of $X$ of representable shape correspond to pairs $(A,
x)$ with $A \in \scat{A}$ and $x \in X(A)$.  In other words, they are the
objects of the category of elements.

\item 
In topology, a subspace $A$ of a space $B$ is called dense%
%
\index{density}
%
if every point in $B$ can be obtained as a limit of points in $A$.  This
provides some explanation for the name of Theorem~\ref{thm:density}: the
category $\scat{A}$ is `dense' in $\pshf{\scat{A}}$ because every object of
$\pshf{\scat{A}}$ can be obtained as a colimit of objects of $\scat{A}$.
%
\index{presheaf!colimit of representables@as colimit of representables|)}%
\index{functor!representable!colimit of representables|)}%
%
\end{enumerate}
\end{remarks}


\exs


\begin{question}
Fix a small category $\scat{A}$.
% 
\begin{enumerate}[(b)]
\item
Let $\cat{S}$ be a locally small category with pullbacks.  Show that a
natural transformation
\[
\xymatrix{
\scat{A} \rtwocell^X_Y{\alpha} &\cat{S}
}
\]
is monic (as a map in $\ftrcat{\scat{A}}{\cat{S}}$) if and only if $\alpha_A$
is monic for all $A \in \scat{A}$.  (Hint: use Lemma~\ref{lemma:monic-pb}.)

\item   
\label{part:monic-epic-transf}
Describe explicitly the monics and epics in $\pshf{\scat{A}}$.%
%
\index{presheaf!category of presheaves!monics and epics in}%
%

\item 
Can you do part~\bref{part:monic-epic-transf} without relying on the fact
that limits and colimits of presheaves are computed pointwise?
\end{enumerate}
\end{question}


\begin{question}
\begin{enumerate}[(b)]
\item 
Prove that representables have the following connectedness%
%
\index{connectedness}
%
property: given a locally small category $\cat{A}$ and $A \in \cat{A}$, if
$X, Y \in \pshf{\cat{A}}$ with $\h_A \iso X + Y$, then either $X$ or $Y$ is
the constant functor $\emptyset$.

\item
Deduce that the sum%
%
\index{functor!representable!sum of representables}
%
of two representables is never representable.
\end{enumerate}
\end{question}


\begin{question}
Show how a category of elements can be described as a comma category.
\end{question}


\begin{question}
Let $X$ be a presheaf on a locally small category.  Show that $X$ is
representable if and only if its category of elements has a terminal
object.

(Since a terminal object is a limit of the empty diagram, this implies that
the concept of representability can be derived from the concept of limit.
Since a terminal object of a category $\cat{E}$ is also a right adjoint to
the unique functor $\cat{E} \to \One$, the concept of representability can
also be derived from the concept of adjoint.)
\end{question}


\begin{question}
Prove that every slice%
%
\index{presheaf!category of presheaves!slice of}%
\index{slice category!presheaf category@of presheaf category}
%
of a presheaf category is again a presheaf category.
That is, given a small category $\scat{A}$ and a presheaf $X$ on $\scat{A}$,
prove that $\pshf{\scat{A}}/X$ is equivalent to $\pshf{\scat{B}}$ for some
small category $\scat{B}$.
\end{question}


\begin{question}
\label{ex:kan}
Let $F\from \scat{A} \to \scat{B}$ be a functor between small categories.
For each object $B \in \scat{B}$, there is a comma category $\comma{F}{B}$
(defined dually to the comma category in Example~\ref{eg:comma-obj-ftr}),
and there is a projection functor $P_B \from \comma{F}{B} \to \scat{A}$.
% 
\begin{enumerate}[(b)]
\item 
Let $X \from \scat{A} \to \cat{S}$ be a functor from $\scat{A}$ to a
category $\cat{S}$ with small colimits.  For each $B \in \scat{B}$, let
$(\Lan_F X)(B)$ be the colimit of the diagram
\[
\comma{F}{B} \toby{P_B} \scat{A} \toby{X} \cat{S}.
\]
Show that this defines a functor $\Lan_F X\from \scat{B} \to \cat{S}$, and
that for functors $Y \from \scat{B} \to \cat{S}$, there is a canonical
bijection between natural transformations $\Lan_F X \to Y$ and natural
transformations $X \to Y \of F$.

\item
\label{part:left-kan-exists}
Deduce that for any category $\cat{S}$ with small colimits, the functor
\[
\dashbk \of F \from 
\ftrcat{\scat{B}}{\cat{S}} \to
\ftrcat{\scat{A}}{\cat{S}}
\]
has a left adjoint.  (This left adjoint, $\Lan_F$, is called
\demph{left Kan%
%
\index{Kan extension}
%
extension} along $F$.)

\item
Part~\bref{part:left-kan-exists} and its dual imply that when $\cat{S}$
has small limits and colimits, the functor $\dashbk \of F$ has both left
and right adjoints.  Revisit Exercise~\ref{ex:G-set-adjns} with this
in mind, taking $F$ to be either the unique functor $\One \to G$%
%
\index{group!action of}%
\index{G-set@$G$-set}%
\index{representation!group or monoid@of group or monoid!linear}
%
or the unique functor $G \to \One$.
\end{enumerate}
\end{question}



\section{Interactions between adjoint functors and limits}
\label{sec:adj-lim}


We saw in Proposition~\ref{propn:ladj-rep} that any set-valued functor with
a left adjoint is representable, and in Proposition~\ref{propn:reps-cts}
that any representable preserves limits.  Hence, any set-valued functor
with a left adjoint preserves limits.  In fact, this conclusion holds not
only for set-valued functors, but in complete generality.

\begin{thm}     
\label{thm:adjts-cts}
%
\index{limit!preservation of!adjoint@by adjoint}%
\index{adjunction!limits preserved in}
%
Let $\hadjnli{\cat{A}}{\cat{B}}{F}{G}$ be an adjunction.  Then $F$ preserves
colimits and $G$ preserves limits.
\end{thm}

\begin{pf}
By duality, it is enough to prove that $G$ preserves limits.  Let $D\from
\scat{I} \to \cat{B}$ be a diagram for which a limit exists.  Then
% 
\begin{align}
\cat{A} \biggl(A, G\biggl(\lt{\scat{I}} D \biggr)\biggr) &
\iso   
\cat{B}\biggl(F(A), \lt{\scat{I}} D\biggr)        
\label{eq:adj-lim-1}    \\
        &
\iso   
\lt{\scat{I}} \cat{B}(F(A), D)        
\label{eq:adj-lim-2}    \\
        &
\iso   
\lt{\scat{I}} \cat{A}(A, G \of D)      
\label{eq:adj-lim-3}    \\
        &
\iso   
\Cone(A, G \of D)
\label{eq:adj-lim-4}
\end{align}
% 
naturally in $A \in \cat{A}$.  Here, the isomorphism~\eqref{eq:adj-lim-1}
is by adjointness, \eqref{eq:adj-lim-2} is because representables preserve
limits, \eqref{eq:adj-lim-3} is by adjointness again, and
\eqref{eq:adj-lim-4} is by Lemma~\ref{lemma:cone-rep}.  So
$G\biggl(\lt{\scat{I}} D\biggr)$ represents $\Cone(\dashbk, G \of D)$; that
is, it is a limit of $G \of D$.
\end{pf}


\begin{example}
Forgetful functors from categories of algebras to $\Set$ have left
adjoints, but hardly ever right adjoints.  Correspondingly, they preserve%
%
\index{functor!forgetful!preserves limits}
%
all limits, but rarely all colimits.
\end{example}

\begin{example}
\label{eg:arith-set}
Every set $B$ gives rise to an adjunction $(\dashbk \times B) \ladj
(\dashbk)^B$ of functors from $\Set$ to $\Set$ (Example~\ref{eg:adjn:cc}).  So
$\dashbk \times B$ preserves colimits  and
$(\dashbk)^B$ preserves limits.  In particular, $\dashbk \times B$ preserves
finite sums and $(\dashbk)^B$ preserves finite products, giving isomorphisms
% 
\begin{align}
\label{eq:Set-dist}
0 \times B &\iso 0,
&
(A_1 + A_2) \times B &\iso (A_1 \times B) + (A_2 \times B),      \\
\label{eq:Set-codist}
1^B &\iso 1,
&
(A_1 \times A_2)^B &\iso A_1^B \times A_2^B.
\end{align}
% 
These are the analogues of standard rules of arithmetic.%
%
\index{arithmetic}
%
(See also Example \ref{eg:lims-lims} and the `Digression on arithmetic' on
page~\pageref{p:arith}.)  Indeed, if we know~\eqref{eq:Set-dist}
and~\eqref{eq:Set-codist} for just finite sets then by taking cardinality
on both sides, we obtain exactly these standard rules.  The natural%
%
\index{natural numbers}
%
numbers are, after all, just the isomorphism classes of finite sets.
\end{example}

\begin{example}
Given a category $\cat{A}$ with all limits of shape $\scat{I}$, we have the
adjunction
$\hadjnli{\cat{A}}{\ftrcat{\scat{I}}{\cat{A}}}{\Delta}{\lt{\scat{I}}}$
(Proposition~\ref{propn:lim-const-adjn}).  Hence $\lt{\scat{I}}$ preserves
limits, or equivalently, limits of shape $\scat{I}$ commute with (all)
limits.  This gives another proof that limits commute%
%
\index{limit!commutativity with limits}
%
with limits (Proposition~\ref{propn:lims-lims}), at least in the case where
the category has all limits of one of the shapes concerned.
\end{example}

\begin{example} 
\label{eg:no-free-field}
Theorem~\ref{thm:adjts-cts} is often used to prove that a functor does
\emph{not}%
%
\index{adjunction!nonexistence of adjoints}
%
have an adjoint.  For instance, it was claimed in
Example~\ref{egs:adjns-alg}\bref{egs:adjns-alg:fields} that the forgetful
functor $U\from \Field \to \Set$%
%
\index{field}
%
does not have a left adjoint.  We can now prove this.  If $U$ had a left
adjoint $F\from \Set \to \Field$, then $F$ would preserve colimits, and in
particular, initial objects.  Hence $F(\emptyset)$ would be an initial
object of $\Field$.  But $\Field$ has no initial object, since there are no
maps between fields of different characteristic.  Further examples of
nonexistence of adjoints can be found in Exercise~\ref{ex:no-adjt}.
\end{example}


\minihead{Adjoint functor theorems}
%
\index{adjoint functor theorems|(}
%

Every functor with a left adjoint preserves limits, but limit-preservation
alone does not guarantee the existence of a left adjoint.  For example, let
$\cat{B}$ be any category.  The unique functor $\cat{B} \to \One$ always
preserves limits, but by Example~\ref{eg:init-term}, it only has a left
adjoint if $\cat{B}$ has an initial object.

On the other hand, if we have a limit-preserving functor $G\from \cat{B}
\to \cat{A}$ \emph{and $\cat{B}$ has all limits}, then there is an
excellent chance that $G$ has a left adjoint.  It is still not always true,
but counterexamples are harder to find.  For instance (taking $\cat{A} =
\One$ again), can you find a category $\cat{B}$ that has all limits but no
initial object?

The condition of having all limits is so important that it has its own word:
% 
\begin{defn}    
\label{defn:complete}
A category is \demph{complete}%
%
\index{category!complete}%
\index{complete}
%
(or properly, \demph{small complete}) if it has all limits.
\end{defn}

There are various results called adjoint functor theorems, all of the
following form:
% 
\begin{displaytext} \it
Let $\cat{A}$ be a category, $\cat{B}$ a complete category, and $G\from \cat{B}
\to \cat{A}$ a functor.  Suppose that $\cat{A}$, $\cat{B}$ and $G$ satisfy
certain further conditions.  Then 
\[
G \text{ has a left adjoint}
\iff
G \text{ preserves limits}.
\]
\end{displaytext}
% 
The forwards implication is immediate from Theorem~\ref{thm:adjts-cts}.
It is the backwards implication that concerns us here.

Typically, the `further conditions' involve the distinction between small
and large collections.  But there is a special%
%
\index{ordered set!adjunction between|(}
%
case in which these complications disappear, and I will use it to explain
the main idea behind the proofs of the adjoint functor theorems.  It is the
case where the categories $\cat{A}$ and $\cat{B}$ are ordered sets.

As we saw in Section~\ref{sec:lims-basics}, limits in ordered sets are
meets.  More precisely, if $D\from \scat{I} \to \scat{B}$ is a diagram in
an ordered set $\scat{B}$, then
\[
\lt{\scat{I}} D 
= 
\Meet_{I \in \scat{I}} D(I), 
\]
with one side defined if and only if the other is.  So an ordered set is
complete if and only if every subset has a meet.  Similarly, a map $G\from
\scat{B} \to \scat{A}$ of ordered sets preserves limits if and only if
\[
G\Biggl(\Meet_{i \in I} B_i\Biggr) 
= 
\Meet_{i \in I} G(B_i)
\]
whenever $(B_i)_{i \in I}$ is a family of elements of $\scat{B}$ for which a
meet exists.


We now show that for ordered sets, there is an adjoint functor theorem of the
simplest possible kind: there are no `further conditions' at all.

\begin{propn}[Adjoint functor theorem for ordered sets]
\label{propn:OAFT}
Let $\scat{A}$ be an ordered set, $\scat{B}$ a complete ordered set, and
$G\from \scat{B} \to \scat{A}$ an order-preserving map.  Then
\[
G \text{ has a left adjoint}
\iff
G \text{ preserves meets}.
\]
\end{propn}

\begin{pf}
Suppose that $G$ preserves meets.  By Corollary~\ref{cor:pre-AFT}, it is
enough to show that for each $A \in \scat{A}$, the comma category
$\comma{A}{G}$ has an initial object.  Let $A \in \scat{A}$.  Then
$\comma{A}{G}$ is an ordered set, namely, $\{ B \in \scat{B} \such A \leq
G(B) \}$ with the order inherited from $\scat{B}$.  We have to show that
$\comma{A}{G}$ has a least element.

Since $\scat{B}$ is complete, the meet $\Meet_{B \in \scat{B}\from A \leq
  G(B)} B$ exists in $\scat{B}$.  This is the meet of all the elements of
$\comma{A}{G}$, so it suffices to show that the meet is itself an element
of $\comma{A}{G}$.  And indeed, since $G$ preserves meets, we have
\[
G \Biggl( \Meet_{B \in \scat{B}\from A \leq G(B)} B \Biggr)
=
\Meet_{B \in \scat{B}\from A \leq G(B)} G(B)
\geq
A,
\]
as required.
\end{pf}

In the general setting of Corollary~\ref{cor:pre-AFT}, the initial object
of $\comma{A}{G}$ is the pair $\Bigl(F(A),\, A \toby{\eta_A} GF(A)\Bigr)$,
where $F$ is the left adjoint and $\eta$ is the unit map.  So in
Proposition~\ref{propn:OAFT}, the left adjoint $F$ is given by
% 
\begin{equation}        
\label{eq:ladj-recipe}
F(A) 
= 
\Meet_{B \in \scat{B}\from A \leq G(B)} B.
\end{equation}

\begin{example} 
\label{eg:oaft-least}
%
\index{limit!colimit@vs.\ colimit|(}
%
Consider Proposition~\ref{propn:OAFT} in the case $\scat{A} = \One$.  The
unique functor $G\from \scat{B} \to \One$ automatically preserves meets,
and, as observed above, a left adjoint to $G$ is an initial object of
$\scat{B}$.  So in the case $\scat{A} = \One$, the proposition states that
a complete ordered set has a least element.  This is not quite trivial,
since completeness means the existence of all meets, whereas a least
element is an empty \emph{join}.

By~\eqref{eq:ladj-recipe}, the least%
%
\index{least element!meet@as meet}
%
element of $\scat{B}$ is $\Meet_{B \in \scat{B}} B$.  Thus, a least element
is not only a colimit of the functor $\emptyset \to \scat{B}$; it is also a
limit of the identity functor $\scat{B} \to \scat{B}$.

The synonym `least upper bound' for `join' suggests a theorem: that a poset
with all meets also has all joins.  Indeed, given a poset $\scat{B}$ with
all meets, the join of a subset of $\scat{B}$ is simply the meet of its
upper bounds: quite literally, its least upper bound.%
%
\index{limit!colimit@vs.\ colimit|)}
%
\end{example}

Let us now attempt to extend Proposition~\ref{propn:OAFT} from ordered sets
to categories, starting with a limit-preserving functor $G$ from a complete
category $\cat{B}$ to a category $\cat{A}$.  In the case of ordered sets,
we had for each $A \in \cat{A}$ an inclusion map $P_A \from \comma{A}{G}
\incl \scat{B}$, and we showed that the left adjoint $F$ was given by
% 
\begin{equation}        
\label{eq:ladj-po}
F(A) = \lt{\comma{A}{G}} P_A.  
\end{equation}
% 
In the general case, the analogue of the inclusion functor is the
projection functor
% 
\begin{equation}
\label{eq:pjn-aft}
\begin{array}{cccc}
P_A\from    &\comma{A}{G}                       &\to            &\cat{B}\\
            &\Bigl(B,\, A \toby{f} G(B)\Bigr)   &\mapsto        &B.
\end{array}
\end{equation}
% 
The case of ordered sets suggests that in general,
equation~\eqref{eq:ladj-po} might define a left adjoint $F$ to $G$.  And
indeed, it can be shown that if this limit in $\cat{B}$ exists and is
preserved by $G$, then \eqref{eq:ladj-po} really does give a left adjoint
(Theorem~X.1.2 of \citeCWM).

This might seem to suggest that our adjoint functor theorem generalizes
smoothly from ordered sets to arbitrary categories, with no need for
further conditions.  But it does not, for reasons that are quite subtle.

Those reasons are more easily explained if we relax our terminology
slightly.  When we defined limits, we built in the condition that the shape
category $\scat{I}$ was small.%
%
\index{limit!large|(}%
\index{limit!small|(}
%
However, the definition of limit makes sense for an arbitrary category
$\scat{I}$.  In this discussion, we will need to refer to this more
inclusive notion of limit, so let us temporarily suspend the convention
that the shape categories $\scat{I}$ of limits are always small.

Now, in the template for adjoint functor theorems stated above (after
Definition~\ref{defn:complete}), it was only required that $\cat{B}$ has, and
$G$ preserves, \emph{small} limits.  But if $\cat{B}$ is a large category
then $\comma{A}{G}$ might also be large, since to specify an object or map in
$\comma{A}{G}$, we have to specify (among other things) an object or map in
$\cat{B}$.  So, the limit~\eqref{eq:ladj-po} defining the left adjoint is not
guaranteed to be small.  Hence there is no guarantee that this limit exists in
$\cat{B}$, nor that it is preserved by $G$.  It follows that the functor $F$
`defined' by~\eqref{eq:ladj-po} might not be defined at all, let alone a left
adjoint.

(The reader experiencing difficulty with reasoning about small and large
collections might usefully compare finite and infinite collections.  For
instance, if $\cat{B}$ is a finite category and $\cat{A}$ has finite
hom-sets then $\comma{A}{G}$ is also finite, but otherwise $\comma{A}{G}$
might be infinite.)

Proposition~\ref{propn:OAFT} still stands, since there we were dealing with
ordered \emph{sets}, which as categories are small.  We might hope to
extend it from posets to arbitrary small categories, since the problem just
described affects only large categories.  But this turns out not to be very
fruitful, since in fact, complete posets are the \emph{only}%
%
\index{ordered set!complete small category is}
%
complete small categories (Exercise~\ref{ex:small-complete}).%
%
\index{ordered set!adjunction between|)}
%

Alternatively, we could try to salvage the argument by assuming that
$\cat{B}$ has, and $G$ preserves, \emph{all} (possibly large) limits.  But
again, this is unhelpful: there are almost no such categories $\cat{B}$.

The situation therefore becomes more complicated.  Each of the best-known
adjoint functor theorems imposes further conditions implying that the large
limit $\lt{\comma{A}{G}} P_A$ can be replaced by a small limit in some
clever way.  This allows one to proceed with the argument above.%
%
\index{limit!large|)}%
\index{limit!small|)}
%

The two most famous adjoint functor theorems are the `general' and the
`special'.  Their exact statements and proofs are perhaps less significant
than their consequences.

\begin{defn}
Let $\cat{C}$ be a category.  A \demph{weakly%
%
\index{set!weakly initial}%
\index{weakly initial}
%
initial set} in $\cat{C}$ is a set $\scat{S}$ of objects with the property
that for each $C \in \cat{C}$, there exist an element $S \in \scat{S}$ and
a map $S \to C$.
\end{defn}
% 
Note that $\scat{S}$ must be a set, that is, small.  So, the existence of a
weakly initial set is some kind of size restriction.  Such size
restrictions are comparable to finiteness conditions in algebra.

\begin{thm}[General adjoint functor theorem]     
\label{thm:gaft}
%
\index{adjoint functor theorems!general}%
\index{general adjoint functor theorem (GAFT)}
%
Let $\cat{A}$ be a category, $\cat{B}$ a complete category, and $G\from
\cat{B} \to \cat{A}$ a functor.  Suppose that $\cat{B}$ is locally small and
that for each $A \in \cat{A}$, the category $\comma{A}{G}$ has a weakly
initial set.  Then
\[
G \text{ has a left adjoint}
\iff
G \text{ preserves limits}.
\]
\end{thm}

\begin{pf}
See the appendix.
\end{pf}

\begin{example} 
\label{eg:gaft-free-alg}
The general adjoint functor theorem (GAFT) implies that for any category
$\cat{B}$ of algebras ($\Grp$, $\Vect_k$, \ldots), the forgetful%
%
\index{functor!forgetful!left adjoint to}
%
functor $U\from \cat{B} \to \Set$ has a left adjoint.  Indeed, we saw in
Example~\ref{eg:lims-alg} that $\cat{B}$ has all limits, and in
Example~\ref{eg:gp-creation} that $U$ preserves them.  Also, $\cat{B}$ is
locally small.  To apply GAFT, we now just have to check that for each $A
\in \Set$, the comma category $\comma{A}{U}$ has a weakly initial set.
This requires a little cardinal%
%
\index{cardinality}%
\index{arithmetic!cardinal}
%
arithmetic, omitted here; see Exercise~\ref{ex:free-gp-card}.

So GAFT tells us that, for instance, the free group%
%
\index{group!free}
%
functor exists.  In Examples~\ref{egs:free-functors}\bref{eg:free-group}
and~\ref{egs:adjns-alg}\bref{egs:adjns-alg:gp}, we began to see the
trickiness of explicitly constructing the free group on a generating set
$A$.  One has to define the set of `formal expressions' (such as $x^{-1} y
x^2 z y^{-3}$, with $x, y, z \in A$), then say what it means for two such
expressions to be equivalent (so that $x^{-2} x^5 y$ is equivalent to $x^3
y$), then define $F(A)$ to be the set of all equivalence classes, then
define the group structure, then check the group axioms, then prove that
the resulting group has the universal property required.  But using GAFT,
we can avoid these complications entirely.

The price to be paid is that GAFT does not give us an explicit%
%
\index{explicit description}
%
description of free groups (or left adjoints more generally).  When people
speak of knowing some object `explicitly', they usually mean knowing its
elements.  An element of an object is a map \emph{into} it, and we have no
handle on maps into $F(A)$: since $F$ is a left adjoint, it is maps
\emph{out} of $F(A)$ that we know about.  This is why explicit descriptions
of left adjoints are often hard to come by.
\end{example}

\begin{example}
More generally, GAFT guarantees that forgetful%
%
\index{functor!forgetful!left adjoint to}
%
functors between categories of algebras, such as
\[
\Ab \to \Grp, 
\quad
\Grp \to \Mon,
\quad
\Ring \to \Mon,
\quad
\Vect_\complexes \to \Vect_\reals,
\]
have left adjoints.  (Some of them are described in
Examples~\ref{egs:adjns-alg}.)  This is `more generally' because $\Set$ can
be seen as a degenerate example of a category of algebras, in the sense of
Remark~\ref{rmk:alg-thy}: a group, ring, etc., is a set equipped with some
operations satisfying some equations, and a set is a set equipped with no
operations satisfying no equations.
\end{example}

The special adjoint functor theorem (SAFT) operates under much tighter
hypotheses than GAFT, and is much less widely applicable.  Its main
advantage is that it removes the condition on weakly initial sets.  Indeed,
it removes \emph{all} further conditions on the functor $G$.

\begin{thm}[Special adjoint functor theorem]
%
\index{adjoint functor theorems!special}%
\index{special adjoint functor theorem}%
\index{SAFT (special adjoint functor theorem)}
%
Let $\cat{A}$ be a category, $\cat{B}$ a complete category, and $G\from
\cat{B} \to \cat{A}$ a functor.  Suppose that $\cat{A}$ and $\cat{B}$ are
locally small, and that $\cat{B}$ satisfies certain further conditions.  Then
\[
G \text{ has a left adjoint}
\iff
G \text{ preserves limits}.
\]
\end{thm}
% 
A precise statement and proof can be found in Section~V.8 of \citeCWM.

\begin{example}
Here is the classic application of SAFT.  Let $\CptHff$%
%
\index{topological space!compact Hausdorff}
%
be the category of compact Hausdorff spaces, and $U\from \CptHff \to \Tp$
the forgetful functor.  SAFT tells us that $U$ has a left adjoint $F$,
turning any space into a compact Hausdorff space in a canonical way.

The existence of this left adjoint is far from obvious, and verifying the
hypotheses of SAFT (or indeed, constructing $F$ in any other way) requires
some deep theorems of topology.  Given a space $X$, the resulting compact
Hausdorff space $F(X)$ is called its \demph{Stone--\v{C}ech%
%
\index{Stone-Cech compactification@Stone--\v{C}ech compactification}
%
compactification}.  Provided that $X$ satisfies some mild separation
conditions, the unit of the adjunction at $X$ is an embedding, so that
$UF(X)$ contains $X$ as a subspace. 

Another advantage of SAFT is that one can extract from its proof a fairly
explicit formula for the left adjoint.  In this case, it tells us that
$F(X)$ is the closure of the image of the canonical map
\[
X \to [0, 1]^{\Tp(X, [0, 1])}, 
\]
where the codomain is a power of $[0, 1]$ in $\Tp$.  
%
\index{adjoint functor theorems|)}
%
\end{example}


\minihead{Cartesian closed categories}
%
\index{category!cartesian closed|(}%
\index{cartesian closed category|(}
%

We have seen that for every set $B$, there is an adjunction $(\dashbk\times B)
\ladj (\dashbk)^B$ (Example~\ref{eg:adjn:cc}), and that for every category
$\cat{B}$, there is an adjunction $(\dashbk \times \cat{B}) \ladj
\ftrcat{\cat{B}}{\dashbk}$
(Remark~\ref{rmks:global-hom}\bref{rmks:global-hom:cc}).

\begin{defn}
A category $\cat{A}$ is \demph{cartesian closed} if it has finite products
and for each $B \in \cat{A}$, the functor $\dashbk \times B\from \cat{A}
\to \cat{A}$ has a right adjoint.
\end{defn}
% 
We write the right adjoint as $(\dashbk)^B$,%
% 
\ntn{exp-cc}
%
and, for $C \in \cat{A}$, call
$C^B$ an \demph{exponential}.%
%
\index{exponential}
%
We may think of $C^B$ as the space of maps from $B$ to $C$.  Adjointness
says that for all $A, B, C \in \cat{A}$,
\[
\cat{A}(A \times B, C)
\iso
\cat{A}\bigl(A, C^B\bigr)
\]
naturally in $A$ and $C$.  In fact, the isomorphism is natural in $B$ too;
that comes for free.

\begin{example}
$\Set$ is cartesian closed; $C^B$ is the function%
%
\index{set!functions@of functions}%
\index{function!set of functions}
%
set $\Set(B, C)$.  
\end{example}

\begin{example}
\hspace*{-.5pt}%
$\CAT$ is cartesian closed; $\cat{C}^\cat{B}$ is the functor%
%
\index{functor!category}
%
category $\ftrcat{\cat{B}}{\cat{C}}$.
\end{example}

In any cartesian closed category with finite sums, the
isomorphisms~\eqref{eq:Set-dist} and~\eqref{eq:Set-codist} of
Example~\ref{eg:arith-set} hold, for the same reasons as stated there.  The
objects of a cartesian closed category therefore possess an arithmetic%
%
\index{arithmetic}
%
like that of the natural numbers.  This thought can be developed in several
interesting directions, but here we just note that these isomorphisms
provide a way of proving that a category is \emph{not} cartesian closed.

\begin{example}
$\Vect_k$%
%
\index{vector space!category of vector spaces!cartesian closed@is not
  cartesian closed} 
%
is not cartesian closed, for any field $k$.  It does have finite products,
as we saw in Example~\ref{eg:prod-vs}: binary product is direct sum
$\oplus$, and the terminal object is the trivial vector space $\{0\}$,
which is also initial.  But if $\Vect_k$ were cartesian closed then
equations~\eqref{eq:Set-dist} would hold, so that $\{0\} \oplus B \iso
\{0\}$ for all vector spaces $B$.  This is plainly false.
\end{example}

\begin{remark}
For any vector spaces $V$ and $W$, the set $\Vect_k(V, W)$ of linear maps
can itself be given the structure of a vector space, as in
Example~\ref{eg:fns-on-vs}.  Let us now call this vector space $[V, W]$.

Given that exponentials are supposed to be `spaces of maps', you might
expect $\Vect_k$ to be cartesian closed, with $[\dashbk, \dashbk]$ as its
exponential.  We have just seen that this cannot be so.  But as it turns
out, the linear maps $U \to [V, W]$ correspond to the \emph{bi}linear%
%
\index{map!bilinear}
%
maps $U \times V \to W$, or equivalently the linear maps $U \otimes V \to
W$.%
%
\index{tensor product}
%
In the jargon, $\Vect_k$ is an example of a `monoidal%
%
\index{monoidal closed category}%
\index{category!monoidal closed}
%
closed category'.  These are like cartesian closed categories, but with the
cartesian (categorical) product replaced by some other operation called
`product', in this case the tensor product of vector spaces.
\end{remark}

For any set $I$, the product category $\Set^I$ is cartesian closed, just
because $\Set$ is.  (Exponentials in $\Set^I$, as well as products, are
computed pointwise.)  Put another way, $\pshf{\scat{A}}$ is cartesian
closed whenever $\scat{A}$ is discrete.  We now show that, in fact,
$\pshf{\scat{A}}$ is cartesian closed for any small category $\scat{A}$
whatsoever.  

In preparation for proving this, let us conduct a thought%
%
\index{thought experiment}
%
experiment.  Write $\psh{\scat{A}}%
%
\ntn{psh}
%
 = \pshf{\scat{A}}$.  If $\psh{\scat{A}}$ \emph{is} cartesian
closed, what must exponentials in $\psh{\scat{A}}$ be?  In other words, given
presheaves $Y$ and $Z$, what must $Z^Y$ be in order that
% 
\begin{equation}        
\label{eq:cc-pshf}
\psh{\scat{A}}\bigl(X, Z^Y\bigr)
\iso
\psh{\scat{A}}(X \times Y, Z)
\end{equation}
% 
for all presheaves $X$?  If this is true for all presheaves $X$, then in
particular it is true when $X$ is representable, so
\[
Z^Y(A)
\iso
\psh{\scat{A}}\bigl(\h_A, Z^Y\bigr)
\iso
\psh{\scat{A}}(\h_A \times Y, Z) 
\]
for all $A \in \scat{A}$, the first step by Yoneda.  This tells us what
$Z^Y$ must be.  Notice that $Z^Y(A)$ is not simply $Z(A)^{Y(A)}$, as one
might at first guess: exponentials in a presheaf category are \emph{not}
generally computed pointwise.%
%
\index{pointwise}
%


\begin{thm}     
\label{thm:pshf-cc}
For any small category $\scat{A}$, the presheaf%
%
\index{presheaf!category of presheaves!cartesian closed@is cartesian closed}
%
 category $\psh{\scat{A}}$
is cartesian closed.
\end{thm}
% 
Here is the strategy of the proof.  The argument in the thought experiment
gives us the isomorphism~\eqref{eq:cc-pshf} whenever $X$ is representable.
A general presheaf $X$ is not representable, but it is a colimit of
representables, and this allows us to bootstrap our way up.

\begin{pf}
We know that $\psh{\scat{A}}$ has all limits, and in particular, finite
products.  It remains to show that $\psh{\scat{A}}$ has exponentials.  Fix
$Y \in \psh{\scat{A}}$.

First we prove that $\dashbk \times Y\from \psh{\scat{A}} \to
\psh{\scat{A}}$ preserves colimits.  (Eventually we will prove that
$\dashbk\times Y$ has a right adjoint, from which preservation of colimits
follows, but our proof that it has a right adjoint will \emph{use}
preservation of colimits.)  Indeed, since products and colimits in
$\psh{\scat{A}}$ are computed pointwise, it is enough to prove that for any
set $S$, the functor $\dashbk \times S\from \Set \to \Set$ preserves
colimits, and this follows from the fact that $\Set$ is cartesian closed.

For each presheaf $Z$ on $\scat{A}$, let $Z^Y$ be the presheaf defined by
\[
Z^Y (A) = \psh{\scat{A}} (\h_A \times Y, Z)
\]
for all $A \in \scat{A}$.  This defines a functor $(\dashbk)^Y \from
\psh{\scat{A}} \to \psh{\scat{A}}$.  

I claim that $(\dashbk \times Y) \ladj (\dashbk)^Y$.  Let $X, Z \in
\psh{\scat{A}}$.  Write $P\from \elt{X} \to \scat{A}$ for the projection
(as in Definition~\ref{defn:cat-elts}), and write $\h_P = \h_\bl \of P$.
Then
% 
\begin{align}
\label{eq:pcc:density}
\psh{\scat{A}}\bigl(X, Z^Y\bigr) &
\iso   
\psh{\scat{A}} \biggl( \colt{\elt{X}} \h_P, Z^Y \biggr)  \\      
\label{eq:pcc:out}
        &
\iso   
\lt{\elt{X}^{\op}} \psh{\scat{A}}\bigl(\h_P, Z^Y\bigr)   \\
\label{eq:pcc:yon}
        &
\iso   
\lt{\elt{X}^{\op}} Z^Y(P)                       \\
\label{eq:pcc:defn}
        &
\iso   
\lt{\elt{X}^{\op}} \psh{\scat{A}}(\h_P \times Y, Z)     \\
\label{eq:pcc:in}
        &
\iso   
\psh{\scat{A}}\biggl( \colt{\elt{X}} (\h_P \times Y), Z \biggr) \\
\label{eq:pcc:cocts} 
       &
\iso   
\psh{\scat{A}}\biggl(\biggl(\colt{\elt{X}} \h_P\biggr) \times Y, Z\biggr) \\
\label{eq:pcc:density-again}
        &
\iso   
\psh{\scat{A}}(X \times Y, Z)
\end{align}
% 
naturally in $X$ and $Z$.  Here~\eqref{eq:pcc:density}
and~\eqref{eq:pcc:density-again} follow from Theorem~\ref{thm:density};
\eqref{eq:pcc:out} and~\eqref{eq:pcc:in} are because representables
preserve limits (as rephrased in Remark~\ref{rmk:rep-pres});
\eqref{eq:pcc:yon} is by Yoneda; \eqref{eq:pcc:defn} is by definition of
$Z^Y$; and~\eqref{eq:pcc:cocts} is because $\dashbk\times Y$ preserves
colimits.
\end{pf}

This result can be seen as a step along the road to topos%
%
\index{topos}
%
theory.  A topos is a category with certain special properties.  Topos
theory unifies, in an extraordinary way, important aspects of logic and
geometry.

For instance, a topos can be regarded as a `universe of sets':%
%
\index{set!category of sets!topos@as topos}
%
$\Set$ is the most basic example of a topos, and every topos shares enough
features with $\Set$ that one can reason with its objects as if they were
sets of some exotic kind.  On the other hand, a topos can be regarded as a
generalized topological space:%
%
\index{topological space!topos@as topos}
%
every space gives rise to a topos (namely, the category of sheaves%
%
\index{sheaf}
%
on it), and topological properties of the space can be reinterpreted in a
useful way as categorical properties of its associated topos.

By definition, a topos is a cartesian closed category with finite limits
and with one further property: the existence of a so-called subobject%
%
\index{subobject!classifier}
%
classifier.  For example, the two-element%
%
\index{set!two-element}
%
set $2$ is the subobject classifier of $\Set$, which means, informally,
that subsets of a set $A$ correspond one-to-one with maps $A \to 2$.
Exercises~\ref{ex:soc} and~\ref{ex:pshf-topos} give the formal definition
of subobject classifier, then guide you through the proof that $\Set$, and,
more generally, every presheaf category, is a topos.%
%
\index{category!cartesian closed|)}%
\index{cartesian closed category|)}
%


\exs


\begin{question}        
\label{ex:no-adjt}
\begin{enumerate}[(b)]
\item 
Prove that the forgetful functor $U\from \Grp \to \Set$ has no right
adjoint.

\item 
Prove that the chain of adjunctions $C \ladj D \ladj O \ladj I$%
%
\index{category!category of categories!adjunctions with $\Set$}
%
in Exercise~\ref{ex:cdoi} extends no further in either direction.

\item
Does the chain of adjunctions in Exercise~\ref{ex:pshf-adjns} extend
further in either direction?
\end{enumerate}
\end{question}


\begin{question}
Let $\cat{A}$ be a locally small category.  For functors $U\from \cat{A}
\to \Set$, consider the following three conditions: (A)~$U$ has a left
adjoint; (R)~$U$ is representable; (L)~$U$ preserves limits.  
% 
\begin{enumerate}[(b)]
\item 
Show that (A) $\textonlyif$ (R) $\textonlyif$ (L).

\item 
Show that if $\cat{A}$ has sums then (R) $\textonlyif$ (A).%
%
\index{functor!representable!adjoints@and adjoints}
%
\end{enumerate}
% 
(If $\cat{A}$ satisfies the hypotheses of the special adjoint functor
theorem then also (L) $\textonlyif$ (A), so the three conditions are
equivalent.)
\end{question}


\begin{question}        
\label{ex:small-complete}
\begin{enumerate}[(b)]
\item 
Prove that every preordered%
%
\index{ordered set!preordered set@vs.\ preordered set}
%
set is equivalent (as a category) to an ordered set.

\item 
Let $\cat{A}$ be a category with all small products.  Suppose that
$\cat{A}$ is not a preorder, so that there exists a parallel pair of maps
$\parpairi{A}{B}{f}{g}$ in $\cat{A}$ with $f \neq g$.  By considering the
maps $A \to B^I$ for each set $I$, prove that $\cat{A}$ is not small.

\item 
Deduce that every small category with small products is equivalent to a
complete ordered set.%
%
\index{ordered set!complete small category is}
%

\item 
Adapt the argument to prove that every finite category with finite
products is equivalent to a complete ordered set.
\end{enumerate}
\end{question}


\begin{question}        
\label{ex:free-gp-card}
Probably the most important application of the general adjoint functor
theorem is to proving that forgetful functors between categories of
algebras have left adjoints (Example~\ref{eg:gaft-free-alg}).  Verifying
the hypotheses can be done with some cardinal%
%
\index{cardinality}%
\index{arithmetic!cardinal}
%
arithmetic.  Here is a typical example.
% 
\begin{enumerate}[(b)]
\item   
\label{part:gen-subgp-bound}
Let $A$ be a set.  Prove that for any group $G$ and family $(g_a)_{a \in
A}$ of elements of $G$, the subgroup of $G$ generated by $\{g_a \such a
\in A\}$ has cardinality at most $\max \{\crd{\nat}, \crd{A}\}$.

\item   
\label{part:small-coll-gps}
Prove that for any set $S$, the collection of isomorphism classes of groups
of cardinality at most $\crd{S}$ is small.

\item 
Let $U\from \Grp \to \Set$%
%
\index{group!free}
%
be the forgetful functor from groups to sets.  Deduce
from~\bref{part:gen-subgp-bound} and~\bref{part:small-coll-gps} that for
every set $A$, the comma category $\comma{A}{U}$ has a weakly initial set.

\item 
Use GAFT to conclude that $U$ has a left adjoint.
\end{enumerate}
\end{question}


\begin{question}
Let $\scat{A}$ be a small cartesian closed category.  Prove that the
Yoneda%
%
\index{Yoneda embedding!preserves exponentials}%
\index{exponential!preserved by Yoneda embedding}
%
embedding $\scat{A} \to \pshf{\scat{A}}$ preserves the whole cartesian
closed structure (exponentials as well as products).
\end{question}


\begin{question}        
\label{ex:soc}
Recall from Exercise~\ref{ex:subobjects} the notion of subobject.  A 
category $\cat{A}$ is \demph{well-powered}%
%
\index{category!well-powered}%
\index{well-powered}
%
if for each $A \in \cat{A}$, the class of subobjects of $A$ is small, that
is, a set.  (All of our usual examples of categories are well-powered.)
Let $\cat{A}$ be a well-powered category with pullbacks, and write
$\Sub(A)$ for the set of subobjects of an object $A \in \cat{A}$.
% 
\begin{enumerate}[(b)]
\item
Deduce from Exercise~\ref{ex:pb-monic} that any map $A' \toby{f} A$ in
$\cat{A}$ induces a map $\Sub(f)\from \Sub(A) \to \Sub(A')$.

\item 
Show that this determines a functor $\Sub\from \cat{A}^\op \to \Set$.
(Hint: use Exercise~\ref{ex:pb-pasting}.)

\item 
For some categories $\cat{A}$, the functor $\Sub$ is representable.  A
\demph{subobject%
%
\index{subobject!classifier}
%
classifier} for $\cat{A}$ is an object $\Omega \in
\cat{A}$ such that $\Sub \iso \h_\Omega$.  Prove that $2$ is a subobject
classifier for $\Set$.
\end{enumerate}
% 
A \demph{topos}%
%
\index{topos}
%
is a cartesian closed category with finite limits and a subobject
classifier.  You have just completed the proof that $\Set$ is a topos.
\end{question}


\begin{question}        
\label{ex:pshf-topos}
This exercise follows on from the last, culminating in the proof that every
presheaf category%
%
\index{presheaf!category of presheaves!topos@is topos}
%
is a topos.%
%
\index{topos}
%
Let $\scat{A}$ be a small category.
% 
\begin{enumerate}[(b)]
\item
By conducting a thought%
%
\index{thought experiment}
%
experiment similar to the one before the statement of
Theorem~\ref{thm:pshf-cc}, find out what the subobject classifier $\Omega$
of $\pshf{\scat{A}}$ must be if it exists.

\item 
Prove that this $\Omega$ is indeed a subobject classifier.

\item 
Conclude that $\pshf{\scat{A}}$ is a topos.
\end{enumerate}
\end{question}


