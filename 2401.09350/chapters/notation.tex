\chapter*{Notation}
\label{appendix:notation}
\addcontentsline{toc}{chapter}{Notation}

This section summarizes the special symbols and notation used throughout this work.
We often repeat these definitions in context as a reminder, especially if we choose to
abuse notation for brevity or other reasons.

\begin{svgraybox}
    Paragraphs that are highlighted in a gray box such as this contain important
    statements, often conveying key findings or observations, or a detail that will
    be important to recall in later chapters.
\end{svgraybox}

\subsection*{Terminology}
We use the terms ``vector'' and ``point'' interchangeably.
In other words, we refer to an ordered list of $d$ real values
as a $d$-dimensional vector or a point in $\mathbb{R}^d$.

We say that a point is a \emph{data} point if it is part of the collection
of points we wish to sift through. It is a \emph{query} point if it is the input
to the search procedure, and for which we are expected to return the top-$k$
similar data points from the collection.


\subsection*{Symbols}

\subsubsection*{Reserved Symbols}
\begin{longtable*}{p{0.3\linewidth}p{0.7\linewidth}}
$\mathcal{X}$ & Used exclusively to denote a collection of vectors. \\
$m$ & We use this symbol exclusively to denote the cardinality of a collection of data points, $\mathcal{X}$. \\
$q$ & Used singularly to denote a query point. \\
$d$ & We use this symbol exclusively to refer to the number of dimensions. \\
$e_1, e_2, \ldots, e_d$ & Standard basis vectors in $\mathbb{R}^d$ \\
\end{longtable*}

\subsubsection*{Sets}
\begin{longtable*}{p{0.3\linewidth}p{0.7\linewidth}}
$\mathcal{J}$ & Calligraphic font typically denotes sets. \\
$\lvert \cdot \rvert$ & The cardinality (number of items) of a finite set. \\
$[n]$ & The set of integers from $1$ to $n$ (inclusive): $\{ 1, 2, 3, \ldots, n\}$. \\
$B(u, r)$ & The closed ball of radius $r$ centered at point $u$: $\{ v \;|\; \delta(u, v) \leq r \}$ where
$\delta(\cdot, \cdot)$ is the distance function.\\
$\setminus$ & The set difference operator: $\mathcal{A} \setminus \mathcal{B} = \{ x \in \mathcal{A} \;|\; x \notin \mathcal{B} \}$. \\
$\triangle$ & The symmetric difference of two sets. \\
$\mathbbm{1}_p$ & The indicator function. It is $1$ if the predicate $p$ is true, and $0$ otherwise. \\
\end{longtable*}

\subsubsection*{Vectors and Vector Space}
\begin{longtable*}{p{0.3\linewidth}p{0.7\linewidth}}
$[a, b]$ & The closed interval from $a$ to $b$. \\
$\mathbb{Z}$ & The set of integers. \\
$\mathbb{R}^d$ & $d$-dimensional Euclidean space. \\
$\mathbb{S}^{d-1}$ & The hypersphere in $\mathbb{R}^d$. \\
$u, v, w$ & Lowercase letters denote vectors. \\
$u_i, v_i, w_i$ & Subscripts identify a specific coordinate of a vector, so that $u_i$ is the $i$-th coordinate of vector $u$. \\
\end{longtable*}

\subsubsection*{Functions and Operators}
\begin{longtable*}{p{0.3\linewidth}p{0.7\linewidth}}
$\mathit{nz}(\cdot)$ & The set of non-zero coordinates of a vector: $\mathit{nz}(u) = \{ i \;|\; u_i \neq 0 \}$. \\
$\delta(\cdot, \cdot)$ & We use the symbol $\delta$ exclusively to denote the distance function,
taking two vectors and producing a real value. \\
$J(\cdot, \cdot)$ & The Jaccard similarity index of two vectors: $J(u, v) = \lvert \mathit{nz}(u) \cap \mathit{nz}(v) \rvert / \lvert \mathit{nz}(u) \cup \mathit{nz}(v) \rvert$. \\
$\langle \cdot, \cdot \rangle$ & Inner product of two vectors: $\langle u, v \rangle = \sum_i u_i v_i$. \\
$\lVert \cdot \rVert_p$ & The $L_p$ norm of a vector: $\lVert u \rVert_p = (\sum_i \lvert u_i \rvert^p)^{1/p}$. \\
$\oplus$ & The concatenation of two vectors. If $u, v \in \mathbb{R}^d$, then $u \oplus v \in \mathbb{R}^{2d}$. \\
\end{longtable*}

\subsubsection*{Probabilities and Distributions}
\begin{longtable*}{p{0.3\linewidth}p{0.7\linewidth}}
$\ev[\cdot]$ & The expected value of a random variable. \\
$\var[\cdot]$ & The variance of a random variable. \\
$\probability[\cdot]$ & The probability of an event. \\
$\land$, $\lor$ & Logical AND and OR operators. \\
$Z$ & We generally use uppercase letters to denote random variables. \\
\end{longtable*}
