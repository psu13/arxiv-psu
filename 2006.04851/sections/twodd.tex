% !TEX root = ../lifeonbrane3.tex
%

\subsection{The case of two dimensions}\label{sec:two-d}

Recall that the curvature terms in the induced action \reef{diver1} have coefficients with inverse powers of $(d-2)$ and so we must reconsider the calculation of this brane action for $d=2$, \ie when the bulk space is (locally) AdS$_3$ and the induced geometry on the brane is AdS$_2$. This section sketches the necessary calculations, which are largely the same as those performed in higher dimensions, but with a few important differences.

Let us add that in contrast to the induced action, the calculations in section \ref{BranGeo}, where the position of the brane is determined using the Israel junction conditions, need no modifications for $d=2$.
Therefore we can simply substitute $d=2$ into eqs.~\reef{position} and \reef{secondorder} for the brane position to find
\beq\label{2dposition}
\s^2 \simeq 2L^2\veps+ \frac{L}{4 \pi \Gbk T_o}\,\veps^2+\cdots\,, \qquad{\rm with}\quad
\veps=1-4 \pi \Gbk L T_o \,.
\eeq
Of course, we must be able to reproduce the same result using the new induced gravity action.


\subsubsection*{Integration of bulk action}

As discussed in section \ref{indyaction}, one can determine the structure of the terms in the induced action by a careful examination of the FG expansion near the asymptotic boundary \cite{Skenderis:1999nb,deHaro:2000wj,deHaro:2000vlm}. However, we can take the simpler route here, since in two dimensions the Riemann curvature has a single component and therefore the entire induced action can be expressed in terms of the Ricci scalar $\ric(\tilde g)$. Therefore, we evaluate the on-shell bulk action and match the boundary divergences to an expansion in $\ric(\tilde g)$. That is, we substitute the metric \reef{metric3} into the bulk action \reef{act2} plus the corresponding  Gibbons-Hawking-York surface term \cite{PhysRevLett.28.1082,Gibbons:1976ue} and integrate over the radial direction $z$. The result can be expressed as a boundary integral with a series of divergences as $\s\to0$,\footnote{This expression also includes ${\cal O}(\s^2)$ contributions, which are necessary to match eq.~\reef{2dposition} to ${\cal O}(\veps^2)$ in the following. Further, note that we are ignoring the contributions coming from asymptotic boundaries at  $z\to\infty$.}
\beq\label{induced act2}
I_\mt{diver}=\frac{L}{16 \pi \Gbk}\int d^2x\sqrt{-g^{\ads2}} \left[\frac{1}{\s^2} +\frac{1}{L^2}\,\log\Big(\frac{\s}{L}\Big) -\frac{\s^2}{16L^4} +\cdots\right]\,.
\eeq
%where $\sir$ is some IR scale introduced by the upper region of the $z$ integration.
Now we rewrite the above expression in terms of the induced metric and the corresponding Ricci scalar combining eqs.~\reef{metric3}, \reef{curve1} and \reef{Ricky2}, which yield
\beq\label{AdS3}
\sqrt{-\tilde{g}} =  \frac{L^2}{\s^2}\left(1 + \frac{\s^2}{4L^2}\right)^2 \sqrt{-g^{\ads2}} \,, \qquad\quad
\ric = -2\,\frac{\s^2}{L^4}\left(1+\frac{\s^2}{4L^2}\right)^{-2}\,.
\eeq
Using these expressions, %as well as $\lir\equiv L^2/\sir$, 
the induced action becomes\footnote{Our derivation of eq.~\eqref{ind-action2} will miss terms involving derivatives of $\tilde{R}$ as these vanish for the constant curvature geometry of our brane. However,  such terms will only appear at higher orders, \ie in the `$\cdots$' (other than the total derivative $\tilde\Box\tilde{R}$).} 
%The same issue arises in \eqref{sol4}, where we assume the Liouville field solves its equation of motion \eqref{eom4} for constant $\tilde{R}$.}
\beq\label{ind-action2}
I_\mt{diver}=\frac{L}{16 \pi \Gbk}\int d^2x\sqrt{-\tilde{g}} \Big[\frac{2}{L^2} -  \frac12\, \tilde{R} \,\log\left(-\frac{L^2 }{2}\tR\right) + \frac{1}{2}\,\tilde{R}+\frac{L^2}{16}\,\tilde{R}^2 +\cdots\Big]\,.
\eeq

The most striking feature of this induced action is the term proportional to $\ric\log|\ric|$. The appearance of this logarithm is related to the conformal anomaly \cite{Henningson:1998gx,Henningson:1998ey,Burgess:1999vb}, and points towards the fact that the corresponding gravitational action in nonlocal,\footnote{Similar nonlocalities appear in the curvature-squared or four-derivative contributions with $d=4$, or more generally in the interactions with $d/2$ curvatures for higher (even) $d$. Hence they do not play a role in higher dimensions if we work in the regime where the induced action \reef{act3} is well approximated by Einstein gravity coupled to a cosmological constant.} as we discuss next.  Further, since the Einstein-Hilbert term is topological in two dimensions, it turns out that this unusual action is precisely what is needed to match the dynamics of the bulk gravity described above, \ie the position of the brane in eq.~\reef{2dposition}.

The logarithmic contribution  should correspond to that coming from the nonlocal Polyakov action \cite{Skenderis:1999nb}. Schematically, we would have
\beq
I_\mt{bulk}\simeq I_\mt{Poly}= -\frac{\alpha\,L}{16 \pi \Gbk}\int d^2x\sqrt{-\tilde{g}} \, \tilde{R}\,\frac1{\tilde \Box}\,\tilde{R} \,,
\label{induced-action3}
\eeq
where we have introduced an arbitrary constant $\alpha$ here but it will be fixed by comparing with the divergences in the integrated action. Of course, $\frac1{\tilde \Box}\,\tilde{R}$ indicates a convolution of the Ricci scalar with the scalar Green's function, but there are subtleties here in dealing with constant curvatures. The latter are ameliorated by making the action \reef{induced-action3} local by introducing a auxiliary field $\phi$ (\eg see \cite{Skenderis:1999nb,Alvarez:1982zi}),
\beq
 I_\mt{Poly}=\frac{\alpha\,L}{8\pi \Gbk}\int d^2x\sqrt{-\tilde{g}} \, \left[
 -\frac12\,\tilde g^{ij}\tilde\nabla_i\phi\tilde\nabla_j\phi
 +\phi\,\tilde R + \chi\,e^{-\phi}
\right] \,.
\label{PolyAct2}
\eeq
where $\chi$ is a fixed constant.\footnote{The last term is needed to take care of zero mode problem \cite{Alvarez:1982zi}. Examining the equation of motion \reef{eom4}, one can think of $\phi$ as a conformal factor relating the metric $\tg_{ij}$ to a canonical constant curvature metric $\hat g_{ij}$, \ie $\tg_{ij}=e^{\phi}\hat g_{ij}$ with $\hat R(\hat g)=\chi$ \cite{Alvarez:1982zi,Frolov:1996hd}.
Hence we choose $\chi$ to be negative to match the sign of $\tilde R$. Further, note that with the interaction $\chi e^{-\phi}$ in the action \reef{PolyAct2}, $\phi$ becomes an interacting field \cite{Skenderis:1999nb}.
\label{ZZZ}}

The equation of motion resulting from eq.~\reef{PolyAct2}  is
\beq
0=\tilde\Box \phi +\tilde R - \chi\,e^{-\phi}\,,
\label{eom4}
\eeq
which has a simple solution when $\tilde R$ is a constant, namely,
\beq
 \phi=\phi_0 = \log(\chi/\tilde R)\,.
\label{sol4}
\eeq
% \vc{Are we saying that \eqref{ind-action2} and \eqref{induct} should only be trusted for $\tilde{R}$ constant? How do we know that there aren't terms involving derivatives of $\tilde{R}$ which happen to vanish on our constant $\tilde{R}$ brane. Since we should only be including `divergent' things in the induced gravity action, perhaps we just want to extract a constant logarithmically divergent term, as in Section 4.1 in \cite{Almheiri:2019psf} --- more explanation in comment below \eqref{shift0}; also, related to the comment below \eqref{ind-action2}.}
Evaluating the Polyakov action with $\phi = \phi_0$ yields
\beq
 I_\mt{Poly}\big|_{\phi=\phi_0}=-\frac{\alpha\,L}{8\pi \Gbk}\int d^2x\sqrt{-\tilde{g}} \, \left[
 \tilde R \,\log(\tilde R/\chi) - \tilde R\, \right] \,.
\label{PolyAct3}
\eeq
Comparing this expression with the log term in eq.~\reef{ind-action2}, we fix $\alpha=\frac14$ and $\chi=-\frac2{L^2}$.

Varying the action \reef{PolyAct2} with respect to the metric, we find the corresponding contribution to the `gravitational' equations of motion
\beqa
T^\mt{Poly}_{ij}=-\frac2{\sqrt{-g}}\,\frac{\delta I_\mt{Poly}}{\delta g^{ij}}
&=&\frac{L}{32\pi \Gbk}\Big[
\tilde\nabla_i\phi\tilde\nabla_j\phi
 +2\,\tilde\nabla_i\tilde\nabla_j\phi
\label{eom5}\\
&&\qquad\qquad\left.
-\tg_{ij}\left(\frac12\,(\tilde\nabla\phi)^2
 +2\,\tilde\Box \phi- \chi\,e^{-\phi} \right)
\right]\,,
\nonumber
\eeqa
where we have used $\tilde R_{ij} -\frac12\tg_{ij}\tilde R=0$ for $d=2$ to eliminate the terms linear in $\phi$ (without any derivatives). Now substituting $\phi_0$, we find that this expression reduces to
\beq
T^\mt{Poly}_{ij}\big|_{\phi=\phi_0}=
\frac{L}{32\pi \Gbk}\,\tg_{ij}\, \tilde R\,,
\label{onshell5}
\eeq
which we will substitute into evaluating the equations of motion below to fix the position of the brane. As an aside, we can take the trace of the above expression to find
that it reproduces the trace anomaly, \eg \cite{Duff:1977ay,Duff:1993wm}
\beq\label{trace}
\langle T^i{}_i \rangle = \frac{c}{24\pi}\,\tilde R \,,
\eeq
where we recall that $c=\frac{3L}{2\Gbk}$ for the boundary CFT. In our case, the trace anomaly will be twice as large, since there are two copies of the CFT living on the brane.

The induced action $I_\mt{induced} = 2\,I_\mt{diver} + I_\mt{brane}$ can be written as
%\footnote{At this point, we have made the convenient choice to equate $\lir=\ell_\mt{eff}$.}
\beq\label{induct}
I_\mt{induced} = \frac{1}{16 \pi G_\mt{eff}}\int d^2x\sqrt{-\tilde{g}} \Big[\frac{2}{\ell_\mt{eff}^2} -  \tilde{R} \,\log\left(-\frac{L^2 }{2}\tR\right)+\tR +\frac{L^2}{8}\,\tilde{R}^2 +\cdots\Big]\,,
\eeq
where $\ell_\mt{eff}$ is given by the expression in eq.~\reef{Newton2} with $d=2$, \ie
\beq\label{lobster}
\frac{L^2}{{\ell}_\mt{eff}^2}=2\( 1-4 \pi \Gbk L T_o\)\,,
\eeq
however, we have set $\Geff =  \Gbk/L$ here. The metric variation then yields the following equation of motion
\beq\label{gamble3}
0=\frac{2}{\ell_\mt{eff}^2}\,\tilde{g}_{ij}+\tilde{g}_{ij} \,
\tilde{R} + \frac{L^2}8\,\tR\(\tg_{ij}\,\tR-4 \tR_{ij}\)+\cdots\,,
\eeq
where we dropped the terms involving derivatives of curvatures arising from the variation of the $\tR^2$ term.
To leading order, we find $\tR\sim -2/\ell_\mt{eff}^2 = -4\veps/L^2$ in agreement with eqs.~\reef{curve2} and \reef{Ricky2}. Hence, the gravitational equations of motion again fix the (leading-order) position of the brane for $d=2$, and further it is a straightforward exercise to match to second order corrections in eq.~\reef{2dposition} using the curvature-squared contributions in eq.~\reef{gamble3}.


\subsubsection*{Adding JT gravity}

Much of the recent literature on quantum extremal islands examines models involving two-dimensional gravity, \eg \cite{Almheiri:2019psf, Almheiri:2019hni, Almheiri:2019yqk, Chen:2019uhq, Penington:2019kki, Almheiri:2019qdq, Chen:2019iro}, however, the gravitational theory in these models is Jackiw-Teitelboim (JT) gravity \cite{Jackiw:1984je,Teitelboim:1983ux}. One can incorporate JT gravity into the current model by dropping the usual tension term \reef{braneaction}, and instead using the following brane action\footnote{Alternatively, one could simply add $I_\mt{JT}$ to the usual tension term. With this approach, an extra source term appears in eq.~\reef{fulleom}, but it can be eliminated by shifting the dilaton in a manner similar to eq.~\reef{shift1}.} 
\beq\label{braneact2}
I_\mt{brane}= I_\mt{JT} + I_\mt{ct}\,,
\eeq
where the JT action takes the usual form,
\beq\label{JTee}
I_\mt{JT} =\frac{1}{16\pi G_\mt{brane}}\int d^2x\sqrt{-\tilde{g}}\left[\Phi_0\,\tilde{R}+ \Phi\left(\tilde{R}+\frac{2}{\ell^2_\mt{JT}}
\right)\right]\,.
\eeq
Here, as in previous actions, we have ignored the boundary terms associated with the JT action, \eg see \cite{Maldacena:2016upp}, and we have introduced the dilaton $\Phi$. Recall that $\Phi_0$ is simply a constant and so the first term is topological but contributes to the generalized entropy. In eq.~\reef{braneact2}, we have also included a counterterm 
\beq\label{count123}
I_\mt{ct}=-\frac{1}{4\pi \Gbk L}\int d^2x\sqrt{-\tilde{g}}\,,
\eeq
which is tuned to cancel the induced cosmological constant on the brane. This choice ensures that the JT gravity \reef{JTee} couples to the boundary CFT in the expected way, \eg as in  \cite{Almheiri:2019psf,Maldacena:2016upp} -- see further comments below.

The full induced action now takes the form
\begin{align}
&&I_\mt{induced}=\frac{1}{16 \pi G_\mt{eff}}\int d^2x\sqrt{-\tilde{g}} \Big[ -  \tilde{R} \,\log\left(-\frac{L^2 }{2}\tR\right) +\frac{L^2}{8}\,\tilde{R}^2 +\cdots\Big]
\nonumber\\
&&\qquad+\frac{1}{16\pi G_\mt{brane}}\int d^2x\sqrt{-\tilde{g}}\left[\tilde\Phi_0\,\tilde{R}+ \Phi\left(\tilde{R}+\frac{2}{\ell^2_\mt{JT}}
\right)\right]\,,
\label{fullindyact}
\end{align}
where we have combined the two topological contributions in the second line with\footnote{In \cite{Almheiri:2019psf}, $\tilde \Phi_0$ would also absorb a logarithmic constant $-2\log(L/z_\mt{B})$, which would be accompanied by a shift in the prefactor in the argument of the logarithmic term in eq.~\reef{fullindyact}, \ie $2/L^2\to2/ z^2_\mt{B}$.}
\beq\label{shift0}
\tilde \Phi_0 =\Phi_0 + G_\mt{brane}/G_\mt{eff}\,.
\eeq

Now, with the JT action \reef{JTee}, the dilaton equation of motion fixes $\tilde R=-2/\ell^2_\mt{JT}$, \ie the brane geometry is locally AdS$_2$ everywhere with $\ell_\mt{B}= \ell_\mt{JT}$. Then the position $\s$ of the brane is fixed by eq.~\reef{curve1} and implicitly we assume that $\ell_\mt{JT}\gg L$, which ensures that $\s\ll L$ as in our previous discussions. The gravitational equation of motion coming from the variation of the metric becomes
\beq\label{fulleom}
-\nabla_{i}\nabla_{j}\Phi+\tilde{g}_{ij}\(\nabla^2\Phi-\frac{\Phi}{\ell^2_\mt{JT}}\)= 8\pi G_\brane\, \widetilde T^\mt{CFT}_{ij}
= -\frac{\Gbr}{ \Geff}\, \frac{1 }{\hat{\ell}_\mt{eff}^2}\,\tilde{g}_{ij} \,,
\eeq
where $\hat{\ell}_\mt{eff}$ is the effective curvature scale produced by $\ell_\mt{JT}$.
That is, in the case without JT gravity, we can combine eqs.~\reef{positionbrane}, \reef{curve1} and \reef{Newton2} to find
\beq\label{curve33}
\frac{L^2}{{\ell}_\mt{eff}^2}= f\!\(\frac{L^2}{\ell_\mt{B}^2}\)\equiv 2\(1-\sqrt{1-\frac{L^2}{\ell_\mt{B}^2}}\,\)  \,.
\eeq
We can understand this expression as the gravitational equation of motion coming from the two-dimensional action \reef{induct}, where a Taylor expansion of the right-hand side for $L/\ell_\mt{B}\ll1$ corresponds to varying the curvature terms and subsequently substituting $\tilde R_{ij}=-\frac{1}{\ell^2_\mt{B}}\,\tilde g_{ij}$, as in eq.~\reef{Ricky2}.  Now in the JT equation of motion \reef{fulleom}, the effective curvature scale $\hat{\ell}_\mt{eff}$ satisfies ${L^2}/{\hat{\ell}_\mt{eff}^2}= f\!\({L^2}/{\ell_\mt{JT}^2}\)$. We have indicated in eq.~\reef{fulleom} that the left-hand side corresponds to the stress tensor of the boundary CFT which lives on the brane. In the present arrangement,\footnote{In  more interesting scenarios, \eg with evaporating black holes as in \cite{Almheiri:2019psf,Almheiri:2019hni,Chen:2019uhq}, it is more appropriate to work directly with the CFT's stress tensor, rather than replacing these degrees of freedom by an effective gravity action after integrating out the CFT.} this takes a particularly simple form, with $T^\mt{CFT}_{ij}\propto \tilde{g}_{ij}$. Of course, this source term in eq.~\reef{fulleom} can be easily absorbed by shifting the dilaton, 
\beq\label{shift1}
\tilde\Phi\equiv \Phi- \frac{\Gbr}{ \Geff}\, \frac{\ell^2_\mt{JT} }{\hat{\ell}_\mt{eff}^2}\,,
\eeq
so that $\tilde\Phi$ satisfies the usual source-free equation studied in \eg \cite{Maldacena:2016upp}.

At this point, we observe that  the trace of eq.~\reef{fulleom} yields on the right-hand side,
\beq\label{almost}
\langle \big[\widetilde T^\mt{CFT}\big]^i{}_i \rangle = -\frac{L}{ 4\pi\Gbk}\, \frac{1 }{\hat{\ell}_\mt{eff}^2}=-\frac{L}{ 4\pi\Gbk}\, \frac{1 }{\ell_\mt{JT}^2}\(1+\frac14\,\frac{L^2}{\ell_\mt{JT}^2}+\frac18\,\frac{L^4}{\ell_\mt{JT}^4}+\cdots\)\,,
\eeq
where in the final expression, we are Taylor expanding $f(L^2/\ell_\mt{JT}^2)$ assuming $L^2/\ell_\mt{JT}^2\ll 1$, as above. Noting that $\tilde R=-2/\ell^2_\mt{JT}$ and comparing to eq.~\reef{trace},\footnote{Recall that the central charge here is twice that appearing in eq.~\reef{trace} because the brane supports two (weakly interacting) copies of the boundary CFT.} we see that the expected trace anomaly has recieved a infinite series of higher order corrections. We can interprete the latter as arising from the finite UV cutoff on the brane, recalling that $\tilde\delta\simeq L$ as discussed at the end of section \ref{indyaction}. 

