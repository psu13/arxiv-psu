% !TEX root = ../lifeonbrane3.tex

\subsection{Explicit Calculations}\label{sec:examples}

In this section, we explicitly evaluate the holographic EE and examine the transition between the two classes of RT surfaces. While we set up the calculations for general $d>2$, our explicit results are given for $d=3$ in which case the bulk spacetime locally has the geometry of AdS$_{4}$. We add some comments about $d=2$, and the addition of Jackiw-Teitelboim gravity \reef{JTee} on the brane, in the discussion section.

\subsubsection*{Setting up the calculation for general dimension}

In section \ref{sec:enzyme}, we reviewed two different coordinate systems in AdS$_{d+1}$. The AdS$_d$ foliation \reef{metric3a} was well suited to discuss the brane geometry, while the global coordinates are adapted to discuss the background geometry of the boundary CFT. However, our explicit calculations of the holographic EE are best performed in a new `cylindrical' coordinate system. In particular, following \cite{Krtous:2014pva}, we introduce cylindrical coordinates $P,\,\zeta$ where $\zeta$ specifies the position along the axis of the cylinder while $P$ measure the distance from the axis. These are related to the global coordinates in eq.~\reef{metric2s} by
\begin{align}\label{cylie}
\cosh r&=\sqrt{P^2+1}\,\cosh\zeta\,,\\
%\sinh r&=\sqrt{\frac{P^2+\tanh^2\zeta}{1-\tanh^2\zeta}}\,,\\
\tan\theta&=\frac{P}{\sqrt{1+P^2}}\,\frac{1}{\sinh\zeta}\,,
\end{align}
while the rest of the spherical angles remain unchanged. With this transformation, the metric becomes
\beq\label{cylindd}
ds^2=L^2\[ -(P^2+1)\cosh^2\zeta\, dt^2+\frac{dP^2}{1+P^2}+\left( 1+P^2 \right)d\zeta^2+P^2\,d\Omega_{d-2}^2\]\,.
\eeq
The range of these coordinates is $P\in (0,\infty)$ and $\zeta\in(-\infty,\infty)$. The conformal boundary is reached with $P\to \infty$ (or $\zeta\to\pm\infty$ with fixed $P$).  The upper ($0\le\theta\le\pi/2$) and lower ($\pi/2\le\theta\le\pi$) hemispheres are mapped to the upper ($\zeta\ge0$) and lower ($\zeta\le0$) halves of the cylindrical system. The conformal defect is positioned at $\zeta=0$. As noted above, the RT surfaces will be restricted to a constant time surface and hence the convenience of the cylindrical coordinates becomes evident, \ie  $\zeta$ becomes an extra Killing coordinate in the corresponding spatial geometry.


A few more technical details are needed  for our calculations:
in cylindrical coordinates \reef{cylindd}, the boundary entangling surface corresponds to the two circles $\zeta=\pm\zeb$, where
\beq\label{zeta0}
\sinh\zeb=\tan\thb\,,
\eeq
seen in the limit $P\to\infty$ of the second line in eq.~\reef{cylie}. Using the AdS foliation of eq.~\reef{metric3a}, the position of the brane was $z=\s$. Using eq.~\reef{eq:foobar}, the brane position can be specified in cylindrical coordinates \reef{cylindd} according to 
\beq\label{eq:foobar2}
 \(1+P^2\)\sinh^2\!\zeta=\frac{L^2}{\s^2}\(1-\frac{\s^2}{4L^2}\)^2\,.
\eeq
Recall that the brane intersects the asymptotic boundary at the position of the conformal defect, \ie at $\theta=\pi/2$ with $r\to\infty$, which corresponds to $\zeta= 0$ with $P\to \infty$ in cylindrical coordinates. Further recall that RT surface areas are UV divergent since they extend to the asymptotic boundaries.  Hence we introduced a UV regulator surface at $r=r_\mt{UV}$, which in cylindrical coordinates becomes
\beq\label{regular}
(P^2+\tanh^2\zeta)\cosh^2\zeta=\sinh^2\! r_\mt{UV}\,.
\eeq
We will be mainly interested in comparing the areas of different surfaces for fixed $\zeb$, as discussed above. Since the UV divergent terms only depend of the geometry of the entangling surface, they will cancel in the difference of the two areas. Hence, we can then safely take the UV cutoff to infinity.

As noted, the RT surfaces all lie in a fixed time slice and thus we only need consider configurations with cylindrical symmetry (\ie rotational symmetry on the $S^{d-2}$). Hence it is convenient to use the cylindrical coordinates \reef{cylindd} and  parametrize the profile of the bulk surfaces as $\zeta=\zeta(P)$. The bulk contribution to the holographic EE is given by
\begin{align}\label{area}
S_\mt{bulk}= \frac{L^{d-1}\, \Omega_{d-2}}{2\,\Gbk} \int dP P^{d-2} \sqrt{\frac{1}{1+P^2}+(1+P^2)\,\zeta'^2}
\end{align}
where again $\Omega_{d-2}$ is the area of the unit $(d-2)$-sphere -- see footnote \ref{footsphere}. As in eq.~\reef{area0}, an overall factor of 2 is included here to account for the reflection symmetry of the profile $\zeta(P)$ about the brane. Since this expression does not contain an explicit $\zeta$ dependence, it is straightforward to derive
\begin{align}\label{zetap}
\zeta'(P)=\pm \frac{1}{1+P^2}\,\sqrt{\frac{P_0^{2(d-2)}\(1+P_0^2\)}{P^{2(d-2)}\(1+P^2\)-P_0^{2(d-2)}\(1+P_0^2\)}}
\end{align}
where the two branches correspond to two identical surfaces related by a reflection with respect to $\zeta=0$. $P_0$ corresponds to the turning point, where the surface makes its closest approach to the symmetry axis.

We now discuss the disconnected phase described at the beginning of this section. It corresponds to the `trivial' solution with $P_0=0$. We find $\zeta(P)=\pm\zeb$, which in cylindrical coordinates looks simply as a pair of disks anchored at the boundary entangling surface. Substituting $\zeta'=0$ into eq.~\reef{area}, the area of the two discs can be integrated up to some cutoff radius $P_\mt{UV}$, and the corresponding holographic EE is
\begin{align}\label{A_disc}
S_\mt{disc}=\frac{L^{d-1}\, \Omega_{d-2}}{2(d-1)\,\Gbk} \ \puv^{d-1}\, {}_{2}F_1\left[ \frac{1}{2},\frac{d-1}{2},\frac{d+1}{2},-\puv^2 \right]\,.
\end{align}
In this case, the entanglement wedge corresponds to two identical disconnected pieces contained between each component of the RT surface and the asymptotic boundary, \ie the regions $\zeta\ge+\zeb$ and $\zeta\le-\zeb$, as sketched in the upper panel of figure \ref{fig:RTPhases}. 
%\dn{changed non-existent reference to figure \ref{fig:RTPhases}. Note that there are no a) / b) labels in the figure though. Do you just want to refer to \ref{fig:RTPhases}, or should we add a new figure to the beginning of section 4 which shows the two possible RT surfaces in the style of \ref{fig:cutoffs}?}
%
%\begin{figure}[h]
%\begin{center}
%\includegraphics[scale=.5]{images/EWs}
%\caption{Sketch of fixed time slices of our setup, showing the two possible configurations. The shaded region corresponds to the entanglement wedge.}
%\label{fig:EWs}
%\end{center}
%\end{figure}


\begin{figure}
	\def\svgwidth{0.8\linewidth}
	\centering{
		\input{RTPhases2.pdf_tex}
		\caption{Sketch of fixed time slices of our symmetric setup, showing the two possible configurations. The shaded red region corresponds to the entanglement wedge. The connected solution contains an island on the brane, where gravity is dynamical.}
		\label{fig:RTPhases}
	}
\end{figure}

The connected phase corresponds to $P_0>0$, which leads to a cylindrical RT surface. Integrating eq.~\reef{zetap} yields a family of bulk surfaces, which are symmetric about the brane and which are anchored on the asymptotic boundary at $\zeta=\pm\zeb$. Recalling the discussion below eq.~\reef{area0}, we observe that in this configuration, $P_0$ is the second integration constant which must be tuned in order to satisfy the appropriate boundary condition \reef{ortho1} at the brane, see the lower panel of figure \ref{fig:RTPhases}.

Before we calculate the entropy in the most general setting, let us consider the case of a zero-tension brane with $1/\Gbr=0$, \ie empty AdS$_{d+1}$. In this case, the brane is positioned at $\s=2L$ or simply, $\zeta=0$. Now, the `plus' branch of eq.~\eqref{zetap} can be integrated to produce a profile extending from $P=\puv$ at $\zeta=+\zeb$ to the maximal depth $P=P_0$ at some $\zeta=\zeta_0(\zeb,P_0)<\zeb$. Since eq.~\reef{ortho1} indicates that the RT surface must intersect the brane orthogonally, we must tune $P_0$ (with fixed $\zeb$) such that $\zeta_0=0$, \ie the RT surface reaches its maximal depth at the brane position. Now, substituting eq.~\reef{zetap} into eq.~\reef{area}, the holographic EE (for empty AdS$_{d+1}$) becomes
\begin{align}\label{A_conn}
S_\mt{conn}(T_o=0)
%&=2L^{d-3}S_{d-2} \int_{P_0}^P dp \frac{p^{d-2}}{\sqrt{1+p^2}} \sqrt{1+\frac{P_0^{2(d-2)}+P_0^{2(d-1)}}{p^{2(d-2)}+p^{2(d-1)}-P_0^{2(d-2)}-P_0^{2(d-1)}}}\\
&=\frac{L^{d-1}\, \Omega_{d-2}}{2\,\Gbk} \int_{P_0}^{\puv}\!\!\! dP\,  \frac{P^{2(d-2)}}{\sqrt{P^{2(d-2)}(1+P^2)-P_0^{2(d-2)}(1+P_0^2)}}\,.
\end{align}

In the general case, this exercise is slightly more complicated for the case of interest with a finite-tension DGP brane at some $z=\s\ll L$, and the geometry of the corresponding RT surface is illustrated in the lower panel of figure \ref{fig:RTPhases}. The RT surface is again symmetric about the brane and so as above, we focus on the portion starting at $\zeta=+\zeb$ at the asymptotic boundary (\ie at $P=\puv$). As before, the `plus' branch of eq.~\eqref{zetap} produces a surface reaching its maximal depth $P=P_0$ at some $\zeta=\zeta_0(\zeb,P_0)<\zeb$.\footnote{In fact, $\zeta_0(\zeb,P_0)$ is precisely the same function introduced above, since the turning point of the RT surfaces are completely independent of the brane properties.}  Now one continues from this point using the `minus' branch of eq.~\reef{zetap}, which then meets the brane as some $P=P_\mt{B}(\zeb,P_0)$ and $\zeta=\zeta_\mt{B}(\zeb,P_0)$.\footnote{Of course, $P_\mt{B}$ and $\zeta_\mt{B}$ are related as in eq.~\reef{eq:foobar2}.} One would again tune $P_0$ (for fixed $\zeb$) to ensure the appropriate boundary condition \reef{ortho1} is satisfied at the brane.
The bulk contribution to the holographic EE then becomes
\beqa
S_\mt{conn}(T_o>0)&=&\frac{L^{d-1}\, \Omega_{d-2}}{2\,\Gbk}\[ \int_{P_0}^{\puv}\!\!\!  dP\,  \frac{P^{2(d-2)}}{\sqrt{P^{2(d-2)}(1+P^2)-P_0^{2(d-2)}(1+P_0^2)}}\right.
\labell{Acon2}\\
&&\qquad\qquad\qquad+\left. \int_{P_0}^{\pb}\!\!  dP\,  \frac{P^{2(d-2)}}{\sqrt{P^{2(d-2)}(1+P^2)-P_0^{2(d-2)}(1+P_0^2)}}\]\,.
\nonumber
\eeqa
Of course, if there is no gravitational term on the brane (\eg as in eq.~\reef{newbran}), then this expression yields the entire generalized entropy \reef{eq:sgen_intro} for the connected phase. Now
rather than explicitly examining the brane boundary condition \reef{ortho1} in cylindrical coordinates, we will simply evaluate the generalized entropy and find the minimum numerically in the following. Hence to proceed further we will have to choose a specific value for the boundary dimension $d$.

\subsubsection*{Explicit results for $d=3$}
In this section, we consider the above discussion for $d=3$, in which case the boundary geometry becomes $\Rbb\times S^{2}$, the bulk spacetime is locally AdS$_4$, and the branes have an AdS$_3$ geometry. We will also consider supplementing the the four-dimensional bulk action \reef{act2} with a Gauss-Bonnet term, %\iar{I suggest including this term with an extra $L^2/\Gbk$, so that 1) it is of the same 'order' as the area term when $\lgb\sim 1$, and 2) we can remove this global prefactor in e.g. fig \ref{figdeltaA} }
 \beq
I_\mt{top} = \frac{\lgb}{16\pi^2}  \int \mathrm{d}^4x \, \sqrt{-g}\, \left[ R_{abcd}R^{abcd}-4\,R_{ab}R^{ab}+R^2 \right]\,.
 \labell{top2}
 \eeq
Note that we have ignored the necessary boundary terms which ensure that this interaction is proportional to the Euler density, \eg see \cite{Myers:1987yn}. Although this curvature-squared term does not effect the bulk equations of motion, it will contribute to the generalized entropy \cite{Dong:2013qoa,Hung:2011xb}\footnote{One may worry that the topological nature of $I_\mt{top}$ undercuts the usual derivations of the generalized entropy. However, individually the three terms in eq.~\reef{top2} are dynamical and one can apply the results of \cite{Dong:2013qoa} for each separately and then take the sum of the corresponding contributions to the holographic entropy, which one finds matches the result in eq.~\reef{Euler3}.}
 \beq
S_\mt{JM} = \frac{\lgb}{4\pi} \int_{\Sigma_\xR} d^2x\sqrt{h}\,\mR +\frac{\lgb}{2\pi} \int_{\partial \Sigma_\xR}
dx\sqrt{h}\,\mK_g \,,
 \labell{Euler3}
 \eeq
where $\mR$ denotes the Ricci scalar for the intrinsic geometry on the RT surface $\Sigma_\xR$. Similarly, $\mK_g$ denotes the geodesic curvature of the boundary $\partial \Sigma_\xR$. Of course, eq.~\reef{Euler3} gives a topological contribution proportional to the Euler character of the two-dimensional extremal surfaces\footnote{The normalization is chosen so that for an RT surface with two-sphere topology, $S_\mt{JM} = 2 \lgb$.} and so their geometry remains unaffected by this term. However, in the following, this additional contribution will  provide an extra parameter which allows us to adjust the transition between the connected and disconnected phases.

For $d=3$, some analytic expressions for the extremal surfaces can be obtained \cite{Krtous:2014pva}. For example,
integrating eq.~\eqref{zetap} yields the following profile for the extremal surface in empty AdS$_4$ \cite{Krtous:2014pva}
\beqa
&&\zeta_\pm(P;P_0,\zeta_0)=\zeta_0\pm \frac{P_0}{\sqrt{(1+P_0^2)(1+2P_0^2)}} \labell{zetasol} \\
&&\times\left[ (1+P_0^2)\, F\!\left( \mbox{Arcos} \frac{P_0}{P},\sqrt{\frac{1+P_0^2}{1+2P_0^2}} \right)-P_0^2\, \Pi\!\left( \mbox{Arccos}\frac{P_0}{P},\frac{1}{1+P_0^2},\sqrt{\frac{1+P_0^2}{1+2P_0^2}} \right) \right]
\nonumber
\eeqa
where $F$ and $\Pi$ correspond to incomplete elliptic integrals of the first and third kind, respectively.\footnote{Our notation for the elliptic integrals matches that in \cite{Gradshteyn:1702455}, section 8.1.} Again, the $\pm$ branches correspond to the two portions of the surface, symmetric with respect to $\zeta_0=0$. Of course, we need to know where this surface is anchored at the boundary. Hence we define
\beqa
\zeta_\infty&\equiv&\zeta_+(P\to \infty;P_0,\zeta_0)-\zeta_0
\labell{alphadog}\\
&=&\frac{P_0\left[ (1+P_0^2)\,K\!\left( \sqrt{ \frac{1+P_0^2}{1+2P_0^2} }\right) - P_0^2\, \Pi\!\left( \frac{1}{1+P_0^2},\sqrt{\frac{1+P_0^2}{1+2P_0^2}} \right) \right]}{\sqrt{(1+P_0^2)(1+2P_0^2)}}
\nonumber
\eeqa
and the surface reaches the asymptotic boundary at $\zeta_\pm(P\to \infty)=\zeta_0\pm \zeta_\infty$. Hence the two components of the entangling surface in the boundary theory are separated by $2\zeta_\infty$, in the cylindrical coordinates.

Figure \ref{figzetainfty} plots $\zeta_\infty$ as a function of $P_0$. The maximum is obtained at $P_0=P_0^{\mt{crit}}\approx 0.51633$ with $\zeta_\infty=\zeta_\infty^{\mt{crit}}\approx 0.5011$. An interesting observation in \cite{Krtous:2014pva} was that, for $P_0<P_0^{\mt{crit}}$, there exist \textit{two} values of $P_0$ with the same $\zeta_\infty$. That is, if the two components of the entangling surface are sufficiently `close' on the boundary sphere, there actually exist \textit{two} extremal RT surfaces that connect them in the bulk. However, one branch (with the smaller value of $P_0$) is always subdominant, and therefore will be of little interest in our analysis. On the other hand, if the separation of the two entangling spheres  is larger than the critical value $2\zeta_\infty^{\text{max}}$ (in cylindrical coordinates), there is no connected extremal surface that joins them.
%
%\begin{figure}[h]
%\begin{center}
%\includegraphics[scale=0.4]{images/zetainfty}
%\caption{Plot of the `height' of the RT surface in cylindrical coordinates, as a function of the turning point $P_0$ characterising the surface. For $\zeta_\infty<\zeta_\infty^{\mt{crit}}$, there are two minimal surfaces anchored at the same regions; otherwise there exists none. }
%\label{figzetainfty}
%\end{center}
%\end{figure}

\begin{figure}[h]
	\def\svgwidth{0.5\linewidth}
	\centering{
		\input{rtHeight.pdf_tex}
		\caption{Plot of the `height' of the RT surface in cylindrical coordinates, as a function of the turning point $P_0$ characterising the surface. For $\zeta_\infty<\zeta_\infty^{\mt{crit}}$, there are two minimal surfaces anchored at the same regions; otherwise there exists none. 
		}
		\label{figzetainfty}
	}
\end{figure}

Let us now describe the solutions corresponding to different values of the tension and DGP term:
%\dn{Should we mention somewhere around here that $T_o = 0$ is not einstein gravity on the brane?}\iar{Not needed here, since T_o=0 means the brane doesn't exist}.
\paragraph{a) $T_o=0;\ 1/\Gbr=0$\ :} First we consider the holographic EE in empty AdS$_4$ as a lead-in to the case with a brane. As emphasized above, the area of these surfaces is divergent, and so one introduces a UV regulator surface,  integrating of the area from $P_0$ to some $\puv\gg1$ \cite{Krtous:2014pva}. For the disconnected solution (\ie a pair of disks), eq.~\eqref{A_disc} with $d=3$ gives
\begin{align}\label{Sdisc}
S_\mt{disc}(\puv)
=& \frac{\pi L^2}{\Gbk}\,\left( \sqrt{1+\puv^2}-1 \right)+2 \lgb
\\
=& \frac{A(S^1_{\puv})}{4G_\mt{eff}}
- \frac{\pi L^2}{G_\bulk} + 2\lgb
+ \Ocal(\puv^{-1})\,.
\label{eq:lonely}
\end{align}
where
\beq\label{sample}
\frac{A(S^1_P)}{4G_\mt{eff}}=\frac{\pi L^2}{\Gbk}\,P\,,
\eeq
is the length of $S^1_P$, a circle with radius $P$, and we used eq.~\reef{Newton2} to write $\frac{1}{G_\mt{eff}}=\frac{2\,L}{\Gbk}$.
We have included in eq.~\eqref{Sdisc} the topological contribution in eq.~\reef{Euler3}. On the other hand, for the connected surfaces the area formula \eqref{A_conn} yields
\begin{align}\label{eq:dietCoke}
\begin{split}
\MoveEqLeft[3]
S_\mt{conn}(\puv,P_0)
\\
=& \frac{\pi L^2}{\Gbk}\,\frac{P_0^2}{\sqrt{1+2P_0^2}}\, \Pi\!\left( \mbox{Arccos}\frac{P_0}{\puv},1,\sqrt{\frac{1+P_0^2}{1+2P_0^2}} \right)
\end{split}
\\
\begin{split}
=& \frac{A(S^1_{\puv})}{4G_\mt{eff}}
+ \frac{\pi L^2}{G_\bulk}\left[
-\sqrt{1+2P_0^2} E\left(\sqrt{\frac{1+P_0^2}{1+2P_0^2}}\right)
+ \frac{P_0^2}{\sqrt{1+2P_0^2}} K\left(\sqrt{\frac{1+P_0^2}{1+2P_0^2}}\right)
\right]
\\
&+ \Ocal(\puv^{-1})\,,
\end{split}
\label{eq:friendless}
\end{align}
where $E$ is the elliptic integral of the second kind. We emphasize that this result only applies for  vanishing $T_o$ and vanishing $1/\Gbr$, \ie  for the AdS$_4$ vacuum. Note that the Euler character of the cylindrical RT surface is zero and hence there is no contribution proportional to $\lgb$. As expected, the divergence in the $\puv\to\infty$ limit matches for the areas of the connected and disconnected surfaces. Hence we can safely take the limit when considering the difference
\begin{align}\label{dAP01}
\Delta S(P_0)&=\lim_{\puv\to \infty}\left( S_\mt{conn}(\puv,P_0)-S_\mt{disc} (\puv)\right)\,,
\end{align}
given by the difference in $O\left( \left( P_0/\puv \right)^0 \right)$ terms in eq.~\eqref{eq:friendless} and eq.~\eqref{eq:lonely}.
A plot of $\Delta S$ is shown in figure \ref{figdeltaA}. When $\Delta S>0$, the disconnected RT surface is the dominant saddle, while for $\Delta S<0$, the connected solution dominates. Notice that with a larger (positive) topolgical coupling $\lgb$, the entropy in eq.~\reef{Sdisc} increases while eq.~\reef{eq:dietCoke} is unaffected, and hence the range of the disconnected phase is decreased in figure \ref{figdeltaA}.

%\footnote{Of course, $\Delta S(P_0)$ is closely related to the mutual information. \rcm{words:} Mutual information between two subsystems $A$ and $B$ is defined via $I=S_{A\cup B}-S_A-S_B$. Notice that whenever $\Delta S>0$, phase $I$ dominates and therefore the mutual information vanishes, since $S_{A\cup B}=S_A+S_B$.}
%
%\begin{figure}[h]
%\begin{center}
%\includegraphics[scale=0.3]{images/dAP0}
%\caption{Renormalised entropy from eq. \eqref{dAP01}. The connected (disconnected) surface dominates when $\Delta S<0\,(\Delta S>0)$. When $\lgb$ to becomes very large, $\lgb\sim c_T$, the connected solution becomes favoured. }
%\label{figdeltaA}
%\end{center}
%\end{figure}

\begin{figure}[th]
	\def\svgwidth{0.8\linewidth}
	\centering{
		\input{figdeltaA2.pdf_tex}
		\caption{Renormalised entropy from eq. \eqref{dAP01}. The connected (disconnected) surface dominates when $\Delta S<0\,(\Delta S>0)$. When $\lgb$ becomes very large, $\lgb\sim c_T$, the connected solution becomes favoured.}
		\label{figdeltaA}
	}
\end{figure}

\paragraph{b) $T_o\ne 0;\ 1/\Gbr=0$\ :} The next step is to introduce the brane, however, we do not include a gravitational term in the brane action yet, \ie $1/\Gbr=0$. In this case, we saw  in eq.~\reef{Acon2} that there is an additional contribution as the RT surface extends from the maximal depth $P_0$ back out to meet the brane at $P_\mt{B}$. Both contributions in eq.~\reef{Acon2} take the same form except for the limits of integration, hence the $d=3$ result in eq.~\reef{eq:dietCoke} is replaced by
\beqa
S_\mt{conn}(\puv,P_0)&=& \frac{\pi L^2}{\Gbk}\,\frac{P_0^2}{\sqrt{1+2P_0^2}}\,\[ \Pi\!\left( \mbox{Arccos}\frac{P_0}{\puv},1,\sqrt{\frac{1+P_0^2}{1+2P_0^2}} \right)\right.
\labell{CokeZero}\\
&& \left.\qquad\qquad+\Pi\!\left( \mbox{Arccos}\frac{P_0}{\pb},1,\sqrt{\frac{1+P_0^2}{1+2P_0^2}} \right)\]\,.
\nonumber
\eeqa


Of course, the entropy for the disconnected phase remains the same as in eq.~\reef{Sdisc} and we can consider the difference of the generalized entropy evaluated on the connected and disconnected extremal surfaces, as in eq.~\reef{dAP01}. Just as we saw a leading divergent contribution in eq.~\reef{eq:dietCoke} for $\puv\to\infty$, we expect that eq.~\reef{CokeZero} will contain an analogous large contribution for $\pb\gg P_0$. However, this term will not be cancelled in $\Delta S$. In fact, in this regime, we can expand the difference as
\begin{align}\label{radishes}
\begin{split}
\MoveEqLeft[2]\Delta S(P_0)
\\
=& \frac{A(\sigma_\xR)}{4G_\mt{eff}}
+ \frac{\pi L^2}{G_\bulk}\left[
1-2\sqrt{1+2P_0^2} E\left(\sqrt{\frac{1+P_0^2}{1+2P_0^2}}\right)
+ \frac{2P_0^2}{\sqrt{1+2P_0^2}} K\left(\sqrt{\frac{1+P_0^2}{1+2P_0^2}}\right)
\right]
\\
&-2 \lgb
+ \Ocal(\pb^{-1})\,.
\end{split}
\end{align}
Here, the intersection $\sigma_\xR$ of the RT surface and the brane is a circle of radius $\pb$ with area $A(\sigma_\xR)=2\pi L\,\pb$ given by eq.~\eqref{cylindd}. The fact that the leading term can be expressed as the gravitational entropy for the induced gravity action \reef{act3} on the brane is in perfect agreement with our discussion in the previous section. As we will see below, the finite terms will play a role once we turn on the DGP term, allowing for the appearance of a different island on the brane. 

From the above expansion, we see that there is a strong penalty for having a large $\sigma_\xR$ in the connected phase. From the brane perspective, the gravitational entropy results in a large penalty against forming an island on the brane. In fact, generally we expect that $\Delta S>0$ in this regime and hence the disconnected solution provides the dominant saddle point. However, if we tune the topological coupling $\lgb$ to be large\footnote{We note that this requires $\lgb\sim {L^2}/{\Gbk}\sim \cT$, the central charge of the boundary CFT -- see further discussion in section \reef{sec:discussion}.} (and positive), this contribution can compensate for the leading gravitational entropy term, at least for $\sigma_\xR$ up to a certain size.

On the other hand, we must note that $\pb$  is not an independent parameter. Rather it is implicitly determined by $\zeb$ and the brane tension $T_o$, as well as the value of $P_0$ that minimises the area functional in eq. \eqref{radishes}. $\pb$ can be determined in the following way (see figure \ref{fig:RTPhases}). One begins by solving for $\zeta_0$ using
$\zeta_0+\zeta_\infty(P_0)=\zeb$ where $\zeta_\infty(P_0)$ is given in eq.~\reef{alphadog}. Then one finds `sample' values of $\pb,\zeta_\mt{B}$ where the extremal surface meets the brane by combining eqs.~\reef{eq:foobar2} and \reef{zetasol} and simultaneously solving
\beqa
 \(1+\pb^2\)\sinh^2\!\zeta_\mt{B}&=&\frac{L^2}{\s^2}\(1-\frac{\s^2}{4L^2}\)^2\,,\nonumber\\
 \zeta_-(\pb;P_0,\zeta_0)&=&\zeta_\mt{B}\,.
 \label{solver3}
\eeqa
This yields $P_B$ as a function of $P_0,\zeta_{\mt{CFT}}$ and $T_o$, and substituting $\pb$ into eq.~\reef{CokeZero} gives the area of the associated extremal surface. Below, we perform this calculation numerically. However, we have not yet considered the boundary conditions \reef{ortho1} in this analysis. Rather than explicitly examining the latter, we simply evaluate the area (or rather the difference $\Delta S$) over the range of possible $P_0$ (with fixed $\zeb,T_o$), as shown in figure \ref{fig:dS0}a. The correct RT surfaces are then identified as the minima in these plots. Further, the examples in the figure illustrate that without the topological contribution, $\Delta S>0$ for all minima and so the disconnected phase dominates, as generally expected. That is, no quantum extremal islands form on the brane in this case. However, as shown in figure \ref{fig:dS0}b, we see that with a sufficiently large topological coupling $\lgb$ one can achieve $\Delta S<0$, where a first order transition leads to the formation of an island. 

Although the above recipe is valid for arbitrary brane tensions, in the limit of very large tension we can approximate the solution analytically. Since, as stated above, the leading contribution to the entropy \eqref{sample} scales as $A(\sigma_\xR)\sim \pb$, the RT surface corresponds to that which has the minimal value of $\pb$. Moreover, since the function $\zeta_{\mt{B}}(P)$ defining embedding of the brane in \eqref{solver3} is monotonically decreasing with $P$, the surface must maximise its hight $\zeta_\infty(P_0)$, which is achieved for $P_0=P_0^{\mt{crit}}$, by definition (see discussion around figure \ref{figzetainfty}). This can be readily checked in figure \ref{fig:dS0}a, where the curves attain a minimum around $\mbox{arctan}(P_0^{\mt{crit}})\approx 0.47$, with a small correction due to the finite terms in \eqref{radishes}, which becomes smaller and smaller as we increase the tension. We shall refer to this solution with $P_0\approx P_0^{\mt{crit}}$ as the \textit{small island}, in order to distinguish it from a second island appearing below which corresponds to a circle with a larger radius.  


%\begin{figure}[h]
%\begin{center}
%\includegraphics[scale=0.3]{images/setup}
%\caption{A half of the geometric setup in cylindrical coordinates, with the radial direction compactified to a finite coordinate value. The `tube' (orange) corresponds to the connected RT surface $\sigma$, anchored at conformal infinity and intersecting the brane (green) at the codim-$3$ surface $\del\sigma_{brane}$ (red). The disk (blue) corresponds to the trivial disconnected phase. A second copy of this is glued along the brane, as depicted in figure \ref{fig:brane2}.  }
%\label{figsetup}
%\end{center}
%\end{figure}

%The total renormalised area of the connected phase is given in eq. \eqref{Acon2}. Notice that the intersection point $\pb=\pb(\s,\zeb,P_0)$ depends on the location of the brane via $\s$, together with the anchoring hight $\zeb$ and the turning point $P_0$ the surface. To find the entanglement entropy, we fix $\s$ and $\zeb$, and vary $P_0$ in order to find the minimum of the area
%\begin{align}\label{dAtilde}
%\Delta \tilde A=S_{\mt{conn}}(T_0> 0)-S_{\mt{disc}}
%\end{align}
%where the first and second term are given by eqs. \eqref{Acon2} and \eqref{Sdisc} respectively. We perform this computation numerically. In figure \ref{figClassAreas} where we plot the area of minimal surfaces that intersect the brane, for fixed anchoring point $\zeb$ as we vary the turning point $P_0$, for different values of $\s$, the parameter controlling the tension of the brane. For large tensions $L/\s\gg 1$, it can be checked that the minimum is always attained for $P_0=\bar P_0\approx 0.44$.


%\begin{figure}
%\begin{center}
%  \begin{subfigure}[b]{0.4\textwidth}
%    \includegraphics[width=\textwidth]{images/ClassAreas}
%    \caption{}
%  \end{subfigure}
%  %
%  \begin{subfigure}[b]{0.4\textwidth}
%    \includegraphics[width=\textwidth]{images/GB_u}
%    \caption{}
%  \end{subfigure}
%  \end{center}
%  \caption{a) Renormalised area from eq. \eqref{radishes} of connected RT surfaces, anchored at $\zeb=0$, with $\lgb=0$. b) Critical value of $\lgb$ such that $\mbox{min} \Delta S<0$.}
%  \label{fig:dS}
%\end{figure}

\begin{figure}[h]
	\def\svgwidth{0.9\linewidth}
	\centering{
		\input{dS2.pdf_tex}
		\caption{Panel a. illustrates the renormalised area from eq. \eqref{radishes} of connected RT surfaces, anchored at $\zeb=0$, with $\lgb=0$. Panel b. is a plot of the critical value of $\lgb$ such that $\mbox{min}( \Delta S)<0$.
		}
		\label{fig:dS0}
	}
\end{figure}

\paragraph{c) $T_o\ne 0;\ 1/\Gbr\ne 0$\ :} Finally, we examine the holographic EE in the presence of a DGP brane. The only difference in this analysis is the additional contribution coming at the intersection of the RT surface with the brane in eq.~\reef{eq:sad}.
In the present setting, this means that we add the following,
\beq\label{sample2}
S_\mt{brane}=\frac{A(\sigma_\xR)}{4\Gbr}=\frac{\pi L}{2\Gbr}\,\pb\,,%\frac{\pi L^2}{\Gbr}\,\pb\,,
\eeq
to the bulk contribution in eq.~\reef{CokeZero}. In fact, the  expansion of $\Delta S$ for $\pb\gg P_0$ takes precisely the same form as in eq.~\reef{radishes}. The only difference is that the induced Newton's constant on the brane is now given by eq.~\reef{Newton33}, \ie $\frac{1}{G_\mt{eff}}=\frac{2\,L}{\Gbk}+\frac{1}{\Gbr}$.

Generally, we might think of $1/\Gbr$ as a positive quantity, and so the DGP contribution \reef{sample2} would simply increase the penalty for having a large $\sigma_\xR$ in the connected phase, and enhance the dominance of the disconnected phase. However, there is no apriori reason why we should not also consider a negative gravitational coupling on the brane,\footnote{For example, integrating out quantum fields on the brane could produce either a positive or negative shift in Newton's constant. In particular, it can be negative for gauge fields or nonminimally coupled scalar fields, as discussed in the context of EE in \cite{Larsen:1995ax,Kabat:1995eq} -- see further discussion in section \ref{sec:discussion} and appendix \ref{bubble}.} in which case the DGP term serves as another mechanism to reduce the penalty for forming an island on the brane. It is this scenario that we will examine further here -- as well as in appendix \ref{bubble}.

It will prove convenient to work with the ratio $\lamb$ introduced in eq.~\reef{newdefs}. Let us recall what parameters are in play. The tension of the brane is controlled by $\s$, which we keep small but finite. The dimensionless ratio between the bulk and brane gravitational constants is controlled by $\lamb$. As discussed above, interesting things happen when $\lamb<0$, which is when $\Gbr<0$ while $\Gbk>0$.

\begin{figure}[h]
	\def\svgwidth{\linewidth}
	\centering{
		\input{IslandPlot.pdf_tex}
		\caption{ Panel a.: Generalised (renormalised) area from eq. \eqref{radishes} as function of $P_0$, for different values of the DGP coupling $\lamb$. Notice the appearance of a `large' island when $\lamb$ approaches $-1$, due to the partial cancellation of the induced and DGP area terms. Panel b.: Phase diagram: the black lines correspond to first order phase transitions, while the blue one at $\lamb=-1$ indicates the region where gravity becomes unstable. Both plots are done for fixed $L/\s=100$. 
		}
		\label{fig:dS}
	}
\end{figure}
%\begin{figure}
%\begin{center}
%  \begin{subfigure}[b]{0.5\textwidth}
%    \includegraphics[width=\textwidth]{images/A_gen}
%    \caption{}
%  \end{subfigure}
%  %
%  \begin{subfigure}[b]{0.4\textwidth}
%    \includegraphics[width=\textwidth]{images/GB_B}
%    \caption{}
%  \end{subfigure}
%  \end{center}
%  \caption{ a) Generalised (renormalised) area from eq. \eqref{radishes} as function of $P_0$, for different values of the DGP coupling $\lamb$. Notice the appearance of a `large' island when $\lamb$ approaches $-1$, due to the partial cancellation of the induced and DGP area terms. b) Phase diagram: the black lines correspond to first order phase transitions, while the blue one at $\lamb=-1$ indicates the region where gravity becomes unstable. Both plots are done for fixed $L/\s=100$. }
%  \label{fig:dS}
%\end{figure}
Using the same approach described above, we can explore the transition between the connected and disconnected phases numerically. In figure \ref{fig:dS}a, we plot $\Delta S$ as function of $P_0$ for a fixed $\zeb=0.095,L/\s=100$ and $\lgb=0$, for different values of $\lamb$. These plots are analogous to those presented in figure \ref{fig:dS}a where $\lamb=0$ (but $L/\s$ is varied). Again, these plots are made in lieu of a detailed examination of the boundary conditions where the RT surfaces meet the brane, rather the correct boundary conditions \reef{ortho1} will be achieved where $P_0$ is tuned to produced an minimum in these plots. For small $\lamb$ the curves show a single minimum but $\Delta S>0$, indicating that the disconnected solution dominates in this case. As $\lamb$ becomes more negative, the curves are pulled down and eventually $\Delta S$ enters the negative region so that the connected solution becomes the dominant saddle point. This behaviour is as expected but we note that $\lamb$ is very close to $-1$ in this regime, which according to eq.~\reef{newdefs} means there is almost a complete cancelation between the induced gravitation coupling $1/G_\mt{RS}$ and the DGP term $1/\Gbr$. Of course, this near cancellation is alleviated by turning on the topological coupling $\lgb$, as shown in figure \ref{fig:dS}b. 

Another interesting feature shown in figure \ref{fig:dS}a is the appearance of a second minimum in the curves. This second solution occurs at a larger value of $P_0$ and also of $P_B$, and corresponds to a larger circle $\sigma_\xR$ on the brane, and therefore we refer to it as a \textit{large island}. The existence of this second island is due to the finite terms in \eqref{radishes}. Indeed, these terms are essentially what is plotted in figure \ref{figdeltaA}, and they are unbounded from below for large $P_0$. Therefore, when $\lamb$ becomes sufficiently negative as to produce a significant cancellation between the induced and DGP gravitational entropies, there is a new competition, now between $A(\sigma_\xR)/4G_\mt{eff}$ and the finite terms, producing the large island. As $\lamb\to-1$, the minimum rolls down to infinity ($P\to \infty,\Delta S_{\mt{gen}}\to-\infty$), indicating an instability at this point, which we explore further in appendix \ref{bubble}.

Figure \ref{fig:dS}b summarises the phase diagram of the system, for a fixed value of the tension $L/\s=100$, as we vary both the DGP coupling $\lamb$ and the topological coupling $\lambda_{\mt{GB}}$. The lines between no/small/large islands correspond to first order phase transitions, while the blue line at $\lamb$ indicates the region where the theory becomes unstable. 


