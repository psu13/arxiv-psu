% !TEX root = ../lifeonbrane3.tex
In this section we relate the brane setup to the calculation of entanglement entropies for regions on the boundary CFT whose RT surface falls deep into the bulk and connects to the brane.

In AdS/CFT, branes such as the ones describes above play an essential role in the description of Boundary CFT (BCFT). If the CFT itself is living on a manifold with a boundary, the holographic prescription states that the dual geometry consists also of a manifold with a boundary - an AdS-like spacetime that ends on a brane, which is itself anchored at the boundary of the CFT (an `end-of-the-world brane') \cite{ads/bcft}.

In general, the holographic recipe shows that the entanglement entropy of a subregion $A$ of the field theory is given by an extremal surface $\sigma$ that is anchored at the asymptotic boundary at $A$, that extends into the bulk and remains homologous to $A$. In the case of a BCFT, the new ingredient is that the R-T surface $\sigma$ is also allowed to end up at the brane. This introduces a new set of interesting `candidate' surfaces that compete in the minimization process.

Recently, a new formula for the computation of holographic entanglement entropy was proposed, relevant for situations when the quantum system of interest is in contact with another system that contains gravity as a dynamical field. We will investigate this formula using the same bulk setup as described in previous sections. \iar{explain some more}

\subsection{Bulk RT to Brane Wald}\label{sec:genera}

In this subsection we study the leading order contributions to holography entanglement entropies for our brane setup in arbitrary dimension. We find that the Wald-Dong entropies associated to the induced $d$-dimensional gravity action match the leading order contributions from the RT surface in the $(d+1)$-dimensional bulk.


\subsubsection{Bulk RT surface contributions \label{bulkrt}}

As before, the bulk consists of two copies of a patch of global AdS$_{d+1}$, glued together along that surface, which creates a spacetime with a tensionfull brane located at that surface. The asymptotic boundary now consists of two copies of a hemisphere cross time, $S^{d-1}\times \mathbb R_t$, glued along a defect. As our entangling region, we choose two identical discs, centered at the pole of each hemisphere, see figure \ref{fig:rtsurface}, which depicts half of the geometry. In this setting, the R-T surface $\sigma$ has two different `phases'. The disconnected or `trivial' phase (phase "I") consists of two independent surfaces with the topology of a disc. On the other hand, in phase II the R-T surface ends on the brane at some codimension-$3$ surface $\del \sigma_B$, which due to symmetry will correspond to a higher dimensional sphere. \iar{add figure of phase I and II}.

%\begin{figure}[h]
%	\centering
%	\includegraphics[width=0.5\linewidth]{images/rtSurface}
%	\caption{Sketch of RT surface for AdS$_4$. Sorry but I messed up the labellings... $\sigma_A$ is should be on the top and $\sigma_B$ is on the bottom. \iar{Can we actually stick to the labelling as in the figure? looks much more natural} }
%	\label{fig:rtsurface}
%\end{figure}


\begin{figure}[h]
	\def\svgwidth{0.6\linewidth}
	\centering{
		\input{3dBrane.pdf_tex}
		\caption{A timeslice of AdS$_{d+1}$ space. Constant $\rho$ hyperbolic foliations are drawn and a brane lives on an AdS$_d$ slice at large $\rho$. On the other boundary of the bulk spacetime lies a $d$-dimensional CFT.
		}
		\label{fig:cutoffs}
	}
\end{figure}



\josh{the area labelling notation should be cleaned up.}

Following Faulkner et al. \cite{Faulkner:2013ana}, we expect \iar{But we are going to prove this later?}
\beq\label{gen}
S_{EE} = \text{ext}\Big(2\frac{A(\sigma)}{4 \Gbk}+ \frac{A(\partial\sigma_\mt{brane})}{4 \Gbr}\Big)\,,
\eeq where $\partial \sigma_\mt{brane}$ is the area of the intersection of the surface $\sigma$ and the brane. The factor of two in the bulk area term accounts for our two bulks that are glued together along the brane.


Before solving for the area of the surface between a disk on the lower AdS boundary and the brane we will first understand the dominant behaviour at very large tension and small $\delta_\mt{CFT}$ cutoff, which is a UV cutoff around the CFT boundary. Since we are considering the leading order behaviour, we simply apply the Fefferman-Graham coordinates for some asymptotic metric and set the cutoff in the radial $z$ coordinate. Since the geometry in question is time-independent, we perform all calculations on a fixed timeslice.

Recall that to compute an extremal areas we parametrize the surface by parameters $\xi^i$ and set up an variational problem in terms of an action for the surface area. For instance, in the Fefferman-Graham coordinates for AdS$_{d+1}$, we have
\beq
I_\mt{extremal}=\int d^{d-1}\xi \frac{L^{d-1}}{z^{d-1}}\sqrt{\det\Big(\frac{\partial z}{\partial \xi^i} \frac{\partial z}{\partial \xi^j} + g_{k l} (x,z)\frac{\partial x^{k}}{\partial \xi^i} \frac{\partial x^{l}}{\partial \xi^i} \Big)}\,.
\eeq

Following Rangamani and Takayanagi's book \cite{Rangamani_2017}, we find that around small $z$ in Fefferman-Graham coordinates the metric on the extremal surface $\sigma$ is decomposed as
\beq
ds^2_{\sigma}=\frac{L^2}{z^2}(dz^2 + \tilde{\gamma}_{a b}d\alpha^a d\alpha^b)\,,
\eeq where $\alpha_i$ are coordinates on the entangling surface. Assuming that the minimal surface falls straight into the bulk from the CFT boundary, we fix $\xi^1$ to be $z$ and find that the leading contribution to the area at this boundary is
\beq\label{area}
A(\sigma)_{\delta_B} = \int_{\delta_B}^{z_{IR}}\frac{L^{d-1}dz}{z^{d-1}}\int d\xi^{d-2}\sqrt{\tilde{\gamma}}+\dots=\frac{ L^{d-1}A(\partial \sigma_\mt{CFT})}{(d-2)\delta_B^{d-2}}+\dots\,,
\eeq where $\partial \sigma_\mt{CFT}$ is the area of the entangling surface on CFT boundary.\iar{clarify here} We can play the same game at the $\rho=\rho_*$ end of the surface, although in this calculation we ``rebundle" the divergent quantity into the induced metric on the brane, as in (\ref{bundle}).

Explicitly, the calculation goes as follows:
\beq\label{branearea}
A(\sigma)_{brane} = \int_s^{z_{IR}} dz \Big[\frac{L^{d-1}}{z^{d-1}}\Big(1+\frac{1}{2}\frac{z^2}{L^2}+\frac{1}{16}\frac{z^4}{L^4}\Big)^{\frac{d-2}{2}}\Big]\int d^{d-2}x \sqrt{-\bar{g}_{AdS_d}}\,,
\eeq where here $\bar{g}_{AdS_d}$ related to the induced metric on the intersection of the brane and the RT surface (denoted $\bar{g}_{ij}$) by
\beq
\bar{g}_{ij} = \frac{L^2}{s^2}\Big(1+\frac{1}{2}\frac{s^2}{L^2}+\frac{1}{16}\frac{s^4}{L^4}\Big) \bar{g}_{ij}^{AdS_d}\,.
\eeq We now simply integrate over the $z$ direction and find, after expanding at small $s$,
\beq\label{brane area}
A(\sigma)_{brane} = \frac{L}{d-2}A(\partial \sigma_\mt{brane}) + \frac{1}{2(d-4)}\frac{s^2}{L}A(\partial \sigma_\mt{brane}) + \mathcal{O}(s^4/L^4)\,.
\eeq
From the calculations above we expect that to leading order
\beq\label{bulk area}
\frac{A(\sigma)}{4 \Gbk}=\frac{L^{d-1}A(\partial \sigma_\mt{CFT})}{4 (d-2)\Gbk \delta_B^{d-2}} + \frac{L A(\partial \sigma_\mt{brane})}{4(d-2) \Gbk}\,.
\eeq Therefore we find
\beq
S_{EE} = \frac{2 L^{d-1}A(\partial \sigma_\mt{CFT})}{4 (d-2)\Gbk \delta_B^{d-2}} + \frac{A(\partial \sigma_\mt{brane})}{4 \Geff}\,,
\eeq where
\beq
\frac{1}{\Geff} = \frac{2 L}{(d-2) }\frac{1}{\Gbk} + \frac{1}{\Gbr}\,,
\eeq which matches (\ref{Geff}) as was desired. Note that we included two contributions of the bulk area (\ref{bulk area}) to account for both sides of the AdS space. Hence we see that the Wald entropy associated to the induced action term is really contained in the bulk RT surface and not a genuine generalized correction to the entanglement entropy. \iar{Good, but need to clarify}. In section \ref{Wald}, we compare the second order term of equation (\ref{brane area}) to the Wald entropy for the curvature-squared terms of the induced action.


\subsubsection{Wald entropies \label{Wald}}

In the generalized entanglement entropy there are contributions from the brane action which take the form of the Wald entropy, as justified in section \ref{waldjustification}.

We start by computing the Wald entropy for the standard Einstein-Hilbert action because of its relevance to both the induced brane action and physical brane action. The Wald entropy formula is \cite{Myers:2010tj}
\beq
S = -2\pi \int_{\Sigma}d^{d-2}x\sqrt{-h}\frac{\partial\mathcal{L}}{\partial R^{ab}_{\,\,\,\,\,cd}}\hat{\epsilon}^{ab}\hat{\epsilon}_{cd}\,,
\eeq where $\mathcal{L}$ is the Lagrangian density and $\hat{\epsilon}_{ab}$ is the binormal to the region $\Sigma$, and $h_{ij}=\tilde{g}_{ij}$ is the induced metric along that region. In our setup, $\Sigma\equiv \partial\sigma_\mt{brane}$ is the ($d-2$)-dimensional intersection surface between the brane and the bulk Ryu-Takayanagi (RT) surface. As in section \ref{bulkrt}, we refer to the induced metric along this surface by $\bar{g}_{ij}$, which is simply the restriction of $\tilde{g}$ to the surface $\partial\sigma_\mt{brane}$.

Note that the integrand is constant along the brane in our geometry so the equation reduces to
\beq
S = -2\pi \frac{\partial\mathcal{L}}{\partial R^{ab}_{\,\,\,\,\,cd}}\hat{\epsilon}^{ab}\hat{\epsilon}_{cd}\int_{\partial\sigma_\mt{brane}}d^{d-2}x\sqrt{-\bar{g}} = -2\pi \frac{\partial\mathcal{L}}{\partial R^{ab}_{\,\,\,\,\,cd}}\hat{\epsilon}^{ab}\hat{\epsilon}_{cd} A(\partial\sigma_\mt{brane}) \,.
\eeq
The Einstein-Hilbert Lagrangian density takes the form
\beq
\mathcal{L} = \alpha \tilde{R} = \alpha \tilde{R}^{ab}_{\,\,\,\,\,cd} \tilde{g}^{ic}\tilde{g}^{jd}\tilde{g}_{ai}\tilde{g}_{bj}\,
\eeq from which we compute
\beq
\hat{\epsilon}^{ab}\hat{\epsilon}_{cd}\frac{\partial\mathcal{L}}{\partial \tilde{R}^{ab}_{\,\,\,\,\,cd}} =\alpha \hat{\epsilon}^{ab}\hat{\epsilon}_{cd} \tilde{g}^{ic}\tilde{g}^{jd}\tilde{g}_{ai}\tilde{g}_{bj} = -2 \alpha\,,
\eeq where we used $\hat{\epsilon}^{ij}\hat{\epsilon}_{ij} = -2$ from \cite{Myers:2010tj}. For the actions on the brane, this gives the standard result
\beq
S_{brane} = \frac{A(\partial\sigma_\mt{brane})}{4 \Gbr}\,\,\,\,\,\text{and}\,\,\,\,\, S_{induced}^{(1)} = \frac{L A(\partial\sigma_\mt{brane})}{4(d-2)\Gbk}\,.
\eeq These standard results were already used at the end of section \ref{bulkrt}. As promised, we can use this formalism to study the higher curvature terms in the induced brane action to ensure that they provide the same information as the area of the bulk RT surface.

We first consider the two curvature-squared terms independently, which will be combined by linearity. Consider the Lagrangian density
\beq
\mathcal{L} = \kappa R^2\implies \hat{\epsilon}^{ab}\hat{\epsilon}_{cd}\frac{\partial\mathcal{L}}{\partial R^{ab}_{\,\,\,\,\,cd}} = - 4 \kappa R\,.
\eeq Also, we need to consider
\beq\label{RicciTensorTerm}
\mathcal{L} = \gamma R^{de}R_{de} =\gamma R^{ab}_{\,\,\,\,\,cd}R^{\,\,\,\,\,ld}_{jk}g_{ai}g_{be}g^{ic}g_{jm}g_{ml}g^{ke}\implies \hat{\epsilon}^{ab}\hat{\epsilon}_{cd}\frac{\partial\mathcal{L}}{\partial R^{ab}_{\,\,\,\,\,cd}} =-4\gamma\hat{R}\,,
\eeq
whenever $R_{ij} = \hat{R}g_{ij}$.
Now we turn to the curvature-squared terms of the induced bulk action (\ref{bulkaction}). Applying the variations above, we find
\beq
S_{induced}^{(2)} = \frac{8 L^3 s^2}{(d-4)(4L^2+s^2)^2}\frac{A(\partial\sigma_\mt{brane})}{4 \Gbk}\,.
\eeq Now expanding this to leading order in $s$, we find
\beq
S_{induced}^{(2)} = \frac{ s^2}{2(d-4)L}\frac{A(\partial\sigma_\mt{brane})}{4 \Gbk}\,,
\eeq which matches the analogous term from \ref{brane area}.

\subsubsection{Dong entropies and extrinsic curvature corrections}

So far we ignored the extrinsic curvatures of the entangling surface on the brane. Following \cite{Bhattacharyya_2014_2, Dong_2014}, we consider a second derivative theory of gravity ($\mathcal{L}^{(2)} = \lambda_1 \tilde{R}^2 + \lambda_2 \tilde{R}_{\mu\nu}\tilde{R}^{\mu\nu}$) with entropy
\beq\label{dong}
S = -4\pi \int d^{d-2}x \sqrt{-\bar{g}}\Big[2 \lambda_1 \tilde{R} +\lambda_2 \Big(\tilde{R}_{\mu\nu}n_i^{\mu}n_i^{\nu}-\frac{1}{2}\tilde{\mathcal{K}}_i\tilde{\mathcal{K}}^i\Big)\Big]\,,
\eeq where $n_i$'s are the two normals to the entangling surface. Note that in our setup,
\beq
\tilde{R}_{\mu\nu}n_i^{\mu}n_i^{\nu} = \hat{R}\tilde{g}_{\mu\nu}n_i^{\mu}n_i^{\nu} = \hat{R}\sum_{i=1}^2 n_i^{\mu}n_{i \mu} = 2\hat{R}\,,
\eeq where we used the normalization of the normal vector. This contribution to the entropy agrees with equation (\ref{RicciTensorTerm}). In our calculations below we consider only the one-sided bulk. Including both contributions simply requires a factor of two multiplying the Dong entropy.

We now present an argument to show that the minimal surface calculation also includes the extrinsic term of equation (\ref{dong}). The argument follows \cite{Hung_2011}.

First we describe the embedding of the RT surface by the bulk coordinates $X^{\mu} = {x^i,z}$, which can be expanded near the brane surface as
\beq
X^i(y^a, z) = X^{(0)i}(y^a) + \frac{z^2}{L^2}X^{(1)i}(y^a)+\dots\,,
\eeq where $y^a$ are the coordinates along the bulk surface. The induced metric along the surface is given by
\beq
h_{\alpha\beta} = \partial_{\alpha}X^{\mu}\partial_{\beta}X^{\nu}G_{\mu\nu}[X]\,,
\eeq which can be expanded around the brane as
\beq
h_{zz} = \frac{L^2}{z^2}\Big(1+\frac{z^2}{L^2}h_{zz}^{(1)}+\dots\Big),\,\,\,\,\, h_{ab}=\frac{L^2}{z^2}\Big(h^{(0)}_{ab}+\frac{z^2}{L^2}h^{(1)}_{ab}+\dots\Big)\,.
\eeq
Note that $h_{ab}=\tilde{g}_{ab}|_{\sigma_b}$ is the induced metric at the intersection of the RT surface and the brane. Now when we studied the behaviour of the RT surface in equation (\ref{branearea}), we assumed that the brane fell straight into the bulk, which is why the extrinsic curvature contributions did not appear. The correct expansion around the brane is \cite{Hung_2011}
\begin{multline}\label{correctedarea}
A(\sigma)_{brane} = \int_s^{z_{IR}}dz \Big[\frac{L^{d-1}}{z^{d-1}}\Big(1+\frac{1}{2}\frac{z^2}{L^2}+\frac{1}{16}\frac{z^4}{L^4}\Big)^{\frac{d-2}{2}}\Big]\int d^{d-2}x \sqrt{-\bar{g}_{AdS_d}}\\\times\Big[1+\Big(h^{(1)}_{zz}+h^{(0)ab}h^{(1)}_{ab}\Big)\frac{z^2}{2L^2}+\dots \Big]\,,
\end{multline} where we notice that the first-order term was the contribution evaluated already. We now focus on the second order term. To raise the indices in the $h^{(0)ab}$, we used the metric $h^{ab}$, which is order $z^2/L^2$. Hence we see that at second order we only need to focus on the $h^{(1)}_{zz}$ term. \josh{Check this is correct, i.e. this second term is indeed subdominant. ALSO, double-check that $\tilde{\mathcal{K}}_i$ is the same as the $K_i$ in Dong.}

It was shown in \cite{Hung_2011} that the second-order contribution to the RT surface embedding coordinates can be written in terms of the extrinsic curvature,
\beq
X^{(1)i} = \frac{L^2 \tilde{\mathcal{K}}^i}{2(d-2)}\,,
\eeq and the induced metric can be expanded via
\beq
h_{zz} = \frac{L^2}{z^2}\Big(1+\frac{4z^2}{L^4}(X^{(1)i})^2+\dots\Big)\implies h^{(1)}_{zz} = \frac{L^2 (\tilde{\mathcal{K}}^i)^2}{(d-2)^2}\,.
\eeq
We now substitute this expression into equation (\ref{correctedarea}) and integrate out the $z$ direction. Upon expanding at small $s$, we obtain
\beq
A(\sigma)^{(1)}_{brane} = \frac{L s^2}{2 (d-4)(d-2)^2}(\tilde{\mathcal{K}}^i)^2 \int d^{d-2}x \sqrt{-\tilde{g}}\,,
\eeq and hence the second order contribution to the entanglement entropy is
\beq
S^{(1)} = \frac{1}{4 \Gbk} \frac{L s^2}{2 (d-4)(d-2)^2}(\tilde{\mathcal{K}}^i)^2 A(\partial\sigma_\mt{brane})\,,
\eeq which precisely matches equation (\ref{dong}) after computing $\lambda_2$ from the induced brane action (\ref{bulkaction}),
\beq
\lambda_2= \frac{L^3}{16 \pi(d-4)(d-2)^2 \Gbk}\,.
\eeq
\josh{I need to be a bit more careful... the left over $s^2/L^2$ comes from the raising and lowering of the $i$ indices compared to Dong's formula or from the metric in Dong's formula?.}
From these calculations we conclude that the entropies associated to the induced action on the brane in the Randall-Sundrum picture give exactly the leading contributions to the RT surface area in the $(d+1)$-dimensional bulk!







\subsubsection{A comment on $d=2$ case}

We see that the equations are slightly different in $d=2$ since the calculation in (\ref{area}) will introduce a log rather than a powerlaw behaviour. This will again match the RS-picture side.

The leading order behaviour of the bulk RT surface is just a single integral,
\beq
S = \frac{1}{4 G_3}\int_{s}^{\lambda} \frac{L}{z}dz = -\frac{L}{4G_3} \log\Big(\frac{s}{L}\Big)\,,
\eeq where we have ignored the infrared contribution. We also take the variation of the $d=2$ induced action to compute its Wald entropy \cite{Dong_2014}. We find
\begin{multline}
S = \frac{1}{4 G_{3}}\frac{\partial \mathcal{L}}{\partial \tilde{R}} = \frac{1}{4 G_{3}}\frac{\partial }{\partial \tilde{R}}\Big(\frac{2}{L} -\frac{1}{2}L \tilde{R} \log\Big(-\frac{L^2}{ 2} \tilde{R}\Big) + \frac{L}{2}\tilde{R}+\frac{L^3}{16}\tilde{R}^2\Big)\\=-\frac{L}{4G_3} \log\Big(\frac{s}{L}\Big)\,,
\end{multline} which matches the RT surface calculation (after expanding to leading order in $s$)!



\subsection{Examples}\label{sec:examples}

\rcm{Let's stream line this, \ie minimize the number of coordinate systems/transformations. Don't need embedding space or Poincare coordinates, etcetera:}

Recall that eq.~\reef{metric} described the AdS$_{d+1}$ geometry foliated by AdS$_d$ slices with
\beq\label{metric1a}
ds^2 %= g_{ab}\,dx^{a}dx^{b}
= d\rho^2 + \cosh^2\left({\rho}/{L}\right)\, g_{ij}^{\mt{AdS}_d}\,dx^{i}dx^{j}\,.
\eeq
but we really discussed the brane geometry and the background solution in terms of eq.~\reef{metric3}
\beq\label{metric3a}
ds^2=\frac{L^2}{z^2}\left[dz^2 +  \left(1 + \frac{z^2}{4\,L^2}\right)^2 g_{ij}^{\mt{AdS}_d}\,dx^{i}dx^{j} \right]\,.
\eeq 
So I would really prefer to start with these coordinates! While these coordinates are ideally suited to discuss the brane geometry and the back-reacted background, we would also like to consider `global' coordinates for the AdS$_{d+1}$ geometry
\beq\label{metric2s}
ds^2=L^2\left[-\cosh^2\!r\,dt^2+dr^2+\sinh^2\!r\,d\Omega_{d-1}^2 \right]\,.
\eeq 
These coordinates are better adapted to discuss the boundary theory, which we consider to be on an $R\times S^{d-1}$ background. While we refered to these as `global' coordinates, they do not cover the entire back-reacted bulk solution depicted in figure \ref{fig:brane2}. Rather we could use the coordinates in eq.~\reef{metric2s} to cover two patches on either side of the brane and near the asymptotic AdS$_{d+1}$ boundary.

\rcm{So what is the coordinate transformation between eqs.~\reef{metric3a} and \reef{metric2s}?? need to pick some `nice' coordinates for $g_{ij}^{\mt{AdS}_d}\,dx^{i}dx^{j}$ in eq.~\reef{metric3a}. It would be convenient if we can use the same time $t$ in both metrics. Is this transformation needed? Yes because we will need to describe the brane location $z=z_\mt{B}$ in the following.} 

Now following \cite{Krtous:2014pva}, we introduce cylindrical coordinates
\beq\label{cylindd}
ds^2=L^2\[ -\rcm{???}dt^2+\frac{dP^2}{1+P^2}+\left( 1+P^2 \right)d\zeta^2+P^2\,d\Omega_{d-2}^2\]
\eeq
\rcm{Note that I want to show the full spacetime metric and I want to show a metric for AdS$_{d+1}$. Where is the $r$ cutoff in global/spherical coordinates? Shouldn't matter in comparing areas but be careful.}

\rcm{Now what is the coordinate transformation between eq.~\reef{metric2s} and \reef{cylindd}?? What is the position of the brane in the cylindrical coordinates?}

\rcm{Now should be able to start, \eg set up eqs.~\reef{area} and \reef{zetap}, in general dimensions. Only then proceed to $d=3$ and proceed with explicit calculations for the example. Somewhere in there (\ie $d=3$) we have to talk about adding Gauss-Bonnet term to bulk gravity and adding Euler character to RT formula!
 (Working first part with general $d$ allows us to do bubbles in general dimensions in next section.)}


\rcm{*****************************}

In this section, we consider concrete examples of the construction outlined in the previous sections. We work out explicitly the cases of bulk AdS$_{d+1}$ spacetimes with $d=2,3,4$. As before, we begin by examining $d>2$ first, since it is simpler in that the analysis can be done purely in terms of Einstein (or higher derivative) gravity. We then turn to $d=2$, which involves the addition of a Jackiw-Teitelboim term for the dilaton.

In particular, we are interested in exploring the physics of quantum systems that are entangled with degrees of freedom in a theory with gravity. According to a recent proposal, the formula for the generalised entanglement entropy must include an extra term, that of eq. \eqref{gen} \cite{Almheiri:2019hni}. Our setup is one of the simplest scenarios where to test this proposal explicitly.

As usual, the areas of surfaces in both the connected and disconnected phases are UV divergent. Since we will be mainly interested in the connected solution (phase II), we will always work with the renormalised area obtained by subtracting the contribution of phase I, rendering a finite result. In this way, one can safely take the UV cutoff to infinity from the beginning of the calculations.

We begin with a general analysis for an AdS$_d$ brane embedded in AdS$_{d+1}$. We then consider the particular case of $d=3$, where most computations can be done analytically. In order to introduce the coordinate system, we begin with the basics. As usual, AdS$_{d+1}$ is defined by
\begin{align}\label{adsdef}
-U^2-V^2+\sum_{i=1}^{d} X_i^2=-L^2
\end{align}
embedded into $\mathbb R^{2,d}$. As is well known, the following change of coordinates
\begin{align}\label{}
U&=\frac{z}{2}\left( 1+\frac{L^2-t^2+\vec{x}^2}{z^2} \right) \\
X_{d}&= \frac{z}{2} \left( 1-\frac{L^2+t^2-\vec{x}^2}{z^2} \right) \\
V&=\frac{Lt}{z}\\
X_j&=\frac{Lx_j}{z}\ \ ,\ \ j=1,\hdots,d-1
\end{align}
with $\vec{x}=(x_1,\hdots,x_{d-1})$ leads to the Poincare half space metric,
\begin{align}\label{Poincare}
ds^2=L^2 \frac{dz^2-dt^2+ d\vec{x}^2}{z^2}
\end{align}

Another representation that will be useful for us is achieved by performing a further change of variables in the $(z,x_j)$ for a given $j$. Without loss of generality, let us simply choose $j=1$, and transform according to
\begin{align}\label{zx1}
z=\frac{y}{\cosh \left( \rho/L \right)}\ \ ,\ \ x_1=y\tanh\left( \rho/L \right)
\end{align}
while leaving all other coordinates unchanged. This leads to the metric already given in \eqref{metric},
\begin{align}\label{hyper}
ds^2=d\rho^2+\cosh^2\left( \rho/L \right) ds^2_{\text{AdS}_{d}}
\end{align}
where the AdS$_d$ metric take the Poincare form
\begin{align}\label{}
ds^2_{\text{AdS}_{d}}&=L^2 \frac{dy^2-dt^2+dx_2^2+\hdots dx_{d-1}^2}{y^2}
\end{align}

The metric \eqref{hyper} represents a \textit{hyperbolic} foliation of AdS space: surfaces of constant $\rho$ correspond also to AdS$_{d}$, embedded in the global AdS$_{d+1}$. Moreover, these surfaces possess a constant value of the extrinsic curvature scalar $\mathcal{K}$. In terms of the Poincare coordinates \eqref{Poincare}, these are defined by planes of a given slope in the $(x_1,z)$ directions which can be parametrized by their respective value for $\rho$:
\begin{align}\label{}
\frac{x_1}{z}=\frac{X_1}{L}=\sinh \left( \rho/L \right)
\end{align}

Of course, as mentioned above, due to symmetry the same would apply to any other $x_{j}$ (or $X_j$) for $j=1,\hdots,d-1$, had we chosen it instead of $x_1$ ($X_1$) in \eqref{zx1}. This simply reflects the $SO(d-1)$ rotational symmetry of \eqref{adsdef}.

Finally, another representation of AdS$_{d+1}$ is the well known global chart in spherical coordinates:
\begin{align}\label{}
U&=L\cosh r \cos \tau\\
V&=L\cosh r \sin \tau\\
X_i&=L\sinh r\ \hat{x}_i\ \ \ ,\ \ i=1,\hdots,d
\end{align}
with $\sum_{i=1}^{d}\hat{x}_i^2=1$ so the $\hat{x}_i$ parametrize the unit sphere $S^{d-1}$, and the metric reads
\begin{align}\label{globalads}
ds^2=L^2\left( -\cosh^2r\, d\tau^2+dr^2+\sinh^2 r\, d\Omega_{d-1}^2 \right)
\end{align}

In these coordinates, a hypersurface of constant $X_j$ - equivalent to a fixed $\rho$ as discussed above - corresponds to
\begin{align}\label{brane}
\frac{X_i}{L}=\sinh r\, \hat{x}_i=\sinh \left(  \rho/L \right)
\end{align}
For simplicity, we will use the freedom to choose the orientation of the spacial spherical axes such that both the entangling surface and the brane have azimuthal symmetry.

\subsubsection{Explicit results in AdS$_{4}$}

In this section we will particularize the discussion to $d=3$, that is our bulk spacetime will be AdS$_4$ and we will consider branes with an induced AdS$_3$ geometry. For this case, some analytic expressions can be obtained. We start by reviewing the problem of finding minimal surfaces in empty global AdS$_4$ first with no brane inserted, as explored in \cite{Krtous:2014pva}. In section \ref{AdS/BCFT} we will use the standard AdS/BCFT tools \cite{Takayanagi:2011zk} to deal with a CFT with a boundary, which requires the introduction of the brane.

Our starting point is the AdS$_4$ metric in global coordinates \eqref{globalads},
\begin{align}\label{}
\frac{ds^2}{L^2}=-\cosh^2 r d\tau^2+dr^2+\sinh^2r \left( d\theta^2+\sin^2\theta d\varphi^2 \right)
\end{align}

Since we will focus on entanglement entropies, we shall fix $\tau=0$ and work with three-dimensional hyperbolic space $\mathbb H^3$
\begin{align}\label{H3}
\frac{ds^2}{L^2}=dr^2+\sinh^2r \left( d\theta^2+\sin^2\theta d\varphi^2 \right)
\end{align}
The main strategy of \cite{Krtous:2014pva} was to move from the spherical coordinates $r,\theta$ to cylindrical coordinates $P,\zeta$ where $\zeta$ specifies the position along the axis of the cylinder while $P$ measure the distance from that axis. The azimuthal angle $\varphi$ is left unchanged.



The transformation is given by
\begin{align}\label{}
\sinh r&=\sqrt{\frac{P^2+\tanh^2\zeta}{1-\tanh^2\zeta}}\\
\tan\theta&=\frac{P}{\sqrt{1+P^2}}\frac{\sqrt{1-\tanh^2\zeta}}{\tanh\zeta}
\end{align}
with $P\in (0,\infty),\zeta\in(-\infty,\infty)$. The equator of the sphere $\theta=\pi/2$ is mapped to the `equator' of the cylinder $\zeta=0$, while the upper (lower) hemisphere is mapped to the upper (lower) half space in the cylindrical system. The conformal boundary at infinity corresponds to $P\to \infty$.

In this coordinates, the spatial section \eqref{H3} takes the particularly simple form
\begin{align}\label{H3cyl}
\frac{ds^2}{L^2}=\frac{dP^2}{1+P^2}+\left( 1+P^2 \right)d\zeta^2+P^2d\varphi^2
\end{align}
The convenience of these coordinates is now evident: the metric \eqref{H3cyl} has now \textit{two} manifest Killing vectors, as it doesn't depend explicitly on either $\varphi$ nor $\zeta$.

In order to make use of this symmetry, we will restrict to settings that possess cylindrical symmetry around $P=0$. We will consider entangling surfaces (circles for $d=3$) that are centred around this axis, and similarly for the brane in section \ref{AdS/BCFT}. Therefore, and without further loss of generality, we choose the axis $\hat{x}_1$ of the brane as the $\theta=0$ line in spherical coordinates, such that both the brane and the entangling surface will be symmetric around $\varphi$.



\paragraph{Minimal surfaces in global AdS$_4$}

As we restrict to settings with cylindrical symmetry, it is convenient to parametrize a surface by $\zeta=\zeta(P)$. The associated area functional is
\begin{align}\label{area}
\frac{A}{L^2}=\int dP P \sqrt{\frac{1}{1+P^2}+(1+P^2)\zeta'^2}
\end{align}
\vc{missing factor of $2\pi$ on RHS from $\varphi$ integral?}The Euler-Lagrange equation associated to \eqref{area} becomes particularly simple since the metric in cylindrical coordinates is independent of $\zeta$. The solution for the derivative is given by
\begin{align}\label{zetap}
\zeta'(P)=\pm \frac{1}{1+P^2}\sqrt{\frac{P_0^4+P_0^2}{P^4+P^2-P_0^4-P_0^2}}
\end{align}
where the two branches correspond to two identical surfaces related by a reflection with respect to $\zeta=0$. As is well known in the holographic context, the minimal surface anchored to a fixed entangling region at conformal infinity needs not be unique. For a given setting there may be multiple local minima of the area functional, and one is instructed to pick the global minima amongst them. This leads to possible `phase transitions' in the RT surface as we now discuss.

The disconnected `trivial' solution corresponds uniquely to $P_0=0$. This implies $\zeta(P)=\zeta_0$ which in cylindrical coordinates looks simply as a disk anchored at the boundary region. The entanglement wedge corresponds to two identical disconnected pieces.

On the other hand, for $P_0>0$ the surface is very nontrivial, and the two branches of \eqref{zetap} meet at the locus $\zeta=\zeta_0,P=P_0$ which is where the minimal surface reaches its maximal depth into the bulk. Integrating the explicit form of the curve \eqref{zetap}, one finds
\begin{align}\label{zetasol}
\zeta(P;P_0,\zeta_0)&=\zeta_0\pm \frac{P_0}{\sqrt{(1+P_0^2)(1+2P_0^2)}} \nonumber \\
&\left[ (1+P_0^2) F\left( \mbox{Arcos} \frac{P_0}{P},\sqrt{\frac{1+P_0^2}{1+2P_0^2}} \right)-P_0^2 \Pi\left( \mbox{Arccos}\frac{P}{P_0},\frac{1}{1+P_0^2},\sqrt{\frac{1+P_0^2}{1+2P_0^2}} \right) \right]
\end{align}

The integration constant $\zeta_0$ corresponds to the value of $\zeta$ when the surfaces reaches the deepest into the bulk, while $F$ and $P$ correspond to the elliptic integrals.


Notice that we are characterising the minimal surfaces \eqref{zetasol} in terms of $\zeta_0$ and $P_0$, which does not directly provide the location at which the surface is anchored at the boundary.  To find this location - the entangling region corresponding to a given $\zeta_0,P_0$ - we can consider the inverse function, $\zeta_\infty(P_0)$, defined via $\zeta(P\to \infty)=\zeta_0\pm \zeta_\infty$, and given by
\begin{align}\label{}
\zeta_{\infty}(P_0)=\frac{P_0\left( (1+P_0^2)K\left( \frac{1+P_0^2}{1+2P_0^2} \right) - P_0^2 \Pi \left( \frac{1}{1+P_0^2},\frac{1+P_0^2}{1+2P_0^2} \right) \right)}{\sqrt{(1+P_0^2)(1+2P_0^2)}}
\end{align}

In figure \ref{figzetainfty} we plot the asymptotic value $\zeta_\infty$ as a function of $P_0$. One of the interesting observations of \cite{Krtous:2014pva} was that, for a given $\zeta_\infty<\zeta_\infty^{\text{max}}\approx 0.5$, there exist \textit{two} values of $P_0$ with the same $\zeta_\infty$. This implies that, for two discs that are sufficiently `close' on the sphere, there actually exist \textit{two} minimal surfaces that connect them. However, one of them (the branch which has smaller value of $P_0$ is always subdominant, and therefore will be of little interest in our analysis). On the other hand, if the `height' of the RT surface (in cylindrical coordinates) is larger than a critical value, there exist no connected minimal surface that joins them.

\begin{figure}[h]
\begin{center}
\includegraphics[scale=0.4]{images/zetainfty}
\caption{Plot of the `height' of the RT surface in cylindrical coordinates, as a function of the parameter $P_0$ characterising the surface. For $2\zeta_\infty\leq 2\zeta_\infty^{\text{max}}\approx 1$, there are two minimal surfaces anchored at the same regions; for $2\zeta_\infty> 2\zeta_\infty^{\text{max}}$ there exists no connected minimal surface joining them. }
\label{figzetainfty}
\end{center}
\end{figure}

Next we consider the areas. Although the area of the surfaces is divergent, one can compute the integral of the area element from $P_0$ to some $P>P_0$ along the surface\cite{Krtous:2014pva}. For the disconnected solution (phase I), the area of the pair of disks of radius $P$ is
\begin{align}\label{zetainfty}
\frac{1}{L^2}A_{\text {I}}(P)=2\times2\pi\left( \sqrt{1+P^2}-1 \right)
\end{align}

On the other hand, for the connected surfaces one can compute the area from $P_0$ to some $P>P_0$, yielding
\begin{align}\label{}
\frac{1}{L^2}A_{\text {II}}(P,P_0)&= 2\times\frac{2P_0^2}{\sqrt{1+2P_0^2}} \Pi\left( \mbox{Arccos}\frac{P_0}{P},1,\sqrt{\frac{1+P_0^2}{1+2P_0^2}} \right)
\end{align}
The factor of two is due to the fact that, in both the disconnected and connected cases, the surface is identical on both sides of the brane.

Of course, as usual in AdS/CFT, these areas possess a divergent contribution for large radius $P$. We define the renormalised area $\Delta A$ by subtracting the area of the disconnected solution:
\begin{align}\label{dAP01}
\Delta A(P_0)&=\lim_{P\to \infty}\left( A_{\text {II}}(P,P_0)-A_{\text {I}} (P)\right)
\end{align}
where we can safely take the limit, since the divergences match. The plot of $\Delta A$ is shown in figure \ref{figdeltaA}. Of course, the renormalised area is closely related to the mutual information. Mutual information between two subsystems $A$ and $B$ is defined via $I=S_{A\cup B}-S_A-S_B$. Notice that whenever $\Delta A>0$, phase $I$ dominates and therefore the mutual information vanishes, since $S_{A\cup B}=S_A+S_B$.

\begin{figure}[h]
\begin{center}
\includegraphics[scale=0.3]{images/dAP0}
\caption{Renormalised area from eq. \eqref{dAP01}. For $\Delta A>0$, the disconnected surface I has less area and dominates, while for $\Delta A<0$ the connected solution II is the global minimum.}
\label{figdeltaA}
\end{center}
\end{figure}


\subsubsection{Introducing a brane: R-T surfaces in AdS/BCFT}\label{AdS/BCFT}

With this choice, the equation for the brane \eqref{brane} becomes
\begin{align}\label{}
\sinh r\, \cos\theta=\left( 1+P^2 \right)\sinh^2\zeta=\sinh\left( \rho/L \right):=u^2=\cot\theta_\brane
\label{eq:foobar}
\end{align}
\vc{second expression (and probably third expression also) should be squarerooted.}
From here we see that the brane intersects the conformal boundary at the equator $\theta\to\pi/2,r\to\infty$ or equivalently $\zeta\to 0,P\to \infty$ in the cylindrical coordinates.

In the small $\Gbk$ expansion, the standard R-T formula states that the entanglement entropy of a subregion $\Sigma$ in the CFT is given by the area of a classical minimal surface that can probe the bulk but is anchored at $\partial \Sigma$ and remains homologous to $\Sigma$. The extension of this recipe to the case of boundary CFTs (BCFT) is a natural extrapolation of the dictionary\,\cite{Takayanagi:2011zk}: the bulk spacetime is instructed to `end' at a brane that lives in the bulk and asymptotes to the boundary of the CFT. This introduces a new physical effect in the system: the R-T surfaces are now allowed to end at the brane and thus a new set of `competing' surfaces enter the minimisation problem, which must also be taken into account. In cases where there are multiple local extrema, the EE is then given by the area of the global minimum.

As the first step towards understanding the physics of the formula \eqref{Sgenint}, in this subsection we consider the problem of the BCFT at the leading order in $1/\Gbk$. That is, we solve the problem of finding the R-T surfaces anchored at the CFT subregion that end on the brane, see figure \ref{figsetup}.
\begin{figure}[h]
\begin{center}
\includegraphics[scale=0.3]{images/setup}
\caption{A half of the geometric setup in cylindrical coordinates, with the radial direction compactified to a finite coordinate value. The `tube' (orange) represents the RT surface $\sigma$ in phase II, anchored at conformal infinity and intersecting the brane (green) at the codim-$3$ surface $\del\sigma_{brane}$ (red). The disk (blue) corresponds to the trivial phase I. We subtract their areas to obtain a finite result. }
\label{figsetup}
\end{center}
\end{figure}

The total renormalised area of phase II has now two contributions\footnote{Since we will focus on the regime of branes of large tensions, the second term term in eq. \eqref{A2tobrane} indeed comes with a positive sign. For small tensions, some minimal surfaces would intersect the brane before reaching the point where $P=P_0$, and therefore the second term would appear with a negative sign in front. }
\begin{align}\label{A2tobrane}
\Delta \tilde A(u,\zeta_{cft},P_0)=\Delta A[P_0]+A[P_b,P_0]\ \ ,\ \ P_b=P_b(u,\zeta_{cft},P_0)
\end{align}
The first term is simply the renormalised area term introduced in \eqref{dAP01}, which involves an integral from $P_0$ to $\infty$ in the lower half of the surface of figure \ref{figsetup}. In the second, we integrate from $P_0$ to $P_b$, defined as the intersection of the surface with the brane. We have emphasised that the radius of intersection $P_b$ depends on the location of the brane via $u$, together with the anchoring hight $\zeta_{cft}$ and the parameter $P_0$ of the given curve. To find the entanglement entropy of the disk in the BCFT, one must take the area \eqref{A2tobrane} and vary $P_0$ in order to find the minimum, for fixed $u$ and $\zeta_{cft}$.

We perform this computation numerically, since the special functions involved in the calculation are difficult to invert. Some sample curves are displayed in figure \ref{}, where we plot the area of the R-T surface for fixed anchoring point $\zeta_{cft}$ as we vary $P_0$, for different values of $u$, the parameter controlling the tension of the brane. In particular, we notice that the renormalised area is always positive for sufficiently large $u$. Therefore we find that, although a local minimum exists, the renormalised area up to the brane is always larger than that of the disk -  that is, the disconnected R-T solution always dominates. Thus, in the regime of large $u$, the disconnected phase always dominates. By itself, this phase would not be too interesting, since the presence of the brane would not play a significant role, and the mutual information between the two discs would simply vanish.
\begin{figure}[h]
\begin{center}
\includegraphics[scale=0.3]{images/ClassAreas}
\caption{ Renormalised area $\Delta\tilde A$ of connected R-T surfaces as a function of $P_0$, from the anchoring point on the CFT up to the intersection with the brane, for different values of $u$. }
\label{figClassAreas}
\end{center}
\end{figure}

However, the story has a twist: as discussed recently in the context of black hole evaporation models and also discussed above, the RT formula to compute entanglement entropies must be corrected if the quantum system of interest is entangled with another system where gravity is dynamical. This is precisely the case at hand, since the CFT is entangled with degrees of freedom on the RS brane, where we have an Einstein-Hilbert gravity action.


\subsubsection{The generalised entropy on the brane}\label{sec:generalised}

Let us define the dimensionless `generalised area' functional as
\begin{align}\label{Agen}
\mathcal{A}_{gen}(\sigma)=\frac{A(\sigma)}{4\Gbk}+\frac{A(\partial\sigma_{brane})}{4\Gbr}
\end{align}
Then, the generalised entropy is defined as the minimum (or extremum for the covariant case) of the generalised area under variations of the surface $\sigma$:
\begin{align}\label{}
S_{gen}=\min_{\sigma} \mathcal{A}_{gen}(\sigma)
\end{align}
Notice that each $\sigma$ determines its own $\partial\sigma_{brane}$, which is not an independent variable.

As before, we are interested in the renormalised area of the connected surface, where we subtract the disconnected contribution. Notice that this affects only the bulk area term, and not the second term of \eqref{Agen}. Now let us specify to the concrete case of our geometric setup. We must consider
\begin{align}\label{dAgen}
\Delta \mathcal{A}_{gen}&=\frac{\Delta \tilde A(\sigma)}{4\Gbk}+\frac{A_{\text{II}}(\partial\sigma_{brane})}{4\Gbr}
\end{align}

The first term, $\Delta \tilde A=A_{\text{II}}(\sigma)-A_{\text{I}}(\sigma)$, is simply the renormalised area up to the brane, already considered in section \ref{AdS/BCFT}. The new term, $A_{\text{I}}(\partial\sigma_{brane})$, the area of the intersection between $\sigma$ and the brane. In this case, due to the axial symmetry around $P=0$, this consists of a circle of radius $P_b$, which according to the metric \eqref{H3cyl} has an area (perimeter) of
\begin{align}\label{ring}
A_{\text{II}}(\partial\sigma_{brane})=2\pi L P_b
\end{align}

It will prove convenient to work with the dimensionless ratio
\begin{align}\label{lambdaratio}
\lambda:=\frac{\Gbk}{2L\Gbr}
\end{align}
which controls the relative strength between the bulk and brane Newton constants, and therefore the effect of the extra term in the generalised entropy.

We explore the generalised area and their associated entropies numerically. In figure \ref{figA_gen} we plot the renormalised generalised area of eq. \eqref{dAgen} as function of $P_0$, for different values of the anchoring point $\zeta_{cft}$, brane tension $u$ and the ratio $\lambda$.  Finally, the renormalised generalised entropy is the minimum over all allowed $P_0$:
\begin{align}\label{}
\Delta S_{gen}=\min_{P_0} \Delta \mathcal{A}_{gen}
\end{align}

Let us recall what parameters are in play. The tension of the brane is controlled by $u$, which we keep large but finite. The dimensionless ratio between the bulk and brane gravitational constants is controlled by $\lambda$. In this regime, the interesting physics arises when $\lambda<0$, which in this context is obtained by allowing $\Gbr<0$ while $\Gbk>0$.  As discussed above, a negative gravitational constant on the brane can be naturally obtained via quantum corrections. Moreover, the parameter $P_0$ is not free, but must be varied in order to find the minimum of the total area functional. For a given $\zeta_{cft}$, only those R-T surfaces with $P_0$ such that they actually intersect the brane are allowed to `compete' in the minimisation problem.
\begin{figure}[h]
\begin{center}
\includegraphics[scale=0.3]{images/A_gen}
\caption{Generalised area for different values of $-1<\lambda<0$, as function of $P_0$. For each curve, the generalised entropy is given by the global minimum. The critical value for which the minimum lies below the axis signals the transition when phase II dominates.}
\label{figA_gen}
\end{center}
\end{figure}

Figure \ref{figA_gen} contains information about the phase diagram of our system. First, we see the existence of a first order phase transition between two different R-T surfaces that end on the brane. Notice that this non-trivial structure is induced by the second term in the generalised entropy. Indeed, when $\lambda=0$, one obtains the results of figure \ref{figClassAreas}, where there is a single global and local minimum, for each value of the brane tension.

\iar{Figure: plot $\lambda_c$ vs $u$}



%\newpage

%\subsubsection{Higher dimensions}

%\paragraph{Mutual information of two discs}

\paragraph{Entanglement Wedge cross sections}

$P_0$ is defined as the point where $\zeta'$ diverges, we need some characterisation in terms of CFT data, the same for mutual info
