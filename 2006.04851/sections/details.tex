% !TEX root = ../lifeonbrane3.tex
%



We will give some detailed calculations which show how the change in bulk parameters is related to changes in the brane/boundary picture. For simplicity, we consider the case of a Poincare patch with a time-like co-dimension one defect which passes through the origin. The defect breaks the conformal symmetry from $SO(d,2)$ to $SO(d-1,2)$.

\subsection{Interface OPE expansion in holography}
An operator in the CFT has an expansion in terms of operators living at the interface. If we denote by $x$ the direction perpendicular to the brane and by $y$ the direction along the brane, we have that
\begin{align}
    O_{\Delta}(x,y) = \sum_{\Delta'} c_{O_{\Delta}O_{\Delta'}} \frac{O_{\Delta'}(y)}{|x|^{\Delta - \Delta'}}.
\end{align}
We will now discuss how this expansion is realized holographically. To this end we choose AdS slicing coordinates
\begin{align}
    ds_{AdS_{d+1}}^2 = \frac{L^2}{\sin^2 \theta}(d\theta^2 + ds_{AdS_d}^2).
\end{align}
The boundaries are located at $\theta = \pm \pi$.
As mentioned above, we will for simplicity assume that $ds_{AdS_d}^2$ is the AdS metric in Poincare coordinates,
\begin{align}
    ds_{AdS_d}^2 = \frac{dx^\mu dx_\mu + dz^2}{z^2}
\end{align}
Consider a scalar field of mass $m$. We will make a the ansatz
\begin{align}
    \phi(x,z,\theta;p) = e^{i p x} \psi(z) f(\theta).
\end{align}
The equation of motion then yields
\begin{align}
    \psi(z) \left(f''(\theta )-(d-1) \cot (\theta ) f'(\theta )\right)\\
    +f(\theta ) \left(z \left(-(d-2) \psi'(z) +z \psi''(z)-z \psi(z) p^2\right)-m^2 \psi(z) \csc ^2(\theta )\right) = 0
\end{align}
We need to solve
\begin{align}
    z \left(-(d-2) \psi'(z)+z \psi''(z) - z \psi(z) p^2 \right)=\lambda  \psi(z).
\end{align}
This is the $z$ dependent part for the equation of a scalar field on AdS$_d$ with mass-squared $\lambda$. As we will see immediately, this means that $\lambda$ is bounded from below by the Breitenlohner-Freedman bound, $\lambda \geq - (\frac{d-1}{2})^2$. As is well known, the general solution can be expressed in terms of Bessel functions,
\begin{align}
    \psi(z) = c_1 z^{\frac{d-1}{2}} J_{\frac{1}{2} \sqrt{d^2-2 d+4 \lambda +1}}\left(z \sqrt{-p^2}\right)+c_2 z^{\frac{d-1}{2}} Y_{\frac{1}{2} \sqrt{d^2-2 d+4 \lambda +1}}\left(z \sqrt{-p^2}\right).
\end{align}
The equation for $f(\theta)$ then becomes
\begin{align}
\left(f''(\theta )-(d-1) \cot (\theta ) f'(\theta )\right)+f(\theta ) \left(\lambda - m^2 \csc ^2(\theta )\right)=0
\end{align}
with solutions
\begin{align}
    \left( c_1 P_{\frac{1}{2} \left(\sqrt{d^2-2 d+4 \lambda +1}-1\right)}^{\frac{1}{2} \sqrt{d^2+4 m^2}}(\cos (\theta ))+c_2 Q_{\frac{1}{2} \left(\sqrt{d^2-2 d+4 \lambda +1}-1\right)}^{\frac{1}{2} \sqrt{d^2+4 m^2}}(\cos (\theta ))\right)\left(\sin ^2(\theta )\right)^{d/4},
\end{align}
where $P$ and $Q$ are Legendre functions. Note that $\lambda$ has not been fixed yet, so we have two continuous families of solutions. We will see in a bit that by carefully considering the situation at the bulk dual of the defect, $\lambda$ will be quantized. The defect operators will then be given as the dual operators to the $d$-dimensional fields $\psi[z]e^{ipx}$.


The Legendre functions are somewhat hard to work with, since for generic mass both solutions diverge as $\theta \to - \pi$ and only a particular linear combination forms the normalizable modes. It would therefore probably be useful to re-express them in terms of hypergeometric functions. We will not do this here, however, but continue discussing a special case in which $P$ becomes the normalizable mode. This happens for $m=0$, a massless field in the bulk.

In the case discussed in this paper, we have two copies of the spacetime between the boundary and the brane, which are identified along the brane. Since the asymptotic boundary is compact it is clear that the bulk modes must be quantized and in fact, it is well known that requiring regularity of the normalizable mode in the bulk in global AdS imposes a quantization condition on the bulk modes. By $\mathbb Z_2$ symmetry of the setup, we know that all modes are either even or odd under reflection across the brane. Without any additional sources added, the solution for a scalar field is is regular everywhere.\footnote{We will revisit this assumption below.} The even modes have vanishing normal derivative at the brane, while the odd modes vanish at the brane. We must also impose this condition on our solutions. This condition quantizes $\lambda$. The exact value must be determined numerically, however for simple cases, it can also be done analytically.

\paragraph{Example. } Let us take $x=0$ as the defect, i.e., we do not introduce a defect, but simply pretend that $x=0$ is a special place. The defect OPE then becomes simply the Taylor expansion of an operator around $x=0$. For a generalized free fields we have that the conformal family simply consists of derivatives of the operators, i.e., operators or dimensions $\Delta' = \Delta + n$. The relation between $\lambda$ and $\Delta'$ is 
\begin{align}
    \lambda = \Delta'(\Delta'-(d-1)).
\end{align}
In the case at hand we assumed a massless scalar field, so $\Delta = d$. The correct boundary condition is Dirichlet conditions at both boundaries. One can confirm explicitly by a numerical computation, that the quantization condition yields
\begin{align}
    \lambda = (d+n)(n+1).
\end{align}
In the case we present in the paper, we need to focus on even modes, since only those induce dynamical fields on the brane.

\subsection{Bound states and mode counting}
What happens if we introduce a defect and couple bulk fields to it? Before we answer this question, let us consider a somewhat simpler situation, namely that of a Randall--Sundrum brane. The aim of this section is to understand under which conditions bound states appear and which other modes we have to integrate out in order to obtain an effective theory for such modes. Other questions are about the number and orthogonality of modes. Lastly, we are interested in the holographic dictionary for the effective theory on the RS brane. 

We will take AdS in Poincare coordinates and consider a scalar field. The ansatz for the wavefunction reads
\begin{align}
    \psi(t,\vec x,z) = \iint d\omega d\vec p \; c_{\omega p }\; e^{i \omega t + i \vec p \vec x} f(z;p,\omega).
\end{align}
Using this, we want to solve the equations of motion in $d+1$ dimensions,
\begin{align}
    z^2 f''(z;p,\omega) - z (d-1) f'(z;p,\omega) + z^2 (-\omega^2 + p^2) f(z;p,\omega) + m^2 f(z;p,\omega) = 0.
\end{align}
The solutions are well known and given in terms of a linear combination of Bessel functions,
\begin{align}
    f(z;p,\omega) = c_1 z^{\frac{d}{2}} J_{\frac 1 2 \sqrt{d^2 - 4m^2}}(k z) + c_2 z^{\frac{d-1}{2}} Y_{\frac 1 2 \sqrt{d^2 - 4m^2}}(k z),
\end{align}
with $\omega^2 - p^2 = k^2$. If $\frac 1 2 \sqrt{d^2 - 4m^2}$ is not integer, we could have replaced $Y_{\frac 1 2 \sqrt{d^2 - 4m^2}}(k z) \to J_{-\frac 1 2 \sqrt{d^2 - 4m^2}}(k z)$ to obtain another set of linearly independent functions. Both, $Y_\alpha(k z)$ and $J_\alpha(kz)$ diverge as $z \to 0$, where the asymptotic boundary is located. Since we do not intend to turn on sources we drop those solutions and are left with $J_\alpha$. It obeys the identity
\begin{align}
    \int_0^\infty x J_{\alpha}(ux) J_{\alpha}(vx) dx = \frac 1 u \delta(u-v),
\end{align}
which we can use to normalize the solutions. The reason is that this integral takes the same form as the Klein Gordon norm along the $z$-direction.
\begin{align}
    \sim \int dz z^{-d+1} (\psi_1 \partial_t \psi_2 - \psi_2 \partial_t \psi_1).
\end{align}
Note that if we require regularity of the solution, we also have that $k^2 > 0$.

We now introduce a Randall--Sundrum brane at $z = z_*$. In order to understand how different boundary conditions change the number of modes, consider first the case where we impose Dirichlet conditions at the brane. It is clear already that now $k$ will be quantized. At leading order, the large $z$ expansion of $J_\nu$ is
\begin{align}
    J_\nu(k z) = \sqrt{\frac{2} {\pi k z}} \cos(k z - \frac 1 2 \nu \pi - \frac 1 4 \pi).
\end{align}
Clearly, for $k z_* \gg 1$ the zeroes are at 
\begin{align}
    k = \frac{\pi}{z_*} (n + \frac 1 2 \nu - \frac 1 4).
\end{align}
How does this change if we introduce non-trivial boundary conditions at the brane? In analogy with the gravitational case, let us pick Robin boundary conditions,
\begin{align}
    \phi'(z_*) = c \phi(z_*).
\end{align}
It is important that our boundary condition takes this form (as opposed to, say, $\phi'(z_*) = c \phi(z_*)$), since otherwise a linear combination of solutions would not be a solution anymore. Using the large $z$ expansion, we find that (assuming $c \geq \mathcal O(1)$)
\begin{align}
   \sin(k z_* - \frac 1 2 \nu \pi - \frac 1 4 \pi) = -\frac{c}{k} \cos(k z_* - \frac 1 2 \nu \pi - \frac 1 4 \pi).
\end{align}
The key point is that the modes are spaced at order $\frac 1 {z_*}$ and changes in the spectrum for a range of modes of $\Delta k < \mathcal O(1)$ (these are still order $z_* k$ modes) can be neglected. In other words: the spectrum changes with respect to the one with Dirichlet boundary conditions, but only very mildly.
\\
\\
Open question: What happens for $k z_* \sim \mathcal O(1)$?
\\
\\
Crucially, thanks to the modified boundary condition, a new state appears with $k^2 < 0$. The relevant solution is $I_\nu(k z)$. at large $k z$ this takes the form
\begin{align}
    I_\nu(k z) \sim \frac{e^{kz}}{\sqrt{2 \pi k z}}.
\end{align}
Clearly, this could never satisfy a Dirichlet boundary condition $\phi = 0$ at large $z_*$, but the Robin condition becomes at leading order
\begin{align}
    k = c.
\end{align}

In order to compare the effects, we should normalize the modes. The bound state very roughly has a norm of
\begin{align}
    \int dz z{^-d} e^{2 k z} \sim \frac{e^{2 k z}}{z^{d-1}}, 
\end{align}
so that to properly normalized wave function in the large $kz$ limit roughly becomes 
\begin{align}
   \psi \sim z_*^{\frac{d-1}{2}}\frac{e^{k(z-z_*)}}{\sqrt{2 \pi k z}}.
\end{align}
The positive energy states are roughly normalized and thus we see that evaluated at the brane, the bound state is bigger by a factor of $\sim z_*^{\frac{d-1}{2}}$.

At the asymptotic boundary, we have that
\begin{align}
    J_\nu(kz) \sim (\frac 1 2 kz)^\nu \Gamma(\nu + 1)^{-1} \\
    \psi \sim z_*^{\frac{d-1}{2}} e^{-k z_*} (\frac 1 2 kz)^\nu \Gamma(\nu + 1)^{-1}.
\end{align}
As a result of the normalization, we see that the bound state is exponentially suppressed. We conclude that by integrating out all KK modes, we produce the CFT at the asymptotic boundary together with a scalar field on the brane. However, both, the CFT at the asymptotic boundary as well as the theory on the Randall-Sundrum brane are only effective theories. The corrections to the theory on the RS brane are a power-law in the coordinate distance of the RS brane to the asymptotic boundary, while the corrections to the CFT are exponentially small.

For massless particles we expect that $k\sim 0$ and it is unclear how much of the above analysis still holds.


\subsection{The CFT on the brane}
Let us consider a fixed time, e.g. $t=0$. The bulk modes can be classified into even and odd modes under the $\mathbb Z_2$ symmetry coming from reflecting the system at the brane. A certain linear combination of even and odd modes only has support on one side of the brane, while the orthogonal linear combination has support on the other side. Furthermore, all odd modes vanish on the brane and only even induce fields.

It should be clear from the preceding section that the acceptable values of $\lambda$ and thus also the defect operator spectrum is determined by the location of the brane. Let us assume that we locate the brane very close to $\theta = \pi$. Imposing Neumann boundary conditions at the brane, one can easily see that the allowed values for $\lambda$ are very close to the allowed values for $\lambda$ had we simply quantized AdS without a brane. This means that the OPE spectrum coming from one side of the brane looks very much like the one of a CFT with a ``fake defect'' as discussed in the example above. This suggests that the theory on the brane has approximately the same light spectrum as the CFT on the asymptotic AdS boundary. Since there are theories on either side of the brane, the brane carries a second copy of the theory.

The operators of the brane CFT are given by the limiting value of the bulk fields. Let's say the brane is placed at $\epsilon_{UV} \equiv \epsilon$. We can calculate correlation functions of brane operators $\bar {\mathcal O} = \epsilon^\Delta \bar \phi$ on the brane by introducing a source to the bulk Lagrangian,
\begin{align}
    \dots + \int_{\partial} \bar J \bar {\mathcal O}.
\end{align}
Adding this term changes the Neumann boundary condition of the bulk field $\phi$ from $n \cdot \nabla \phi = 0$ to $(n \cdot \nabla \phi - J) = 0$. We thus see that in the holographic dictionary of the brane theory, the role of sources is played by the Neumann boundary condition (as opposed to Dirichlet conditions at the asymptotic AdS boundary). Correlation functions can also simply be calculated by using a modified version of the extrapolate dictionary, where we do not divide out by $z^\Delta$, but include this factor into the definition of the operators. This factor is precisely what gives the induced field its scaling dimension. If we are interested in calculating correlation functions between CFT operators and operators on the brane, the easiest way should be by just using the extrapolate dictionary.

\subsection{Adding DGP terms}
We can add additional terms to the brane, which as we will now see change the bulk solution, change the dynamics of the brane theory, and modify the defect CFT.
\subsubsection{Scalar fields}
The simplest term we can add to the brane is a source term
\begin{align}
    S_{DGP} = -\alpha \int \bar \phi,
\end{align}
where $\bar \phi$ is the induced field on the brane. This modifies the background solution, but not the equation for fluctuations around the solution. The reason is clear: Any contribution to the equations of motion of fluctuations around the classical solution must be at order $\mathcal O(\delta \phi^2)$, but the term discussed here vanishes at this order. However, what does change is the classical solution and with it the vacuum expectation value of $\mathcal O$, the operator dual to the field $\phi$. The reduced conformal invariance allows scalar one-point functions of the form 
\begin{align}
    \langle \mathcal O \rangle = \frac a {|x|}
\end{align}
and such a term is turned on by coupling $\phi$ linearly to the brane.\footnote{I have not calculated the exact coefficient.} As mentioned already above, it now becomes clear that a minimal coupling of scalar fields to the brane does change the background, but is not sufficient to create a bound state on the brane. For this to happen, we must try harder.

The next term we will discuss takes the form of a kinetic term
\begin{align}
    S_{DGP} = - \frac 1 2 \int (\partial \bar \phi)^2 + \bar m^2 \phi^2.
\end{align}
If we vary the action, this contributes an extra term to the equations of motion.
\begin{align}
    \Box \phi + \frac{\delta(\theta - \theta_B)}{\sin^2 \theta_B} (\bar \Box - \bar m^2) \bar \phi = 0.
\end{align}
If we expand $\phi$ in eigenfunctions of the D'alembert operator along the brane as before, we obtain for a mode with eigenvalue $\lambda$
\begin{align}
    \Box \phi_\lambda(\theta) + \frac{\delta(\theta - \theta_B)}{\sin^2 \theta_B}  (\lambda - \bar m^2) \bar \phi_\lambda = 0.
\end{align}
Thus, we note that a kinetic term induces a source term, which depends on the mode we are considering. Integrating this equation around $\theta_B$ will give a discontinuity in the first derivative of the field at the location of the brane
\begin{align}
    (\bar \phi'_{\lambda,L} - \bar \phi'_{\lambda,R}) +  (\lambda - \bar m^2) \bar \phi_\sigma = 0.
\end{align}
For the $\mathbb Z_2$ even solution we find
\begin{align}
        2 \partial_\theta \log \bar \phi_{\lambda} +  (\lambda - \bar m^2) = 0,
\end{align}
which quantizes $\lambda$, now however as a function of the brane location, $\bar m$ and $\lambda$ itself. Note that those terms essentially are Robin bounday conditions. The mass-term changes the boundary condition universally, while the kinetic term changes it for different modes. This discussion only affects the modes which previously had Neumann boundary conditions. The modes with Dirichlet conditions at the location of the brane are unaffected, since the source term vanishes.


The action for fluctuations $\psi$ around this solution obtains two additional terms.
\begin{align}
    \dots + \int (\bar \Box \bar \phi - \bar m^2 \bar \phi ) \psi - \frac 1 2 \int (\partial \psi)^2 - \bar m^2 \psi^2
\end{align}
It is thus easy to see that $\psi$ is also sourced and that the equations of motion for $\psi$ might allow a bound-state close to the brane.
This means, amongst other things, that the normal derivative at the brane changes which induces a change in the operator spectrum.


\subsubsection{The gravitational field}
Since the gravitational field is non-linear, the situation is slightly richer here. As discussed above, adding a simple tension term to the brane,
\begin{align}
    S_\text{brane} = - T \int \sqrt{|h|}, 
\end{align}
warps spacetime. In the case of the $\mathbb Z_2$ quotient of our setting, this can also be interpreted as moving the brane closer to (or further away from -- depending on the value of $T$) the second asymptotic boundary. Before tackling the global AdS situation we are considering, let us look at the analogous situation in Randall-Sundrum. There, metric fluctuations around this background obey an equation of the form
\begin{align}
    \left[\frac{-m^2}{2} e^{2 k |y|} - \frac 1 2 \partial_y^2 - 2 k \delta(y) + 2 k^2 \right] \psi(z) = 0.
\end{align}
Here, we just used their notation in which metric fluctuations can be split into contributions orthogonal and transverse to the brane directions, $h(x,y) = \psi(y) e^{ipx}$, $p^2 = m^2$ and $k$ is a parameter in their solution which controls the ratio between the bulk cosmological constant and the brane tension, $k^{-1} \sim L^2 G_\text{bulk} T_\text{brane} \sim \frac 1 L$. The relation between those parameters is a special feature of the Randall-Sundrum model.

An important observation is that the appearance of the $\delta$ function term is responsible for the support of a bound state. The shape of the wave function is controlled by $\frac{e^{k|y|}}{k}$, where $y=0$ is the location of the defect, and so we see that small $k$ means the bound state is wider. Note that in this case, larger $T_\brane$ means a wider bound state. The reason is that the Randall-Sundrum solution does not allow to fix $G_\mt{bulk}$ and $L$ while changing $T_\brane$.

Consider now the case where we have added an additional Einstein-Hilbert DGP term to the action of our brane. Note that we are in the bulk picture, where we have not intergrated out the directions orthogonal to the brane. The variation of the action obtains a new source-term
\begin{align}
    \int\sqrt{h}(\frac{G_{ab}}{16 \pi G_\brane}   - \frac 1 2 T h_{ab} + \frac 1 2 \Delta T h_{ab}) \delta h^{ab}.
\end{align}
In order not to change the brane position, we have that 
\begin{align}
     G_{ab} = - 8 \pi G_\brane \Delta T h_{ab}. 
\end{align}
In the RS scenario, $\Delta T =0$, since the location of the branes is independent of the brane tension. 
Let us now look at fluctuations around this background. The relevant terms appear at second order in the above equation. Recall that the factor $k$ which determines the width of the bound state is proportional the the tension $T$. The question now is whether the other two terms we introduced (and fixed by making them cancel) modify $k$. The relevant contributions come from the second order expansion of the brane action. We get the relevant second order terms by looking at the first-order contribution from 
\begin{align}
    \delta(\frac{G_{ab}}{16 \pi G_\brane}),
\end{align}
which however are just the linearized field equations. We find\footnote{http://www.physics.fau.edu/~cbeetle/PHY6938.07F/linearized.pdf} that for zero modes, i.e., those which obey Einsteins equations, the linearized equations also cancel and clearly, even for zero mode fluctuations, there is no additional contribution at the location of the brane. However, for KK modes a coupling 
\begin{align}
 - \frac{m^2}{32 \pi G_\brane}
\end{align}
is generated.



To get closer to our situation, imagine that the theory on the brane was not on a flat, but on an AdS background. In this case we would need to add an additional $\Delta T$-term. The metric on the brane obeys
\begin{align}
\delta G_{ab} = \frac 1 2 \frac{(d-1)(d-2)}{\ell_B^2} h_{ab}.
\end{align}
Fluctuations $\delta h_{ab}$ around that background obey
\begin{align}
\delta G_{ab} = \frac 1 2 \frac{(d-1)(d-2)}{\ell_B^2} \delta h_{ab},
\end{align}
where $\delta G_{ab}$ is evaluated on the AdS background. The quantity $\ell_B$ is determined by the choice of $T_0$. To add DGP terms, we modify the action by
\begin{align}
\Delta I_\brane = \frac 1 {16 \pi G_\brane} \int \sqrt h R + \int \sqrt h \Delta T.
\end{align}
Here, we finally added a $\Delta T$ which is fixed by the requirement that $\Delta I_\brane$ does not modify the location of the brane. In order to ensure this, we require $\delta \Delta I_\brane = 0$, which fixes
\begin{align}
\Delta T = \frac{(d-1)(d-2)}{16 \pi G_\brane \ell_B^2}.
\end{align}
Let us now consider bulk fluctuations. Their leading order action is given by expanding the bulk action including brane terms to second order in the fluctuations. However, the DGP-term vanishes even at second order for zero-modes,
\begin{align}
\delta^2 \Delta I_\brane =& \int \sqrt \frac 1 2 h^{ab} \delta h_{ab} \left( \frac{G_{cd}}{16 \pi G_\brane} - \frac 1 2 \Delta T h_{cd}) \delta h^{cd} \right) \\
&+ \int \sqrt{h} \left( \frac{\delta G_{cd}}{16 \pi G_\brane} - \frac 1 2 \Delta T \delta h_{cd}\right) \delta h^{cd}.
\end{align}
The first equation vanishes by definition of $\Delta T$, while the second line vanishes for on-shell $\delta h_{ab}$. Note that the on-shell modes are precisely the zero-modes of Randall-Sundrum. The KK-modes are off-shell. For them, adding DGP terms changes the coupling.


\subsubsection{Randall--Sundrum}
We can treat the RS scenario as a toy model for our case. We aim at understanding what the effect of adding a DGP term is. Without a DGP term, the graviton bulk solutions take the form
\begin{align}
\psi = A \left(|z| + \frac 1 k\right)^{1/2} J_2\left( m\left(|z| + \frac 1 k\right) \right) + B \left(|z| + \frac 1 k\right)^{1/2} Y_2\left( m\left(|z| + \frac 1 k\right) \right).
\end{align}
Close to the origin, where the condition
\begin{align}
\psi'(0) = - \frac 3 2 k \psi(0)
\end{align}
holds, we can approximate
\begin{align}
J_2 \sim \frac{m^2 (|z| + \frac 1 k)^2}{8} \\
Y_2 \sim - \frac{4}{\pi m^2 (|z| + \frac 1 k)^2} - \frac 1 \pi.
\end{align}
From this we obtain
\begin{align}
\psi(0) = A \frac {m^2} 8 \frac 1 {k^{5/2}} - B \left( \frac{4 k^{3/2}}{\pi m^2} + \frac 1 {\pi k^{1/2}} \right) \\
\psi'(0) = A \frac 5 2 \frac {m^2} 8 \frac 1 {k^{3/2}} + B \left( \frac 3 2 \frac{4 k^{5/2}}{\pi m^2} - \frac {k^{1/2}}{2 \pi } \right).
\end{align}

Adding a DGP term modifies the coupling to the brane to
\begin{align}
T_\brane \to T_\brane (1 + \frac{m^2}{16 \pi G_\brane T_\brane}) = T_\brane (1 + \frac{m^2 L^2}{12} \frac{G_\mt{bulk}}{L G_\brane}) \equiv T_\brane \lambda.
\end{align}
The term which multiplies the delta-function in the equations of motion thus gets multiplied, $k \to \lambda k$. While this leaves the bulk solution unchanged, the condition at the brane changes to
\begin{align}
\psi'(0) = - \frac 3 2 k \lambda \psi(0).
\end{align}
In terms of the solutions we find that
\begin{align}
\frac A B = \frac 8 {5 + 3\lambda} \frac{k^2}{\pi m^2} \left(12 (\lambda - 1) \frac {k^2}{m^2} + (3 \lambda + 1) \right).
\end{align}
The case without DGP terms is $\lambda = 1$ and we find
\begin{align}
\frac A B = \frac{4 k^2}{\pi m^2}.
\end{align}
The vanishing of the first term, however, suggests some fine-tuning.
If we substitute the expression for $\lambda$ we end up with ($2G_\mt{bulk} = G_\mt{eff} L$)
\begin{align}
\label{eq:ratioAB2}
\frac A B = \frac 8 {5 + 3\lambda} \frac{k^2}{\pi m^2} \left(\frac{G_\mt{eff}}{2 G_\brane} + 4 + \frac{m^2}{4 k^2} \frac{G_\mt{eff}}{2 G_\brane} \right).
\end{align}
We are now especially interested in the coupling of the fairly light KK modes, \ie $m L \ll 1$. For positive values of $G_\brane$, we see that the relative strength of both terms in the linear combination changes, 
\begin{align}
\frac A B = \frac{4 k^2}{\pi m^2} \left(1 + \frac{G_\mt{eff}}{8 G_\brane} \right).
\end{align}
However, if we choose $G_\brane$ negative and are close to regime where
\begin{align}
\frac{G_\mt{eff}}{G_\brane} \sim - 8, 
\end{align}
we can make the first two terms in the parenthesis cancel. We end up with 
\begin{align}
\frac A B \sim - \frac 4 \pi \sim \text{constant}.
\end{align}
This would mean that the ``non-normalizable'' modes are not suppressed for light masses anymore.

The next obvious question is what is the forbidden range for $\frac{G_\mt{eff}}{2 G_\brane}$? Since the effect of massive KK modes on the gravitational coupling is exponentially suppressed by a Yukawa term, we can focus on only the light KK modes. Requiring that the first two terms of equation \eqref{eq:ratioAB2} dominate and approximating the last term by $0$, we find 
\begin{align}
 \frac{4}{G_\mt{eff}} \gg - \frac{1}{2 G_\brane}.
\end{align}
Up to numerical factors of order $\mathcal O(1)$, this agrees with the condition on $G_\brane$ in the main text, which came from a lower bound on $(1+\lambda_B)$.




\subsubsection{Our setting}
Let us discuss gravity in our case in more detail. We consider slicing coordinates
\begin{align}
\frac{1}{\sin(\mu)^2} \left(d\mu + ds^2_{AdS, d}\right).
\end{align}
Let us call indices in the $d+1$-dimensional spacetime $M,N,\dots$ and in the slices $i,j,\dots$. The non-vanishing Christoffel symbols are
\begin{align}
\Gamma^\mu_{\mu\mu} = -cot(\mu) && \Gamma^i_{j\mu} = -cot(\mu) \delta^i_j && \Gamma^\mu_{ij} = cot(\mu) \tilde g_{ij} && \Gamma^i_{jk} = \tilde \Gamma^i_{jk},
\end{align}
where tilde denotes objects with respect to the metric on the slices.
Gauge freedom allows us to fix $d+1$ components on the metric fluctuation. In order make good use of the symmetries of our system we go to a Fefferman-Graham-like gauge, where we set all components of the metric involving $\mu$ to zero.
If we assume a Lagrangian of the form
\begin{align}
L = R - (d-1)(d-2) \Lambda,
\end{align}
the linearized equations of motion in the bulk can be brought to the form\footnote{Note that in transverse-traceless gauge, the equations simply take the form $\frac 1 2 \Box h_{ab} - \Lambda h_{ab} = 0$.}
\begin{align}
\frac 1 2 \Box h_{ab} + \frac 1 2 g_{ab} \nabla^d \nabla^c h_{cd} - \frac 1 2 g_{ab} \nabla_d \nabla^d h^{c}_{c} + \frac 1 2 \nabla_a \nabla_b h^c_c - \frac 1 2 \nabla^c \nabla_a h_{bc}- \frac 1 2 \nabla^c \nabla_b h_{ac} + (d-1) \Lambda (h_{ab} - \frac 1 2 g_{ab} h^c_c) = 0.
\end{align}
We will start by computing the terms one by one, everthing with indices upstairs, though. The calculations are done by expanding the derivatives with xAct and then continuing by hand.
\begin{align}
(\Box h)^{\mu\mu} &= 2 \Gamma^{\mu}_{ij} \Gamma^\mu_{kl} g^{ik} h^{jl} \\
&= 2 \cos^2 \mu \sin^2 \mu h^c_c\\
(\Box h)^{\mu i} &= \partial_l (h^{ki} \Gamma^\mu_{jk} g^{lj}) + \Gamma^\mu_{jk} (\tilde \nabla^l h^{ik}) + \Gamma^i_{ml} g^{jl} \Gamma^\mu_{kj} h^{mk} - \Gamma^k_{kl} g^{lm} \Gamma^\mu_{mj} h^{ij}\\
(\Box h)^{ij} &= 
\end{align}
\begin{align}
\nabla^d \nabla^c h_{cd} &= \tilde \nabla_i \tilde \nabla_j h^{ij} + \partial_\mu(\Gamma^\mu_{ij} h^{ij}) + h^{ij} \Gamma^\mu_{ij} \Gamma^\mu_{\mu\mu} \\
&= \tilde \nabla_i \tilde \nabla_j h^{ij} + \sin \mu \partial_\mu (\cos \mu  h^c_c)
\end{align}
\begin{align}
\nabla_i \nabla_j h^c_c &= \tilde \nabla_i \tilde \nabla_j h^c_c - \Gamma^\mu_{ij} \partial_\mu h^c_c\\
&=\tilde \nabla_i \tilde \nabla_j h^c_c - \cos(\mu)\sin(\mu) g_{ij} \partial_\mu h^c_c\\
\nabla_\mu \nabla_\mu h^{c}_c &= \partial_\mu \partial_\mu h^c_c - \Gamma^\mu_{\mu\mu} \partial_\mu h^c_c.
\end{align}
\begin{align}
\nabla^i \nabla^j h^M_M &= \tilde \nabla^i \tilde \nabla^j h^M_M + \Gamma^i_{k\mu} g^{kj} g^{\mu\mu} \partial_\mu h^M_M\\
\nabla^\mu \nabla^\mu h^{M}_M &= g^{\mu\mu} \partial_\mu(g^{\mu\mu} \partial_\mu h^M_M) + g^{\mu\mu}g^{\mu\mu} \Gamma^\mu_{\mu\mu} \partial_\mu h^M_M
\end{align}
\begin{align}
\nabla_c \nabla^\mu h^{\mu c} = \Gamma^\mu_{cf} g^{\mu\mu} (\partial_\mu h^{fc} + \Gamma^c_{\mu e} h^{fe} + \Gamma^f_{\mu d} h^{dc}) + \Gamma^\mu_{cf} g^{fe} \Gamma^\mu_{ed} h^{dc}
\end{align}
The $\mu\mu$ component of the equations of motion thus becomes
\begin{align}
\frac 1 2 \Box h^{\mu\mu} + \frac 1 2 g^{\mu\mu} \nabla^M \nabla^N h_{MN} - \frac 1 2 g^{\mu\mu} \nabla_M \nabla^M h^N_N + \frac 1 2 \nabla^\mu \nabla^\mu h^M_M - \nabla^M \nabla^\mu h^\mu_M - \frac{\Lambda}{2}(d-1) g^{\mu\mu} h^M_M = 0. 
\end{align}

The goal now is to impose on-shell-ness of the gravitons the $d$-dimensional slice at the location of the brane (these are the zero-modes) and solve for the behavior in $\mu$ direction, after we have added a source term at the boundary.

\subsection{More detailed description about how holographic works}
everything is commented out.
 %Nonetheless, we expect that the CFT on the dynamical theory can still be treated as approximately holographic. On the asymptotic AdS boundary the Dirichlet boundary condition plays the role of sources while on the Planck brane, sources are the Neumann boundary conditions. \dn{Not sure anymore whether the story is that simple} \vc{I'm not sure how Neumann boundary conditions can be treated as sources. (For Dirichlet, we can say that a source is given by the coefficient of the $\sim z^{d-\Delta}$ part of a bulk field fixed by the Dirichlet boundary conditions; For Neumann boundary conditions, we don't to fix the $\sim z^{d-\Delta}$ asymptotic value of the fields, so I'm not sure what to equate the source to.)}

% At any fixed time-slice, we can separate the low lying operator spectrum into two pieces. There are light bulk-CFT operators which are dual to the fields away from the brane. Their dimensions are almost the same as those of the CFT without the defect. Secondly, there are defect operators, which describe excitations close to the brane. This set of operators will again almost look like the operator spectrum of a holographic CFT. \dn{$G_{eff}$ characterizes the ``number of channels'' between the brane CFT and the asymptotic CFT. This is fairly obvious from the gravitational perspective: Large $G_{eff}$ corresponds to a large overlap of the graviton wavefunction with modes away from the brane, \ie the effective brane theory loses information into the bath through a non-local(?) effect. What does this correspond to in the boundary perspective?}

%\dn{Can we make an argument, where we approximate the correlator by the length of a geodesic and relate this to the boundary entropy? For example in 2D, the boundary entropy should scale with be $\tilde c_T$ and }

% \dn{Preliminary:} We can be a bit more precise: Consider the CFT correlator
%\begin{align}
%\langle \mathcal O_\mt{bulk} \mathcal O_\mt{bdry} \rangle \sim \left(\frac{c_T}{\tilde c_T}\right)^{m/2}.
%\end{align}
%The scaling can be heuristically motivated as follows. At large $N$, a single trace operator $\mathcal O$ creates a single particle state with unit amplitude. The state consists of some power, $(\tilde c_T)^{m/2}$, of the degrees of freedom with equal probability. If $c_T$ is much smaller than $\tilde c_T$, then 

%For the stress energy two-point function, this indicates that 
%\begin{align}
%\langle T_{brane} T_{bath} \rangle \sim \sqrt{c_{bulk} c_{bdry}} \frac{c_{bulk}}{c_{bdry}}.
%\end{align}
%Holographically, this becomes  
%\begin{align}
%\langle g_{bulk} g_{bulk} \rangle \sim \sqrt{\frac{1}{G_\mt{bulk} G_\mt{eff}}}  \frac{G_\mt{eff}}{G_\mt{bulk}},
%\end{align}
%where the propagator on the left is the bulk-to-brane propagator. Note that this the correlator between graviton bound state and bulk gravitons must be much smaller.

%Again, we can remove the first square-root by choosing the canonical normalization of the two-point function. The second factor is the coupling between the gravity theory and the bath. It would be interesting if we could derive this using more detailed CFT methods.  
