% !TEX root = ../lifeonbrane3.tex
%

\subsection{DGP Gravity on the Brane} \label{sec:DGP}
The previous discussion of $d=2$ motivates that it is interesting to add an intrinsic gravity term to the brane action. Here, we extend this discussion to higher dimensions, \ie extend the brane action to include an Einstein-Hilbert term. Of course, this scenario can be viewed as a version of Dvali-Gabadadze-Porrati (DGP) gravity \cite{Dvali:2000hr} in an AdS background. Hence, it combines features of both RS and DGP gravity theories. We discuss the modifications of the brane dynamics and the induced action below, but it also produces interesting modifications of the generalized entropy, as discussed in sections \ref{HEE} and appendices \ref{generalE} and \ref{bubble}.

We write the extended brane action, replacing eq.~\reef{braneaction}, as
\beq\label{newbran}
I_\mt{brane} = -(T_o-\Delta T)\int d^dx \sqrt{-\tilde{g}} + \frac{1}{16\pi \Gbr}\int d^dx \sqrt{-\tilde{g}} \tilde{R}\,.
\eeq
In general, for a fixed brane tension, the position of the brane will be modified with the additional Einstein-Hilbert term. Hence we have parametrized the full brane tension as $T_o-\Delta T$ and the contribution $\Delta T$ will be tuned to keep the position of the brane fixed. This choice will facilitate the comparison of the generalized entropy between different scenarios in the following.

As in section \ref{BranGeo}, the position of the brane can be determined using the Israel junction conditions \reef{Israel1}. Hence we begin by evaluating the brane's stress tensor,
\beq\label{stressbran}
S_{ij}\equiv -\frac{2}{\sqrt{-\tilde g}}\,\frac{\delta I_\mt{brane}}{\delta \tg^{ij}}=-\tilde{g}_{ij}(T_o-\Delta T) -\frac{1}{8\pi \Gbr}\left(\ric_{ij}-\frac12\tg_{ij}\,\ric\right)\,.
\eeq
As commented above, we choose $\Delta T$ to cancel the curvature contributions in this expression, \ie the stress tensor reduces to $S_{ij}=-T_o\,\tilde{g}_{ij}$. With this tuning, the Israel junction conditions in eq.~\reef{Israel1} are unchanged as the analysis which follows from there. Therefore the brane position and curvature remain identical to those determined in eqs.~\reef{positionbrane} and \reef{curve1}. This allows use to determine the desired tuning as
\beq\label{tune3}
\Delta T = \frac{(d-1)(d-2)}{16\pi \Gbr\,\ell_\mt{B}^2}\ 
\simeq \frac{(d-1)(d-2)}{8\pi \Gbr\,L^2}\,\veps\,.
\eeq
We have used eq.~\reef{curve2} to show that the shift in the brane tension is small in the $\veps$ expansion. 

We return to the induced gravitational action on the brane that takes the same form as in eq.~\reef{act3} but with the effective Newton's constant in eq.~\reef{Newton2} replaced by
\beq
\frac{1}{G_\mt{eff}}=\frac{2L}{(d-2)\,G_\mt{bulk}}+\frac{1}{\Gbr}\,.
\label{Newton33}
\eeq
By construction, $\ell_\mt{eff}$ and the position of the brane are unchanged. Note that the gravitational couplings in the Einstein terms and in the higher curvature interactions, \ie in the first and second lines of eq.~\reef{act3},  are now distinct. That is, $G_\mt{eff}$ no longer equals $G_\mt{RS}$. 

In the following, it will be useful to define the ratio
\beq\label{newdefs}
\lamb=\frac{G_\mt{RS}}{\Gbr}
\qquad{\rm with}\quad \frac{1}{G_\mt{RS}}=\frac{2L}{(d-2)\,G_\mt{bulk}}\,,
\eeq
where $G_\mt{RS}$ is the induced Newton's constant on an RS brane appearing in eq.~\reef{Newton2}, while the dimensionless ratio $\lamb$ controls the relative strength of the Newton's constants in the bulk and on the brane. With these definitions, the induced Newton's constant on the DGP brane, in eq.~\reef{Newton33}, can be rewritten as
\beq\label{Newton34}
\frac{1}{G_\mt{eff}}=\frac{1}{G_\mt{RS}}\(1+\lamb\)\,.
\eeq\\

Of course, one can also consider other modifications of the brane action beyond adding the Einstein-Hilbert term in eq.~\reef{newbran} -- see discussion in the next subsection and \cite{domino}. Further, we will discuss adding topological gravitational terms on the brane or in the bulk in sections \ref{HEE} and \ref{sec:discussion}. In particular, we will see in section \ref{sec:examples} that adding a Gauss-Bonnet term to the four-dimensional bulk gravity theory yields another tuneable parameter which, for a certain parameter range, makes it possible to find quantum extremal islands in the absence of black holes. 
