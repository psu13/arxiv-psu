% !TEX root = ../lifeonbrane3.tex
%

As described in the introduction, we are studying a holographic system where the boundary theory is a $d$-dimensional CFT which lives on a spherical cylinder $R\times S^{d-1}$ (where the $R$ is the time direction). Further, this CFT is coupled to a (codimension-one) conformal defect positioned on the equator of the sphere. Hence, the defect spans the geometry $R\times S^{d-2}$ and supports a ($d-1$)-dimensional CFT. The bulk description of this system involves an asymptotically AdS$_{d+1}$ spacetime with a codimension-one brane spread through the middle of the space (and extending to the position of the defect at asymptotic infinity). In this setup, the brane has an AdS$_d$ geometry and further, we consider the case in which the brane has a substantial tension and backreacts on the bulk geometry. If the brane tension is appropriately tuned, the backreaction produces Randall-Sundrum gravity  supported on the brane \cite{Randall:1999vf,Randall:1999ee}, \ie in the backreacted geometry, new (normalizable) modes of the bulk graviton are localized near the brane inducing an effective theory of dynamical gravity on the brane. In the following, we review the bulk geometry produced by the backreaction of the brane, and also the gravitational action induced on the brane.

\subsection{Brane Geometry}\label{BranGeo}

In the bulk, we have Einstein gravity with a negative cosmological constant in $d+1$ dimensions, \ie
\beq
I_\mt{bulk} = \frac{1}{16 \pi G_\mt{bulk}}\int d^{d+1}x\sqrt{-g}
\[{R}(g) + \frac{d(d-1)}{L^2} \] \,,
\label{act2}
\eeq
where $g_{ab}$ denotes the bulk metric, and we are ignoring the corresponding surface terms here \cite{PhysRevLett.28.1082,Gibbons:1976ue,Emparan:1999pm}.
We also introduce a codimension-one (\ie $d$-dimensional) brane in the bulk gravity theory. The brane action is simply given by
\beq\label{braneaction}
I_\mt{brane} = -T_o\int d^dx\sqrt{-\tilde{g}}\,.
\eeq
where $T_o$ is the brane tension and $\tilde g_{ij}$ denotes the induced metric on the brane.

Away from the brane, the spacetime geometry locally takes the form of AdS$_{d+1}$ with the curvature scale set by $L$. As described above, the induced geometry on the brane will be an AdS$_d$ space, and so it is useful to consider the following metric where the AdS$_{d+1}$ geometry is foliated by AdS$_d$ slices
\beq\label{metric}
ds^2 %= g_{ab}\,dx^{a}dx^{b}
= d\rho^2 + \cosh^2\left({\rho}/{L}\right)\, g_{ij}^{\mt{AdS}_d}\,dx^{i}dx^{j}\,.
\eeq
Implicitly here, $L$ also sets the curvature of the AdS$_d$ metric, \eg in global coordinates,
\beq\label{metric2}
g_{ij}^{\mt{AdS}_d}\,dx^{i}dx^{j}=L^2\left[-\cosh^2\!\tdr\,dt^2+d\tdr^2+\sinh^2\!\tdr\,d\Omega_{d-2}^2
\right]\,.
%-\left(1 + \frac{r^2}{L^2}\right)dt^2+\frac{dr^2}{1 + \frac{r^2}{L^2}}+r^2d\Omega_{d-2}^2\,.
\eeq
With the above choices, we approach the asymptotic boundary with $\rho\to\pm\infty$, or with fixed $\rho$ and $\tdr\to\infty$. In the latter case, we arrive at the equator of the boundary $S^{d-1}$, where the conformal defect is located. For the following, it will be convenient to replace $\rho$ with a Fefferman-Graham-like coordinate \cite{FG,Fefferman:2007rka},
\beq\label{zrho}
z = 2 L e^{-\rho/L}\,,
\eeq
with which the metric \reef{metric} becomes
\beq\label{metric3}
ds^2=\frac{L^2}{z^2}\left[dz^2 +  \left(1 + \frac{z^2}{4\,L^2}\right)^2 g_{ij}^{\AdS_d}\,dx^{i}dx^{j} \right]\,.
%ds^2=\frac{L^2}{z^2}\left[dz^2 +  \left(1 + \frac{1}{2}\frac{z^2}{L^2} + \frac{1}{16}\frac{z^4}{L^4}\right)g_{ij}^{\mt{AdS}_d}\,dx^{i}dx^{j} \right]\,.
\eeq
In these coordinates we approach the asymptotic boundary with $z\to0$ and with $z\to\infty$. Below, we will focus on the region near $z\sim 0$.

\begin{figure}[h]
	\def\svgwidth{1\linewidth}
	\centering{
		\input{Gluing2.pdf_tex}
		\caption{Panel (a): Our Randall-Sundrum construction involves foliating  with AdS$_d$ slices. Then identical portions of two such AdS$_{d+1}$ geometries are glued together along an common AdS$_d$ slice. Panel (b): The jump in the extrinsic curvature across the interface between the two geometries is supported by a(n infinitely) thin brane. The brane is represented by a green line in the figures and the bulk AdS$_{d+1}$ spacetime is blue with a $d$-dimensional CFT at the asymptotic boundary.} \label{fig:brane2}
	}
\end{figure}

As described above, the brane spans an AdS$_d$ geometry in the middle of the backreacted spacetime. Following the usual Randall-Sundrum approach, we construct the desired solution
by cutting off the AdS$_{d+1}$ geometry at some $z=\s$, and then
complete the space by gluing this geometry to another copy of itself
-- see figure \ref{fig:brane2}.
Then the Israel junction conditions (\eg see \cite{israel1966singular,Misner:1974qy}) fix $\s$ by relating the discontinuity of the extrinsic curvature across this surface to the stress tensor introduced by the brane, \ie
\beq\label{Israel1}
 \Delta{K}_{ij}-\tilde{g}_{ij}\,\Delta{K}_{k}{}^{k} = 8 \pi \Gbk\, S_{ij} = - 8 \pi \Gbk T_o\,\tilde{g}_{ij}\,,
\eeq
where $\Delta{K}_{ij}={K}^+_{ij} - {K}^-_{ij}= 2{K}_{ij}$, given the symmetry of our construction.
The extrinsic curvature is calculated as \cite{Misner:1974qy}
\beq \label{extrinsic}
{K}_{ij} =\frac{1}{2}\frac{\partial g_{ij}}{\partial n}\bigg|_{z=\s} =-\frac{z}{2L} \frac{\partial g_{ij}}{\partial z} \bigg|_{z=\s} = \frac{1}{L}\frac{4 L^2 -\s^2}{4L^2 +\s^2}\,\tilde{g}_{ij}\,,
\eeq
where $\partial_n = -\frac{z}{L}\partial_z$ is an outward directed unit normal vector. Further, we are using the notation introduced above where $\tilde g_{ij}$ corresponds to the induced metric on the surface $z=\s$, \ie on the brane. Combining eqs.~\reef{Israel1} and \reef{extrinsic}, we arrive at
\beq\label{positionbrane}
\frac{4L^2 -\s^2}{4L^2 +\s^2} = \frac{4\pi \Gbk L\, T_o}{d-1}\,.
\eeq

Now if we consider $\s\ll L$, it will ensure that the defect is well approximated by the holographic gravity theory on the brane -- see the discussion in the next subsection. In this regime, we can solve eq.~\reef{positionbrane} in a small $\s$ expansion, and to leading order, we find that
\beq\label{position}
\s^2\simeq z_\mt{0}^2 = 2L^2\left(1-\frac{4 \pi \Gbk L T_o}{d-1}\right)\,.
\eeq
Hence to achieve this result, we must tune the expression in brackets on the right to be small, \ie
\beq\label{tune}
\veps\equiv 1-\frac{4 \pi \Gbk L T_o}{d-1}\ll 1\,.
\eeq
As the notation suggests, we can think of this quantity $\veps$ as an expansion parameter in solving for the brane position from eq.~\reef{positionbrane}.
A useful check of our calculations below will come from carrying the solution to the next order, \ie
$\s^2= z_\mt{0}^2+\delta[ \s^2]_\mt{2}+\cdots$ with
\beq\label{secondorder}
\delta[ \s^2]_\mt{2} =\frac{(d-1)L}{4 \pi \Gbk T_o}\,\veps^2
\ = \frac{(d-1)L}{4 \pi \Gbk T_o}\left(1-\frac{4 \pi \Gbk L T_o}{d-1}\right)^2\,.
\eeq

To conclude, we consider the intrinsic geometry of the brane. As we noted above, the curvature scale of $g_{ij}^{\mt{AdS}_d}$ is simply $L$, and hence given the full bulk metric \reef{metric3}, we can read off the curvature scale of the surface $z=\s$ as
\beq\label{curve1}
\lb=\frac{L^2}{\s}\left(1 + \frac{\s^2}{4\,L^2}\right)\,.
\eeq
Note that since we are considering $\s/L\ll 1$, it follows that
$\lb/L\gg 1$, \ie the brane is weakly curved. Using eq.~\reef{position}, we can solve for $\lb$ to leading order in the $\veps$ expansion to find
\beq\label{curve2}
\frac{L^2}{\lb^2}\simeq
2\,\veps\ =2\left(1-\frac{4 \pi \Gbk L T_o}{d-1}\right)\,.
\eeq
It will be useful to have the following expressions for the Ricci tensor and scalar evaluated for the brane geometry, and these are compactly written using eq.~\reef{curve1} as
\beq\label{Ricky2}
\tilde{R}_{ij}(\tilde g)=-\frac{d-1}{\lb^2}\, \tilde{g}_{ij}\,,\qquad \tilde{R}(\tilde g)=-\frac{d(d-1)}{\lb^2}\,.
\eeq




\subsection{Gravitational Action on the Brane}\label{indyaction}


As noted above, following the usual Randall-Sundrum scenario \cite{Randall:1999vf,Randall:1999ee,Karch:2000ct}, new (normalizable) modes of the bulk graviton are localized near the brane in the backreacted geometry, and this induces an effective theory of dynamical gravity on the brane. The gravitational action can be determined as follows:
First, one considers  a Fefferman-Graham (FG) expansion near the boundary of an asymptotic AdS geometry \cite{FG,Fefferman:2007rka}. Then integrating the bulk action (including the Gibbons-Hawking-York surface term \cite{PhysRevLett.28.1082,Gibbons:1976ue}) over the radial direction out to some regulator surface produces a series of divergent terms, which through the FG expansion can be associated with various geometric terms involving the intrinsic curvature of the boundary metric. Usually in AdS/CFT calculations, a series of boundary counterterms are added to the action to remove these divergences, as the regulator surface is taken to infinity \cite{Emparan:1999pm}. In the present braneworld construction, the regulator surface is replaced by the brane, which remains at a finite radius, and no additional counterterms are added. Rather the `divergent' terms become contributions to the gravitational action of the brane theory, and hence the latter from previous discussions of the boundary counterterms \cite{Emparan:1999pm}, \ie
\begin{multline}\label{diver1}
I_\mt{diver}=\frac{1}{16\pi \Gbk}\int d^dx \sqrt{-\tilde{g}}\left[\frac{2(d-1)}{L}+\frac{L}{(d-2)}\tilde{R}
\right.\\
+\left. \frac{L^3}{(d-4)(d-2)^2} \left(\tilde{R}^{ij}\tilde{R}_{ij}-\frac{d}{4(d-1)}\,\tilde{R}^2\right) +\cdots\right]\,.
\end{multline}

Several comments are in order at this point: First of all, we note that the above expression is written in terms of the induced metric $\tilde g_{ij}$ on the brane (as in \cite{Emparan:1999pm}) rather than the boundary metric $\overscript{g}{0}_{ij}$ that enters the FG expansion. Using the standard results, \eg \cite{Skenderis:2002wp,deHaro:2000vlm}, we can relate the two with
\beq\label{relate}
\tilde g_{ij}(x_k) = \frac{L^2}{\s^2}\,\overscript{g}{0}_{ij}(x_k) +  \overscript{g}{1}_{ij}(x_k)+\frac{\s^2}{L^2}\, \overscript{g}{2}_{ij}(x_k)+\cdots\,,
\eeq
where the higher order terms can be expressed in terms of the curvatures of $\overscript{g}{0}_{ij}$, \eg
\beq\label{oneg}
\overscript{g}{1}_{ij} = -\frac{L^2}{d-2}\left(R_{ij}\big[\overscript{g}{0}\big] -\frac{\overscript{g}{0}_{ij}}{2(d-1)}\,R\big[\overscript{g}{0}\big]\right)\,.
\eeq
In other words, the two metrics are related by a Weyl scaling and a field redefinition. Further, we see a factor of $(d-2)$ appearing in the denominator of the second term, \ie the Einstein-Hilbert term, in eq.~\reef{diver1}. Hence this expression only applies for $d\ge3$ and must be reevaluated for $d=2$, which we do in section \ref{sec:two-d}. Similar factors, as well as a factor of $d-4$, appear in the denominator of the third term, which again indicates that this expression must be reconsidered for $d=4$.

In any event, the gravitational action on the brane is given by combining the above expression with the brane action \reef{braneaction},
\beq\label{totaction}
I_\mt{induced} = 2\, I_\mt{diver} +  I_\mt{brane}\,,
\eeq
where the factor of two in the first term accounts for integrating over the bulk geometry on both sides of the brane.
The combined result can be written as
\beqa
I_\mt{induced}&=&\frac{1}{16 \pi G_\mt{eff}}\int d^{d}x\sqrt{-\tilde{g}}
\[\frac{(d-1)(d-2)}{\ell_\mt{eff}^2} + \tilde{R}(\tilde{g})\right]
\labell{act3}\\
&&\qquad\quad
+\frac{1}{16 \pi G_\mt{RS}}\int d^{d}x\sqrt{-\tilde{g}}\left[ \frac{L^2}{(d-4)(d-2)}\(\tilde{R}^{ij}\tilde{R}_{ij}-
\frac{d}{4(d-1)} \tilde{R}^2\)+\cdots\]\,,
\nonumber
\eeqa
where
\beq
\frac{1}{G_\mt{eff}}=\frac{1}{G_\mt{RS}}=\frac{2\,L}{(d-2)\,G_\mt{bulk}}\,,
\qquad\qquad
\frac{1}{\ell_\mt{eff}^2}=\frac{2}{L^2}\left(1-\frac{4 \pi \Gbk L T_o}{d-1}\right)\,.
\label{Newton2}
\eeq
In the present discussion $G_\mt{eff}$ and $G_\mt{RS}$ are equal, but by adding terms to the brane action this can change. We will explain this in section \ref{sec:DGP}. Comparing eqs.~\reef{curve2} and \reef{Newton2}, we see that $\ell_\mt{eff}$ (which sets the cosmological constant term in $I_\mt{induced}$) precisely matches the leading order expression for the brane curvature $\lb$. Hence if we only consider the first two terms in eq.~\reef{act3}, the resulting Einstein equations would reproduce the leading expression (in the $\veps$ expansion) for the curvatures in eq.~\reef{Ricky2}. Further, it is a straightforward exercise to show that if the contribution of the curvature squared terms is also included in the gravitation equations of motion, the curvature is shifted to precisely reproduce the $\veps^2$ term in eq.~\reef{Ricky2}. Hence rather than using the Israel junction condtions, we could determine the position of the brane in the backreacted geometry by first solving the gravitational equations of the brane action \reef{act3} and then finding the appropriate surface $z=\s$ with the corresponding curvature. More generally, the fact that these two approaches match was verified by \cite{deHaro:2000wj},\footnote{See also earlier discussions, \eg \cite{Shiromizu:1999wj,Verlinde:1999fy,Gubser:1999vj}.} which argued the bulk Einstein equations combined with the Israel junction conditions are equivalent to the brane gravity equations of motion.\footnote{We note that the brane graviton acquires a small mass through interactions with the CFT residing there \cite{Karch:2000ct,Karch:2001jb,Porrati:2001gx}. However, this mass plays no role in the following as it is negligible in the regime of interest, \ie $L/\ell_\mt{eff}\ll 1$ -- see further discussion in section  \ref{face}. This point was emphasized in \cite{Geng:2020qvw}.}

Of course, the gravitational approach only provides an effective approach in the limit that $\ell_\mt{eff}\gg L$ since otherwise the contributions of the higher curvature terms cannot be ignored.
For example, if the curvatures are proportional to $1/\ell_\mt{eff}^2$ at leading order, then the curvature squared term is suppressed by a factor of $L^2/\ell_\mt{eff}^2$ relative the first two terms. Similarly the higher order curvature terms denoted by the ellipsis in eq.~\reef{act3} are further suppressed by a further factor of $L^2/\ell_\mt{eff}^2$ for each additional curvature appearing these terms. From eq.~\reef{curve2}, we can write $\frac{L^2}{\ell_\mt{eff}^2}=2\veps$ and hence we see that the gravitational brane action and the resulting equations of motion can be organized in the same small $\veps$ expansion discussed in the previous section.\footnote{Note that we have distinguished the gravitational couplings in the Einstein terms and in the higher curvature interactions, \ie in the first and second lines of eq.~\reef{act3}, even though $G_\mt{eff}=G_\mt{RS}$ here. However, this distinction will become important in section \ref{sec:DGP}.}

Recall that we can give a holographic description of this system involving (two weakly interacting copies of) the boundary CFT living on the brane. However, this CFT has a finite UV cutoff because the brane resides at a finite radius in the bulk, \eg see \cite{deHaro:2000vlm,Emparan:2006ni,Myers:2013lva}. The action \reef{diver1} is then the induced gravitational action resulting from integrating out the CFT degrees of freedom. The UV cutoff is usually discussed in the context of the boundary metric $g^{\ssc (0)}_{ij}$, where the short distance cutoff would be given by $\delta\simeq\s$.  However, recall that the gravitational action \reef{act3} is expressed in terms of the induced metric $\tilde g_{ij}$ and so the conformal transformation in eq.~\reef{relate} yields $\tilde\delta\simeq L$ for this description of the brane theory. Therefore the $\veps$ expansion corresponds to an expansion in powers of the short distance cutoff, \ie $\veps\sim\tilde \delta^2/\ell_\mt{eff}^2$.
