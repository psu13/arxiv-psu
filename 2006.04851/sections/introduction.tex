% !TEX root = ../lifeonbrane3.tex
%

Almost half a century ago, it was discovered that black holes behave as quantum objects, with an associated temperature, entropy and other thermodynamic properties \cite{Hawking:1974sw,Hawking:1974rv,Hawking:1976de,Bekenstein:1972tm,Bekenstein:1973ur}. One realization of these ideas is the Bekenstein-Hawking (BH) formula, which states that the black hole entropy is a quarter of its horizon area measured in Planck units, \ie $S_\mt{BH}=A/4 G_\mt{N}$. These concepts gained a wider scope in the context of the AdS/CFT correspondence, where the Ryu-Takayanagi (RT) prescription \cite{Ryu:2006ef,Ryu:2006bv,Hubeny:2007xt,Rangamani:2016dms} applies the same geometric expression to extremal bulk surfaces in evaluating the entanglement entropy for generic subregions on the boundary theory. Indeed, this was later derived as a special case of the generalized gravitational entropy in \cite{Lewkowycz:2013nqa}.

However, as pointed out by Hawking early on \cite{Hawking:1976ra}, a standard semiclassical analysis seemingly leads to an inconsistency in describing the time evolution of black holes. If a pure state of matter collapses to form a black hole, which is then allowed to completely evaporate via Hawking radiation, the final quantum state appears to be mixed, contradicting unitary evolution. This is the black hole information paradox.  On the other hand, arguments from the AdS/CFT correspondence suggest that unitarity should remain valid, \eg \cite{Polchinski:2016hrw,Harlow:2014yka}. There, one expects that after an initial rise of the entanglement entropy of the Hawking radiation, subtle correlations between the quanta emitted at early and late times lead to a purification of the final state and a decrease in the late-time entropy. This qualitative behaviour of the entropy is known as the Page curve \cite{Page:1993wv} -- see also \cite{Harlow:2014yka}.

As emphasized with the generalized second law \cite{Bekenstein:1974ax} (see also \cite{Wall:2009wm,Wall:2011hj}), the geometric BH entropy is naturally combined with the entanglement entropy of quantum fields outside the event horizon to produce a finite quantity known as the generalized entropy. In the context of holographic entropy, this leads to an extension of the RT prescription to include quantum corrections in the bulk \cite{Faulkner:2013ana,Engelhardt:2014gca}
\begin{align}
	\label{eq:sgen_intro}
	 S_\EE(\xR) = {\rm min}\left\{\extr\,
 S_\gen(\xV)\right\}={\rm min}\left\{\extr
  \( \frac{A(\xV)}{4 G_\mt{N}} + S_\mt{QFT}\)\right\}\,,
\end{align}
where $\xV$ is a bulk surface homologous to the boundary subregion $\xR$, while $S_\mt{QFT}$ is the entropy of the quantum fields on a partial Cauchy surface extending from $\xV$ to $\xR$ on the asymptotic boundary. The surface which extremizes the generalized entropy in the above expression is then referred to as a Quantum Extremal Surface (QES) \cite{Engelhardt:2014gca}. Further, the `min' indicates that in the situation where there is more than one extremal surface, one chooses that which yields the minimum value for $S_\gen(\xV)$. 

This approach produced some surprising new insights with holographic models of black hole evaporation \cite{Almheiri:2019psf, Penington:2019npb, Almheiri:2019hni}. In particular, at late stages in the evaporation, the quantum term can compete with the classical BH contribution in eq.~\reef{eq:sgen_intro} to produce new saddle points for the QES, which could describe the late-time phase of the Page curve. Perhaps the biggest surprise is that the Page curve can be reproduced from saddlepoint calculations in semi-classical gravity, \ie in a situation where the details of the black hole microstates or of the encoding of information in the Hawking radiation are still not revealed. Further, the evaluation of the entanglement entropy of the Hawking radiation is seen to be encapsulated by the so-called `island rule' \cite{Almheiri:2019hni},
\beq
 S_\EE(\xR) ={\rm min}\left\{\extr
  \(  S_\mt{QFT}(\braneR \cup \text{islands}) + \frac{A\(\partial(\islands)\)}{4 G_\mt{N}}\)\right\}\,.
 \labell{wonderA}
\eeq
That is, the entropy of the radiation collected in a nongravitating reservoir is evaluated as the contributions from the quantum fields in the reservoir but possibly also on a quantum extremal island (QEI) in the gravitating region, \ie a separate region near the black hole, as well as a geometric BH contribution from the boundary of the island. In the early phase of the Hawking evaporation, extremizing this expression yields the empty set for the island, \ie there is no island. However, at late times, a QEI appears to reduce the radiation's entropy and yields the expected late-time behaviour of the Page curve. These results have sparked further progress with a variety of new investigations, \eg \cite{Almheiri:2019yqk, Almheiri:2019psy, Almheiri:2019qdq, Penington:2019kki, Akers:2019nfi, Rozali:2019day, Chen:2019uhq, Bousso:2019ykv, Gautason:2020tmk, Hartman:2020swn, Marolf:2020xie, Hollowood:2020cou, Anegawa:2020ezn, Hashimoto:2020cas, Sully:2020pza, Balasubramanian:2020hfs, Alishahiha:2020qza, Geng:2020qvw, Krishnan:2020oun}. %\rcm{More? Note \cite{Hashimoto:2020cas} is in higher dimensions.}

In this paper, we aim to explore the island formula \eqref{wonderA} in further generality. Recall that the latter was motivated by the `doubly holographic' model presented in \cite{Almheiri:2019hni}, who in turn began with the two-dimensional model of \cite{Almheiri:2019psf}. The latter consists of a bath, \ie a two-dimensional CFT on a half line, and a pair of quantum mechanical systems, which are assumed to be holographically dual to Jackiw-Teitelboim (JT) gravity on AdS$_2$ coupled to the same CFT as in the bath. Hence if the quantum mechanical systems begin in a thermofield double state, the dual description is given by a two-sided AdS$_2$ black hole. If the boundary of the bath is then coupled to one of the quantum  systems, \ie to the asymptotic boundary of one side of the black hole, the black hole begins to evaporate as Hawking radiation leaks into the bath. Now the insight of \cite{Almheiri:2019hni} was to examine the case where the two-dimensional CFT is itself \textit{holographic}, and so can be replaced with a locally AdS$_3$ bulk. The boundary of this bulk geometry has two components: the asymptotically AdS boundary, on which the bath lives, and the Planck brane, where the JT gravity is supported. This third perspective on the system has the advantage that the generalized entropy in eq.~\reef{eq:sgen_intro} or \reef{wonderA} is realized completely geometrically. That is, the entanglement entropy of the boundary CFT is computed by RT surfaces in the three-dimensional bulk, and the geometric BH contribution is given by the usual expression for JT gravity. Further, calculations in this doubly holographic model produce the expected Page curve, with RT surfaces ending on the Planck brane manifesting the island rule \eqref{wonderA}. 

In the present paper, we generalize this doubly holographic model to higher dimensions as follows (see also figure \ref{fig:threetales}): We consider a $d$-dimensional holographic CFT coupled to a codimension-one conformal defect. As usual, the gravitational dual corresponds to an asymptotically AdS$_{d+1}$ spacetime, containing a codimension-one brane anchored on the asymptotic boundary at the position of the defect. The gravitational backreaction of the brane warps the geometry creating localized graviton modes in its vicinity, as per the usual Randall-Sundrum (RS) scenario \cite{Randall:1999ee,Randall:1999vf,Karch:2000ct}. Hence at sufficiently long wavelengths, the system can then also be described by an effective theory of Einstein gravity coupled to (two copies of) the holographic CFT on the brane, all coupled to the CFT on the static boundary geometry.\footnote{Some tuning of the parameters characterizing the brane is required to achieve this effective description. Note that the fact that the RS gravity on the brane has a finite cutoff \cite{Randall:1999ee,Randall:1999vf} makes conspicuous that this is only an effective theory.} To better emulate the previous model with JT gravity \cite{Almheiri:2019hni}, we also consider introducing an intrinsic Einstein term to the brane action, analogous to the construction of Dvali, Gabadadze and Porrati (DGP) \cite{Dvali:2000hr}.\footnote{Without the DGP term, our construction resembles that in \cite{Rozali:2019day} in many respects. Our model resembles the setup in \cite{Almheiri:2019hni} even more closely if we make a $\mathbb Z_2$ orbifold quotient across the brane.}  In any event, this more or less standard holographic model can be viewed from three perspectives in analogy with \cite{Almheiri:2019hni}: the bulk gravity perspective, with a brane coupled to gravity in an asymptotically AdS$_{d+1}$ space; the boundary perspective, with the boundary CFT coupled to a conformal defect; and the brane perspective, with a region where the holographic CFT couples to Einstein gravity and another region where the same CFT propagates on a fixed background geometry.

From the bulk gravity perspective, the entanglement entropy is realized in a completely geometric way in terms of the areas of RT surfaces, with a contribution in the bulk and another contribution on the brane. That is, we have an extension of the usual RT prescription with
\beq
 S_\EE(\xR) = {\rm min}\left\{\extr\,
 S_\gen(\xV)\right\}={\rm min}\left\{\extr
  \(
  \frac{A(\xV)}{4 G_\bulk} + \frac{A(\xV \cap {\rm brane})}{4 G_\brane}\)\right\}\,,
 \label{eq:sad0}
\eeq
where again, where $\xV$ is a bulk surface homologous to the boundary subregion $\xR$ (see figure \ref{fig:first}). Note that the brane contribution seems natural here, we will argue for its presence by extending the derivation in \cite{Myers:2010tj}. In contrast to eq.~\reef{eq:sgen_intro}, we are not considering quantum field contributions in the AdS$_{d+1}$ bulk. However, from the brane perspective, the usual RT term, \ie the first term on the right-hand side of eq.~\reef{eq:sad0}, is interpreted as the leading planar contribution of the boundary CFT to $S_\EE(\xR)$, and the island rule \eqref{wonderA} is realized in situations where the RT surface cross over the brane. 

\begin{figure}[h]
	\def\svgwidth{1\linewidth}
	\centering{
		\input{first_vincentsEdit.pdf_tex}
		\caption{A sketch of our holographic setup illustrating the various elements appearing in eq.~\eqref{eq:sad0}, which manifests the island rule in our analysis. }
                \label{fig:first} 	}
\end{figure}

We emphasize the underlying \textit{simplicity} of our holographic model. In particular, the elements of construction are more or less standard, and the entropies are evaluated with the geometric formula for holographic entanglement entropy. Hence we generalize the island rule to any number of dimensions but also cast it in a framework where many of its features follow simply from the properties of the RT prescription -- and in fact, can be understood analytically. In particular, we will be able to address several issues which appeared puzzling in \cite{Almheiri:2019hni}. Other recent analyses in higher dimensions were undertaken numerically in \cite{Almheiri:2019psy}, in an effective theory in flat space \cite{Hashimoto:2020cas} and using a Randall-Sundrum-inspired toy model in \cite{Geng:2020qvw}.

The remainder of this paper is organized as follows: In section \ref{sec:branegravity}, we begin by studying a certain class of $d$-dimensional branes embedded in AdS$_{d+1}$. We show how the Randall-Sundrum gravity induced on the brane is equivalent to the bulk description of the brane embedded in the higher dimensional geometry. In section \ref{face}, we elucidate the different holographic perspectives of this system as described above, \ie we can describe the system as a $d$-dimensional boundary CFT coupled to a conformal defect, a $d$-dimensional CFT which contains a region with dynamical gravity, or a ($d$+1)-dimensional theory of gravity coupled to a codimension-one brane. Section \ref{HEE} investigates the relation between the appearance of quantum extremal islands using eq.~\reef{wonderA} and the bulk picture using eq.~\reef{eq:sad0} with RT surfaces crossing the brane. In the same section, we present some explicit calculations explicitly illustrating appearance of such QEI for $d=3$. Section \ref{sec:discussion} concludes with a discussion of our results. In appendix \ref{generalE}, we extend  the arguments in \cite{Myers:2010tj} to support the appearance of the brane contribution to the generalized entropy in eq.~\reef{eq:sad0}. Appendix \ref{bubble} examines a surprising class of spherical RT surfaces, which can be supported at finite size by the brane. 

We must note that most of our discussion is quite general and not necessarily linked to the physics of black holes. In fact, the explicit calculations in section \ref{sec:examples} evaluate the entanglement entropy of entangling regions (with components on either side of the conformal defect) in the vacuum state of the boundary system.\footnote{Further, let us note that the formation of QEIs on branes in the `Einstein gravity regime' require us to introduce somewhat unconventional couplings. That is, we must consider a negative Newton's constant on the brane and/or a Gauss-Bonnet interaction in the four-dimensional bulk gravity.} This illustrates that QEIs are not a feature exclusive to the black hole information problem, but may play a role in more general settings where gravity and entanglement are involved. Nevertheless, it is indeed possible within our model to also discuss black holes. In a forthcoming publication \cite{QEI}, we will apply the methods developed here to the case of eternal black holes coupled to a thermal bath in higher dimensions, similar to \cite{Almheiri:2019yqk}. 



%%% Local Variables:
%%% mode: latex
%%% TeX-master: "../lifeonbrane3"
%%% End:
