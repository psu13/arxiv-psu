% !TEX root = ../lifeonbrane3.tex
%

In sections \ref{sec:two-d} and \ref{sec:DGP}, we introduced intrinsic gravitational terms to the brane action. Following \cite{Almheiri:2019hni},\footnote{See also \cite{Almheiri:2019psf, Almheiri:2019yqk, Chen:2019uhq, Penington:2019kki, Almheiri:2019qdq}.} we assumed that these terms contribute to the generalized entropy, \eg see eq.~\reef{eq:sad0} or \reef{eq:sad}.
In this appendix, we present a extended version of an argument in \cite{Myers:2010tj}, which will support this assumption and our formula for generalized entropy. 

As in the main text, we begin with a $d$-dimensional holographic CFT on $R\times S^{d-1}$ with a conformal defect on the equator of the sphere, sweeping out $R\times S^{d-2}$. On a fixed time-slice, we choose an entangling surface $\SCFT$ which divides the sphere into two equal halves along a maximal $S^{d-2}$ which lies orthogonal to the conformal defect. Now we wish to determine the entanglement entropy between the two halves
of the system, as sketched in figure \ref{fig:defect}. Recall that with the geometric approach \cite{Callan:1994py}, we must evaluate the partition function on a (Euclidean) background geometry with an infinitesimal conical defect. In order to construct a symmetric geometry where introducing such a defect is well-defined, we perform a Wick rotation on the boundary time (\ie $t_\mt{E}=it$) and then conformally transform the Euclidean background metric to a round $S^{d}$ with the conformal defect lying on a maximal $S^{d-1}$ on this background. Now $\SCFT$ remains a maximal $S^{d-2}$ which runs orthogonal to the defect and pierces the latter on a $S^{d-3}$. With this construction, there is a rotational symmetry in the two dimensions orthogonal to $\SCFT$. To evaluate the corresponding entanglement entropy, we construct $\mathcal{M}_{1-\eps}$, the `$n$-fold cover' with $n=1-\eps$, by introducing an infinitesimal conical defect at $\SCFT$. The entanglement entropy is then given by
\beq\label{entro9}
S = \lim_{\eps\to0}\left( \frac{\partial\ }{\partial\eps}+1 \right)\log
Z_{1-\eps}\,,
\eeq
where $Z_{1-\eps}$ is the partition function of the holographic CFT on the covering space $\mathcal{M}_{1-\eps}$. Of course, the latter has a dual description in terms of the bulk gravity, and using the usual saddle point approximation, eq.~\reef{entro9} becomes \cite{Myers:2010tj}
\beq\label{entropylimit}
S =-\lim_{\epsilon\to0}\Big(\frac{\partial}{\partial \epsilon} + 1\Big) I_{E, 1-\epsilon}\,,
\eeq 
where $I_{E,1-\epsilon}$ is the Euclidean bulk action evaluated on the appropriate dual solution.
%
%\begin{figure}[h]
%	\def\svgwidth{0.3\linewidth}
%	\centering{
%		\input{defect.pdf_tex}
%		\caption{A timeslice of our $d$-dimensional CFT setup with entangling surface $\Sigma_\mt{CFT}$. An infinitesimal conical defect $\Sigma_\xR$ runs through the bulk and intersects the brane at $\sigma_\xR$.} \label{fig:defect}
%	}
%\end{figure}
\begin{figure}[h]
	\def\svgwidth{0.6\linewidth}
	\centering{
		\input{defect31.pdf_tex}
		\caption{A timeslice of our $d$-dimensional CFT setup with entangling surface $\Sigma_\mt{CFT}$ and an equatorial conformal defect (the green line). In the right panel, one dimension is suppressed relative to the left panel.} \label{fig:defect}}
\end{figure}

Setting $n=1$ for a moment, the bulk dual of $\mathcal{M}_{1}$ is simply the Euclidean version of the geometry constructed in section \ref{BranGeo}, which we denote $\widetilde{\mathcal{M}}_{1}$. Recall the boundary geometry is $S^{d}$ and the conformal defect runs around a maximal $S^{d-1}$. In the bulk, the geometry is locally EAdS$_{d+1}$ everywhere away from the brane, and the brane has a EAdS$_d$ geometry which extends out to the conformal defect at the asymptotic boundary and with the curvature scale given by eq.~\reef{curve1} -- see figure \ref{fig:defect2}. Now the  entangling surface $\SCFT$ on the asymptotic AdS boundary is the boundary of an extremal surface $\Sigma_\xR$ in the bulk, which runs  straight across the bulk solution and has a EAdS$_{d-1}$ geometry with curvature scale $L$.  This surface pierces the brane at a right angle and the intersection, another extremal surface $\sigma_\xR$, has the geometry of a EAdS$_{d-2}$ with curvature scale $\ell_\mt{B}$ -- see figure \ref{fig:defect2}. Now because of the symmetry of this configuration, the rotational symmetry about the entangling surface in the boundary extends to a rotational symmetry about $\Sigma_\xR$ in the bulk. Hence we can calculate the entanglement entropy with the same geometric approach as we applied in the boundary. That is, we construct $\widetilde{\mathcal{M}}_{1-\eps}$, the $n$-fold cover (with $n=1-\eps$) of the bulk solution  with a infinitesimal conical defect at $\Sigma_\xR$ and by extension, at $\sigma_\xR$ on the brane. 

\begin{figure}[h]
	\def\svgwidth{0.6\linewidth}
	\centering{
		\input{defect32.pdf_tex}
		\caption{A cross-section of the Euclidean geometry $\widetilde{\mathcal{M}}_{1}$. The orange semicircle and its complement along a time slice represent the orange shaded region of figure \ref{fig:defect} and its complement. The rotation that keeps $\Sigma_\mt{CFT}$ fixed represents euclidean time. An infinitesimal conical defect $\Sigma_\xR$ runs through the bulk and intersects the brane at $\sigma_\xR$.} \label{fig:defect2}
	}
\end{figure}

That is, the angle around $\Sigma_\xR$ runs through a range $2\pi(1-\epsilon)$. Now  \cite{Fursaev:1994ea,Fursaev:1995ef} developed a description of such conical defects in which the singular geometry is replaced by a `regulator' geometry where the region
around the conical singularity is smoothed out. Applying their key result, we can write the bulk Riemann tensor  as a ``smooth" contribution away from $\Sigma_\xR$, the conical defect, and a singular order $\epsilon$ contribution at $\Sigma_\xR$,\footnote{This order $\eps$ contribution is universal, whereas the details of the regulator come into play at order $\eps^2$ and higher.}
\beq\label{separation}
^{(\epsilon)}R^{ab}{}_{cd} = R^{ab}{}_{cd}+2\pi \epsilon\, {\varepsilon}^{ab}{\varepsilon}_{cd}\,\delta_{\Sigma_\xR}\,,
\eeq 
where ${\varepsilon}_{ab}$ is the Euclidean volume form in the two-dimensional transverse space to $\Sigma_\xR$, and $R^{ab}{}_{cd}$ is the ``smooth" curvature piece. The $\delta_{\Sigma_\xR}$ is a two-dimensional delta function defined in \cite{Myers:2010tj}. The conical singularity intersects the brane at $\sigma_\xR$ and so we have a similar decomposition for the Riemann tensor on the brane,
\beq\label{separation2}
^{(\epsilon)}\tilde R^{ij}{}_{k\ell} = \tilde R^{ij}{}_{k\ell}+2\pi \epsilon \,\tilde{\varepsilon}^{ij}\tilde{\varepsilon}_{k\ell}\,\delta_{\sigma_\xR}\,.
\eeq 

Now recall that our aim is to evaluate the Euclidean action in eq.~\reef{entropylimit}. This action is the sum of the Euclidean versions\footnote{Note that the difference in signs in going between Minkowski and Euclidean signatures \cite{Myers:2010tj}.} of the bulk and brane actions in eqs.~\reef{act2} and \reef{newbran} (or perhaps eq.~\reef{JTee} for $d=2$), as well as the associated boundary terms. Equipped with eqs.~\reef{separation} and \reef{separation2}, it can be shown that in the limit of small $\epsilon$ that the Euclidean action can be expanded as
\beqa\label{epsilonaction}
I_{E,1-\epsilon}& =& (1-\epsilon)I_{E,1} +
 \int_\mt{bulk}\!\! d^{d+1}x \sqrt{g}\, 2\pi \epsilon {\varepsilon}^{ab}{\varepsilon}_{cd}\,\delta_{\Sigma_\xR}\, \frac{\partial \mathcal{L}_\mt{E,bulk}}{\partial R^{ab}{}_{cd}}\\
 %
&&\qquad+\int_\mt{brane}\!\!\!\! d^{d}x \sqrt{\tilde g} \, 2\pi \epsilon \tilde{\varepsilon}^{ij}\tilde{\varepsilon}_{k\ell}\,\delta_{\sigma_\xR}\,  \frac{\partial \mathcal{L}_\mt{E,brane}}{\partial \tilde{R}^{ij}{}_{k\ell}} +\mathcal{O}(\epsilon^2)\,.
\eeqa
Noting the symmetry of our configuration, \ie the curvatures are constant everywhere along the surfaces $\Sigma_\xR$ and $\sigma_\xR$,
we then find the entropy in eq.~(\ref{entropylimit}) is given by
\beq\label{fish9}
S = -2\pi \frac{\partial \mathcal{L}_\mt{E,bulk}}{\partial R^{ab}{}_{cd}} {\varepsilon}^{ab}{\varepsilon}_{cd} \int_{\Sigma_\xR} d^{d-1}x \sqrt{h}-2\pi \frac{\partial \mathcal{L}_\mt{E,brane}}{\partial \tilde{R}^{ij}{}_{k\ell}} \tilde{\varepsilon}^{ij}\tilde{\varepsilon}_{k\ell} \int_{\sigma_\xR} d^{d-2}x \sqrt{h'}\,,
\eeq 
where $h$ and $h'$ are the induced metrics along the $\Sigma_\xR$ and $\sigma_\xR$, respectively. Hence we see that there is a contribution of the Wald entropy from both the bulk action and the brane action. Further, let us note that various signs appear upon analytically continuing back to Lorentzian spacetime, \ie in the Lagrangian and the transverse volume form \cite{Myers:2010tj}. 


For the case where the Einstein-Hilbert action appears both in the bulk and on the brane, as in eqs.~\reef{act2} and \reef{newbran}, we find the formula for the generalized entropy \reef{fish9} becomes
\beq\label{fish88}
S = \frac{A(\Sigma_\xR)}{4 G_\mt{bulk}}+ \frac{A(\sigma_\xR)}{4 G_\mt{brane}}\,,
\eeq
as given in equation \reef{eq:sad}. The present derivation only applies to special symmetric configuration, as in \cite{Myers:2010tj}. The symmetry of this configuration preculdes finding any extrinsic curvature terms in eq.~\reef{fish9}, as would be expected for the Dong entropy \cite{Dong:2013qoa}. We note however that no such terms would correct eq.~\reef{fish88} for the generalized entropy coming from the Einstein-Hilbert term. It would, of course, be interesting to extend our derivation to more general configurations involving bulk DGP branes, along the lines of \cite{Lewkowycz:2013nqa} or \cite{Dong:2016hjy}.

%As a final note here, let us observe that our derivation of eq.~\reef{fish88} involved explicitly introducing a  comparing this singular approach to \cite{Lewkowycz:2013nqa} as Callan-Wilczek \cite{Callan:1994py} is to Gibbons-Hawking \rcm{ref??}
