% !TEX root = ../lifeonbrane3.tex
%



\rcm{lots of new references to read}

In order to derive the island formula, the appearance of wormholes in the replica trick is a crucial ingredient.
In the two-dimensional models involving JT gravity studied so far \cite{Almheiri:2019qdq,Penington:2019kki}, the existence of wormholes follows from the fact that JT gravity is defined by ensemble averaging over an ensemble of Hamiltonians.\footnote{For example, JT gravity emerges as the low energy effective description of the SYK model \cite{}, or has a UV complete definition in terms of a matrix model \cite{Saad:2019lba}.} In fact, by assuming ensemble averaging, one can easily reproduce many features \cite{Marolf:2020xie} demonstrated in the 2d models. 

However, it is generally assumed that in higher dimensions gravity on asymptotically AdS spacetimes is dual to a single, unitary theory.\footnote{The best known example is the duality between $\mathcal N=4$ super Yang-Mills theory and supergravity (or more precisely Type IIB string theory) on AdS${}_5\times S^5$.} In fact, in the higher dimensional case one would expect replica wormholes to be absent in any local theory. This can be seen by considering a theory on a higher-dimensional spacetime which couples to gravity in a subregion $U_\text{grav}$, and lives on a fixed background in the complement $U_\text{bath}$. If we want to compute the density matrix of a region $A \subset U_\text{bath}$, we are instructed to integrate out all degrees of freedom in the gravitating region $U_\text{grav}$. If we take products of the reduced density matrix -- like we would do in the replica trick -- no integral over the gravitating region is left to be performed and no Euclidean wormholes appear. More generally, in two dimensions, replica Wormholes appear in calculations of the spectral form factor and yield to a non-factorizability of products of the partition function. However, for a CFT${}_d$ the partition function is a number and a product of many partition functions clearly must factorize.

This opens the question if and how islands and replica wormholes appear in the higher dimensional case. As we have seen, in our model, they are readily explained from the bulk perspective, but in order to go beyond the case of holographic matter, it is important to also understand how they arise in the brane picture. Here, we will explain a possible mechanism suggested by the model discussed in this paper, and its relation to ensemble averaging.

As discussed in this paper, the description of the brane theory as a local CFT with a cutoff coupled to gravity in some region of space is only an effective one. In fact, we see from the bulk picture that the degrees of freedom in $U_\text{bath}$ and $U_\text{grav}$ are not independent. A particularly clear sign is the fact that under certain conditions parts of the gravitating region $\mathcal I_A \subset U_\text{grav}$ lie in the entanglement wedge of bath subregions $A \subset U_\text{bath}$. This signalled the appearance of an island on the brane picture. In this case, as is clear from EW reconstruction, specifying the state in $A$ also specifies the state in the island $\mathcal I_A$. In the presence of a brane, new bulk modes appear, which can be identified with the now-dynamical metric and sources of other primary operators on the brane. Specifying the state in $\mathcal I_A$ thus in particular means to specify the states of the metric and all other sources. The description that takes into account the link between degrees of freedom in the bath and the gravitating region corresponds to the full quantum gravity description of the state of \cite{Almheiri:2019yqk}.

In an effective, local and semi-classical approximation this link between the bath and the gravitating region is ignored so that $A$ and its associated island $\mathcal I_A$ are treated as independent subregions. If we are interested in a reduced density matrix $\rho^\text{s.c.}_A$ of a $A$ in the semi-classical approximation, we can obtain it from a fully quantum gravity density matrix $\rho^\text{q.g.}_A$, containing information of $A$ and $\mathcal I_A$, by treating the degrees of freedom in $\mathcal I_A$ as independent and tracing over them. Now, consider taking a product of $n$ density matrices $(\rho^\text{q.g.}_A)^n$. If we want to go to the semi-classical picture, we need to trace over the state on $\mathcal I_A$,
\begin{align}
(\rho^\text{s.c.}_A)^n = \tr_{\mathcal I_A}((\rho^\text{q.g.}_A)^n)
\end{align}
This correlates the states of the metric, quantum fields and other sources in any of the replica copies of $\mathcal I_A$ in the product. Moreover, since the trace also runs over modes which are associated with sources on the brane, this procedure effectively looks like ensemble averaging the sources on $\mathcal I_A$. This ensemble averaging then has an immediate description in terms of including Euclidean wormholes in the path integral.\\
