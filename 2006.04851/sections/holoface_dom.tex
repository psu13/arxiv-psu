% !TEX root = ../lifeonbrane3.tex
%

Our setup can be interpreted from three different `holographic' perspectives, which are analogous to the three descriptions of \cite{Almheiri:2019hni}, suitably generalised to arbitrary dimensions. A set of analogous descriptions for gravity on a brane in higher dimensions was discussed in the context of the Karch-Randall model \cite{Karch:2000ct}, and in fact, these are the models discussed here with the addition of the DGP term \reef{newbran}. In this section we review each of the dual descriptions, and explore their relation. 

First, consider the {\it bulk gravity perspective} corresponding to the geometric picture portrayed in section \ref{BranGeo}: we have an AdS$_{d+1}$ bulk region where gravity is dynamical, containing a DGP brane with tension running through the middle of the spacetime  -- see figure \ref{fig:threetales}a. The induced geometry on the brane is AdS$_d$. In the second picture, we integrate out the bulk action from the asymptotic boundary where gravity is frozen up to the brane, giving rise to Randall-Sundrum gravity \cite{Randall:1999vf,Randall:1999ee,Karch:2000ct} on the brane. From the resulting {\it brane perspective}, the CFT$_d$ is then supported in a region with dynamical gravity (\ie the brane) and another non-dynamical one (\ie the asymptotic boundary) -- figure \ref{fig:threetales}b. Finally, the third description makes full use of the AdS/CFT dictionary, by using holography {along} the brane. This {\it boundary perspective} describes the system as a CFT$_d$ coupled to a conformal defect that is located at the position where the brane intersects the asymptotic boundary -- see figure \ref{fig:threetales}c. 

A holographic system was presented in \cite{Almheiri:2019hni} to describe the evaporation of two-dimensional black holes in JT gravity. This system has three descriptions analogous to those above. Of course, it also includes certain elements that we did not introduce in our model, \ie end-of-the-world branes to give a holographic description of conformal boundaries separating various components  \cite{Takayanagi:2011zk,Fujita:2011fp} and performing a $\mathbb Z_2$ orbifold quotient across the Planck brane, \ie the brane supporting JT gravity. However, the essential ingredients are the same as above. The boundary perspective in \cite{Almheiri:2019hni} describes the system as a two-dimensional holographic conformal field theory with a boundary, at which it couples to a (one-dimensional) quantum mechanical system -- figure \ref{fig:threetales}f. With the brane perspective, the quantum mechanical system is replaced by its holographic dual, the Planck brane supporting JT gravity coupled to another copy of the two-dimensional holographic CFT -- see figure \ref{fig:threetales}e. Finally, the bulk gravity perspective replaces the holographic CFT with three-dimensional Einstein gravity in an asymptotically AdS$_3$ geometry. Because of the $\mathbb Z_2$ orbifolding, the latter effectively has two boundaries, the standard asymptotically AdS boundary and the dynamical Planck brane -- see figure \ref{fig:threetales}d.

This initial model \cite{Almheiri:2019hni} raised a number of intriguing puzzles. For example, as emphasized in \cite{Almheiri:2019yqk}, implicitly two different notions of the radiation degrees of freedom are being used: one being the semi-classical approximation and the other one in the purely quantum theory. Here, we will explain some details of the higher dimensional construction which allow us to provide a resolution of several of these questions in section \ref{sec:discussion}.

\begin{figure}[h]
	\def\svgwidth{0.9\linewidth}
	\centering{
		\input{ThreeTales_EditcopyJS4.pdf_tex}
		\caption{This figure shows the relation between a time-slice in our construction and the holographic setup of \cite{Almheiri:2019hni}. The top row illustrates three perspectives with which the system discussed here can be described, while the bottom row displays the analogous descriptions for the model in \cite{Almheiri:2019hni}. The comparison can be made more precise by performing a $\mathbb Z_2$ orbifold quotient across the bulk brane/conformal defect in the top row. \\
			\textbf{a.} Bulk gravity perspective, with an asymptotically AdS$_{d+1}$ space (shaded blue) which contains a co-dimension one Randall-Sundrum brane (shaded grey).\\ 
			\textbf{b.} Brane perspective, with dual CFT$_d$ on the asymptotic boundary geometry (blue) and also extending on the AdS$_d$ region (shaded green) where gravity is dynamical.\\
			\textbf{c.} Boundary perspective, with the holographic CFT$_d$ on $S^{d-1}$ (blue) coupled to a codimension-one conformal defect (green).\\
			\textbf{d.} AdS$_3$ formulation with two boundary components: the flat asymptotic boundary (straight black line) and a ``Planck brane'' (curved black line) with an AdS$_2$ geometry.\\ 
			\textbf{e.} The holographic CFT extends over a region with a fixed metric (blue) and an AdS$_2$ region with JT gravity (green).\\
			\textbf{f.} The microscopic description as a two-dimensional BCFT (blue) coupled to a quantum mechanical system at its boundary (green).\\
		}
		\label{fig:threetales}
	}
\end{figure}




\begin{figure}[h]
	\def\svgwidth{1\linewidth}
	\centering{
		\input{bulkmodes_JSedit.pdf_tex}
		\caption{This figure illustrates the spatial profile of the first few normalized graviton modes in the presence of a large tension brane, and a $\mathbb Z_2$ orbifolding across the brane. We use the spatial coordinate $\mu$, related to $\rho$ in eq.~\reef{metric} by $\cot \mu = \sinh\rho/L$. The tension is adjusted such that the location of the brane is at $\mu = \mu_\mt{B}$ with $\mu_\mt{B}\lesssim\pi$. As discussed in the main text, the presence of the brane creates new bulk modes (orange), which are highly localized at the brane, and which play the role of a (nearly massless) graviton on the brane. The remaining bulk modes appear as KK modes in the brane theory.   }
                \label{fig:boundstate} 	}
\end{figure}


\paragraph{Bulk gravity perspective:} As discussed in section \ref{BranGeo}, the system has a bulk description in terms of gravity on an asymptotically AdS$_{d+1}$ spacetime containing a codimension-one brane, which splits the bulk into two halves -- see figure \ref{fig:threetales}a. The brane is characterized by the tension $T_o$ and also the DGP coupling $1/\Gbr$, introduced in eqs.~\eqref{braneaction} and \eqref{newbran}, respectively. We can use the Israel junction conditions \reef{Israel1} to determine the location of the brane as embedded in the higher dimensional space. The backreaction causes warping around the brane, and after a change of coordinates, tuning the brane tension can be understood as moving the brane further into a new asymptotic AdS region, as seen in eq.~\reef{positionbrane} or \reef{position}. For large brane tension, \ie with $\veps \ll 1$, the spectrum of graviton fluctuations in the bulk is almost unchanged with respect to the modes in empty AdS space. However, a new set of graviton states also appear localized at the brane \cite{Randall:1999vf,Randall:1999ee}, as illustrated figure \ref{fig:boundstate}. These are created by the nonlinear coupling of gravity to the brane. Unlike in the Randall-Sundrum model with a flat or de Sitter brane, the new graviton modes are not actually massless on the brane, but merely very light states whose wavefunction peaks around the brane \cite{Karch:2000ct,Karch:2001jb}. The remaining bulk graviton modes appear as a tower of  Kaluza-Klein states, from the point of view of the theory on the brane, with masses of ${\cal O}(1/\ell_\mt{eff})$ set by the curvature scale of the $d$-dimensional AdS geometry on the brane. These results have been studied in quite some detail \cite{Karch:2000ct,Karch:2001jb,Porrati:2001db,Miemiec:2000eq,Schwartz:2000ip,Porrati:2001gx} for Randall-Sundrum branes, but it is interesting to examine how the spectrum is modified by the DGP term \reef{newbran}. We will make some qualitative statements about this question below, but leave a detailed quantitative discussion and the interpretation of this mechanism from the point of view of the CFT for future work \cite{domino}. 
%A similar set of localized modes appears for each bulk field, however, their boundary conditions at the brane differ from that of the graviton.\rcm{??} 





\paragraph{Brane perspective:} This second perspective, discussed in section \ref{indyaction}, effectively integrates out the spatial direction between the asymptotically AdS boundary and the brane to produce an effective action \reef{act3} for Randall-Sundrum/DGP gravity on the brane, with the new localized graviton state playing the role of the $d$-dimensional graviton. Hence, we are left with a $d$-dimensional theory of gravity coupled to (two copies of) the dual CFT on the brane -- see figure \ref{fig:threetales}b. As discussed in the description of the bulk perspective, amongst the new localized bulk modes, we have an almost massless graviton but also a tower of massive Kaluza-Klein states with masses of ${\cal O}(1/\ell_\mt{eff})$.  In section \ref{indyaction}, we demonstrated the consistency between the bulk gravity perspective and the brane perspective by observing how the equations of motion of the new effective action fix the brane position in the ambient spacetime. Of course, the bulk physics is also dual to the dual CFT on the asymptotic AdS$_{d+1}$ boundary, and so this description is completed by coupling the gravitational and CFT degrees of freedom on the brane to the  CFT on the fixed boundary geometry. We refer to that latter as the {\it bath} CFT. Next, we discuss how different parameters in the brane perspective are related to bulk parameters. 

There are four independent parameters which characterize the gravitational theory on the brane: the curvature scale $\ell_\mt{eff}$, the effective Newton's constant $G_\mt{eff}$, the central charge of the boundary CFT $\cT$, and the effective short-distance cutoff $\tilde\delta$. These emerge from the bulk theory through the four parameters characterizing the latter: the bulk curvature scale $L$, the bulk Newton's constant $\Gbk$, the brane Newton's constant $\Gbr$ and the brane tension $T_o$.\footnote{Recall that $\Delta T$ is determined by these parameters in eq.~\reef{tune3}, as well as eqs.~\reef{positionbrane} and \reef{curve1}.} From eq.~\reef{Newton2}, we see that $\ell_\mt{eff}$ is determined by a specific combination of $T_o$, $\Gbk$ and $L$. Similarly, $G_\mt{eff}$ is determined by $\Gbr$, $\Gbk$ and $L$ in eq.~\reef{Newton33}. The central charge of the boundary CFT is given by the standard expression $\cT\sim L^{d-1}/G_\mt{bulk}$, \eg see \cite{Buchel:2009sk}. 

Lastly, as discussed in section \ref{indyaction},  the theory on the brane comes with a short-distance cutoff $\tilde \delta$ \cite{deHaro:2000vlm,Emparan:2006ni,Myers:2013lva} at which the description of the brane theory in terms of (two copies of) the boundary CFT coupled to Einstein gravity breaks down. Following a standard bulk analysis, one would see that correlators of local operators (with appropriate gravitational dressings) now longer exhibit the expected CFT behaviour at short distances of order 
\beq
\tilde\delta_\mt{CFT}\sim L\,.
\label{ctoff2}
\eeq
We denote this cutoff with the subscript `CFT' to emphasize that the description of the matter degrees of freedom on the brane as a local $d$-dimensional CFT is failing at distances smaller than this short-distance cutoff. However, we stress that there is another scale $\tilde \delta_\mt{GR}$, which is the distance at which the approximation of Einstein gravity on the brane breaks down. The simple parameter counting above shows that this cannot be an independent scale. For the brane perspective, the true cutoff $\tilde \delta$ where the description in terms of the dual CFT coupled to Einstein gravity fails is 
\beq\label{ctoff}
\tilde \delta={\rm max}\left\{\tilde \delta_\mt{CFT}\,,\ \tilde \delta_\mt{GR}\right\}\,.
\eeq
We now discuss how $\tilde \delta_\mt{GR}$ is related to the other scales in the brane theory.

Recall that integrating out the bulk degrees of freedom produces a series of higher curvature terms in the effective action \reef{act3}, and hence demanding that $d$-dimensional Einstein gravity provides a good approximation of the brane theory introduces constraints. The suppression of these higher curvature corrections requires that the ratio $L/\ell_\mt{eff}$ be small. However, if we examine eq.~\reef{act3} carefully and note the distinction $G_\mt{eff}\ne G_\mt{RS}$, then suppressing the curvature-squared terms requires that
\beq
\frac{1}{1+\lamb}\,\frac{L^2}{\ell_\mt{eff}^2}
\ll 1\,,
\label{lmfao}
\eeq
using eq.~\reef{Newton34}. Note that for fixed bulk and boundary curvature scales, this implies a lower bound on the DGP term, such that $\lamb$ cannot be arbitrarily close to $-1$. For a pure RS brane with no additional DGP gravity, \ie $\lambda_b=0$, we conclude that the cutoff below which we find Einstein gravity coincides with the CFT cutoff $\tilde\delta_\mt{GR} \sim \tilde\delta_\mt{CFT}\sim L$. More generally then, the above expression suggests that the DGP term \reef{newbran} affects a shift producing a new short-distance cutoff for gravity,
\beq
\tilde\delta_\mt{GR} \sim \frac{L}{\sqrt{1+\lamb}}\ \sim
\frac{\tilde\delta_\mt{CFT}}{\sqrt{1+\lamb}}\,.
\label{haiku}
\eeq
Hence the true cutoff \reef{ctoff} depends on the sign of $\lamb$ -- we return to this point below. We should note that this result only applies for $d>4$. For $d=4$, the coefficient of the curvature-squared term is logarithmic in the cutoff, while for $d=2$ or $3$, this interaction is not associated with a UV divergence. 


While the above are UV effects, there are also IR effects resulting from having a large number of matter degrees of freedom propagating on the brane, as explained in \cite{Dvali:2007hz,Dvali:2007wp,Reeb:2009rm}. The usual regime of validity for QFT in semiclassical gravity lies at energy scales below the Planck mass, or at distance scales larger than $G_\eff^{1/(d-2)}$. However, the boundary CFT has a large number of degrees of freedom, as indicated by the large $\cT$, and hence the semiclassical description of gravity in fact breaks down much earlier. A direct way to see this breakdown \cite{Dvali:2007wp} is to consider the computation of the (canonically normalized) graviton two-point function. In the high energy approximation, \ie ignoring the AdS geometry, we have here:\footnote{This propagator argument can also be applied for the higher curvature terms discussed above. For example, the curvature-squared terms gives a perturbative correction: $\langle h(p) \,h(-p)\rangle \sim p^{-2}\[1 + \frac{L^2}{1+\lamb}\, p^{2}+\cdots\]$. Hence this approach yields the same result for the cutoff in eq.~\reef{haiku}.}
\beq
\langle h(p)\, h(-p)\rangle \sim p^{-2}\[1 + \cT\, G_\eff\, p^{d-2}+\cdots\]\,.
\label{Dvali1}
\eeq
The leading correction arises from a diagram involving the external gravitons coupling to the CFT stress tensor two-point function. We see that such corrections are only suppressed relative to the `tree-level' result for momenta below a cutoff scale of order $(\cT G_\eff)^{-1/(d-2)}$. 
For our model, the gravitational theory of the brane can therefore only be treated semiclassically for distance scales larger than
\beq
\tilde\delta_\mt{GR} \sim (\cT G_\eff)^{1/(d-2)} \sim \frac{L}{(1+\lambda_b)^{1/(d-2)}}\sim \frac{\tilde\delta_\mt{CFT}}{(1+\lambda_b)^{1/(d-2)}}\,.
\label{Dvali2}
\eeq
Again, for a pure RS brane with $\lambda_b=0$, the cutoffs for Einstein gravity and the CFT agree, yielding $\tilde\delta \sim L$. However, the addition of a DGP gravity term modifies the cutoff, but in a manner distinct from eq.~\reef{haiku}, produced by the higher curvature terms. Note that the above result applies for $d\ge 3$.

The distinction between these two cutoffs indicates that these are really two different physical phenomena contributing to the breakdown of Einstein gravity in the brane perspective. Note that $\lamb>0$, in both eqs.~\reef{haiku} and \reef{Dvali2}, the effect is to produce a shorter cutoff scale, however, the second limit \reef{Dvali2} is the first to contribute (where we are assuming $d>4$). However, this result is smaller that $\tilde\delta_\mt{CFT}$ and hence from eq.~\reef{ctoff}, we find
\beq\label{ctoffplus}
\lamb>0\ \ :\qquad \tilde \delta \sim \tilde\delta_\mt{CFT}\sim L\,.
\eeq
On the other hand with $\lamb<0$, the cutoff $\tilde\delta_\mt{GR}$ is pushed to larger distance scales. In this case, eq.~\reef{haiku} is the first to modify the gravitational physics on the brane as we move to smaller distances. Further since this result is now larger than the CFT cutoff, in this regime, eq.~\reef{ctoff} yields
\beq\label{ctoffminus}
\lamb<0\ \ :\qquad \tilde \delta \sim \tilde\delta_\mt{GR}\sim \frac{L}{\sqrt{1+\lamb}}\,.
\eeq


Let us also note that the latter effect, \ie CFT corrections to the graviton propagator, are also responsible for the mass of the brane graviton \cite{Porrati:2001db}. It is interesting to note that if we take the high energy limit of the corrections to the graviton propagator, eq.~\eqref{Dvali1}, we can estimate a mass correction for low energy gravitons mode of roughly
\begin{align}
\label{eq:mass_correction}
\frac{\cT G_\eff}{\ell_\mt{eff}^{d}} \sim \frac{1}{(1 + \lambda_b)\, \ell_\mt{eff}^{\,2}} \left( \frac{L}{\ell_\mt{eff}}\right)^{d-2},
\end{align}
where we have substituted the $d$-dimensional AdS scale as a lower bound on the momentum. The scaling with the d-dimensional cosmological constant $- \frac 1 {\ell^2_\text{eff}}$ agrees with predictions in the Karch-Randall model \cite{Miemiec:2000eq, Schwartz:2000ip}. However, we caution the reader that the above argument by which we obtained the scaling is heuristic at best. Importantly, whether or not the graviton actually obtains a mass correction depends on the boundary conditions of the matter fields in AdS and can therefore not be determined by a local argument alone
\cite{Porrati:2001db}. However, taking eq.~\eqref{eq:mass_correction} at face value, we also see that a negative DGP coupling increases the mass scale, and vice versa for a positive coupling. This can be confirmed explicitly from bulk calculations \cite{domino}. 

\paragraph{Boundary perspective:} As the preceding discussion has made clear, the theory obtained by integrating out the bulk between the asymptotic boundary and the brane, has an effective description of the brane in terms of a local $d$-dimensional CFT coupled to Einstein gravity up to some cutoff \reef{ctoff}. However, the standard rules of AdS/CFT also allow for a fully microscopic description of the system in terms of the boundary theory. This is obtained by integrating out the bulk -- including the brane -- and the result  is given by the bath CFT on the fixed $d$-dimensional boundary geometry coupled to a ($d-1$)-dimensional conformal defect (positioned where the brane reaches the asymptotic boundary, \ie the equator of the boundary sphere) -- see figure \ref{fig:threetales}c. 

The bath CFT is characterized by the central charge $\cT\sim L^{d-1}/G_\mt{bulk}$, while the defect is characterized by its defect central charge $\tilde{c}_\mt{T}\sim \ell_\mt{eff}^{d-2}/G_\mt{eff}$. We note that in the absence of a DGP term, increasing the brane tension increases the defect central charge $\tilde c_\mt{T}$. Further, we note that the ratio of these two charges is given by
\beq\label{eq:eq:large_central_charges}
\frac{\tilde c_\mt{T}}{\cT} \sim \left(\frac{\ell_\mt{eff}}{L}\right)^{d-2} (1+\lambda_b) \,.
\eeq
Following the standard AdS/CFT dictionary, the ratio $\ell_\mt{eff}/L$ also  translates to a ratio of couplings in the defect and bath CFTs,\footnote{Remember that the AdS/CFT dictionary tells us that $G_N\sim \ell_\mt{AdS}^{d-1}/ N_{dof}$ and $\lambda_\text{Hooft}\sim (\ell_\mt{AdS}/\ell_\mt{s})^{d}$.}
\beq\label{ratou}
{\tilde \lambda}/{\lambda}\sim{\ell_\mt{eff}}/{L}  \,.
\eeq
Since we do not have a particular string construction in mind here, $\lambda$ should be thought of some positive power of the `t Hooft coupling of the bath CFT, while $\tilde\lambda$ will be some (different) positive power of the analogous coupling for the defect CFT.

Now the parameters in this boundary description must be constrained if we want to be in the regime where the brane perspective is valid. In particular, the latter requires that the brane curvature scale must be much larger than the effective cutoff, \ie
\beq
\ell_\mt{eff}/\tilde\delta \gg1\,.
\label{ratou3}
\eeq
Now as described above, the cutoff has a separate form depending on whether $\lamb$ is positive or negative. Eq.~\reef{ctoffplus} applies for $\lamb>0$, which then yields $\ell_\mt{eff}/L \gg1$. Hence we must have ${\tilde \lambda}/{\lambda}\gg1$ and also $\tilde c_\mt{T}/\cT\gg1$ since $1+\lamb>1$ in this case. Similarly for $\lamb<0$, combining eqs.~\reef{ctoffminus} and \reef{ratou3} yields $\ell_\mt{eff}/L \gg1/\sqrt{1+\lamb}$. In this case, $1+\lamb<1$ and it is straightforward to again show that the ratios must be constrained in the same manner. Hence for either sign of $\lamb$, we have 
\beq
{\tilde \lambda}/{\lambda}\gg1\qquad{\rm and}\qquad \tilde c_\mt{T}/\cT\gg1\,.
\label{ratou4}
\eeq
The large ratio of the central charges can also be heuristically understood requiring that energy and information are only leaking very slowly from the dynamical gravity region into the bath \cite{Rozali:2019day}. It has been argued that this ratio also sets the Page time \cite{Rozali:2019day}. With the boundary perspective, this can be understood as a requirement which ensures that the degrees of freedom on the defect and the CFT only slowly mix.

Lastly, the $d$-dimensional graviton can be understood as a field dual to the lightest operator appearing in the boundary OPE expansion of the CFT stress energy tensor \cite{Aharony:2003qf}. At weak coupling, one would naively assume that the lightest operator has dimension $\Delta = d$. However, due to strong coupling effects it becomes possible that a negative anomalous dimension of roughly $-1$ is obtained, so that the corresponding operator can act as the holographic dual to a $d$-dimensional graviton. The mass of the lightest state then signals that the anomalous dimension is not quite $-1$, such that the dimension of the boundary operator dual to the graviton is $\Delta \geq d-1$. 


%%% Local Variables:
%%% mode: latex
%%% TeX-master: "../lifeonbrane3"
%%% End:
