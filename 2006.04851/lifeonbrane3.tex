\documentclass[onecolumn,amsmath,amssymb,nofootinbib,12pt]{article}
\usepackage{../helpers/jheppub}
\usepackage{ifpdf}

% \usepackage[bb=boondox]{mathalfa}

%\usepackage[notcite,notref]{showkeys}

%\documentclass[11pt,a4paper,oneside]{article}
%[preprint]{revtex4}
%Changed subfigure to subcaption
\usepackage{graphicx,subcaption}
\graphicspath{ {figures/} }
\usepackage{amsfonts}
\usepackage{amsmath}
\usepackage{amssymb}
\usepackage{mathtools}
\usepackage{tensor}
\usepackage{epsfig}
% \usepackage{BOONDOX-cal}
%\usepackage[usenames]{color}
\usepackage{pdfpages}
\usepackage{bbm}
\usepackage{graphicx,epstopdf}
\usepackage[numbers]{natbib}
\usepackage[makeroom]{cancel}
\usepackage{hyperref}
\usepackage{tikzsymbols}
\hypersetup{pdftex,colorlinks=true,allcolors=blue}
%\usepackage[margin=1.4in]{geometry}
\usepackage{array}
\usepackage[export]{adjustbox}

\usepackage[normalem]{ulem}

\usepackage[papersize={7.0in, 10.0in}, left=.5in, right=.5in, top=1in, bottom=.9in]{geometry}
\linespread{1.05}
\sloppy
\raggedbottom
\pagestyle{plain}

%####################################
% VINCENT'S MACROS
\newcommand{\Rbb}{\mathbb{R}}
\newcommand{\Lcal}{\mathcal{L}}
\newcommand{\Zbb}{\mathbb{Z}}
\newcommand{\Ocal}{\mathcal{O}}

\newcommand{\Brane}{{\rm brane}}
\newcommand{\brane}{\mt{brane}}
\newcommand{\bulk}{\mt{bulk}}
\newcommand{\ren}{\mt{ren}}
\newcommand{\area}{A}
\newcommand{\gen}{\mt{gen}}
\newcommand{\eff}{\mt{eff}}
\newcommand{\dive}{\mt{div}}
\newcommand{\CFT}{\mt{CFT}}
\newcommand{\hyperF}{\tensor[_2]{F}{_1}}
\newcommand{\CFTR}{\mathbf{R}}
\newcommand{\braneR}{R}
\newcommand{\AdS}{\mathrm{AdS}}
\newcommand{\bulkbndry}{\sigma}
\newcommand{\bulkreg}{\Sigma}
\newcommand{\RTsurf}{\Sigma_\xR} %{\bulkbndry_\CFTR}
\newcommand{\RTbulkreg}{\bulkreg_\CFTR}
\newcommand{\AdSdmetric}{{g}^{\AdS_d}} %{\bar{g}^{\AdS_d}}
\newcommand{\RTmetric}{h}
\newcommand{\QESmetricNoz}{\mathfrak{h}}
\newcommand{\twoDmetric}{f}
\newcommand{\curvK}{\mathcal{K}}
\newcommand{\inducedK}{\tilde{\curvK}}
\newcommand{\inducedR}{\tilde{R}}
\newcommand{\inducedg}{\tilde{g}}
\newcommand{\inducednabla}{\tilde{\nabla}}
\newcommand{\inducedBox}{\tilde{\Box}}
\newcommand{\UV}{\mathrm{UV}}
\newcommand{\simUnder}[1]{\underset{#1}{\sim}}
\newcommand{\simUV}{\simUnder{\mathrm{UV}}}
\newcommand{\extr}{{\rm ext}}
\newcommand{\islands}{\mathrm{islands}}
\newcommand{\RT}{\mt{RT}}
\newcommand{\DGP}{\mt{DGP}}
\newcommand{\IR}{\mt{IR}}
\newcommand{\cT}{c_\mt{T}}
\newcommand{\overscript}[2]{\overset{\scriptscriptstyle{(#2)}}{#1}{}}
%\DeclareMathOperator*{\extr}{ext}
%####################################

\newcommand{\xR}{\mathbf{R}}
\newcommand{\xV}{\mathbf{V}}
\newcommand{\EE}{\mt{EE}}
\newcommand{\tilh}{{\tilde h}}


\newcommand{\RN}[1]{%
	\textup{\uppercase\expandafter{\romannumeral#1}}%
}
\newcommand{\labell}[1]{\label{#1}} %{\label{#1}} %
\newcommand{\comment}[1]{\textcolor{red}{\bf [[[#1]]]}}

\newcommand{\del}{\partial}
\newcommand{\vac}{\text{vac}}
\newcommand{\surf}{\text{surf}}
\newcommand{\BH}{\text{BH}}
%\newcommand{\bulk}{\text{bulk}}
\newcommand{\tot}{\text{tot}}
\newcommand{\corner}{\text{corner}}
\newcommand{\mx}{\text{max}}
\newcommand{\past}{\text{past}}
\newcommand{\future}{\text{future}}
\newcommand{\jnt}{\text{jnt}}
\newcommand{\phys}{\rm phys}
%\usepackage{musixtex}
%\usepackage{lilyglyphs}

\definecolor{limegreen}{rgb}{0.2, 0.8, 0.2}
\definecolor{deepcarrotorange}{rgb}{0.91, 0.41, 0.17}
\definecolor{darkviolet}{rgb}{0.58, 0.0, 0.83}
\definecolor{cyan}{rgb}{0.0, 0.72, 0.92}
\definecolor{plum}{rgb}{0.67, 0.0, 0.55}

\newcommand{\schmm}[1]{\textcolor{red}{\bf #1}}
\newcommand{\rcm}[1]{\textcolor{red}{\bf [[Rob: #1]]}}
\newcommand{\josh}[1]{\textcolor{plum}{\bf [[Joshua: #1]]}}
\newcommand{\iar}[1]{\textcolor{blue}{\bf [[IR: #1]]}}
\newcommand{\vc}[1]{\textcolor{magenta}{\bf [[Vincent: #1]]}}
\newcommand{\dn}[1]{\textcolor{limegreen}{\bf [[Dominik: #1]]}}

\newcommand{\hd}[1]{\noindent{\bf #1}\ }

\newcommand{\eg}{{\it e.g.,}\ }
\newcommand{\ie}{{\it i.e.,}\ }
\newcommand{\reef}[1]{(\ref{#1})}
\newcommand{\ssc}{\scriptscriptstyle}
\newcommand{\mt}[1]{\textrm{\tiny #1}}

%\newcommand{\s}{\sigma}
\newcommand{\s}{z_\mt{B}}
\newcommand{\sir}{z_\mt{IR}}
\newcommand{\lir}{\ell_\mt{IR}}
\newcommand{\lb}{\ell_\mt{B}}
\newcommand{\lamb}{\lambda_{b}}
\newcommand{\lgb}{\lambda_\mt{GB}}
\newcommand{\thb}{\theta_\mt{CFT}}
\newcommand{\zeb}{\zeta_\mt{CFT}}
%\newcommand{\thb}{\theta_{\partial\CFTR}}
%\newcommand{\zeb}{\zeta_{\partial\CFTR}}
\newcommand{\SCFT}{\Sigma_\mt{CFT}}
%\newcommand{\SCFT}{\partial\CFTR}
\newcommand{\puv}{P_\mt{UV}}
\newcommand{\pb}{P_\mt{B}}
\newcommand{\pbo}{P_\mt{B,0}}
\newcommand{\veps}{\varepsilon}
\newcommand{\sgen}{S_\mt{gen}}
\newcommand{\Agen}{\mathcal{A}_\mt{gen}}


\newcommand{\CC}{\mathcal{C}}
\newcommand{\mS}{\mathcal{S}}
\newcommand{\Gn}{G_\mt{N}}
\newcommand{\eps}{\epsilon}
\newcommand{\tk}{{\tilde k}}
\newcommand{\mA}{\mathcal{A}}
\newcommand{\mB}{\mathcal{B}}
\newcommand{\mD}{\mathcal{D}}
\newcommand{\mM}{\mathcal{M}}
\newcommand{\mK}{{\cal K}} %{\mathcal{K}}
\newcommand{\mR}{{\cal R}} %{\mathcal{R}}
\newcommand{\ric}{{\tilde R}}

\newcommand{\ads}[1]{{\mt{AdS}_{\tiny #1}}}
\newcommand{\tg}{{\tilde g}}
\newcommand{\tR}{{\tilde R}}
\newcommand{\tdr}{{\tilde r}}


\newcommand{\beq}{\begin{equation}}
\newcommand{\eeq}{\end{equation}}
\newcommand{\beqa}{\begin{eqnarray}}
\newcommand{\eeqa}{\end{eqnarray}}

\newcommand{\bea}{\begin{eqnarray}}
\newcommand{\eea}{\end{eqnarray}}
\newcommand{\nn}{\nonumber}
\newcommand{\pa}{\partial}
\newcommand{\vk}{{\vec k}}

\newcommand{\pen}{\frak{a}}
\newcommand{\tx}{\tilde{x}}
\newcommand{\tp}{\tilde{p}}
\newcommand{\hx}{\hat{x}}
\newcommand{\hp}{\hat{p}}
\newcommand{\hX}{{\widehat X}}
\newcommand{\hV}{{\widehat V}}
\newcommand{\hW}{{\widehat W}}
\newcommand{\hZ}{{\widehat Z}}

\newcommand{\lp}{\left(}
\newcommand{\rp}{\right)}
\newcommand{\op}{\mathcal{O}}
\newcommand{\cev}[1]{\reflectbox{\ensuremath{\vec{ \reflectbox{\ensuremath{#1}}}}}}
\newcommand{\tr}{{\rm tr}}
\newcommand{\bx}{\mathbf{x}}

%Some useful commands for QM
\newcommand{\bra}[1]{\left< #1 \right|}
\newcommand{\ket}[1]{\left| #1 \right>}
\newcommand{\expVal}[1]{\left< #1 \right>}
\newcommand{\braket}[2]{\left<#1|#2\right>}



\renewcommand{\(}{\left(}
\renewcommand{\)}{\right)}
\renewcommand{\[}{\left[}
\renewcommand{\]}{\right]}
\newcommand{\dslash}{\delta^{\!\!\!\!-}\!}
\def\Tr{{\text{Tr}}}

\newcommand{\psT}{\psi_\mt{T}}
\newcommand{\psR}{\psi_\mt{R}}
\newcommand{\wrr}{\omega_{\mt R}}
\newcommand{\id}{\mathbbm{1}}
\newcommand{\zero}{\mathbbm{0}}
\newcommand{\mC}{\mathcal{C}}
\newcommand{\mO}{\mathcal{O}}
\newcommand{\bR}{\mathbb{R}}
\newcommand{\mH}{\mathbb{H}}
\newcommand{\mV}{\mathcal{V}}
\newcommand{\mL}{\mathcal{L}}
\newcommand{\vq}{{\vec q}}
\newcommand{\vqp}{{\vec q}^{\,\, \prime}}
\newcommand{\Gbr}{G_\mt{brane}}
\newcommand{\Gbk}{G_\mt{bulk}}
\newcommand{\Geff}{G_\mt{eff}}
\newcommand{\leff}{\ell_\mt{eff}}

\newcommand{\RNum}[1]{\uppercase\expandafter{\romannumeral #1\relax}}


\usepackage[thinlines]{easytable}
%\usepackage{tikzsymbols}
\usepackage{textcomp}
%\usepackage{parskip}

%\newcommand{\max}{\text{max}}

\setcounter{secnumdepth}{3}
\setcounter{tocdepth}{2}

\title{\boldmath Quantum Extremal Islands Made Easy, Part I:\\
Entanglement on the Brane}
%\\ Ryu-Takayanagi meets Randall-Sundrum}
%\title{\boldmath Life on the Planck Brane}


\author[a,b]{Hong Zhe Chen,}
\author[a]{Robert C. Myers,}
\author[a]{Dominik Neuenfeld,}
\author[c]{Ignacio A. Reyes}
\author[a,b]{and Joshua Sandor}

\affiliation[a]{Perimeter Institute for Theoretical Physics, Waterloo, ON N2L 2Y5, Canada}
\affiliation[b]{Dept.~of Physics $\&$ Astronomy, University of Waterloo, Waterloo, ON N2L 3G1, Canada}
\affiliation[c]{Max-Planck-Institut f\"ur Gravitationsphysik, Am M\"uhlenberg 1, 14476 Potsdam, Germany}

\emailAdd{hchen2@pitp.ca}
\emailAdd{rmyers@pitp.ca}
\emailAdd{dneuenfeld@pitp.ca}
\emailAdd{ignacio.reyes@aei.mpg.de}
\emailAdd{jsandor@perimeterinstitute.ca}




\abstract{Recent progress in our understanding of the black hole information paradox has lead to a new prescription for calculating entanglement entropies, which involves special subsystems in regions where gravity is dynamical, called \textit{quantum extremal islands}. We present a simple holographic framework where the emergence of quantum extremal islands can be understood in terms of the standard Ryu-Takayanagi prescription, used for calculating entanglement entropies in the boundary theory. Our setup describes a $d$-dimensional boundary CFT coupled to a ($d$--1)-dimensional defect, which are dual to global AdS${}_{d+1}$ containing a codimension-one brane. Through the Randall-Sundrum mechanism, graviton modes become localized at the brane, and in a certain parameter regime, an effective description of the brane is given by Einstein gravity on an AdS${}_d$ background coupled to two copies of the boundary CFT. Within this effective description, the standard RT formula implies the existence of quantum extremal islands in the gravitating region, whenever the RT surface crosses the brane. This indicates that islands are a universal feature of effective theories of gravity and need not be tied to the presence of black holes.}


\begin{document}
\maketitle

%\setcounter{section}{-1}
%\section{Proposed Outline}
%\vskip0.3cm


%\rcm{Use the notation of new note!! \eg $G_\mt{brane},\ G_\mt{bulk},\ G_\mt{eff},\ \ell_\mt{eff}$}\\

%\rcm{Use notation like `eq.' or `eqs.' in front of equation numbers. Use `figure' rather than `fig.' or `Figure'. Get bibtex references from INSPIRE.}

%%%%%%%%%%%%%%%%%%%%%%%%%%%%%%%%%%%%%%%%%%%%%%%%%%%%%%%%%%%%%%%%%%%%%%%%%%%

\newpage
\section{Introduction}\label{sec:introduction}

\section{Introduction \label{sec:introduction}}

When probed at very short wavelengths, QCD is essentially a theory of
free \index{Partons}`partons' --- quarks and gluons --- which only
scatter off one another through relatively small quantum corrections,
that can be systematically calculated. 
But at longer wavelengths, of order the size of the proton $\sim
1\mathrm{fm} = 10^{-15}\mathrm{m}$,  
we see strongly bound towers of hadron resonances emerge, with string-like
potentials building up if we try to separate their partonic
constituents. Due to our
inability to perform analytic calculations in 
strongly coupled field theories, QCD is therefore 
still only partially solved. Nonetheless,  all its features, across all
distance scales, are believed to be encoded in a single one-line
formula of alluring simplicity; the
\index{QCD!Lagrangian}%
Lagrangian\footnote{Throughout these notes we let it be implicit that
  ``Lagrangian'' really refers to Lagrangian density, ${\cal L}$, the
  four-dimensional space-time integral of which is the action.} of QCD.

The consequence for collider physics is that some parts of QCD can be
calculated in terms of the fundamental parameters of the Lagrangian,
whereas others must be expressed through models or functions whose effective 
parameters are not a priori calculable but which can be constrained
by fits to data. 
However, even in the absence of a
perturbative expansion, there are still several strong theorems which
hold, and which can be used to give relations between seemingly
different processes. (This is, e.g., the reason it makes sense to 
measure the partonic substructure of the proton in $ep$ collisions and
then re-use the same parametrisations for $pp$
collisions.) Thus, in the chapters 
dealing with phenomenological models we shall emphasise that the loss
of a factorised perturbative expansion is not equivalent to a total
loss of predictivity.   

An alternative approach would be to give up on calculating QCD 
and use leptons instead. Formally, this amounts to summing inclusively over
strong-interaction phenomena, when such are present. While such a
strategy might succeed in replacing what we do know about QCD by
``unity'', however, even the most adamant chromophobe would acknowledge
the following basic facts of collider physics for the next decade(s): 
1) At the LHC, the initial states are
hadrons, and hence, at
the very least, well-understood and precise parton distribution
functions (PDFs) will be required; 2) high precision will mandate
 calculations to higher orders in perturbation theory, 
which in turn will involve more QCD; 3) the requirement of lepton
\emph{isolation} makes the very definition of a lepton
 depend implicitly on QCD and 4) 
 the rate of jets that are misreconstructed as leptons in
 the experiment depends explicitly on it. 
And, 5) though many new-physics signals \emph{do} give observable
signals in the lepton sector, this is far from guaranteed, nor is it
exclusive when it occurs. 
 It would therefore be  unwise not to attempt to solve QCD to the best
 of our ability, the better to prepare ourselves for both the largest
 possible discovery reach and the highest attainable subsequent
 precision. 

Furthermore, QCD is the richest gauge theory we have so far
 encountered. Its emergent phenomena, unitarity properties, colour structure, 
 non-perturbative dynamics, quantum vs.\ classical limits, 
interplay between scale-invariant and
 scale-dependent properties, and its wide
 range of phenomenological applications, are still very much topics of
 active investigation, about which we continue to learn.  

In addition, or perhaps as a consequence, the field of QCD is
currently experiencing something of a revolution. On the perturbative
side, new methods to compute scattering amplitudes with very high
particle multiplicities are being developed, together with advanced
techniques for combining such amplitudes with all-orders resummation
frameworks. On the non-perturbative side, the wealth of data on
soft-physics processes from the LHC is
forcing us to reconsider the reliability of the standard fragmentation
models, and heavy-ion collisions are providing new insights into
the collective behavior of hadronic matter. The
study of cosmic rays impinging on the Earth's
atmosphere challenges our ability to extrapolate fragmentation models
from collider energy scales to the region of ultra-high energy cosmic
rays. And finally, dark-matter annihilation processes in space  may produce 
hadrons, whose spectra are sensitive to the modeling 
of fragmentation.

In the following, we shall focus on QCD for mainstream 
collider physics. This includes the basics of SU(3), colour factors, the running
of $\alpha_s$, factorisation, 
hard processes, infrared safety, parton showers and matching, event generators, hadronisation, and the so-called underlying event. 
While not covering everything, hopefully these topics can also serve
at least as stepping stones to more specialised
issues that have been left out, such as twistor-inspired techniques, 
heavy flavours, polarisation, or forward physics, or to topics more tangential to
other fields, such as axions, lattice QCD, or heavy-ion physics.  

\subsection{A First Hint of Colour}
Looking for new physics, as we do now at the LHC, it is instructive to 
consider the story of the discovery of colour. The first hint was
arguably the $\Delta^{++}$ \index{Baryons}baryon, discovered in 
1951~\cite{Brueckner:1952zz}. The title and part of the abstract from this
historical paper are reproduced in \figRef{fig:Delta}.
\begin{figure}[t]
\begin{center}
\begin{tabular}{c}
\colorbox{gray}{\includegraphics*[scale=0.75]{DeltaTitle.pdf}}\\[5mm]
\hspace*{2mm}\begin{minipage}{0.88\textwidth}
\small\sl  ``[...] It is concluded that the apparently anomalous features of the
scattering can be interpreted to be an indication of a resonant
meson-nucleon interaction corresponding to a nucleon isobar with spin
$\frac32$, isotopic spin $\frac32$, and with an excitation energy of
$277\,$MeV.''\\[1mm]
\end{minipage}
\end{tabular}
\caption{The title and part of the abstract of the 1951 paper
  \cite{Brueckner:1952zz} (published in 1952) in which the existence 
  of the $\Delta^{++}$ baryon was deduced, based on data from Sachs and
  Steinberger at Columbia~\cite{Chedester:1951sc}  and from Anderson,
  Fermi, Nagle, et al.~at Chicago~\cite{Fermi:1952zz}. Further studies 
  at Chicago were quickly performed
  in~\cite{Anderson:1952nw,Anderson:1952zza}. See also the memoir by
  Nagle~\cite{nagle1984delta}. 
\label{fig:Delta}}  
\end{center}
\end{figure}
In the context of the \index{Quarks}quark model --- which first
had to be developed, successively joining together the notions of 
spin, isospin, strangeness, and 
the \index{Eightfold way}eightfold way\footnote{In physics, the ``eightfold way''
refers to the classification of the lowest-lying pseudoscalar
\index{Mesons}mesons and 
\index{SU(3)!Of Flavour}%
spin-1/2 \index{Baryons}baryons within \index{Octet}octets in SU(3)-flavour space ($u,d,s$). The
$\Delta^{++}$ is part of a spin-3/2 baryon \index{Decuplet}decuplet, a ``tenfold way'' in this
terminology.} 
--- the \index{Flavour}flavour and spin content of the $\Delta^{++}$
baryon is: 
\begin{equation}
\left\vert \Delta^{++} \right> = \left\vert
\,u_\uparrow\ u_\uparrow\ u_\uparrow \right>~,
\end{equation} 
clearly a highly symmetric configuration. However, since 
the $\Delta^{++}$ is a fermion, it must have an overall
antisymmetric wave function. In 1965, fourteen years after its
discovery, this was finally understood by the introduction of colour
\index{SU(3)}%
\index{SU(3)!Of Colour}%
as a new quantum number associated with the group SU(3)
\cite{Greenberg:1964pe,Han:1965pf}. The $\Delta^{++}$ wave function can now be made
antisymmetric by arranging its three quarks antisymmetrically 
in this new degree of freedom, 
\begin{equation}
\left\vert \Delta^{++} \right> = \epsilon^{ijk} \left\vert
\,u_{i\uparrow}\ u_{j\uparrow}\ u_{k\uparrow}\right>~,
\end{equation} 
hence solving the mystery.

More direct experimental tests of the number of colours were provided first by
measurements of the decay width of $\pi^0\to \gamma\gamma$ decays, which 
is proportional to $N_C^2$, 
and later by the famous ``R'' ratio in
$e^+e^-$ collisions ($R=\sigma(e^+e^-\to q\bar{q})/\sigma(e^+e^-\to
\mu^+\mu^-)$), which is proportional to $N_C$, see
e.g.~\cite{Dissertori:2003pj}. 
Below, in \SecRef{sec:L} we shall see how to
calculate such colour factors. 

\subsection{The Lagrangian of QCD \label{sec:L}}
\index{QCD!Lagrangian}%
Quantum Chromodynamics is based on the gauge group
\index{SU(3)}$\mrm{SU(3)}$, the 
Special Unitary group in 3 (complex) dimensions, whose elements 
are the set of unitary $3\times 3$ matrices with determinant one. 
\index{Fundamental representation}%
\index{SU(3)!Fundamental representation}%
Since there are 9 linearly independent unitary complex
matrices\footnote{A complex $N\times N$ matrix has $2N^2$ degrees of
  freedom, on which unitarity provides $N^2$ constraints.}, one of
which has determinant $-1$, there are a total of 8
independent directions in this matrix space, corresponding to eight
different generators as compared
with the single one of QED. In the context of QCD, we normally
represent this group using the 
so-called \emph{fundamental}, or \emph{defining}, representation, in
which the generators of $\mrm{SU(3)}$ appear as a set of eight traceless and
hermitean matrices, to which we return below.  
We shall refer to indices enumerating
the rows and columns of these matrices  (from 1 to 3) as
\emph{fundamental} indices, and we use the letters $i$,
$j$, $k$, \ldots, to denote them.
\index{Adjoint representation}%
\index{SU(3)!Adjoint representation}%
We refer to indices enumerating the generators (from 1 to 8),
as \emph{adjoint} 
indices\footnote{The dimension of the \emph{adjoint}, or
  \emph{vector}, representation is equal to the number of generators,
  $N^2-1=8$ for $\mrm{SU(3)}$, while the  
\index{Fundamental representation}%
\index{SU(3)!Fundamental representation}%
dimension of the fundamental representation is
  the degree of the group, $N=3$ for $\mrm{SU(3)}$.}, and we use the first
letters of the alphabet ($a$, $b$, $c$, \ldots) to denote them. 
These matrices can operate both on each other (representing
combinations of successive gauge transformations) and on a set of
$3$-vectors, the latter of 
which represent \index{Quarks}quarks in colour 
space; the quarks are \emph{triplets} under $\mrm{SU(3)}$. The matrices can be
thought of as representing gluons in colour 
space (or, more precisely, the gauge transformations carried out by
gluons), hence there are
eight different gluons; the gluons are \emph{octets} under $\mrm{SU(3)}$. 

\index{QCD!Lagrangian}%
The Lagrangian density of QCD is 
\begin{equation}
{\cal L} = \bar{\psi}_q^i(i\gamma^\mu)(D_\mu)_{ij}\psi_q^j - m_q
\bar{\psi}_q^i\psi_{qi} - \frac14 F^a_{\mu\nu}F^{a\mu\nu}~,\label{eq:L}
\end{equation}
where $\psi_q^i$ denotes a quark field with
(fundamental) colour index $i$, 
$\psi_q = ({\textcolor{red}{\psi_{qR}}},{\color{green}\psi_{qG}}, 
{\color{blue}\psi_{qB}})^T$, 
$\gamma^\mu$ is a Dirac matrix that expresses the
vector nature of the strong interaction, with $\mu$ being a Lorentz
vector index, $m_q$ allows for the
possibility of non-zero \index{Quarks}quark masses (induced by the
standard Higgs 
mechanism or similar), $F^a_{\mu\nu}$ is the gluon field strength 
tensor for a gluon\footnote{The definition of the gluon field strength
  tensor will be given below in \eqRef{eq:F}.} with (adjoint) 
colour index $a$ (i.e., $a\in[1,\ldots,8]$), 
and $D_\mu$ is the covariant derivative in QCD,
\begin{equation}
(D_{\mu})_{ij} = \delta_{ij}\partial_\mu - i g_s t_{ij}^a A_\mu^a~,\label{eq:D}
\end{equation}
\index{QCD!Coupling}
with $g_s$ the \index{alphaS@$\alpha_s$}strong coupling (related to
$\alpha_s$ by $g_s^2 = 4\pi 
\alpha_s$; we return to the strong coupling in more detail below), 
$A^a_\mu$  the gluon field with 
colour index $a$, and $t_{ij}^a$ proportional to the hermitean and
traceless \index{Gell-Mann matrices|see{SU(3)}}Gell-Mann matrices of $\mrm{SU(3)}$, 
\index{SU(3)!Generators}%
\begin{equation}
\mbox{\includegraphics*[scale=1.0]{gell-mann}}~.
\end{equation}
These generators are just the $\mrm{SU(3)}$ analogs of the
Pauli matrices in 
$\mrm{SU(2)}$. 
By convention, the constant of proportionality is normally
taken to 
be 
\begin{equation}
t^a_{ij} = \frac12 \lambda^a_{ij}~. \label{eq:t}
\end{equation}
\index{QCD!Coupling}
This choice in turn determines the normalisation of the coupling
$g_s$, via \eqRef{eq:D}, and
fixes the values of the $\mrm{SU(3)}$ \index{Casimirs}Casimirs and structure constants, to which we return below. 

An example of the colour flow for a
quark-gluon interaction in colour 
space is given in \figRef{fig:qg}.
\begin{figure}[t]
\begin{center}
\begin{minipage}[h]{4.6cm}
\begin{center}
$A^1_\mu$\\
\includegraphics*[scale=0.75]{qgv.pdf}\\[-3mm]
$\psi_{q\textcolor{green}{G}}$\hfill$\psi_{q\textcolor{red}{R}}$
\end{center}
\end{minipage}~~~
\parbox{0.4\textwidth}{
$
\begin{array}{ccccc}
\propto & - \frac{i}{2} g_s & \bar{\psi}_{q\color{red}R}  & \lambda^{1} & \psi_{q\color{green}G} 
\\[2mm]
= & -\frac{i}{2}g_s & \left(\begin{array}{ccc} \textcolor{red}{1} & \color{green} 0 &
  \color{blue} 0 
\end{array}\right) & 
\left(\begin{array}{ccc}
0 & 1 & 0  \\
1 & 0 & 0 \\
0 & 0 & 0
\end{array}\right) & 
 \left(\begin{array}{c}
\textcolor{red}{0} \\
\color{green}1 \\
\color{blue}0
\end{array}\right) \end{array}
$}
\caption{Illustration of a 
\index{Quarks}\index{Gluons}$qqg$ vertex in QCD, before
  summing/averaging over colours: a gluon in a state represented by $\lambda^1$
  interacts with quarks in the states $\psi_{qR}$ and
  $\psi_{qG}$. \label{fig:qg}}
\end{center}
\end{figure}
Normally, of course, we sum over all the colour indices, so this
example merely gives a pictorial representation of what one particular
(non-zero) term in the colour sum looks like.


\subsection{Colour Factors}
\index{QCD!Colour factors}
\index{Colour factors}%
\index{Colour-space indices|see{Colour connections}}%
\index{Matrix elements}%
Typically, we do not measure colour in the final state ---
instead we average over all possible incoming colours and sum over all
possible outgoing ones, wherefore QCD scattering amplitudes (squared) in
practice always contain sums over quark fields contracted with
\index{SU(3)!Generators}Gell-Mann matrices. These contractions in turn
produce traces  
which yield the \index{Colour factors}\emph{colour factors} that are associated to each QCD
process, and which basically count the number of ``paths through
colour space'' that the process at hand can take\footnote{The
  convention choice represented by \eqRef{eq:t} introduces a
  ``spurious'' factor of 2 for each power of the coupling $\alpha_s$. 
Although one could in principle absorb that factor into a redefinition
of the coupling, effectively redefining the normalisation of ``unit
colour charge'', the standard definition of $\alpha_s$ is now so
entrenched that alternative choices would be counter-productive, at
least in the context of a pedagogical review.}.

A very simple example of a colour factor is given by the decay process $Z\to
q\bar{q}$. This vertex contains a simple $\delta_{ij}$ in colour
space; the outgoing quark and antiquark must have identical 
(anti-)col\-ours. Squaring the corresponding matrix element and summing over
final-state colours yields a colour factor of
\begin{equation}
e^+e^-\to Z \to q\bar{q}~~~:~~~\sum_{\mrm{colours}}|M|^2 \propto
\delta_{ij}\delta_{ji} = \mrm{Tr}\{\delta\} = N_C = 3~,
\end{equation}
since $i$ and $j$ are quark (i.e., 3-dimensional
fundamental) indices. This factor corresponds directly to the 3 different
``paths through colour space'' that the process at hand can take; the
produced quarks can be red, green, or blue. 

A next-to-simplest example is given by $q\bar{q}\to
\gamma^*/Z\to\ell^+\ell^-$ (usually referred to as the
\index{Drell-Yan}Drell-Yan 
process~\cite{Drell:1970wh}),  
which is just a crossing of the previous one. By crossing
symmetry, the squared matrix element, including the colour factor, is
exactly the same as before, but since the quarks are here incoming, we
must \emph{average} rather than sum over their colours, leading to
\begin{equation}
q\bar{q}\to Z\to e^+e^-~~~:~~~\frac{1}{9}\sum_{\mrm{colours}}|M|^2 \propto \frac19\delta_{ij}\delta_{ji} = \frac19 \mrm{Tr}\{\delta\} = \frac13~,
\end{equation}
where the colour factor now expresses a \emph{suppression} which can
be interpreted as due to the fact that only quarks of matching colours
are able to collide and produce a $Z$ boson. The chance that a quark
and an antiquark picked at random from the colliding hadrons have 
matching colours is $1/N_C$. 
\begin{figure}[t]
\end{figure}

Similarly, $\ell q \to
\ell q$ via $t$-channel photon exchange (usually called Deep
Inelastic Scattering --- \index{DIS}\index{Deep inelastic scattering|see{DIS}}DIS --- with ``deep'' referring to a 
large virtuality of the exchanged photon), constitutes yet another
crossing of the same basic process, 
see \figRef{fig:Zcrossings}. \index{Colour factors}The colour factor in this case 
comes out as unity. 
\begin{figure}[t]
\centering\vspace*{-8mm}
\begin{tabular}{ccc}
\rotatebox{360}{\includegraphics*[scale=0.93]{ee2qq}} \ \ 
& \ \ \includegraphics*[scale=0.93,angle=180,origin=c]{ee2qq}
\ \ & \ \ \includegraphics*[scale=0.9,angle=297,origin=c]{ee2qq}\\
Hadronic $Z$ decay & \index{Drell-Yan}Drell-Yan & \index{DIS}DIS \\[1mm]
$e^-e^+ \to \gamma^*/Z^0 \to q\bar{q}$ &
$q\bar{q} \to \gamma^*/Z^0 \to \ell^+\ell^-$ &
$\ell \bar{q} \stackrel{\gamma^*/Z^*}{\to} \ell \bar{q}$
\\[2mm] 
$\propto N_C$ & $\propto 1/N_C$ & $\propto 1$
\end{tabular}
\caption{Illustration of the three crossings of the interaction of a
  lepton current (black) with a \index{Quarks}quark current (red) 
  via an intermediate photon or
  $Z$ boson, with corresponding colour factors. \label{fig:Zcrossings}}
\end{figure}

To illustrate what happens when we insert (and sum over)
quark-gluon
vertices, such as the one depicted in \figRef{fig:qg}, we take
the process $Z\to3\,$jets. \index{Colour factors}The colour factor for
this process can be 
computed as follows, with the accompanying illustration showing a
corresponding diagram (squared) with explicit colour-space indices on
each vertex:\\
\index{Colour connections}
\begin{equation}
\mbox{
\begin{tabular}{cc}
\parbox{5.2cm}{
$Z \to qg\bar{q}$~~~:~~~\\
\[
\begin{array}{rcl}
\displaystyle\sum_{\mrm{colours}}|M|^2 & \propto & \displaystyle
\delta_{ij}t_{jk}^a t_{k\ell
    }^a\delta_{\ell i} \\
& = & \displaystyle
\mrm{Tr}\{t^at^a\}\\[4mm] & = & \displaystyle
  \frac12\mrm{Tr}\{\delta\} = 4~,
\end{array}
\]}
&
\parbox{8.5cm}{\includegraphics*[scale=0.6]{colFacZ3.pdf}
}
\end{tabular}}
\end{equation}
where the last $\mrm{Tr}\{\delta\} = 8$, since the trace runs over
the 8-dimensional adjoint indices. If we
want to ``count the paths through colour space'', we should leave out
the factor $\frac12$ which comes from the normalisation convention for
the $t$ matrices, \eqRef{eq:t}, hence this process can take 8
different paths through colour space, one for each gluon basis state.

The tedious task of taking traces over $t$
matrices can be greatly alleviated by use of the relations given in
\TabRef{tab:lambda}.  
\index{Traces in SU(3)|see{SU(3)}}%
\index{SU(3)!Trace relations}%
\index{QCD!Trace relations|see{SU(3)}}%
\begin{table}
\begin{center}
\scalebox{1.04}{\begin{tabular}{ccc}
\toprule
\index{SU(3)!Trace relations}Trace Relation & Indices & Occurs in Diagram Squared
\\
\midrule
$\mrm{Tr}\{t^at^b\} = T_R\, \delta^{ab}$ & $a,b\in[1,\ldots,8]$
& \parbox[c]{4cm}{\includegraphics*[scale=0.5]{traces1}}\\
$\sum_a t^a_{ij}t^a_{jk} = C_F\, \delta_{ik}$ &%
\parbox[c]{3cm}{\begin{center}
$a\in[1,\ldots,8]$\\
$i,j,k\in[1,\ldots,3]$\end{center}}
& \parbox[c]{4cm}{\includegraphics*[scale=0.5]{traces2}}\\
$\sum_{c,d} f^{acd} f^{bcd} = C_A\, \delta^{ab}$ & $a,b,c,d\in[1,\ldots,8]$
& \parbox[c]{4cm}{\includegraphics*[scale=0.5]{traces3}}\\
$ t^a_{ij}t^a_{k\ell} = T_R \left(\delta_{jk}\delta_{i\ell}
- \frac{1}{N_C}\delta_{ij}\delta_{k\ell}\right)$ & $i,j,k,\ell\in[1,\ldots,3]$
& \parbox[c]{4cm}{\includegraphics*[scale=0.5]{traces4}}\hspace*{-0.2cm}(Fierz)\\
\bottomrule
\end{tabular}}
\caption{Trace relations for $t$ matrices (convention-independent). 
 More relations
  can be found in \cite[Section 1.2]{Ellis:1991qj} and in 
  \cite[Appendix A.3]{Peskin:1995ev}.
\label{tab:lambda}}
\end{center}
\end{table}
In the standard normalisation convention for the \index{SU(3)}$\mrm{SU(3)}$ generators,
\eqRef{eq:t}, the \index{Casimirs}Casimirs of $\mrm{SU(3)}$ appearing in
\TabRef{tab:lambda} are\footnote{See, e.g., \cite[Appendix
    A.3]{Peskin:1995ev} for how to obtain the Casimirs in other
  normalisation conventions. As an example, choosing $t^a_{ij} = \lambda_{ij}^a/\sqrt{2}$ would yield $T_R=1$, $C_F=T_R(N_C^2-1)/N_C=8/3$, $C_A=3$.} 
\index{Casimirs}\index{TR@$T_R$}\index{CA@$C_A$}\index{CF@$C_F$}
\begin{equation}
T_R = \frac12 \hspace*{2cm} C_F = \frac43 \hspace*{2cm} C_A = N_C = 3~.
\end{equation}
In addition, the gluon self-coupling on the third line in
\TabRef{tab:lambda} involves factors of $f^{abc}$. These
\index{QCD!Structure constants|see{SU(3)}}%
are called the \index{SU(3)!Structure constants}\emph{structure constants} of QCD and they enter via 
the non-Abelian term in the \index{Gluons}gluon field strength tensor appearing in
\eqRef{eq:L}, 
\begin{equation}
F^a_{\mu\nu} = \underbrace{\partial_\mu A_\nu^a - \partial_\nu
  A^a_\mu}_{\mathrm{Abelian}} +
\underbrace{ g_s f^{abc} A_\mu^b A_\nu^c}_{\mathrm{non-Abelian}}~. \label{eq:F}
\end{equation}

\noindent\begin{minipage}[t]{0.46\textwidth}
The structure constants of $\mrm{SU(3)}$ are listed in the table to the
right. They define the \emph{adjoint}, or \emph{vector}, representation of $\mrm{SU(3)}$
and are related to the fundamental-representation generators via the
commutator relations
\begin{equation}
t^at^b - t^bt^a = [t^a,t^b] = i f^{abc} t_c~,
\end{equation} 
or equivalently,
\begin{equation}
if^{abc}~=~2\mrm{Tr}\{t^c[t^a,t^b]\}~.
\end{equation}
Thus, it is a matter of choice whether one prefers to express colour
space on a basis of fundamental-representation $t$ matrices, or via
the structure constants $f$, and one can go back and forth between the
two.
\end{minipage}%
\hfill%
\colorbox{darkgray}{%
\colorbox{lightgray}{%
\begin{minipage}[t]{0.46\textwidth}
\vspace*{3mm}\begin{center}
\textbf{Structure Constants of SU(3)}
\begin{equation}
f_{123} = 1
\end{equation}
\begin{equation}
f_{147} = f_{246} = f_{257} = f_{345} = \frac12
\end{equation}
\begin{equation}
f_{156} = f_{367} = -\frac12
\end{equation}
\begin{equation}
f_{458} = f_{678} = \frac{\sqrt{3}}{2}
\end{equation}
Antisymmetric in all indices\\[3mm]
All other $f_{abc}=0$\vspace*{3mm}\\
\end{center}
\end{minipage}%
}}\vskip1mm

\begin{figure}[t]
\begin{center}
\begin{minipage}[h]{4.6cm}
\begin{center}
$A_\nu^4(k_2)$\\
\includegraphics*[scale=0.75]{ggv.pdf}\\[-3mm]
$A^6_\rho(k_1)$\hfill$A_\mu^2(k_3)$
\end{center}
\end{minipage}~~~
\parbox{0.35\textwidth}{
$
\begin{array}{cccc}
\propto & - g_s \ f^{246} \!\! & \!\! [ (k_3 - k_2)^\rho g^{\mu\nu}  \\ 
& & +(k_2 - k_1)^\mu g^{\nu\rho} \\ 
& &+(k_1 - k_3)^\nu g^{\rho\mu}]
\end{array}
$}\vspace*{1mm}
\caption{Illustration of a \index{Gluons}$ggg$ vertex in QCD, before
  summing/averaging over colours: interaction between gluons in the 
  states $\lambda^2$, $\lambda^4$, and $\lambda^6$ is represented by
  the structure constant $f^{246}$. 
\label{fig:gg}}
\end{center}
\end{figure}
 Expanding the $F_{\mu\nu}F^{\mu\nu}$ term of the
Lagrangian using \eqRef{eq:F}, we see that there is a 3-gluon and a
4-gluon vertex that involve $f^{abc}$, the latter of which has two
powers of $f$ and two powers of the coupling. 

Finally, the last line of \TabRef{tab:lambda} is not really a trace
relation but instead a useful so-called Fierz transformation, which
expresses products of $t$ matrices in terms of Kronecker $\delta$ functions. 
It is often used, for instance, in shower Monte Carlo
applications, to assist in mapping between colour flows in $N_C = 3$,
in which cross sections and splitting probabilities are calculated, 
and those in $N_C\to\infty$ (``leading colour''), used to represent colour flow in
the MC ``event record''.

A \index{Gluons}gluon self-interaction vertex is
illustrated in \figRef{fig:gg}, to be compared with the quark-gluon
one in \figRef{fig:qg}. We remind the reader that gauge boson
self-interactions are a hallmark of non-Abelian theories and that their
presence leads to some of the main differences between QED and
QCD. One should also keep in mind 
that the \index{Colour factors}colour factor for the vertex in \figRef{fig:gg}, \index{CA@$C_A$}$C_A$, 
is roughly twice as large as that for a quark, \index{CF@$C_F$}$C_F$.

\subsection{The Strong Coupling \label{sec:coupling}}
\index{QCD!Coupling}
\index{Jets}
\index{alphaS@$\alpha_s$}To first approximation, QCD is 
\index{QCD!Scale invariance}\emph{scale invariant}. That is, if one
``zooms in'' on a QCD jet, one will find a repeated self-similar 
pattern of jets within jets within jets, reminiscent of
fractals. 
In the context of QCD, this property was originally 
called \index{Lightcone scaling|see{QCD Scale invariance}}light-cone scaling, or 
\index{Bjorken scaling|see{QCD Scale invariance}}Bj{\o}rken scaling. 
This type of scaling is closely related to the class of
angle-preserving symmetries, called \index{Conformal
invariance}\emph{conformal} symmetries. In physics 
today, the terms ``conformal'' and ``scale invariant'' are used 
interchangeably\footnote{Strictly speaking, conformal symmetry is more
restrictive than just scale invariance, but examples of
scale-invariant field theories that are not conformal are rare.}.
Conformal invariance is a mathematical property of several
QCD-``like'' theories which are now being studied (such as $N=4$
supersymmetric relatives of QCD). It is also 
related to the physics of so-called ``unparticles'', though that is a
relation that goes beyond the scope of these lectures.

Regardless of the labelling, 
if the  \index{alphaS@$\alpha_s$}strong coupling did not run (we shall
return to the running 
of the coupling below), Bj{\o}rken scaling would be absolutely true. QCD
would be a theory with a fixed coupling, the same at all scales. 
This simplified picture already captures some of the most important
properties of QCD, as we shall discuss presently.  

\index{QCD!Scale invariance}%
In the limit of exact Bj{\o}rken scaling --- QCD at fixed coupling
--- properties of high-energy interactions are determined 
only by \emph{dimensionless} kinematic quantities, such as scattering
angles (pseudorapidities) and ratios of energy
scales\footnote{Originally, the observed approximate agreement with
this was used as a powerful argument
for pointlike substructure in hadrons; since measurements at different
energies are sensitive to different resolution scales, independence of the absolute
energy scale is indicative of the absence of other fundamental
scales in the problem and hence of pointlike constituents.}.
For applications of QCD to high-energy collider physics, an important
consequence of Bj{\o}rken scaling is thus that the rate of 
\index{Parton showers}%
\index{Bremsstrahlung|see{Parton showers}}
bremsstrahlung
jets, with a given transverse momentum, scales in direct proportion to
the hardness 
of the fundamental partonic scattering process they are produced in
association with. This agrees well with our intuition about accelerated
charges; the harder you ``kick'' them, the harder the radiation they
produce.  

For instance, in the limit of exact scaling, a
measurement of the rate of 10-GeV jets produced in association with an
ordinary $Z$ 
boson could be used as a direct prediction of the rate of 100-GeV jets
that would be 
produced in association with a 900-GeV $Z'$ boson, and so 
forth. Our intuition about how many bremsstrahlung jets a given type of
process is likely to have should therefore be governed first and
foremost by the \emph{ratios} of scales that appear in that particular
process, as has been  highlighted in a number of studies focusing on
the mass and $p_\perp$ scales appearing, e.g., in
Beyond-the-Standard-Model (BSM) 
physics processes
\cite{Plehn:2005cq,Alwall:2008qv,Papaefstathiou:2009hp,Krohn:2011zp}. 
\index{QCD!Scale invariance}Bj{\o}rken scaling 
\index{Scale invariance|see{QCD}}
is also fundamental to the understanding of jet substructure in QCD, see, e.g.,
\cite{Vermilion:2011nm,Altheimer:2012mn}.  

\index{alphaS@$\alpha_s$!Running coupling}%
On top of the underlying scaling behavior, the running coupling will
introduce a dependence on the absolute scale, implying more radiation
at low scales than at high ones. The running is logarithmic with
\index{alphaS@$\alpha_s$!beta function}%
energy, and is governed by the so-called \emph{beta function}, 
\index{alphaS@$\alpha_s$}
\begin{equation}
Q^2 \frac{\partial \alpha_s}{\partial Q^2} = \frac{\partial
  \alpha_s}{\partial \ln Q^2} =
\beta(\alpha_s)~, \label{eq:running}
\end{equation}
where the function driving the energy dependence, the \index{Beta function}{beta
  function}, is defined as
\begin{equation}
\beta(\alpha_s) = -\alpha_s^2(b_0 +
b_1\alpha_s + b_2\alpha_s^2 + \ldots)~,\label{eq:beta}
\end{equation}
with LO (1-loop) and NLO (2-loop) coefficients
\begin{eqnarray}
b_0 & = & \frac{11C_A - 4 T_R n_f}{12\pi}~,\\[3mm]
b_1 & = & \frac{17C_A^2 - 10 T_R C_A n_f - 6 T_R C_F n_f}{24\pi^2} ~=~
\frac{153-19 n_f}{24\pi^2}~.\label{eq:b}
\end{eqnarray}
In the $b_0$ coefficient, the first term is due to
\index{Gluons!Contribution to beta function}gluon loops while the
second is due to \index{Quarks!Contribution to beta function}quark
ones. Similarly, the first 
term of the $b_1$ coefficient arises from double gluon loops,
while the second and third represent mixed quark-gluon ones. 
At higher loop orders, the $b_i$ coefficients depend explicitly on the
renormalisation scheme that is used. A brief discussion can be found in the
PDG review on QCD~\cite{pdg2012}, with more elaborate ones
contained in \cite{Dissertori:2003pj,Ellis:1991qj}. 
Note that, if there are additional coloured particles beyond the
Standard-Model ones, loops involving those particles enter
 at energy scales above the masses of the
new particles, thus modifying the  \index{alphaS@$\alpha_s$}running of the coupling at high scales. 
This is discussed, e.g., for supersymmetric models in
\cite{Martin:1997ns}. For the running of other SM couplings, see
e.g.,~\cite{Langacker:2010zza}. 

\index{alphaS@$\alpha_s$!Running coupling}%
Numerically, the value of the  \index{alphaS@$\alpha_s$}strong coupling is usually specified by
giving its value at the specific 
reference scale $Q^2=M^2_Z$, from which we can obtain its
value at any other scale by solving \eqRef{eq:running}, 
\begin{equation}
\alpha_s(Q^2) = \alpha_s(M_Z^2) \frac{1}{1+b_0
  \alpha_s(M_Z^2)\ln\frac{Q^2}{M_Z^2} + {\cal O}(\alpha_s^2)}~,
\label{eq:alphaq2}
\end{equation}
with relations including the ${\cal O}(\alpha_s^2)$ terms 
available, e.g., in \cite{Ellis:1991qj}. 
Relations between scales 
not involving $M_Z^2$ can obviously be obtained by just replacing $M_Z^2$
by some other scale $Q'^2$ everywhere in \eqRef{eq:alphaq2}. A
comparison of running at one- and two-loop order, in both cases starting from
$\alpha_s(M_Z)=0.12$, is given in \figRef{fig:asRun}.
\begin{figure}[t]
\centering
\includegraphics*[scale=0.45]{vc-alphaS.pdf}
\caption{Illustration of the running of
 $\alpha_s$ at 1- (open 
  circles) and 2-loop
  order (filled circles), 
starting from the same value of $\alpha_s(M_Z)=0.12$. 
\label{fig:asRun}}
\end{figure}
As is evident from the figure, the 2-loop running is somewhat faster
than the 1-loop one.

\index{alphaS@$\alpha_s$!Running coupling}%
As an application, let us prove that the 
logarithmic running of the coupling implies that an intrinsically 
multi-scale problem can be converted to a single-scale one, up to
corrections suppressed by two powers of $\alpha_s$, 
by taking the geometric mean of the scales involved. This follows from
expanding an arbitrary product of individual  \index{alphaS@$\alpha_s$}$\alpha_s$ factors around an
arbitrary scale $\mu$, using \eqRef{eq:alphaq2}, 
\begin{eqnarray}
\alpha_s(\mu_1)\alpha_s(\mu_2)\cdots\alpha_s(\mu_n) & = &
\prod_{i=1}^{n} \alpha_s(\mu) \left(1 +
b_0\,\alpha_s\ln\left(\frac{\mu^2}{\mu_i^2}\right) + {\cal O}(\alpha_s^2)\right)
\nonumber\\[2mm]
& = & \alpha_s^n(\mu) \left(1 + b_0\, \alpha_s \ln \left(
 \frac{\mu^{2n}}{\mu_1^2\mu_2^2\cdots\mu_n^2}\right) +  {\cal
   O}(\alpha_s^2) \right)~,
\end{eqnarray}
whereby the specific single-scale choice $\mu^n =
\mu_1\mu_2\cdots\mu_n$ (the geometric mean) can
be seen to push the difference between the two sides of the equation one order higher
than would be the case for any other combination of scales\footnote{In
  a fixed-order calculation, the individual scales $\mu_i$,
would correspond, e.g., to the $n$ hardest scales appearing in an infrared
safe sequential clustering algorithm applied to the given momentum
configuration.}. 

The appearance of the number of \index{Flavour}flavours, $n_f$, in $b_0$ implies that the
slope of the running depends on the number of contributing
\index{Flavour}flavours. Since full QCD is best approximated by $n_f=3$
below the charm threshold, by $n_f=4$ and $5$ from there to the $b$
and $t$ thresholds, respectively, and then by $n_f=6$ at scales
higher than $m_t$, it is therefore important to be aware that 
the running changes slope across quark \index{Flavour}flavour
thresholds. Likewise, it would change across the threshold for any coloured
new-physics particles that might exist, with a magnitude depending on
the particles' colour and spin quantum numbers.

\index{alphaS@$\alpha_s$!Running coupling}%
\index{alphaS@$\alpha_s$}
The negative overall sign of \eqRef{eq:beta}, combined with the fact
that $b_0 > 0$ (for $n_f \le 16$), leads to the famous
result\footnote{
Perhaps the highest pinnacle of fame for \eqRef{eq:beta} was reached
when the sign of it featured in an episode of the TV series ``Big Bang
Theory''.} 
that the QCD coupling effectively \emph{decreases} with
 energy, called \index{Asymptotic freedom}asymptotic 
freedom, for the discovery of which the \index{Nobel prize}Nobel prize in physics was
awarded to D.~Gross, H.~Politzer, and F.~Wilczek in 2004. An extract
of the prize announcement runs as follows:
\begin{center}
\begin{minipage}{0.84\textwidth}
\sl  What this year's Laureates discovered was something that, at
first sight, seemed completely contradictory. The interpretation of
their mathematical result was that the closer the quarks are to each
other, the \emph{weaker} is the ``colour charge''. When the quarks are
really close to each other, the force is so weak that they behave
almost as free particles\footnote{More correctly, it is the coupling
  rather than the  
  force which becomes weak as the distance decreases. 
  The $1/r^2$ Coulomb singularity of the force is only dampened, not removed, 
  by the diminishing coupling.}. 
This phenomenon is called ``asymptotic
freedom''. The converse is true when the quarks move apart: the force
becomes stronger when the distance increases\footnote{More correctly,
 it is the potential which grows, linearly, while the force becomes
 constant.}. 
\end{minipage}
\end{center}

\index{Running coupling|see{alphaS@$\alpha_s$}}%
\index{alphaS@$\alpha_s$!Running coupling}%
Among the consequences of \index{Asymptotic freedom}asymptotic freedom is that perturbation
theory becomes better behaved at higher absolute energies, due to the
effectively decreasing coupling. Perturbative calculations for our
900-GeV $Z'$ boson from before should therefore be slightly faster
converging than equivalent calculations for the 90-GeV one. 
Furthermore, since the running of  \index{alphaS@$\alpha_s$}$\alpha_s$ explicitly
breaks Bj{\o}rken scaling, we also expect to see small changes in jet
shapes and in jet production ratios as we vary the energy. For
instance, since high-$p_\perp$ jets
start out with a smaller effective coupling, their intrinsic shape
(irrespective of boost effects) is
somewhat narrower than for low-$p_\perp$ jets, an issue which can be
important for jet calibration. Our current understanding of the
running of the QCD coupling is summarised by the plot in
\figRef{fig:alphas}, taken from a recent comprehensive review by S.\ Bethke
\cite{pdg2012,Bethke:2012jm}. A complementary up-to-date overview of
$\alpha_s$ determinations can be found in~\cite{d'Enterria:2015toz}. 

\index{alphaS@$\alpha_s$!Running coupling}%
As a final remark on \index{Asymptotic freedom}asymptotic freedom, note
that the decreasing 
value of the  \index{alphaS@$\alpha_s$}strong coupling with energy must eventually cause it to
become comparable to the electromagnetic and weak ones, at some energy
scale. Beyond that point, which may lie at energies of order
$10^{15}-10^{17}\,$GeV (though it may be lower if as yet undiscovered
particles generate large corrections to the running), 
we do not know  what the further evolution of the combined theory will 
actually look like, or whether it will continue to exhibit
\index{Asymptotic freedom}asymptotic
freedom. 

\index{alphaS@$\alpha_s$}%
\index{alphaS@$\alpha_s$!Running coupling}%
\index{alphaS@$\alpha_s$!LambdaQCD@$\Lambda_{\mathrm{QCD}}$}%
Now consider what happens when we run the coupling in the other
direction, towards smaller energies. 
\begin{figure}[t]
\begin{center}\hspace*{-0.25cm}
\parbox[c]{3.1cm}{\includegraphics*[scale=0.65]{arr-ir.pdf}}
\parbox[c]{8cm}{\includegraphics*[scale=0.5]{asq-2011.pdf}}\hspace*{-1mm}
\parbox[c]{3.1cm}{\includegraphics*[scale=0.65]{arr-uv.pdf}}
\caption{Illustration of the running of $\alpha_s$ in a theoretical
  calculation (band) and in physical processes at
  different characteristic scales, from
  \cite{pdg2012,Bethke:2012jm}. The little kinks at $Q=m_{c}$ and
  $Q=m_b$ are
  caused by discontinuities in the running across the flavour
  thresholds.\label{fig:alphas}}  
\end{center}           
\end{figure}
Taken at face value, the numerical value of the coupling diverges
rapidly at scales below 1 GeV, as illustrated by the curves
disappearing off the left-hand edge of the plot in
\figRef{fig:alphas}. To make this divergence
explicit, one can rewrite
\eqRef{eq:alphaq2} in the following form, 
 \index{alphaS@$\alpha_s$}
\begin{equation}
\alpha_s(Q^2) = \frac{1}{b_0 \ln \frac{Q^2}{\Lambda^2}}~,\label{eq:alphasLam}
\end{equation}
where 
\begin{equation}
\Lambda \sim 200\, \mbox{MeV}
\end{equation}
\index{alphaS@$\alpha_s$!LambdaQCD@$\Lambda_{\mathrm{QCD}}$}%
\index{alphaS@$\alpha_s$!Landau Pole|see{$\Lambda_{\mathrm{QCD}}$}}%
\index{LambdaQCD@$\Lambda_{\mathrm{QCD}}$|see{alphaS@$\alpha_s$}}%
specifies the energy scale at which the perturbative coupling would nominally become
infinite, called the Landau pole. (Note, however, that this only
parametrises the purely \emph{perturbative} result, which is not
reliable at \index{Strong coupling}strong coupling, so \eqRef{eq:alphasLam} should 
not be taken to imply that the physical behavior of full QCD should
exhibit a divergence for $Q\to \Lambda$.) 

\index{alphaS@$\alpha_s$}%
\index{alphaS@$\alpha_s$!Running coupling}%
\index{alphaS@$\alpha_s$!LambdaQCD@$\Lambda_{\mathrm{QCD}}$}%
Finally, one should be aware that there is a multitude of different
ways of defining both $\Lambda$ and $\alpha_s(M_Z)$. At the very
least, the numerical value one obtains depends both on the
renormalisation scheme used (with the dimensional-regularisation-based
``modified minimal subtraction'' scheme, $\overline{\mbox{MS}}$, being the
most common one) and on the perturbative order of the calculations 
used to extract them. As a rule of thumb, fits to experimental data typically yield 
smaller values for $\alpha_s(M_Z)$ the higher the order of the
calculation used to extract it (see, e.g.,
\cite{Bethke:2009jm,Dissertori:2009ik,Bethke:2012jm,pdg2012}), with  $
\alpha_s(M_Z)\vert_{\mrm{LO}} \gsim \alpha_s(M_Z)\vert_{\mrm{NLO}}
\gsim \alpha_s(M_Z)\vert_{\mrm{NNLO}}$. 
Further, since the number of \index{Flavour}flavours changes the slope
of the running, the location of the Landau pole for fixed
$\alpha_s(M_Z)$ depends explicitly on the number of \index{Flavour}flavours used in
the running. Thus each value of $n_f$ is associated with its own
value of $\Lambda$, with the following matching relations across
thresholds guaranteeing continuity of the coupling at one loop,
\index{LambdaQCD@$\Lambda_{\mathrm{QCD}}$|see{$\alpha_s$}}
\index{alphaS@$\alpha_s$!LambdaQCD@$\Lambda_{\mathrm{QCD}}$}%
\begin{eqnarray}
n_f = 5 \leftrightarrow 6 ~~~:~~~~~~\Lambda_6 = \Lambda_5
  \left(\frac{\Lambda_5}{m_t}\right)^{\frac{2}{21}} & & 
\Lambda_5 = \Lambda_6
  \left(\frac{m_t}{\Lambda_6}\right)^{\frac{2}{23}} ~, \\[2mm]
n_f = 4 \leftrightarrow 5 ~~~:~~~~~~\Lambda_5 = \Lambda_4
  \left(\frac{\Lambda_4}{m_b}\right)^{\frac{2}{23}} & & 
\Lambda_4 = \Lambda_5
  \left(\frac{m_b}{\Lambda_5}\right)^{\frac{2}{25}} ~, \\[2mm]
n_f = 3 \leftrightarrow 4 ~~~:~~~~~~\Lambda_4 = \Lambda_3 
  \left(\frac{\Lambda_3}{m_c}\right)^{\frac{2}{25}} & &
\Lambda_3 = \Lambda_4 
  \left(\frac{m_c}{\Lambda_4}\right)^{\frac{2}{27}} ~.
\end{eqnarray}

\index{alphaS@$\alpha_s$}%
\index{alphaS@$\alpha_s$!Running coupling}%
It is sometimes stated that QCD only has a single free
parameter, the  \index{alphaS@$\alpha_s$}strong coupling. 
However, even in the perturbative
region, the beta function depends explicitly on the number of
quark \index{Flavour}flavours, as we have seen, and thereby also on the quark
masses. Furthermore, in the non-perturbative region around or below
$\Lambda_{\mrm{QCD}}$, the value of the 
perturbative coupling, as obtained, e.g., from \eqRef{eq:alphasLam},
gives little or no insight into the behavior of the full theory. 
Instead, universal functions (such as parton densities, form factors,
fragmentation functions, etc), effective theories (such as the
Operator Product Expansion, Chiral Perturbation Theory, or Heavy Quark
Effective Theory), or phenomenological models (such as Regge Theory or
the String and Cluster Hadronisation Models) must be used, which in
turn depend on additional non-perturbative parameters whose relation to, e.g.,
$\alpha_s(M_Z)$, is not a priori known. 

\index{Lattice QCD}
For some of these questions,
such as hadron masses, lattice QCD can furnish important
additional insight, but for multi-scale and/or time-evolution
problems, the applicability of lattice methods is still severely
restricted; the lattice formulation of QCD requires 
  a Wick rotation to
  Euclidean space. The time-coordinate can then be treated on an
  equal footing with the other dimensions, but intrinsically
  Minkowskian problems, such as the time evolution of a system, are
   inaccessible. The limited size of current lattices
  also severely constrain the scale hierarchies that it is possible to
  ``fit'' between the lattice spacing and the lattice size. 

\index{Landau pole|see{$\alpha_s$}}%
\index{QCD!Landau Pole|see{$\alpha_s$}}%
\index{Renormalisation|see{$\alpha_s$}}%
\index{QCD!Renormalisation|see{$\alpha_s$}}%

\subsection{Colour States}
\index{Coherence}%
A final example of the application of the underlying $\mrm{SU(3)}$ group
theory to QCD is given by considering which colour states we can
obtain by combinations of quarks and gluons. The simplest example of
this is the combination of a quark and antiquark. We can form a total
of nine different colour-anticolour combinations, which fall into two
irreducible representations of $\mrm{SU(3)}$:
\begin{equation}
3 \otimes \overline{3} = 8 \oplus 1~.\label{eq:33bar}
\end{equation}
The singlet corresponds to the symmetric wave function 
$\frac{1}{\sqrt{3}}\left(\left|R\bar{R}\right>+\left|G\bar{G}\right>+\left|B\bar{B}\right>\right)$, 
which is invariant under $\mrm{SU(3)}$ transformations (the definition of a
singlet). The other eight linearly independent 
combinations (which can be represented by one for each Gell-Mann
matrix, with the singlet corresponding to the identity matrix) transform
into each other under $\mrm{SU(3)}$. Thus, although we sometimes talk about
colour-singlet states as 
being made up, e.g., of ``red-antired'', that is not quite precise
language. The actual state $\left|R\bar{R}\right>$ is \emph{not} a
pure colour singlet.  Although it does
have a non-zero \emph{projection} onto the singlet wave function
above, it also has non-zero projections onto the two members of
the octet that correspond to the diagonal Gell-Mann
matrices. Intuitively, one can also easily realise this by noting that
an $\mrm{SU(3)}$ rotation of $\left|R\bar{R}\right>$ would in general turn it into a
different state, say $\left|B\bar{B}\right>$, whereas a true colour singlet
would be invariant. 
Finally, we can also realise from \eqRef{eq:33bar} that a random
(colour-uncorrelated) quark-antiquark pair has a $1/N^2=1/9$ 
chance to be in an overall colour-singlet state; otherwise it is in
an octet. 

Similarly, there are also nine possible quark-quark (or
antiquark-antiquark) combinations, six of which are symmetric
under interchange of the two quarks and three of which are antisymmetric:
\index{Sextet}%
\begin{equation}
6 ~=~ \left(\begin{array}{c}
\left|RR\right>\\
\left|GG\right>\\
\left|BB\right>\\
\frac{1}{\sqrt{2}}\left(\left|RG\right> + \left|GR\right>\right)\\
\frac{1}{\sqrt{2}}\left(\left|GB\right> + \left|BG\right>\right)\\
\frac{1}{\sqrt{2}}\left(\left|BR\right> + \left|RB\right>\right)
\end{array}\right)
~~~~~~~~~
\bar{3} = \left(\begin{array}{c}
\frac{1}{\sqrt{2}}\left(\left|RG\right> - \left|GR\right>\right)\\
\frac{1}{\sqrt{2}}\left(\left|GB\right> - \left|BG\right>\right)\\
\frac{1}{\sqrt{2}}\left(\left|BR\right> - \left|RB\right>\right)
\end{array}\right)~.
\end{equation}
The members of the sextet transform into (linear combinations of) 
each other under $\mrm{SU(3)}$ transformations, and similarly for the
members of the antitriplet, hence neither of these can be reduced
further. The breakdown into
irreducible $\mrm{SU(3)}$ multiplets is therefore
\begin{equation}
3 \otimes 3 = 6 \oplus \overline{3}~.
\end{equation}
Thus, an uncorrelated pair of quarks has a $1/3$ chance to add to an overall
anti-triplet state (corresponding to coherent
superpositions like ``red + green = antiblue''\footnote{In the context of
  hadronisation models, 
  this coherent superposition of two quarks in an overall antitriplet
  state is sometimes called a
  \index{Diquarks}``diquark'' (at low $m_{qq}$)
  \index{String junctions}or a ``string junction'' (at high $m_{qq}$), see
  \secRef{sec:stringModel}; it corresponds to the antisymmatric ``red
  + green = antiblue'' combination needed to create a baryon
  wavefunction. }); otherwise it is in an overall 
sextet state. 

Note that the emphasis on
the quark-(anti)quark pair being \emph{uncorrelated} is important;
production processes that correlate the produced partons, like $Z\to q\bar{q}$ or $g\to q\bar{q}$, will
project out specific components (here the singlet and octet,
respectively). 
Note also that, if the quark
and (anti)quark are on opposite sides of the universe (i.e., living in
two different hadrons), the QCD \emph{dynamics} will not care what
overall colour state they 
are in, so for the formation of multi-partonic states in QCD, obviously the
spatial part of the wave functions (causality at the very least) 
will also play a role. Here, we are considering \emph{only} the colour part
of the wave functions. 
Some additional examples are 
\begin{eqnarray}
8\otimes 8 & = & 27 \oplus 10 \oplus \overline{10} \oplus 8 \oplus 8
\oplus 1 ~,\\ 
3 \otimes 8 & = & 15 \oplus 6 \oplus 3~,\\
3 \otimes 6 & = & 10 \oplus 8~,\\
3\otimes3\otimes3 & = & (6 \oplus \overline{3}) \otimes 3 = 10 \oplus 8
\oplus 8 \oplus 1 ~.
\end{eqnarray}
Physically, the 27 in the first line corresponds to a completely
incoherent addition of the colour charges of two gluons;
\index{Decuplet}the decuplets are slightly more coherent (with a lower
total colour charge), the octets
yet more, and the singlet corresponds to the combination of two gluons
that have precisely equal and opposite colour charges, so that their
total charge is zero. 
Further extensions and generalisations of these combination rules can
\index{Young tableaux}be obtained, e.g., using the method of Young
tableaux~\cite{young1901,youngSagan}.  



\section{Brane Gravity}\label{sec:branegravity}
% !TEX root = ../lifeonbrane3.tex
%

As described in the introduction, we are studying a holographic system where the boundary theory is a $d$-dimensional CFT which lives on a spherical cylinder $R\times S^{d-1}$ (where the $R$ is the time direction). Further, this CFT is coupled to a (codimension-one) conformal defect positioned on the equator of the sphere. Hence, the defect spans the geometry $R\times S^{d-2}$ and supports a ($d-1$)-dimensional CFT. The bulk description of this system involves an asymptotically AdS$_{d+1}$ spacetime with a codimension-one brane spread through the middle of the space (and extending to the position of the defect at asymptotic infinity). In this setup, the brane has an AdS$_d$ geometry and further, we consider the case in which the brane has a substantial tension and backreacts on the bulk geometry. If the brane tension is appropriately tuned, the backreaction produces Randall-Sundrum gravity  supported on the brane \cite{Randall:1999vf,Randall:1999ee}, \ie in the backreacted geometry, new (normalizable) modes of the bulk graviton are localized near the brane inducing an effective theory of dynamical gravity on the brane. In the following, we review the bulk geometry produced by the backreaction of the brane, and also the gravitational action induced on the brane.

\subsection{Brane Geometry}\label{BranGeo}

In the bulk, we have Einstein gravity with a negative cosmological constant in $d+1$ dimensions, \ie
\beq
I_\mt{bulk} = \frac{1}{16 \pi G_\mt{bulk}}\int d^{d+1}x\sqrt{-g}
\[{R}(g) + \frac{d(d-1)}{L^2} \] \,,
\label{act2}
\eeq
where $g_{ab}$ denotes the bulk metric, and we are ignoring the corresponding surface terms here \cite{PhysRevLett.28.1082,Gibbons:1976ue,Emparan:1999pm}.
We also introduce a codimension-one (\ie $d$-dimensional) brane in the bulk gravity theory. The brane action is simply given by
\beq\label{braneaction}
I_\mt{brane} = -T_o\int d^dx\sqrt{-\tilde{g}}\,.
\eeq
where $T_o$ is the brane tension and $\tilde g_{ij}$ denotes the induced metric on the brane.

Away from the brane, the spacetime geometry locally takes the form of AdS$_{d+1}$ with the curvature scale set by $L$. As described above, the induced geometry on the brane will be an AdS$_d$ space, and so it is useful to consider the following metric where the AdS$_{d+1}$ geometry is foliated by AdS$_d$ slices
\beq\label{metric}
ds^2 %= g_{ab}\,dx^{a}dx^{b}
= d\rho^2 + \cosh^2\left({\rho}/{L}\right)\, g_{ij}^{\mt{AdS}_d}\,dx^{i}dx^{j}\,.
\eeq
Implicitly here, $L$ also sets the curvature of the AdS$_d$ metric, \eg in global coordinates,
\beq\label{metric2}
g_{ij}^{\mt{AdS}_d}\,dx^{i}dx^{j}=L^2\left[-\cosh^2\!\tdr\,dt^2+d\tdr^2+\sinh^2\!\tdr\,d\Omega_{d-2}^2
\right]\,.
%-\left(1 + \frac{r^2}{L^2}\right)dt^2+\frac{dr^2}{1 + \frac{r^2}{L^2}}+r^2d\Omega_{d-2}^2\,.
\eeq
With the above choices, we approach the asymptotic boundary with $\rho\to\pm\infty$, or with fixed $\rho$ and $\tdr\to\infty$. In the latter case, we arrive at the equator of the boundary $S^{d-1}$, where the conformal defect is located. For the following, it will be convenient to replace $\rho$ with a Fefferman-Graham-like coordinate \cite{FG,Fefferman:2007rka},
\beq\label{zrho}
z = 2 L e^{-\rho/L}\,,
\eeq
with which the metric \reef{metric} becomes
\beq\label{metric3}
ds^2=\frac{L^2}{z^2}\left[dz^2 +  \left(1 + \frac{z^2}{4\,L^2}\right)^2 g_{ij}^{\AdS_d}\,dx^{i}dx^{j} \right]\,.
%ds^2=\frac{L^2}{z^2}\left[dz^2 +  \left(1 + \frac{1}{2}\frac{z^2}{L^2} + \frac{1}{16}\frac{z^4}{L^4}\right)g_{ij}^{\mt{AdS}_d}\,dx^{i}dx^{j} \right]\,.
\eeq
In these coordinates we approach the asymptotic boundary with $z\to0$ and with $z\to\infty$. Below, we will focus on the region near $z\sim 0$.

\begin{figure}[h]
	\def\svgwidth{1\linewidth}
	\centering{
		\input{Gluing2.pdf_tex}
		\caption{Panel (a): Our Randall-Sundrum construction involves foliating  with AdS$_d$ slices. Then identical portions of two such AdS$_{d+1}$ geometries are glued together along an common AdS$_d$ slice. Panel (b): The jump in the extrinsic curvature across the interface between the two geometries is supported by a(n infinitely) thin brane. The brane is represented by a green line in the figures and the bulk AdS$_{d+1}$ spacetime is blue with a $d$-dimensional CFT at the asymptotic boundary.} \label{fig:brane2}
	}
\end{figure}

As described above, the brane spans an AdS$_d$ geometry in the middle of the backreacted spacetime. Following the usual Randall-Sundrum approach, we construct the desired solution
by cutting off the AdS$_{d+1}$ geometry at some $z=\s$, and then
complete the space by gluing this geometry to another copy of itself
-- see figure \ref{fig:brane2}.
Then the Israel junction conditions (\eg see \cite{israel1966singular,Misner:1974qy}) fix $\s$ by relating the discontinuity of the extrinsic curvature across this surface to the stress tensor introduced by the brane, \ie
\beq\label{Israel1}
 \Delta{K}_{ij}-\tilde{g}_{ij}\,\Delta{K}_{k}{}^{k} = 8 \pi \Gbk\, S_{ij} = - 8 \pi \Gbk T_o\,\tilde{g}_{ij}\,,
\eeq
where $\Delta{K}_{ij}={K}^+_{ij} - {K}^-_{ij}= 2{K}_{ij}$, given the symmetry of our construction.
The extrinsic curvature is calculated as \cite{Misner:1974qy}
\beq \label{extrinsic}
{K}_{ij} =\frac{1}{2}\frac{\partial g_{ij}}{\partial n}\bigg|_{z=\s} =-\frac{z}{2L} \frac{\partial g_{ij}}{\partial z} \bigg|_{z=\s} = \frac{1}{L}\frac{4 L^2 -\s^2}{4L^2 +\s^2}\,\tilde{g}_{ij}\,,
\eeq
where $\partial_n = -\frac{z}{L}\partial_z$ is an outward directed unit normal vector. Further, we are using the notation introduced above where $\tilde g_{ij}$ corresponds to the induced metric on the surface $z=\s$, \ie on the brane. Combining eqs.~\reef{Israel1} and \reef{extrinsic}, we arrive at
\beq\label{positionbrane}
\frac{4L^2 -\s^2}{4L^2 +\s^2} = \frac{4\pi \Gbk L\, T_o}{d-1}\,.
\eeq

Now if we consider $\s\ll L$, it will ensure that the defect is well approximated by the holographic gravity theory on the brane -- see the discussion in the next subsection. In this regime, we can solve eq.~\reef{positionbrane} in a small $\s$ expansion, and to leading order, we find that
\beq\label{position}
\s^2\simeq z_\mt{0}^2 = 2L^2\left(1-\frac{4 \pi \Gbk L T_o}{d-1}\right)\,.
\eeq
Hence to achieve this result, we must tune the expression in brackets on the right to be small, \ie
\beq\label{tune}
\veps\equiv 1-\frac{4 \pi \Gbk L T_o}{d-1}\ll 1\,.
\eeq
As the notation suggests, we can think of this quantity $\veps$ as an expansion parameter in solving for the brane position from eq.~\reef{positionbrane}.
A useful check of our calculations below will come from carrying the solution to the next order, \ie
$\s^2= z_\mt{0}^2+\delta[ \s^2]_\mt{2}+\cdots$ with
\beq\label{secondorder}
\delta[ \s^2]_\mt{2} =\frac{(d-1)L}{4 \pi \Gbk T_o}\,\veps^2
\ = \frac{(d-1)L}{4 \pi \Gbk T_o}\left(1-\frac{4 \pi \Gbk L T_o}{d-1}\right)^2\,.
\eeq

To conclude, we consider the intrinsic geometry of the brane. As we noted above, the curvature scale of $g_{ij}^{\mt{AdS}_d}$ is simply $L$, and hence given the full bulk metric \reef{metric3}, we can read off the curvature scale of the surface $z=\s$ as
\beq\label{curve1}
\lb=\frac{L^2}{\s}\left(1 + \frac{\s^2}{4\,L^2}\right)\,.
\eeq
Note that since we are considering $\s/L\ll 1$, it follows that
$\lb/L\gg 1$, \ie the brane is weakly curved. Using eq.~\reef{position}, we can solve for $\lb$ to leading order in the $\veps$ expansion to find
\beq\label{curve2}
\frac{L^2}{\lb^2}\simeq
2\,\veps\ =2\left(1-\frac{4 \pi \Gbk L T_o}{d-1}\right)\,.
\eeq
It will be useful to have the following expressions for the Ricci tensor and scalar evaluated for the brane geometry, and these are compactly written using eq.~\reef{curve1} as
\beq\label{Ricky2}
\tilde{R}_{ij}(\tilde g)=-\frac{d-1}{\lb^2}\, \tilde{g}_{ij}\,,\qquad \tilde{R}(\tilde g)=-\frac{d(d-1)}{\lb^2}\,.
\eeq




\subsection{Gravitational Action on the Brane}\label{indyaction}


As noted above, following the usual Randall-Sundrum scenario \cite{Randall:1999vf,Randall:1999ee,Karch:2000ct}, new (normalizable) modes of the bulk graviton are localized near the brane in the backreacted geometry, and this induces an effective theory of dynamical gravity on the brane. The gravitational action can be determined as follows:
First, one considers  a Fefferman-Graham (FG) expansion near the boundary of an asymptotic AdS geometry \cite{FG,Fefferman:2007rka}. Then integrating the bulk action (including the Gibbons-Hawking-York surface term \cite{PhysRevLett.28.1082,Gibbons:1976ue}) over the radial direction out to some regulator surface produces a series of divergent terms, which through the FG expansion can be associated with various geometric terms involving the intrinsic curvature of the boundary metric. Usually in AdS/CFT calculations, a series of boundary counterterms are added to the action to remove these divergences, as the regulator surface is taken to infinity \cite{Emparan:1999pm}. In the present braneworld construction, the regulator surface is replaced by the brane, which remains at a finite radius, and no additional counterterms are added. Rather the `divergent' terms become contributions to the gravitational action of the brane theory, and hence the latter from previous discussions of the boundary counterterms \cite{Emparan:1999pm}, \ie
\begin{multline}\label{diver1}
I_\mt{diver}=\frac{1}{16\pi \Gbk}\int d^dx \sqrt{-\tilde{g}}\left[\frac{2(d-1)}{L}+\frac{L}{(d-2)}\tilde{R}
\right.\\
+\left. \frac{L^3}{(d-4)(d-2)^2} \left(\tilde{R}^{ij}\tilde{R}_{ij}-\frac{d}{4(d-1)}\,\tilde{R}^2\right) +\cdots\right]\,.
\end{multline}

Several comments are in order at this point: First of all, we note that the above expression is written in terms of the induced metric $\tilde g_{ij}$ on the brane (as in \cite{Emparan:1999pm}) rather than the boundary metric $\overscript{g}{0}_{ij}$ that enters the FG expansion. Using the standard results, \eg \cite{Skenderis:2002wp,deHaro:2000vlm}, we can relate the two with
\beq\label{relate}
\tilde g_{ij}(x_k) = \frac{L^2}{\s^2}\,\overscript{g}{0}_{ij}(x_k) +  \overscript{g}{1}_{ij}(x_k)+\frac{\s^2}{L^2}\, \overscript{g}{2}_{ij}(x_k)+\cdots\,,
\eeq
where the higher order terms can be expressed in terms of the curvatures of $\overscript{g}{0}_{ij}$, \eg
\beq\label{oneg}
\overscript{g}{1}_{ij} = -\frac{L^2}{d-2}\left(R_{ij}\big[\overscript{g}{0}\big] -\frac{\overscript{g}{0}_{ij}}{2(d-1)}\,R\big[\overscript{g}{0}\big]\right)\,.
\eeq
In other words, the two metrics are related by a Weyl scaling and a field redefinition. Further, we see a factor of $(d-2)$ appearing in the denominator of the second term, \ie the Einstein-Hilbert term, in eq.~\reef{diver1}. Hence this expression only applies for $d\ge3$ and must be reevaluated for $d=2$, which we do in section \ref{sec:two-d}. Similar factors, as well as a factor of $d-4$, appear in the denominator of the third term, which again indicates that this expression must be reconsidered for $d=4$.

In any event, the gravitational action on the brane is given by combining the above expression with the brane action \reef{braneaction},
\beq\label{totaction}
I_\mt{induced} = 2\, I_\mt{diver} +  I_\mt{brane}\,,
\eeq
where the factor of two in the first term accounts for integrating over the bulk geometry on both sides of the brane.
The combined result can be written as
\beqa
I_\mt{induced}&=&\frac{1}{16 \pi G_\mt{eff}}\int d^{d}x\sqrt{-\tilde{g}}
\[\frac{(d-1)(d-2)}{\ell_\mt{eff}^2} + \tilde{R}(\tilde{g})\right]
\labell{act3}\\
&&\qquad\quad
+\frac{1}{16 \pi G_\mt{RS}}\int d^{d}x\sqrt{-\tilde{g}}\left[ \frac{L^2}{(d-4)(d-2)}\(\tilde{R}^{ij}\tilde{R}_{ij}-
\frac{d}{4(d-1)} \tilde{R}^2\)+\cdots\]\,,
\nonumber
\eeqa
where
\beq
\frac{1}{G_\mt{eff}}=\frac{1}{G_\mt{RS}}=\frac{2\,L}{(d-2)\,G_\mt{bulk}}\,,
\qquad\qquad
\frac{1}{\ell_\mt{eff}^2}=\frac{2}{L^2}\left(1-\frac{4 \pi \Gbk L T_o}{d-1}\right)\,.
\label{Newton2}
\eeq
In the present discussion $G_\mt{eff}$ and $G_\mt{RS}$ are equal, but by adding terms to the brane action this can change. We will explain this in section \ref{sec:DGP}. Comparing eqs.~\reef{curve2} and \reef{Newton2}, we see that $\ell_\mt{eff}$ (which sets the cosmological constant term in $I_\mt{induced}$) precisely matches the leading order expression for the brane curvature $\lb$. Hence if we only consider the first two terms in eq.~\reef{act3}, the resulting Einstein equations would reproduce the leading expression (in the $\veps$ expansion) for the curvatures in eq.~\reef{Ricky2}. Further, it is a straightforward exercise to show that if the contribution of the curvature squared terms is also included in the gravitation equations of motion, the curvature is shifted to precisely reproduce the $\veps^2$ term in eq.~\reef{Ricky2}. Hence rather than using the Israel junction condtions, we could determine the position of the brane in the backreacted geometry by first solving the gravitational equations of the brane action \reef{act3} and then finding the appropriate surface $z=\s$ with the corresponding curvature. More generally, the fact that these two approaches match was verified by \cite{deHaro:2000wj},\footnote{See also earlier discussions, \eg \cite{Shiromizu:1999wj,Verlinde:1999fy,Gubser:1999vj}.} which argued the bulk Einstein equations combined with the Israel junction conditions are equivalent to the brane gravity equations of motion.\footnote{We note that the brane graviton acquires a small mass through interactions with the CFT residing there \cite{Karch:2000ct,Karch:2001jb,Porrati:2001gx}. However, this mass plays no role in the following as it is negligible in the regime of interest, \ie $L/\ell_\mt{eff}\ll 1$ -- see further discussion in section  \ref{face}. This point was emphasized in \cite{Geng:2020qvw}.}

Of course, the gravitational approach only provides an effective approach in the limit that $\ell_\mt{eff}\gg L$ since otherwise the contributions of the higher curvature terms cannot be ignored.
For example, if the curvatures are proportional to $1/\ell_\mt{eff}^2$ at leading order, then the curvature squared term is suppressed by a factor of $L^2/\ell_\mt{eff}^2$ relative the first two terms. Similarly the higher order curvature terms denoted by the ellipsis in eq.~\reef{act3} are further suppressed by a further factor of $L^2/\ell_\mt{eff}^2$ for each additional curvature appearing these terms. From eq.~\reef{curve2}, we can write $\frac{L^2}{\ell_\mt{eff}^2}=2\veps$ and hence we see that the gravitational brane action and the resulting equations of motion can be organized in the same small $\veps$ expansion discussed in the previous section.\footnote{Note that we have distinguished the gravitational couplings in the Einstein terms and in the higher curvature interactions, \ie in the first and second lines of eq.~\reef{act3}, even though $G_\mt{eff}=G_\mt{RS}$ here. However, this distinction will become important in section \ref{sec:DGP}.}

Recall that we can give a holographic description of this system involving (two weakly interacting copies of) the boundary CFT living on the brane. However, this CFT has a finite UV cutoff because the brane resides at a finite radius in the bulk, \eg see \cite{deHaro:2000vlm,Emparan:2006ni,Myers:2013lva}. The action \reef{diver1} is then the induced gravitational action resulting from integrating out the CFT degrees of freedom. The UV cutoff is usually discussed in the context of the boundary metric $g^{\ssc (0)}_{ij}$, where the short distance cutoff would be given by $\delta\simeq\s$.  However, recall that the gravitational action \reef{act3} is expressed in terms of the induced metric $\tilde g_{ij}$ and so the conformal transformation in eq.~\reef{relate} yields $\tilde\delta\simeq L$ for this description of the brane theory. Therefore the $\veps$ expansion corresponds to an expansion in powers of the short distance cutoff, \ie $\veps\sim\tilde \delta^2/\ell_\mt{eff}^2$.

% !TEX root = ../lifeonbrane3.tex
%

\subsection{The case of two dimensions}\label{sec:two-d}

Recall that the curvature terms in the induced action \reef{diver1} have coefficients with inverse powers of $(d-2)$ and so we must reconsider the calculation of this brane action for $d=2$, \ie when the bulk space is (locally) AdS$_3$ and the induced geometry on the brane is AdS$_2$. This section sketches the necessary calculations, which are largely the same as those performed in higher dimensions, but with a few important differences.

Let us add that in contrast to the induced action, the calculations in section \ref{BranGeo}, where the position of the brane is determined using the Israel junction conditions, need no modifications for $d=2$.
Therefore we can simply substitute $d=2$ into eqs.~\reef{position} and \reef{secondorder} for the brane position to find
\beq\label{2dposition}
\s^2 \simeq 2L^2\veps+ \frac{L}{4 \pi \Gbk T_o}\,\veps^2+\cdots\,, \qquad{\rm with}\quad
\veps=1-4 \pi \Gbk L T_o \,.
\eeq
Of course, we must be able to reproduce the same result using the new induced gravity action.


\subsubsection*{Integration of bulk action}

As discussed in section \ref{indyaction}, one can determine the structure of the terms in the induced action by a careful examination of the FG expansion near the asymptotic boundary \cite{Skenderis:1999nb,deHaro:2000wj,deHaro:2000vlm}. However, we can take the simpler route here, since in two dimensions the Riemann curvature has a single component and therefore the entire induced action can be expressed in terms of the Ricci scalar $\ric(\tilde g)$. Therefore, we evaluate the on-shell bulk action and match the boundary divergences to an expansion in $\ric(\tilde g)$. That is, we substitute the metric \reef{metric3} into the bulk action \reef{act2} plus the corresponding  Gibbons-Hawking-York surface term \cite{PhysRevLett.28.1082,Gibbons:1976ue} and integrate over the radial direction $z$. The result can be expressed as a boundary integral with a series of divergences as $\s\to0$,\footnote{This expression also includes ${\cal O}(\s^2)$ contributions, which are necessary to match eq.~\reef{2dposition} to ${\cal O}(\veps^2)$ in the following. Further, note that we are ignoring the contributions coming from asymptotic boundaries at  $z\to\infty$.}
\beq\label{induced act2}
I_\mt{diver}=\frac{L}{16 \pi \Gbk}\int d^2x\sqrt{-g^{\ads2}} \left[\frac{1}{\s^2} +\frac{1}{L^2}\,\log\Big(\frac{\s}{L}\Big) -\frac{\s^2}{16L^4} +\cdots\right]\,.
\eeq
%where $\sir$ is some IR scale introduced by the upper region of the $z$ integration.
Now we rewrite the above expression in terms of the induced metric and the corresponding Ricci scalar combining eqs.~\reef{metric3}, \reef{curve1} and \reef{Ricky2}, which yield
\beq\label{AdS3}
\sqrt{-\tilde{g}} =  \frac{L^2}{\s^2}\left(1 + \frac{\s^2}{4L^2}\right)^2 \sqrt{-g^{\ads2}} \,, \qquad\quad
\ric = -2\,\frac{\s^2}{L^4}\left(1+\frac{\s^2}{4L^2}\right)^{-2}\,.
\eeq
Using these expressions, %as well as $\lir\equiv L^2/\sir$, 
the induced action becomes\footnote{Our derivation of eq.~\eqref{ind-action2} will miss terms involving derivatives of $\tilde{R}$ as these vanish for the constant curvature geometry of our brane. However,  such terms will only appear at higher orders, \ie in the `$\cdots$' (other than the total derivative $\tilde\Box\tilde{R}$).} 
%The same issue arises in \eqref{sol4}, where we assume the Liouville field solves its equation of motion \eqref{eom4} for constant $\tilde{R}$.}
\beq\label{ind-action2}
I_\mt{diver}=\frac{L}{16 \pi \Gbk}\int d^2x\sqrt{-\tilde{g}} \Big[\frac{2}{L^2} -  \frac12\, \tilde{R} \,\log\left(-\frac{L^2 }{2}\tR\right) + \frac{1}{2}\,\tilde{R}+\frac{L^2}{16}\,\tilde{R}^2 +\cdots\Big]\,.
\eeq

The most striking feature of this induced action is the term proportional to $\ric\log|\ric|$. The appearance of this logarithm is related to the conformal anomaly \cite{Henningson:1998gx,Henningson:1998ey,Burgess:1999vb}, and points towards the fact that the corresponding gravitational action in nonlocal,\footnote{Similar nonlocalities appear in the curvature-squared or four-derivative contributions with $d=4$, or more generally in the interactions with $d/2$ curvatures for higher (even) $d$. Hence they do not play a role in higher dimensions if we work in the regime where the induced action \reef{act3} is well approximated by Einstein gravity coupled to a cosmological constant.} as we discuss next.  Further, since the Einstein-Hilbert term is topological in two dimensions, it turns out that this unusual action is precisely what is needed to match the dynamics of the bulk gravity described above, \ie the position of the brane in eq.~\reef{2dposition}.

The logarithmic contribution  should correspond to that coming from the nonlocal Polyakov action \cite{Skenderis:1999nb}. Schematically, we would have
\beq
I_\mt{bulk}\simeq I_\mt{Poly}= -\frac{\alpha\,L}{16 \pi \Gbk}\int d^2x\sqrt{-\tilde{g}} \, \tilde{R}\,\frac1{\tilde \Box}\,\tilde{R} \,,
\label{induced-action3}
\eeq
where we have introduced an arbitrary constant $\alpha$ here but it will be fixed by comparing with the divergences in the integrated action. Of course, $\frac1{\tilde \Box}\,\tilde{R}$ indicates a convolution of the Ricci scalar with the scalar Green's function, but there are subtleties here in dealing with constant curvatures. The latter are ameliorated by making the action \reef{induced-action3} local by introducing a auxiliary field $\phi$ (\eg see \cite{Skenderis:1999nb,Alvarez:1982zi}),
\beq
 I_\mt{Poly}=\frac{\alpha\,L}{8\pi \Gbk}\int d^2x\sqrt{-\tilde{g}} \, \left[
 -\frac12\,\tilde g^{ij}\tilde\nabla_i\phi\tilde\nabla_j\phi
 +\phi\,\tilde R + \chi\,e^{-\phi}
\right] \,.
\label{PolyAct2}
\eeq
where $\chi$ is a fixed constant.\footnote{The last term is needed to take care of zero mode problem \cite{Alvarez:1982zi}. Examining the equation of motion \reef{eom4}, one can think of $\phi$ as a conformal factor relating the metric $\tg_{ij}$ to a canonical constant curvature metric $\hat g_{ij}$, \ie $\tg_{ij}=e^{\phi}\hat g_{ij}$ with $\hat R(\hat g)=\chi$ \cite{Alvarez:1982zi,Frolov:1996hd}.
Hence we choose $\chi$ to be negative to match the sign of $\tilde R$. Further, note that with the interaction $\chi e^{-\phi}$ in the action \reef{PolyAct2}, $\phi$ becomes an interacting field \cite{Skenderis:1999nb}.
\label{ZZZ}}

The equation of motion resulting from eq.~\reef{PolyAct2}  is
\beq
0=\tilde\Box \phi +\tilde R - \chi\,e^{-\phi}\,,
\label{eom4}
\eeq
which has a simple solution when $\tilde R$ is a constant, namely,
\beq
 \phi=\phi_0 = \log(\chi/\tilde R)\,.
\label{sol4}
\eeq
% \vc{Are we saying that \eqref{ind-action2} and \eqref{induct} should only be trusted for $\tilde{R}$ constant? How do we know that there aren't terms involving derivatives of $\tilde{R}$ which happen to vanish on our constant $\tilde{R}$ brane. Since we should only be including `divergent' things in the induced gravity action, perhaps we just want to extract a constant logarithmically divergent term, as in Section 4.1 in \cite{Almheiri:2019psf} --- more explanation in comment below \eqref{shift0}; also, related to the comment below \eqref{ind-action2}.}
Evaluating the Polyakov action with $\phi = \phi_0$ yields
\beq
 I_\mt{Poly}\big|_{\phi=\phi_0}=-\frac{\alpha\,L}{8\pi \Gbk}\int d^2x\sqrt{-\tilde{g}} \, \left[
 \tilde R \,\log(\tilde R/\chi) - \tilde R\, \right] \,.
\label{PolyAct3}
\eeq
Comparing this expression with the log term in eq.~\reef{ind-action2}, we fix $\alpha=\frac14$ and $\chi=-\frac2{L^2}$.

Varying the action \reef{PolyAct2} with respect to the metric, we find the corresponding contribution to the `gravitational' equations of motion
\beqa
T^\mt{Poly}_{ij}=-\frac2{\sqrt{-g}}\,\frac{\delta I_\mt{Poly}}{\delta g^{ij}}
&=&\frac{L}{32\pi \Gbk}\Big[
\tilde\nabla_i\phi\tilde\nabla_j\phi
 +2\,\tilde\nabla_i\tilde\nabla_j\phi
\label{eom5}\\
&&\qquad\qquad\left.
-\tg_{ij}\left(\frac12\,(\tilde\nabla\phi)^2
 +2\,\tilde\Box \phi- \chi\,e^{-\phi} \right)
\right]\,,
\nonumber
\eeqa
where we have used $\tilde R_{ij} -\frac12\tg_{ij}\tilde R=0$ for $d=2$ to eliminate the terms linear in $\phi$ (without any derivatives). Now substituting $\phi_0$, we find that this expression reduces to
\beq
T^\mt{Poly}_{ij}\big|_{\phi=\phi_0}=
\frac{L}{32\pi \Gbk}\,\tg_{ij}\, \tilde R\,,
\label{onshell5}
\eeq
which we will substitute into evaluating the equations of motion below to fix the position of the brane. As an aside, we can take the trace of the above expression to find
that it reproduces the trace anomaly, \eg \cite{Duff:1977ay,Duff:1993wm}
\beq\label{trace}
\langle T^i{}_i \rangle = \frac{c}{24\pi}\,\tilde R \,,
\eeq
where we recall that $c=\frac{3L}{2\Gbk}$ for the boundary CFT. In our case, the trace anomaly will be twice as large, since there are two copies of the CFT living on the brane.

The induced action $I_\mt{induced} = 2\,I_\mt{diver} + I_\mt{brane}$ can be written as
%\footnote{At this point, we have made the convenient choice to equate $\lir=\ell_\mt{eff}$.}
\beq\label{induct}
I_\mt{induced} = \frac{1}{16 \pi G_\mt{eff}}\int d^2x\sqrt{-\tilde{g}} \Big[\frac{2}{\ell_\mt{eff}^2} -  \tilde{R} \,\log\left(-\frac{L^2 }{2}\tR\right)+\tR +\frac{L^2}{8}\,\tilde{R}^2 +\cdots\Big]\,,
\eeq
where $\ell_\mt{eff}$ is given by the expression in eq.~\reef{Newton2} with $d=2$, \ie
\beq\label{lobster}
\frac{L^2}{{\ell}_\mt{eff}^2}=2\( 1-4 \pi \Gbk L T_o\)\,,
\eeq
however, we have set $\Geff =  \Gbk/L$ here. The metric variation then yields the following equation of motion
\beq\label{gamble3}
0=\frac{2}{\ell_\mt{eff}^2}\,\tilde{g}_{ij}+\tilde{g}_{ij} \,
\tilde{R} + \frac{L^2}8\,\tR\(\tg_{ij}\,\tR-4 \tR_{ij}\)+\cdots\,,
\eeq
where we dropped the terms involving derivatives of curvatures arising from the variation of the $\tR^2$ term.
To leading order, we find $\tR\sim -2/\ell_\mt{eff}^2 = -4\veps/L^2$ in agreement with eqs.~\reef{curve2} and \reef{Ricky2}. Hence, the gravitational equations of motion again fix the (leading-order) position of the brane for $d=2$, and further it is a straightforward exercise to match to second order corrections in eq.~\reef{2dposition} using the curvature-squared contributions in eq.~\reef{gamble3}.


\subsubsection*{Adding JT gravity}

Much of the recent literature on quantum extremal islands examines models involving two-dimensional gravity, \eg \cite{Almheiri:2019psf, Almheiri:2019hni, Almheiri:2019yqk, Chen:2019uhq, Penington:2019kki, Almheiri:2019qdq, Chen:2019iro}, however, the gravitational theory in these models is Jackiw-Teitelboim (JT) gravity \cite{Jackiw:1984je,Teitelboim:1983ux}. One can incorporate JT gravity into the current model by dropping the usual tension term \reef{braneaction}, and instead using the following brane action\footnote{Alternatively, one could simply add $I_\mt{JT}$ to the usual tension term. With this approach, an extra source term appears in eq.~\reef{fulleom}, but it can be eliminated by shifting the dilaton in a manner similar to eq.~\reef{shift1}.} 
\beq\label{braneact2}
I_\mt{brane}= I_\mt{JT} + I_\mt{ct}\,,
\eeq
where the JT action takes the usual form,
\beq\label{JTee}
I_\mt{JT} =\frac{1}{16\pi G_\mt{brane}}\int d^2x\sqrt{-\tilde{g}}\left[\Phi_0\,\tilde{R}+ \Phi\left(\tilde{R}+\frac{2}{\ell^2_\mt{JT}}
\right)\right]\,.
\eeq
Here, as in previous actions, we have ignored the boundary terms associated with the JT action, \eg see \cite{Maldacena:2016upp}, and we have introduced the dilaton $\Phi$. Recall that $\Phi_0$ is simply a constant and so the first term is topological but contributes to the generalized entropy. In eq.~\reef{braneact2}, we have also included a counterterm 
\beq\label{count123}
I_\mt{ct}=-\frac{1}{4\pi \Gbk L}\int d^2x\sqrt{-\tilde{g}}\,,
\eeq
which is tuned to cancel the induced cosmological constant on the brane. This choice ensures that the JT gravity \reef{JTee} couples to the boundary CFT in the expected way, \eg as in  \cite{Almheiri:2019psf,Maldacena:2016upp} -- see further comments below.

The full induced action now takes the form
\begin{align}
&&I_\mt{induced}=\frac{1}{16 \pi G_\mt{eff}}\int d^2x\sqrt{-\tilde{g}} \Big[ -  \tilde{R} \,\log\left(-\frac{L^2 }{2}\tR\right) +\frac{L^2}{8}\,\tilde{R}^2 +\cdots\Big]
\nonumber\\
&&\qquad+\frac{1}{16\pi G_\mt{brane}}\int d^2x\sqrt{-\tilde{g}}\left[\tilde\Phi_0\,\tilde{R}+ \Phi\left(\tilde{R}+\frac{2}{\ell^2_\mt{JT}}
\right)\right]\,,
\label{fullindyact}
\end{align}
where we have combined the two topological contributions in the second line with\footnote{In \cite{Almheiri:2019psf}, $\tilde \Phi_0$ would also absorb a logarithmic constant $-2\log(L/z_\mt{B})$, which would be accompanied by a shift in the prefactor in the argument of the logarithmic term in eq.~\reef{fullindyact}, \ie $2/L^2\to2/ z^2_\mt{B}$.}
\beq\label{shift0}
\tilde \Phi_0 =\Phi_0 + G_\mt{brane}/G_\mt{eff}\,.
\eeq

Now, with the JT action \reef{JTee}, the dilaton equation of motion fixes $\tilde R=-2/\ell^2_\mt{JT}$, \ie the brane geometry is locally AdS$_2$ everywhere with $\ell_\mt{B}= \ell_\mt{JT}$. Then the position $\s$ of the brane is fixed by eq.~\reef{curve1} and implicitly we assume that $\ell_\mt{JT}\gg L$, which ensures that $\s\ll L$ as in our previous discussions. The gravitational equation of motion coming from the variation of the metric becomes
\beq\label{fulleom}
-\nabla_{i}\nabla_{j}\Phi+\tilde{g}_{ij}\(\nabla^2\Phi-\frac{\Phi}{\ell^2_\mt{JT}}\)= 8\pi G_\brane\, \widetilde T^\mt{CFT}_{ij}
= -\frac{\Gbr}{ \Geff}\, \frac{1 }{\hat{\ell}_\mt{eff}^2}\,\tilde{g}_{ij} \,,
\eeq
where $\hat{\ell}_\mt{eff}$ is the effective curvature scale produced by $\ell_\mt{JT}$.
That is, in the case without JT gravity, we can combine eqs.~\reef{positionbrane}, \reef{curve1} and \reef{Newton2} to find
\beq\label{curve33}
\frac{L^2}{{\ell}_\mt{eff}^2}= f\!\(\frac{L^2}{\ell_\mt{B}^2}\)\equiv 2\(1-\sqrt{1-\frac{L^2}{\ell_\mt{B}^2}}\,\)  \,.
\eeq
We can understand this expression as the gravitational equation of motion coming from the two-dimensional action \reef{induct}, where a Taylor expansion of the right-hand side for $L/\ell_\mt{B}\ll1$ corresponds to varying the curvature terms and subsequently substituting $\tilde R_{ij}=-\frac{1}{\ell^2_\mt{B}}\,\tilde g_{ij}$, as in eq.~\reef{Ricky2}.  Now in the JT equation of motion \reef{fulleom}, the effective curvature scale $\hat{\ell}_\mt{eff}$ satisfies ${L^2}/{\hat{\ell}_\mt{eff}^2}= f\!\({L^2}/{\ell_\mt{JT}^2}\)$. We have indicated in eq.~\reef{fulleom} that the left-hand side corresponds to the stress tensor of the boundary CFT which lives on the brane. In the present arrangement,\footnote{In  more interesting scenarios, \eg with evaporating black holes as in \cite{Almheiri:2019psf,Almheiri:2019hni,Chen:2019uhq}, it is more appropriate to work directly with the CFT's stress tensor, rather than replacing these degrees of freedom by an effective gravity action after integrating out the CFT.} this takes a particularly simple form, with $T^\mt{CFT}_{ij}\propto \tilde{g}_{ij}$. Of course, this source term in eq.~\reef{fulleom} can be easily absorbed by shifting the dilaton, 
\beq\label{shift1}
\tilde\Phi\equiv \Phi- \frac{\Gbr}{ \Geff}\, \frac{\ell^2_\mt{JT} }{\hat{\ell}_\mt{eff}^2}\,,
\eeq
so that $\tilde\Phi$ satisfies the usual source-free equation studied in \eg \cite{Maldacena:2016upp}.

At this point, we observe that  the trace of eq.~\reef{fulleom} yields on the right-hand side,
\beq\label{almost}
\langle \big[\widetilde T^\mt{CFT}\big]^i{}_i \rangle = -\frac{L}{ 4\pi\Gbk}\, \frac{1 }{\hat{\ell}_\mt{eff}^2}=-\frac{L}{ 4\pi\Gbk}\, \frac{1 }{\ell_\mt{JT}^2}\(1+\frac14\,\frac{L^2}{\ell_\mt{JT}^2}+\frac18\,\frac{L^4}{\ell_\mt{JT}^4}+\cdots\)\,,
\eeq
where in the final expression, we are Taylor expanding $f(L^2/\ell_\mt{JT}^2)$ assuming $L^2/\ell_\mt{JT}^2\ll 1$, as above. Noting that $\tilde R=-2/\ell^2_\mt{JT}$ and comparing to eq.~\reef{trace},\footnote{Recall that the central charge here is twice that appearing in eq.~\reef{trace} because the brane supports two (weakly interacting) copies of the boundary CFT.} we see that the expected trace anomaly has recieved a infinite series of higher order corrections. We can interprete the latter as arising from the finite UV cutoff on the brane, recalling that $\tilde\delta\simeq L$ as discussed at the end of section \ref{indyaction}. 


% !TEX root = ../lifeonbrane3.tex
%

\subsection{DGP Gravity on the Brane} \label{sec:DGP}
The previous discussion of $d=2$ motivates that it is interesting to add an intrinsic gravity term to the brane action. Here, we extend this discussion to higher dimensions, \ie extend the brane action to include an Einstein-Hilbert term. Of course, this scenario can be viewed as a version of Dvali-Gabadadze-Porrati (DGP) gravity \cite{Dvali:2000hr} in an AdS background. Hence, it combines features of both RS and DGP gravity theories. We discuss the modifications of the brane dynamics and the induced action below, but it also produces interesting modifications of the generalized entropy, as discussed in sections \ref{HEE} and appendices \ref{generalE} and \ref{bubble}.

We write the extended brane action, replacing eq.~\reef{braneaction}, as
\beq\label{newbran}
I_\mt{brane} = -(T_o-\Delta T)\int d^dx \sqrt{-\tilde{g}} + \frac{1}{16\pi \Gbr}\int d^dx \sqrt{-\tilde{g}} \tilde{R}\,.
\eeq
In general, for a fixed brane tension, the position of the brane will be modified with the additional Einstein-Hilbert term. Hence we have parametrized the full brane tension as $T_o-\Delta T$ and the contribution $\Delta T$ will be tuned to keep the position of the brane fixed. This choice will facilitate the comparison of the generalized entropy between different scenarios in the following.

As in section \ref{BranGeo}, the position of the brane can be determined using the Israel junction conditions \reef{Israel1}. Hence we begin by evaluating the brane's stress tensor,
\beq\label{stressbran}
S_{ij}\equiv -\frac{2}{\sqrt{-\tilde g}}\,\frac{\delta I_\mt{brane}}{\delta \tg^{ij}}=-\tilde{g}_{ij}(T_o-\Delta T) -\frac{1}{8\pi \Gbr}\left(\ric_{ij}-\frac12\tg_{ij}\,\ric\right)\,.
\eeq
As commented above, we choose $\Delta T$ to cancel the curvature contributions in this expression, \ie the stress tensor reduces to $S_{ij}=-T_o\,\tilde{g}_{ij}$. With this tuning, the Israel junction conditions in eq.~\reef{Israel1} are unchanged as the analysis which follows from there. Therefore the brane position and curvature remain identical to those determined in eqs.~\reef{positionbrane} and \reef{curve1}. This allows use to determine the desired tuning as
\beq\label{tune3}
\Delta T = \frac{(d-1)(d-2)}{16\pi \Gbr\,\ell_\mt{B}^2}\ 
\simeq \frac{(d-1)(d-2)}{8\pi \Gbr\,L^2}\,\veps\,.
\eeq
We have used eq.~\reef{curve2} to show that the shift in the brane tension is small in the $\veps$ expansion. 

We return to the induced gravitational action on the brane that takes the same form as in eq.~\reef{act3} but with the effective Newton's constant in eq.~\reef{Newton2} replaced by
\beq
\frac{1}{G_\mt{eff}}=\frac{2L}{(d-2)\,G_\mt{bulk}}+\frac{1}{\Gbr}\,.
\label{Newton33}
\eeq
By construction, $\ell_\mt{eff}$ and the position of the brane are unchanged. Note that the gravitational couplings in the Einstein terms and in the higher curvature interactions, \ie in the first and second lines of eq.~\reef{act3},  are now distinct. That is, $G_\mt{eff}$ no longer equals $G_\mt{RS}$. 

In the following, it will be useful to define the ratio
\beq\label{newdefs}
\lamb=\frac{G_\mt{RS}}{\Gbr}
\qquad{\rm with}\quad \frac{1}{G_\mt{RS}}=\frac{2L}{(d-2)\,G_\mt{bulk}}\,,
\eeq
where $G_\mt{RS}$ is the induced Newton's constant on an RS brane appearing in eq.~\reef{Newton2}, while the dimensionless ratio $\lamb$ controls the relative strength of the Newton's constants in the bulk and on the brane. With these definitions, the induced Newton's constant on the DGP brane, in eq.~\reef{Newton33}, can be rewritten as
\beq\label{Newton34}
\frac{1}{G_\mt{eff}}=\frac{1}{G_\mt{RS}}\(1+\lamb\)\,.
\eeq\\

Of course, one can also consider other modifications of the brane action beyond adding the Einstein-Hilbert term in eq.~\reef{newbran} -- see discussion in the next subsection and \cite{domino}. Further, we will discuss adding topological gravitational terms on the brane or in the bulk in sections \ref{HEE} and \ref{sec:discussion}. In particular, we will see in section \ref{sec:examples} that adding a Gauss-Bonnet term to the four-dimensional bulk gravity theory yields another tuneable parameter which, for a certain parameter range, makes it possible to find quantum extremal islands in the absence of black holes. 


\section{Three perspectives: Bulk/Brane/Boundary}\label{face}
% !TEX root = ../lifeonbrane3.tex
%

Our setup can be interpreted from three different `holographic' perspectives, which are analogous to the three descriptions of \cite{Almheiri:2019hni}, suitably generalised to arbitrary dimensions. A set of analogous descriptions for gravity on a brane in higher dimensions was discussed in the context of the Karch-Randall model \cite{Karch:2000ct}, and in fact, these are the models discussed here with the addition of the DGP term \reef{newbran}. In this section we review each of the dual descriptions, and explore their relation. 

First, consider the {\it bulk gravity perspective} corresponding to the geometric picture portrayed in section \ref{BranGeo}: we have an AdS$_{d+1}$ bulk region where gravity is dynamical, containing a DGP brane with tension running through the middle of the spacetime  -- see figure \ref{fig:threetales}a. The induced geometry on the brane is AdS$_d$. In the second picture, we integrate out the bulk action from the asymptotic boundary where gravity is frozen up to the brane, giving rise to Randall-Sundrum gravity \cite{Randall:1999vf,Randall:1999ee,Karch:2000ct} on the brane. From the resulting {\it brane perspective}, the CFT$_d$ is then supported in a region with dynamical gravity (\ie the brane) and another non-dynamical one (\ie the asymptotic boundary) -- figure \ref{fig:threetales}b. Finally, the third description makes full use of the AdS/CFT dictionary, by using holography {along} the brane. This {\it boundary perspective} describes the system as a CFT$_d$ coupled to a conformal defect that is located at the position where the brane intersects the asymptotic boundary -- see figure \ref{fig:threetales}c. 

A holographic system was presented in \cite{Almheiri:2019hni} to describe the evaporation of two-dimensional black holes in JT gravity. This system has three descriptions analogous to those above. Of course, it also includes certain elements that we did not introduce in our model, \ie end-of-the-world branes to give a holographic description of conformal boundaries separating various components  \cite{Takayanagi:2011zk,Fujita:2011fp} and performing a $\mathbb Z_2$ orbifold quotient across the Planck brane, \ie the brane supporting JT gravity. However, the essential ingredients are the same as above. The boundary perspective in \cite{Almheiri:2019hni} describes the system as a two-dimensional holographic conformal field theory with a boundary, at which it couples to a (one-dimensional) quantum mechanical system -- figure \ref{fig:threetales}f. With the brane perspective, the quantum mechanical system is replaced by its holographic dual, the Planck brane supporting JT gravity coupled to another copy of the two-dimensional holographic CFT -- see figure \ref{fig:threetales}e. Finally, the bulk gravity perspective replaces the holographic CFT with three-dimensional Einstein gravity in an asymptotically AdS$_3$ geometry. Because of the $\mathbb Z_2$ orbifolding, the latter effectively has two boundaries, the standard asymptotically AdS boundary and the dynamical Planck brane -- see figure \ref{fig:threetales}d.

This initial model \cite{Almheiri:2019hni} raised a number of intriguing puzzles. For example, as emphasized in \cite{Almheiri:2019yqk}, implicitly two different notions of the radiation degrees of freedom are being used: one being the semi-classical approximation and the other one in the purely quantum theory. Here, we will explain some details of the higher dimensional construction which allow us to provide a resolution of several of these questions in section \ref{sec:discussion}.

\begin{figure}[h]
	\def\svgwidth{0.9\linewidth}
	\centering{
		\input{ThreeTales_EditcopyJS4.pdf_tex}
		\caption{This figure shows the relation between a time-slice in our construction and the holographic setup of \cite{Almheiri:2019hni}. The top row illustrates three perspectives with which the system discussed here can be described, while the bottom row displays the analogous descriptions for the model in \cite{Almheiri:2019hni}. The comparison can be made more precise by performing a $\mathbb Z_2$ orbifold quotient across the bulk brane/conformal defect in the top row. \\
			\textbf{a.} Bulk gravity perspective, with an asymptotically AdS$_{d+1}$ space (shaded blue) which contains a co-dimension one Randall-Sundrum brane (shaded grey).\\ 
			\textbf{b.} Brane perspective, with dual CFT$_d$ on the asymptotic boundary geometry (blue) and also extending on the AdS$_d$ region (shaded green) where gravity is dynamical.\\
			\textbf{c.} Boundary perspective, with the holographic CFT$_d$ on $S^{d-1}$ (blue) coupled to a codimension-one conformal defect (green).\\
			\textbf{d.} AdS$_3$ formulation with two boundary components: the flat asymptotic boundary (straight black line) and a ``Planck brane'' (curved black line) with an AdS$_2$ geometry.\\ 
			\textbf{e.} The holographic CFT extends over a region with a fixed metric (blue) and an AdS$_2$ region with JT gravity (green).\\
			\textbf{f.} The microscopic description as a two-dimensional BCFT (blue) coupled to a quantum mechanical system at its boundary (green).\\
		}
		\label{fig:threetales}
	}
\end{figure}




\begin{figure}[h]
	\def\svgwidth{1\linewidth}
	\centering{
		\input{bulkmodes_JSedit.pdf_tex}
		\caption{This figure illustrates the spatial profile of the first few normalized graviton modes in the presence of a large tension brane, and a $\mathbb Z_2$ orbifolding across the brane. We use the spatial coordinate $\mu$, related to $\rho$ in eq.~\reef{metric} by $\cot \mu = \sinh\rho/L$. The tension is adjusted such that the location of the brane is at $\mu = \mu_\mt{B}$ with $\mu_\mt{B}\lesssim\pi$. As discussed in the main text, the presence of the brane creates new bulk modes (orange), which are highly localized at the brane, and which play the role of a (nearly massless) graviton on the brane. The remaining bulk modes appear as KK modes in the brane theory.   }
                \label{fig:boundstate} 	}
\end{figure}


\paragraph{Bulk gravity perspective:} As discussed in section \ref{BranGeo}, the system has a bulk description in terms of gravity on an asymptotically AdS$_{d+1}$ spacetime containing a codimension-one brane, which splits the bulk into two halves -- see figure \ref{fig:threetales}a. The brane is characterized by the tension $T_o$ and also the DGP coupling $1/\Gbr$, introduced in eqs.~\eqref{braneaction} and \eqref{newbran}, respectively. We can use the Israel junction conditions \reef{Israel1} to determine the location of the brane as embedded in the higher dimensional space. The backreaction causes warping around the brane, and after a change of coordinates, tuning the brane tension can be understood as moving the brane further into a new asymptotic AdS region, as seen in eq.~\reef{positionbrane} or \reef{position}. For large brane tension, \ie with $\veps \ll 1$, the spectrum of graviton fluctuations in the bulk is almost unchanged with respect to the modes in empty AdS space. However, a new set of graviton states also appear localized at the brane \cite{Randall:1999vf,Randall:1999ee}, as illustrated figure \ref{fig:boundstate}. These are created by the nonlinear coupling of gravity to the brane. Unlike in the Randall-Sundrum model with a flat or de Sitter brane, the new graviton modes are not actually massless on the brane, but merely very light states whose wavefunction peaks around the brane \cite{Karch:2000ct,Karch:2001jb}. The remaining bulk graviton modes appear as a tower of  Kaluza-Klein states, from the point of view of the theory on the brane, with masses of ${\cal O}(1/\ell_\mt{eff})$ set by the curvature scale of the $d$-dimensional AdS geometry on the brane. These results have been studied in quite some detail \cite{Karch:2000ct,Karch:2001jb,Porrati:2001db,Miemiec:2000eq,Schwartz:2000ip,Porrati:2001gx} for Randall-Sundrum branes, but it is interesting to examine how the spectrum is modified by the DGP term \reef{newbran}. We will make some qualitative statements about this question below, but leave a detailed quantitative discussion and the interpretation of this mechanism from the point of view of the CFT for future work \cite{domino}. 
%A similar set of localized modes appears for each bulk field, however, their boundary conditions at the brane differ from that of the graviton.\rcm{??} 





\paragraph{Brane perspective:} This second perspective, discussed in section \ref{indyaction}, effectively integrates out the spatial direction between the asymptotically AdS boundary and the brane to produce an effective action \reef{act3} for Randall-Sundrum/DGP gravity on the brane, with the new localized graviton state playing the role of the $d$-dimensional graviton. Hence, we are left with a $d$-dimensional theory of gravity coupled to (two copies of) the dual CFT on the brane -- see figure \ref{fig:threetales}b. As discussed in the description of the bulk perspective, amongst the new localized bulk modes, we have an almost massless graviton but also a tower of massive Kaluza-Klein states with masses of ${\cal O}(1/\ell_\mt{eff})$.  In section \ref{indyaction}, we demonstrated the consistency between the bulk gravity perspective and the brane perspective by observing how the equations of motion of the new effective action fix the brane position in the ambient spacetime. Of course, the bulk physics is also dual to the dual CFT on the asymptotic AdS$_{d+1}$ boundary, and so this description is completed by coupling the gravitational and CFT degrees of freedom on the brane to the  CFT on the fixed boundary geometry. We refer to that latter as the {\it bath} CFT. Next, we discuss how different parameters in the brane perspective are related to bulk parameters. 

There are four independent parameters which characterize the gravitational theory on the brane: the curvature scale $\ell_\mt{eff}$, the effective Newton's constant $G_\mt{eff}$, the central charge of the boundary CFT $\cT$, and the effective short-distance cutoff $\tilde\delta$. These emerge from the bulk theory through the four parameters characterizing the latter: the bulk curvature scale $L$, the bulk Newton's constant $\Gbk$, the brane Newton's constant $\Gbr$ and the brane tension $T_o$.\footnote{Recall that $\Delta T$ is determined by these parameters in eq.~\reef{tune3}, as well as eqs.~\reef{positionbrane} and \reef{curve1}.} From eq.~\reef{Newton2}, we see that $\ell_\mt{eff}$ is determined by a specific combination of $T_o$, $\Gbk$ and $L$. Similarly, $G_\mt{eff}$ is determined by $\Gbr$, $\Gbk$ and $L$ in eq.~\reef{Newton33}. The central charge of the boundary CFT is given by the standard expression $\cT\sim L^{d-1}/G_\mt{bulk}$, \eg see \cite{Buchel:2009sk}. 

Lastly, as discussed in section \ref{indyaction},  the theory on the brane comes with a short-distance cutoff $\tilde \delta$ \cite{deHaro:2000vlm,Emparan:2006ni,Myers:2013lva} at which the description of the brane theory in terms of (two copies of) the boundary CFT coupled to Einstein gravity breaks down. Following a standard bulk analysis, one would see that correlators of local operators (with appropriate gravitational dressings) now longer exhibit the expected CFT behaviour at short distances of order 
\beq
\tilde\delta_\mt{CFT}\sim L\,.
\label{ctoff2}
\eeq
We denote this cutoff with the subscript `CFT' to emphasize that the description of the matter degrees of freedom on the brane as a local $d$-dimensional CFT is failing at distances smaller than this short-distance cutoff. However, we stress that there is another scale $\tilde \delta_\mt{GR}$, which is the distance at which the approximation of Einstein gravity on the brane breaks down. The simple parameter counting above shows that this cannot be an independent scale. For the brane perspective, the true cutoff $\tilde \delta$ where the description in terms of the dual CFT coupled to Einstein gravity fails is 
\beq\label{ctoff}
\tilde \delta={\rm max}\left\{\tilde \delta_\mt{CFT}\,,\ \tilde \delta_\mt{GR}\right\}\,.
\eeq
We now discuss how $\tilde \delta_\mt{GR}$ is related to the other scales in the brane theory.

Recall that integrating out the bulk degrees of freedom produces a series of higher curvature terms in the effective action \reef{act3}, and hence demanding that $d$-dimensional Einstein gravity provides a good approximation of the brane theory introduces constraints. The suppression of these higher curvature corrections requires that the ratio $L/\ell_\mt{eff}$ be small. However, if we examine eq.~\reef{act3} carefully and note the distinction $G_\mt{eff}\ne G_\mt{RS}$, then suppressing the curvature-squared terms requires that
\beq
\frac{1}{1+\lamb}\,\frac{L^2}{\ell_\mt{eff}^2}
\ll 1\,,
\label{lmfao}
\eeq
using eq.~\reef{Newton34}. Note that for fixed bulk and boundary curvature scales, this implies a lower bound on the DGP term, such that $\lamb$ cannot be arbitrarily close to $-1$. For a pure RS brane with no additional DGP gravity, \ie $\lambda_b=0$, we conclude that the cutoff below which we find Einstein gravity coincides with the CFT cutoff $\tilde\delta_\mt{GR} \sim \tilde\delta_\mt{CFT}\sim L$. More generally then, the above expression suggests that the DGP term \reef{newbran} affects a shift producing a new short-distance cutoff for gravity,
\beq
\tilde\delta_\mt{GR} \sim \frac{L}{\sqrt{1+\lamb}}\ \sim
\frac{\tilde\delta_\mt{CFT}}{\sqrt{1+\lamb}}\,.
\label{haiku}
\eeq
Hence the true cutoff \reef{ctoff} depends on the sign of $\lamb$ -- we return to this point below. We should note that this result only applies for $d>4$. For $d=4$, the coefficient of the curvature-squared term is logarithmic in the cutoff, while for $d=2$ or $3$, this interaction is not associated with a UV divergence. 


While the above are UV effects, there are also IR effects resulting from having a large number of matter degrees of freedom propagating on the brane, as explained in \cite{Dvali:2007hz,Dvali:2007wp,Reeb:2009rm}. The usual regime of validity for QFT in semiclassical gravity lies at energy scales below the Planck mass, or at distance scales larger than $G_\eff^{1/(d-2)}$. However, the boundary CFT has a large number of degrees of freedom, as indicated by the large $\cT$, and hence the semiclassical description of gravity in fact breaks down much earlier. A direct way to see this breakdown \cite{Dvali:2007wp} is to consider the computation of the (canonically normalized) graviton two-point function. In the high energy approximation, \ie ignoring the AdS geometry, we have here:\footnote{This propagator argument can also be applied for the higher curvature terms discussed above. For example, the curvature-squared terms gives a perturbative correction: $\langle h(p) \,h(-p)\rangle \sim p^{-2}\[1 + \frac{L^2}{1+\lamb}\, p^{2}+\cdots\]$. Hence this approach yields the same result for the cutoff in eq.~\reef{haiku}.}
\beq
\langle h(p)\, h(-p)\rangle \sim p^{-2}\[1 + \cT\, G_\eff\, p^{d-2}+\cdots\]\,.
\label{Dvali1}
\eeq
The leading correction arises from a diagram involving the external gravitons coupling to the CFT stress tensor two-point function. We see that such corrections are only suppressed relative to the `tree-level' result for momenta below a cutoff scale of order $(\cT G_\eff)^{-1/(d-2)}$. 
For our model, the gravitational theory of the brane can therefore only be treated semiclassically for distance scales larger than
\beq
\tilde\delta_\mt{GR} \sim (\cT G_\eff)^{1/(d-2)} \sim \frac{L}{(1+\lambda_b)^{1/(d-2)}}\sim \frac{\tilde\delta_\mt{CFT}}{(1+\lambda_b)^{1/(d-2)}}\,.
\label{Dvali2}
\eeq
Again, for a pure RS brane with $\lambda_b=0$, the cutoffs for Einstein gravity and the CFT agree, yielding $\tilde\delta \sim L$. However, the addition of a DGP gravity term modifies the cutoff, but in a manner distinct from eq.~\reef{haiku}, produced by the higher curvature terms. Note that the above result applies for $d\ge 3$.

The distinction between these two cutoffs indicates that these are really two different physical phenomena contributing to the breakdown of Einstein gravity in the brane perspective. Note that $\lamb>0$, in both eqs.~\reef{haiku} and \reef{Dvali2}, the effect is to produce a shorter cutoff scale, however, the second limit \reef{Dvali2} is the first to contribute (where we are assuming $d>4$). However, this result is smaller that $\tilde\delta_\mt{CFT}$ and hence from eq.~\reef{ctoff}, we find
\beq\label{ctoffplus}
\lamb>0\ \ :\qquad \tilde \delta \sim \tilde\delta_\mt{CFT}\sim L\,.
\eeq
On the other hand with $\lamb<0$, the cutoff $\tilde\delta_\mt{GR}$ is pushed to larger distance scales. In this case, eq.~\reef{haiku} is the first to modify the gravitational physics on the brane as we move to smaller distances. Further since this result is now larger than the CFT cutoff, in this regime, eq.~\reef{ctoff} yields
\beq\label{ctoffminus}
\lamb<0\ \ :\qquad \tilde \delta \sim \tilde\delta_\mt{GR}\sim \frac{L}{\sqrt{1+\lamb}}\,.
\eeq


Let us also note that the latter effect, \ie CFT corrections to the graviton propagator, are also responsible for the mass of the brane graviton \cite{Porrati:2001db}. It is interesting to note that if we take the high energy limit of the corrections to the graviton propagator, eq.~\eqref{Dvali1}, we can estimate a mass correction for low energy gravitons mode of roughly
\begin{align}
\label{eq:mass_correction}
\frac{\cT G_\eff}{\ell_\mt{eff}^{d}} \sim \frac{1}{(1 + \lambda_b)\, \ell_\mt{eff}^{\,2}} \left( \frac{L}{\ell_\mt{eff}}\right)^{d-2},
\end{align}
where we have substituted the $d$-dimensional AdS scale as a lower bound on the momentum. The scaling with the d-dimensional cosmological constant $- \frac 1 {\ell^2_\text{eff}}$ agrees with predictions in the Karch-Randall model \cite{Miemiec:2000eq, Schwartz:2000ip}. However, we caution the reader that the above argument by which we obtained the scaling is heuristic at best. Importantly, whether or not the graviton actually obtains a mass correction depends on the boundary conditions of the matter fields in AdS and can therefore not be determined by a local argument alone
\cite{Porrati:2001db}. However, taking eq.~\eqref{eq:mass_correction} at face value, we also see that a negative DGP coupling increases the mass scale, and vice versa for a positive coupling. This can be confirmed explicitly from bulk calculations \cite{domino}. 

\paragraph{Boundary perspective:} As the preceding discussion has made clear, the theory obtained by integrating out the bulk between the asymptotic boundary and the brane, has an effective description of the brane in terms of a local $d$-dimensional CFT coupled to Einstein gravity up to some cutoff \reef{ctoff}. However, the standard rules of AdS/CFT also allow for a fully microscopic description of the system in terms of the boundary theory. This is obtained by integrating out the bulk -- including the brane -- and the result  is given by the bath CFT on the fixed $d$-dimensional boundary geometry coupled to a ($d-1$)-dimensional conformal defect (positioned where the brane reaches the asymptotic boundary, \ie the equator of the boundary sphere) -- see figure \ref{fig:threetales}c. 

The bath CFT is characterized by the central charge $\cT\sim L^{d-1}/G_\mt{bulk}$, while the defect is characterized by its defect central charge $\tilde{c}_\mt{T}\sim \ell_\mt{eff}^{d-2}/G_\mt{eff}$. We note that in the absence of a DGP term, increasing the brane tension increases the defect central charge $\tilde c_\mt{T}$. Further, we note that the ratio of these two charges is given by
\beq\label{eq:eq:large_central_charges}
\frac{\tilde c_\mt{T}}{\cT} \sim \left(\frac{\ell_\mt{eff}}{L}\right)^{d-2} (1+\lambda_b) \,.
\eeq
Following the standard AdS/CFT dictionary, the ratio $\ell_\mt{eff}/L$ also  translates to a ratio of couplings in the defect and bath CFTs,\footnote{Remember that the AdS/CFT dictionary tells us that $G_N\sim \ell_\mt{AdS}^{d-1}/ N_{dof}$ and $\lambda_\text{Hooft}\sim (\ell_\mt{AdS}/\ell_\mt{s})^{d}$.}
\beq\label{ratou}
{\tilde \lambda}/{\lambda}\sim{\ell_\mt{eff}}/{L}  \,.
\eeq
Since we do not have a particular string construction in mind here, $\lambda$ should be thought of some positive power of the `t Hooft coupling of the bath CFT, while $\tilde\lambda$ will be some (different) positive power of the analogous coupling for the defect CFT.

Now the parameters in this boundary description must be constrained if we want to be in the regime where the brane perspective is valid. In particular, the latter requires that the brane curvature scale must be much larger than the effective cutoff, \ie
\beq
\ell_\mt{eff}/\tilde\delta \gg1\,.
\label{ratou3}
\eeq
Now as described above, the cutoff has a separate form depending on whether $\lamb$ is positive or negative. Eq.~\reef{ctoffplus} applies for $\lamb>0$, which then yields $\ell_\mt{eff}/L \gg1$. Hence we must have ${\tilde \lambda}/{\lambda}\gg1$ and also $\tilde c_\mt{T}/\cT\gg1$ since $1+\lamb>1$ in this case. Similarly for $\lamb<0$, combining eqs.~\reef{ctoffminus} and \reef{ratou3} yields $\ell_\mt{eff}/L \gg1/\sqrt{1+\lamb}$. In this case, $1+\lamb<1$ and it is straightforward to again show that the ratios must be constrained in the same manner. Hence for either sign of $\lamb$, we have 
\beq
{\tilde \lambda}/{\lambda}\gg1\qquad{\rm and}\qquad \tilde c_\mt{T}/\cT\gg1\,.
\label{ratou4}
\eeq
The large ratio of the central charges can also be heuristically understood requiring that energy and information are only leaking very slowly from the dynamical gravity region into the bath \cite{Rozali:2019day}. It has been argued that this ratio also sets the Page time \cite{Rozali:2019day}. With the boundary perspective, this can be understood as a requirement which ensures that the degrees of freedom on the defect and the CFT only slowly mix.

Lastly, the $d$-dimensional graviton can be understood as a field dual to the lightest operator appearing in the boundary OPE expansion of the CFT stress energy tensor \cite{Aharony:2003qf}. At weak coupling, one would naively assume that the lightest operator has dimension $\Delta = d$. However, due to strong coupling effects it becomes possible that a negative anomalous dimension of roughly $-1$ is obtained, so that the corresponding operator can act as the holographic dual to a $d$-dimensional graviton. The mass of the lightest state then signals that the anomalous dimension is not quite $-1$, such that the dimension of the boundary operator dual to the graviton is $\Delta \geq d-1$. 


%%% Local Variables:
%%% mode: latex
%%% TeX-master: "../lifeonbrane3"
%%% End:


\section{Holographic EE on the Brane}\label{HEE}
\input{sections/HEE1}
% !TEX root = ../lifeonbrane3.tex

\subsection{Explicit Calculations}\label{sec:examples}

In this section, we explicitly evaluate the holographic EE and examine the transition between the two classes of RT surfaces. While we set up the calculations for general $d>2$, our explicit results are given for $d=3$ in which case the bulk spacetime locally has the geometry of AdS$_{4}$. We add some comments about $d=2$, and the addition of Jackiw-Teitelboim gravity \reef{JTee} on the brane, in the discussion section.

\subsubsection*{Setting up the calculation for general dimension}

In section \ref{sec:enzyme}, we reviewed two different coordinate systems in AdS$_{d+1}$. The AdS$_d$ foliation \reef{metric3a} was well suited to discuss the brane geometry, while the global coordinates are adapted to discuss the background geometry of the boundary CFT. However, our explicit calculations of the holographic EE are best performed in a new `cylindrical' coordinate system. In particular, following \cite{Krtous:2014pva}, we introduce cylindrical coordinates $P,\,\zeta$ where $\zeta$ specifies the position along the axis of the cylinder while $P$ measure the distance from the axis. These are related to the global coordinates in eq.~\reef{metric2s} by
\begin{align}\label{cylie}
\cosh r&=\sqrt{P^2+1}\,\cosh\zeta\,,\\
%\sinh r&=\sqrt{\frac{P^2+\tanh^2\zeta}{1-\tanh^2\zeta}}\,,\\
\tan\theta&=\frac{P}{\sqrt{1+P^2}}\,\frac{1}{\sinh\zeta}\,,
\end{align}
while the rest of the spherical angles remain unchanged. With this transformation, the metric becomes
\beq\label{cylindd}
ds^2=L^2\[ -(P^2+1)\cosh^2\zeta\, dt^2+\frac{dP^2}{1+P^2}+\left( 1+P^2 \right)d\zeta^2+P^2\,d\Omega_{d-2}^2\]\,.
\eeq
The range of these coordinates is $P\in (0,\infty)$ and $\zeta\in(-\infty,\infty)$. The conformal boundary is reached with $P\to \infty$ (or $\zeta\to\pm\infty$ with fixed $P$).  The upper ($0\le\theta\le\pi/2$) and lower ($\pi/2\le\theta\le\pi$) hemispheres are mapped to the upper ($\zeta\ge0$) and lower ($\zeta\le0$) halves of the cylindrical system. The conformal defect is positioned at $\zeta=0$. As noted above, the RT surfaces will be restricted to a constant time surface and hence the convenience of the cylindrical coordinates becomes evident, \ie  $\zeta$ becomes an extra Killing coordinate in the corresponding spatial geometry.


A few more technical details are needed  for our calculations:
in cylindrical coordinates \reef{cylindd}, the boundary entangling surface corresponds to the two circles $\zeta=\pm\zeb$, where
\beq\label{zeta0}
\sinh\zeb=\tan\thb\,,
\eeq
seen in the limit $P\to\infty$ of the second line in eq.~\reef{cylie}. Using the AdS foliation of eq.~\reef{metric3a}, the position of the brane was $z=\s$. Using eq.~\reef{eq:foobar}, the brane position can be specified in cylindrical coordinates \reef{cylindd} according to 
\beq\label{eq:foobar2}
 \(1+P^2\)\sinh^2\!\zeta=\frac{L^2}{\s^2}\(1-\frac{\s^2}{4L^2}\)^2\,.
\eeq
Recall that the brane intersects the asymptotic boundary at the position of the conformal defect, \ie at $\theta=\pi/2$ with $r\to\infty$, which corresponds to $\zeta= 0$ with $P\to \infty$ in cylindrical coordinates. Further recall that RT surface areas are UV divergent since they extend to the asymptotic boundaries.  Hence we introduced a UV regulator surface at $r=r_\mt{UV}$, which in cylindrical coordinates becomes
\beq\label{regular}
(P^2+\tanh^2\zeta)\cosh^2\zeta=\sinh^2\! r_\mt{UV}\,.
\eeq
We will be mainly interested in comparing the areas of different surfaces for fixed $\zeb$, as discussed above. Since the UV divergent terms only depend of the geometry of the entangling surface, they will cancel in the difference of the two areas. Hence, we can then safely take the UV cutoff to infinity.

As noted, the RT surfaces all lie in a fixed time slice and thus we only need consider configurations with cylindrical symmetry (\ie rotational symmetry on the $S^{d-2}$). Hence it is convenient to use the cylindrical coordinates \reef{cylindd} and  parametrize the profile of the bulk surfaces as $\zeta=\zeta(P)$. The bulk contribution to the holographic EE is given by
\begin{align}\label{area}
S_\mt{bulk}= \frac{L^{d-1}\, \Omega_{d-2}}{2\,\Gbk} \int dP P^{d-2} \sqrt{\frac{1}{1+P^2}+(1+P^2)\,\zeta'^2}
\end{align}
where again $\Omega_{d-2}$ is the area of the unit $(d-2)$-sphere -- see footnote \ref{footsphere}. As in eq.~\reef{area0}, an overall factor of 2 is included here to account for the reflection symmetry of the profile $\zeta(P)$ about the brane. Since this expression does not contain an explicit $\zeta$ dependence, it is straightforward to derive
\begin{align}\label{zetap}
\zeta'(P)=\pm \frac{1}{1+P^2}\,\sqrt{\frac{P_0^{2(d-2)}\(1+P_0^2\)}{P^{2(d-2)}\(1+P^2\)-P_0^{2(d-2)}\(1+P_0^2\)}}
\end{align}
where the two branches correspond to two identical surfaces related by a reflection with respect to $\zeta=0$. $P_0$ corresponds to the turning point, where the surface makes its closest approach to the symmetry axis.

We now discuss the disconnected phase described at the beginning of this section. It corresponds to the `trivial' solution with $P_0=0$. We find $\zeta(P)=\pm\zeb$, which in cylindrical coordinates looks simply as a pair of disks anchored at the boundary entangling surface. Substituting $\zeta'=0$ into eq.~\reef{area}, the area of the two discs can be integrated up to some cutoff radius $P_\mt{UV}$, and the corresponding holographic EE is
\begin{align}\label{A_disc}
S_\mt{disc}=\frac{L^{d-1}\, \Omega_{d-2}}{2(d-1)\,\Gbk} \ \puv^{d-1}\, {}_{2}F_1\left[ \frac{1}{2},\frac{d-1}{2},\frac{d+1}{2},-\puv^2 \right]\,.
\end{align}
In this case, the entanglement wedge corresponds to two identical disconnected pieces contained between each component of the RT surface and the asymptotic boundary, \ie the regions $\zeta\ge+\zeb$ and $\zeta\le-\zeb$, as sketched in the upper panel of figure \ref{fig:RTPhases}. 
%\dn{changed non-existent reference to figure \ref{fig:RTPhases}. Note that there are no a) / b) labels in the figure though. Do you just want to refer to \ref{fig:RTPhases}, or should we add a new figure to the beginning of section 4 which shows the two possible RT surfaces in the style of \ref{fig:cutoffs}?}
%
%\begin{figure}[h]
%\begin{center}
%\includegraphics[scale=.5]{images/EWs}
%\caption{Sketch of fixed time slices of our setup, showing the two possible configurations. The shaded region corresponds to the entanglement wedge.}
%\label{fig:EWs}
%\end{center}
%\end{figure}


\begin{figure}
	\def\svgwidth{0.8\linewidth}
	\centering{
		\input{RTPhases2.pdf_tex}
		\caption{Sketch of fixed time slices of our symmetric setup, showing the two possible configurations. The shaded red region corresponds to the entanglement wedge. The connected solution contains an island on the brane, where gravity is dynamical.}
		\label{fig:RTPhases}
	}
\end{figure}

The connected phase corresponds to $P_0>0$, which leads to a cylindrical RT surface. Integrating eq.~\reef{zetap} yields a family of bulk surfaces, which are symmetric about the brane and which are anchored on the asymptotic boundary at $\zeta=\pm\zeb$. Recalling the discussion below eq.~\reef{area0}, we observe that in this configuration, $P_0$ is the second integration constant which must be tuned in order to satisfy the appropriate boundary condition \reef{ortho1} at the brane, see the lower panel of figure \ref{fig:RTPhases}.

Before we calculate the entropy in the most general setting, let us consider the case of a zero-tension brane with $1/\Gbr=0$, \ie empty AdS$_{d+1}$. In this case, the brane is positioned at $\s=2L$ or simply, $\zeta=0$. Now, the `plus' branch of eq.~\eqref{zetap} can be integrated to produce a profile extending from $P=\puv$ at $\zeta=+\zeb$ to the maximal depth $P=P_0$ at some $\zeta=\zeta_0(\zeb,P_0)<\zeb$. Since eq.~\reef{ortho1} indicates that the RT surface must intersect the brane orthogonally, we must tune $P_0$ (with fixed $\zeb$) such that $\zeta_0=0$, \ie the RT surface reaches its maximal depth at the brane position. Now, substituting eq.~\reef{zetap} into eq.~\reef{area}, the holographic EE (for empty AdS$_{d+1}$) becomes
\begin{align}\label{A_conn}
S_\mt{conn}(T_o=0)
%&=2L^{d-3}S_{d-2} \int_{P_0}^P dp \frac{p^{d-2}}{\sqrt{1+p^2}} \sqrt{1+\frac{P_0^{2(d-2)}+P_0^{2(d-1)}}{p^{2(d-2)}+p^{2(d-1)}-P_0^{2(d-2)}-P_0^{2(d-1)}}}\\
&=\frac{L^{d-1}\, \Omega_{d-2}}{2\,\Gbk} \int_{P_0}^{\puv}\!\!\! dP\,  \frac{P^{2(d-2)}}{\sqrt{P^{2(d-2)}(1+P^2)-P_0^{2(d-2)}(1+P_0^2)}}\,.
\end{align}

In the general case, this exercise is slightly more complicated for the case of interest with a finite-tension DGP brane at some $z=\s\ll L$, and the geometry of the corresponding RT surface is illustrated in the lower panel of figure \ref{fig:RTPhases}. The RT surface is again symmetric about the brane and so as above, we focus on the portion starting at $\zeta=+\zeb$ at the asymptotic boundary (\ie at $P=\puv$). As before, the `plus' branch of eq.~\eqref{zetap} produces a surface reaching its maximal depth $P=P_0$ at some $\zeta=\zeta_0(\zeb,P_0)<\zeb$.\footnote{In fact, $\zeta_0(\zeb,P_0)$ is precisely the same function introduced above, since the turning point of the RT surfaces are completely independent of the brane properties.}  Now one continues from this point using the `minus' branch of eq.~\reef{zetap}, which then meets the brane as some $P=P_\mt{B}(\zeb,P_0)$ and $\zeta=\zeta_\mt{B}(\zeb,P_0)$.\footnote{Of course, $P_\mt{B}$ and $\zeta_\mt{B}$ are related as in eq.~\reef{eq:foobar2}.} One would again tune $P_0$ (for fixed $\zeb$) to ensure the appropriate boundary condition \reef{ortho1} is satisfied at the brane.
The bulk contribution to the holographic EE then becomes
\beqa
S_\mt{conn}(T_o>0)&=&\frac{L^{d-1}\, \Omega_{d-2}}{2\,\Gbk}\[ \int_{P_0}^{\puv}\!\!\!  dP\,  \frac{P^{2(d-2)}}{\sqrt{P^{2(d-2)}(1+P^2)-P_0^{2(d-2)}(1+P_0^2)}}\right.
\labell{Acon2}\\
&&\qquad\qquad\qquad+\left. \int_{P_0}^{\pb}\!\!  dP\,  \frac{P^{2(d-2)}}{\sqrt{P^{2(d-2)}(1+P^2)-P_0^{2(d-2)}(1+P_0^2)}}\]\,.
\nonumber
\eeqa
Of course, if there is no gravitational term on the brane (\eg as in eq.~\reef{newbran}), then this expression yields the entire generalized entropy \reef{eq:sgen_intro} for the connected phase. Now
rather than explicitly examining the brane boundary condition \reef{ortho1} in cylindrical coordinates, we will simply evaluate the generalized entropy and find the minimum numerically in the following. Hence to proceed further we will have to choose a specific value for the boundary dimension $d$.

\subsubsection*{Explicit results for $d=3$}
In this section, we consider the above discussion for $d=3$, in which case the boundary geometry becomes $\Rbb\times S^{2}$, the bulk spacetime is locally AdS$_4$, and the branes have an AdS$_3$ geometry. We will also consider supplementing the the four-dimensional bulk action \reef{act2} with a Gauss-Bonnet term, %\iar{I suggest including this term with an extra $L^2/\Gbk$, so that 1) it is of the same 'order' as the area term when $\lgb\sim 1$, and 2) we can remove this global prefactor in e.g. fig \ref{figdeltaA} }
 \beq
I_\mt{top} = \frac{\lgb}{16\pi^2}  \int \mathrm{d}^4x \, \sqrt{-g}\, \left[ R_{abcd}R^{abcd}-4\,R_{ab}R^{ab}+R^2 \right]\,.
 \labell{top2}
 \eeq
Note that we have ignored the necessary boundary terms which ensure that this interaction is proportional to the Euler density, \eg see \cite{Myers:1987yn}. Although this curvature-squared term does not effect the bulk equations of motion, it will contribute to the generalized entropy \cite{Dong:2013qoa,Hung:2011xb}\footnote{One may worry that the topological nature of $I_\mt{top}$ undercuts the usual derivations of the generalized entropy. However, individually the three terms in eq.~\reef{top2} are dynamical and one can apply the results of \cite{Dong:2013qoa} for each separately and then take the sum of the corresponding contributions to the holographic entropy, which one finds matches the result in eq.~\reef{Euler3}.}
 \beq
S_\mt{JM} = \frac{\lgb}{4\pi} \int_{\Sigma_\xR} d^2x\sqrt{h}\,\mR +\frac{\lgb}{2\pi} \int_{\partial \Sigma_\xR}
dx\sqrt{h}\,\mK_g \,,
 \labell{Euler3}
 \eeq
where $\mR$ denotes the Ricci scalar for the intrinsic geometry on the RT surface $\Sigma_\xR$. Similarly, $\mK_g$ denotes the geodesic curvature of the boundary $\partial \Sigma_\xR$. Of course, eq.~\reef{Euler3} gives a topological contribution proportional to the Euler character of the two-dimensional extremal surfaces\footnote{The normalization is chosen so that for an RT surface with two-sphere topology, $S_\mt{JM} = 2 \lgb$.} and so their geometry remains unaffected by this term. However, in the following, this additional contribution will  provide an extra parameter which allows us to adjust the transition between the connected and disconnected phases.

For $d=3$, some analytic expressions for the extremal surfaces can be obtained \cite{Krtous:2014pva}. For example,
integrating eq.~\eqref{zetap} yields the following profile for the extremal surface in empty AdS$_4$ \cite{Krtous:2014pva}
\beqa
&&\zeta_\pm(P;P_0,\zeta_0)=\zeta_0\pm \frac{P_0}{\sqrt{(1+P_0^2)(1+2P_0^2)}} \labell{zetasol} \\
&&\times\left[ (1+P_0^2)\, F\!\left( \mbox{Arcos} \frac{P_0}{P},\sqrt{\frac{1+P_0^2}{1+2P_0^2}} \right)-P_0^2\, \Pi\!\left( \mbox{Arccos}\frac{P_0}{P},\frac{1}{1+P_0^2},\sqrt{\frac{1+P_0^2}{1+2P_0^2}} \right) \right]
\nonumber
\eeqa
where $F$ and $\Pi$ correspond to incomplete elliptic integrals of the first and third kind, respectively.\footnote{Our notation for the elliptic integrals matches that in \cite{Gradshteyn:1702455}, section 8.1.} Again, the $\pm$ branches correspond to the two portions of the surface, symmetric with respect to $\zeta_0=0$. Of course, we need to know where this surface is anchored at the boundary. Hence we define
\beqa
\zeta_\infty&\equiv&\zeta_+(P\to \infty;P_0,\zeta_0)-\zeta_0
\labell{alphadog}\\
&=&\frac{P_0\left[ (1+P_0^2)\,K\!\left( \sqrt{ \frac{1+P_0^2}{1+2P_0^2} }\right) - P_0^2\, \Pi\!\left( \frac{1}{1+P_0^2},\sqrt{\frac{1+P_0^2}{1+2P_0^2}} \right) \right]}{\sqrt{(1+P_0^2)(1+2P_0^2)}}
\nonumber
\eeqa
and the surface reaches the asymptotic boundary at $\zeta_\pm(P\to \infty)=\zeta_0\pm \zeta_\infty$. Hence the two components of the entangling surface in the boundary theory are separated by $2\zeta_\infty$, in the cylindrical coordinates.

Figure \ref{figzetainfty} plots $\zeta_\infty$ as a function of $P_0$. The maximum is obtained at $P_0=P_0^{\mt{crit}}\approx 0.51633$ with $\zeta_\infty=\zeta_\infty^{\mt{crit}}\approx 0.5011$. An interesting observation in \cite{Krtous:2014pva} was that, for $P_0<P_0^{\mt{crit}}$, there exist \textit{two} values of $P_0$ with the same $\zeta_\infty$. That is, if the two components of the entangling surface are sufficiently `close' on the boundary sphere, there actually exist \textit{two} extremal RT surfaces that connect them in the bulk. However, one branch (with the smaller value of $P_0$) is always subdominant, and therefore will be of little interest in our analysis. On the other hand, if the separation of the two entangling spheres  is larger than the critical value $2\zeta_\infty^{\text{max}}$ (in cylindrical coordinates), there is no connected extremal surface that joins them.
%
%\begin{figure}[h]
%\begin{center}
%\includegraphics[scale=0.4]{images/zetainfty}
%\caption{Plot of the `height' of the RT surface in cylindrical coordinates, as a function of the turning point $P_0$ characterising the surface. For $\zeta_\infty<\zeta_\infty^{\mt{crit}}$, there are two minimal surfaces anchored at the same regions; otherwise there exists none. }
%\label{figzetainfty}
%\end{center}
%\end{figure}

\begin{figure}[h]
	\def\svgwidth{0.5\linewidth}
	\centering{
		\input{rtHeight.pdf_tex}
		\caption{Plot of the `height' of the RT surface in cylindrical coordinates, as a function of the turning point $P_0$ characterising the surface. For $\zeta_\infty<\zeta_\infty^{\mt{crit}}$, there are two minimal surfaces anchored at the same regions; otherwise there exists none. 
		}
		\label{figzetainfty}
	}
\end{figure}

Let us now describe the solutions corresponding to different values of the tension and DGP term:
%\dn{Should we mention somewhere around here that $T_o = 0$ is not einstein gravity on the brane?}\iar{Not needed here, since T_o=0 means the brane doesn't exist}.
\paragraph{a) $T_o=0;\ 1/\Gbr=0$\ :} First we consider the holographic EE in empty AdS$_4$ as a lead-in to the case with a brane. As emphasized above, the area of these surfaces is divergent, and so one introduces a UV regulator surface,  integrating of the area from $P_0$ to some $\puv\gg1$ \cite{Krtous:2014pva}. For the disconnected solution (\ie a pair of disks), eq.~\eqref{A_disc} with $d=3$ gives
\begin{align}\label{Sdisc}
S_\mt{disc}(\puv)
=& \frac{\pi L^2}{\Gbk}\,\left( \sqrt{1+\puv^2}-1 \right)+2 \lgb
\\
=& \frac{A(S^1_{\puv})}{4G_\mt{eff}}
- \frac{\pi L^2}{G_\bulk} + 2\lgb
+ \Ocal(\puv^{-1})\,.
\label{eq:lonely}
\end{align}
where
\beq\label{sample}
\frac{A(S^1_P)}{4G_\mt{eff}}=\frac{\pi L^2}{\Gbk}\,P\,,
\eeq
is the length of $S^1_P$, a circle with radius $P$, and we used eq.~\reef{Newton2} to write $\frac{1}{G_\mt{eff}}=\frac{2\,L}{\Gbk}$.
We have included in eq.~\eqref{Sdisc} the topological contribution in eq.~\reef{Euler3}. On the other hand, for the connected surfaces the area formula \eqref{A_conn} yields
\begin{align}\label{eq:dietCoke}
\begin{split}
\MoveEqLeft[3]
S_\mt{conn}(\puv,P_0)
\\
=& \frac{\pi L^2}{\Gbk}\,\frac{P_0^2}{\sqrt{1+2P_0^2}}\, \Pi\!\left( \mbox{Arccos}\frac{P_0}{\puv},1,\sqrt{\frac{1+P_0^2}{1+2P_0^2}} \right)
\end{split}
\\
\begin{split}
=& \frac{A(S^1_{\puv})}{4G_\mt{eff}}
+ \frac{\pi L^2}{G_\bulk}\left[
-\sqrt{1+2P_0^2} E\left(\sqrt{\frac{1+P_0^2}{1+2P_0^2}}\right)
+ \frac{P_0^2}{\sqrt{1+2P_0^2}} K\left(\sqrt{\frac{1+P_0^2}{1+2P_0^2}}\right)
\right]
\\
&+ \Ocal(\puv^{-1})\,,
\end{split}
\label{eq:friendless}
\end{align}
where $E$ is the elliptic integral of the second kind. We emphasize that this result only applies for  vanishing $T_o$ and vanishing $1/\Gbr$, \ie  for the AdS$_4$ vacuum. Note that the Euler character of the cylindrical RT surface is zero and hence there is no contribution proportional to $\lgb$. As expected, the divergence in the $\puv\to\infty$ limit matches for the areas of the connected and disconnected surfaces. Hence we can safely take the limit when considering the difference
\begin{align}\label{dAP01}
\Delta S(P_0)&=\lim_{\puv\to \infty}\left( S_\mt{conn}(\puv,P_0)-S_\mt{disc} (\puv)\right)\,,
\end{align}
given by the difference in $O\left( \left( P_0/\puv \right)^0 \right)$ terms in eq.~\eqref{eq:friendless} and eq.~\eqref{eq:lonely}.
A plot of $\Delta S$ is shown in figure \ref{figdeltaA}. When $\Delta S>0$, the disconnected RT surface is the dominant saddle, while for $\Delta S<0$, the connected solution dominates. Notice that with a larger (positive) topolgical coupling $\lgb$, the entropy in eq.~\reef{Sdisc} increases while eq.~\reef{eq:dietCoke} is unaffected, and hence the range of the disconnected phase is decreased in figure \ref{figdeltaA}.

%\footnote{Of course, $\Delta S(P_0)$ is closely related to the mutual information. \rcm{words:} Mutual information between two subsystems $A$ and $B$ is defined via $I=S_{A\cup B}-S_A-S_B$. Notice that whenever $\Delta S>0$, phase $I$ dominates and therefore the mutual information vanishes, since $S_{A\cup B}=S_A+S_B$.}
%
%\begin{figure}[h]
%\begin{center}
%\includegraphics[scale=0.3]{images/dAP0}
%\caption{Renormalised entropy from eq. \eqref{dAP01}. The connected (disconnected) surface dominates when $\Delta S<0\,(\Delta S>0)$. When $\lgb$ to becomes very large, $\lgb\sim c_T$, the connected solution becomes favoured. }
%\label{figdeltaA}
%\end{center}
%\end{figure}

\begin{figure}[th]
	\def\svgwidth{0.8\linewidth}
	\centering{
		\input{figdeltaA2.pdf_tex}
		\caption{Renormalised entropy from eq. \eqref{dAP01}. The connected (disconnected) surface dominates when $\Delta S<0\,(\Delta S>0)$. When $\lgb$ becomes very large, $\lgb\sim c_T$, the connected solution becomes favoured.}
		\label{figdeltaA}
	}
\end{figure}

\paragraph{b) $T_o\ne 0;\ 1/\Gbr=0$\ :} The next step is to introduce the brane, however, we do not include a gravitational term in the brane action yet, \ie $1/\Gbr=0$. In this case, we saw  in eq.~\reef{Acon2} that there is an additional contribution as the RT surface extends from the maximal depth $P_0$ back out to meet the brane at $P_\mt{B}$. Both contributions in eq.~\reef{Acon2} take the same form except for the limits of integration, hence the $d=3$ result in eq.~\reef{eq:dietCoke} is replaced by
\beqa
S_\mt{conn}(\puv,P_0)&=& \frac{\pi L^2}{\Gbk}\,\frac{P_0^2}{\sqrt{1+2P_0^2}}\,\[ \Pi\!\left( \mbox{Arccos}\frac{P_0}{\puv},1,\sqrt{\frac{1+P_0^2}{1+2P_0^2}} \right)\right.
\labell{CokeZero}\\
&& \left.\qquad\qquad+\Pi\!\left( \mbox{Arccos}\frac{P_0}{\pb},1,\sqrt{\frac{1+P_0^2}{1+2P_0^2}} \right)\]\,.
\nonumber
\eeqa


Of course, the entropy for the disconnected phase remains the same as in eq.~\reef{Sdisc} and we can consider the difference of the generalized entropy evaluated on the connected and disconnected extremal surfaces, as in eq.~\reef{dAP01}. Just as we saw a leading divergent contribution in eq.~\reef{eq:dietCoke} for $\puv\to\infty$, we expect that eq.~\reef{CokeZero} will contain an analogous large contribution for $\pb\gg P_0$. However, this term will not be cancelled in $\Delta S$. In fact, in this regime, we can expand the difference as
\begin{align}\label{radishes}
\begin{split}
\MoveEqLeft[2]\Delta S(P_0)
\\
=& \frac{A(\sigma_\xR)}{4G_\mt{eff}}
+ \frac{\pi L^2}{G_\bulk}\left[
1-2\sqrt{1+2P_0^2} E\left(\sqrt{\frac{1+P_0^2}{1+2P_0^2}}\right)
+ \frac{2P_0^2}{\sqrt{1+2P_0^2}} K\left(\sqrt{\frac{1+P_0^2}{1+2P_0^2}}\right)
\right]
\\
&-2 \lgb
+ \Ocal(\pb^{-1})\,.
\end{split}
\end{align}
Here, the intersection $\sigma_\xR$ of the RT surface and the brane is a circle of radius $\pb$ with area $A(\sigma_\xR)=2\pi L\,\pb$ given by eq.~\eqref{cylindd}. The fact that the leading term can be expressed as the gravitational entropy for the induced gravity action \reef{act3} on the brane is in perfect agreement with our discussion in the previous section. As we will see below, the finite terms will play a role once we turn on the DGP term, allowing for the appearance of a different island on the brane. 

From the above expansion, we see that there is a strong penalty for having a large $\sigma_\xR$ in the connected phase. From the brane perspective, the gravitational entropy results in a large penalty against forming an island on the brane. In fact, generally we expect that $\Delta S>0$ in this regime and hence the disconnected solution provides the dominant saddle point. However, if we tune the topological coupling $\lgb$ to be large\footnote{We note that this requires $\lgb\sim {L^2}/{\Gbk}\sim \cT$, the central charge of the boundary CFT -- see further discussion in section \reef{sec:discussion}.} (and positive), this contribution can compensate for the leading gravitational entropy term, at least for $\sigma_\xR$ up to a certain size.

On the other hand, we must note that $\pb$  is not an independent parameter. Rather it is implicitly determined by $\zeb$ and the brane tension $T_o$, as well as the value of $P_0$ that minimises the area functional in eq. \eqref{radishes}. $\pb$ can be determined in the following way (see figure \ref{fig:RTPhases}). One begins by solving for $\zeta_0$ using
$\zeta_0+\zeta_\infty(P_0)=\zeb$ where $\zeta_\infty(P_0)$ is given in eq.~\reef{alphadog}. Then one finds `sample' values of $\pb,\zeta_\mt{B}$ where the extremal surface meets the brane by combining eqs.~\reef{eq:foobar2} and \reef{zetasol} and simultaneously solving
\beqa
 \(1+\pb^2\)\sinh^2\!\zeta_\mt{B}&=&\frac{L^2}{\s^2}\(1-\frac{\s^2}{4L^2}\)^2\,,\nonumber\\
 \zeta_-(\pb;P_0,\zeta_0)&=&\zeta_\mt{B}\,.
 \label{solver3}
\eeqa
This yields $P_B$ as a function of $P_0,\zeta_{\mt{CFT}}$ and $T_o$, and substituting $\pb$ into eq.~\reef{CokeZero} gives the area of the associated extremal surface. Below, we perform this calculation numerically. However, we have not yet considered the boundary conditions \reef{ortho1} in this analysis. Rather than explicitly examining the latter, we simply evaluate the area (or rather the difference $\Delta S$) over the range of possible $P_0$ (with fixed $\zeb,T_o$), as shown in figure \ref{fig:dS0}a. The correct RT surfaces are then identified as the minima in these plots. Further, the examples in the figure illustrate that without the topological contribution, $\Delta S>0$ for all minima and so the disconnected phase dominates, as generally expected. That is, no quantum extremal islands form on the brane in this case. However, as shown in figure \ref{fig:dS0}b, we see that with a sufficiently large topological coupling $\lgb$ one can achieve $\Delta S<0$, where a first order transition leads to the formation of an island. 

Although the above recipe is valid for arbitrary brane tensions, in the limit of very large tension we can approximate the solution analytically. Since, as stated above, the leading contribution to the entropy \eqref{sample} scales as $A(\sigma_\xR)\sim \pb$, the RT surface corresponds to that which has the minimal value of $\pb$. Moreover, since the function $\zeta_{\mt{B}}(P)$ defining embedding of the brane in \eqref{solver3} is monotonically decreasing with $P$, the surface must maximise its hight $\zeta_\infty(P_0)$, which is achieved for $P_0=P_0^{\mt{crit}}$, by definition (see discussion around figure \ref{figzetainfty}). This can be readily checked in figure \ref{fig:dS0}a, where the curves attain a minimum around $\mbox{arctan}(P_0^{\mt{crit}})\approx 0.47$, with a small correction due to the finite terms in \eqref{radishes}, which becomes smaller and smaller as we increase the tension. We shall refer to this solution with $P_0\approx P_0^{\mt{crit}}$ as the \textit{small island}, in order to distinguish it from a second island appearing below which corresponds to a circle with a larger radius.  


%\begin{figure}[h]
%\begin{center}
%\includegraphics[scale=0.3]{images/setup}
%\caption{A half of the geometric setup in cylindrical coordinates, with the radial direction compactified to a finite coordinate value. The `tube' (orange) corresponds to the connected RT surface $\sigma$, anchored at conformal infinity and intersecting the brane (green) at the codim-$3$ surface $\del\sigma_{brane}$ (red). The disk (blue) corresponds to the trivial disconnected phase. A second copy of this is glued along the brane, as depicted in figure \ref{fig:brane2}.  }
%\label{figsetup}
%\end{center}
%\end{figure}

%The total renormalised area of the connected phase is given in eq. \eqref{Acon2}. Notice that the intersection point $\pb=\pb(\s,\zeb,P_0)$ depends on the location of the brane via $\s$, together with the anchoring hight $\zeb$ and the turning point $P_0$ the surface. To find the entanglement entropy, we fix $\s$ and $\zeb$, and vary $P_0$ in order to find the minimum of the area
%\begin{align}\label{dAtilde}
%\Delta \tilde A=S_{\mt{conn}}(T_0> 0)-S_{\mt{disc}}
%\end{align}
%where the first and second term are given by eqs. \eqref{Acon2} and \eqref{Sdisc} respectively. We perform this computation numerically. In figure \ref{figClassAreas} where we plot the area of minimal surfaces that intersect the brane, for fixed anchoring point $\zeb$ as we vary the turning point $P_0$, for different values of $\s$, the parameter controlling the tension of the brane. For large tensions $L/\s\gg 1$, it can be checked that the minimum is always attained for $P_0=\bar P_0\approx 0.44$.


%\begin{figure}
%\begin{center}
%  \begin{subfigure}[b]{0.4\textwidth}
%    \includegraphics[width=\textwidth]{images/ClassAreas}
%    \caption{}
%  \end{subfigure}
%  %
%  \begin{subfigure}[b]{0.4\textwidth}
%    \includegraphics[width=\textwidth]{images/GB_u}
%    \caption{}
%  \end{subfigure}
%  \end{center}
%  \caption{a) Renormalised area from eq. \eqref{radishes} of connected RT surfaces, anchored at $\zeb=0$, with $\lgb=0$. b) Critical value of $\lgb$ such that $\mbox{min} \Delta S<0$.}
%  \label{fig:dS}
%\end{figure}

\begin{figure}[h]
	\def\svgwidth{0.9\linewidth}
	\centering{
		\input{dS2.pdf_tex}
		\caption{Panel a. illustrates the renormalised area from eq. \eqref{radishes} of connected RT surfaces, anchored at $\zeb=0$, with $\lgb=0$. Panel b. is a plot of the critical value of $\lgb$ such that $\mbox{min}( \Delta S)<0$.
		}
		\label{fig:dS0}
	}
\end{figure}

\paragraph{c) $T_o\ne 0;\ 1/\Gbr\ne 0$\ :} Finally, we examine the holographic EE in the presence of a DGP brane. The only difference in this analysis is the additional contribution coming at the intersection of the RT surface with the brane in eq.~\reef{eq:sad}.
In the present setting, this means that we add the following,
\beq\label{sample2}
S_\mt{brane}=\frac{A(\sigma_\xR)}{4\Gbr}=\frac{\pi L}{2\Gbr}\,\pb\,,%\frac{\pi L^2}{\Gbr}\,\pb\,,
\eeq
to the bulk contribution in eq.~\reef{CokeZero}. In fact, the  expansion of $\Delta S$ for $\pb\gg P_0$ takes precisely the same form as in eq.~\reef{radishes}. The only difference is that the induced Newton's constant on the brane is now given by eq.~\reef{Newton33}, \ie $\frac{1}{G_\mt{eff}}=\frac{2\,L}{\Gbk}+\frac{1}{\Gbr}$.

Generally, we might think of $1/\Gbr$ as a positive quantity, and so the DGP contribution \reef{sample2} would simply increase the penalty for having a large $\sigma_\xR$ in the connected phase, and enhance the dominance of the disconnected phase. However, there is no apriori reason why we should not also consider a negative gravitational coupling on the brane,\footnote{For example, integrating out quantum fields on the brane could produce either a positive or negative shift in Newton's constant. In particular, it can be negative for gauge fields or nonminimally coupled scalar fields, as discussed in the context of EE in \cite{Larsen:1995ax,Kabat:1995eq} -- see further discussion in section \ref{sec:discussion} and appendix \ref{bubble}.} in which case the DGP term serves as another mechanism to reduce the penalty for forming an island on the brane. It is this scenario that we will examine further here -- as well as in appendix \ref{bubble}.

It will prove convenient to work with the ratio $\lamb$ introduced in eq.~\reef{newdefs}. Let us recall what parameters are in play. The tension of the brane is controlled by $\s$, which we keep small but finite. The dimensionless ratio between the bulk and brane gravitational constants is controlled by $\lamb$. As discussed above, interesting things happen when $\lamb<0$, which is when $\Gbr<0$ while $\Gbk>0$.

\begin{figure}[h]
	\def\svgwidth{\linewidth}
	\centering{
		\input{IslandPlot.pdf_tex}
		\caption{ Panel a.: Generalised (renormalised) area from eq. \eqref{radishes} as function of $P_0$, for different values of the DGP coupling $\lamb$. Notice the appearance of a `large' island when $\lamb$ approaches $-1$, due to the partial cancellation of the induced and DGP area terms. Panel b.: Phase diagram: the black lines correspond to first order phase transitions, while the blue one at $\lamb=-1$ indicates the region where gravity becomes unstable. Both plots are done for fixed $L/\s=100$. 
		}
		\label{fig:dS}
	}
\end{figure}
%\begin{figure}
%\begin{center}
%  \begin{subfigure}[b]{0.5\textwidth}
%    \includegraphics[width=\textwidth]{images/A_gen}
%    \caption{}
%  \end{subfigure}
%  %
%  \begin{subfigure}[b]{0.4\textwidth}
%    \includegraphics[width=\textwidth]{images/GB_B}
%    \caption{}
%  \end{subfigure}
%  \end{center}
%  \caption{ a) Generalised (renormalised) area from eq. \eqref{radishes} as function of $P_0$, for different values of the DGP coupling $\lamb$. Notice the appearance of a `large' island when $\lamb$ approaches $-1$, due to the partial cancellation of the induced and DGP area terms. b) Phase diagram: the black lines correspond to first order phase transitions, while the blue one at $\lamb=-1$ indicates the region where gravity becomes unstable. Both plots are done for fixed $L/\s=100$. }
%  \label{fig:dS}
%\end{figure}
Using the same approach described above, we can explore the transition between the connected and disconnected phases numerically. In figure \ref{fig:dS}a, we plot $\Delta S$ as function of $P_0$ for a fixed $\zeb=0.095,L/\s=100$ and $\lgb=0$, for different values of $\lamb$. These plots are analogous to those presented in figure \ref{fig:dS}a where $\lamb=0$ (but $L/\s$ is varied). Again, these plots are made in lieu of a detailed examination of the boundary conditions where the RT surfaces meet the brane, rather the correct boundary conditions \reef{ortho1} will be achieved where $P_0$ is tuned to produced an minimum in these plots. For small $\lamb$ the curves show a single minimum but $\Delta S>0$, indicating that the disconnected solution dominates in this case. As $\lamb$ becomes more negative, the curves are pulled down and eventually $\Delta S$ enters the negative region so that the connected solution becomes the dominant saddle point. This behaviour is as expected but we note that $\lamb$ is very close to $-1$ in this regime, which according to eq.~\reef{newdefs} means there is almost a complete cancelation between the induced gravitation coupling $1/G_\mt{RS}$ and the DGP term $1/\Gbr$. Of course, this near cancellation is alleviated by turning on the topological coupling $\lgb$, as shown in figure \ref{fig:dS}b. 

Another interesting feature shown in figure \ref{fig:dS}a is the appearance of a second minimum in the curves. This second solution occurs at a larger value of $P_0$ and also of $P_B$, and corresponds to a larger circle $\sigma_\xR$ on the brane, and therefore we refer to it as a \textit{large island}. The existence of this second island is due to the finite terms in \eqref{radishes}. Indeed, these terms are essentially what is plotted in figure \ref{figdeltaA}, and they are unbounded from below for large $P_0$. Therefore, when $\lamb$ becomes sufficiently negative as to produce a significant cancellation between the induced and DGP gravitational entropies, there is a new competition, now between $A(\sigma_\xR)/4G_\mt{eff}$ and the finite terms, producing the large island. As $\lamb\to-1$, the minimum rolls down to infinity ($P\to \infty,\Delta S_{\mt{gen}}\to-\infty$), indicating an instability at this point, which we explore further in appendix \ref{bubble}.

Figure \ref{fig:dS}b summarises the phase diagram of the system, for a fixed value of the tension $L/\s=100$, as we vary both the DGP coupling $\lamb$ and the topological coupling $\lambda_{\mt{GB}}$. The lines between no/small/large islands correspond to first order phase transitions, while the blue line at $\lamb$ indicates the region where the theory becomes unstable. 





\section{Discussion}\label{sec:discussion}
% !TEX root = ../lifeonbrane3.tex
%

We have described a holographic framework where quantum extremal surfaces and the island rule \reef{wonderA} can be examined in higher dimensions, \ie for gravity theories in $d\ge2$. In particular, the background is simple enough that the construction given in section \ref{sec:branegravity} is straightforward and purely analytic, in contrast to the numerical approach of \cite{Almheiri:2019psy}. In section \ref{face}, we were also able to describe the system from three different perspectives, analogous to the three descriptions of the two-dimensional system examined in \cite{Almheiri:2019hni}. In particular, we have the boundary perspective, where the system is described as a $d$-dimensional CFT coupled to a ($d-1$)-dimensional conformal defect;
the bulk gravity perspective, where ($d+1$)-dimensional gravity with a negative cosmological constant is coupled to a codimension-one brane; and the brane perspective, where the boundary CFT is coupled to an AdS$_d$ region which supports Einstein gravity and two copies of the same CFT, which are weakly coupled to each other. As we emphasized, this last perspective is an effective theory, as is made clear by the cut-off arising in this Randall-Sundrum braneworld scenario. As discussed and examined in some detail in section \ref{HEE}, this effective gravity theory lends itself to the appearance of quantum extremal islands in the brane perspective, although these have a conventional interpretation from the bulk gravity perspective, in terms of RT surfaces which cross the brane for certain of choices of the entangling geometry on the boundary.\\

\hd{Unconventional features:} Of course, the analysis presented in our paper is somewhat unusual in that we are finding quantum extremal islands but there are no black holes, no horizons and no Hawking radiation involved. Rather we simply considered the entanglement entropy of various entangling regions in the vacuum state of the boundary system. However, to favour the formation of these quantum extremal islands, and at the same time have the brane in the `Einstein gravity regime,' \ie $L/\leff\ll1$, we had to introduce somewhat unconventional couplings. That is, we considered a negative Newton's constant on the brane $\lamb<0$ and nonzero Gauss-Bonnet coupling $\lgb$ for a four-dimensional bulk. Both of these choices were enhancing the connected RT surfaces over the disconnected RT surfaces in calculating the holographic EE. Of course, an interesting question is the interpretation of these `exotic' bulk couplings in terms of data describing the boundary CFT (and the conformal defect). While we do not have a precise interpretation, some qualitative results can be stated.

As observed in section \ref{face}, using standard holographic techniques, one finds that the gravitational coupling in the DGP brane action \reef{newbran} affects the spectrum of defect operators in the boundary theory \cite{domino}. Now let us reiterate that there is no apriori reason not to consider $\lamb<0$. For example, integrating out quantum fields on the brane could produce either a positive or negative shift of Newton's constant. In particular, the shift can be negative for gauge fields or nonminimally coupled scalar fields, as was discussed in the context of EE in \cite{Larsen:1995ax,Kabat:1995eq} -- see also discussion is appendix \ref{bubble}. However, this scenario is not the one we are describing here. In particular, additional brane fields such as these would make significant contributions to the EE which are not accounted for in our calculations. Hence, implicitly, we simply assume that the gravitational coupling $1/\Gbr$ (either positive or negative) is induced by some unknown UV physics.

Introducing the Gauss-Bonnet term \reef{top2} does not modify the gravitational dynamics in the four-dimensional bulk, considered in section \ref{sec:examples}, and hence the correlators of the stress tensor are not modified in the dual three-dimensional boundary theory.\footnote{Of course, such modifications arise for holographic constructions in higher dimensions \cite{Buchel:2009sk}.} However, the topological coupling $\lgb$ affects the entanglement structure of the boundary CFT states. To see this, consider calculating the entanglement entropy holographically for two nearby regions in the boundary. The phase transition between connected and disconnected phase of the RT surfaces is sensitive to a Gauss-Bonnet term. For positive $\lgb$, the transition from disconnected to connected phase takes place earlier (and vice versa for negative $\lgb$). This means that with $\lgb>0$, the mutual information between these two regions remains of order $c_\mt{T}$ for larger separations, \eg \cite{Headrick:2010zt}. Note, however, that choosing positive $\lgb$ favours higher genus surfaces. A concern with this choice might be if higher genus extremal surfaces exist, they may produce unusual results. Finally, we note that the topological coupling appears directly in the expressions for the holographic EE, \eg see eq.~\reef{Sdisc}. Therefore to have an appreciable effect, we must choose this coupling to be of the order of the central charge of the boundary theory, \ie $\lgb\sim L^2/\Gbk\sim \cT$.

Let us add that in section \ref{sec:examples}, we focused on the example of $d=3$ with a four-dimensional bulk. In this case, the natural topological term to add to the bulk gravity is the Gauss-Bonnet term \reef{top2}. Of course, the scenario extends straightforwardly to any $d=2n-1$ for which there is a corresponding topological term which can be added to the bulk gravity action, \ie the Euler character for $2n$-dimensional manifolds, \eg see \cite{Hung:2011xb}. Similarly, for even boundary dimensions ($d=2n$), the analogous topological terms could be added to the brane action, where they would not modify the dynamics of gravity on the brane but they would modify the gravitational entropy associated with the boundary of the quantum extremal islands. 

In light of these unconventional features, a natural question therefore is whether we find quantum extremal islands in our analysis with both $\lamb =0= \lgb$. The answer is affirmative, however, one must reduce to the tension of the brane to reduce its backreaction and the extent of the additional geometry in the vicinity of the  brane's location. As a result, the connected RT surfaces will have a smaller (bulk) area contribution as they cross the brane. However, in this case, the curvature of the AdS geometry on the brane is also smaller, and hence the effective description of the brane theory in terms of Einstein gravity breaks down. That is, with $\leff\sim L$, the contributions of the higher curvature corrections in the induced action \reef{act3} are no longer suppressed relative to the Einstein term and these new interactions play an important role in the dynamics of gravity in the brane perspective. Furthermore, the cutoff of the corresponding CFT on the brane will be much lower. Alternatively, one could think about computing the EE in settings beyond the vacuum state that we studied here. In fact, in \cite{QEI}, we will explicitly show without additional Gauss-Bonnet or DGP couplings that quantum extremal islands appear for (nonextremal) eternal black holes in equilibrium with an external heat bath, \ie in a higher dimensional analog of the analysis in \cite{Almheiri:2019yqk}.

Let us conclude here by comparing our approach with the recent work \cite{Geng:2020qvw}, which appeared while the present paper was prepared for submission. The latter examines essentially the same model (with no DGP term) but concentrates on a very different regime. The authors of \cite{Geng:2020qvw} focused on the formation of islands for the case of a tensionless brane, where the brane gravity becomes very nonstandard, as explained above. Further, in the limit where the graviton becomes massless, \ie $\ell_\mt{eff}\to \infty$, they  observe that no islands form \cite{Geng:2020qvw}. On the other hand, the present work focuses the regime of large brane tension, where the theory on the brane can be well approximated by Einstein gravity (\ie the graviton mass and higher curvature interactions are negligible). We moreover show that by allowing either a topological term or a negative $\Gbr$, islands can appear even in the absence of horizons.\\ 

\hd{Resolving Puzzles:} Our construction clarifies certain conceptual puzzles that arose in early discussions of quantum extremal islands in a holographic framework, \eg for the two-dimensional gravity models introduced in \cite{Almheiri:2019hni} and studied in \cite{Almheiri:2019yqk, Chen:2019uhq}. For example in these models the Planck brane, which supports the JT gravity theory, appears at the boundary of the three-dimensional bulk spacetime. Hence one might have wondered if the brane degrees of freedom (including the JT gravity) are a part of the boundary theory or part of the bulk theory. In our construction, the Planck brane is in the middle of the spacetime geometry and so this question does not arise -- these degrees of freedom belong to the bulk. An important corrolary of this observation is that when a quantum extremal island appears on the brane, \eg see the lower panel in figure \ref{fig:RTPhases}, we are able to recover information about the island with data from the boundary CFT in the corresponding boundary subregion, by applying standard entanglement wedge reconstruction \cite{EW1,EW2,EW3,Jafferis:2015del,Dong:2016eik,Faulkner:2017vdd,Cotler:2017erl}. Of course, the latter would not apply if the brane degrees of freedom were a part of the boundary theory.

Further, our construction circumvents the question of whether RT surfaces are allowed to end on the Planck brane. Rather in our paper, the extremal surfaces just pass through the bulk and only end on the asymptotic boundary as usual. It is simply that in certain situations, the RT surfaces will pass through the brane, which of course, corresponds to the formation of a quantum extremal island.

Another `novel' feature of the two-dimensional JT gravity model of \cite{Almheiri:2019hni} was that the holographic entanglement entropy included an extra boundary term, \ie the gravitational entropy of the JT model, where the RT surface terminated on the Planck brane. That is, the holographic entanglement entropy was given by extremizing the sum of the bulk area of the RT surface and this additional boundary term. An analogous gravitational entropy term on the brane arises in our construction with a DGP brane -- see eq.~\reef{eq:sad}. In fact, our derivation in appendix \ref{generalE} suggests that if the brane supports intrinsic gravitational interactions then the corresponding Wald-Dong entropy on the brane is part of the holographic entanglement entropy formula, as shown in eq.~\reef{fish9}. Hence this general result agrees with the boundary term introduced in the two-dimensional JT gravity models, mentioned above. A shortcoming of the derivation in appendix \ref{generalE} is that the geometric configuration involved a high degree of symmetry, which precluded  finding the expected extrinsic curvature terms \cite{Dong:2013qoa}. Therefore it would be interesting to extend our construction there to more general configurations  along the lines of \cite{Lewkowycz:2013nqa,Dong:2016hjy}.

We want to emphasize the above discussion is distinct from finding in section \ref{sec:enzyme} that the leading contribution to the holographic EE where the RT surface crosses the brane matches the Wald-Dong entropy of the induced gravitational action on the brane\reef{act3}.\footnote{Recall that this analysis was general enough to see the extrinsic curvature contributions coming from the higher curvature interactions in eq.~\reef{act3}.} For example, the leading contribution is $\area(\sigma_\xR)/{4G_\mt{eff}}$, where $\sigma_\xR$ is the cross-section of the RT surface on the brane. As shown in eq.~\reef{eq:bazinga2}, the DGP term is one important contribution to this result, but the bulk area of the RT surface in the vicinty of the brane is also necessary. Of course, we still find the leading contributions reproduce the gravitational entropy of the induced gravity theory on the brane even without the DGP term, \ie with $1/\Gbr=0$. This must be closely related to the fact that the bulk Einstein equations combined with the Israel junction conditions are equivalent to the gravity equations of motion on the brane in the Randall-Sundrum scenario \cite{deHaro:2000wj}.

In passing we note here that $d=2$ is distinguished in the above discussion. In this case, the leading contribution corresponds to the Wald-Dong entropy for the the Polyakov-Liouville action \eqref{PolyAct2} and takes the form given in eq.~\reef{arc}. However, since it only depends on the curvature scalar which is constant across the AdS$_2$ geometry of the brane, this contribution takes the same value no matter where the RT surface  crosses the brane. This contrasts with the higher dimensional result $\area(\sigma_\xR)/{4G_\mt{eff}}$, which rapidly grows as the position of $\sigma_\xR$ moves to larger radii on the brane. That is, there is an enormous penalty against forming large quantum extremal islands for $d\ge3$. In contrast, no such penalty arises for $d=2$ facilitating the formation of islands, as discussed in detail in
\cite{Rozali:2019day}. Of course, if one adds JT gravity \reef{JTee} to the two-dimensional brane action, as in eq.~\reef{braneact2}, then the gravitational entropy on the brane includes $\(\Phi_0+
\Phi(x)\)/4\Gbr$, which will favour smaller quantum extremal islands because the dilaton profile grows with the radius on the brane \cite{Maldacena:2016upp}.

Of course, we can modify our higher dimensional construction to make it more analogous to the two-dimensional model introduced in \cite{Almheiri:2019hni} by taking a $\mathbb Z_2$ orbifold quotient across the brane. With this orbifold, the brane appears as the edge of the bulk geometry but clearly the association with the bulk degrees of freedom has not changed. The brane now only supports a a single copy of the boundary CFT and there are factors of 1/2 appearing in various expressions, \eg we make the following replacement in eq.~\reef{Newton2}: ${1}/{G_\mt{eff}}=L/((d-2)\Gbk)$. Similarly, the RT surfaces will now end on the orbifolded brane while satisfying the boundary condition,
\beq\label{ortho8}
 0  =  \tg_j{}^\nu\(g_{\mu\nu}\,\partial_{n}X^\mu
  + \frac{G_\bulk}{G_\brane}\,\inducedK_i \,\partial_\nu x^i\)\,,
\eeq
%\rcm{Vincent: please confirm}\vc{I'm not sure about the factor of 2: if there are really two identical bulks, then \eqref{ortho8} is just a special case of \eqref{ortho7} and the factor of 2 is correct; but, if there is only one bulk (as suggested by stripping off a factor of $2$ in \eqref{Newton2} to get ${1}/{G_\mt{eff}}=L/((d-2)\Gbk)$ mentioned above), then the $\partial_{n_L} X^\mu$ term in \eqref{ortho7} is just not present, so there should not be a factor of 2 in \eqref{ortho8}.}
which replaces eq.~\reef{ortho7}. Further, the conformal defect becomes a conformal boundary in the orbifolded theory, \ie the spatial geometry on which the CFT lives is now a ($d-1$)-dimensional hemisphere with the conformal boundary being the $S^{d-2}$ at the edge of the hemisphere. 

Other questions that may have arisen from the early discussions of quantum extremal islands which focussed on JT gravity might include the importance of having a low spacetime dimension, \ie $d=2$, or of the JT model itself. The early work of \cite{Penington:2019npb} considered black hole evaporation with Einstein gravity in higher dimensions, and the holographic model of \cite{Almheiri:2019hni} was extended to a holographic framework with $d=4$ in \cite{Almheiri:2019psy} using numerical calculations. Hence our paper reinforces these results by describing quantum extremal islands in a new setting, in particular, in higher dimensions and with Einstein gravity. Our construction is also simple enough that further investigations of the role of quantum extremal islands in higher dimensions are straightforward, \eg see \cite{QEI}. Let us add that JT gravity can be seen as the gravitational dual of the so-called SYK model \cite{Maldacena:2016hyu,Sachdev:1992fk,Sachdev:2010um,Ktalks}. This duality involves an ensemble average over the couplings in the boundary quantum mechanics and so one may expect that this averaging plays a role in the appearance of quantum extremal islands. However, it seems that this is not the case as our construction relies on the standard holographic rules of the AdS/CFT correspondence where there is no such averaging of the couplings in the boundary theory.

One other perplexing issue with the island rule \reef{rule1} is the appearance of the entanglement of the CFT degrees of freedom in the region $\CFTR$ on both sides of the equation \cite{Almheiri:2019hni}. As explained in \cite{Almheiri:2019yqk}, we should distinguish the ``full quantum description'' of, \eg the Hawking radiation in the presence of black holes on the left-hand side from the ``semiclassical description'' which includes the outgoing radiation and purifying partners on the quantum extremal island on the right-hand side. Our holographic construction makes clear that the description of quantum states with islands in the brane picture is on a different footing than that solely in terms of the boundary theory. In particular, referring to the three perspectives discussed in section \ref{face}, it is clear that the boundary perspective (with the boundary CFT coupled to a conformal defect) gives a complete description of quantum state.  By the standard rules of the AdS/CFT correspondence, the bulk perspective (where Einstein gravity with a negative cosmological constant is coupled to a codimension-one brane) gives an equivalent description.\footnote{In this paper, we modeled the CFT defect with a simple brane in the bulk. This bottom-up approach is neither sufficient, nor completely correct. For example, in the case of $\mathcal N=4$ SYM theory on $S^4$, the presence of an interface breaks at least half of the supersymmetry generators and the $R$ symmetry. In a complete description, this will result in a deformation of the bulk $S^5$. For top-down models, see \cite{Karch:2001cw,DeWolfe:2001pq, DHoker:2007hhe, DHoker:2008rje, Chiodaroli:2009yw, Chiodaroli:2011nr, Chiodaroli:2012vc}. }
However, the brane perspective has a different character. In particular, the description in terms of a CFT coupled to the dynamical AdS$_d$ region is only an effective one. Indeed, as emphasized in section \ref{face}, the Randall-Sundrum gravity is only valid down to  the short distance cutoff $\tilde\delta\sim L$, \ie see eqs.~\reef{ctoffplus} and \reef{ctoffminus}. Beyond this cutoff, gravity is no longer localized to the brane and the additional `Kaluza-Klein' modes of the graviton are strongly coupled to the brane and their contribution cannot be ignored. 

Further, this brane perspective also provides an effective description of the coupling to the defect CFT. That is, it only accounts for the couplings localized at the defect, which dominate at low energies, but ignores the subtle nonlocal couplings, which could be seen as coming through the bulk AdS geometry in the dual description. Of course, the quantum extremal islands in the effective description of the brane perspective are a clear example of this. These islands are a remnant of replica wormholes in the limit $n\to1$ \cite{Penington:2019kki,Hartman:2020swn}. However, in the replica trick construction of the corresponding Renyi entropies in the bath CFT, one can ask why the gravity on the different branes in the replica copies should connect with one another. However, these effective gravity theories are UV completed by a single theory of gravity in the bulk and so it is natural to consider geometries connecting the branes, \ie replica wormholes if the effective theory. Hence the connection of the brane and boundary through the bulk provides a simple explanation of these wormholes.  Given the simplicity of our construction, it may provide a useful framework in which to understand further subtleties in distinguishing the various expressions in the island rule.

As a final note here, we observe that the finite cutoff for the CFT on the brane has noticeable effects even for $d=2$, \eg see eq.~\reef{almost}. In contrast, the early discussions of \eg \cite{Almheiri:2019hni,Almheiri:2019psf, Almheiri:2019yqk, Chen:2019uhq, Penington:2019kki, Almheiri:2019qdq} assumed that one could use standard formulae for conformal transformations in the $d=2$ CFT in the gravitational region (\ie on the brane). It would be interesting to understand if the cutoff modifies any of this analysis in a significant way \cite{QEI}.\\

\hd{Brane geometry, Part I:} As described in section \ref{sec:branegravity}, we choose the brane tension to produce a negative cosmological constant in the gravity theory on the brane, in accord with eqs.~\reef{act3} and \reef{Newton2}. As a result, the $d$-dimensional geometry on the brane is AdS space. However, it is straightforward to consider the case where the brane tension takes its critical value, such that $1/\ell_\mt{eff}^2=0$, as is usually done in the Randall-Sundrum scenario \cite{Randall:1999ee,Randall:1999vf}. In this case, the analogous brane geometry is simply flat space, and the brane is easily embedded in the bulk AdS$_{d+1}$ geometry on a slice of constant radius (or constant $z$) in standard Poincar\'e coordinates. An interesting feature of this embedding is that the brane reaches the asymptotic AdS$_{d+1}$ boundary along the null boundaries of the flat space geometry (as well as a timelike and spacelike infinity) \eg see \cite{Karch:2001cw}. 
%\dn{I've added Karch-Randall as a ref, since they draw nice pictures (see figure 5 in ''locally localized gravity''. However, I don't think there is an explicit reference which discusses these constructions in detail. It seems to always have been common knowledge.} 
Hence we can naturally investigate quantum extremal surfaces and the island formula in flat space using the usual expressions for holographic entanglement entropy in this construction as long as we consider regions on null infinity. Notably this matches the approach pursued in \cite{Hartman:2020swn}, but contrasts with studies of \eg \cite{Gautason:2020tmk} which considered spacelike regions. It would, of course, be interesting to use this framework to study quantum extremal islands in the context of asymptotically flat braneworld black holes, \eg as described in \cite{Emparan:1999wa,Emparan:1999fd}. We should note however that there are undoubtedly subtleties with the proposed construction, \eg as the brane completely cuts out the asymptotic AdS$_{d+1}$ boundary (except for a single point) on constant time slices. 

Of course, one can also consider the case where the brane tension is chosen such that $1/\ell_\mt{eff}^2<0$. That is, the brane gravity theory would have a positive cosmological constant and the corresponding brane geometry becomes de Sitter space. In this case, one constructs a foliation of the bulk AdS$_{d+1}$ geometry in terms of $d$-dimensional de Sitter slices and the brane can be embedded along the slice with the appropriate curvature, \eg see \cite{Karch:2001cw}. In this case, the brane reaches the asymptotic AdS$_{d+1}$ boundary on the future and past timelike infinities of the de Sitter geometry. 
%\dn{Don't know of any reference, but again seems like common knowledge.} 
Hence, this construction provides a framework to use holographic entanglement entropy for investigating the island formula in de Sitter space as long as we consider regions on the timelike future of the latter geometry. Let us add that this would be similar to upcoming work of \cite{dSone,dStwo}, which studies related questions in the context of JT gravity with a positive cosmological constant 
\cite{Maldacena:2019cbz}. The de Sitter evolution of the Hartle-Hawking vacuum prepares a two-dimensional CFT state on circle and the entanglement entropy of various regions in the latter state are investigated, revealing new islands in the de Sitter geometry \cite{dSone,dStwo}.\\ 
%\dn{I'm not sure what the last sentence refers to.}\\

\hd{Brane geometry, Part II:}



The geometry of the setup presented in this paper might look unconventional. As seen from the brane perspective, we have the bath CFT on the asymptotic boundary with geometry $S^{d-1} \times \mathbb R$, and two copies of the same CFT on the brane with an AdS$_d$ geometry. These two geometries are joined by introducing a cutoff surface (with topology $S^{d-2} \times \mathbb R$) near the asymptotic boundary of the AdS$_d$ geometry and gluing it to the equator of the  $S^{d-1} \times \mathbb R$ geometry. In particular, the resulting geometry is not a manifold in the vicinity of the gluing region -- see the left panel of figure \ref{fig:no_mfld}. Of course, we can obtain a manifold by taking the $\mathbb Z_2$ quotient which identifies the two halves of the bath CFT, such that the theory is again defined on a manifold with topology $S^{d-1} \times \mathbb R$. However, we will ignore this simplification here. Rather, we want to comment on the theory before taking the $\mathbb Z_2$ quotient. 

\begin{figure}[t]
\centering
\includegraphics[scale = 0.7]{no_mfld}
\caption{Left: In the brane perspective, the bath CFT on the asymptotic boundary (blue) is connected to two copies of the effective CFT on the brane (green) but the resulting geometry is not a manifold. Right: For excitations below the effective CFT cutoff the system behaves as if it consists of two systems on a manifold which are weakly coupled in the gravitational region (green).}
\label{fig:no_mfld}
\end{figure}

First, we note that constructions where multiple CFTs are joined at a common defect are not rare. For example they appear in the study of boundary and interface CFTs (\eg see \cite{Chiodaroli:2012vc}), and sometimes seem to be required to remove anomalies \cite{Ooguri:2020sua}.

Second, we would like to argue that in the regime where the defect theory can be described by two copies of the boundary CFT coupled to Einstein gravity, we can approximately think of the full theory as two copies of the orbifolded theory (each living on a manifold), which are weakly coupled in the gravitational region -- see the right panel of figure \ref{fig:no_mfld}. This is particularly easy to see from the bulk perspective. For brevity we restrict ourselves to the discussion of graviton modes, but a similar story applies to all bulk fields. 

Let us begin by recalling that for $\veps \ll 1$, the spectrum of graviton fluctuations in the bulk is almost unchanged with respect to the modes in (two copies of) empty AdS space. Hence much of the corresponding physics should be very similar that of two copies of the the AdS$_{d+1}$, or to two copies of the dual CFT$_d$ on the boundaries of two independent AdS$_{d+1}$ geometries. Of course, one exception to the preceding is that upon gluing the two AdS$_{d+1}$ geometries together, a new set of very light graviton states localized in the vicinity of the brane \cite{Randall:1999vf,Randall:1999ee,Karch:2000ct,Karch:2001jb}, as discussed in section \ref{face}. For simplicity, we refer to the latter as the brane graviton modes, while we refer to the former as the standard normalizable modes.\footnote{These bulk modes are $\mathbb Z_2$ graded under reflection across the Planck brane, and the even modes survive the $\mathbb Z_2$ orbifold discussed above include the brane graviton states as well as half of the standard normalizable modes. However, this organization of the modes is not useful for the following discussion.}

On a fixed time slice, as shown in the right panel of figure \ref{fig:brane2}, the standard normalizable modes will describe stress energy excitations in the dual CFT on both the left and right halves of the asymptotic boundary. If we assume an approximate extrapolate dictionary \cite{Harlow:2011ke} for the brane theory as well, these normalizable modes will also describe analogous excitations for the effective CFT on the brane. However, there will be two sets of such excitations: those described by bulk excitations\footnote{We stress here that the localized excitations considered here do not correspond to individual energy eigenmodes, which were implicit in the previous paragraph. Rather they will consist of linear combinations of such eigenmodes evaluated on the fixed time slice being examined here. Of course, having superpositions of energy eigenmodes is what produces the complicated time evolution described below.} with support primarily in the right copy of the AdS$_{d+1}$ geometry, and those described by the analogous excitations primarily in the left AdS$_{d+1}$ geometry. Hence, the stress tensor on the brane can be decomposed into two pieces which correspond to subsectors of the brane theory, each of which is determined by bulk excitations which essentially live on one side of the brane. If these subsectors were truly superselection sectors (\eg as one might imagine arises in the limit $\veps\to0$), our brane theory would contain two independent copies of the boundary CFT  and each of these copies would only interact with the bath CFT on the corresponding half of the asymptotic boundary. That is, each of these systems would live on an independent manifold with topology $S^{d-1} \times \mathbb R$. 

However, this is not strictly correct and the two copies of the CFT on the brane are weakly coupled with $\veps\ll1$ but finite. In particular, localized stress energy excitations of the form considered above will not remain localized with time evolution. Rather they will eventually spread across the entire asymptotic boundary if time evolves for a sufficiently long time. For example, an excitation localized on the right asymptotic boundary will evolve to eventually produce excitations of the stress tensors on the left asymptotic boundary and on the brane as well. From the boundary perspective, excitations moving onto the brane correspond to excitations that are absorbed by the conformal defect (and remain there for a long time).

The spreading of the localized excitations can be seen to arise through two physical effects: First, the bulk excitations can tunnel between the two AdS$_{d+1}$ regions shown in figure \ref{fig:brane2}. Recall that (the radial part of) the linearized bulk equation of motion can be reduced to a Schroedinger equation with a double-well potential, where the height of the barrier is determined by the brane tension \cite{Karch:2000ct}. With $\veps\ll1$ but finite, the barrier height while large remains finite and there will be a finite probability for a bulk excitation on one side of the Planck brane to tunnel to the other. A second independent coupling comes because the stress tensors of the two copies of the CFT couple to the same gravity theory on the brane. From the bulk perspective, the nonlinear Einstein equation produces interactions between the brane graviton modes with excitations on either side of the brane. Hence bulk excitation excitations on one side can leak to the other side by scattering process involving the brane gravitons. However, we note that both effects become smaller as the brane tension approaches its critical value, \ie as $\veps$ approaches zero. Thus, to a good approximation, the brane theory can be treated at two copies of the boundary CFT which only interact weakly. \\


\hd{Entanglement wedge cross-sections:} Recent work \cite{Takayanagi:2017knl,Nguyen:2017yqw} has drawn attention to
the entanglement wedge cross-section, \ie for disconnected boundary regions, the codimension-two surfaces in the bulk which have minimal area and which split the entanglement wedge in two. In particular, there are a number of proposals relating these holographic surfaces to various entanglement measures: entanglement of purification \cite{Takayanagi:2017knl,Nguyen:2017yqw}, reflected entropy \cite{Dutta:2019gen},  odd entanglement entropy \cite{Tamaoka:2018ned,Kusuki:2019evw,Kusuki:2019rbk}, or entanglement negativity \cite{Kudler-Flam:2018qjo,Kusuki:2019zsp}. 

Turning to our model and examining figure \ref{fig:RTPhases}, we see that there are two such minimal surfaces in the connected phase, for which a quantum extremal island appears on the brane. These surfaces are simply disks of radius $P=P_0$ on either side of the brane, with area
\beq\label{reflw}
A=\frac{2\,L^{d-1}\, \Omega_{d-2}}{d-1} \ P_0^{d-1}\, {}_{2}F_1\left[ \frac{1}{2},\frac{d-1}{2},\frac{d+1}{2},-P_0^2 \right]\,,
\eeq
as can be seen from eq.~\reef{A_disc}. The fact that both disks have the same area results from the fact that the corresponding boundary regions are symmetric on either of the conformal defect -- see figure \ref{EEprob}. Of course, if one of the two caps comprising the boundary regions was smaller, the minimal area disk closer to this cap would provide the global minimum and hence become the entanglement wedge cross-section. It would be interesting to understand if the second minimal disk also plays an interesting role in characterizing the entanglement of the boundary state. In this vein, let us add that there are also two additional extremal disks which divide the entanglement wedge in two but their area is actually a local maximum. These disks again lie on either side of the brane but end on $\sigma_\xR$, the intersection of the RT surface with the brane. Again, it is natural to wonder if these surfaces have an interpretation in terms of the boundary entanglement. Let us note that similar surfaces appear in the following discussion.\\

\hd{RT Bubbles and Wormholes:} 

	In appendix \ref{bubble}, we consider a surprising class of RT surfaces with the topology of a sphere, \ie $S^{d-1}$ in the ($d+1$)-dimensional bulk. The appearance of these extremal `bubbles' is quite unusual as they are homologous to the entire boundary. Hence the standard RT prescription would assign an entropy to the ground state of the dual boundary system. Further, presence of a `zero mode' which allows the bubbles to be translated along the brane makes their interpretation even more puzzling. An essential feature for the appearance of the RT bubbles was that the gravitational coupling in the DGP term \reef{newbran} was negative, \ie $\lamb<0$. We also noted that the bubbles do not appear to be macroscopic objects in the brane theory. Rather, as shown in eq.~\reef{haiku2}, their size is always of order of the effective cutoff $\tilde\delta$.
	
Despite the unusual features of these RT bubbles, the discussion in appendix \ref{bubble} highlights a general feature of the quantum extremal islands in a simple way. In particular, as discussed below eq.~\reef{genbubble1}, there are two competing terms contributing to the generalized entropy of these surfaces: the bulk area which describes the entropy of the CFT fields on the brane enclosed by the bubble and the area of the boundary where they intersect the brane, which appears in the gravitational entropy of the DGP term. The bulk contribution naturally acts to contract the bubble but with $\lamb<0$, the brane contribution acts to expand the bubble. As described in the appendix, there is an equilibrium radius where these two effects balance one another. Of course, with $\lamb>0$, the brane contribution also acts to contract the boundary of the bubble and so no closed extremal surfaces appear, as expected.

As noted above, a similar competition is a general feature in the formation of quantum extremal islands. However, in this case as discussed in section \ref{sec:enzyme}, the bulk and brane contributions combine to produce a Bekenstein-Hawking term $\area(\sigma_\xR)/{4G_\mt{eff}}$ on the boundary of the island. This contribution, of course, imposes a large penalty to the formation of a large island and acts to contract the boundary towards a smaller (\ie vanishing) radius. For an island to appear, this contraction must be balanced by an expanding contribution. From the bulk perspective, this is simply coming from the remaining\footnote{We combined part of the bulk area into the Bekenstein-Hawking term above.} bulk area contribution of the RT surface, which we can ascribe to the quantum EE of the CFT state from the brane perspective. The point to be noted here is that for this to provide an expansion the RT surface must be anchored far from the island, \ie in the asymptotic (nongravitational) region associated with the boundary CFT. While perhaps self-evident, this discussion highlights the nonlocal nature of the physics producing the quantum extremal islands.

Let us add that the quantum extremal islands discussed here (as well as the RT bubbles) are remnants of replica wormholes in the limit $n\to1$. This follows from the fact that we are simply studying holographic EE with RT surfaces in a new bulk background, \ie with a back-reacted brane. Hence the analysis of \cite{Lewkowycz:2013nqa}\footnote{Following \cite{Dong:2016hjy,Faulkner:2017vdd}, the same applies for general time dependent situations.} introduces a smooth $n$-fold covering geometry for the corresponding Renyi entropies with positive integer indices. These covering geometries produce smooth wormhole geometries on brane analogous to those discussed in \cite{Almheiri:2019qdq,Penington:2019kki} for two dimensions. 

Now assuming replica symmetry, one can then take a $\mathbb Z_n$ orbifold quotient which leaves a single copy of the boundary geometry but the bulk solution now contains a codimension-two cosmic brane with tension $T_n=(n-1)/(4\Gbk\,n)$. In the presence of a DGP brane, we expect that there is an additional contribution where the two branes intersect, \ie the intersection surface carries an intrinsic tension $\widehat T_n=(n-1)/(4\Gbr\,n)$. In this setting, our discussion above for the formation of quantum extremal islands extends to the Renyi entropies in a relatively straightforward way. In particular, we expect that an area contribution associated with the boundary of the island now carries an effective tension $\tilde T_n=(n-1)/(4G_\mt{eff}\,n)$, which combines the intrinsic tension of this intersection surface and the contribution of the cosmic brane in the vicinity of the Planck brane. The contraction created by this term must be balance by the expansion provided by the remaining cosmic brane contributions. However, to provide an expansion the cosmic brane must be anchored by a twist operator in the asymptotic (nongravitational) boundary. Again, this highlights the nonlocal nature of the physics which implicitly supports the replica wormholes.

Of course, these dynamical considerations are emergent in the topological models considered in \cite{Marolf:2020xie,Penington:2019kki}. Hence it would be interesting to understand the implications of this dynamics to extend the new discussions of baby universes and ensembles to higher dimensions.\\



To conclude, let us comment that we will build on the holographic model constructed here to study the Page curve and the appearance of quantum extremal islands for higher dimensional black holes in \cite{QEI}. In particular, we study eternal black holes coming to equilibrium with an external heat bath (prepared at the same temperature) in a higher dimensional analog of the analysis appearing in \cite{Almheiri:2019yqk}. Let us reiterate that unconventional features (\ie Gauss-Bonnet and DGP couplings) introduced to favour quantum extremal islands here are unimportant in the discussion of higher dimensional black holes.\\





%%% Local Variables:
%%% mode: latex
%%% TeX-master: "../lifeonbrane3"
%%% End:




\section*{Acknowledgments}
We would like to thank Ahmed Almheiri, Raphael Bousso, Xi Dong, Netta Engelhardt,  Zach Fisher, Greg Gabadadze, Juan Hernandez, Don Marolf, Shan-Ming Ruan, Edgar Shaghoulian, Antony Speranza and Raman Sundrum for useful comments and discussions. Research at Perimeter Institute is supported in part by the Government of Canada through the Department of Innovation, Science and Economic Development Canada and by the Province of Ontario through the Ministry of Colleges and Universities. RCM is supported in part by a Discovery Grant from the Natural Sciences and Engineering Research Council of Canada, and by the BMO Financial Group. HZC is supported by the Province of Ontario and the University of Waterloo through an Ontario Graduate Scholarship. RCM and DN also received funding from the Simons Foundation through the ``It from Qubit'' collaboration. The work of IR is funded by the Gravity, Quantum Fields and Information group at AEI, which is generously supported by the Alexander von Humboldt Foundation and the Federal Ministry for Education and Research  through the Sofja Kovalevskaja Award. IR also acknowledges the support of the Perimeter Visiting Graduate Fellows program and the hospitality of Perimeter Institute, where part of this work was done. JS acknowledges the support of the Natural Sciences and Engineering Research Council of Canada (NSERC).

\appendix

\section{Generalized Entropy on the Brane}\label{generalE}
% !TEX root = ../lifeonbrane3.tex
%

In sections \ref{sec:two-d} and \ref{sec:DGP}, we introduced intrinsic gravitational terms to the brane action. Following \cite{Almheiri:2019hni},\footnote{See also \cite{Almheiri:2019psf, Almheiri:2019yqk, Chen:2019uhq, Penington:2019kki, Almheiri:2019qdq}.} we assumed that these terms contribute to the generalized entropy, \eg see eq.~\reef{eq:sad0} or \reef{eq:sad}.
In this appendix, we present a extended version of an argument in \cite{Myers:2010tj}, which will support this assumption and our formula for generalized entropy. 

As in the main text, we begin with a $d$-dimensional holographic CFT on $R\times S^{d-1}$ with a conformal defect on the equator of the sphere, sweeping out $R\times S^{d-2}$. On a fixed time-slice, we choose an entangling surface $\SCFT$ which divides the sphere into two equal halves along a maximal $S^{d-2}$ which lies orthogonal to the conformal defect. Now we wish to determine the entanglement entropy between the two halves
of the system, as sketched in figure \ref{fig:defect}. Recall that with the geometric approach \cite{Callan:1994py}, we must evaluate the partition function on a (Euclidean) background geometry with an infinitesimal conical defect. In order to construct a symmetric geometry where introducing such a defect is well-defined, we perform a Wick rotation on the boundary time (\ie $t_\mt{E}=it$) and then conformally transform the Euclidean background metric to a round $S^{d}$ with the conformal defect lying on a maximal $S^{d-1}$ on this background. Now $\SCFT$ remains a maximal $S^{d-2}$ which runs orthogonal to the defect and pierces the latter on a $S^{d-3}$. With this construction, there is a rotational symmetry in the two dimensions orthogonal to $\SCFT$. To evaluate the corresponding entanglement entropy, we construct $\mathcal{M}_{1-\eps}$, the `$n$-fold cover' with $n=1-\eps$, by introducing an infinitesimal conical defect at $\SCFT$. The entanglement entropy is then given by
\beq\label{entro9}
S = \lim_{\eps\to0}\left( \frac{\partial\ }{\partial\eps}+1 \right)\log
Z_{1-\eps}\,,
\eeq
where $Z_{1-\eps}$ is the partition function of the holographic CFT on the covering space $\mathcal{M}_{1-\eps}$. Of course, the latter has a dual description in terms of the bulk gravity, and using the usual saddle point approximation, eq.~\reef{entro9} becomes \cite{Myers:2010tj}
\beq\label{entropylimit}
S =-\lim_{\epsilon\to0}\Big(\frac{\partial}{\partial \epsilon} + 1\Big) I_{E, 1-\epsilon}\,,
\eeq 
where $I_{E,1-\epsilon}$ is the Euclidean bulk action evaluated on the appropriate dual solution.
%
%\begin{figure}[h]
%	\def\svgwidth{0.3\linewidth}
%	\centering{
%		\input{defect.pdf_tex}
%		\caption{A timeslice of our $d$-dimensional CFT setup with entangling surface $\Sigma_\mt{CFT}$. An infinitesimal conical defect $\Sigma_\xR$ runs through the bulk and intersects the brane at $\sigma_\xR$.} \label{fig:defect}
%	}
%\end{figure}
\begin{figure}[h]
	\def\svgwidth{0.6\linewidth}
	\centering{
		\input{defect31.pdf_tex}
		\caption{A timeslice of our $d$-dimensional CFT setup with entangling surface $\Sigma_\mt{CFT}$ and an equatorial conformal defect (the green line). In the right panel, one dimension is suppressed relative to the left panel.} \label{fig:defect}}
\end{figure}

Setting $n=1$ for a moment, the bulk dual of $\mathcal{M}_{1}$ is simply the Euclidean version of the geometry constructed in section \ref{BranGeo}, which we denote $\widetilde{\mathcal{M}}_{1}$. Recall the boundary geometry is $S^{d}$ and the conformal defect runs around a maximal $S^{d-1}$. In the bulk, the geometry is locally EAdS$_{d+1}$ everywhere away from the brane, and the brane has a EAdS$_d$ geometry which extends out to the conformal defect at the asymptotic boundary and with the curvature scale given by eq.~\reef{curve1} -- see figure \ref{fig:defect2}. Now the  entangling surface $\SCFT$ on the asymptotic AdS boundary is the boundary of an extremal surface $\Sigma_\xR$ in the bulk, which runs  straight across the bulk solution and has a EAdS$_{d-1}$ geometry with curvature scale $L$.  This surface pierces the brane at a right angle and the intersection, another extremal surface $\sigma_\xR$, has the geometry of a EAdS$_{d-2}$ with curvature scale $\ell_\mt{B}$ -- see figure \ref{fig:defect2}. Now because of the symmetry of this configuration, the rotational symmetry about the entangling surface in the boundary extends to a rotational symmetry about $\Sigma_\xR$ in the bulk. Hence we can calculate the entanglement entropy with the same geometric approach as we applied in the boundary. That is, we construct $\widetilde{\mathcal{M}}_{1-\eps}$, the $n$-fold cover (with $n=1-\eps$) of the bulk solution  with a infinitesimal conical defect at $\Sigma_\xR$ and by extension, at $\sigma_\xR$ on the brane. 

\begin{figure}[h]
	\def\svgwidth{0.6\linewidth}
	\centering{
		\input{defect32.pdf_tex}
		\caption{A cross-section of the Euclidean geometry $\widetilde{\mathcal{M}}_{1}$. The orange semicircle and its complement along a time slice represent the orange shaded region of figure \ref{fig:defect} and its complement. The rotation that keeps $\Sigma_\mt{CFT}$ fixed represents euclidean time. An infinitesimal conical defect $\Sigma_\xR$ runs through the bulk and intersects the brane at $\sigma_\xR$.} \label{fig:defect2}
	}
\end{figure}

That is, the angle around $\Sigma_\xR$ runs through a range $2\pi(1-\epsilon)$. Now  \cite{Fursaev:1994ea,Fursaev:1995ef} developed a description of such conical defects in which the singular geometry is replaced by a `regulator' geometry where the region
around the conical singularity is smoothed out. Applying their key result, we can write the bulk Riemann tensor  as a ``smooth" contribution away from $\Sigma_\xR$, the conical defect, and a singular order $\epsilon$ contribution at $\Sigma_\xR$,\footnote{This order $\eps$ contribution is universal, whereas the details of the regulator come into play at order $\eps^2$ and higher.}
\beq\label{separation}
^{(\epsilon)}R^{ab}{}_{cd} = R^{ab}{}_{cd}+2\pi \epsilon\, {\varepsilon}^{ab}{\varepsilon}_{cd}\,\delta_{\Sigma_\xR}\,,
\eeq 
where ${\varepsilon}_{ab}$ is the Euclidean volume form in the two-dimensional transverse space to $\Sigma_\xR$, and $R^{ab}{}_{cd}$ is the ``smooth" curvature piece. The $\delta_{\Sigma_\xR}$ is a two-dimensional delta function defined in \cite{Myers:2010tj}. The conical singularity intersects the brane at $\sigma_\xR$ and so we have a similar decomposition for the Riemann tensor on the brane,
\beq\label{separation2}
^{(\epsilon)}\tilde R^{ij}{}_{k\ell} = \tilde R^{ij}{}_{k\ell}+2\pi \epsilon \,\tilde{\varepsilon}^{ij}\tilde{\varepsilon}_{k\ell}\,\delta_{\sigma_\xR}\,.
\eeq 

Now recall that our aim is to evaluate the Euclidean action in eq.~\reef{entropylimit}. This action is the sum of the Euclidean versions\footnote{Note that the difference in signs in going between Minkowski and Euclidean signatures \cite{Myers:2010tj}.} of the bulk and brane actions in eqs.~\reef{act2} and \reef{newbran} (or perhaps eq.~\reef{JTee} for $d=2$), as well as the associated boundary terms. Equipped with eqs.~\reef{separation} and \reef{separation2}, it can be shown that in the limit of small $\epsilon$ that the Euclidean action can be expanded as
\beqa\label{epsilonaction}
I_{E,1-\epsilon}& =& (1-\epsilon)I_{E,1} +
 \int_\mt{bulk}\!\! d^{d+1}x \sqrt{g}\, 2\pi \epsilon {\varepsilon}^{ab}{\varepsilon}_{cd}\,\delta_{\Sigma_\xR}\, \frac{\partial \mathcal{L}_\mt{E,bulk}}{\partial R^{ab}{}_{cd}}\\
 %
&&\qquad+\int_\mt{brane}\!\!\!\! d^{d}x \sqrt{\tilde g} \, 2\pi \epsilon \tilde{\varepsilon}^{ij}\tilde{\varepsilon}_{k\ell}\,\delta_{\sigma_\xR}\,  \frac{\partial \mathcal{L}_\mt{E,brane}}{\partial \tilde{R}^{ij}{}_{k\ell}} +\mathcal{O}(\epsilon^2)\,.
\eeqa
Noting the symmetry of our configuration, \ie the curvatures are constant everywhere along the surfaces $\Sigma_\xR$ and $\sigma_\xR$,
we then find the entropy in eq.~(\ref{entropylimit}) is given by
\beq\label{fish9}
S = -2\pi \frac{\partial \mathcal{L}_\mt{E,bulk}}{\partial R^{ab}{}_{cd}} {\varepsilon}^{ab}{\varepsilon}_{cd} \int_{\Sigma_\xR} d^{d-1}x \sqrt{h}-2\pi \frac{\partial \mathcal{L}_\mt{E,brane}}{\partial \tilde{R}^{ij}{}_{k\ell}} \tilde{\varepsilon}^{ij}\tilde{\varepsilon}_{k\ell} \int_{\sigma_\xR} d^{d-2}x \sqrt{h'}\,,
\eeq 
where $h$ and $h'$ are the induced metrics along the $\Sigma_\xR$ and $\sigma_\xR$, respectively. Hence we see that there is a contribution of the Wald entropy from both the bulk action and the brane action. Further, let us note that various signs appear upon analytically continuing back to Lorentzian spacetime, \ie in the Lagrangian and the transverse volume form \cite{Myers:2010tj}. 


For the case where the Einstein-Hilbert action appears both in the bulk and on the brane, as in eqs.~\reef{act2} and \reef{newbran}, we find the formula for the generalized entropy \reef{fish9} becomes
\beq\label{fish88}
S = \frac{A(\Sigma_\xR)}{4 G_\mt{bulk}}+ \frac{A(\sigma_\xR)}{4 G_\mt{brane}}\,,
\eeq
as given in equation \reef{eq:sad}. The present derivation only applies to special symmetric configuration, as in \cite{Myers:2010tj}. The symmetry of this configuration preculdes finding any extrinsic curvature terms in eq.~\reef{fish9}, as would be expected for the Dong entropy \cite{Dong:2013qoa}. We note however that no such terms would correct eq.~\reef{fish88} for the generalized entropy coming from the Einstein-Hilbert term. It would, of course, be interesting to extend our derivation to more general configurations involving bulk DGP branes, along the lines of \cite{Lewkowycz:2013nqa} or \cite{Dong:2016hjy}.

%As a final note here, let us observe that our derivation of eq.~\reef{fish88} involved explicitly introducing a  comparing this singular approach to \cite{Lewkowycz:2013nqa} as Callan-Wilczek \cite{Callan:1994py} is to Gibbons-Hawking \rcm{ref??}


\section{RT Bubbles}
\label{bubble}
\input{sections/bubbles}

%\section{Thinking about Ensembles?}
%\label{ensign}
%% !TEX root = ../lifeonbrane3.tex
%



\rcm{lots of new references to read}

In order to derive the island formula, the appearance of wormholes in the replica trick is a crucial ingredient.
In the two-dimensional models involving JT gravity studied so far \cite{Almheiri:2019qdq,Penington:2019kki}, the existence of wormholes follows from the fact that JT gravity is defined by ensemble averaging over an ensemble of Hamiltonians.\footnote{For example, JT gravity emerges as the low energy effective description of the SYK model \cite{}, or has a UV complete definition in terms of a matrix model \cite{Saad:2019lba}.} In fact, by assuming ensemble averaging, one can easily reproduce many features \cite{Marolf:2020xie} demonstrated in the 2d models. 

However, it is generally assumed that in higher dimensions gravity on asymptotically AdS spacetimes is dual to a single, unitary theory.\footnote{The best known example is the duality between $\mathcal N=4$ super Yang-Mills theory and supergravity (or more precisely Type IIB string theory) on AdS${}_5\times S^5$.} In fact, in the higher dimensional case one would expect replica wormholes to be absent in any local theory. This can be seen by considering a theory on a higher-dimensional spacetime which couples to gravity in a subregion $U_\text{grav}$, and lives on a fixed background in the complement $U_\text{bath}$. If we want to compute the density matrix of a region $A \subset U_\text{bath}$, we are instructed to integrate out all degrees of freedom in the gravitating region $U_\text{grav}$. If we take products of the reduced density matrix -- like we would do in the replica trick -- no integral over the gravitating region is left to be performed and no Euclidean wormholes appear. More generally, in two dimensions, replica Wormholes appear in calculations of the spectral form factor and yield to a non-factorizability of products of the partition function. However, for a CFT${}_d$ the partition function is a number and a product of many partition functions clearly must factorize.

This opens the question if and how islands and replica wormholes appear in the higher dimensional case. As we have seen, in our model, they are readily explained from the bulk perspective, but in order to go beyond the case of holographic matter, it is important to also understand how they arise in the brane picture. Here, we will explain a possible mechanism suggested by the model discussed in this paper, and its relation to ensemble averaging.

As discussed in this paper, the description of the brane theory as a local CFT with a cutoff coupled to gravity in some region of space is only an effective one. In fact, we see from the bulk picture that the degrees of freedom in $U_\text{bath}$ and $U_\text{grav}$ are not independent. A particularly clear sign is the fact that under certain conditions parts of the gravitating region $\mathcal I_A \subset U_\text{grav}$ lie in the entanglement wedge of bath subregions $A \subset U_\text{bath}$. This signalled the appearance of an island on the brane picture. In this case, as is clear from EW reconstruction, specifying the state in $A$ also specifies the state in the island $\mathcal I_A$. In the presence of a brane, new bulk modes appear, which can be identified with the now-dynamical metric and sources of other primary operators on the brane. Specifying the state in $\mathcal I_A$ thus in particular means to specify the states of the metric and all other sources. The description that takes into account the link between degrees of freedom in the bath and the gravitating region corresponds to the full quantum gravity description of the state of \cite{Almheiri:2019yqk}.

In an effective, local and semi-classical approximation this link between the bath and the gravitating region is ignored so that $A$ and its associated island $\mathcal I_A$ are treated as independent subregions. If we are interested in a reduced density matrix $\rho^\text{s.c.}_A$ of a $A$ in the semi-classical approximation, we can obtain it from a fully quantum gravity density matrix $\rho^\text{q.g.}_A$, containing information of $A$ and $\mathcal I_A$, by treating the degrees of freedom in $\mathcal I_A$ as independent and tracing over them. Now, consider taking a product of $n$ density matrices $(\rho^\text{q.g.}_A)^n$. If we want to go to the semi-classical picture, we need to trace over the state on $\mathcal I_A$,
\begin{align}
(\rho^\text{s.c.}_A)^n = \tr_{\mathcal I_A}((\rho^\text{q.g.}_A)^n)
\end{align}
This correlates the states of the metric, quantum fields and other sources in any of the replica copies of $\mathcal I_A$ in the product. Moreover, since the trace also runs over modes which are associated with sources on the brane, this procedure effectively looks like ensemble averaging the sources on $\mathcal I_A$. This ensemble averaging then has an immediate description in terms of including Euclidean wormholes in the path integral.\\



%\section{Details of AdS/ICFT} \label{details}
%% !TEX root = ../lifeonbrane3.tex
%



We will give some detailed calculations which show how the change in bulk parameters is related to changes in the brane/boundary picture. For simplicity, we consider the case of a Poincare patch with a time-like co-dimension one defect which passes through the origin. The defect breaks the conformal symmetry from $SO(d,2)$ to $SO(d-1,2)$.

\subsection{Interface OPE expansion in holography}
An operator in the CFT has an expansion in terms of operators living at the interface. If we denote by $x$ the direction perpendicular to the brane and by $y$ the direction along the brane, we have that
\begin{align}
    O_{\Delta}(x,y) = \sum_{\Delta'} c_{O_{\Delta}O_{\Delta'}} \frac{O_{\Delta'}(y)}{|x|^{\Delta - \Delta'}}.
\end{align}
We will now discuss how this expansion is realized holographically. To this end we choose AdS slicing coordinates
\begin{align}
    ds_{AdS_{d+1}}^2 = \frac{L^2}{\sin^2 \theta}(d\theta^2 + ds_{AdS_d}^2).
\end{align}
The boundaries are located at $\theta = \pm \pi$.
As mentioned above, we will for simplicity assume that $ds_{AdS_d}^2$ is the AdS metric in Poincare coordinates,
\begin{align}
    ds_{AdS_d}^2 = \frac{dx^\mu dx_\mu + dz^2}{z^2}
\end{align}
Consider a scalar field of mass $m$. We will make a the ansatz
\begin{align}
    \phi(x,z,\theta;p) = e^{i p x} \psi(z) f(\theta).
\end{align}
The equation of motion then yields
\begin{align}
    \psi(z) \left(f''(\theta )-(d-1) \cot (\theta ) f'(\theta )\right)\\
    +f(\theta ) \left(z \left(-(d-2) \psi'(z) +z \psi''(z)-z \psi(z) p^2\right)-m^2 \psi(z) \csc ^2(\theta )\right) = 0
\end{align}
We need to solve
\begin{align}
    z \left(-(d-2) \psi'(z)+z \psi''(z) - z \psi(z) p^2 \right)=\lambda  \psi(z).
\end{align}
This is the $z$ dependent part for the equation of a scalar field on AdS$_d$ with mass-squared $\lambda$. As we will see immediately, this means that $\lambda$ is bounded from below by the Breitenlohner-Freedman bound, $\lambda \geq - (\frac{d-1}{2})^2$. As is well known, the general solution can be expressed in terms of Bessel functions,
\begin{align}
    \psi(z) = c_1 z^{\frac{d-1}{2}} J_{\frac{1}{2} \sqrt{d^2-2 d+4 \lambda +1}}\left(z \sqrt{-p^2}\right)+c_2 z^{\frac{d-1}{2}} Y_{\frac{1}{2} \sqrt{d^2-2 d+4 \lambda +1}}\left(z \sqrt{-p^2}\right).
\end{align}
The equation for $f(\theta)$ then becomes
\begin{align}
\left(f''(\theta )-(d-1) \cot (\theta ) f'(\theta )\right)+f(\theta ) \left(\lambda - m^2 \csc ^2(\theta )\right)=0
\end{align}
with solutions
\begin{align}
    \left( c_1 P_{\frac{1}{2} \left(\sqrt{d^2-2 d+4 \lambda +1}-1\right)}^{\frac{1}{2} \sqrt{d^2+4 m^2}}(\cos (\theta ))+c_2 Q_{\frac{1}{2} \left(\sqrt{d^2-2 d+4 \lambda +1}-1\right)}^{\frac{1}{2} \sqrt{d^2+4 m^2}}(\cos (\theta ))\right)\left(\sin ^2(\theta )\right)^{d/4},
\end{align}
where $P$ and $Q$ are Legendre functions. Note that $\lambda$ has not been fixed yet, so we have two continuous families of solutions. We will see in a bit that by carefully considering the situation at the bulk dual of the defect, $\lambda$ will be quantized. The defect operators will then be given as the dual operators to the $d$-dimensional fields $\psi[z]e^{ipx}$.


The Legendre functions are somewhat hard to work with, since for generic mass both solutions diverge as $\theta \to - \pi$ and only a particular linear combination forms the normalizable modes. It would therefore probably be useful to re-express them in terms of hypergeometric functions. We will not do this here, however, but continue discussing a special case in which $P$ becomes the normalizable mode. This happens for $m=0$, a massless field in the bulk.

In the case discussed in this paper, we have two copies of the spacetime between the boundary and the brane, which are identified along the brane. Since the asymptotic boundary is compact it is clear that the bulk modes must be quantized and in fact, it is well known that requiring regularity of the normalizable mode in the bulk in global AdS imposes a quantization condition on the bulk modes. By $\mathbb Z_2$ symmetry of the setup, we know that all modes are either even or odd under reflection across the brane. Without any additional sources added, the solution for a scalar field is is regular everywhere.\footnote{We will revisit this assumption below.} The even modes have vanishing normal derivative at the brane, while the odd modes vanish at the brane. We must also impose this condition on our solutions. This condition quantizes $\lambda$. The exact value must be determined numerically, however for simple cases, it can also be done analytically.

\paragraph{Example. } Let us take $x=0$ as the defect, i.e., we do not introduce a defect, but simply pretend that $x=0$ is a special place. The defect OPE then becomes simply the Taylor expansion of an operator around $x=0$. For a generalized free fields we have that the conformal family simply consists of derivatives of the operators, i.e., operators or dimensions $\Delta' = \Delta + n$. The relation between $\lambda$ and $\Delta'$ is 
\begin{align}
    \lambda = \Delta'(\Delta'-(d-1)).
\end{align}
In the case at hand we assumed a massless scalar field, so $\Delta = d$. The correct boundary condition is Dirichlet conditions at both boundaries. One can confirm explicitly by a numerical computation, that the quantization condition yields
\begin{align}
    \lambda = (d+n)(n+1).
\end{align}
In the case we present in the paper, we need to focus on even modes, since only those induce dynamical fields on the brane.

\subsection{Bound states and mode counting}
What happens if we introduce a defect and couple bulk fields to it? Before we answer this question, let us consider a somewhat simpler situation, namely that of a Randall--Sundrum brane. The aim of this section is to understand under which conditions bound states appear and which other modes we have to integrate out in order to obtain an effective theory for such modes. Other questions are about the number and orthogonality of modes. Lastly, we are interested in the holographic dictionary for the effective theory on the RS brane. 

We will take AdS in Poincare coordinates and consider a scalar field. The ansatz for the wavefunction reads
\begin{align}
    \psi(t,\vec x,z) = \iint d\omega d\vec p \; c_{\omega p }\; e^{i \omega t + i \vec p \vec x} f(z;p,\omega).
\end{align}
Using this, we want to solve the equations of motion in $d+1$ dimensions,
\begin{align}
    z^2 f''(z;p,\omega) - z (d-1) f'(z;p,\omega) + z^2 (-\omega^2 + p^2) f(z;p,\omega) + m^2 f(z;p,\omega) = 0.
\end{align}
The solutions are well known and given in terms of a linear combination of Bessel functions,
\begin{align}
    f(z;p,\omega) = c_1 z^{\frac{d}{2}} J_{\frac 1 2 \sqrt{d^2 - 4m^2}}(k z) + c_2 z^{\frac{d-1}{2}} Y_{\frac 1 2 \sqrt{d^2 - 4m^2}}(k z),
\end{align}
with $\omega^2 - p^2 = k^2$. If $\frac 1 2 \sqrt{d^2 - 4m^2}$ is not integer, we could have replaced $Y_{\frac 1 2 \sqrt{d^2 - 4m^2}}(k z) \to J_{-\frac 1 2 \sqrt{d^2 - 4m^2}}(k z)$ to obtain another set of linearly independent functions. Both, $Y_\alpha(k z)$ and $J_\alpha(kz)$ diverge as $z \to 0$, where the asymptotic boundary is located. Since we do not intend to turn on sources we drop those solutions and are left with $J_\alpha$. It obeys the identity
\begin{align}
    \int_0^\infty x J_{\alpha}(ux) J_{\alpha}(vx) dx = \frac 1 u \delta(u-v),
\end{align}
which we can use to normalize the solutions. The reason is that this integral takes the same form as the Klein Gordon norm along the $z$-direction.
\begin{align}
    \sim \int dz z^{-d+1} (\psi_1 \partial_t \psi_2 - \psi_2 \partial_t \psi_1).
\end{align}
Note that if we require regularity of the solution, we also have that $k^2 > 0$.

We now introduce a Randall--Sundrum brane at $z = z_*$. In order to understand how different boundary conditions change the number of modes, consider first the case where we impose Dirichlet conditions at the brane. It is clear already that now $k$ will be quantized. At leading order, the large $z$ expansion of $J_\nu$ is
\begin{align}
    J_\nu(k z) = \sqrt{\frac{2} {\pi k z}} \cos(k z - \frac 1 2 \nu \pi - \frac 1 4 \pi).
\end{align}
Clearly, for $k z_* \gg 1$ the zeroes are at 
\begin{align}
    k = \frac{\pi}{z_*} (n + \frac 1 2 \nu - \frac 1 4).
\end{align}
How does this change if we introduce non-trivial boundary conditions at the brane? In analogy with the gravitational case, let us pick Robin boundary conditions,
\begin{align}
    \phi'(z_*) = c \phi(z_*).
\end{align}
It is important that our boundary condition takes this form (as opposed to, say, $\phi'(z_*) = c \phi(z_*)$), since otherwise a linear combination of solutions would not be a solution anymore. Using the large $z$ expansion, we find that (assuming $c \geq \mathcal O(1)$)
\begin{align}
   \sin(k z_* - \frac 1 2 \nu \pi - \frac 1 4 \pi) = -\frac{c}{k} \cos(k z_* - \frac 1 2 \nu \pi - \frac 1 4 \pi).
\end{align}
The key point is that the modes are spaced at order $\frac 1 {z_*}$ and changes in the spectrum for a range of modes of $\Delta k < \mathcal O(1)$ (these are still order $z_* k$ modes) can be neglected. In other words: the spectrum changes with respect to the one with Dirichlet boundary conditions, but only very mildly.
\\
\\
Open question: What happens for $k z_* \sim \mathcal O(1)$?
\\
\\
Crucially, thanks to the modified boundary condition, a new state appears with $k^2 < 0$. The relevant solution is $I_\nu(k z)$. at large $k z$ this takes the form
\begin{align}
    I_\nu(k z) \sim \frac{e^{kz}}{\sqrt{2 \pi k z}}.
\end{align}
Clearly, this could never satisfy a Dirichlet boundary condition $\phi = 0$ at large $z_*$, but the Robin condition becomes at leading order
\begin{align}
    k = c.
\end{align}

In order to compare the effects, we should normalize the modes. The bound state very roughly has a norm of
\begin{align}
    \int dz z{^-d} e^{2 k z} \sim \frac{e^{2 k z}}{z^{d-1}}, 
\end{align}
so that to properly normalized wave function in the large $kz$ limit roughly becomes 
\begin{align}
   \psi \sim z_*^{\frac{d-1}{2}}\frac{e^{k(z-z_*)}}{\sqrt{2 \pi k z}}.
\end{align}
The positive energy states are roughly normalized and thus we see that evaluated at the brane, the bound state is bigger by a factor of $\sim z_*^{\frac{d-1}{2}}$.

At the asymptotic boundary, we have that
\begin{align}
    J_\nu(kz) \sim (\frac 1 2 kz)^\nu \Gamma(\nu + 1)^{-1} \\
    \psi \sim z_*^{\frac{d-1}{2}} e^{-k z_*} (\frac 1 2 kz)^\nu \Gamma(\nu + 1)^{-1}.
\end{align}
As a result of the normalization, we see that the bound state is exponentially suppressed. We conclude that by integrating out all KK modes, we produce the CFT at the asymptotic boundary together with a scalar field on the brane. However, both, the CFT at the asymptotic boundary as well as the theory on the Randall-Sundrum brane are only effective theories. The corrections to the theory on the RS brane are a power-law in the coordinate distance of the RS brane to the asymptotic boundary, while the corrections to the CFT are exponentially small.

For massless particles we expect that $k\sim 0$ and it is unclear how much of the above analysis still holds.


\subsection{The CFT on the brane}
Let us consider a fixed time, e.g. $t=0$. The bulk modes can be classified into even and odd modes under the $\mathbb Z_2$ symmetry coming from reflecting the system at the brane. A certain linear combination of even and odd modes only has support on one side of the brane, while the orthogonal linear combination has support on the other side. Furthermore, all odd modes vanish on the brane and only even induce fields.

It should be clear from the preceding section that the acceptable values of $\lambda$ and thus also the defect operator spectrum is determined by the location of the brane. Let us assume that we locate the brane very close to $\theta = \pi$. Imposing Neumann boundary conditions at the brane, one can easily see that the allowed values for $\lambda$ are very close to the allowed values for $\lambda$ had we simply quantized AdS without a brane. This means that the OPE spectrum coming from one side of the brane looks very much like the one of a CFT with a ``fake defect'' as discussed in the example above. This suggests that the theory on the brane has approximately the same light spectrum as the CFT on the asymptotic AdS boundary. Since there are theories on either side of the brane, the brane carries a second copy of the theory.

The operators of the brane CFT are given by the limiting value of the bulk fields. Let's say the brane is placed at $\epsilon_{UV} \equiv \epsilon$. We can calculate correlation functions of brane operators $\bar {\mathcal O} = \epsilon^\Delta \bar \phi$ on the brane by introducing a source to the bulk Lagrangian,
\begin{align}
    \dots + \int_{\partial} \bar J \bar {\mathcal O}.
\end{align}
Adding this term changes the Neumann boundary condition of the bulk field $\phi$ from $n \cdot \nabla \phi = 0$ to $(n \cdot \nabla \phi - J) = 0$. We thus see that in the holographic dictionary of the brane theory, the role of sources is played by the Neumann boundary condition (as opposed to Dirichlet conditions at the asymptotic AdS boundary). Correlation functions can also simply be calculated by using a modified version of the extrapolate dictionary, where we do not divide out by $z^\Delta$, but include this factor into the definition of the operators. This factor is precisely what gives the induced field its scaling dimension. If we are interested in calculating correlation functions between CFT operators and operators on the brane, the easiest way should be by just using the extrapolate dictionary.

\subsection{Adding DGP terms}
We can add additional terms to the brane, which as we will now see change the bulk solution, change the dynamics of the brane theory, and modify the defect CFT.
\subsubsection{Scalar fields}
The simplest term we can add to the brane is a source term
\begin{align}
    S_{DGP} = -\alpha \int \bar \phi,
\end{align}
where $\bar \phi$ is the induced field on the brane. This modifies the background solution, but not the equation for fluctuations around the solution. The reason is clear: Any contribution to the equations of motion of fluctuations around the classical solution must be at order $\mathcal O(\delta \phi^2)$, but the term discussed here vanishes at this order. However, what does change is the classical solution and with it the vacuum expectation value of $\mathcal O$, the operator dual to the field $\phi$. The reduced conformal invariance allows scalar one-point functions of the form 
\begin{align}
    \langle \mathcal O \rangle = \frac a {|x|}
\end{align}
and such a term is turned on by coupling $\phi$ linearly to the brane.\footnote{I have not calculated the exact coefficient.} As mentioned already above, it now becomes clear that a minimal coupling of scalar fields to the brane does change the background, but is not sufficient to create a bound state on the brane. For this to happen, we must try harder.

The next term we will discuss takes the form of a kinetic term
\begin{align}
    S_{DGP} = - \frac 1 2 \int (\partial \bar \phi)^2 + \bar m^2 \phi^2.
\end{align}
If we vary the action, this contributes an extra term to the equations of motion.
\begin{align}
    \Box \phi + \frac{\delta(\theta - \theta_B)}{\sin^2 \theta_B} (\bar \Box - \bar m^2) \bar \phi = 0.
\end{align}
If we expand $\phi$ in eigenfunctions of the D'alembert operator along the brane as before, we obtain for a mode with eigenvalue $\lambda$
\begin{align}
    \Box \phi_\lambda(\theta) + \frac{\delta(\theta - \theta_B)}{\sin^2 \theta_B}  (\lambda - \bar m^2) \bar \phi_\lambda = 0.
\end{align}
Thus, we note that a kinetic term induces a source term, which depends on the mode we are considering. Integrating this equation around $\theta_B$ will give a discontinuity in the first derivative of the field at the location of the brane
\begin{align}
    (\bar \phi'_{\lambda,L} - \bar \phi'_{\lambda,R}) +  (\lambda - \bar m^2) \bar \phi_\sigma = 0.
\end{align}
For the $\mathbb Z_2$ even solution we find
\begin{align}
        2 \partial_\theta \log \bar \phi_{\lambda} +  (\lambda - \bar m^2) = 0,
\end{align}
which quantizes $\lambda$, now however as a function of the brane location, $\bar m$ and $\lambda$ itself. Note that those terms essentially are Robin bounday conditions. The mass-term changes the boundary condition universally, while the kinetic term changes it for different modes. This discussion only affects the modes which previously had Neumann boundary conditions. The modes with Dirichlet conditions at the location of the brane are unaffected, since the source term vanishes.


The action for fluctuations $\psi$ around this solution obtains two additional terms.
\begin{align}
    \dots + \int (\bar \Box \bar \phi - \bar m^2 \bar \phi ) \psi - \frac 1 2 \int (\partial \psi)^2 - \bar m^2 \psi^2
\end{align}
It is thus easy to see that $\psi$ is also sourced and that the equations of motion for $\psi$ might allow a bound-state close to the brane.
This means, amongst other things, that the normal derivative at the brane changes which induces a change in the operator spectrum.


\subsubsection{The gravitational field}
Since the gravitational field is non-linear, the situation is slightly richer here. As discussed above, adding a simple tension term to the brane,
\begin{align}
    S_\text{brane} = - T \int \sqrt{|h|}, 
\end{align}
warps spacetime. In the case of the $\mathbb Z_2$ quotient of our setting, this can also be interpreted as moving the brane closer to (or further away from -- depending on the value of $T$) the second asymptotic boundary. Before tackling the global AdS situation we are considering, let us look at the analogous situation in Randall-Sundrum. There, metric fluctuations around this background obey an equation of the form
\begin{align}
    \left[\frac{-m^2}{2} e^{2 k |y|} - \frac 1 2 \partial_y^2 - 2 k \delta(y) + 2 k^2 \right] \psi(z) = 0.
\end{align}
Here, we just used their notation in which metric fluctuations can be split into contributions orthogonal and transverse to the brane directions, $h(x,y) = \psi(y) e^{ipx}$, $p^2 = m^2$ and $k$ is a parameter in their solution which controls the ratio between the bulk cosmological constant and the brane tension, $k^{-1} \sim L^2 G_\text{bulk} T_\text{brane} \sim \frac 1 L$. The relation between those parameters is a special feature of the Randall-Sundrum model.

An important observation is that the appearance of the $\delta$ function term is responsible for the support of a bound state. The shape of the wave function is controlled by $\frac{e^{k|y|}}{k}$, where $y=0$ is the location of the defect, and so we see that small $k$ means the bound state is wider. Note that in this case, larger $T_\brane$ means a wider bound state. The reason is that the Randall-Sundrum solution does not allow to fix $G_\mt{bulk}$ and $L$ while changing $T_\brane$.

Consider now the case where we have added an additional Einstein-Hilbert DGP term to the action of our brane. Note that we are in the bulk picture, where we have not intergrated out the directions orthogonal to the brane. The variation of the action obtains a new source-term
\begin{align}
    \int\sqrt{h}(\frac{G_{ab}}{16 \pi G_\brane}   - \frac 1 2 T h_{ab} + \frac 1 2 \Delta T h_{ab}) \delta h^{ab}.
\end{align}
In order not to change the brane position, we have that 
\begin{align}
     G_{ab} = - 8 \pi G_\brane \Delta T h_{ab}. 
\end{align}
In the RS scenario, $\Delta T =0$, since the location of the branes is independent of the brane tension. 
Let us now look at fluctuations around this background. The relevant terms appear at second order in the above equation. Recall that the factor $k$ which determines the width of the bound state is proportional the the tension $T$. The question now is whether the other two terms we introduced (and fixed by making them cancel) modify $k$. The relevant contributions come from the second order expansion of the brane action. We get the relevant second order terms by looking at the first-order contribution from 
\begin{align}
    \delta(\frac{G_{ab}}{16 \pi G_\brane}),
\end{align}
which however are just the linearized field equations. We find\footnote{http://www.physics.fau.edu/~cbeetle/PHY6938.07F/linearized.pdf} that for zero modes, i.e., those which obey Einsteins equations, the linearized equations also cancel and clearly, even for zero mode fluctuations, there is no additional contribution at the location of the brane. However, for KK modes a coupling 
\begin{align}
 - \frac{m^2}{32 \pi G_\brane}
\end{align}
is generated.



To get closer to our situation, imagine that the theory on the brane was not on a flat, but on an AdS background. In this case we would need to add an additional $\Delta T$-term. The metric on the brane obeys
\begin{align}
\delta G_{ab} = \frac 1 2 \frac{(d-1)(d-2)}{\ell_B^2} h_{ab}.
\end{align}
Fluctuations $\delta h_{ab}$ around that background obey
\begin{align}
\delta G_{ab} = \frac 1 2 \frac{(d-1)(d-2)}{\ell_B^2} \delta h_{ab},
\end{align}
where $\delta G_{ab}$ is evaluated on the AdS background. The quantity $\ell_B$ is determined by the choice of $T_0$. To add DGP terms, we modify the action by
\begin{align}
\Delta I_\brane = \frac 1 {16 \pi G_\brane} \int \sqrt h R + \int \sqrt h \Delta T.
\end{align}
Here, we finally added a $\Delta T$ which is fixed by the requirement that $\Delta I_\brane$ does not modify the location of the brane. In order to ensure this, we require $\delta \Delta I_\brane = 0$, which fixes
\begin{align}
\Delta T = \frac{(d-1)(d-2)}{16 \pi G_\brane \ell_B^2}.
\end{align}
Let us now consider bulk fluctuations. Their leading order action is given by expanding the bulk action including brane terms to second order in the fluctuations. However, the DGP-term vanishes even at second order for zero-modes,
\begin{align}
\delta^2 \Delta I_\brane =& \int \sqrt \frac 1 2 h^{ab} \delta h_{ab} \left( \frac{G_{cd}}{16 \pi G_\brane} - \frac 1 2 \Delta T h_{cd}) \delta h^{cd} \right) \\
&+ \int \sqrt{h} \left( \frac{\delta G_{cd}}{16 \pi G_\brane} - \frac 1 2 \Delta T \delta h_{cd}\right) \delta h^{cd}.
\end{align}
The first equation vanishes by definition of $\Delta T$, while the second line vanishes for on-shell $\delta h_{ab}$. Note that the on-shell modes are precisely the zero-modes of Randall-Sundrum. The KK-modes are off-shell. For them, adding DGP terms changes the coupling.


\subsubsection{Randall--Sundrum}
We can treat the RS scenario as a toy model for our case. We aim at understanding what the effect of adding a DGP term is. Without a DGP term, the graviton bulk solutions take the form
\begin{align}
\psi = A \left(|z| + \frac 1 k\right)^{1/2} J_2\left( m\left(|z| + \frac 1 k\right) \right) + B \left(|z| + \frac 1 k\right)^{1/2} Y_2\left( m\left(|z| + \frac 1 k\right) \right).
\end{align}
Close to the origin, where the condition
\begin{align}
\psi'(0) = - \frac 3 2 k \psi(0)
\end{align}
holds, we can approximate
\begin{align}
J_2 \sim \frac{m^2 (|z| + \frac 1 k)^2}{8} \\
Y_2 \sim - \frac{4}{\pi m^2 (|z| + \frac 1 k)^2} - \frac 1 \pi.
\end{align}
From this we obtain
\begin{align}
\psi(0) = A \frac {m^2} 8 \frac 1 {k^{5/2}} - B \left( \frac{4 k^{3/2}}{\pi m^2} + \frac 1 {\pi k^{1/2}} \right) \\
\psi'(0) = A \frac 5 2 \frac {m^2} 8 \frac 1 {k^{3/2}} + B \left( \frac 3 2 \frac{4 k^{5/2}}{\pi m^2} - \frac {k^{1/2}}{2 \pi } \right).
\end{align}

Adding a DGP term modifies the coupling to the brane to
\begin{align}
T_\brane \to T_\brane (1 + \frac{m^2}{16 \pi G_\brane T_\brane}) = T_\brane (1 + \frac{m^2 L^2}{12} \frac{G_\mt{bulk}}{L G_\brane}) \equiv T_\brane \lambda.
\end{align}
The term which multiplies the delta-function in the equations of motion thus gets multiplied, $k \to \lambda k$. While this leaves the bulk solution unchanged, the condition at the brane changes to
\begin{align}
\psi'(0) = - \frac 3 2 k \lambda \psi(0).
\end{align}
In terms of the solutions we find that
\begin{align}
\frac A B = \frac 8 {5 + 3\lambda} \frac{k^2}{\pi m^2} \left(12 (\lambda - 1) \frac {k^2}{m^2} + (3 \lambda + 1) \right).
\end{align}
The case without DGP terms is $\lambda = 1$ and we find
\begin{align}
\frac A B = \frac{4 k^2}{\pi m^2}.
\end{align}
The vanishing of the first term, however, suggests some fine-tuning.
If we substitute the expression for $\lambda$ we end up with ($2G_\mt{bulk} = G_\mt{eff} L$)
\begin{align}
\label{eq:ratioAB2}
\frac A B = \frac 8 {5 + 3\lambda} \frac{k^2}{\pi m^2} \left(\frac{G_\mt{eff}}{2 G_\brane} + 4 + \frac{m^2}{4 k^2} \frac{G_\mt{eff}}{2 G_\brane} \right).
\end{align}
We are now especially interested in the coupling of the fairly light KK modes, \ie $m L \ll 1$. For positive values of $G_\brane$, we see that the relative strength of both terms in the linear combination changes, 
\begin{align}
\frac A B = \frac{4 k^2}{\pi m^2} \left(1 + \frac{G_\mt{eff}}{8 G_\brane} \right).
\end{align}
However, if we choose $G_\brane$ negative and are close to regime where
\begin{align}
\frac{G_\mt{eff}}{G_\brane} \sim - 8, 
\end{align}
we can make the first two terms in the parenthesis cancel. We end up with 
\begin{align}
\frac A B \sim - \frac 4 \pi \sim \text{constant}.
\end{align}
This would mean that the ``non-normalizable'' modes are not suppressed for light masses anymore.

The next obvious question is what is the forbidden range for $\frac{G_\mt{eff}}{2 G_\brane}$? Since the effect of massive KK modes on the gravitational coupling is exponentially suppressed by a Yukawa term, we can focus on only the light KK modes. Requiring that the first two terms of equation \eqref{eq:ratioAB2} dominate and approximating the last term by $0$, we find 
\begin{align}
 \frac{4}{G_\mt{eff}} \gg - \frac{1}{2 G_\brane}.
\end{align}
Up to numerical factors of order $\mathcal O(1)$, this agrees with the condition on $G_\brane$ in the main text, which came from a lower bound on $(1+\lambda_B)$.




\subsubsection{Our setting}
Let us discuss gravity in our case in more detail. We consider slicing coordinates
\begin{align}
\frac{1}{\sin(\mu)^2} \left(d\mu + ds^2_{AdS, d}\right).
\end{align}
Let us call indices in the $d+1$-dimensional spacetime $M,N,\dots$ and in the slices $i,j,\dots$. The non-vanishing Christoffel symbols are
\begin{align}
\Gamma^\mu_{\mu\mu} = -cot(\mu) && \Gamma^i_{j\mu} = -cot(\mu) \delta^i_j && \Gamma^\mu_{ij} = cot(\mu) \tilde g_{ij} && \Gamma^i_{jk} = \tilde \Gamma^i_{jk},
\end{align}
where tilde denotes objects with respect to the metric on the slices.
Gauge freedom allows us to fix $d+1$ components on the metric fluctuation. In order make good use of the symmetries of our system we go to a Fefferman-Graham-like gauge, where we set all components of the metric involving $\mu$ to zero.
If we assume a Lagrangian of the form
\begin{align}
L = R - (d-1)(d-2) \Lambda,
\end{align}
the linearized equations of motion in the bulk can be brought to the form\footnote{Note that in transverse-traceless gauge, the equations simply take the form $\frac 1 2 \Box h_{ab} - \Lambda h_{ab} = 0$.}
\begin{align}
\frac 1 2 \Box h_{ab} + \frac 1 2 g_{ab} \nabla^d \nabla^c h_{cd} - \frac 1 2 g_{ab} \nabla_d \nabla^d h^{c}_{c} + \frac 1 2 \nabla_a \nabla_b h^c_c - \frac 1 2 \nabla^c \nabla_a h_{bc}- \frac 1 2 \nabla^c \nabla_b h_{ac} + (d-1) \Lambda (h_{ab} - \frac 1 2 g_{ab} h^c_c) = 0.
\end{align}
We will start by computing the terms one by one, everthing with indices upstairs, though. The calculations are done by expanding the derivatives with xAct and then continuing by hand.
\begin{align}
(\Box h)^{\mu\mu} &= 2 \Gamma^{\mu}_{ij} \Gamma^\mu_{kl} g^{ik} h^{jl} \\
&= 2 \cos^2 \mu \sin^2 \mu h^c_c\\
(\Box h)^{\mu i} &= \partial_l (h^{ki} \Gamma^\mu_{jk} g^{lj}) + \Gamma^\mu_{jk} (\tilde \nabla^l h^{ik}) + \Gamma^i_{ml} g^{jl} \Gamma^\mu_{kj} h^{mk} - \Gamma^k_{kl} g^{lm} \Gamma^\mu_{mj} h^{ij}\\
(\Box h)^{ij} &= 
\end{align}
\begin{align}
\nabla^d \nabla^c h_{cd} &= \tilde \nabla_i \tilde \nabla_j h^{ij} + \partial_\mu(\Gamma^\mu_{ij} h^{ij}) + h^{ij} \Gamma^\mu_{ij} \Gamma^\mu_{\mu\mu} \\
&= \tilde \nabla_i \tilde \nabla_j h^{ij} + \sin \mu \partial_\mu (\cos \mu  h^c_c)
\end{align}
\begin{align}
\nabla_i \nabla_j h^c_c &= \tilde \nabla_i \tilde \nabla_j h^c_c - \Gamma^\mu_{ij} \partial_\mu h^c_c\\
&=\tilde \nabla_i \tilde \nabla_j h^c_c - \cos(\mu)\sin(\mu) g_{ij} \partial_\mu h^c_c\\
\nabla_\mu \nabla_\mu h^{c}_c &= \partial_\mu \partial_\mu h^c_c - \Gamma^\mu_{\mu\mu} \partial_\mu h^c_c.
\end{align}
\begin{align}
\nabla^i \nabla^j h^M_M &= \tilde \nabla^i \tilde \nabla^j h^M_M + \Gamma^i_{k\mu} g^{kj} g^{\mu\mu} \partial_\mu h^M_M\\
\nabla^\mu \nabla^\mu h^{M}_M &= g^{\mu\mu} \partial_\mu(g^{\mu\mu} \partial_\mu h^M_M) + g^{\mu\mu}g^{\mu\mu} \Gamma^\mu_{\mu\mu} \partial_\mu h^M_M
\end{align}
\begin{align}
\nabla_c \nabla^\mu h^{\mu c} = \Gamma^\mu_{cf} g^{\mu\mu} (\partial_\mu h^{fc} + \Gamma^c_{\mu e} h^{fe} + \Gamma^f_{\mu d} h^{dc}) + \Gamma^\mu_{cf} g^{fe} \Gamma^\mu_{ed} h^{dc}
\end{align}
The $\mu\mu$ component of the equations of motion thus becomes
\begin{align}
\frac 1 2 \Box h^{\mu\mu} + \frac 1 2 g^{\mu\mu} \nabla^M \nabla^N h_{MN} - \frac 1 2 g^{\mu\mu} \nabla_M \nabla^M h^N_N + \frac 1 2 \nabla^\mu \nabla^\mu h^M_M - \nabla^M \nabla^\mu h^\mu_M - \frac{\Lambda}{2}(d-1) g^{\mu\mu} h^M_M = 0. 
\end{align}

The goal now is to impose on-shell-ness of the gravitons the $d$-dimensional slice at the location of the brane (these are the zero-modes) and solve for the behavior in $\mu$ direction, after we have added a source term at the boundary.

\subsection{More detailed description about how holographic works}
everything is commented out.
 %Nonetheless, we expect that the CFT on the dynamical theory can still be treated as approximately holographic. On the asymptotic AdS boundary the Dirichlet boundary condition plays the role of sources while on the Planck brane, sources are the Neumann boundary conditions. \dn{Not sure anymore whether the story is that simple} \vc{I'm not sure how Neumann boundary conditions can be treated as sources. (For Dirichlet, we can say that a source is given by the coefficient of the $\sim z^{d-\Delta}$ part of a bulk field fixed by the Dirichlet boundary conditions; For Neumann boundary conditions, we don't to fix the $\sim z^{d-\Delta}$ asymptotic value of the fields, so I'm not sure what to equate the source to.)}

% At any fixed time-slice, we can separate the low lying operator spectrum into two pieces. There are light bulk-CFT operators which are dual to the fields away from the brane. Their dimensions are almost the same as those of the CFT without the defect. Secondly, there are defect operators, which describe excitations close to the brane. This set of operators will again almost look like the operator spectrum of a holographic CFT. \dn{$G_{eff}$ characterizes the ``number of channels'' between the brane CFT and the asymptotic CFT. This is fairly obvious from the gravitational perspective: Large $G_{eff}$ corresponds to a large overlap of the graviton wavefunction with modes away from the brane, \ie the effective brane theory loses information into the bath through a non-local(?) effect. What does this correspond to in the boundary perspective?}

%\dn{Can we make an argument, where we approximate the correlator by the length of a geodesic and relate this to the boundary entropy? For example in 2D, the boundary entropy should scale with be $\tilde c_T$ and }

% \dn{Preliminary:} We can be a bit more precise: Consider the CFT correlator
%\begin{align}
%\langle \mathcal O_\mt{bulk} \mathcal O_\mt{bdry} \rangle \sim \left(\frac{c_T}{\tilde c_T}\right)^{m/2}.
%\end{align}
%The scaling can be heuristically motivated as follows. At large $N$, a single trace operator $\mathcal O$ creates a single particle state with unit amplitude. The state consists of some power, $(\tilde c_T)^{m/2}$, of the degrees of freedom with equal probability. If $c_T$ is much smaller than $\tilde c_T$, then 

%For the stress energy two-point function, this indicates that 
%\begin{align}
%\langle T_{brane} T_{bath} \rangle \sim \sqrt{c_{bulk} c_{bdry}} \frac{c_{bulk}}{c_{bdry}}.
%\end{align}
%Holographically, this becomes  
%\begin{align}
%\langle g_{bulk} g_{bulk} \rangle \sim \sqrt{\frac{1}{G_\mt{bulk} G_\mt{eff}}}  \frac{G_\mt{eff}}{G_\mt{bulk}},
%\end{align}
%where the propagator on the left is the bulk-to-brane propagator. Note that this the correlator between graviton bound state and bulk gravitons must be much smaller.

%Again, we can remove the first square-root by choosing the canonical normalization of the two-point function. The second factor is the coupling between the gravity theory and the bath. It would be interesting if we could derive this using more detailed CFT methods.  



\bibliography{bibliography}
\bibliographystyle{utphys}

\end{document}


%%% JUNK %%%%%

\dn{Extras for version 2}
Extras:
\begin{enumerate}
\item \rcm{intersection of brane and boundary is not a manifold??} 
\item \dn{Flat limit?}
\item \dn{de Sitter?}
\item \dn{Ensemble averaging!}
%\item {\it What to do with:} \rcm{Drop} While we specify the position and shape of the entangling surface in the asymptotic boundary, the position and shape of the RT surface as it crosses the brane, \ie the quantum extremal surface, is determined dynamically. There is a tension with \cite{Myers:2013lva}.
\end{enumerate}


\subsection{Old stuff}
\dn{Remove this subsection if not needed anymore}

\rcm{Tell the story of three descriptions of the same system; make the parallel with the three descriptions in \cite{Almheiri:2019hni}; old words:} There are two sides to the brane story that we will consider in this section. In the first perspective (section \ref{bulkgravity}) we view the brane as properly lying in the ($d+1$)-dimensional spacetime. We can use the Israel junction conditions to relate the bulk stress-energy tensor across the brane to the discontinuity of the extrinsic curvature across the brane, which determines the location of the brane as embedded in higher dimensional space. The second perspective (section \ref{inducedgravity}) applies the Randall-Sundrum technique to realize an effective action on the brane in which we are left only with a $d$-dimensional theory. We see the consistency between the pictures by observing how the equations of motion fix the brane position in the ambient spacetime.


\rcm{mention:} Our Planck brane is in the middle of the space and so there is no confusion about whether or not these degrees of freedom belong to the boundary or the bulk. It's the bulk!

\rcm{mention:} brane theory as an `ensemble', \ie not a CFT with a cutoff but an ensemble of CFTs \dn{So far I still disagree. I don't think that there are Randall-Sundrum modes for sources of scalar fields. The sources in the RS case correspond to the value of the Neumann BC of the non-gravitational fields towards the brane, which (at least in the $\mathbb Z_2$ orbifold) needs to be fixed.}

\rcm{Move the discussion of cutoff from section \ref{wormy} here:} Due to the large number of matter degrees of freedom, the formation of bubbles can occur above the Planck scale $A\sim G_\eff$. However, for the same reason, it becomes questionable whether the semiclassical description of gravity does in fact hold down to the Planck scale. Roughly speaking, $c$-many matter fields should backreact $c$-times as strongly: backreactions which are ordinarily suppressed by $G_\eff$ are now only suppressed by $c G_\eff$. In terms of Feynman diagrams, while loop order is ordinarily suppressed by $G_\eff$, matter loops now carry factors of $c$. This seems to suggest that with a large $c$ number of matter degrees of freedom, gravity can only be treated semiclassically down to scales of order\footnote{For related discussions, see \cite{Dvali:2007hz,Dvali:2007wp,Reeb:2009rm}. \vc{``Rob should take a look at them in more detail -- in particular, \cite{Dvali:2007wp}."}} $c G_\eff$ rather than $G_\eff$. Indeed, the exchange in dominance between `classical' area term $A/4G_\eff$ and the quantum entropy $S_\ren$ at scales $A\sim c G_\eff$ appears to support this claim. However, as argued in the previous paragraph, the formation of bubbles is not possible at scales $A\gtrsim c G_\eff$. In fact, the bubble size found in \eqref{eq:cookie} lies well below this scale. Hence, it would seem that the formation of bubbles is a mere by-product of attempting to probe length scales below the range of validity of semiclassical gravity on the brane.




%%% Local Variables:
%%% mode: latex
%%% TeX-master: t
%%% End:
