\documentclass[12pt]{article}
\def\fnote#1#2{\begingroup\def\thefootnote{#1}\footnote{#2}\addtocounter
{footnote}{-1}\endgroup}
\def\BM#1{\mbox{\boldmath{$#1$}}}

%\usepackage{amsmath,amsthm,amscd,amssymb}
\usepackage[colorlinks=true
,breaklinks=true
,urlcolor=blue
,anchorcolor=blue
,citecolor=blue
,filecolor=blue
,linkcolor=blue
,menucolor=blue
,linktocpage=true]{hyperref}
\hypersetup{
bookmarksopen=true,
bookmarksnumbered=true,
bookmarksopenlevel=10
}
\usepackage[noBBpl,sc]{mathpazo}
\linespread{1.05}
\usepackage[papersize={6.9in, 10.0in}, left=.5in, right=.5in, top=.6in, bottom=.9in]{geometry}
\sloppy
\raggedbottom

% these include amsmath and that can cause trouble in older docs.
\makeatletter
\@ifpackageloaded{amsmath}{}{\RequirePackage{amsmath}}

\DeclareFontFamily{U}  {cmex}{}
\DeclareSymbolFont{Csymbols}       {U}  {cmex}{m}{n}
\DeclareFontShape{U}{cmex}{m}{n}{
    <-6>  cmex5
   <6-7>  cmex6
   <7-8>  cmex6
   <8-9>  cmex7
   <9-10> cmex8
  <10-12> cmex9
  <12->   cmex10}{}

\def\Set@Mn@Sym#1{\@tempcnta #1\relax}
\def\Next@Mn@Sym{\advance\@tempcnta 1\relax}
\def\Prev@Mn@Sym{\advance\@tempcnta-1\relax}
\def\@Decl@Mn@Sym#1#2#3#4{\DeclareMathSymbol{#2}{#3}{#4}{#1}}
\def\Decl@Mn@Sym#1#2#3{%
  \if\relax\noexpand#1%
    \let#1\undefined
  \fi
  \expandafter\@Decl@Mn@Sym\expandafter{\the\@tempcnta}{#1}{#3}{#2}%
  \Next@Mn@Sym}
\def\Decl@Mn@Alias#1#2#3{\Prev@Mn@Sym\Decl@Mn@Sym{#1}{#2}{#3}}
\let\Decl@Mn@Char\Decl@Mn@Sym
\def\Decl@Mn@Op#1#2#3{\def#1{\DOTSB#3\slimits@}}
\def\Decl@Mn@Int#1#2#3{\def#1{\DOTSI#3\ilimits@}}

\let\sum\undefined
\DeclareMathSymbol{\tsum}{\mathop}{Csymbols}{"50}
\DeclareMathSymbol{\dsum}{\mathop}{Csymbols}{"51}

\Decl@Mn@Op\sum\dsum\tsum

\makeatother

\makeatletter
\@ifpackageloaded{amsmath}{}{\RequirePackage{amsmath}}

\DeclareFontFamily{OMX}{MnSymbolE}{}
\DeclareSymbolFont{largesymbolsX}{OMX}{MnSymbolE}{m}{n}
\DeclareFontShape{OMX}{MnSymbolE}{m}{n}{
    <-6>  MnSymbolE5
   <6-7>  MnSymbolE6
   <7-8>  MnSymbolE7
   <8-9>  MnSymbolE8
   <9-10> MnSymbolE9
  <10-12> MnSymbolE10
  <12->   MnSymbolE12}{}

\DeclareMathSymbol{\downbrace}    {\mathord}{largesymbolsX}{'251}
\DeclareMathSymbol{\downbraceg}   {\mathord}{largesymbolsX}{'252}
\DeclareMathSymbol{\downbracegg}  {\mathord}{largesymbolsX}{'253}
\DeclareMathSymbol{\downbraceggg} {\mathord}{largesymbolsX}{'254}
\DeclareMathSymbol{\downbracegggg}{\mathord}{largesymbolsX}{'255}
\DeclareMathSymbol{\upbrace}      {\mathord}{largesymbolsX}{'256}
\DeclareMathSymbol{\upbraceg}     {\mathord}{largesymbolsX}{'257}
\DeclareMathSymbol{\upbracegg}    {\mathord}{largesymbolsX}{'260}
\DeclareMathSymbol{\upbraceggg}   {\mathord}{largesymbolsX}{'261}
\DeclareMathSymbol{\upbracegggg}  {\mathord}{largesymbolsX}{'262}
\DeclareMathSymbol{\braceld}      {\mathord}{largesymbolsX}{'263}
\DeclareMathSymbol{\bracelu}      {\mathord}{largesymbolsX}{'264}
\DeclareMathSymbol{\bracerd}      {\mathord}{largesymbolsX}{'265}
\DeclareMathSymbol{\braceru}      {\mathord}{largesymbolsX}{'266}
\DeclareMathSymbol{\bracemd}      {\mathord}{largesymbolsX}{'267}
\DeclareMathSymbol{\bracemu}      {\mathord}{largesymbolsX}{'270}
\DeclareMathSymbol{\bracemid}     {\mathord}{largesymbolsX}{'271}

\def\horiz@expandable#1#2#3#4#5#6#7#8{%
  \@mathmeasure\z@#7{#8}%
  \@tempdima=\wd\z@
  \@mathmeasure\z@#7{#1}%
  \ifdim\noexpand\wd\z@>\@tempdima
    $\m@th#7#1$%
  \else
    \@mathmeasure\z@#7{#2}%
    \ifdim\noexpand\wd\z@>\@tempdima
      $\m@th#7#2$%
    \else
      \@mathmeasure\z@#7{#3}%
      \ifdim\noexpand\wd\z@>\@tempdima
        $\m@th#7#3$%
      \else
        \@mathmeasure\z@#7{#4}%
        \ifdim\noexpand\wd\z@>\@tempdima
          $\m@th#7#4$%
        \else
          \@mathmeasure\z@#7{#5}%
          \ifdim\noexpand\wd\z@>\@tempdima
            $\m@th#7#5$%
          \else
           #6#7%
          \fi
        \fi
      \fi
    \fi
  \fi}

\def\overbrace@expandable#1#2#3{\vbox{\m@th\ialign{##\crcr
  #1#2{#3}\crcr\noalign{\kern2\p@\nointerlineskip}%
  $\m@th\hfil#2#3\hfil$\crcr}}}
\def\underbrace@expandable#1#2#3{\vtop{\m@th\ialign{##\crcr
  $\m@th\hfil#2#3\hfil$\crcr
  \noalign{\kern2\p@\nointerlineskip}%
  #1#2{#3}\crcr}}}

\def\overbrace@#1#2#3{\vbox{\m@th\ialign{##\crcr
  #1#2\crcr\noalign{\kern2\p@\nointerlineskip}%
  $\m@th\hfil#2#3\hfil$\crcr}}}
\def\underbrace@#1#2#3{\vtop{\m@th\ialign{##\crcr
  $\m@th\hfil#2#3\hfil$\crcr
  \noalign{\kern2\p@\nointerlineskip}%
  #1#2\crcr}}}

\def\bracefill@#1#2#3#4#5{$\m@th#5#1\leaders\hbox{$#4$}\hfill#2\leaders\hbox{$#4$}\hfill#3$}

\def\downbracefill@{\bracefill@\braceld\bracemd\bracerd\bracemid}
\def\upbracefill@{\bracefill@\bracelu\bracemu\braceru\bracemid}

\DeclareRobustCommand{\downbracefill}{\downbracefill@\textstyle}
\DeclareRobustCommand{\upbracefill}{\upbracefill@\textstyle}

\def\upbrace@expandable{%
  \horiz@expandable
    \upbrace
    \upbraceg
    \upbracegg
    \upbraceggg
    \upbracegggg
    \upbracefill@}
\def\downbrace@expandable{%
  \horiz@expandable
    \downbrace
    \downbraceg
    \downbracegg
    \downbraceggg
    \downbracegggg
    \downbracefill@}

\DeclareRobustCommand{\overbrace}[1]{\mathop{\mathpalette{\overbrace@expandable\downbrace@expandable}{#1}}\limits}
\DeclareRobustCommand{\underbrace}[1]{\mathop{\mathpalette{\underbrace@expandable\upbrace@expandable}{#1}}\limits}

\makeatother


\usepackage[small]{titlesec}
\usepackage{cite}

%\usepackage[dotinlabels]{titletoc}
%\titlelabel{{\thetitle}.\quad}
%\usepackage{titletoc}
\usepackage[small]{titlesec}

\titleformat{\section}[block]
  {\fillast\medskip}
  {\bfseries{\thesection. }}
  {1ex minus .1ex}
  {\bfseries}
 
\titleformat*{\subsection}{\itshape}
\titleformat*{\subsubsection}{\itshape}

\setcounter{tocdepth}{2}

\titlecontents{section}
              [2.3em] 
              {\bigskip}
              {{\contentslabel{2.3em}}}
              {\hspace*{-2.3em}}
              {\titlerule*[1pc]{}\contentspage}
              
\titlecontents{subsection}
              [4.7em] 
              {}
              {{\contentslabel{2.3em}}}
              {\hspace*{-2.3em}}
              {\titlerule*[.5pc]{}\contentspage}

% hopefully not used.           
\titlecontents{subsubsection}
              [7.9em]
              {}
              {{\contentslabel{3.3em}}}
              {\hspace*{-3.3em}}
              {\titlerule*[.5pc]{}\contentspage}
%\makeatletter
\renewcommand\tableofcontents{%
    \section*{\contentsname
        \@mkboth{%
           \MakeLowercase\contentsname}{\MakeLowercase\contentsname}}%
    \@starttoc{toc}%
    }
\def\@oddhead{{\scshape\rightmark}\hfil{\small\scshape\thepage}}%
\def\sectionmark#1{%
      \markright{\MakeLowercase{%
        \ifnum \c@secnumdepth >\m@ne
          \thesection\quad
        \fi
        #1}}}
        
\makeatother

%\makeatletter

 \def\small{%
  \@setfontsize\small\@xipt{13pt}%
  \abovedisplayskip 8\p@ \@plus3\p@ \@minus6\p@
  \belowdisplayskip \abovedisplayskip
  \abovedisplayshortskip \z@ \@plus3\p@
  \belowdisplayshortskip 6.5\p@ \@plus3.5\p@ \@minus3\p@
  \def\@listi{%
    \leftmargin\leftmargini
    \topsep 9\p@ \@plus3\p@ \@minus5\p@
    \parsep 4.5\p@ \@plus2\p@ \@minus\p@
    \itemsep \parsep
  }%
}%
 \def\footnotesize{%
  \@setfontsize\footnotesize\@xpt{12pt}%
  \abovedisplayskip 10\p@ \@plus2\p@ \@minus5\p@
  \belowdisplayskip \abovedisplayskip
  \abovedisplayshortskip \z@ \@plus3\p@
  \belowdisplayshortskip 6\p@ \@plus3\p@ \@minus3\p@
  \def\@listi{%
    \leftmargin\leftmargini
    \topsep 6\p@ \@plus2\p@ \@minus2\p@
    \parsep 3\p@ \@plus2\p@ \@minus\p@
    \itemsep \parsep
  }%
}%
\def\open@column@one#1{%
 \ltxgrid@info@sw{\class@info{\string\open@column@one\string#1}}{}%
 \unvbox\pagesofar
 \@ifvoid{\footsofar}{}{%
  \insert\footins\bgroup\unvbox\footsofar\egroup
  \penalty\z@
 }%
 \gdef\thepagegrid{one}%
 \global\pagegrid@col#1%
 \global\pagegrid@cur\@ne
 \global\count\footins\@m
 \set@column@hsize\pagegrid@col
 \set@colht
}%

\def\frontmatter@abstractheading{%
\bigskip
 \begingroup
  \centering\large
  \abstractname
  \par\bigskip
 \endgroup
}%

\makeatother

%\DeclareSymbolFont{CMlargesymbols}{OMX}{cmex}{m}{n}
%\DeclareMathSymbol{\sum}{\mathop}{CMlargesymbols}{"50}

\begin{document}

\raggedbottom


\hfill{UTTG-01-16 }


\vspace{36pt}


\begin{center}
{\large {\bf {What Happens in a Measurement?}}}


\vspace{36pt}
Steven Weinberg\fnote{*}{Electronic address:
weinberg@physics.utexas.edu}\\
{\em Theory Group, Department of Physics, University of
Texas\\
Austin, TX, 78712}


\vspace{30pt}

\noindent
{\bf Abstract}
\end{center}

\vspace{10pt}

\noindent
It is assumed that in a measurement the system  under study interacts with a macroscopic measuring apparatus, in such a way that the density matrix of the measured system  evolves according to the Lindblad equation.  Under an assumption of non-decreasing von Neumann entropy, conditions on the operators appearing in this equation are given that are necessary and sufficient for the late-time limit of the density matrix to take the form appropriate for a measurement. Where these conditions are satisfied, the Lindblad equation can be solved explicitly.  The probabilities appearing in the late-time limit of this general solution are found to agree with the Born rule, and are independent of the details of the operators in the Lindblad equation.


\vfill

\pagebreak


\pdfbookmark{INTRODUCTION}{I}

\begin{center}
{\bf I. Introduction}
\end{center}		

According to the Copenhagen interpretation of quantum mechanics, during a complete measurement the initial density matrix $\rho_{\rm initial}$ undergoes a collapse 
\begin{equation}
\rho_{\rm initial}\mapsto \rho_{\rm final}=\sum_\alpha p_\alpha\Lambda_\alpha
\end{equation}
where $\Lambda_\alpha=|\alpha\rangle\langle\alpha|$ are projection operators onto the complete set of orthonormal eigenvectors $|\alpha\rangle$  of whatever is being measured, satisfying the usual conditions
\begin{equation}
\Lambda_\alpha\Lambda_\beta=\delta_{\alpha\beta}\Lambda_\alpha~~~~~~~\sum_\alpha\Lambda_\alpha=\BM{1}~~~~~~~{\rm Tr}\Lambda_\alpha=1~~~~~\Lambda_\alpha^\dagger=\Lambda_\alpha\;,
\end{equation}
and $p_\alpha$ are probabilities, given by the Born rule 
\begin{equation}
p_\alpha=\langle\alpha|\rho_{\rm initial}|\alpha\rangle={\rm Tr}\Big(\Lambda_\alpha \rho_{\rm initial}\Big)\;.
\end{equation}
In a closed system in ordinary quantum mechanics the state vector evolves unitarily and deterministically, so as well known the collapse (1) cannot occur if the initial density matrix $\rho_{\rm initial}$ describes a pure state (or an ensemble of fewer pure states than the number of terms in $\rho_{\rm final}$).  In the original formulation[1] of the Copenhagen interpretation it was simply accepted that the change in a system during measurement in principle  departs from quantum mechanics.  We will instead adopt the popular modern  view that  the Copenhagen interpretation refers to open systems in which the transition (1) is driven by the interaction of the microscopic system under study with a suitable environment, a macroscopic external measuring apparatus (which may include an observer) chosen to bring this transition about.

Of course, this view of the Copenhagen interpretation just pushes  the hard problems of interpreting quantum mechanics to a  larger scale.  We make no attempt to address these problems in the present paper, beyond noting the conjecture [2] that  the unitary evolution of microscopic systems is merely a very good approximation, while the density matrix of combined systems with macroscopic parts in general evolves rapidly  and non-unitarily, and in particular undergoes the collapse (1) during a measurement.   Although in this paper we are focusing on 
ordinary quantum mechanics in an open system,  most of our analysis applies equally to closed systems in modified versions of quantum mechanics.





Whether in  open systems in ordinary quantum mechanics or in closed systems in some modified version of quantum mechanics,  in order to avoid instantaneous communication at a distance in entangled states, it is important to require that the density matrix at one time depends  on the density matrix at any earlier time, but not otherwise on  the state vector at the earlier time [3].  This evolution can be linear but non-unitary in ordinary quantum mechanics if the system under study interacts with an environment that fluctuates randomly more rapidly than the rate at which the density matrix evolves (set by the interaction strength) if we average over these fluctuations.   We do not need to go into details regarding this interaction with the environment, because it is known that in the most general linear evolution that preserves the unit trace and Hermiticity of the density matrix and satisfies the condition of complete positivity[4], the density matrix satisfies the Lindblad equation [5]:
\begin{equation}
\frac{d\rho(t)}{dt}=-i\Big[{\cal H},\rho(t)\Big]+\sum_n \left(L_n\rho(t)L_n^\dagger-\frac{1}{2}L_n^\dagger L_n \rho(t)
-\frac{1}{2}\rho(t) L_n^\dagger L_n\right)\;,
\end{equation}
with constant matrices\fnote{**}{We limit the considerations of this paper to a Hilbert space of a finite dimensionality $d$.  Presumably they can be extended to infinite dimensional spaces, on which ${\cal H}$ and the $L_n$ act as suitably defined operators.} $L_n$ and ${\cal H}$.  
We then face three questions:
\begin{enumerate}
  \item What are the necessary conditions on the operators  $L_n$  and ${\cal H}$ for  the density matrix to approach a time-independent linear combination such as (1) of specific projection operators $\Lambda_\alpha$ at late times?
  \item  Are these conditions sufficient?  
  \item For such $L_n$  and ${\cal H}$, are the coefficients $p_\alpha$ of the $\Lambda_\alpha$ in this linear combination given by the Born rule (3)? 
 \end{enumerate}  
The answer to the first question is given in Section II, under the assumption that the $L_n$ satisfy the necessary and sufficient condition[6] that the von Neumann entropy $-{\rm Tr}(\rho\ln\rho)$ should never decrease:
\begin{equation}
\sum_n L_n^\dagger L_n=\sum_n L_n L^\dagger_n\;.
\end{equation}
The second and third questions are answered in Section III, where we give a general solution of the Lindblad equation, under the conditions found in Section II.


\newpage

\pdfbookmark{NECESSARY CONDITIONS FOR A MEASUREMENT}{II}

\begin{center}
{\bf II. Necessary Conditions for a Measurement}
\end{center}

First, let us consider some general aspects of the late-time behavior of the solutions of the Lindblad equation (4), without yet specializing to $L_n$ satisfying Eq.~(5).  Because Eq.~(4) is linear with time-independent coefficients, it has solutions that are generically of the form
\begin{equation}
\rho(t)=\sum_k v_k \exp\Big(\lambda_k t\Big)\;,
\end{equation}
where $v_k$ and $\lambda_k$ are the eigenmatrices and eigenvalues of the operator ${\cal L}$ in Eq.~(4):
\begin{equation}
{\cal L}v_k=\lambda_k v_k\;,
\end{equation}
\begin{equation}
{\cal L}v\equiv -i\Big[{\cal H},v\Big]+\sum_n \left(L_n\,v\,L_n^\dagger-\frac{1}{2}L_n^\dagger L_n\, v
-\frac{1}{2}v\, L_n^\dagger L_n\right)\;,
\end{equation}
with the normalization of each $v_k$ in Eq.~(6) of course depending on initial conditions.
(It is only  for the non-degenerate case that the solution of Eq.~(4) necessarily takes the form (6); if an eigenvalue $\lambda_k$ has an $N$-fold degeneracy, then $\exp(\lambda_kt)$ may be accompanied with a polynomial in $t$ of order up to $N-1$.)  Because ${\cal L}$ is in general not Hermitian the eigenvalues may be complex, and  the individual $v_k$ need not be Hermitian or positive, though the sum (6) must be both Hermitian and positive, 

Even if there are  eigenvalues $\lambda_k$ with positive-definite real parts, such terms cannot contribute to the sum (6).  If they did contribute then the sum of such terms would dominate $\rho(t)$  at late times. But ${\rm Tr}\rho(t)$ must remain constant, so the sum of terms with ${\rm Re}\lambda_k>0$ would have to be traceless.  Also, $\rho(t)$ must remain Hermitian and positive, so the sum of terms with ${\rm Re}\lambda_k>0$ would have to be Hermitian and positive. But then the eigenvalues of this sum would have to be real and positive and add up to zero, which is impossible unless all the eigenvalues vanish, in which case the sum vanishes.  The  same argument rules out any contribution of powers of time for any eigenvalues with ${\rm Re}\lambda_k=0$.  So we conclude that the asymptotic behavior of $\rho(t)$ is dominated by the sum of $v_k\exp(\lambda_k t)$ over all eigenmatrices with ${\rm Re}\lambda_k=0$, if there are any.  

In fact, as required by the constancy of the trace, there always is at least one eigenmatrix with $\lambda_k=0$.  We can think of ${\cal L}$ as a $d^2\times d^2$ matrix, acting on the space of $d\times d$ matrices.  Because Eq.~(4) preserves the trace of $\rho$, the unit $d\times d$ matrix ${\BM 1}$ is a left eigenvector of ${\cal L}$ with eigenvalue zero, so ${\rm Det}{\cal L}=0$, and therefore ${\cal L}$ also has a right eigenvector (not necessarily the unit matrix) with eigenvalue zero.  But in general there may be several  $v_k$ with $\lambda_k=0$.

In order to separate the real and imaginary parts of general eigenvalues, let us consider the quantity
\begin{equation}
{\rm Tr}\Big(v_k^\dagger v_k\Big)\,\lambda_k= {\rm Tr}\Big(v_k^\dagger {\cal L}v_k\Big)\;.
\end{equation}
A straightforward calculation gives
\begin{align}
{\rm Tr}\Big(v_k^\dagger v_k\Big){\rm Re}\lambda_k &=-\frac{1}{2}{\rm Tr}\left(\sum_n [v_k\,,\,L_n^\dagger]^\dagger[v_k\,,\,L_n^\dagger]\right)\nonumber\\
&\quad\quad-\frac{1}{2}{\rm Tr}\left(v_k v_k^\dagger \sum_n\Big(L_n^\dagger L_n-L_n L_n^\dagger\Big)\right)\\
{\rm Tr}\Big(v_k^\dagger v_k\Big){\rm Im}\lambda_k &=-{\rm Tr}\Big(v_k^\dagger [{\cal H},v_k]\Big)
+{\rm Im}{\rm Tr}\sum_n L_nv_k^\dagger[v_k,L_n^\dagger ]\end{align}
(See Appendix A.)  
It is difficult to make further progress without invoking some assumption that limits the nature of the $L_n$.  As mentioned in Sec. I, we shall assume that the $L_n$ satisfy the necessary and sufficient condition (5) for non-decreasing entropy.  In this case, Eq.~(10) simplifies to 
\begin{equation}
{\rm Tr}\Big(v_k^\dagger v_k\Big){\rm Re}\lambda_k=-\frac{1}{2}{\rm Tr}\left(\sum_n [v_k\,,\,L_n^\dagger]^\dagger[v_k\,,\,L_n^\dagger]\right)\;.
\end{equation}
We see immediately that  the real parts of all $\lambda_k$ are negative or zero.  The behavior of $\rho(t)$ for $t\rightarrow\infty$ is then dominated by the modes $v_k$ for which ${\rm Re}\lambda_k=0$, for which according to Eq. (12) $v_k$ must commute with all $L^\dagger_n$, and hence with all $L_n$.  (By taking the adjoint of Eq.~(7) we see that if $v_k$ is an eigenmatrix of ${\cal L}$ then so is $v_k^\dagger$, which must appear in (6) along with $v_k$ to keep $\rho$ Hermitian.  The adjoint of the condition that $v_k^\dagger$ commutes with $L_n^\dagger$ tells us that $v_k$ must commute with $L_n$.)  

Also, for such modes Eq.~(11) gives
\begin{equation}{\rm Tr}\Big(v_k^\dagger v_k\Big){\rm Im}\lambda_k=-{\rm Tr}\Big(v_k^\dagger [{\cal H},v_k]\Big)\;.
\end{equation}
But it must not be thought that a $v_k$ that commutes with all $L_n$ is necessarily an eigenmatrix of ${\cal L}$ with the real part of the eigenvalue zero and its imaginary part of first order in ${\cal H}$.    With $L_n$ subject to Eq.~(5) and $v_k$ commuting with all $L_n$, the eigenvalue equation (7) becomes
$$\lambda_k v_k=-i[{\cal H},v_k]\;,$$
but this is impossible if the space of matrices that commute with all $L_n$ is not invariant under commutation with ${\cal H}$.  Otherwise the commutator with ${\cal H}$ in Eq.~(7) will mix the eigenmatrices $v_k$ that commute with all $L_n$ with other matrices that do not commute with some ${L_n}$, giving an eigenvalue with negative-definite real part, whose contribution vanishes for $t\rightarrow\infty$.  

Here is an example. Take $d=2$, with a single $L_n$ given by $L=\ell\sigma_3$ (which trivially satisfies Eq.~(5)), and ${\cal H}=h\sigma_1$, with $h$ real.  This $L$ commutes with the projection operators $(1\pm \sigma_3)/2$, as required in a measurement of $\sigma_3$, but for $h\neq 0$ the commutator of ${\cal H}$ with these projection operators does not commute with them, so the measurment doesn't work.  We can see this in the late-time behavior of the solutions of the Lindblad equation.   In general the eigenmatrices  of ${\cal L}$ are $v_0\propto\BM{1}$, with eigenvalue zero; $v_{1}\propto\sigma_1$, with eigenvalue $-2|\ell|^2<0$; and two mixtures of $\sigma_2$ and $\sigma_3$, with eigenvalues $-|\ell|^2\pm(|\ell|^4-4h^2)^{1/2}$.  For $h=0$ there are two eigenmatrices with eigenvalue zero, which can be taken as $\BM{1}$ and $\sigma_3$, or equivalently as the projection matrices $(\BM{1}+\sigma_3)/2$ and       $(\BM{1}-\sigma_3)/2$ as needed in a measurement of $\sigma_3$; the other eigenmatrices both have eigenvalues $-2|\ell|^2$, corresponding to modes that disappear for $t\rightarrow \infty$.  On the other hand the late-time behavior of the density matrix is entirely different if $h$ is non-zero, though arbitrarily small.  In this case all eigenvalues have negative-definite real part, except the eigenvalue $\lambda_0=0$ associated with $v_0\propto\BM{1}$, and  for $t\rightarrow \infty$ the density matrix approaches the maximum entropy matrix $\BM{1}/2$, for which all probabilities are the same.   




With this background, let us now consider what happens in a measurement.  We suppose that the microscopic system under study interacts with a macroscopic measuring apparatus, in such a way that the density matrix of the microscopic system evolves according to the Lindblad equation (4), with the measuring apparatus chosen so that the matrices $L_n$ and ${\cal H}$ have whatever properties are needed so that $\rho(t)$ at late times approaches a linear combination of projection operators $\Lambda_\alpha$ on the eigenstates $|\alpha\rangle$ of whatever is being measured.  As we have seen in Eq.~(12), in order for this to be the case without putting any constraints on the initial conditions that determine the coefficients in this linear combination, it is necessary that the matrices $L_n$ should commute with any linear combination of the $\Lambda_\alpha$, and hence with each 
$\Lambda_\alpha$:
\begin{equation}[L_n,\Lambda_\alpha]=0\;,          \end{equation}
from which it follows immediately that each $L_n$ must itself be a linear combination of the $\Lambda_\alpha$:
\begin{equation}
 L_n=\sum_\alpha \ell_{n\alpha}\Lambda_\alpha\;,
\end{equation}
with coefficients $\ell_{n\alpha}$ that are in general complex numbers.
(From Eq.~(14) it follows that  the eigenstates $|\alpha\rangle$  satisfying $\Lambda_\beta|\alpha\rangle=\delta_{\alpha\beta}|\alpha\rangle$ must be eigenstates of the $L_n$:
$$ L_n|\alpha\rangle=L_n\Lambda_\alpha|\alpha\rangle=\Lambda_\alpha L_n|\alpha\rangle=|\alpha\rangle \langle \alpha|L_n|\alpha\rangle$$
Then $L_n$ has the same action on any $|\alpha\rangle$ as does the sum (15) with $\ell_{n\alpha}= \langle \alpha|L_n|\alpha\rangle$, and since the $|\alpha\rangle$ form a complete set, $L_n$ must equal the sum (15).)  From Eq.~(15) the condition (5) for non-decreasing entropy follows trivially.

This leaves us with the matrix ${\cal H}$.  As remarked earlier, in order that the limiting behavior of $\rho(t)$ for general initial conditions should be a linear combination of the $\Lambda_\alpha$, it is necessary that the space of such linear combinations should be invariant under commutation with ${\cal H}$:
$$
[{\cal H},\Lambda_\alpha]=\sum_\beta h_{\alpha\beta}\Lambda_\beta\;.
$$
By multiplying this commutator on both the left and right with any $\Lambda_\beta$, we see that $0=h_{\alpha\beta}$, and therefore ${\cal H}$ must commute with all $\Lambda_\alpha$.  By the same argument used above for the $L_n$, we see then that ${\cal H}$ must be a linear combination of the $\Lambda_\alpha$:
\begin{equation}
{\cal H}=\sum_\alpha h_\alpha\Lambda_\alpha\;,
\end{equation}
with real coefficients $h_\alpha$.

This is a good place to bring up a complication.  The late-time behavior (1) is expected only for a {\em complete} measurement.  It is more common for measurements to be incomplete, in the sense that they do not lead to definite states $|\alpha\rangle$ with definite probabilities, but to equivalence classes of  states that are not distinguished by the measurement.  For instance, in a system consisting of two spins $1/2$, we might measure only the first spin, leaving the other undisturbed.  The states then fall into two classes, labeled by the $z$-components of the two spins: one class consists of $|1/2,1/2\rangle$ and  $|1/2,-1/2\rangle$, and the other consists of $|-1/2,1/2\rangle$ and  $|-1/2,-1/2\rangle$.  In incomplete measurements, instead of (1), the expected late-time limit of the density matrix is
\begin{equation}
\rho_{\rm initial}\mapsto\rho_{\rm final}=\sum_C \Lambda_C\rho_{\rm initial}\Lambda_C \;,
\end{equation}
where 
\begin{equation}
\Lambda_C=\sum_{\alpha\in C}\Lambda_\alpha\;,
\end{equation}
As far as the states within a single class are concerned, $\Lambda_C$ acts just like a unit matrix, so Eq,~(17) says that the measurement does nothing to what is not being measured.   For a complete measurement, where each  state belongs to a different class, Eq.~(17) reduces to Eqs.~(1) and (3).

 Eq.~(12) shows that in order for $\rho(t)$ to have some given asymptotic limit $\rho_{\rm final}$, it is necessary  for all $L_n$ to commute with 
this limit, and since this must be true for all $\rho_{\rm initial}$, the $L_n$ here must in particular commute with $\sum_C\Lambda_C\Lambda_\alpha\Lambda_C=\Lambda_\alpha$.  The same argument as given above for complete measurements then shows that, here too, eash $L_n$ must be a linear combination (15) of the $\Lambda_\alpha$.  Only now there is a constraint on the coefficients.  The commutator of the sum (15) with the limit (17) is 
$$ \Big[\sum_\alpha \ell_{n\alpha} \Lambda_\alpha\,,\,\sum_C \Lambda_C\rho_{\rm initial}\Lambda_C\Big]
=\sum_C\sum_{\beta,\gamma\in C}[\ell_{n\beta}-\ell_{n\gamma}]\Lambda_\beta\rho_{\rm initial}\Lambda_\gamma$$
which vanishes for all initial density matrices if $\ell_{n\beta}=\ell_{n\gamma}$ for all $\beta$ and $\gamma$ in the same class.  The same argument shows that $h_\beta=h_\gamma$ if $\beta$ and $\gamma$ are in the same class.
This is reasonable, because for an incomplete measurement the Lindblad equation must not distinguish between different states in the same class,  We will see in the next section that in this case 
the late-time limit of the density matrix does have the form (17).

\pdfbookmark{COLLAPSE OF THE DENSITY MATRIX}{III}
\begin{center}
{\bf III. Collapse of the Density Matrix}
\end{center}

First let us give the solution of the Lindblad equation under the condition that the matrices $L_n$ and ${\cal H}$ in this equation are linear combinations (15), (16) of projection operators $\Lambda_\alpha$ satisfying Eq.~(2):
$$ 
L_n=\sum_{\alpha}\ell_{n\alpha}\Lambda_\alpha\;,~~~~~~~
{\cal H}=\sum_{\alpha}h_{\alpha}\Lambda_\alpha~~\;.$$
It is straightforward to check that Eq.~(4) is then satisfied by
\begin{equation}
\rho(t)=\sum_{\alpha\beta}\Lambda_\alpha M \Lambda_\beta\;\exp(\lambda_{\alpha\beta}t)\;,
\end{equation}
where
\begin{equation}
\lambda_{\alpha\beta}=-\frac{1}{2}\sum_n\Big|\ell_{n\alpha}-\ell_{n\beta}\Big|^2+i\;{\rm Im}\sum_n \ell_{n\alpha}\ell^*_{n\beta}
-i\Big(h_{\alpha}-h_{\beta}\Big)\;.
\end{equation}
and $M$ is an arbitrary  matrix, independent of $\alpha$, $\beta$, and time. [See Appendix B.]  To relate $M$ to the initial value of $\rho(t)$ at $t=0$, set $t=0$ in Eq.~(19) and use the completeness condition $\sum_\alpha\Lambda_\alpha=\BM{1}$.  We see that $\rho(0)=M$, and so
\begin{equation}
\rho(t)=\sum_{\alpha\beta}\Lambda_\alpha \rho(0) \Lambda_\beta\;\exp(\lambda_{\alpha\beta}t)\;,
\end{equation}
This is our general solution.[7]

Now consider the behavior of this solution at late times.  The only terms in the sum (21) that do not decay exponentially are those with $\ell_{n\alpha}=\ell_{n\beta}$ for all $n$.  
If for the moment we rule out degeneracy,  so that $\ell_{n\alpha}$ can equal $\ell_{n\beta}$ for all $n$ only for $\alpha=\beta$, then all $\lambda_{\alpha\beta}$ have negative-definite real part except those with $\alpha=\beta$, for which the imaginary as well as the real parts of $\lambda_{\alpha\alpha}$ vanish.  These terms then dominate the asymptotic behavior[8]  of the density matrix for $t\rightarrow \infty$:
\begin{equation}
\rho(t)\rightarrow \sum_\alpha \Lambda_\alpha\rho(0)\Lambda_\alpha=\sum_\alpha \Lambda_\alpha \langle \alpha|\rho(0)|\alpha\rangle
\end{equation}
This is just the behavior (1) called for by the Copenhagen interpretation, with probabilities $p_\alpha$ given by the Born rule (3).  

The case of degeneracy arises in an incomplete measurement, in which  we only measure whether the system is in some state or other in a class  of states that are not distinguished by the measurement.  As indicated at the end of the previous section, in this case we expect $\ell_{n\alpha}$ to equal $\ell_{n\beta}$ for all $n$ and $h_\beta=h_\gamma$ if (and only if) $|\alpha\rangle$ and $|\beta\rangle$ are in the same class.  Then Eq.~(21) has the expected late-time behavior (17).


It is striking that although the detailed time-dependence of the density matrix depends on the coefficients $\ell_{n\alpha}$ and $h_\alpha$ appearing in the matrices in the Lindblad equation, the asymptotic limit for $t\rightarrow\infty$ for both complete and incomplete measurements does not depend on these details, depending only on the initial condition $\rho(0)$ and on what it is that  is being measured.  
This, of course, is just what we require of a measurement. 



\vspace{20pt}

\begin{center}
{\bf Acknowledgments}
\end{center}

I am grateful for correspondence with P. Pearle, and regarding the condition for non-decreasing entropy, with H. Narnhofer,  D. Reeb, and R. Werner.  This material is based upon work supported by the National Science Foundation under Grant Number PHY-1316033 and with support from The Robert A. Welch Foundation, Grant No. F-0014.


\vspace{10pt}

\pdfbookmark{Appendix A}{A}
\begin{center}
{\bf Appendix A: Derivation of Eqs.~(10) and (11) }
\end{center}
\renewcommand{\theequation}{A.\arabic{equation}}
\setcounter{equation}{0}

We start with the desired result, and work back to the problem it solves.  For a general matrix $v$, consider the quantity 
\begin{eqnarray}
&&R\equiv-\frac{1}{2}{\rm Tr}\left(\sum_n [v\,,\,L_n^\dagger]^\dagger[v\,,\,L_n^\dagger]\right)-\frac{1}{2}{\rm Tr}\left(v v^\dagger \sum_n\Big(L_n^\dagger L_n-L_n L_n^\dagger\Big)\right)\nonumber\\
&&+i{\rm Im}{\rm Tr}\sum_n L_nv^\dagger[v,L_n^\dagger ]-i{\rm Tr}\Big(v^\dagger [{\cal H},v]\Big)
\end{eqnarray}
Expanding each term, this is
\begin{eqnarray}
&&R=-\frac{1}{2}{\rm Tr}\sum_n L_n v^\dagger v\,L_n^\dagger
+\frac{1}{2}{\rm Tr}\sum_n  v^\dagger L_n v\,L_n^\dagger
+\frac{1}{2}{\rm Tr}\sum_n L_n v^\dagger L_n^\dagger\,v
-\frac{1}{2}{\rm Tr}\sum_n v^\dagger L_n  \,L_n^\dagger\,v\nonumber\\&&
-\frac{1}{2}{\rm Tr}\sum_n v v^\dagger  L_n^\dagger L_n
+\frac{1}{2}{\rm Tr} \sum_n v v^\dagger  L_n L^\dagger_n\nonumber\\
&&+\frac{1}{2}{\rm Tr}\sum_n v^\dagger L_n v  L_n^\dagger-\frac{1}{2}{\rm Tr}\sum_n L_n v^\dagger L_n^\dagger v\nonumber\\&&
-i{\rm Tr}\Big(v^\dagger [{\cal H},v]\Big)\;.
\end{eqnarray}
The third and eighth terms cancel; the fourth and sixth terms cancel; the second and seventh terms add to give the term
${\rm Tr} v^\dagger\sum_n L_n v L_n^\dagger$ in ${\rm Tr}v^\dagger{\cal L}v$;  the first and fifth terms give the terms
$-{\rm Tr} v^\dagger v \sum_n  L_n^\dagger L_n /2$ and $-{\rm Tr} v^\dagger\sum_n L^\dagger_n  L_n v/2$ in ${\rm Tr}v^\dagger{\cal L}v$; and the last term gives the Hamiltonian term in ${\rm Tr}v^\dagger{\cal L}v$.  We conclude that 
\begin{equation}
R={\rm Tr}\Big(v^\dagger{\cal L}v\Big)\;.
\end{equation}
The first two terms in (A.1) are real, while the last two are imaginary, so
\begin{equation}
{\rm Re}{\rm Tr}\,v^\dagger{\cal L}v=-\frac{1}{2}{\rm Tr}\left(\sum_n [v\,,\,L_n^\dagger]^\dagger[v\,,\,L_n^\dagger]\right)-\frac{1}{2}{\rm Tr}\left(v v^\dagger \sum_n\Big(L_n^\dagger L_n-L_n L_n^\dagger\Big)\right)
\end{equation}
\begin{equation}
{\rm Im}\,{\rm Tr}\,v^\dagger{\cal L}v={\rm Im}{\rm Tr}\sum_n L_nv^\dagger[v,L_n^\dagger ]\,-{\rm Tr}\Big(v^\dagger [{\cal H},v]\Big)
\end{equation}
Taking $v$ to be one of the eigenmatrices $v_k$ of ${\cal L}$, with ${\cal L}v_k=\lambda_kv_k$ then gives Eqs.~(10) and (11).

\vspace{10pt}

\pdfbookmark{Appendix B}{B}
\begin{center}
{\bf Appendix B: Derivation of Eqs. (19) and (20) }
\end{center}
\renewcommand{\theequation}{B.\arabic{equation}}
\setcounter{equation}{0}

We try a solution of the Lindblad equation
\begin{equation}
\rho(t)=\sum_{\alpha\beta}\Lambda_\alpha M \Lambda_\beta f_{\alpha\beta}(t)\;.
\end{equation}
With $\Lambda_\alpha$ and ${\cal H}$ given by Eqs.~(15) and (16), the Lindblad equation (4) becomes
\begin{equation}`\sum_{\alpha\beta}\Lambda_\alpha M \Lambda_\beta\frac{d}{dt} f_{\alpha\beta}(t)=\sum_{\alpha\beta}\lambda_{\alpha\beta}\Lambda_\alpha M \Lambda_\beta f_{\alpha\beta}(t)\;,
\end{equation}`
where 
\begin{equation}
\lambda_{\alpha\beta}=C_{\alpha\beta}-\frac{1}{2}C_{\alpha\alpha}-\frac{1}{2}C_{\beta\beta}-i(h_\alpha-h_\beta)
\end{equation}
and
\begin{equation}
C_{\alpha\beta}=\sum_n \ell_{n\alpha}\ell_{n\beta}^*\;.
\end{equation}
This has an obvious solution of the same form as (19):
\begin{equation}
f_{\alpha\beta}(t)=\exp\Big(\lambda_{\alpha\beta}t\Big)f_{\alpha\beta}(0)\;.
\end{equation}
To get a more useful expression for $\lambda_{\alpha\beta}$, we note that 
$$-\frac{1}{2}\left|\ell_{\alpha}-\ell_{\beta}\right|^2=-\frac{1}{2}C_{\alpha\alpha}-\frac{1}{2}C_{\beta\beta}+{\rm Re}\sum_n \ell_{n\alpha}\ell_{n\beta}^*=C_{\alpha\beta}-\frac{1}{2}C_{\alpha\alpha}-\frac{1}{2}C_{\beta\beta}-i{\rm Im}\sum_n \ell_{n\alpha}\ell_{n\beta}^* $$
so Eq.~(B.3) is the same as Eq.~(20).



\vspace{10pt}

\begin{center}
{\bf ---------}
\end{center}

\vspace{10pt}

\begin{enumerate}


\item N. Bohr, Nature {\bf 121}, 580 (1928).
\item G. C. Ghirardi, A. Rimini,
and T. Weber, Phys. Rev. D {\bf 34}, 470 (1986);  P. Pearle, Phys. Rev. A {\bf 39}, 2277 (1989), and in {\em Quantum Theory: A Two-Time Success Story} (Yakir Aharonov Festschrift), eds. D. C. Struppa \& J. M. Tollakson (Springer, 2013), Chapter 9. [arXiv:1209.5082]
\item N. Gisin, Helv. Phys. Acta {\bf 62}, 363 (1989); Phys. Lett. A {\bf 143}, 1 (1990).  This is discussed in a wider context by J. Polchinski, Phys. Rev. Lett. {\bf 66}, 397 (1991).
\item  If any entangled density matrix for a compound system ${\cal S}\otimes {\cal S}$ consisting of two isolated copies of a system ${\cal S}$ remains positive for a  range of future times if it is positive at an initial time, then the linear mapping $\rho(t)\rightarrow\rho(t')$ of the density matrix of ${\cal S}$  for $t'>t$ in this range is completely positive,  as shown by F. Benatti, R. Floreanini, and R. Romano,  J. Phys. A Math. Gen.  {\bf 35}, L551 (2002).  For complete positivity see W. F. Stinnespring, Proc. Am. Math. Soc. {\bf 6}, 211 (1955); M. D. Choi, J. Canada Math. {\bf 24}, 520 (1972).  For its  implications, see  M. D. Choi, {\it Linear Algebra and its Applications} {\bf 10}, 285 (1975).  
\item G. Lindblad, Commun. Math. Phys. {\bf 48}, 119 (1976); V. Gorini, A. Kossakowski and E. C. G. Sudarshan, J. Math. Phys. {\bf 17}, 821 (1976).  For a straightforward derivation, see P. Pearle, Eur. J. Phys. {\bf 33}, 805 (2012).
\item F. Benatti and R. Narnhofer, Lett. Math. Phys. {\bf 15}, 325 (1988).  (Their result, which applies for infinite as well as finite Hilbert spaces, takes the form of an inequality.  When limited to finite Hilbert spaces, it iw equivalent to  the equality (5).)  It was earlier shown by T. Banks, M. Peskin, and L. Susskind, Nuclear Phys. B {\bf 244}, 125 (1984), that a sufficient (though not necessary) condition for non-decreasing entropy is that the $L_n$ are Hermitian.  Of course, if the $L_n$ are Hermitian then Eq.~(5) is automatically satisfied.
\item This solution was  given in the second edition of S. Weinberg, {\em Lectures on Quantum Mechanics} (Cambridge University Press, Cambridge, UK, 2015), Section 6.9, for the special case where all $L_n$ are Hermitian.
\item This behavior is seen in several of the examples presented by Pearle in ref. [5].
\end{enumerate}

  \end{document}
\begin{enumerate}
\item 
\item F. Benatti, R. Floreanini, and R. Romano, J. Phys. A Math. Gen. {\bf 35}, L351 (2002),  For complete continuity, see W. F. Stinnespring, Proc. Am. Math. Soc. {\bf 6}, 211 (1955); M. D. Choi, J. Canada Math. {\bf 24}, 520 91972).
\item M. D. Choi, {it Linear Algebra and its Applications} {\bf 10}285 (1975).
\item G. Lindblad, Commun. Math. Phys. {\bf 48}, 119 (1976); V. Gorini, A. Kossakowski and E. C. G. Sudarshan, J. Math. Phys. {\bf 17}, 821 (1976). 
\item G. C. Ghirardi, A. Rimini,
and T. Weber, Phys. Rev. D {\bf 34}, 470 (1986);  P. Pearle, Phys. Rev. A {\bf 39}, 2277 (1989), and in {\em Quantum Theory: A Two-Time Success Story} (Yakir Aharonov Festschrift), eds. D. C. Struppa \& J. M. Tollakson (Springer, 2013), Chapter 9. [arXiv:1209.5082]
\item This was added in the second edition of S. Weinberg, {\em Lectures on Quantum Mechanics} (Cambridge University Press, 
Cambridge, UK,  2015), Sec. 6.9.
\item T. Banks, M. Peskin, and L. Susskind, Nuclear Phys. B {\bf 244}, 125 (1984). A modified proof is given below.


\end{enumerate}


 
\end{document}

To avoid both the dualism of the Copenhagen interpretation and   the endless creation of inconceivably many branches of history of the many-worlds approach, while at the same time holding on to a realist description of the evolution of physical states from moment to moment,  we may try to modify quantum mechanics so that during measurement the wave function undergoes a probabilistic nonlinear collapse to a single result[4], but then a new problem may intrude.  In an entangled state in ordinary quantum mechanics  an intervention in the wave function affecting one part of a system can instantaneously affect the wave function describing a distant subsystem.  This is bad enough if we take the wave function seriously as a description of reality, but at least in ordinary quantum mechanics no measurement in one subsystem can reveal what measurement was done in an isolated subsystem.  If the wave function evolves nonlinearly it is difficult even to formulate what we mean by isolated subsystems, much less to prevent instantaneous communication between them[5].  

Gisin[5] has pointed out that if the density matrix evolves linearly (whether or not in the course of a measurement), as it does in ordinary quantum mechanics, then in an entangled state no change in the evolution of the density matrix in one isolated subsystem can affect the evolution of the density matrix in another isolated subsystem.  It therefore seems worth considering yet another interpretation of quantum mechanics:  The density matrix rather than the wave function  is to be taken as a description of reality.  

This is very different from giving the same status to the wave function, because the density matrix contains much less information.    If we know that a system is in any one of a number of normalized but not necessarily orthonormal states $\Psi_r$, with  probabilities $P_r$, then we know that the density matrix is $\rho=\sum_r P_r\Lambda_r$, where $\Lambda_r$ is the projection operator on state  
$\Psi_r$, but this does not work in reverse; for a given $\rho$, there are any number of collections of not necessarily orthonormal states and probabilities that give the same density matrix.  Instead of defining the density matrix in terms of state vectors, we may take its physical interpretation to be the postulate that the average value of any physical quantity $A$ is ${\rm Tr}(A\rho)$, which since it applies also to powers of $A$ allows us to find from the density matrix the probability distribution  for values of $A$.  These may be regarded as objective probabilities, independent of whether or not anything is actually being measured.  

The question then is whether, in some modified version of quantum mechanics, the density matrix might naturally collapse in the way called for by the Copenhagen interpretation, without attributing any special status to measurement.  This behavior of the density matrix has already been seen in the interesting modifications of quantum mechanics based on the idea of spontaneous localization[6].  The purpose of this note is to explore this sort of collapse of the density matrix in a more general context.


To avoid instantaneous communication, the density matrix at time $t'$ is assumed to be given in terms of the matrix at time $t$ by a general linear relation
\begin{equation}
\rho_{M'N'}(t')=\sum_{MN}K_{M'M,N'N}(t',t)\,\rho_{MN}(t)\;,
\end{equation}
where $K$ is some  general kernel.   We will take the indices $M$, $N$, etc. to run over a finite number $d$ of values, leaving the extension to Hilbert spaces of infinite dimensionality for future work.  Instead of a deterministic nonlinear evolution of the probability distribution of the state vector, as in ref. [4], we have here a deterministic linear evolution of the density matrix.
We  require that ${\rm Tr}\rho(t')=1$ for all $\rho(t)$ with ${\rm Tr}\rho(t)=1$, which implies that  $K$ satisfies the condition:
\begin{equation}
\sum_{M'}K_{M'M,M'N}(t',t)=\delta_{MN}\;.
\end{equation}
We also require that $\rho(t')$ should be Hermitian for all Hermitian $\rho(t)$, which implies that the kernel is  Hermitian, in the sense that
\begin{equation}
K^*_{M'M,N'N}(t',t)=K_{N'N,M'M}(t',t)
\end{equation}
Finally, we make an assumption of complete positivity[7], that all eigenvalues of 
$K$ are positive, which  which ensures that $\rho(t')$ is a positive matrix if $ \rho(t)$ is.



It is well-known in the quantum theory of open systems that under these assumptions the time-dependence of the density matrix is given by a  first-order differential equation that can be derived from the terms in Eq.~(1) up to first order in $t'-t$.  This is the Lindblad equation[8], which takes the form
\begin{equation}
\dot{\rho}(t)=[{\cal L}_9+{\cal L}_1]\,\rho(t)\;,
\end{equation}
where for any matrix $f$,
\begin{eqnarray}
&& {\cal L}_0\,f\equiv -i[H,f]\;,~~~~~~{\cal L}_1\,f\equiv\sum_\alpha \Bigg[L_\alpha\,f\,L_\alpha^\dagger
-\frac{1}{2}L^\dagger_\alpha\,L_\alpha\,f
-\frac{1}{2}f\,L_\alpha^\dagger\,L_\alpha\Bigg]\;.~~~
\end{eqnarray}
Here $H$ is a Hermitian matrix that can be identified with the Hamiltonian of ordinary quantum mechanics, and  the sum  runs over  $\leq d^2-1$ matrices $L_\alpha$, which represent the departure from ordinary quantum mechanics.  In writing Eqs.~(4) and (5) we are assuming time-translation invariance, in which case the kernel $K(t',t)$ depends only on $t'-t$, and the matrices $H$ and 
$L_\alpha$ are time-independent.  
It may be assumed that the $L_\alpha$ are negligible at the scale of atoms or molecules, where ordinary quantum mechanics applies, but are large at macroscopic scales of mass or distance, as when (but not only when) measurements are made, so that the density matrix evolves very rapidly.  But to what?

The solution of the Lindblad equation can be expressed in terms of eigenvalues $\lambda_r$ and eigenmatrices $f_r$ of ${\cal L}_0+{\cal L}_1$, satisfying $[{\cal L}_0+{\cal L}_1]f_r=\lambda_rf_r$.  In the generic case, 

We will first consider the special case where the $L_\alpha$ are Hermitian matrices.  This is the case encountered in several of the examples presented by Pearle[8], and it provides a helpful orientation for the more general problem.  The linear operator (5) in the Lindblad equation here takes the form:
\begin{equation}
{\cal L}f\equiv \sum_\alpha \Big[L_\alpha\,f\,L_\alpha
-\frac{1}{2}L_\alpha^2\,f
-\frac{1}{2}f\,L_\alpha^2\Big]=-\frac{1}{2}\sum_\alpha[L_\alpha,\,[L_\alpha\,f]]\;.
\end{equation}  
It is useful in this case to think of $ \rho(t)$ as an element of a $d^2$-dimensional Hilbert space, in which the scalar product of two $d\times d$ matrices $f$ and $g$ is taken as $(f,g)\equiv {\rm Tr}(f^\dagger g)$.  
Because the linear operator (6) is Hermitian in the sense that ${\rm Tr}(f^\dagger{\cal L} g)={\rm Tr}([{\cal L}f]^\dagger g)$, it has a complete set of $d^2$ orthogonal eigenmatrices $f_r$ with real eigenvalues $\lambda_r$, and the general solution of the Lindblad equation is therefore of the form
$\rho(t)=\sum_r f_r\exp(\lambda_r t)$.  Furthermore, ${\cal L}$ is {\em negative}, in the sense that for any matrix $f$ we have 
\begin{equation}
(f,{\cal L}f)=-\frac{1}{2}\sum_\alpha {\rm Tr}([L_\alpha,f]^\dagger [L_\alpha,f])\leq 0\;,
\end{equation}
so all eigenvalues of ${\cal L}$ have $\lambda_r\leq 0$.  The late-time behaviour will thus be dominated by the zero-modes of ${\cal L}$, those eigenmatrices $f_r$ for which $\lambda_r=0$.  Inspection of Eq.~(7) shows immediately that a necessary and sufficient condition for a matrix $f$ to be a zero-mode of ${\cal L}$ is that it should commute with all the Lindblad matrices $L_\alpha$.  There is always at least one such zero-mode, the unit matrix $\BM{1}$, and there may be others.  






There is a subcase  that perhaps represents something like what happens during a measurement.  Suppose that the $L_\alpha$ take the form $c_\alpha \Lambda_\alpha$, where $c_\alpha$ are  real coefficients, and $\Lambda_\alpha$ are the  projection matrices $[\Lambda_\alpha]_{MN}=\psi^{(\alpha)}_M \psi^{(\alpha)*}_N$ on a complete set of orthonormal state vectors $\psi^{(\alpha)}_M$, for which $\Lambda_\alpha\Lambda_\beta=\delta_{\alpha\beta}\Lambda_\alpha$ and $\sum_\alpha\Lambda_\alpha=\BM{1}$.    
The zero-modes here are the Lindblad matrices themselves, which all commute, and the solution of the Lindblad equation here is
\begin{equation}
\rho(t)=\sum_\alpha \Lambda_\alpha\rho(0)\Lambda_\alpha+\sum_{\alpha\neq \beta}e^{-(c^2_\alpha+c^2_\beta)t/2}\Lambda_\alpha\rho(0)\Lambda_\beta\;.
\end{equation}
We see that the density matrix smoothly approaches the result $\sum_\alpha P_\alpha\Lambda_\alpha$, where $P_\alpha=(\psi^{(\alpha)\dagger}\rho(0)\psi^{(\alpha)})$ is the probability postulated in the Copenhagen interpretation.  

Now let us return to the general case governed by Eqs.~(4) and (5).  In general there is no metric in which ${\cal L}$ is Hermitian, but it is still true that in the generic case without degeneracy the eigenmatrices $f_r$ of ${\cal L}$ form a complete set, in which case the general solution of the Lindblad equation is again of the form $\rho(t)=\sum_r f_r\exp(\lambda_r t)$, only now the eigenvalues $\lambda_r$ are not necessarily real.  The degenerate case can be regarded as a limit of the non-degenerate case in which some eigenvalues are continuously merged.  An exponential $\exp(\lambda_r t)$ for which $\lambda_r$ has an $n$-fold degeneracy will thus in general be accompanied with polynomial in $t$ of order $n-1$.

It is not clear whether in the general case  there are  eigenmatrices with ${\rm Re}\lambda_r>0$, which might dominate the late-time behavior of general solutions of the Lindblad equation, but we can easily see that such eigenmatrices cannot appear in the late-time behaviour of $\rho(t)$.  We are assuming that at some initial time $\rho(t)$ is Hermitian, positive and has unit trace, and the Lindblad equation then ensures that these properties are preserved for all future times.  If there were any $f_r$ with ${\rm Re}\lambda_r>0$ and with non-zero trace that contributes to $\rho(t)$, then ${\rm Tr}\rho(t)$ would grow for $t\rightarrow\infty$  at least as fast as $\exp({\rm Re}\lambda_r\,t)$, contradicting the constancy of the trace.  On the other hand, if there were any $f_r$ with ${\rm Re}\lambda_r>0$, all of which are traceless, their sum would be a traceless Hermitian matrix that dominates $ \rho(t)$ for $t\rightarrow\infty$.  Because  in this case the sum of the eigenvalues of this traceless Hermitian matrix would vanish, at least one of these eigenvalues would have to be negative,  contradicting the positivity of $\rho(t)$.

Since no eigenmatrices with ${\rm Re}\lambda_r>0$ contribute to $\rho(t)$, the asymptotic behavior of $\rho(t)$ will be dominated by eigenmatrices with ${\rm Re}\lambda_r=0$, if there are any.  In fact, there are.  In particular,
there will again always be a zero mode of ${\cal L}$, though not in general the unit matrix.  To see this, consider any complete set of $d^2$ independent $d\times d$ matrices $g_u$, and define a $d^2\times d^2$ matrix ${\cal M}_{uv}\equiv {\rm Tr}(g_u^\dagger {\cal L} g_v)$.  Since the $g_u$ are complete, there must be some coefficients $c_u$ for which $\sum_u c_u g_u=\BM{1}$, in which case $\sum_u c^*_u {\cal M}_{uv}=0$ for all $v$, because ${\rm Tr}({\cal L}g)=0$ for all $g$.  So the determinant of ${\cal M}$ vanishes, and therefore there must also be some $b_v$ for which $\sum_v {\cal M}_{uv}b_v=0$ for all $u$, which means that $ \sum_v b_v g_v$ is a zero-mode of ${\cal L}$.  

Repeating our former reasoning, in order for $\rho(t)$ to remain positive, Hermitian , and with unit trace for $t\rightarrow\infty$, the only eigenmatrices  $f_r$ of ${\cal L}$  with ${\rm Re}\lambda_r=0$ that can appear in $\rho(t)$, even if degenerate, must  not be accompanied with powers of time.  Some of these eigenmatrices may have ${\rm Im}\lambda_r\neq 0$, but these  must have zero trace to keep the trace constant for late times.  Of course, $\rho(t)$ must also receive a contribution from a zero-mode with unit trace.



For the purposes of detecting possible departures from ordinary quantum mechanics, it is interesting also to consider the case where the sum ${\cal L}_1$ over operators $L_\alpha$ in Eq.~(5) represents a small perturbation.  The Hamiltonian term ${\cal L}_0\,f\equiv -i[H,f]$ has eigenmatrices $[f^{(ab)}]_{MN}\equiv \psi^{(a)}_M\,\psi^{(b)*}_N$ with eigenvalues $\lambda_0^{(ab)}=-i(E_a-E_b)$, where $\psi^{(a)}$ are the eigenvectors of $H$ with eigenvalue $E_a$.   Since $i{\cal L}_0$ is Hermitian in the metric 
$(f,g)\equiv {\rm Tr}(f^\dagger g)$, and the matrices $f^{(ab)}$ are orthonormal in the sense that $(f^{(a'b')},f^{(ab)})=\delta_{a'a}\delta_{b'b}$, the first-order corrections to these eigenvalues are the eigenvalues of the matrix 
\begin{equation}
M_{a'b',ab}\equiv{\rm Tr}(f^{(a'b')\dagger}{\cal L}_1\,f^{(ab)})=\sum_\alpha \Bigg[[L_\alpha]_{a'a}[L_\alpha^\dagger]_{bb'}-\frac{1}{2}\delta_{a'a}[L_\alpha^\dagger\,L_\alpha]_{bb'}-\frac{1}{2}\delta_{b'b}[L_\alpha^\dagger\,L_\alpha]_{a'a}\Bigg]\;.
\end{equation}

First, consider the perturbations to the zeroth-order eigenvalues $\lambda_0^{(aa)}=0$.  These are all degenerate, so the first-order perturbations to these eigenvalues are the eigenvalues of the submatrix
\begin{equation}
M_{a'a}\equiv M_{a'a',aa}=\ell_{a'a}-\delta_{a'a}\sum_b\ell_{ba}\;,
\end{equation}
where $\ell_{ba}\equiv \sum_\alpha|[L_\alpha]_{ba}|^2$.  The non-Hermitian matrix $M_{a'a}$ has a zero eigenvalue, with a left eigenvector whose components are all equal.  The corresponding right-eigenvector $v_a$, which contributes to $[\rho(t)]_{MN}$  a time-independent term proportional to $\sum_a v_a \psi^{(a)}_M\,\psi^{(a)*}_N$, is considerably more complicated, and depends on the details of the Lindblad matrices.  For instance, in the two-state and three-state cases
$$ v=\left[\begin{array}{c}\ell_{12}\\ \ell_{21}\end{array}\right]\;,~~
v=\left[\begin{array}{c}\ell_{12}\ell_{13}+\ell_{32}\ell_{13}+\ell_{12}\ell_{23}\\
\ell_{21}\ell_{13}+\ell_{21}\ell_{23}+\ell_{23}\ell_{31}\\  \ell_{31}\ell_{12}+\ell_{31}\ell_{32}+\ell_{21}\ell_{32}\end{array}\right]\;.$$
(Note that if all $L_\alpha$ are Hermitian then $\ell_{ba}$ is symmetric, and the right eigenvector is parallel to the left eigenvector, with all components equal.)  
In addition to this eigenvector, there are others that seem (though I have not been able to prove it) generically to have negative real parts.  By the same reasoning as before, even if there are eigenvectors of $M$ with positive real part, they cannot contribute to a solution of the Lindblad equation with initial conditions that make $\rho(t)$ Hermitian and positive.



Now let us consider the perturbations to the zeroth order eigenvalues $\lambda_0^{(ab)}$ with $a\neq b$.  For generic energies these are all non-degenerate, so the first-order perturbation to these eigenvalues are simply the matrix elements:
\begin{eqnarray}
&&\lambda_1^{(ab)}=M_{ab,ab}\nonumber\\&&
=\sum_\alpha \Bigg[[L_\alpha]_{aa}[L_\alpha]^*_{bb}-\frac{1}{2}[L_\alpha^\dagger\,L_\alpha]_{bb}-\frac{1}{2}[L_\alpha^\dagger\,L_\alpha]_{aa}\Bigg]\nonumber\\&&=
\sum_\alpha \Bigg[[L_\alpha]_{aa}[L_\alpha]^*_{bb}-\frac{1}{2}|[L_\alpha]_{bb}|^2-\frac{1}{2}[|L_\alpha]_{aa}|^2-\frac{1}{2}\sum_{c\neq b}|[L_\alpha]_{cb}|^2-\frac{1}{2}\sum_{c\neq a}|[L_\alpha]_{ca}|^2\Bigg]\nonumber\\&&=
\sum_\alpha \Bigg[i{\rm Im}([L_\alpha]_{aa}[L_\alpha]_{bb}^*)-\frac{1}{2}\Big|[L_\alpha]_{aa}-[L_\alpha]_{bb}\Big|^2\nonumber\\&&~~~~~~~~~~~~~~
-\frac{1}{2}\sum_{c\neq b}|[L_\alpha]_{cb}|^2-\frac{1}{2}\sum_{c\neq a}|[L_\alpha]_{ca}|^2\Bigg]  \;.
\end{eqnarray}
We see that these eigenvalues all have negative real part.  The pure oscillation at frequency $E_a-E_b$ found in ordinary quantum mechanics is replaced here with an oscillation at a slightly shifted frequency that decays at a rate $-{\rm Re}\lambda_1^{(ab)}$.  

