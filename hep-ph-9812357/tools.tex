%
%----------------------------------------------------------------------
\section{Theoretical tools\label{sec:tools}}
%----------------------------------------------------------------------
%- {{{ multi-loop diagrams:

%----------------------------------------------------------------------
\subsection{Multi-loop diagrams}\label{secmultiloop}
%----------------------------------------------------------------------
%
The main field for multi-loop calculations certainly is QCD in the
perturbative regime. The reason is, on the one hand, that in general the
complexity of a given Feynman diagram increases rapidly with the number
of dimensional parameters involved.  In the Standard Model, for example, with
its relatively large amount of different particles and scales, with the
currently available technology it is hardly possible to compute
processes for electroweak phenomena beyond two-loop level. In QCD, by
contrast, the gauge bosons are massless, and the quark masses are such
that for many physical reactions one may either neglect them completely
or consider only one quark flavour as massive and all others as massless
particles.  On the other hand, in QED, the fine structure constant is
very small, rendering the higher order corrections negligible in
general.  The coupling constant in QCD is about ten times larger, but
still small enough to play the role of an expansion parameter.

%- }}}
%- {{{ Dimensional Regularization and Minimal Subtraction:

%----------------------------------------------------------------------
\subsubsection{Dimensional Regularization and Minimal Subtraction}
%----------------------------------------------------------------------
%
The momentum integrations of loop diagrams are in general not convergent
in four-dimensional space-time.  This requires the introduction of the
concept of renormalization, rendering physical quantities finite and
attributing a natural interpretation to the parameters and fields of the
underlying theory.  In order to isolate the divergent pieces one
introduces a so-called regularization scheme, the most popular one being
dimensional regularization~\cite{tHoVel72} at that time. The strategy is
to replace the four-dimensional loop integrals appearing in Feynman
diagrams by ``$D$-dimensional'' ones, obeying the basic relations of
standard convergent integrals like linearity {\it etc}.  $D$ is
considered to be a complex parameter. The limit of integer $D$, if it
exists, is required to reproduce the value for the standard integral.
All manipulations are those of convergent integrals, and only {\it
  after} integration one takes the limit $D\to 4$.  The divergences are
then reflected as poles in $1/(D-4)$.

The concept of renormalization consists in compensating these
divergences by adding suitable divergent terms, so-called counter-terms,
to the Lagrange density. Since the only requirement on
the counter-terms is to cancel the divergences, there obviously is a
freedom in choosing their finite parts. A particular choice is called
a renormalization scheme.

A convenient renormalization scheme in combination with dimensional
regularization is the so-called minimal subtraction or
MS scheme~\cite{tHo73}.  It prescribes to precisely subtract the poles in
\begin{eqnarray}
\varepsilon &\equiv& (4-D)/2\,.
\end{eqnarray}
Even more popular is the
\msbar~scheme (modified MS scheme)~\cite{BarBurDukMut78} which is
based on the observation that the poles in $1/\varepsilon$
appear only in the combination
\begin{equation}
\Delta = {1\over \varepsilon} - \gamma_{\rm E} + \ln 4\pi
\end{equation}
and that therefore it is convenient to subtract $\Delta$ instead of
$1/\varepsilon$. In the following we will use the \msbar~scheme throughout
unless stated otherwise. If divergent parts will be quoted explicitly,
the $\gamma_{\rm E}$ and $\ln 4\pi$ terms will be omitted.

Dimensional regularization, like any other regularization
scheme, introduces an arbitrary mass parameter, usually denoted
by $\mu$. This becomes clear by considering, for example, the
one-loop integral
\begin{multline}
  \frac{1}{i}\int {\dd^Dp\over (2\pi)^D} 
  {1\over  (-\left(q-p\right)^2)^a 
           \left(-p^2\right)^b} = \\
           {(-1/q^2)^{a+b-D/2}\over (4\pi)^{D/2}}
  \frac{\Gamma(a+b-D/2)\Gamma(D/2-a)\Gamma(D/2-b)}
       {\Gamma(D-a-b)\Gamma(a)\Gamma(b)}
\,,
\label{eqml1loop}
\end{multline}
where $\Gamma(x)$ is Euler's gamma function. Note that $a$
and $b$ are not necessarily integer.
The Laurent-series of the r.h.s.\ with respect to $\varepsilon$
leads to logarithms with dimensional arguments. Therefore, one multiplies
any quantity by an appropriate $D$-dependent power of $\mu$ such that
the mass dimension of the whole object is independent of $D$. It is clear
that only logarithmic $\mu$-dependence may arise by this procedure.
In terms of the Lagrangian, the artificial mass scale manifests itself
in a coupling constant with $D$-dependent mass dimension.

The combination of dimensional regularization and minimal subtraction
has many computational consequences. One can show, for example, that within
the framework of dimensional regularization massless tadpoles,
i.e.~integrals that do not carry any dimensional parameter except the
integration momenta, may be set to zero consistently. On the other hand,
minimal subtraction guarantees that any renormalization constant is a
series in the coupling constant alone, without explicit dependence on
any dimensional quantity like masses or momenta.  Many calculations can
be considerably simplified by exploiting one of these properties.
Indeed, one of the most powerful tools for the computation of multi-loop
diagrams, the so-called algorithm of integration-by-parts,
strongly resides on the properties of dimensional regularization. It
will be described in more detail below.

There are also certain drawbacks of dimensional regularization as well
as the \msbar~scheme.  One of them, of course, is the lack of any
reference to physical intuition, as one has it for regularization
schemes like introducing cut-offs or a discrete space-time lattice, and
for renormalization schemes like the on-shell scheme.  Besides this,
dimensional regularization generally causes problems whenever explicit
reference to four dimensions is made. For example, the anti-commuting
definition of $\gamma_5$ leads to inconsistencies when working in $D$
dimensions.  A consistent definition of $\gamma_5$ was given
in~\cite{tHoVel72} and formalized in~\cite{BreMai77}. It defines
$\gamma_5$ to anti-commute with $\gamma_0,\ldots,\gamma_3$ and to
commute with all the remaining $\gamma$-matrices. This definition
obviously breaks Lorentz covariance and requires the introduction of so-called
evanescent operators which makes practical calculations quite tedious.
Also in the Supersymmetric Standard
Model, defined in four-dimensional space time, one runs into problems,
but we will not dwell on them here since this will not concern what
follows.

%- }}}
%- {{{ Integration-by-parts\label{sec::IP}:

%----------------------------------------------------------------------
\subsubsection{Integration-by-parts\label{sec::IP}}
%----------------------------------------------------------------------
%
Let us now describe one of the benefits of dimensional regularization
in more detail, namely the integration-by-parts algorithm.  It uses the
fact that the $D$-dimensional integral over a total derivative is equal
to zero:
\begin{eqnarray}
  \int{\rm d}^D p {\partial\over \partial p^\mu} f(p,\ldots) &=& 0\,.
  \label{eqipgen}
\end{eqnarray}
By explicitly performing the differentiations one obtains recurrence
relations connecting a complicated Feynman integral to several simpler
ones.  The proper combination of different recurrence relations allows
any Feynman integral (at least single-scale ones) to be reduced to a
small set of so-called master integrals. The latter ones have to be
evaluated only once and for all, either analytically or numerically.

The integration-by-parts algorithm was initially introduced for 
massless two-point
functions up to three loops \cite{CheTka81}, where two non-trivial
master integrals were to be evaluated. Further
on, the technique was applied to those massive tadpole integrals
contributing to the three-loop QCD corrections of the photon
polarization function in the limit $q^2\ll m^2$~\cite{Bro92}, where $q$
is the external momentum of the correlator and $m$ is the mass of the
heavy quark. One non-trivial master integral results in this case.  The
procedure was extended to apply to the three-loop QCD corrections for
the $\rho$ parameter~\cite{Avd95,CheKueSte951} where two more master
integrals had to be evaluated.

The recurrence relations for all possible three-loop tadpole integrals
with a single mass were derived in~\cite{Avd97} and the (most
complicated) master integrals were calculated in~\cite{Bro98}.  At
four-loop level the integration-by-parts technique was applied to
completely massive
tadpole diagrams, only aiming at their divergent parts, however.  This
restricted problem leads to two four-loop master
integrals~\cite{RitVerLar97}.

The integration-by-parts technique equally well applies to on-shell 
integrals, the
complexity at $n$-loop level being comparable to the tadpole case at
$n+1$ loops, however.  The recurrence relations at two loops were
derived some time ago~\cite{GraBroGraSch90} and were applied to the
fermion propagator in order to determine the relation between the
on-shell and $\overline{\rm MS}$ mass in QCD~\cite{GraBroGraSch90}, and
the wave function renormalization constant~\cite{BroGraSch91}.  The
three-loop on-shell integrals contributing to the anomalous magnetic
moment of the electron could be reduced to 18 master
integrals with the help of the integration-by-parts algorithm, and
thus an 
analytic
evaluation of this quantity could be performed~\cite{LapRem96}.  Very
recently ${\cal O}(\alpha^2)$ corrections of the $\mu$ decay were
calculated~\cite{RitStu98}, and the integration-by-parts method was
used to 
determine the
pole part of the corresponding four-loop on-shell integrals.
%
\begin{figure}[h]
  \begin{center}
  \leavevmode
      \epsfxsize=2.5cm
      \epsffile[189 293 423 500]{bsp2loop.ps}\\
    \parbox{\captionwidth}{
      \caption[]{\label{figtriangle}\sloppy
        Two-loop master diagram. The arrows denote the direction of
        momentum flow.
        }      }
  \end{center}
\end{figure}
%

To demonstrate the power of the integration-by-parts method let us
consider the scalar two-loop diagram of Fig.~\ref{figtriangle}.
The corresponding Feynman integral shall be denoted by
\begin{equation}
I(n_1,\ldots,n_5) = \int{\dd^D p\over (2\pi)^D}{\dd^D k\over (2\pi)^D}
{1\over (p_1^2 + m_1^2)^{n_1}\cdots(p_5^2+m_5^2)^{n_5}}\,,
\end{equation}
where $p_1,\ldots,p_5$ are combinations of the loop momenta $p,k$ and
the external momentum $q$ (we work in Euclidean space here).
$n_1,\ldots,n_5$ are called the indices of the integral.  Consider the
subloop defined by lines 2, 3 and 5, and take its loop momentum to be
$p=p_5$. If we then apply the operator $\sprod{(\partial/\partial
  p_5)}{p_5}$ to the {\it integrand} of $I$, we obtain a relation of the
form (\ref{eqipgen}), where
\begin{equation}
f(p_5,\ldots) = \frac{p_5^{\mu}}
  {(p_5^2+m_5^2)^{n_5} (p_2^2+m_2^2)^{n_2} (p_3^2+m_3^2)^{n_3}}
\,.
\end{equation}
Performing the differentiation and using momentum conservation at each
vertex one derives the following equation:
\begin{multline}
\Big[ 
    - n_3 {\bf 3^+}\left({\bf 5^-}-{\bf 4^-}+m_4^2-m_5^2-m_3^2\right)
    - n_2 {\bf 2^+}\left({\bf 5^-}-{\bf 1^-}+m_1^2-m_5^2-m_2^2\right)
\\
    + D-2n_5-n_3-n_2+2n_5 m_5^2 {\bf 5^+} 
\Big]\, I(n_1,\ldots,n_5) = 0
\,,
\label{eqtrianglerec}
\end{multline}
where the operators ${\bf 1^{\pm}}, {\bf 2^{\pm}}, \ldots$ are used in
order to raise and lower the indices: ${\bf I^{\pm}}I(\ldots,n_i,\ldots)
= I(\ldots,n_i\pm 1,\ldots)$.  In Eq.~(\ref{eqtrianglerec}), generally
referred to as the triangle rule, it is understood that the operators to
the left of $I(n_1,\ldots,n_5)$ are applied {\em before} integration.  If the
condition $m_5=0, m_3=m_4$ and $m_1=m_2$ holds, increasing one index
always means to reduce another one.  Therefore this recurrence relation
may be used to shift the indices $n_1$, $n_4$ or $n_5$ to zero which leads
to much simpler integrals.

The triangle rule constitutes an important building block for the
general recurrence relations. The strategy is to combine several
independent equations of the kind~(\ref{eqtrianglerec}) in order to
arrive at relations connecting one complicated integral to a set of
simpler ones. For example, while the direct evaluation of even the
completely massless case for the diagram in Fig.~\ref{figtriangle} is
non-trivial, application of the triangle rule (\ref{eqtrianglerec})
leads to
\begin{equation}
I(n_1,\ldots,n_5) \,=\, \frac{1}{D-2n_5-n_2-n_3}
\Big[
n_2{\bf 2^+}\left({\bf 5^-} - {\bf 1^-}\right)
+
n_3{\bf 3^+}\left({\bf 5^-} - {\bf 4^-}\right)
\Big]\,I(n_1,\ldots,n_5)
\,.
\label{eqrecI}
\end{equation}
Repeated application of this equation reduces one of the indices $n_1$,
$n_4$ or $n_5$ to zero. For example, for the simplest case
($n_1=n_2=\ldots=n_5=1$) one obtains the equation pictured in
Fig.~\ref{fig2loopIP}: The non-trivial diagram on the l.h.s.\ is
expressed as a sum of two quite simple integrals which can be solved by
applying the one-loop formula of Eq.~(\ref{eqml1loop}) twice. This
example also shows a possible trap of the integration-by-parts
technique. In general its application introduces artificial
$1/\varepsilon$ poles which cancel only after combining all terms. They
require the expansion of the individual terms up to sufficiently high
powers in $\varepsilon$ in order to obtain, for example, the finite part
of the original diagram.  This point must carefully be respected in
computer realizations of the integration-by-parts algorithm: One must
not cut the series at too low powers because then the result goes wrong;
keeping too many terms, on the other hand, may intolerably slow down the
performance.

In our example, the l.h.s.\ in Fig.~\ref{fig2loopIP} is finite, each term
on the r.h.s., however, develops $1/\varepsilon^2$ poles. The first
three orders in the expansion for $\varepsilon\to 0$ cancel, and the
${\cal O}(\varepsilon)$ term of the square bracket, together with the
$1/\varepsilon$ in front of it, leads to the well-known result:
$I(1,1,1,1,1)=6\zeta(3)/q^2$, where $q$ is the external momentum.

\begin{figure}[th]
\leavevmode
 \begin{center}
 \begin{tabular}{cccccc}
   \epsfxsize=2.5cm
   \parbox{1cm}{\epsffile[189 293 423 499]{bspT1.ps}}
&
$\displaystyle =\frac{1}{\varepsilon}\Bigg[$
&
   \epsfxsize=2.5cm
   \parbox{1cm}{\epsffile[189 293 423 499]{bspT2.ps}}
&
---
&
   \epsfxsize=2.5cm
   \parbox{1cm}{\epsffile[189 293 423 499]{bspT3.ps}}
&
$\displaystyle \Bigg]$
 \end{tabular}
\parbox{\captionwidth}{
 \caption[]{\label{fig2loopIP}\sloppy
          Symbolic equation resulting from Eq.~(\ref{eqrecI})
          applied to the diagram $I(1,1,1,1,1)$. The dot indicates that
          the respective denominator appears twice.
         }
}
 \end{center}
\end{figure}


In general, the successive application of recurrence relations generates
a huge number of terms out of a single diagram.  Therefore, a calculation
carried out by hand becomes very tedious and the use of computer algebra
is essential.


At the end of this section let us describe an alternative approach which
tries to avoid the explicit use of recurrence relations and thus the
large number of terms in intermediate steps of the calculation.  The
crucial observation is that an arbitrary integral is expressible as
linear combination of the master integrals where the coefficients simply
depend on the dimension $D$ and the indices of the original integral.
Therefore, an attempt to explicitly solve the system of recurrence
relations in terms of integral representations was made
in~\cite{Bai96,BaiSte98}.  It was even possible to derive additional
recurrence relations over the space-time dimension $D$ in this approach.
It was successfully applied to the class of three-loop tadpoles pictured
in Fig.~\ref{fig3ltad}, where a significant reduction of CPU time could
be achieved.  Further developments in this direction look quite
promising.
%
\begin{figure}[h]
  \begin{center}
    \leavevmode
    \epsfxsize=2.5cm
    \epsffile[212 280 400 500]{batopBN.ps}\\
    \parbox{\captionwidth}{
      \caption[]{\label{fig3ltad}\sloppy
        Three-loop topology for which the system of recurrence relations
        was explicitely solved. Solid lines carry a common mass $M$, dashed
        lines are massless.
        }
      }
  \end{center}
\end{figure}

%- }}}
%- {{{ Tensor decomposition and tensor reduction\label{sectensdec}:

%----------------------------------------------------------------------
\subsubsection{Tensor decomposition and tensor reduction\label{sectensdec}}
%----------------------------------------------------------------------
%
The calculation of Feynman diagrams for realistic field theories
inevitably leads to tensor integrals, i.e.\ Feynman integrals carrying
loop momenta with free Lorentz indices in the numerator. Since allowing
for tensor structure largely increases the number of possible integrals,
it is important to have algorithms that reduce them to a set of basis
integrals. The algorithm described in the present section is referred to
as the Passarino-Veltman method~\cite{PasVel79}.  At one loop-level, it
reduces any tensor integral to integrals with unity in the numerator.
At two-loop level this is no longer true, except for two-loop
propagator-type integrals~\cite{WeiSchBoe94}.  We will briefly introduce
the corresponding technique at the end of this section.

Consider an arbitrary 1-loop integral carrying tensor
structure in the integrand,
\begin{equation}\label{eq::tmunuN}
  I_{\mu\nu\rho\cdots}^N = \mu^{4-D} \int{\dd^D k\over (2\pi)^D}{k_\mu
    k_\nu k_\rho \cdots \over D_0 D_1\cdots D_{N-1}} =
\llangle{k_\mu k_\nu k_\rho \cdots \over D_0 D_1\cdots D_{N-1}}\rrangle_D\,,
\end{equation}
with
\begin{equation}\label{eq::Dnotation}
D_0 = k^2 - m_0^2, \qquad D_i = (k+p_i)^2 - m_i^2
\end{equation}
(the $i\epsilon$ in the denominator is suppressed here).  The statement
is \cite{PasVel79} that it can be expressed in terms of the set of
scalar integrals defined as
\begin{equation}\label{eq::scalint}
I_0^N = \llangle{1\over D_0 D_1\cdots D_{N-1}}\rrangle_D\,.
\end{equation}
The strategy is more or less straightforward and relies on the
decomposition of the integral under consideration into covariants, built
out of $g_{\mu\nu}$ and the external momenta. The general solution to
this problem in terms of a recursive algorithm can be found in
\cite{Denner:habil}. It reduces the rank of the tensor in the numerator
by one in each step. Here we only want to give an example to get an idea
of how the algorithm works.
%
Consider the three-point second-rank tensor integral
%
\begin{equation}\label{eq::tmunu3}
I^3_{\mu\nu} = \llangle {k_\mu k_\nu\over D_0 D_1 D_2} \rrangle_D\,,
\end{equation}
%
with its tensor decomposition
%
\begin{equation}\label{eq::tensdec}
I^3_{\mu\nu} = C_{00} g_{\mu\nu} +
C_{11} p_{1\mu} p_{1\nu} + C_{22} p_{2\mu} p_{2\nu} + C_{12} p_{1\mu} p_{2\nu}
+ C_{21} p_{2\mu} p_{1\nu}\,.
\end{equation}
The $C_{ij}$ are often called {\it Passarino-Veltman} coefficients and
are named by $A,B,C,\ldots$, corresponding to the value of $N=1,2,3,\ldots$ in
Eq.~(\ref{eq::tmunuN}).
%
Defining
%
\begin{equation}\label{eq::rimu}
R_i^\mu = \llangle{k^\mu(\sprod{p_i}{k})\over D_0 D_1 D_2}\rrangle_D\,,\quad
  i=1,2\,,\quad \mbox{and}\qquad R_{00} = \llangle {k^2\over D_0 D_1
  D_2}\rrangle_D\,,
\end{equation}
%
we may write the tensor decomposition of those quantities as
%
\begin{equation}\label{eq::Rtensdec}
R_i^\mu = r_{i1}\, p_1^\mu + r_{i2}\, p_2^\mu\,,\qquad R_{00} = r_{00}\,.
\end{equation}
%
Contracting (\ref{eq::tensdec}) by $p_{i,\nu}$ $(i=1,2)$ one obtains
a set of four equations by separately comparing coefficients of
$p_1^\mu$ and $p_2^\mu$:
%
\begin{equation}\label{eq::rij}
\begin{array}{lllllll}
r_{11}&=&C_{00} & + & C_{11}\, p_1^2 & + & C_{12}\, \sprod{p_1}{p_2}\,,\\[.2ex]
r_{12}&=&       &   & C_{22}\, \sprod{p_1}{p_2} & + & C_{21}\, p_1^2\,,\\[.2ex]
r_{22}&=&C_{00} & + & C_{22}\, p_2^2 & + & C_{21}\, \sprod{p_1}{p_2}\,,\\[.2ex]
r_{21}&=&      &   & C_{11}\, \sprod{p_1}{p_2} & + & C_{12}\, p_2^2\,.
\end{array}
\end{equation}
%
Contraction of (\ref{eq::tensdec}) with $g_{\mu\nu}$, on the other hand,
yields
%
\begin{equation}\label{eq::r00}
r_{00} = D\,C_{00} + C_{11}\,p_1^2 + C_{22}\,p_2^2 +
C_{12}\,\sprod{p_1}{p_2} + C_{21}\, \sprod{p_1}{p_2}\,. 
\end{equation}
%
Together, these are five equations for the five unknowns $C_{ij}$
(actually there are only {\it four} unknowns, since $C_{12} = C_{21}$
because of the symmetry $\mu \leftrightarrow \nu$ in $I^3_{\mu\nu}$; this
redundancy may serve as a useful check in the end).
Combining (\ref{eq::r00}) with the sum of the first and third equation
in (\ref{eq::rij}), one immediately gets the solution for $C_{00}$:
%
\begin{equation}\label{eq::C00sol}
C_{00} = {1\over D-2}\left(r_{00} - r_{11} - r_{22}\right)\,.
\end{equation}
%
The remaining coefficients may be determined by rewriting
(\ref{eq::rij}) as two sets of systems of linear equations:
%
\begin{equation}\label{eq::Cij}
\left(
  \begin{array}{c}
    r_{11} - C_{00} \\
    r_{21}
  \end{array}
\right) =
{\bf X}
\left(
  \begin{array}{c}
    C_{11} \\
    C_{12}
  \end{array}
\right)\,,\quad
\left(
  \begin{array}{c}
    r_{12} \\
    r_{22} - C_{00}
  \end{array}
\right) =
{\bf X}
\left(
  \begin{array}{c}
    C_{21} \\
    C_{22}
  \end{array}
\right)\,,
\end{equation}
where
\begin{equation}
{\bf X} = \left(
\begin{array}{cc}
  p_1^2        & \sprod{p_1}{p_2} \\
  \sprod{p_1}{p_2} & p_2^2
\end{array}
\right)\,.
\end{equation}
%
So, if ${\bf X}$ is invertible, Eq.~(\ref{eq::C00sol}) and the inverse
of Eqs.~(\ref{eq::Cij}) determine the $C_{ij}$ in terms of the $r_{ij}$.

In turn, by rewriting
%
\begin{equation}\label{eq::scalprod}
\sprod{k}{p_i} = {1\over 2}\left[D_i - D_0 - f_i\right]\,,\quad
\mbox{with } f_i = p_i^2 - m_i^2 + m_0^2\,,
\end{equation}
%
and inserting this into the first equation of (\ref{eq::rimu}), one
arrives at
%
\begin{eqnarray}
r_{11}\,p_1^\mu + r_{12}\,p_2^\mu &=& {1\over 2}\left[
\llangle{k_\mu\over D_0 D_2}\rrangle_D -
\llangle{k_\mu\over D_1 D_2}\rrangle_D -
f_1 \llangle{k_\mu\over D_0 D_1 D_2}\rrangle_D\right]\,,
\nonumber\\
r_{21}\,p_1^\mu + r_{22}\,p_2^\mu &=& {1\over 2}\left[
\llangle{k_\mu\over D_0 D_1}\rrangle_D -
\llangle{k_\mu\over D_1 D_2}\rrangle_D -
f_2 \llangle{k_\mu\over D_0 D_1 D_2}\rrangle_D\right]\,,
\end{eqnarray}
%
which allows to compute the $r_{ij}$ through first-rank tensor
integrals.  Thus, the first run-through of recurrence is done. The
second one, in turn, is only concerned with at most first-rank tensors.
In that way, any tensor integral can be reduced to the basic set of
scalar integrals defined in (\ref{eq::scalint}), provided that ${\bf X}$
is invertible.  Several improvements to the original algorithm concerned
with the problem of vanishing $\det{\bf X}$ and also with the numerical
evaluation of the scalar integrals have been worked
out~\cite{Stu88,OldVer89}. An overview and a complete list of references
can be found in~\cite{Stu98}.

At two-loop level this strategy no longer reduces numerators of the
integrals to unity. The reason is that in general one encounters
``irreducible numerators'', i.e.\ scalar products that are not
expressible in terms of the denominator via relations like
(\ref{eq::scalprod}).  But in the case of propagator-type integrals,
i.e.\ those with only one external momentum $p$, the full reduction may
still be achieved by applying the mechanism above first to a
subloop~\cite{WeiSchBoe94}.  For example, consider the diagram in
Fig.~\ref{tensred2l.ps}, and a corresponding first-rank tensor-integral:
%
\begin{figure}
  \begin{center}
    \leavevmode
    \epsfxsize=6.cm
    \epsffile[110 245 465 560]{tensred2l.ps}
    \hfill
    \parbox{\captionwidth}{
    \caption[]{\label{tensred2l.ps}\sloppy Two-loop propagator-type
      diagram. The momenta $k_i$ are linear combinations of the loop
      momenta $l,k$ and the external momentum $p$.  }}
  \end{center}
\end{figure}
%
\begin{equation}
S_\mu = \int{\dd^D k\over (2\pi)^D}{\dd^D l\over (2\pi)^D}{k_\mu\over
  D_k D_l D_{k+l} D_{l+p}} \,.
%= 
%\llangle\llangle {k_\mu\over D_k D_l D_{k+l}
%  D_{l+p}}\rrangle_k \rrangle_l \,.
\end{equation}
The notation here is a bit different from the one in
(\ref{eq::Dnotation}):
\begin{equation}
D_{k_i} = k_i^2 - m_i^2\,,
\end{equation}
with
\begin{equation}
k_2 = k\,,\qquad k_3 = k+l\,,\qquad k_4 = l+p\,,\qquad k_5 = l\,.
\end{equation}
Direct
application of the Passarino-Veltman algorithm through the ansatz $S_\mu
= p_\mu \,S(p^2)$ and contracting this equation by $p^\mu$ leads to the
irreducible numerator $\sprod{p}{k}$, so that no simplification results
in this way.

The idea is instead to perform the tensor reduction of the subintegral
over $k$ first, considering $l$ as its external momentum:
\begin{eqnarray}
S_\mu &=&
\int{\dd^D l}\, {1\over D_l D_{p+l}}
\int{\dd^D k}\, {k_\mu\over D_k D_{k+l}} =
\int{\dd^D l}\, {1\over D_l D_{p+l}}\cdot{l_\mu\over l^2}
\int{\dd^D k}\, {\sprod{k}{l}\over D_k D_{k+l}}\nonumber\\&=&
\int{\dd^D l}\, {\sprod{l}{p}\over l^2 D_l D_{p+l}} 
\int{\dd^D k}\, {\sprod{k}{l}\over D_k D_{k+l}} 
\cdot{p_\mu\over p^2} \nonumber\\&=&
 {p_\mu\over 4 p^2} \,\int{\dd^D k}\int{\dd^D l}\, 
 {\left[D_{k+l} - D_k - l^2 - m_3^2 + m_2^2\right]
 \left[D_{p+l} - D_l - p^2 - m_5^2 + m_4^2\right]\over
   l^2 D_k D_{k+l} D_l D_{p+l}}\,.
\end{eqnarray}
After canceling common factors, the numerator has no dependence on loop
momenta any more which is what we were aiming for. The application to
arbitrary propagator-type tensor integrals can be found
in~\cite{WeiSchBoe94}.

%%%%%%%%%%%%%%%%%%%%%%%%%%%%%%%%%%%%%%%%%%%%%%%%%%%%%%%%%%%%

%- }}}
%- {{{ Tensor reduction by shifting the space-time dimension:

%----------------------------------------------------------------------
\subsubsection{Tensor reduction by shifting the space-time dimension
  \label{sec::tredtar}}
%----------------------------------------------------------------------
%
As already noted in Section~\ref{sectensdec}, the Passarino-Veltman
method reduces the tensor structure in the numerator to unity only at
one-loop level, the reason being ``irreducible numerators'' appearing at
higher loop order. An alternative approach that circumvents this problem
has been worked out at one-loop level in \cite{Dav91} and generalized to
an arbitrary number of loops in \cite{Tar96,Tar97}. However, at two-loop
level it has only been applied to propagator type integrals up to now
(see, e.g., \cite{FleJegTarVer99}).
The basic idea is to express tensor integrals in $D$ dimensions through
scalar ones with a shifted value of $D$.  Again we do not want to
present the algorithm in its full generality here, but try to shed some
light on the main ideas by giving a concrete example.

Consider again the integral (\ref{eq::tmunu3}). By introducing an
auxiliary vector $a_\mu$, it may be written as
%
\begin{eqnarray}\label{eq::tart}
\llangle {k_\mu k_\nu\over D_0 D_1 D_2}\rrangle_D =
\left({1\over i}{\partial\over \partial a^\mu}\right)
\left({1\over i}{\partial\over \partial a^\nu}\right)
\llangle {1\over D_0 D_1 D_2} e^{i\sprod{a}{k}}\rrangle_D \bigg|_{a=0}\,.
\end{eqnarray}
%
Using the Schwinger-parameterization for propagators,
%
\begin{equation}
{1\over k^2-m^2+i\epsilon} = {1\over i}\int_0^\infty\dd\alpha
e^{i\alpha(k^2-m^2+i\epsilon)}\,,
\end{equation}
%
one finds
%
\begin{eqnarray}
  && \llangle {1\over D_0 D_1 D_2} e^{i\sprod{a}{k}}\rrangle_D =
  \int_0^\infty\dd\vec{\alpha} \int{\dd^D k\over (2\pi)^D}
  \,e^{-i\left[\alpha_0m_0^2 + \alpha_1m_1^2 + \alpha_2m_2^2\right]}\,
  \times\nonumber\\&&\mbox{\hspace{1ex}} \exp\left\{i\left[(\alpha_0 +
  \alpha_1 + \alpha_2)k^2 + 2\,\left(\alpha_1 p_1 + \alpha_2 p_2 +
  {a\over 2}\right)\!\cdot\!k + \alpha_1 p_1^2 + \alpha_2
  p_2^2\right]\right\}\,,
\end{eqnarray}
%
where $\dd\vec\alpha = \dd\alpha_0\dd\alpha_1\dd\alpha_2$, and with the
help of
%
\begin{equation}
\int {\dd^D k\over (2\pi)^D}\,\exp\left[i(A\,k^2 + 2\sprod{p}{k})\right] =
i{e^{-i\,{p^2\over A}}\over (4\pi i A)^{D/2}}\,,
\end{equation}
%
one obtains
%
\begin{multline}\label{eq::Ifinal}
 \llangle {1\over D_0 D_1 D_2} e^{i\sprod{a}{k}}\rrangle_D =
{i\over (4\pi i)^{D/2}}\int_0^\infty \dd\vec\alpha \,
{e^{-i(\alpha_0m_0^2 + \alpha_1m_1^2 + \alpha_2m_2^2)} \over 
  (\alpha_0 + \alpha_1 + \alpha_2)^{D/2}}  
\exp\bigg\{\!-i\bigl[\!-\alpha_1(\alpha_0 + \alpha_2)p_1^2 
\\- \alpha_2(\alpha_0 + \alpha_1)p_2^2  +
  2\,\alpha_1\alpha_2
  \sprod{p_1}{p_2} + 
 \alpha_1\,\sprod{p_1}{a} +
  \alpha_2\,\sprod{p_2}{a} + a^2/4\bigr]
  /(\alpha_0 
  + \alpha_1 + \alpha_2)\bigg\}\,.
\end{multline}
%
Inserting (\ref{eq::Ifinal}) into (\ref{eq::tart}) and explicitly
performing the differentiations yields
%
\begin{multline}
\llangle{k_\mu k_\nu\over D_0 D_1 D_2}\rrangle_D =
{i\over (4\pi i)^{D/2}}\int\dd \vec\alpha\,
{e^{-i\left(\alpha_0m_0^2 + \alpha_1m_1^2 + \alpha_2m_2^2\right)}
  \over
  (\alpha_0 + \alpha_1 + \alpha_2)^{D/2+2}}
(\alpha_1 p_1 + \alpha_2 p_2)_\mu (\alpha_1 p_1 + \alpha_2 p_2)_\nu
\, \times\\
\exp\bigg\{-i\Big[-\alpha_1(\alpha_0 + \alpha_2)p_1^2 -
\alpha_2(\alpha_0 + \alpha_1)p_2^2 + 2\,\alpha_1\alpha_2\,
  \sprod{p_1}{p_2} \Big]/(\alpha_0 + \alpha_1 + \alpha_2)\bigg\}\,.
\label{eq::alpha-param}
\end{multline}
%
The technique to derive Eq.~(\ref{eq::alpha-param}) and its
generalization to arbitrary multi-loop diagrams is known since long (see,
e.g., \cite{BreMai77}).  However, instead of differentiating with
respect to $a$ we may equally well apply the following operator on the
scalar integral $\llangle{1\over D_0 D_1 D_2}\rrangle_D$, in that way
getting rid of the auxiliary vector $a$:
%
\begin{eqnarray}
\lefteqn{T_{\mu\nu}\left(\{p_1,p_2\},\left\{{\partial\over \partial
  m_1^2},{\partial\over \partial m_2^2}\right\}, {\bf d^+}\right) \equiv } 
\nonumber\\&\equiv&
\left({1\over i}{\partial\over \partial a^\mu}\right)
\left({1\over i}{\partial\over \partial a^\nu}\right)
  \exp\left[-i(\alpha_1 \sprod{p_1}{a} + \alpha_2\sprod{p_2}{a})
  \rho\right]\bigg|_{a=0,\,\alpha_j = 
    i{\partial\over \partial m_j^2},\,
    \rho    = 4\pi i {\bf d^+}}
  \nonumber\\&=& -\left(p_{1\mu}{\partial\over \partial m_1^2} + 
  p_{2\mu}{\partial\over \partial m_2^2}\right)
  \left(p_{1\nu}{\partial\over \partial m_1^2} + 
  p_{2\nu}{\partial\over \partial m_2^2}\right)\,(4\pi i\,{\bf d^+})^2\,,
\end{eqnarray}
%
where the operator ${\bf d^+}$ increases the space-time dimension by
$2$, i.e.\ ${\bf d^+} \llangle \cdots\rrangle_D = \llangle
\cdots\rrangle_{D+2}$. Finally, we have
%
\begin{eqnarray}
  &&\llangle {k_\mu k_\nu\over D_0 D_1 D_2}\rrangle_D = 
  T_{\mu\nu}\llangle{1\over D_0 D_1 D_2}\rrangle_D = (4\pi)^2 \Bigg[
  2\,p_{1\mu} p_{1\nu}\llangle{1\over D_0 D_1^3 D_2}\rrangle_{D+4} +
  \nonumber\\&&\mbox{\hspace{6ex}} +
  2\,p_{2\mu} p_{2\nu}\llangle{1\over D_0 D_1 D_2^3}\rrangle_{D+4} +
  (p_{1\mu} p_{2\nu} + p_{1\nu} p_{2\mu}) 
  \llangle{1\over D_0 D_1^2 D_2^2}\rrangle_{D+4}\Bigg]\,.
\end{eqnarray}

The algorithm formally applies to an arbitrary number of loops and
external legs.  But this means only that any tensor integral can be
reduced to scalar integrals with a shifted number of space-time
dimension, the latter ones remaining still to be evaluated. A strategy
to cope with these diagrams is to use generalized recurrence
relations~\cite{Tar97}.  At the one-loop level they have been worked out
for arbitrary $n$-point functions in \cite{Tar96}. At the two-loop
level, however, so far they are only published for propagator-type
diagrams~\cite{Tar97}.  Since we feel that these generalized recurrence
relations are beyond the scope of this review, let us refer the
interested reader to the literature \cite{Tar97,TarRhein98}.
The algorithm described above was used in \cite{FleJegTarVer99} for the
computation of the two-loop QCD corrections to the fermion propagator.

%- }}}
%- {{{ Asymptotic expansion of Feynman diagrams\label{subasymp}:
%----------------------------------------------------------------------
\subsection{\label{subasymp}Asymptotic expansion of Feynman diagrams}
%----------------------------------------------------------------------
%- {{{ Generalities:

%
%----------------------------------------------------------------------
\subsubsection{Generalities}
%----------------------------------------------------------------------
As was outlined in Section~\ref{secmultiloop}, the
integration-by-parts algorithm was 
successfully applied to single-scale integrals, i.e.\ massive
tadpole, massless propagator-type, or on-shell integrals.
For an arbitrary multi-scale diagram it is in general rather difficult to
solve recurrence relations.

However, if the scales involved follow a certain hierarchy, a
factorization is possible.  For example, consider the operator product
expansion for the correlator of currents $j(x) = \bar \psi(x) \Gamma
\psi(x)$ in the limit $Q^2 \to \infty$, where $\Gamma$ is some Dirac
matrix and $\psi$ a quark field with mass $m$:
\begin{equation}
i\int \dd x e^{iqx} {\rm T} j(x) j(0) \stackrel{Q^2\to \infty}{\simeq}
\sum_n C_n {\cal O}_n\,,
\end{equation}
where T denotes the time ordered product.  It was realized
in~\cite{CheGorTka82,Tka83,BroGen84,CheSpi87} that if one adopts the
minimal subtraction regularization scheme and abandons normal ordering
of the operators, then $Q$ appears only in the coefficient functions
$C_n$ while the mass $m$ is completely absorbed into the operators
${\cal O}_n$ and appears only in the matrix elements.

The attempts to find an algorithm that produces this factorization for
arbitrary Feynman integrals finally resulted in the prescriptions for
the asymptotic expansion of Feynman
diagrams~\cite{PivTka84,GorLar87,CheSmi87,Smi91,Smi95} (see
also~\cite{Davetal}). These prescriptions provide well defined recipes
that are completely decoupled from any field theoretic derivation and
even lack a rigorous proof, as in the cases (iii) and (iv) below.
However, their success in practical applications justifies them a
posteriori.

At the moment the following procedures are used in the calculations:
\renewcommand{\labelenumi}{(\roman{enumi})}
\begin{enumerate}
\item Large-Momentum Procedure: $Q \gg q,m$
\item Hard-Mass Procedure: $M \gg q,m$
\item Threshold Expansion
\item Expansion with the external momenta on the mass shell.
\end{enumerate}
\renewcommand{\labelenumi}{(\Roman{enumi})}
The first two of them will be considered more closely in the next
subsection.  The presentation will be rather informal, explaining the
procedures in a ready-to-use form.  Currently, the technical apparatus
for the latter two cases is much less developed. For this reason they
will be touched upon only briefly here.

One may treat large-momentum and hard-mass procedure on the same
footing. Thus, in what follows we only present the general formulae in
the case of large external momenta --- the transition to the hard-mass
procedure is straightforward.  The prescription for the large-momentum
procedure is summarized by the following formula:
\begin{eqnarray}
\Gamma(Q,m,q) & \stackrel{Q\to \infty}{\simeq} &
\sum_\gamma \Gamma/\gamma(q,m)
\,\,\star\,\, 
T_{\{q_\gamma,m_\gamma\}}\gamma(Q,m_\gamma,q_\gamma)
\,.
\label{eqasexp}
\end{eqnarray}
Here, $\Gamma$ is the Feynman diagram under consideration, $\{Q\}$
($\{m,q\}$) is the collection of the large (small) parameters, and the
sum goes over all subgraphs $\gamma$ of $\Gamma$ with masses $m_\gamma$
and external momenta $q_\gamma$, subject to certain conditions to be
described below.  $T_{\{q,m\}}$ is an operator performing a Taylor
expansion in $\{q,m\}$ {\em before} any integration is carried out.  The
notation $\Gamma/\gamma\star T_{\{q,m\}}\gamma$ indicates that the
subgraph $\gamma$ of $\Gamma$ is replaced by its Taylor expansion which
should be performed in all masses and external momenta of $\gamma$ that
do not belong to the set $\{Q\}$.  In particular, also those external
momenta of $\gamma$ that appear to be integration momenta in $\Gamma$
have to be considered as small. Only after the Taylor expansions have
been carried out, the loop integrations are performed.  In the following
we will refer to the set $\{\gamma\}$ as {\em hard subgraphs} or simply {\em
  subgraphs}, to $\{\Gamma/\gamma\}$ as {\em co-subgraphs}.

The conditions for the subgraphs $\gamma$ are different for
the hard-mass and the large-momentum procedure\footnote{
  Actually they are very similar and it is certainly possible to merge
  them into one condition using a more abstract language. For our
  purpose, however, it is more convenient to distinguish the two procedures.}.
For the large-momentum procedure, $\gamma$ must
\begin{itemize}
\item contain all vertices where a large momentum enters or leaves the
  graph
\item be one-particle irreducible after identifying these
  vertices.
\end{itemize}
From these requirements it is clear that the hard subgraphs become
massless integrals where the scales are given by the large momenta. In
the simplest case of one large momentum one ends up with propagator-type
integrals.  The co-subgraph, on the other hand, may still contain small
external momenta and masses. However, the resulting integrals are
typically much simpler than the original one.

In the case of hard-mass procedure, $\gamma$ has to
\begin{itemize}
\item contain all propagators carrying a large mass
\item be one-particle irreducible in its connected parts after
  contracting the heavy lines.
\end{itemize}
Here, the hard subgraphs reduce to tadpole integrals with the large masses
setting the scales. The co-subgraphs are again simpler to
evaluate than the initial diagram.


The large-momentum and hard-mass procedure provide expansions for {\it
  off-shell} external momenta which are either large or small as
compared to internal masses. Recently a procedure allowing the
asymptotic expansion of Feynman integrals near threshold was suggested.
Graphical prescriptions similar to those for the large-momentum and
hard-mass procedure have been worked out and applied to
the cross section of quark production at $e^+e^-$ colliders near
threshold~\cite{BenSigSmi98,CzaMel98}. The expansion parameter in this case
is given by the velocity of the produced quarks.

A method to expand on-shell Feynman diagrams was developed in
\cite{Smi97,CzaSmi97}.  Two typical limits for a large external momentum
on a mass shell were considered: one where the mass shell is itself
large, and the other one where the mass shell is zero.
The latter case, called the Sudakov
limit, is a purely Minkowskian phenomenon. This distinguishes it from
the cases described so far which can be formulated completely in
Euclidean space, simplifying rigorous proofs.  The ``philosophy'' of
these expansions is very similar to the hard-mass and large-momentum
procedure.  However, apart from the criteria on the subgraphs, the way of
performing the expansion of propagators is also different. For example,
the expansion of lines carrying large masses and not belonging to a
one-particle-irreducible component of the subgraph (so-called cut lines)
looks as follows:
\begin{eqnarray}
T_\kappa\frac{1}{\kappa k^2 + 2Qk}\Bigg|_{\kappa=1}
\,.
\end{eqnarray}
$k$ is an integration momentum and $Q$ a large external momentum.  The
graphical representation of the procedure becomes much less transparent
because of that.

In~\cite{CzaSmi97} the two-loop master integral with two different
masses and on-shell external momentum was considered: $m\ll M$,
$Q^2=M^2$. The first 19 terms of the expansion in $m/M$ were evaluated.
Similar computations have also been performed for the fermion
propagator~\cite{AvdKal97}.  Two-loop vertex diagrams with external
momenta $p_1$ and $p_2$ obeying the Sudakov limit $p_i^2=0$ and
$(p_1+p_2)^2\to-\infty$ were examined in \cite{Smi97_2}, and an
expansion in $m^2/(p_1+p_2)^2$ was obtained.

Let us finally emphasize that due to analyticity the obtained expansions
often provide valuable information also in other regions of the
parameter space. This knowledge was used to reconstruct the photon
polarization function by combining the results of asymptotic expansions
in different limits (see also Section~\ref{sec::polfunc}).

%- }}}
%- {{{ Examples:

%----------------------------------------------------------------------
\subsubsection{\label{subsubexa}Examples}
%----------------------------------------------------------------------
Let us first consider the one-loop contribution to the photon propagator
shown on the l.h.s.\ of the diagrammatic equation in
Fig.~\ref{fig::lmp1l}.  Both fermion lines are supposed to carry the
same mass, $m$, and $q$ is the external momentum.  The application of
the large-momentum procedure leads to the subdiagrams shown on the
r.h.s.\ of this equation.  The first one represents a simple Taylor
expansion w.r.t.\ $m$, thus leading to massless one-loop integrals which
can be solved with the help of Eq.~(\ref{eqml1loop}). Starting from a
certain order in $m^2$, this subdiagram develops infra-red poles which
are absent in the original diagram. They are due to the massless
denominators arising in the Taylor expansion.  However, they cancel
against twice the ultra-violet poles of the second subgraph.  The factor
of two arises because a symmetric subgraph should be considered as well.
%
\begin{figure}[t]
  \begin{center}
    \parbox{\captionwidth}{
  \leavevmode
  \epsfxsize=2.5cm
  \epsffile[150 260 420 450]{d1q.ps}\hspace{1em}
  \raisebox{2.1em}
  {\Large $= \ \ 1\ \star\,\, $}
  \epsfxsize=2.5cm
  \epsffile[150 260 420 450]{d1q.ps}\hspace{1em}
  \raisebox{2.1em}
  {\Large $+ \ \ 2\times \!\!\!\!$}
  \epsfxsize=2cm
  \raisebox{.5em}{\epsffile[150 260 420 450]{d1q_2c.ps}}\hspace{-.5em}
  \raisebox{2.1em}{\Large $\star\,\,$}
  \epsfxsize=2.5cm
  \epsffile[150 260 420 450]{d1q_2.ps}\hspace{1em}}
%  \begin{center}
    \parbox{\captionwidth}{
      \caption[]{\label{fig::lmp1l}\sloppy
        Large-momentum procedure for the one-loop photon polarization
        function.
        }      }
  \end{center}
\end{figure}
%
To make this cancellation more transparent we present the first four
terms of an expansion in $m^2$ for the transverse part of the one-loop
polarization function (cf.\ Section~\ref{sec::polfunc}):
\begin{eqnarray}
  \Pi^{(0)}_{\rm bare}(q^2)  
&\stackrel{q^2\gg m^2}{=}&
{3\over 16\pi^2}\,\Bigg\{
  {4\over 3 \varepsilon}
  + {20\over 9} 
  - {4\over 3}\,\logqmums
  + 8\,{m^2\over q^2}
  + \left({m^2\over q^2}\right)^2\,\bigg( 
    - {8\over\varepsilon} 
    - 8
    + 8\,\logqmums
  \bigg)
  \nonumber\\&&\mbox{\hspace{3em}}
  + \left({m^2\over q^2}\right)^3\,\bigg( 
    - {32\over 3 \varepsilon} 
    - {80\over 3} + {32\over 3}\,\logqmums
  \bigg)
  \nonumber\\&&\mbox{\hspace{1em}}
  + 2\,\bigg[\left({m^2\over q^2}\right)^2\,\bigg(
    {4\over \varepsilon}
    + 6
    + 4\,\logmum
  \bigg)
  + \left({m^2\over q^2}\right)^3\,\bigg( 
    {16\over 3 \varepsilon}
    + {88\over 9} 
    + {16\over 3}\,\logmum
  \bigg)
  \bigg]
  + \ldots
  \Bigg\}= 
  \nonumber\\
  &=&
  {3\over 16\pi^2}\,\Bigg\{  
  {4\over 3\varepsilon}
  + {20\over 9} - {4\over 3}\,\logqmums
  + 8 {m^2\over q^2}
  + \left({m^2\over q^2}\right)^2\,\bigg( 
    4 
    + 8\,\logqmms
  \bigg)
  \nonumber\\&&\mbox{\hspace{3em}}
  + \left({m^2\over q^2}\right)^3\,\bigg( 
    - {64\over 9} 
    + {32\over 3}\,\logqmms 
  \bigg)
+\ldots
  \Bigg\}\,,
  \label{eq::lmp1l}
\end{eqnarray}
%
with $\logqmums=\ln(-q^2/\mu^2)$, $\logqmms=\ln(-q^2/m^2)$ and
$\logmum=\ln(\mu^2/m^2)$.  The first two lines correspond to the first
subgraph of Fig.~\ref{fig::lmp1l}, and the terms in square brackets are
due to the second subgraph.  The fourth and fifth lines show the sum of
all subgraphs which corresponds to the consistent expansion of the full
one-loop diagrams.  The remaining pole in $1/\varepsilon$ is the
ultra-violet divergency of the full diagram and is usually removed by
renormalization (see below). On the other hand, all spurious poles
cancel in the sum.

Let us now analyze the diagram on the l.h.s.\ of Fig.~\ref{fig::lmp1l} in
the limit $q^2\ll m^2$. In this case the hard-mass procedure applies. It
leads to a trivial Taylor expansion, and one ends up with bubble
integrals.  The first few terms read:
\begin{eqnarray}
\Pi^{(0)}_{\rm bare}(q^2) 
&\stackrel{q^2\ll m^2}{=}&
{3\over 16\pi^2}\,\Bigg\{
  {4\over 3 \varepsilon}
  + {4\over 3}\,\logmum
  + \frac{4}{15}\frac{q^2}{m^2}
  + \frac{1}{35}\left(\frac{q^2}{m^2}\right)^2
  + \frac{4}{945}\left(\frac{q^2}{m^2}\right)^3
  + \ldots
\Bigg\}\,.
\end{eqnarray}
Note that the $1/\varepsilon$ pole is the same as in
Eq.~(\ref{eq::lmp1l}). In the case of the photon propagator
the pole is usually removed by requiring that the polarization
function vanishes for $q^2=0$.
Finally the one-loop polarization function in the two limiting cases
reads:
\begin{eqnarray}
\Pi^{(0)}(q^2) 
&\stackrel{q^2\gg m^2}{=}&
  {3\over 16\pi^2}\,\Bigg\{  
    {20\over 9} - {4\over 3}\,\logqmms
  + 8 {m^2\over q^2}
  + \left({m^2\over q^2}\right)^2\,\bigg( 
    4 
    + 8\,\logqmms
  \bigg)
  \nonumber\\&&\mbox{\hspace{3em}}
  + \left({m^2\over q^2}\right)^3\,\bigg( 
    - {64\over 9} 
    + {32\over 3}\,\logqmms 
  \bigg)
+\ldots
  \Bigg\}\,,
\nonumber\\
\Pi^{(0)}(q^2)
&\stackrel{q^2\ll m^2}{=}&
{3\over 16\pi^2}\,\Bigg\{
    \frac{4}{15}\frac{q^2}{m^2}
  + \frac{1}{35}\left(\frac{q^2}{m^2}\right)^2
  + \frac{4}{945}\left(\frac{q^2}{m^2}\right)^3
  + \ldots
\Bigg\}\,.
\end{eqnarray}

Consider now the the case of the double-bubble diagrams pictured in
Fig.~\ref{figdb}. They provide a gauge invariant subclass of all
three-loop graphs contributing to the photon polarization function.
Note, however, that the particle type is irrelevant for the procedures
described in the previous section; only the mass and momentum
distribution is important.  The outer and the inner mass are denoted by
$m_1$ and $m_2$, respectively, and $q$ is the external momentum.
%
\begin{figure}[t]
\begin{center}
\begin{tabular}{ccc}
\leavevmode
\epsfxsize=5.0cm
\epsffile[142 267 470 525]{diacc1.ps}
&\hspace{2em}&
\leavevmode
\epsfxsize=5.0cm
\epsffile[142 267 470 525]{diacc2.ps} \\
(a) &\hspace{2cm}& (b)
\end{tabular}
\parbox{\captionwidth}{\sloppy
\caption[]{\label{figdb} 
Fermionic double-bubble diagrams with generic masses $m_1$ and $m_2$.
}}
\end{center}
\end{figure}



Of special interest is the high energy expansion of the current
correlators, i.e., $q^2\gg m_1^2,m_2^2$, for example the case where $m_1=0$ and
$m_2=m$.  The imaginary part leads to ${\cal O}(\alpha_s^2)$ corrections
to $R(s)$ in the energy range sufficiently large as compared to the mass of
the produced quarks. One can think of the production of light quarks
($u,d,s$) accompanied by a pair of charm quarks ($m=M_c$) at energies
$\sqrt{s}\gsim5$~GeV, taking into account charm mass effects.

Application of the large-momentum procedure to the diagram of
Fig.~\ref{figdb}(a) results in the subdiagrams displayed in
Fig.~\ref{figdblmp}. We have combined topologically identical diagrams
there. Similar terms arise from the graph of Fig.~\ref{figdb}(b). Since
the hard subgraphs have to be Taylor expanded in $m$ and any
``external'' momentum except $q$, they belong to the class of massless
two-point integrals, and can therefore be computed with the help of the
integration-by-parts technique.  The co-subgraphs, on the other hand,
are massive tadpole integrals, so that also here integration-by-parts
can be applied.  Note that the first of the four subdiagrams in
Fig.~\ref{figdblmp} represents the naive Taylor expansion of the
integrand with respect to $m$. It is clear that starting from a certain
order this term contains infra-red poles which are artificial as the
original diagram is infra-red finite.  The cancellation of these poles
in the sum of all terms on the r.h.s.\ of the equation in
Fig.~\ref{figdblmp} provides a non-trivial check of the
calculation.  The result of this expansion up to ${\cal O}(m^8/q^8)$
looks as follows:
\begin{eqnarray}
\bar{\Pi}_{gs}(q^2) &\stackrel{q^2\gg m^2}{=}&
  {3\over 16\pi^2}\,
  \left({\alpha_s\over \pi}\right)^2\,C_{\rm F}\,T\,\Bigg[
      -{3701\over 648} 
      + {38\over 9}\,\zeta_3 
                + \logqmums\,\bigg(
          {11\over 6} 
          - {4\over 3}\,\zeta_3
          \bigg) 
      - {1\over 6}\,\logqmums^2 
\nonumber\\&&\mbox{\hspace{1em}} 
      + {m^2\over q^2}\,\bigg(
          -{64\over 3} 
          + 16\,\zeta_3
          \bigg) 
      + \left({m^2\over q^2}\right)^2\,\bigg(
          -{67\over 3} 
          + 16\,\zeta_3 
          + \logqmms\,\big(
              -{26\over 3} 
              + 8\,\zeta_3
              \big)
          - \logqmms^2 
          \bigg)
\nonumber\\&&\mbox{\hspace{1em}} 
      + \left({m^2\over q^2}\right)^3\,\bigg(
          -{1552\over 243} 
          + {160\over 27}\,\zeta_3
          - {272\over 243}\,\logqmms 
          + {56\over 81}\,\logqmms^2 
          + {16\over 81}\,\logqmms^3 
          \bigg) 
\nonumber\\&&\mbox{\hspace{1em}} 
      + \left({m^2\over q^2}\right)^4\,\bigg(
          {1435\over 324} 
          - {10\over 3}\,\zeta_3
          + {113\over 54}\,\logqmms 
          - {1\over 18}\,\logqmms^2 
          - {1\over 9}\,\logqmms^3 
          \bigg) 
      \Bigg] + \cdots\,,
\label{eqdblmpres}
\end{eqnarray}
with $\logqmums=\ln(-q^2/\mu^2), \logqmms=\ln(-q^2/m^2)$ and
$\zeta_3=1.202056903\ldots$.  In Section~\ref{subsubexplmp} it is
shown how the computation of all subdiagrams can be done using program
packages designed to automate the large-momentum procedure.  The
logarithmic terms up to the fourth order can be found
in~\cite{CheKue94}, the others are new. We have adopted the
\msbar~scheme, i.e., after taking into account the counter-terms for
$\alpha_s$ induced by the (massless) one and two-loop diagrams, the
local poles are subtracted.  Note that because the mass $m$ is absent in
the lower order terms, it does not need to be renormalized.

\begin{figure}[t]
  \begin{center}
  \leavevmode
   \epsfxsize=3cm
   \epsffile[150 260 420 450]{db.ps}\hspace{1em}
   \raisebox{2.8em}
   {\Large $= \ \ 1\ \star $}
   \epsfxsize=3.cm
   \epsffile[150 260 420 450]{db.ps}\hspace{0em}
   \raisebox{2.8em}
   {\Large $+ \ \ 2\times \!\!$}
   \epsfxsize=1.5cm
   \raisebox{1.7em}{\epsffile[150 260 420 450]{lmpcs1.ps}}\hspace{0em}
   \raisebox{2.8em}{\Large $\star$}
   \epsfxsize=3.cm
   \epsffile[150 260 420 450]{lmp1.ps}\hspace{1em}\\[1em]
   \mbox{\hspace{1em}}\raisebox{2.8em}
   {\Large $+$}
   \epsfxsize=1.5cm
   \raisebox{1.7em}{\epsffile[150 260 420 450]{lmpcs2.ps}}
   \epsfxsize=3.cm
   \raisebox{2.8em}{\Large $\star$}
   \epsffile[150 260 420 450]{lmp2.ps}\hspace{0em}
   \hspace{.5em}\raisebox{2.8em}
   {\Large $+ \ \ 2\times \!\!\!\!$}
   \epsfxsize=3cm
   \raisebox{0em}{\epsffile[150 260 420 450]{lmpcs3.ps}}
   \epsfxsize=3cm
   \raisebox{2.8em}{\Large $\star$}
   \epsffile[150 260 420 450]{lmp3.ps}\hspace{0em}
   \parbox{\captionwidth}{
   \caption[]{\label{figdblmp} \sloppy
     Large-momentum procedure for the double-bubble diagram.
     The inner loop (thick lines)
     carries mass $m$, the outer one is massless, and so are the gluon
     lines. The square of the momentum $q$ flowing through
     the diagram is supposed to be much larger than $m^2$.  It is
     understood that the hard subgraphs (right of ``$\star$'') are to be
     expanded in $m$ and all external momenta except for $q$, and
     reinserted into the fat vertex dots of the co-subgraphs (left of
     ``$\star$'').  Contributions involving massless tadpoles are not
     displayed since they are zero in dimensional regularization.}}
 \end{center}
\end{figure}


As an application of the hard-mass procedure let us consider the hierarchy 
$m_1^2\ll q^2\ll m_2^2$. The imaginary part again leads to contributions
for $R(s)$. This time one may think of charm quark production ($m_1=M_c$)
in the presence of a virtual bottom quark ($m_2=M_b$). It turns out that
already the first term provides a very good approximation almost up
to the threshold $\sqrt{s}=2M_b$~\cite{Che93,HoaJezKueTeu94,Teudiss}.
For simplicity we set $m_1=0$ and $m_2=m$ in the following.

\begin{figure}[t]
  \begin{center}
  \leavevmode
   \epsfxsize=3cm
   \epsffile[150 260 420 450]{db.ps}\hspace{1em}
   \raisebox{2.8em}
   {\Large $=$}
   \epsfxsize=2.cm
   \raisebox{1.em}{\epsffile[150 260 420 450]{born.ps}}\hspace{-1em}
   \raisebox{2.8em}{\Large $\ \ \star \!\!$}
   \epsfxsize=3.cm
   \epsffile[150 260 420 450]{hmp0.ps}\hspace{0em}\\[1em]
   \mbox{\hspace{1em}}
   \raisebox{2.8em}
   {\Large $+ \ \ 2\times \ $}
   \epsfxsize=2.cm
   \raisebox{1.em}{\epsffile[150 260 420 450]{hmpcs1.ps}}\hspace{0em}
   \raisebox{2.8em}{\Large $\ \ \star \!\!$}
   \epsfxsize=3.cm
   \epsffile[150 260 420 450]{hmp1.ps}
   \raisebox{2.8em}
   {\Large $\!\!\!\!+\ \ $}
   \epsfxsize=3cm
   \raisebox{0em}{\epsffile[150 260 420 450]{hmpcs2.ps}}
   \epsfxsize=3.cm
   \raisebox{2.8em}{\Large $\ \ \star \!\!\!\!\!\!$}
   \epsffile[150 260 420 450]{hmp2.ps}\hspace{0em}
%  \begin{center}
    \parbox{\captionwidth}{
   \caption[]{\label{figdbhmp}\sloppy
     Hard-mass procedure for the double-bubble diagram. Now the
     hierarchy $q^2 \ll m^2$ is considered where $m$ is the mass of the
     inner line.  The hard subdiagrams (right of ``$\star$'') are to be
     expanded in all external momenta including $q$ and reinserted into
     the fat vertex dots of the co-subgraphs (left of~``$\star$'').}}
 \end{center}
\end{figure}

The corresponding diagrammatic representation
is shown in Fig.~\ref{figdbhmp}. There are three subdiagrams, one of
which again corresponds to the naive Taylor expansion of the integrand
in the external momentum $q$.
After Taylor expansion, the subdiagrams are reduced to tadpole
integrals with mass scale $m$. The scale of the co-subgraphs is given
by $q$ thus leading to massless propagator-type integrals.
The result for the first three terms reads~\cite{Stediss}:
\begin{eqnarray}
\bar{\Pi}_{gs}(q^2) &\stackrel{q^2\ll m^2}{=}&
   \frac{3}{16\pi^2}
  \left({\alpha_s\over \pi}\right)^2\,C_{\rm F}\,T\,\Bigg[
        \frac{295}{648} 
      + \frac{11}{6}\logqmums
      - \frac{1}{6}\logqmums^2
      - \frac{11}{6}\logqmms 
      + \frac{1}{6}\logqmms^2
\nonumber\\
&&\mbox{}
      - \frac{4}{3}\zeta_3\logqmums 
      + \frac{4}{3}\zeta_3\logqmms
%\nonumber\\
%&&
      + \frac{q^2}{m^2}\left(
         \frac{3503}{10125} 
       - \frac{88}{675}\logqmms
       + \frac{2}{135}\logqmms^2
                       \right)
\nonumber\\
&&
       + \left(\frac{q^2}{m^2}\right)^2\left(
        - \frac{2047}{514500} 
        + \frac{1303}{529200}\logqmms
        - \frac{1}{2520}\logqmms^2
                                        \right)
\Bigg] + \cdots\,.
\label{eqdbhmpres}
\end{eqnarray}
If $m_1$ was different from zero, successive application of the
large-momentum procedure to the co-subgraphs would lead to a subsequent
expansion in $m_1^2/q^2$. An example for such a repeated use of
asymptotic expansions will be described in Section~\ref{sec::zbb}.
Alternatively, one may evaluate the integration analytically as it has
been done for the $q^2/m^2$ term in~\cite{Sei:dipl}, thus leading to the
full $m_1$ dependence of the power-suppressed result.

%- }}}

%- }}}
%- {{{ Helicity-amplitude technique\label{sec::helamp}:

%----------------------------------------------------------------------
\subsection{Helicity-amplitude technique\label{sec::helamp}}
%----------------------------------------------------------------------
The standard way to obtain a cross section or a decay rate in
perturbative quantum field theories is to compute the squared amplitude
and integrate over the phase space for the final state particles. A
typical amplitude is a sum over terms of the form
\begin{equation}
  c\,\prod_i\epsilon_{\mu_i}(k_i,\chi_i)\,\prod_j \bar
  u(p_{j},\lambda_j)\,\Gamma_j\, u'(p'_j,\lambda'_j)\,,
  \label{eq::typamp}
\end{equation}
where $\epsilon$ are polarization vectors of vector particles with
momentum $k_i$ and polarization $\chi_i$, $u$ and $u'$ are spinors of
fermions (or anti-fermions) with momentum $p_j$, $p'_j$ and
helicity $\lambda_j$, $\lambda'_j$, respectively. $\Gamma_j$ are
matrix-valued objects in Dirac space, and $c$ is a scalar function of
momenta, masses, coupling constants etc.  Each term of the form
(\ref{eq::typamp}) corresponds to a certain Feynman diagram.  The
standard way to deal with the Dirac structure is to square the amplitude
{\it before} evaluating the expressions in (\ref{eq::typamp}) any
further. After summing over polarizations of final and initial states, one
employs the relations
\begin{equation}
\sum_{\lambda = \pm 1/2} u(p,\lambda)\bar u(p,\lambda) \sim p\!\!\!/ \pm m
\end{equation}
for fermions/anti-fermions and takes the trace in Dirac space which can
be evaluated with the help of the anti-commutator $\{\gamma_\mu,\gamma_\nu\}
= 2 g_{\mu\nu}$. Specific polarization configurations can be
investigated by introducing suitable projectors in Dirac space. 

However, the higher the order of perturbation theory, the more diagrams
contribute, and because the amplitude must be squared, the number of
terms to be evaluated in the above way increases even quadratically with
the number of diagrams.  As long as one stays with integrated quantities
(total rates), one way out is to apply the optical theorem, i.e., to
take the imaginary part of the forward scattering amplitude (see, e.g.,
Sections~\ref{sec::polfunc} and~\ref{sec::zbb}).
But as soon as one is interested in differential distributions,
this method is no longer applicable. A solution here is the so-called
helicity-amplitude technique. The basic idea is to rewrite
expressions of the form
\begin{equation}
u(p,\lambda)\bar u'(p',\lambda')
\end{equation}
for fixed helicities $\lambda,\lambda'$ in terms of Dirac matrices. This
allows the trace to be taken {\it before} squaring the amplitude. For
given four-momenta one arrives at a single complex number for the full
amplitude, one for each helicity configuration. If desired, the
summation over polarizations is done only {\it after} squaring the
amplitude.

This strategy goes back to \cite{BjoChe66} and was further developed in
\cite{ref::calkul}. Improved algorithms have been worked out in
\cite{ref::helamp,BalMai95}. For detailed discussions of the various
methods let us refer to these original works.

%- }}}

%
%
% end of tools.tex
%
