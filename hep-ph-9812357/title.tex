
\title{
  \vspace{1em}
  \begin{flushright}
    {\bf\normalsize
    TTP98-41, BUTP-98/28, hep-ph/9812357}\\[-.5em]
    {\bf\normalsize December 1998}\\[1em]
  \end{flushright}
  \Large \sc
  Automatic Computation of Feynman Diagrams
  }

\author{{\sc R. Harlander}$^a$ and {\sc M. Steinhauser}$^b$
  \\[3em]
   a) Institut f\"ur Theoretische Teilchenphysik,\\
      Universit\"at Karlsruhe, D-76128 Karlsruhe, Germany
  \\[.5em]
   b) Institut f\"ur Theoretische Physik, \\ Universit\"at
      Bern, CH-3012 Bern, Switzerland
}

\date{}
\maketitle

\begin{abstract} 
  \noindent
  Quantum corrections significantly influence the quantities observed in
  modern particle physics. The corresponding theoretical computations
  are usually quite lengthy which makes their automation mandatory.
  This review reports on the current status of automatic calculation
  of Feynman diagrams in particle physics.  The most important
  theoretical techniques are introduced and their usefulness is
  demonstrated with the help of simple examples. A survey over
  frequently used programs and packages is provided, discussing their
  abilities and fields of applications.  Subsequently, some powerful
  packages which have already been applied to important physical
  problems are described in more detail. The review closes with the
  discussion of a few typical applications for the automated computation
  of Feynman diagrams, addressing current physical questions like
  properties of the $Z$ and Higgs boson, four-loop corrections to
  renormalization group functions and two-loop electroweak corrections.
\end{abstract}

\begin{center}
  {\bf\small Keywords}\\
  {\small Feynman diagrams, computer algebra, calculational techniques,
  radiative corrections, quantum chromodynamics, electroweak physics}
\end{center}
