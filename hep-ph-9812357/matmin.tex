%
\subsubsection{\label{sec::matmin}{\tt MATAD} and {\tt MINCER}}
%
Let us examine the three-loop ladder diagram (see Fig.~\ref{figladder})
in order to see in detail how the programs {\tt MATAD} and {\tt MINCER}
work.  We will focus on the case where the external momentum, $q_1$, is
much smaller than the mass of the fermions, $m$.  Application of the
rules for the hard-mass procedure (Section~\ref{subasymp}) shows that it
suffices to perform a Taylor expansion of the integrand in order to
arrive at an expansion in $q_1^2/m^2$.  The resulting integrals to be
evaluated are of the tadpole type, and the calculation can completely be
passed to {\tt MATAD}.  In the opposite limit, $q_1^2\gg m^2$, in
addition to the naive Taylor expansion of the integrand in $m^2/q_1^2$,
the large-momentum procedure generates non-trivial subgraphs.
In general their evaluation requires to use {\tt MATAD} and
{\tt MINCER} in combination.
%
\begin{figure}[hb]
  \begin{center}
    \begin{tabular}{c}
      \epsfxsize=4cm
      \epsffile[131 278 481 514]{dia11rho.ps}
      \smallskip
    \end{tabular}
    \parbox{\captionwidth}{
      \caption[]{\label{figladder}\sloppy 
        Ladder-type diagram contributing to the
        photon propagator. The solid, wavy and curly lines represent
        quarks, photons and gluons, respectively.  }}
  \end{center}
\end{figure}
%

The input for the Feynman graph is written to a file using the fold
construct of {\tt FORM}~\cite{form}:
\begin{verbatim}
*--#[ ladder:
        ((-1)
        *Dg(nu1,nu2,p7)
        *Dg(nu3,nu4,p8)
        *S(mu1,q1,p3m,nu3,q1,p2m,nu1,q1,p1m,mu2,p6m,nu2,p5m,nu4,p4m)
        );

        #define TOPOLOGY "LA"
*--#] ladder:
\end{verbatim}
Some words concerning the notation are in order: The function
\verb/S(...)/ represents the fermion string and contains momenta
(\verb/p1/ $\widehat = \,p_1$, \ldots) and Lorentz indices (\verb/mu1/
$\widehat = \,\mu_1$, \ldots) as its arguments. The momenta are labeled
in accordance with the {\tt MINCER} conventions for the specific
topology which is \verb/LA/ here, meaning ``ladder''. Note that it
is more convenient to express the input in terms of eight momenta and to
use momentum conservation at a later stage than to impose it already in
the input.  The ending \verb/m/ indicates that massive fermions should be
considered.  A small {\tt FORM} program transforms each Lorentz index to
the corresponding $\gamma$ matrix and the momenta to fermion
propagators.  The expansion in the external momentum, \verb/q1/, is
indicated by the consecutive appearance of \verb/q1/ together with an
integration momentum. The depth of the expansion, which is encoded
as a pre-processor variable in {\tt FORM}, must be defined at program
call. \verb/Dg(...)/ represents the gluon propagator.  After reading the
diagram these functions are replaced according to the Feynman rules.
Then {\tt MATAD} evaluates the fermion trace and expands the
denominators with respect to \verb/q1/.  At this point the user may
influence the procedure by providing an additional file, the so-called
``special {\tt treat}-file''. For the matter in hand, this file contains
a projector on the transverse part
of the diagram:
%
\begin{verbatim}
multiply, (-d_(mu1,mu2)+q1(mu1)*q1(mu2)/q1.q1)*deno(3,-2);
\end{verbatim}
%
where \verb/deno(3,-2)/ means $1/(3-2\varepsilon)$.  In this way one
obtains a scalar expression where \verb/q1/ is absent in the
denominators of the propagators. However, it may still appear in scalar
products with integration momenta.  Application of d'Alembertian
operators performs an averaging in the numerator, i.e. terms like
$(q_1\cdot p)^2$ are replaced by $p^2 q_1^2/D$. This projects out the
coefficients of $(q_1^2)^n$ which have no dependence on the external
momenta any more.  After that it is possible to decompose the numerators
in terms of the denominators.  At this stage the expression may already
have split into millions of different integrals.  Recurrence relations
derived with the integration-by-parts technique reduce them to a small
set of tabulated master integrals which have been evaluated once and for
all.  In the case of the photon propagator, for example, all three-loop
integrals reduce to only three massive three-loop master integrals.  The
same is true for the massless case with non-vanishing external momentum,
which is where {\tt MINCER} applies.

In intermediate steps, when the recurrence relations are applied, it may
happen that the expression blows up and disk space up to several
Gigabytes is needed. But the final result is rather compact and fills
only of the order of one output screen (depending, of course, on the
depth of the expansion).


