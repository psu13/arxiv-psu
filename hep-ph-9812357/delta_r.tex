%
% delta_r.tex
%
%
\subsection[Higgs mass dependence of two-loop corrections to $\Delta
r$]{Higgs mass dependence of two-loop corrections to \bld{\Delta
    r}\label{sec:applic:deltar}}
%
%
As was already pointed out in the introduction, the dependence of
radiative corrections on yet undiscovered particles may be exploited in
order to gain information on the masses of these particles.  This has
been performed very successfully for the top quark and nowadays, as
$M_t$ is measured with reasonable accuracy, the same strategy is used to
pin down the mass of the Higgs boson.  An important quantity in this
respect is $\Delta r$, which comprises the radiative corrections to muon
decay. In this way it relates four fundamental parameters of the
Standard Model to each other --- the electromagnetic coupling $\alpha$,
the Fermi constant $G_F$ and the masses of the $W$ and $Z$ boson:
\begin{eqnarray}
M_W^2\left(1-{M_W^2\over M_Z^2}\right) &=& {\pi\alpha\over
    \sqrt{2}G_F}(1+\Delta r)\,.
\end{eqnarray}
$G_F$, $\alpha$ and $M_Z$ are very well measured quantities,
and the experimental accuracy of $M_W$ is expected to be
considerably improved within the near future by the current and upcoming
measurements at LEP2 and TEVATRON.

Therefore, precise knowledge of the $M_H$-dependence of $\Delta r$ is
highly desirable. The one-loop corrections to $\Delta r$ are known
analytically since long~\cite{MarSir80}, and two-loop corrections up to
next-to-leading order in $1/M_t$ were considered
in~\cite{DegGamVic96,DegGamSir97,DegGamPasSir98}.  It was the concern
of~\cite{BauWei97} to precisely account for the $M_H$ dependence of the
two-loop diagrams containing the top quark and the Higgs boson.  The
idea was to get a result valid for arbitrary values of $M_t$ and $M_H$
without performing an expansion in these masses.  This obviously leads
to two-loop multi-scale diagrams, their number being of the order of one
hundred.  Not only the large number of diagrams but also the complicated
tensor structure and the evaluation of the scalar two-loop integrals
makes the use of highly automated software indispensable.  A further
technical problem arises from the renormalization which has to be known
up to two loops in the electroweak sector of the Standard Model.  In
order to investigate the Higgs-mass dependece of $\Delta r$,
in~\cite{BauWei97} the subtracted quantity
\begin{eqnarray}
\Delta r(M_H) - \Delta r(M_H=65~{\rm GeV})
\end{eqnarray}
was considered which describes the change in the prediction for $\Delta
r$ when $M_H$ is varied.  The value for $\Delta r(M_H=65~{\rm GeV})$ was
taken from \cite{DegGamPasSir98}.

It turned out that the computation of the diagrams contributing to the
considered corrections to $\Delta r$ could be reduced to two-loop
tadpole integrals and two-point functions. This apparently is a task for
the {\tt Mathematica} programs {\tt FeynArts} and {\tt TwoCalc}.  The
various contributions to $\Delta r$ together with the corresponding
counter-terms were generated by means of {\tt FeynArts}.  Contraction of
Lorentz indices, evaluation of the Dirac algebra and the two-loop tensor
reduction was performed by {\tt TwoCalc}.  The resulting expressions
were split into finite and divergent pieces which explicitly
demonstrates the cancellation of poles.  The scalar one- and two-loop
integrals were reduced to one-dimensional integrals allowing a fast
numerical evaluation to high accuracy with the help of special {\tt C}
routines which themselves were fully integrated into the {\tt
  Mathematica} environment~\cite{Bauetal}.  CPU time for this
calculation added up to the order of a few days.

Instead of displaying the full result, we only note that the resulting
prediction for the $W$ mass agrees within a few MeV with the one of the
heavy-top expansion~\cite{DegGamPasSir98}, which is far
below the experimental precision at the moment. (For a more detailed
discussion see \cite{WeiBarc98} and \cite{GamBarc98}.) It is remarkable
that the top-induced corrections turn out to be maximal at the physical
value of the top quark mass within a range of $\pm 50$~GeV around this
physical value.  As shown in Fig.~\ref{fig::MH-MW}, the interrelation
between $M_W$ and $M_t$ favours relatively small values for $M_H$,
consistent with other indirect determinations using the radiative shift
in the weak mixing angle or the leptonic decay rate of the $Z$ boson.
Both of these quantities recently have been calculated with similar
techniques as described in this section~\cite{WeiRhein98}.

\begin{figure}
  \begin{center}
    \leavevmode
    \epsfxsize=10.cm
    \epsffile{mwplotcmpl.eps}
    \hfill
    \parbox{\captionwidth}{
    \caption[]{\label{mwplotcmpl.eps}\sloppy
      Limits on the Higgs mass depending on the values of the top and
      $W$ mass~\cite{WeiBarc98}. The value of $\Delta r(M_H=65~{\rm
      GeV})$ was taken from~\cite{DegGamPasSir98}.
      \label{fig::MH-MW}
      }}
  \end{center}
\end{figure}
%

