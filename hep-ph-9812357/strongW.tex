%
%
\subsection{Strongly interacting vector bosons\label{sec:applic:strongW}}

As an application for the package {\tt CompHEP} let us consider a
scenario where the violation of unitarity in elastic scattering of
massive vector bosons at high energies is avoided by letting them become
strongly interacting.  This is an alternative approach to the Standard
Model solution of introducing a scalar Higgs boson. In~\cite{BHKPYZ98} a
new fundamental strong interaction among the vector bosons is assumed
that occurs at a scale of order 1~TeV.  The model is described by a
Lagrangian with global chiral symmetry which is spontaneously broken in
order to generate the masses of the vector bosons.  The corresponding
interaction terms have been worked out and implemented in {\tt CompHEP}.
Although this would have been much simpler in unitary gauge due to the
absence of ghost particles, the authors of~\cite{BHKPYZ98} decided to
use 't~Hooft-Feynman gauge as then the size of the final expressions is
much smaller.  One possibility to test the new interaction terms are
elastic and quasi-elastic $2\to 2$ scattering experiments with $W^\pm$
and $Z$ bosons which requires colliders with a center-of-mass energy in
the TeV range. Indeed, a future $e^+e^-$ linear collider is expected to
operate at energies of 1.5 to 2~TeV in a second stage~\cite{Acc97}.

The following processes are considered
in~\cite{BHKPYZ98}:
\begin{eqnarray}
e^+e^- \to \nu_e \bar{\nu}_e W^+ W^-
&\quad:&
W^+ W^- \to W^+ W^-
\nonumber\\
e^+e^- \to \nu_e \bar{\nu}_e Z Z
&\quad:&
Z Z \to Z Z
\nonumber\\
e^-e^- \to \nu_e \bar{\nu}_e W^- W^-
&\quad:&
W^- W^- \to W^- W^-
\nonumber\\
e^+e^- \to e^- \bar{\nu}_e W^+ Z
&\quad:&
W^+ Z \to W^+ Z
\nonumber\\
e^+e^- \to e^+ e^- Z Z
&\quad:&
Z Z \to Z Z
\label{eqstrongw1}
\end{eqnarray}
where after the colon the subprocess
characterizing the vector boson scattering is given. However, not only
those diagrams that contain the desired subprocess contribute.
Instead, also background diagrams must be computed where, for example, the
final state vector bosons are radiated off the fermion lines or one of
the massive vector bosons is replaced by a photon.  Furthermore,
diagrams are involved where the neutrinos result from the decay of a $Z$
boson which is in turn generated from a $W^\pm$ pair.  In
Fig.~\ref{figstrongwdia} three sample diagrams are listed.

\begin{figure}[h]
  \leavevmode
  \begin{center}
    \begin{tabular}{ccc}
      \epsfxsize=2.5cm
      \parbox{1cm}{\epsffile[0 0 114 86]{wwgraphs.7}}
      &
      \epsfxsize=2.5cm
      \parbox{1cm}{\epsffile[0 0 114 86]{wwgraphs.9}}
      &
      \epsfxsize=2.5cm
      \parbox{1cm}{\epsffile[0 0 114 86]{wwgraphs.13}}
      \\
      $(a)$&$(b)$&$(c)$
    \end{tabular}
    \parbox{\captionwidth}{
      \caption[]{\label{figstrongwdia}
        Sample diagrams contributing to processes specified in
        Eq.~(\ref{eqstrongw1}). Diagram $(a)$ is part of 
        the signal of the vector boson scattering process whereas $(b)$ and
        $(c)$ belong to the background. 
        }}
  \end{center}
\end{figure}
%

It is straightforward to go through the menus of {\tt CompHEP}, to enter
the different processes, and to produce {\tt FORTRAN} output for the
squared matrix elements.  This is then used in the numerical part of
{\tt CompHEP} to compute the cross sections.  The first run contains no
cuts, but computes all matrix elements corresponding to the generated
set of four-momenta.  In subsequent runs, cuts are applied that reduce
the background and isolate the signal. The data computed in the first
run are taken over so that the CPU time is significantly smaller in this
second step, as already mentioned in Section~\ref{subcomphep}. For
example, the cuts include a lower limit on the invariant mass of the
$\nu_e\bar{\nu_e}$ system, or the selection of central events in
combination with cuts on the transverse momentum of the $W^\pm$,
respectively, the $Z$ boson.  Fig.~\ref{figstrongw} shows how the number
of events containing neutrinos from $Z$ boson decay (c.f.
Fig.~\ref{figstrongwdia}($c$)) is reduced.  Figures of this kind can
easily be produced by using the data files generated by {\tt CompHEP}.

\begin{figure}
\begin{center}
\includegraphics{wwdist.1}
%
% needs feynmp.sty!
%
\parbox{\captionwidth}{
\caption[]{\label{figstrongw}
  Distribution of the invariant mass of the produced vector bosons
  (here generically called $W$) 
  in the process $e^+e^-\to W^+W^-\bar\nu_e\nu_e$ (signal).
  The cut (shaded area) removes events in which the neutrinos are
  generated through $Z$ decays.}} 
\end{center}
\end{figure}

This is not the place to go into details concerning the physical
consequences for which we shall refer to~\cite{BHKPYZ98}.  The key
point, however, is that the couplings of the new interaction terms can
be related to the $2\to 2$ vector boson scattering amplitudes.  Thus,
the investigation of the corresponding total cross sections as well as
several other distributions like the one for invariant masses or
transverse momenta provide important tests on the mechanism responsible
for restoring unitarity at high energies.

%
% end of strongW.tex
%


