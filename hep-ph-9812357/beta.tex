%
\subsection[Four-loop $\beta$ function]{Four-loop \bld{\beta} 
  function\label{sec::beta}}
%
%
The $\beta$ function is an object of common interest in any field
theory, especially in non-Abelian ones. It describes the dependence of
the corresponding coupling constant with respect to the energy scale.
For QCD it is convenient to define
\begin{eqnarray}
\mu^2{\dd\over \dd\mu^2}\frac{\alpha_s(\mu^2)}{\pi} &=& 
\beta(\alpha_s)
\,\,=\,\,
-\left(\frac{\alpha_s(\mu^2)}{\pi}\right)^2
\sum_{i\ge0}\beta_i\left(\frac{\alpha_s(\mu^2)}{\pi}\right)^i
\,.
\end{eqnarray}
The $\beta$ function is directly related to the renormalization factor
$Z_g$ defined through $\alpha_s^0
= Z_g^2\alpha_s $, where $\alpha_s^0$ and $\alpha_s$ are
the bare and renormalized coupling constant of QCD, respectively:
\begin{eqnarray}
  \beta(\alpha_s)= {\alpha_s^2\over \pi}
  {\partial\over \partial\alpha_s} Z_g^{2,(1)}\,,
\label{eqbeta}
\end{eqnarray}
where $Z_g^{2,(1)}$ is the residue of $Z_g^2$ with respect to its
Laurent expansion in $\varepsilon$.
In Eq.~(\ref{eqbeta}) it is already indicated that in mass independent
renormalization schemes, like the \msbar\ scheme, the renormalization
constants and thus also the $\beta$ function only depend on $\alpha_s$.

There are several ways to compute $Z_g$, all of them related through
Slavnov-Taylor identities.  One choice would be to combine the
renormalization constants of the four-gluon vertex and the gluon
propagator.  At four-loop level, however, this requires the computation
of about half a million diagrams.  On the other hand, the approach
of~\cite{RitVerLar97} was based on the relation
\begin{eqnarray}
Z_g &=& \frac{\tilde{Z}_1}{\tilde{Z}_3\sqrt{Z_3}}
\,,
\end{eqnarray}
where $\tilde{Z}_1$ is the renormalization constant of the
ghost-ghost-gluon vertex, and $\tilde{Z}_3$ and $Z_3$ are the ones of
the ghost and gluon propagators, respectively.  Thus the pole parts of
the corresponding Green functions had to be computed up to four loops.

The roughly 50,000 contributing diagrams were generated by means of {\tt
  QGRAF}.  Introducing an artificial mass $M$ in each propagator as
infra-red regulator allowed a Taylor expansion to be performed with
respect to all external momenta, leading to four-loop massive tadpole
integrals.  Recurrence relations based on the integration-by-parts
algorithm were implemented in the program {\tt BUBBLES}~\cite{bubbles}
to reduce the integrals to a minimal set of master integrals. At one-,
two- and three-loop level only one, at four-loop level two master
integrals are required.  The colour factors of the individual diagrams
were determined with the help of a {\tt FORM} program~\cite{color}.  To
cope with the huge number of diagrams a special database-like tool
called {\tt MINOS} was developed. It controls the calculation and allows
to conveniently access the results of single diagrams.  The total CPU
time for this calculation was of the order of a few months.
 
The final result which we will only quote for the physically most
interesting case of QCD reads
\begin{eqnarray}
\renewcommand{\arraystretch}{ 1.3}
 \beta_0 & = & {1\over 4}\left(11 - \frac{2}{3} n_f \right)\,,  \nonumber\\
\beta_1 & = & {1\over 16}\left( 102 - \frac{38}{3} n_f \right)\,, \nonumber \\
 \beta_2 & = & {1\over 64}\left( \frac{2857}{2} - \frac{5033}{18} n_f +
 \frac{325}{54} n_f^2 \right)\,, \nonumber \\
 \beta_3 & = &  {1\over 256} \bigg[
 \left( \frac{149753}{6} + 3564 \zeta_3 \right)
        - \left( \frac{1078361}{162} + \frac{6508}{27} \zeta_3 \right) n_f
  \nonumber \\ & &
       + \left( \frac{50065}{162} + \frac{6472}{81} \zeta_3 \right) n_f^2
       +  \frac{1093}{729}  n_f^3
       \bigg]\,,
\end{eqnarray}
where $n_f$ is the number of active flavours.

%\bibitem{minos}
%J.A.M. Vermaseren, proceedings ...;\\
%see also:
%T. van Ritbergen, J.A.M. Vermaseren, S.A. Larin and P. Nogueira,
%{\it Int. J. Mod. Phys.} {\bf C 6} (1995) 513.
