
\section{Applications\label{secapplications}}
%
This section reports on recent results that would have been impossible
to obtain without the use of computer algebra. The selected examples are
supposed to underline the fields of applications for some of the
programs described above.

The first example is concerned with one of the most classical subjects
for multi-loop calculations, the photon polarization function. We will
describe its evaluation, using the program {\tt MATAD} in the limit
$q^2\ll m^2$, and additionally {\tt MINCER} and {\tt LMP} in the
opposite case, $q^2\gg m^2$.
The technique of repeatedly applying the hard-mass procedure was
successfully applied to the decay rate of the $Z$ boson into $b$ quarks
using {\tt EXP} in the {\tt GEFICOM} environment.
The calculation of renormalization group functions was the
driving item for {\tt BUBBLES} to be developed.
As part of the ``NIKHEF-setup'' it was used to compute four-loop
tadpole diagrams up to their simple poles in $\varepsilon$.
Another example for the calculation of a four-loop quantity is the decay
rate of the Higgs boson into gluons. Applying the approach of an effective
Hamiltonian, {\tt GEFICOM} could be used to compute the corresponding
diagrams.
As an application of the multi-leg program {\tt CompHEP} the
scenario of a strong interaction among electroweak gauge bosons will be
discussed. It is traded as an alternative to the Higgs mechanism to
restore unitarity at high energies.
Calculations in the full electroweak theory usually involve a large
number of different scales. For important contributions to the parameter
$\Delta r$ it was possible to reduce the contributing diagrams
to two-point functions which then were accessible with the help of the
programs {\tt FeynArts} and {\tt TwoCalc}.


