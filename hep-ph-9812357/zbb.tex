\subsection[Corrections of ${\cal O}(\alpha\alpha_s)$ to the decay of
the $Z$ boson into bottom quarks]{Corrections of \bld{{\cal
      O}(\alpha\alpha_s)} to the decay of the \bld{Z} boson into bottom
  quarks\label{sec::zbb}}
%
In the examples considered so far essentially only the QCD part of the
Standard Model is involved, which is, as already noted, the most
important field for multi-loop calculations. The electroweak sector, on
the contrary, resides on a broken gauge symmetry which is the reason why
the diagrams in general carry a lot more scales. Higher order
calculations are thus much more involved.

As an in some sense intermediate case, one may consider mixed
electroweak/QCD corrections. An important example in this class are the
${\cal O}(\alpha\alpha_s)$ corrections to the decay rate of the $Z$
boson into quarks.  One may distinguish two cases, namely the decay into
the light $u,d,s$ or $c$ quarks and the one into $b\bar{b}$.  The latter
case is of special interest as the top quark enters the vertex diagrams
already at one-loop order and leads to corrections proportional to
$M_t^2$.  The complete ${\cal O}(\alpha)$ corrections were considered
in~\cite{AkhBarRie86,BeeHol88}.  Since the mixed ${\cal
  O}(\alpha\alpha_s)$ correction to the $Z$ decay into $u,d,s$ and $c$
turned out to be quite sizable~\cite{CzaKue96}, it was tempting to
consider also the decay into bottom quarks at this order.  The leading
terms of ${\cal O}(G_F M_t^2)$ and ${\cal O}(G_F \ln(M_t^2/M_W^2))$ were
computed in~\cite{FleJegRacTar92,CheKwiSte93,KwiSte95,Per95}.

However, the full $M_t$ dependence to
${\cal O}(\alpha\alpha_s)$ is currently out of reach.  On the other
hand, the top quark is certainly the heaviest particle entering the $Z b
\bar b$ vertex at this order, which makes asymptotic expansions a
well suited tool to obtain a very good approximation to the full answer.

Let us describe the technical aspects of the calculation for $Z\to b
\bar b$ in more detail. The ${\cal O}(\alpha\alpha_s)$ corrections are
computed by applying the optical theorem, i.e., by evaluating the
imaginary part of the $Z$ boson propagator.
%
\begin{figure}[t]
  \begin{center}
    \leavevmode 
    \epsffile[105 558 503 717]{zbbdias.ps}\\
    \parbox{\captionwidth}{
      \caption[]{\label{fig::zbbdias}Diagrams containing a top quark that 
        contribute to $Z\to b\bar b$. Thin lines correspond to bottom
        quarks, thick lines to top quarks, dotted lines to Goldstone
        bosons and inner wavy lines represent $W$ bosons.}}
  \end{center}
\end{figure}
%
Our concerns are only the diagrams involving the top quark.  The ones
contributing to ${\cal O}(\alpha)$ are shown in Fig.~\ref{fig::zbbdias}.
The diagrams to be evaluated to ${\cal O}(\alpha\alpha_s)$ can be
obtained by attaching a gluon in all possible ways, thereby increasing
the number of loops by one.

The different scales in these diagrams are given by $M_t,M_W,M_Z$ and
$M_\Phi$. The latter one is the mass of the charged Goldstone boson. It
is related to $M_W$ via $M_\Phi^2=\xi_W M_W^2$, where $\xi_W$ is the
corresponding gauge parameter which is
arbitrary and should drop out in the final result.  In a first step we
will consider it such that $M_t^2 \gg M_\Phi^2$. For the diagrams (a),
(b), (d) and (e) the application of the hard-mass procedure with respect
to $M_t$ immediately leads to a complete factorization of the
different scales, i.e., only single-scale integrals with at most three
loops are left. For the diagrams (c), (f) and (g), however, some
co-subgraphs still carry more than one scale which makes an evaluation
quite painful, especially as their ${\cal O}(\varepsilon)$ part is also
needed.  It is very suggestive to apply the hard-mass procedure again to
these co-subgraphs.  In order to avoid unwanted imaginary parts arising
from $W$ and $\Phi$ cuts the hierarchy to be chosen among the physical
masses is $M_t^2 \gg M_W^2 \gg M_Z^2$.  The last inequality is seemingly
inadequate, but a closer look at the corresponding diagrams shows that
in this context it is actually equivalent to $4M_W^2 \gg M_Z^2$,
respectively, $(M_W+M_t)^2 \gg M_Z^2$. The mass of the Goldstone boson
may be incorporated using the possible hierarchies $\xi_WM_W^2\gg M_W^2
\gg M_Z^2$ or $M_W^2 \gg \xi_W M_W^2 \gg M_Z^2$.  This procedure leads
to a nested series in $M_W^2/M_t^2$ and $M_Z^2/M_W^2$.
Diagrammatically, an example for this procedure of successively applying
asymptotic expansions looks as follows:
\begin{eqnarray*}
&& \epsfxsize=7em
\raisebox{-2em}{\epsffile[120 260 460 450]{zbblhs.ps}}
%
\stackrel{M_t^2\to \infty}{\longrightarrow}
\,\,\,\,\,
\epsfxsize=8em \raisebox{-1.7em}{\epsffile[120 260 560 450]{zbbrhs1.ps}}
  \star
%
\epsfxsize=.9em
\raisebox{-1.7em}{\epsffile[260 260 310 450]{zbbrhst.ps}}
\,\,+ \,\,\cdots \\[.5em]&&\hspace{8em}
%
\stackrel{\xi M_W^2\to \infty}{\longrightarrow}
%
  \bigg(\!\!\!\!
  \epsfxsize=7em
  \raisebox{-2em}{\epsffile[120 260 460 450]{zbbrhs2.ps}}
  \star\!\!\!\!\!
  \epsfxsize=7em \raisebox{-2em}{\epsffile[120 260 460
  450]{zbbrhs2c.ps}} \hspace{-1em}\bigg)
\star
\epsfxsize=.9em
\raisebox{-1.7em}{\epsffile[260 260 310 450]{zbbrhst.ps}} + \cdots\,,
\end{eqnarray*}
%
where only those terms are displayed which are relevant in the
discussion above and all others contributing to the hard-mass procedure
are merged into the three dots. The mass hierarchy is assumed to be
$M_t^2\gg \xi_W M_W^2 \gg M_W^2 \gg M_Z^2$.

The technical realization of the calculation was performed with the help
of {\tt GEFICOM} (see Section~\ref{subgeficom}). The main new ingredient
in comparison to the examples quoted so far is the use of the automated
version of the hard-mass procedure and its successive application,
implemented in the program {\tt EXP}, in combination with the generator
{\tt QGRAF} and the {\tt FORM} packages {\tt MINCER} and {\tt MATAD}.
This automation not only avoids human errors, but also allows several
checks of the results. For example, since the gauge parameter $\xi_W$ is
arbitrary, it is possible to choose it such that no longer the top
quark, but rather the Goldstone boson is the heaviest particle involved
in the process. Then one is left with the hierarchy $\xi_W M_W^2\gg
M_t^2 \gg M_W^2\gg M_Z^2$.  Note that for this choice the hard-mass
procedure produces completely different subdiagrams than for the cases
above, where $M_t$ was considered to be the largest scale.  Furthermore,
while a manual treatment of both cases would almost double the effort,
the automated version allows one to go from one hierarchy to the other by
simply interchanging two input parameters, as we will see below.

To be concrete, let us consider the input file needed for {\tt GEFICOM}.
It is certainly more sophisticated than the examples mentioned so far,
but this is mainly due to the enriched
particle spectrum.
%
\begin{verbatim}
*** MINCER

* scheme 2
* gauge 1
* exp y ma mc mb q
* powerma 2
* powermb 4
* powermc 4
* mass t Ma 
* mass Wp Mb
* mass pp Mc
* loops 1 true=iprop[t,0,0];
* loops 2 true=iprop[g,0,0];
* loops 2 false=iprop[t,0,0];
* loops 3 true=iprop[g,1,1];
* loops 3 false=iprop[t,0,0];

      list = symbolic ;
      lagfile = 'q.lag' ;
      in = Z[q];
      out = Z[q];
      nloop = ;
      options = onepi;
      true = iprop[Wm,pm,0,2];
      false = iprop[bq,0,1];
\end{verbatim}
%
The meaning of most of the lines can be deduced by comparison with the
example of Section~\ref{subgeficom}.
The mass of the top quark ({\tt t} $\widehat =
\,t$), the $W$ ({\tt Wp} $\widehat = \,W^+$), and the Goldstone boson
({\tt pp} $\widehat =\, \Phi^+$) are denoted by {\tt Ma}, {\tt Mb} and
{\tt Mc}, respectively.  The change in mass hierarchy mentioned before is
achieved by rewriting the line
\begin{verbatim}
* exp y ma mc mb q
\end{verbatim}
to
\begin{verbatim}
* exp y mc ma mb q
\end{verbatim}
This input file is combined together with the one containing
the vertices and propagators, which shall not be displayed here,
in order to generate in a first step the relevant diagrams
at three-loop level. After inserting the Feynman
rules, {\tt EXP} applies
the hard-mass procedure and reduces all diagrams to single-scale
integrals, if necessary by applying the procedure twice.
{\tt EXP} also produces the relevant administrative files which
then call {\tt MINCER} and {\tt MATAD}. 
The runtime for {\tt QGRAF} is of the order of a few seconds
and it takes a few minutes for {\tt EXP} to generate the
subdiagrams.
The time spent in the integration routines strongly depends on the
required depth of the expansion, of course. The computation
in~\cite{HarSeiSte97} took about three weeks.

The contribution of the vertex corrections to the decay rate of the $Z$
boson to bottom quarks, induced by the exchange of a $W$ or a Goldstone
boson, may be written in the following way:
\begin{equation}
\delta\Gamma_b^W = 
\delta\Gamma_d^W + (\delta\Gamma_b^W - \delta\Gamma_d^W) = 
\delta\Gamma_d^W + (\delta\Gamma_b^{0,W} - \delta\Gamma_d^{0,W}) = 
\delta\Gamma_d^W + \delta\Gamma^W_{b-d}\,,
\label{eq::delgamb}
\end{equation}
where $\delta\Gamma_q^{W}$ and $\delta\Gamma_q^{0,W}$ denote
renormalized and unrenormalized quantities. In the difference between
the $b$ and $d$ contributions the relevant counter-terms drop out.  This
means that $\delta\Gamma^W_{b-d}$ is independent of the renormalization
scheme and is therefore well suited for installation in data analyzing
programs like the ones described, e.g., in~\cite{YelRep95}. In addition,
$\delta\Gamma^W_{b-d}$ is also gauge independent. $\delta\Gamma_d^W$ has
been computed in~\cite{CzaKue96}.  
For convenience we list the result for $\delta\Gamma^W_{b-d}$ in
numerical form~\cite{HarSeiSte97}:
\begin{eqnarray}
\delta\Gamma^W_{b-d} &=&
    \Gamma^0 {1\over s^2_\theta}
    {\alpha\over \pi}
%
%
  \bigg\{ - 0.50
%  \nonumber\\[.5em]&&\hbox{\hspace{1em}}
  + (0.71 -0.48)+ (0.08 - 0.29) + (-0.01 - 0.07) + (-0.007 - 0.006)
%
  \nonumber\\[0em]&&\mbox{}
%
  + {\alpha_s\over \pi} \bigg[ 1.16 + (1.21 - 0.49) + (0.30 - 0.65) +
  (0.02 - 0.21 + 0.01)
%
%\nonumber\\&&\mbox{\hspace{1em}}
%
+ (-0.01 - 0.04
  + 0.004) \bigg] \bigg\} =
%
\nonumber\\
%
&=&\Gamma^0 {1\over s^2_\theta} {\alpha\over \pi} \bigg\{- 0.50 - 0.07 +
{\alpha_s\over \pi} \bigg[ 1.16 + 0.13 \bigg]\bigg\}\,,
\label{eq::zbb1l2l}
\end{eqnarray}
with $\Gamma^0 = N_cM_Z\alpha/(12s_\theta^2c_\theta^2$), $s_\theta =
\sin\theta_W$, $\theta_W$ being the weak mixing angle, $c_\theta^2 =
1-s_\theta^2$, and $M_t$ the on-shell top mass.  For
Eq.~(\ref{eq::zbb1l2l}) we used the values $M_t = 175$~GeV,
$M_Z=91.19$~GeV and $s_\theta^2 = 0.223$.  The numbers after the first
equality sign correspond to successively increasing orders in $1/M_t^2$,
where the brackets collect the corresponding constant, $\ln
(M_t^2/M_W^2)$ and, if present, $\ln^2(M_t^2/M_W^2)$ terms. The numbers
after the second equality sign represent the leading $M_t^2$ term and
the sum of the subleading ones.  The ${\cal O}(\alpha)$ and ${\cal
  O}(\alpha\alpha_s)$ results are displayed separately. Comparison of
this expansion of the one-loop terms to the exact result of
\cite{BeeHol88} shows agreement up to $0.01\%$ which gives quite some
confidence in the $\alpha\alpha_s$ contribution.  One can see that
although the $M_t^2$, $M_t^0$ and $M_t^0\ln(M_t^2/M_W^2)$ terms are of the same
order of magnitude, the final result is surprisingly well represented by
the $M_t^2$ term, since the subleading terms largely cancel among each
other.

