
\subsection{Hadronic Higgs decay\label{sec:applic:hgg}}
%
%
%%%%%%%%%%%%%%%%%%%%%%%%%%%%%%%%%%%%%%%%%%%%%%%%%%%%%%%%%%%%

\subsubsection{Introduction}
%
As was already pointed out in the introduction, the only particle of the
Standard Model not yet discovered is the Higgs boson.  Thus it is
necessary to learn as much as possible about potential production and
decay mechanisms of such a particle in order to correctly interpret the
signals in the detector.  An important decay channel of the Higgs boson
is the one into gluons. Although it is a loop-induced process it is
numerically not negligible mainly due to the fact that the ${\cal
  O}(\alpha_s)$ corrections are very
large~\cite{InaKubOka83,DjoSpiZer91}. In this section it is
shown how to compute the ${\cal O}(\alpha_s^2)$ corrections exploiting
the powerful tools of automatic Feynman diagram computation.  For
completeness also the decay into quarks will be discussed.

The Higgs boson will be assumed to be of intermediate-mass range, i.e.,
$M_H\lsim 2M_W$, which suggests the use of the approach of an effective
Lagrangian where the top quark is integrated out. For convenience
of the reader let us, in a first step, collect the relevant formulae.
Detailed derivations can be found
in~\cite{Klu75,Spi84,InaKubOka83,CheKniSte97hbb}.

The starting point is the Yukawa Lagrange density describing the
coupling of the $H$ boson to quarks.  In the limit $m_t\to\infty$ it can
be written as a sum over five operators~\cite{Klu75,Spi84} formed by the
light degrees of freedom, with the corresponding coefficient functions
containing the residual dependence on the top quark:
\begin{eqnarray}
{\cal L}_Y\,\,=\,\,-\frac{H^0}{v^0}
\sum_{q\in\{u,d,s,c,b,t\}}m_{q}^0\bar\psi_{q}^0\psi_{q}^0
&\longrightarrow&
{\cal L}_{\rm eff}\,\,=\,\,
-\frac{H^0}{v^0}\sum_{i=1}^5C_i^0{\cal O}_i^\prime
\,.
\label{eqlagr}
\end{eqnarray}
It turns out that only two of the operators, in the following called
${\cal O}_1^\prime$ and ${\cal O}_2^\prime$, contribute to physical processes.
Expressed in terms of bare fields they read
\begin{eqnarray}
{\cal O}^\prime_1
\,\,=\,\,
\left(G^{0\prime,a}_{\mu\nu}\right)^2\,,
&\quad&
{\cal O}^\prime_2
\,\,=\,\,
\sum_{q\in\{u,d,s,c,b\}}m_{q}^{0\prime}
  \bar\psi_{q}^{0\prime}\psi_{q}^{0\prime}
\,,
\end{eqnarray}
where $G^{0\prime,a}_{\mu\nu}$ is the gluonic field strength tensor.
The renormalization matrix for the coefficient functions 
can be expressed in terms of the well-known renormalization
constants of QCD and may be found in the papers cited above.

To evaluate the ingredients entering the calculation of
$\Gamma(H\to\mbox{hadrons})$ at three-loop level, namely $C_i^0$ and the
imaginary part of the correlators $\langle{\cal O}'_i{\cal O}'_j\rangle$,
a large amount of diagrams must be calculated. Thus it is necessary to
use both a generator for the Feynman diagrams and an efficient interface
transforming the output to a format readable by the integration packages.
In the matter at hand one is either confronted with
propagator-type diagrams or with tadpole integrals where the scales are
given by $M_H$ and $M_t$, respectively.  Therefore the package {\tt
  GEFICOM} in association with {\tt QGRAF} (for the generation) and {\tt
  MATAD/MINCER} (for the computation) is an ideal candidate for the
evaluation of the Feynman diagrams contributing to
$\Gamma(H\to\mbox{hadrons})$.

In the next section we will give an example of the calculation of three-loop
objects in the case of $C_1^0$ in some detail and concentrate in
Section~\ref{subhggcol} on the evaluation of the massless correlators.
{\tt GEFICOM} was even used to compute the ${\cal O}(\alpha_s^3)$
contribution to $\langle {\cal O}'_2 {\cal O}'_2\rangle$ which actually
is a four-loop calculation. It is shown how such a calculation is
nevertheless possible using the three-loop tools provided by {\tt
  GEFICOM}.  Finally, in Section~\ref{subhggres} the physical results
are briefly presented.

%%%%%%%%%%%%%%%%%%%%%%%%%%%%%%%%%%%%%%%%%%%%%%%%%%%%%%%%%%%%

\subsubsection{Coefficient functions}
%
The top quark dependence of the effective Lagrangian is contained in the
coefficient functions $C_i^0$. Thus it is not surprising that their
evaluation can be reduced to massive tadpole integrals with the scale
given by $M_t$.  As an example let us consider $C_1^0$.  There are three
independent ways to obtain $C_1^0$, namely from the sets of $ggH$
three-point, $gggH$ four-point, or $ggggH$ five-point diagrams.  At
three-loop level, these sets contain 657, 7362, and 95004 diagrams,
respectively. A sample diagram of each class is displayed in
Fig.~\ref{fighggc1} (together with a diagram to be evaluated for
$C_2^0$). The first set requires an expansion up to second order in the
external momenta, whereas for the second one it suffices to keep only
linear terms. For the diagrams of the third class all external momenta
may be set to zero from the very beginning. However, this advantage is
counterbalanced by the huge number of contributing diagrams.  Note that the
depth of the expansion is given by the individual terms of the operator
${\cal O}_1$ which dictate the
structure in the external momenta.
%
\begin{figure}[t]
\begin{center}
\leavevmode
\epsfxsize=15cm
\epsffile[76 635 552 724]{agg1fig.ps}\\
\parbox{\captionwidth}{
\caption[]{\label{fighggc1}
  Typical Feynman diagrams contributing to the coefficients
  $C_1^0$ and $C_2^0$.  Looped, bold-faced, and dashed lines represent
  gluons, $t$ quarks, and $H$ bosons, respectively.}}
\end{center}
\end{figure}

Ref.~\cite{CheKniSte97hgg} deals with the $ggH$ channel, using the package {\tt
  GEFICOM} to evaluate the 657 diagrams.  The file defining the process
is rather short because only the package {\tt MATAD} has to be
used:
\begin{verbatim}
*** MATAD
* power 2
* gauge 1
* dala12

      list = symbolic ;
      lagfile = 'q.lag' ;
      in = h[q1p2];
      out = g[q1], g[q2]; 
      nloop = ;
      options =;
      true = bridge[ g,q,c, 0,0 ];  
\end{verbatim}
The second line indicates that an expansion in the external momenta up
to second order should be performed, and {\tt dala12} prompts the
application of projectors for the $\sprod{q_1}{q_2}$ terms since we are dealing
with on-shell gluons ($q_1^2=q_2^2=0$).  The file containing the
propagators and vertices is essentially identical to the pure QCD case
(cf.~Section~\ref{subgeficom}) except for the additional coupling of the Higgs
boson to top quarks.

If one chooses a covariant gauge with arbitrary gauge parameter, the CPU
time for the calculation is of the order of a few days.  This is
drastically reduced after going, for example, to Feynman gauge.
Recently also the decay of a pseudo-scalar Higgs boson into gluons was
considered~\cite{CheKniSteBar98}. The number of diagrams being the same
as in the scalar case, there is a slight complication according to the
treatment of $\gamma^5$ in this case. Therefore the 7362 diagrams of the
three-gluon channel were evaluated in Feynman gauge to have a stringent
check of the calculation at one's disposal. The CPU time amounts to
roughly two weeks.  Let us refrain from listing the
analytical results and refer to \cite{CheKniSte97hgg,CheKniSte98dec}
instead.  Nevertheless, they are incorporated in the numerical formula
given in Section~\ref{subhggres}.

%%%%%%%%%%%%%%%%%%%%%%%%%%%%%%%%%%%%%%%%%%%%%%%%%%%%%%%%%%%%

\subsubsection{\label{subhggcol}Correlators}

The computation of the one-, two- and three-loop correlators
$\langle{\cal O}_i^\prime{\cal O}_j^\prime\rangle$ proceeds in close
analogy to the coefficient functions described in the previous section.
Therefore, let us in this subsection focus on the computation of the
imaginary part of the four-loop contribution to $\langle{\cal
  O}_2^\prime{\cal O}_2^\prime\rangle$ which was originally performed
in~\cite{Che97higgs}. Some sample diagrams are shown in
Fig.~\ref{figo2o2}. Although at first sight this seems to require the
evaluation of four-loop diagrams, the calculation can be reduced to
massless propagator-type and massive tadpole integrals at three-loop level.
%
\begin{figure}[t]
  \begin{center}
    \leavevmode
    \epsfxsize=14.0cm
    \epsffile[125 640 500 730]{fighqq.ps}\\
    \parbox{\captionwidth}{
      \caption{\label{figo2o2}
        Typical Feynman diagrams contributing to
        $\langle{\cal O}_2^\prime{\cal O}_2^\prime\rangle$.
        The solid circles represent the operator
        ${\cal O}_2^\prime$.}}
  \end{center}
\end{figure}
%
%
The underlying idea is quite simple: since the imaginary part of
massless propagator-type diagrams only arises from $\ln(-q^2)$ terms and
the logarithms are in one-to-one correspondence to the poles in
$\varepsilon$ it suffices to compute the divergent parts of the
integrals. It is a feature of dimensional regularization accompanied
with MS-like schemes that the ultra-violet poles and thus the
renormalization constants are polynomials in the masses and external
momenta.  Consequently, for logarithmically divergent diagrams the poles
are independent of any mass scale.  This allows us to assign additional
masses to individual lines of the diagrams and to nullify the external
momentum.  If we take one of the lines at the left vertex to be massive
and set the external momentum to zero, the four-loop integrals can be
expanded in terms of three-loop propagator-type diagrams and one-loop
tadpole integrals for which the technology was described in some detail
above.  The drawback of the described method is that artificial
infra-red singularities are introduced which, of course, influence the
pole structure. They have to be removed using the so-called Infra-red
Rearrangement~\cite{Vla80CheKatTka80} which heavily relies on the
$R^*$ operation.  It was developed on a diagram-by-diagram basis
in~\cite{CheSmi84}.  In order to be able to perform the calculation
automatically a global version of the $R^*$ operation is
mandatory~\cite{Che97R,Che97higgs,Che97radcor96}.

In~\cite{Che97higgs} the computation of
$\Im\langle{\cal O}_2^\prime{\cal O}_2^\prime\rangle$
is presented and explicit formulae are given, demonstrating
that the combination of one-, two- and three-loop
massless propagator-type and massive vacuum integrals are
adequate to get the ${\cal O}(\alpha_s^3)$ corrections.
The computation was performed with the help of {\tt GEFICOM}
which had to be used at four-loop order.
A proper flag in the file defining the process tells
{\tt GEFICOM} to introduce the auxiliary mass and to rename
the momenta in accordance with the factorization taking place.
Again we refrain from listing explicit expressions and refer
to Section~\ref{subhggres} for numerical results.

%%%%%%%%%%%%%%%%%%%%%%%%%%%%%%%%%%%%%%%%%%%%%%%%%%%%%%%%%%%%

\subsubsection{\label{subhggres}Results}

This section summarizes the current knowledge of hadronic Higgs decay by
presenting numerical results for the decay rates $\Gamma(H\to gg)$ and
$\Gamma(H\to b\bar{b})$.

Note that $C_1$ starts at order $\alpha_s$. Hence the combination
$C_1^2\mbox{Im}\langle {\cal O}_1^\prime {\cal O}_1^\prime\rangle$
governing the gluonic decay rate of the Higgs boson is available up to
${\cal O}(\alpha_s^4)$. Normalized to the Born rate it reads 
in numerical form ($\mu=M_H$):
\begin{eqnarray}
\frac{\Gamma(H\to gg)}{\Gamma^{\rm Born}(H\to gg)}
&\approx&
1+17.917\,\frac{\alpha_s^{(5)}(M_H)}{\pi}
+\left(\frac{\alpha_s^{(5)}(M_H)}{\pi}\right)^2
\left(156.808-5.708\,\ln\frac{M_t^2}{M_H^2}\right)
\nonumber\\
&\approx&
1+0.66+0.21
\,,
\label{eqhggfin}
\end{eqnarray}
with $\Gamma^{\rm Born}(H\to gg)
      =G_FM_H^3/(36\pi\sqrt2) \times (\alpha_s^{(5)}(M_H)/\pi)^2$.
In the last line $M_t=175$~GeV and $M_H=100$~GeV is inserted.
The analytical expressions can be found in~\cite{CheKniSte97hgg}.
We observe that the new ${\cal O}(\alpha_s^2)$ term further increases the
well-known ${\cal O}(\alpha_s)$ 
enhancement~\cite{InaKubOka83,DjoSpiZer91} by about one third.
If we assume that this trend continues to ${\cal O}(\alpha_s^3)$ and beyond,
then Eq.~(\ref{eqhggfin})
may already be regarded as a useful approximation to the
full result.
Inclusion of the new ${\cal O}(\alpha_s^2)$ correction leads to an increase of 
the Higgs-boson hadronic width by an amount of order 1\%.

Concerning the decay rate into quarks we restrict ourselves
to the case of bottom quarks. Inserting numerical values into the 
coefficient functions $C_1$ and $C_2$ and the correlators
$\mbox{Im}\langle {\cal O}_1^\prime {\cal O}_2^\prime\rangle$~\cite{CheSte97}
and
$\mbox{Im}\langle {\cal O}_2^\prime {\cal O}_2^\prime\rangle
$~\cite{Che97higgs,CheSte97}
leads to:
\begin{eqnarray}
\Gamma(H\to b\bar{b}) 
&=& 
A_{b\bar{b}}\Bigg\{
1
+ 5.667 \,  a_H^{(5)}
+ 29.147 \, \left(a_H^{(5)}\right)^2
+ 41.758 \left(a_H^{(5)}\right)^3
\label{eqbb}
\\&&\mbox{}
+\left(a_H^{(5)}\right)^2
\left[
3.111
-0.667\,L_t
\right] 
+\left(a_H^{(5)}\right)^3
\left[
50.474
-8.167\,L_t
-1.278\,L_t^2
\right]
\Bigg\},
\nonumber
\end{eqnarray}
with $A_{b\bar b}=3G_FM_Hm_b^2/(4\pi\sqrt{2})$,
$L_t=\ln M_H^2/M_t^2$ and $a_H^{(5)}=\alpha_s^{(5)}(M_H)/\pi$.
In Eq.~(\ref{eqbb})
electromagnetic~\cite{Kat97} and 
electroweak~\cite{KniSte95,CheKniSte97hbb}
corrections have been neglected. Also 
mass effects~\cite{HarSte97}
and second order QCD corrections which are suppressed by the 
top quark mass~\cite{LarRitVer95CheKwi96}
are not displayed.
One observes from Eq.~(\ref{eqbb}) that the top-induced corrections
at ${\cal O}(\alpha_s^3)$ (second line) are of the same order of magnitude
as the ``massless'' corrections (first line).



%%%%%%%%%%%%%%%%%%%%%%%%%%%%%%%%%%%%%%%%%%%%%%%%%%%%%%%%%%%%

%\bibitem{Vla80CheKatTka80}
%A.A. Vladimirov, \tmf{43}{1980}{210};\\
%K.G. Chetyrkin, A.L. Kataev and F.V. Tkachov, \npb{174}{1980}{345}.

%\bibitem{CheSmi84}
%K.G. Chetyrkin and F.V. Tkachov, \plb{114}{1982}{340}.
%K.G. Chetyrkin and V.A. Smirnov, \plb{144}{1984}{419}.

%\bibitem{Che91}
%K.G. Chetyrkin, preprint MPI-PAE/PhT 13/91 (Munich, 1991).

%\bibitem{Che97higgs}
%K.G. Chetyrkin, \plb{390}{1997}{309}.

%\bibitem{Che97R}
%K.G. Chetyrkin, \plb{391}{1997}{402}.

%\bibitem{Che97radcor96}
%K.G. Chetyrkin, \app{28}{1997}{725}.

%\bibitem{CheSte97}
%K.G. Chetyrkin and M. Steinhauser, \plb{408}{1997}{320}.

%\bibitem{KniSte95}
%B.A. Kniehl and M. Steinhauser, \npb{454}{1995}{485}; \plb{365}{1996}{297}.

%\bibitem{CheKniSte97hbb}
%K.G. Chetyrkin, B.A. Kniehl and M. Steinhauser,
%\prl{78}{1997}{594}; \npb{490}{1997}{19}.

%\bibitem{InaKubOka83}
%T. Inami, T. Kubota and Y. Okada, \zpc{18}{1983}{69}.

%\bibitem{DjoSpiZer91}
%A. Djouadi, M. Spira and P.M. Zerwas, \plb{264}{1991}{440}.

%\bibitem{CheKniSte97hgg}
%K.G. Chetyrkin, B.A. Kniehl and M. Steinhauser, \prl{79}{1997}{353}.

%\bibitem{CheKniSte98dec}
%K.G. Chetyrkin, B.A. Kniehl and M. Steinhauser, \npb{510}{1998}{61}.

%\bibitem{Klu75}
%H. Kluberg-Stern and J.B. Zuber, \prd{12}{1975}{467};\\
%N.K. Nielsen, \npb{97}{1975}{527}; \npb{120}{1977}{212}.

%\bibitem{Spi84}
%V.P. Spiridonov, INR Report No.~P--0378 (1984).

%\bibitem{CheKniSteBar98}
%K.G. Chetyrkin, B.A. Kniehl, M. Steinhauser and W.A. Bardeen,
%Report Nos.\ FERMILAB-PUB-98/126-T, MPI/PhT/98--032, 
%NYU-TH/98/04/02, TTP98-21 and hep-ph/9807241 (June 1998);
%({\it Nucl. Phys.} {\bf B} in press).

%\bibitem{CheKniSte97als}
%K.G. Chetyrkin, B.A. Kniehl and M. Steinhauser, \prl{79}{1997}{2184}.

%\bibitem{Kat97}
%A.L. Kataev and V.T. Kim, Report Nos. ENSLAPP-A-407-92, hep-ph/9304282;\\ 
%A.L. Kataev, Report Nos. hep-ph/9708292, to be published in
%{\it Pisma Zh. Eksp. Teor. Fiz.} {\bf v 66}, N5 (1997) 
%({\it JETP Lett.}  {\bf v 66} (1997)).

%\bibitem{HarSte97}
%R. Harlander and M. Steinhauser, \prd{56}{1997}{3980}

%\bibitem{LarRitVer95CheKwi96}
%S.A. Larin, T. van Ritbergen and J.A.M. Vermaseren, \plb{362}{1995}{134};\\
%K.G. Chetyrkin and A. Kwiatkowski, \npb{461}{1996}{3}.

