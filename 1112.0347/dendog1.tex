We now turn to the proof of Theorem~\ref{endogth}.
Our strategy is analogous to that of Chapter~3; we get Completeness {\it via} Prime Completeness.
Firstly, we have:
\begin{theorem}[Soundness]
\label{endogsoun}
For all $M$, $\Gamma$, $\phi$:
\[ M, \Gamma \vdash \phi \;\; \Longrightarrow \;\; M, \Gamma \models \phi . \]
\end{theorem}

\proof\ By a routine induction on the length of proofs in the endogenous logic.
We give two cases for illustration.

\noindent 1. Suppose the last step in the proof is an application of $({\vdash}-{\rightarrow}-I)$:
\[ \frac{M, \Gamma [x \mapsto \phi ] \vdash \psi}{\lambda x . M, \Gamma \vdash (\phi \rightarrow \psi )} \]
By induction hypothesis, $M, \Gamma [x \mapsto \phi ] \models \psi$, i.e for all $\rho \models \Gamma$, $d \in {\cal D}(\sigma )$, 
\[  d \in \lsem \phi \rsem \;\; \Longrightarrow \;\; \lsem M \rsem \rho [x \mapsto d] \in \lsem \psi \rsem , \]
which implies
\[ \lambda x . M, \Gamma \models (\phi \rightarrow \psi ) . \]

\noindent 2. Next we consider $({\vdash}-{\Box}-E)$:
\[ \frac{M, \Gamma \vdash \Box \phi \;\;\; N, \Gamma [x \mapsto \phi ] \vdash \Box \psi}{\uextend{M}{x}{N}, \Gamma \vdash \Box \psi} \]
By induction hypothesis, $M, \Gamma \models \Box \phi$ and $N, \Gamma [x \mapsto \phi ] \models \Box \psi$.
Hence for $\rho \models \Gamma$, $\lsem M \rsem \rho \subseteq \lsem \phi \rsem$,
and for $d \in {\cal D}(\sigma )$,
\[ d \in \lsem \phi \rsem \;\; \Longrightarrow \;\; \lsem N \rsem \rho [x \mapsto d] \subseteq \lsem \psi \rsem . \]
Thus
\[ \begin{array}{ll}
& \bigcup_{d \in \lsem M \rsem \rho} \lsem N \rsem \rho [x \mapsto d] \subseteq \lsem \psi \rsem \\
\Longrightarrow & \lsem \uextend{M}{x}{N} \rsem \rho \subseteq \lsem \psi \rsem \\
\Longrightarrow & \uextend{M}{x}{N}, \Gamma \models \Box \psi . \;\;\; \qed
\end{array} \]

Next, we shall need a technical lemma which  describes our program constructs under the denotational semantics.
\begin{lemma}
\label{techlem}
For $u \in {\cal K}({\cal D}(\sigma ))$, $v \in {\cal K}({\cal D}(\tau ))$, $w \in {\cal K}({\cal D}(\upsilon ))$, 
$X \in \wp_{\sf fne}({\cal K}({\cal D}(\sigma )))$, 
$Y \in \wp_{\sf fne}({\cal K}({\cal D}(\tau )))$, 
$Z \in \wp_{\sf fne}({\cal K}({\cal D}(\sigma \times \tau )))$,
$w_{1} \in {\cal K}({\cal D}({\sf rec} \: t. \, \sigma ))$, $w_{2} \in {\cal K}({\cal D}(\sigma [{\sf rec} \: t. \, \sigma /t]))$:
\[ \begin{array}{rl}
(i) & (u, v) \sqsubseteq \lsem (M, N) \rsem \rho \; \Leftrightarrow \; u \sqsubseteq \lsem M \rsem \rho \: \& \: v \sqsubseteq \lsem N \rsem \rho \\
(ii) & w \sqsubseteq \lsem \mylet{M}{x}{y}{N} \rsem \rho \; \Leftrightarrow \; \exists u, v.  \\
& (u, v) \sqsubseteq \lsem M \rsem \rho \: \& \: w \sqsubseteq \lsem N \rsem \rho [x \mapsto u, y \mapsto v] \\
(iii) & [u, v] \sqsubseteq \lsem \lambda x . M \rsem \rho \; \Leftrightarrow \; v \sqsubseteq \lsem M \rsem \rho [x \mapsto u] \\
(iv) & v \sqsubseteq \lsem MN \rsem \rho \;  \Leftrightarrow \; \exists u. [u, v] \sqsubseteq \lsem M \rsem \rho \: \& \: u \sqsubseteq \lsem N \rsem  \rho \\
(v) & \ltuple 0, u \rtuple \sqsubseteq \lsem \imath (M) \rsem \rho \; \Leftrightarrow \; u \sqsubseteq \lsem M \rsem \rho \\
&  \ltuple 1, v \rtuple \sqsubseteq \lsem \jmath (N) \rsem \rho \; \Leftrightarrow \; v \sqsubseteq \lsem N \rsem \rho \\
(vi) & w \not= \bot \;\; \Longrightarrow \;\; w \sqsubseteq \lsem \mycases{M}{x}{N_{1}}{y}{N_{2}} \rsem \rho \; \Leftrightarrow \\
& \exists u \not= \bot . \, \ltuple 0, u \rtuple \sqsubseteq \lsem M \rsem \rho \: \& \: w \sqsubseteq \lsem N_{1} \rsem \rho [x \mapsto u] \\
& \mbox{or} \\
& \exists v \not= \bot . \, \ltuple 1, v \rtuple \sqsubseteq \lsem M \rsem \rho \: \& \: w \sqsubseteq \lsem N_{2} \rsem \rho [x \mapsto v] \\ 
(vii) & \ltuple 0, u \rtuple \sqsubseteq \lsem {\sf up}(M) \rsem \rho \; \Leftrightarrow \; u \sqsubseteq \lsem M \rsem \rho \\
(viii) & v \not= \bot \;\; \Longrightarrow \;\; v \sqsubseteq \lsem \lift{M}{x}{N} \rsem \rho \; \Leftrightarrow \\
& \exists u. \, \ltuple 0, u \rtuple \sqsubseteq \lsem M \rsem \rho 
\: \& \: v \sqsubseteq \lsem N \rsem \rho [x \mapsto u] \\
(ix) & \converges X \sqsubseteq \lsem \lsing M \rsing_{l} \rsem \rho \; \Leftrightarrow \; \forall x \in X. \, x \sqsubseteq \lsem M \rsem \rho \\
(x) & \converges Y \sqsubseteq \lsem \lextend{M}{x}{N} \rsem \rho \; \Leftrightarrow \; \exists X. \, \converges X \sqsubseteq \lsem M \rsem \rho \\
& \: \& \: \converges Y \sqsubseteq \bigcup_{u \in X} \lsem N \rsem \rho [x \mapsto u] \\
(xi) & \converges X \sqsubseteq \lsem M \uplus_{l} N \rsem \rho \; \Leftrightarrow \; \converges X \sqsubseteq \lsem M \rsem \rho \; \mbox{or} \; \converges X \sqsubseteq \lsem N \rsem \rho \\
(xii) & \converges Z \sqsubseteq \lsem M \otimes_{l} N \rsem \rho \; \Leftrightarrow \; \exists X, Y. \: \converges Z \sqsubseteq \converges X \otimes_{l} \converges Y \\
& \: \& \: \converges X \sqsubseteq \lsem M \rsem \rho \: \& \: \converges Y \sqsubseteq \lsem N \rsem \rho  \\
(xiii) & \diverges X \sqsubseteq \lsem \lsing M \rsing_{u} \rsem \rho \; \Leftrightarrow \; \exists x \in X. \, x \sqsubseteq \lsem M \rsem \rho 
\end{array} \]
\[ \begin{array}{rl}
(xiv) & \diverges Y \sqsubseteq \lsem \uextend{M}{x}{N} \rsem \rho \; \Leftrightarrow \; \exists X. \, \diverges X \sqsubseteq \lsem M \rsem \rho \\
& \: \& \: \diverges Y \sqsubseteq \bigcup_{u \in X} \lsem N \rsem \rho [x \mapsto u] \\ 
(xv) & \diverges X \sqsubseteq \lsem M \uplus_{u} N \rsem \rho \; \Leftrightarrow \; \diverges X \sqsubseteq \lsem M \rsem \rho \; \& \; \diverges X \sqsubseteq \lsem N \rsem \rho \\
(xvi) & \diverges Z \sqsubseteq \lsem M \otimes_{u} N \rsem \rho \; \Leftrightarrow \; \exists X, Y . \: \diverges Z \sqsubseteq \diverges X \otimes_{u} \diverges Y \\
& \: \& \: \diverges X \sqsubseteq \lsem M \rsem \rho \: \& \: \diverges Y \sqsubseteq \lsem N \rsem \rho \\
(xvii) & w_{1} \sqsubseteq \lsem {\sf fold}(M) \rsem \rho \; \Leftrightarrow \; {\foldalph}^{-1}(w_{1}) \sqsubseteq \lsem M \rsem \rho \\
(xviii) & w_{2} \sqsubseteq \lsem {\sf unfold}(M) \rsem \rho \; \Leftrightarrow \; {\foldalph}(w_{2}) \sqsubseteq \lsem M \rsem \rho \\
(xix) & u \sqsubseteq \lsem \mu x .  M \rsem \rho \; \Leftrightarrow \; \exists k \in \omega , \, u_{0}, \ldots , u_{k}. \, u_{0} = \bot \: \& \: u_{k} = u \\
& \: \& \: \forall i: 0 \leq i < k. \, u_{i+1} \sqsubseteq \lsem M \rsem \rho [x \mapsto u_{i}] 
\end{array} \]
\end{lemma}

\proof\ The content of this Lemma is all quite standard, at least in the folklore.
It amounts to a description of the combinators underlying the denotational semantics of terms as {\em approximable mappings}.
Most of it can be found, couched in the language of information systems, in \cite{Sco82}, and for neighbourhood systems in \cite{Sco81}.
We shall just give a couple of the less familiar cases for illustration.

\noindent (xii).
\[ \begin{array}{ll}
\bullet & \converges Z \sqsubseteq \lsem M \otimes_{l} N \rsem \rho \\
\Leftrightarrow & \converges Z \subseteq \bigsqcup \{ \converges X \otimes_{l} \converges Y : \converges X \sqsubseteq \lsem M \rsem \rho \: \& \: \converges Y \sqsubseteq \lsem N \rsem \rho \} \\
& \mbox{since $\otimes_{l}$ is continuous} \\
\Leftrightarrow & \exists X, Y. \: \converges Z \sqsubseteq \converges X 
\otimes_{l} \converges Y \: \& \: \converges X \sqsubseteq \lsem M \rsem \rho 
\; \& \; \converges Y \sqsubseteq \lsem N \rsem \rho \\
& \mbox{since $\converges Z$ is finite.}  \\
\end{array} \]

\noindent (xiv).
\[ \begin{array}{ll}
\bullet & \diverges Y \sqsubseteq \lsem \uextend{M}{x}{N} \rsem \rho \\
\Leftrightarrow & \diverges Y \sqsubseteq \bigsqcup_{\diverges X \sqsubseteq \lsem M \rsem \rho} \bigcup \{ \lsem N \rsem \rho [x \mapsto u] : u \in \diverges X \} \\
& \mbox{since {\sf extend} is continuous} \\
\Leftrightarrow & \exists X. \, \diverges X \sqsubseteq \lsem M \rsem \rho \; \& \; \diverges Y \sqsubseteq \bigcup_{u \in \diverges X} \lsem N \rsem \rho [x \mapsto u] 
\end{array} \]
since $\diverges Y$ is finite.
The argument is completed by observing that
\[ \bigcup_{u \in \diverges X} \lsem N \rsem \rho [x \mapsto u] = \bigcup_{u \in X} \lsem N \rsem \rho [x \mapsto u] . \;\;\; \qed \]

Now for Prime Completeness.

{\bf Notation.} ${\sf CPNF}(\Gamma ) \equiv \forall x \in {\sf Var}. \, {\sf CPNF}(\Gamma x)$.

\begin{theorem}[Prime Completeness]
\label{pricom}
${\sf CPNF}(\Gamma )$ and ${\sf CPNF}(\phi )$ imply that
\[ M, \Gamma \models \phi \;\; \Longrightarrow \;\; M, \Gamma \vdash \phi \]
\end{theorem}

\proof\ We begin by establishing some useful notation.
Given $\Gamma$ with ${\sf CPNF}(\Gamma )$, we define an environment $\rho_{\Gamma}$ by:
\[ \forall x \in {\sf Var}. \, \Gamma x \leftrightsquigarrow \rho_{\Gamma} x . \]
This is well-defined by Proposition~\ref{squig}.
Similarly, let $\phi \leftrightsquigarrow u$.
Now we have:
\begin{equation}
\label{stareq}
M, \Gamma \models \phi \;\; \Longleftrightarrow \;\; u \sqsubseteq \lsem M \rsem \rho_{\Gamma} . 
\end{equation}
The proof proceeds by induction on $M$.
As the various cases all share a common pattern, we shall only give a selection of the more interesting for illustration.

Abstraction. We argue by induction on $\phi$. 
The inductive case, which can only be a conjunction, since
$\phi$ is in {\sf CPNF}, is trivial.
We are left with the case for a generator $(\phi \rightarrow \psi )$, where $\phi$, $\psi$ are in {\sf CPNF}.
Let $\phi \leftrightsquigarrow u$, $\psi \leftrightsquigarrow v$.
Then
\[ \begin{array}{llr}
\bullet & \lambda x . M, \Gamma \models (\phi \rightarrow \psi ) & \\
\Rightarrow & [u, v] \sqsubseteq \lsem \lambda x . M \rsem \rho_{\Gamma} & \ref{stareq} \\
\Rightarrow & v \sqsubseteq \lsem M \rsem \rho_{\Gamma}[x \mapsto u] & \mbox{\ref{techlem}(iii)} \\
\Rightarrow & M, \Gamma [x \mapsto \phi ] \models \psi & \ref{stareq} \\
\Rightarrow & M, \Gamma [x \mapsto \phi ] \vdash \psi & \mbox{ind. hyp.} \\
\Rightarrow & \lambda x . M, \Gamma \vdash (\phi \rightarrow \psi ) & ({\vdash}-{\rightarrow}-I)
\end{array} \]

Application.
\[ \begin{array}{llr}
\bullet & MN, \Gamma \models \phi  &\\
\Rightarrow & u \sqsubseteq \lsem MN \rsem \rho_{\Gamma} & \ref{stareq} \\
\Rightarrow & \exists v. \, [v, u] \sqsubseteq \lsem M \rsem \rho \: \& \: v \sqsubseteq \lsem N \rsem \rho & \mbox{\ref{techlem}(iv)} \\
\Rightarrow & M, \Gamma \models (\psi \rightarrow \phi ) \; \& \; N, \Gamma \models \psi & \ref{stareq} \\
& \mbox{where $\psi \leftrightsquigarrow v$} & \\
\Rightarrow & M, \Gamma \vdash (\psi \rightarrow \phi ) \; \& \; N, \Gamma \vdash \psi & \mbox{ind. hyp.} \\
\Rightarrow & MN, \Gamma \vdash \phi & ({\vdash}-{\rightarrow}-E).
\end{array} \]

Case expression.
\[ \begin{array}{llr}
& \mycases{M}{x}{N_{1}}{y}{N_{2}}, \Gamma \models \phi & \\
\Leftrightarrow & u \sqsubseteq \lsem \mycases{M}{x}{N_{1}}{y}{N_{2}} \rsem \rho_{\Gamma} & \ref{stareq}.
\end{array} \]
If $u = \bot$, then ${\cal L} \vdash \true \leq \phi$, and the required conclusion follows by $({\vdash}-{\wedge})$ and $({\vdash}-{\leq})$.
Otherwise, by~\ref{techlem}(vi), either
\[ (i) \;\; \exists u_{1} \not= \bot . \, \ltuple 0, u_{1} \rtuple \sqsubseteq \lsem M \rsem \rho_{\Gamma} \: \& \: u \sqsubseteq \lsem N_{1} \rsem \rho_{\Gamma} [x \mapsto u_{1}] \]
or
\[ (ii) \;\; \exists u_{2} \not= \bot . \, \ltuple 1, u_{2} \rtuple \sqsubseteq \lsem M \rsem \rho_{\Gamma} \: \& \: u \sqsubseteq \lsem N_{2} \rsem \rho_{\Gamma} [x \mapsto u_{2}]  . \]
We shall consider sub-case (i); (ii) is entirely similar.
Let $\phi_{1} \leftrightsquigarrow u_{1}$. Then
\[ \begin{array}{llr}
\bullet & \ltuple 0, u_{1} \rtuple \sqsubseteq \lsem M \rsem \rho_{\Gamma} \: \& \: u \sqsubseteq \lsem N_{1} \rsem \rho_{\Gamma} [x \mapsto u_{1}]  & \\
\Rightarrow & M, \Gamma \models (\phi_{1} \oplus \false ) \; \& \; N_{1}, \Gamma [x \mapsto \phi_{1}] \models \phi & \ref{stareq} \\
\Rightarrow & M, \Gamma \vdash (\phi_{1} \oplus \false ) \; \& \; N_{1}, \Gamma [x \mapsto \phi_{1}] \vdash \phi & \mbox{ind. hyp.} \\
\Rightarrow & \mycases{M}{x}{N_{1}}{y}{N_{2}}, \Gamma \vdash \phi & 
\mbox{by $({\vdash}-{\oplus}-E)$} \\
& \mbox{since $u_{1} \not= \bot$ implies $\phi_{1} \converges$ by \ref{metap}.} &
\end{array} \]

Tensor product.
We write $\phi \in {\sf CPNF}(P_{u}(\sigma \times \tau ))$ as $\Box \bigvee_{i \in I}(\phi \times \psi )$, and define $Z = \diverges \{ (u_{i}, v_{i}) : i \in I \}$, where
\[ \phi_{i} \leftrightsquigarrow u_{i}, \;\;\; \psi_{i} \leftrightsquigarrow v_{i} \;\;\; (i \in I). \]
Now
\[ \begin{array}{llr}
\bullet & M \otimes_{u} N, \Gamma \models \Box \bigvee_{i \in I}(\phi \times \psi ) & \\
\Rightarrow & Z \sqsubseteq \lsem M \otimes_{u} N \rsem \rho_{\Gamma} & \ref{stareq} \\
\Rightarrow & \exists X, Y. \: \diverges X \sqsubseteq \lsem M \rsem \rho_{\Gamma} \; \& \; \diverges Y \sqsubseteq \lsem N \rsem \rho_{\Gamma} & \\
& \& \; \diverges Z \sqsubseteq \diverges X \otimes_{u} \diverges Y = \diverges (X \times Y) & \mbox{\ref{techlem}(xvi)} 
\end{array} \]
Let $X = \{ u_{k} \}_{k \in K}$, $Y = \{ v_{l} \}_{l \in L}$, and define
\[ \phi_{k} \leftrightsquigarrow u_{k} \;\; (k \in K), \;\;\; \psi_{l} \leftrightsquigarrow v_{l} \;\; (l \in L). \]
Now
\[ \begin{array}{llr}
\bullet & \diverges X \sqsubseteq \lsem M \rsem \rho_{\Gamma} \; \& \; \diverges Y \sqsubseteq \lsem N \rsem \rho_{\Gamma} &  \\
\Rightarrow & M, \Gamma \models \Box \bigvee_{k \in K} \phi_{k} \;\; \& \;\; N, \Gamma \models \Box \bigvee_{l \in L} \psi_{l} & \ref{stareq} \\
\Rightarrow & M, \Gamma \vdash \Box \bigvee_{k \in K} \phi_{k} \;\; \& \;\; N, \Gamma \vdash \Box \bigvee_{l \in L} \psi_{l} & \mbox{ind. hyp.} \\
\Rightarrow & M \otimes_{u} N, \Gamma \vdash \Box ( \bigvee_{k \in K} \phi_{k} \times \bigvee_{l \in L} \psi_{l}) & ({\vdash}-{\Box}-{\otimes}).
\end{array} \]
Finally,
\[ \begin{array}{rclr}
{\cal L} \vdash ( \bigvee_{k \in K} \phi_{k} \times \bigvee_{l \in L} \psi_{l}) & = & \bigvee_{(k,l) \in K \times L} (\phi_{k} \times \psi_{l}) & ({\times}-{\vee}) \\
& \leq & \bigvee_{i \in I} (\phi_{i} \times \psi_{i}) &
\end{array} \]
since $Z \sqsubseteq \diverges X \otimes_{u} \diverges Y$ implies
\[ \forall k, l. \, \exists i. \: {\cal L} \vdash (\phi_{k} \times \psi_{l}) \leq (\phi_{i} \times \psi_{i}). \]
Hence by $({\vdash}-{\leq})$,
\[ M \otimes_{u} N, \Gamma \vdash \Box  \bigvee_{i \in I} (\phi_{i} \times \psi_{i}). \]

Extension.
As in the case for abstraction, it suffices to consider the case when $\phi$ is a generator $\Box \bigvee_{i \in I}\phi_{i}$.
We define $Y = \{ u_{i} \}_{i \in I}$, where $\phi_{i} \leftrightsquigarrow u_{i}$, $(i \in I)$.
Now
\[ \begin{array}{llr}
\bullet & \uextend{M}{x}{N}, \Gamma \models \Box \bigvee_{i \in I} \phi_{i} & \\
\Rightarrow & \diverges Y \sqsubseteq \lsem \uextend{M}{x}{N} \rsem \rho_{\Gamma} & \ref{stareq} \\
\Rightarrow & \exists X. \: \diverges X \sqsubseteq \lsem M \rsem \rho_{\Gamma} \: \& \: \diverges Y \sqsubseteq \bigcup_{u \in X} \lsem N \rsem \rho_{\Gamma}[x \mapsto u] & \mbox{\ref{techlem}(xiv)} \\
\Rightarrow & \exists X. \: \diverges X \sqsubseteq \lsem M \rsem \rho_{\Gamma} \: \& \: \forall u \in X. \, \diverges Y \sqsubseteq \lsem N \rsem \rho_{\Gamma}[x \mapsto u] & 
\end{array} \]
Let $X = \{ v_{j} \}_{j \in J}$, $\psi_{j} \leftrightsquigarrow v_{j}$, $(j \in J)$. Then
\[ \begin{array}{llr}
\bullet &  \diverges X \sqsubseteq \lsem M \rsem \rho_{\Gamma} \: \& \: \forall u \in X. \, \diverges Y \sqsubseteq \lsem N \rsem \rho_{\Gamma}[x \mapsto u] & \\
\Rightarrow & M, \Gamma \models \Box \bigvee_{j \in J} \psi_{j} \; \& \; \forall j \in J. \: N, \Gamma [x \mapsto \psi_{j}] \models \phi & \ref{stareq} \\
\Rightarrow & M, \Gamma \vdash \Box \bigvee_{j \in J} \psi_{j} \; \& \; \forall j \in J. \: N, \Gamma [x \mapsto \psi_{j}] \vdash \phi & \mbox{ind. hyp.} \\
\Rightarrow & M, \Gamma \vdash \Box \bigvee_{j \in J} \psi_{j} \; \& \; N, \Gamma [x \mapsto \bigvee_{j \in J} \psi_{j}] \vdash \phi & ({\vdash}-{\vee}) \\
\Rightarrow & \uextend{M}{x}{N}, \Gamma \vdash  \phi & ({\vdash}-{\Box}-E)
\end{array} \]

Recursive types.
Firstly, we note that for $\phi \in {\cal L}({\sf rec} \: t. \, \sigma )$,
\[ \phi \leftrightsquigarrow u \;\; \Leftrightarrow \;\; \phi \leftrightsquigarrow \alpha^{-1}(u) , \]
since ${\cal L}({\sf rec} \: t. \, \sigma ) = {\cal L}(\sigma 
[{\sf rec} \: t. \,
\sigma /t])$.
Now,
\[ \begin{array}{llr}
\bullet & {\sf fold}(M), \Gamma \models \phi & \\
\Rightarrow & u \sqsubseteq \lsem {\sf fold}(M) \rsem \rho_{\Gamma} & \ref{stareq} \\
\Rightarrow & \alpha^{-1}(u) \sqsubseteq \lsem M \rsem \rho_{\Gamma} & \mbox{\ref{techlem}(xvii)} \\
\Rightarrow & M, \Gamma \models \phi & \ref{stareq} \\
\Rightarrow & M, \Gamma \vdash \phi & \mbox{ind. hyp.} \\
\Rightarrow & {\sf fold}(M), \Gamma \vdash \phi & ({\vdash}-{\sf rec}-I)
\end{array} \]

Recursion.
\[ \begin{array}{llr}
\bullet & \mu x. M, \Gamma \models \phi & \\
\Rightarrow & u \sqsubseteq \lsem \mu x. M \rsem \rho_{\Gamma} & \ref{stareq} \\
\Rightarrow & \exists k \in \omega , u_{0}, \ldots , u_{k}. \: u_{0} = \bot \: \& \: u_{k} = u & \\
& \: \& \: \forall i: 0 \leq i < k. \, u_{i+1} \sqsubseteq \lsem M \rsem \rho_{\Gamma} [x \mapsto u_{i}]  & \mbox{\ref{techlem}(xix).}
\end{array} \]
Let $\| u \|$ be the least such $k$ (as a function of $u$ for $u \sqsubseteq \lsem \mu x. M \rsem \rho_{\Gamma}$, keeping $\mu x. M$, $\Gamma$ fixed).
We complete the proof for this case by induction on $\| u \|$, with $\phi \leftrightsquigarrow u$.

Basis:
\[ \| u \| = 0 \Rightarrow u = \bot \Rightarrow \; \vdash \true \leq \phi \Rightarrow \mu x. M, \Gamma \vdash \phi , \]
by $({\vdash}-{\wedge})$ and $({\vdash}-{\leq})$.

Induction step: $\| u \| = k+1$. Then by definition of $\| u \|$, for some $v$:
\[ u \sqsubseteq \lsem M \rsem \rho_{\Gamma} [x \mapsto v] \; \& \; \| v \| = k. \]
Let $\psi \leftrightsquigarrow v$. Then
\[ \begin{array}{llr}
\bullet & u \sqsubseteq \lsem M \rsem \rho_{\Gamma} [x \mapsto v] \; \& \; \| v \| = k & \\
\Rightarrow & M, \Gamma [x \mapsto \psi ] \models \phi & \ref{stareq} \\
& \mbox{and} \; \mu x. M, \Gamma \vdash \psi & \mbox{inner ind. hyp.} \\
\Rightarrow & M, \Gamma [x \mapsto \psi ] \vdash \phi \; \& \; \mu x. M, \Gamma \vdash \psi & \mbox{outer ind. hyp.} \\
\Rightarrow & \mu x. M, \Gamma \vdash \phi & ({\vdash}-{\mu}-I). \;\;\; \qed
\end{array} \]

Finally, we can prove Theorem~\ref{endogth}.
One half is Theorem~\ref{endogsoun}.
For the converse, suppose $M, \Gamma \models \phi$.
We can assume that $\Gamma x \not= \false$\footnote{ 
meaning $\lsem \Gamma x \rsem \not= \varnothing$, or, equivalently by Theorem~\ref{sdual}, ${\cal L} \nvdash \Gamma x = \false$}  
for all $x \in {\sf Var}$, since otherwise we could apply $({\vdash}-{\vee})$ to obtain $M, \Gamma \vdash \phi$.
Let $V = {\sf FV}(M)$, the {\em free variables} of $M$.
(We omit the formal definition, which should be obvious).
We define $\Gamma_{V}$ by
\[ \Gamma_{V} x = \left\{ \begin{array}{ll}
\Gamma x, & x \in V \\
\true & \mbox{otherwise.}
\end{array}
\right. \]
Then by standard arguments we have:
\begin{eqnarray}
M, \Gamma \models \phi & \Leftrightarrow & M, \Gamma_{V} \models \phi \label{semfv} \\
M, \Gamma \vdash \phi & \Leftrightarrow & M, \Gamma_{V} \vdash \phi  \label{synfv} 
\end{eqnarray}
Now by Lemma~\ref{dptnf}, we have
\[ {\cal L} \vdash \phi = \bigvee_{i \in I} \phi_{i}, \]
and for all $x \in V$,
\[ {\cal L} \vdash \Gamma x = \bigvee_{j \in J_{x}} \psi_{j}, \]
with each $\phi_{i}$, $\psi_{j}$ in {\sf CPNF}.
Moreover, our assumption that $\Gamma x \not= \false$ for all $x$ implies that
$J_{x} \not= \varnothing$ for all $x \in V$.
Given $f \in \prod_{x \in V} J_{x}$ (i.e. a {\em choice function} selecting one of the disjuncts $\psi_{f x}$, $fx \in J_{x}$, for each $x \in V$), we define $\Gamma_{f}$ by:
\[ \Gamma_{f} \: x = \left\{ \begin{array}{ll}
\psi_{f x}, & x \in V \\
\true & \mbox{otherwise.}
\end{array}
\right. \]
Then
\[ \begin{array}{llr}
\bullet & M, \Gamma \models \phi & \\
\Rightarrow & M, \Gamma_{V} \models \phi & \ref{semfv} \\
\Rightarrow & \forall f \in \prod_{x \in V} J_{x}. \: M, \Gamma_{f} \models \bigvee_{i \in I} \phi_{i} & \mbox{$({\vdash}-{\leq})$, Soundness} \\
\Rightarrow & \forall f \in \prod_{x \in V} J_{x}. \, \exists i \in I. \: M, \Gamma_{f} \models \phi_{i} & \\
\Rightarrow & \forall f \in \prod_{x \in V} J_{x}. \, \exists i \in I. \: M, \Gamma_{f} \vdash \phi_{i} &  \mbox{Prime Completeness} \\
\Rightarrow & \forall f \in \prod_{x \in V} J_{x}.  \: M, \Gamma_{f} \vdash \phi & ({\vdash}-{\leq}) \\
\Rightarrow & M, \Gamma_{V} \vdash \phi & ({\vdash}-{\vee}) \\
\Rightarrow & M, \Gamma \vdash \phi & \ref{synfv} \;\;\; \qed 
\end{array} \]


