\chapter{Background: Domains and Locales}
The purpose of this Chapter is to summarise what we assume, to fix notation, and to review some basic definitions and results.
\section{Notation}
Most of the notation from elementary set theory and logic which we will use is standard and should cause no problems to the reader.
We shall use $\equiv$ for {\it definitional equality}; thus $M \equiv N$ 
means ``the expression $M$ is by definition equal to'' (or just: ``is defined to be'') ``$N$''.
We shall use $\omega$ to denote the natural numbers $\{ 0, 1, \ldots \}$ (thought of sometimes as an ordinal, and sometimes as just a set); and $\Bbb N$ to denote the set of {\em positive} integers $\{ 1, 2, \ldots \}$.
Given a set $X$, we write $\wp X$ for the powerset of $X$, $\wp_{\sf f} X$ for the set of {\em finite} subsets of $X$, and $\wp_{\sf fne} X$ for the {\em finite non-empty} subsets.
We write $X \subseteq_{\sf f} Y$ for ``$X$ is a finite subset of $Y$''.

We write substitution of $N$ for $x$ in $M$, where $M$, $N$ are expressions and $x$ is a variable, as $M[N/x]$.
We shall assume the usual notions of free and bound variables, as expounded e.g. in \cite{Bar}.
We shall always take expressions modulo $\alpha$-conversion, and treat substitution as a {\em total} operation in which variable capture is avoided by suitable renaming of bound variables.

Our notations for semantics will follow those standardly used in denotational semantics.
One operation we will frequently need is {\it updating} of environments.
Let ${\sf Env} = {\sf Var} \rightarrow {\cal V}$, where {\sf Var} is a set of variables, and $\cal V$ some value space.
Then for $\rho \in {\sf Env}$, $x \in {\sf Var}$, $v \in {\cal V}$, the expression $\rho [x \mapsto v]$ denotes the environment defined by
\[ ( \rho [ x \mapsto v]) y = \left\{ \begin{array}{ll}
v, & x = y \\
\rho y, & \mbox{otherwise.}
\end{array} \right. \]

Next, we recall some notions concerning posets (partially ordered sets).
Given a poset $P$ and $X \subseteq P$, we write
\[ \begin{array}{lcl}
\converges (X) & = & \{ y \in P : \exists x \in X. \, y \leq x \} \\
\diverges (X) & = & \{ y \in P : \exists x \in X. \, x \leq y \} \\
{\sf Con}(X) & = & \{ y \in P : \exists x, z \in X. \, x \leq y \leq z \}
\end{array} \]
We write $\converges (x)$, $\diverges (x)$ for $\converges (\{ x \} )$, $\diverges (\{ x \} )$.
A set $X$ is {\it left-closed} (or {\it lower-closed}) if $X = \converges (X)$,  {\it right-closed} (or {\it upper-closed}) if $X = \diverges (X)$, and {\it convex-closed} if $X = {\sf Con}(X)$.
When it is important to emphasise $P$ we write $\converges_{P}(X)$, $\diverges_{P}(X)$ etc.
We also have the lower, upper and Egli-Milner {\it preorders} (reflexive and transitive relations) on subsets of $P$:
\[ \begin{array}{lcl}
X \sqsubseteq_{l} Y & \equiv & \forall x \in X. \, \exists y \in Y. \, x \leq y   \\
X \sqsubseteq_{u} Y & \equiv & \forall y \in Y. \, \exists x \in X. \, x \leq y \\
X \sqsubseteq_{EM} Y & \equiv & X \sqsubseteq_{l} Y  \: \& \: X \sqsubseteq_{u} Y 
\end{array} \]
We write {\bf 2} for the two-element lattice $\{ 0, 1 \}$ with $0 < 1$, and $\Oh$ for {\em Sierpinski space}, which has the same carrier as {\bf 2}, and topology $\{ \varnothing, \{ 1 \} , \{ 0, 1 \} \}$.
As we shall see in the section on domains and locales, {\bf 2} and $\Oh$ are 
really two faces of the same structure (a ``schizophrenic object'' in the 
terminology of \cite[Chapter 6]{Joh82}), since $\Oh$ arises from the Scott topology on {\bf 2}, and {\bf 2} from the specialisation order on $\Oh$.
For other basic notions of the theory of partial orders and lattices, we refer to \cite{Compend,Joh82}.

Finally, we shall assume a modicum of familiarity with elementary category theory and general topology; suitable references are \cite{Mac71} and \cite{Dug66} respectively.
\section{Domains}
We shall assume some familiarity with \cite{PloLN}, and use it as our reference for Domain theory.
We shall not review such basic definitions as {\it cpo} (complete partial order---\cite[Chapter 1 p.\  7]{PloLN}), {\it continuous function} ({\it loc. cit.}) etc. here.

By a {\it category of domains} we shall mean a sub-category of {\bf CPO}, the category of complete partial orders and continuous functions ({\it loc. cit.}). ${\bf CPO}_{\bot}$ is the category of {\it strict} functions 
(\cite[Chapter 1 p.\  11]{PloLN}).

The properties of {\bf CPO} which make it a suitable mathematical universe for denotational semantics---a ``tool for making meanings'' in Plotkin's phrase---are:
\begin{enumerate}
\item It admits recursive definitions, both of elements of domains, and of domains themselves.
\item It supports a rich type structure.
\end{enumerate}
The mathematical content of (1) is given by the least fixed point theorem for 
continuous functions on cpo's (\cite[Chapter 1 Theorem 1]{PloLN}), 
and the initial fixed point theorem for continuous functors on {\bf CPO} 
(\cite[Chapter 5 Theorem 1]{PloLN}).
As for (2), the type constructions available over {\bf CPO} are extensively 
surveyed in \cite[Chapters 2 and 3]{PloLN}.
In order to fix notation, we shall catalogue the constructions of which mention will be made in this thesis, with references to the definitions in \cite{PloLN}:
\begin{center}
\begin{tabular}{|l|l|l|} \hline
$A \times B$ & product & Ch. 2 p.\  2 \\
$(A \rightarrow B)$ & function space & Ch. 2 p.\  9 \\
$A \oplus B$ & coalesced sum & Ch. 3 p.\  6 \\
$(A)_{\bot}$ & lifting & Ch. 3 p.\  9 \\
$(A \rightarrow_{\bot} B)$ & strict function space & Ch. 1 p.\  13 \\
$P_{l} A$ & lower (Hoare) powerdomain & Ch. 8 p.\  14 \\
$P_{u} A$ & upper (Smyth) powerdomain & Ch. 8 p.\  45 \\
$P_{p} A$ & convex (Plotkin) powerdomain & Ch. 8 p.\  28 \\ \hline
\end{tabular}
\end{center}
(Note that {\em separated sum} $A + B$ can be defined by: $A + B \equiv (A)_{\bot} \oplus (B)_{\bot}$.)

In this thesis, we shall mainly be concerned with {\em algebraic} domains, i.e. sub-categories of $\omega {\bf ALG}$, the category of $\omega$-algebraic cpo's 
\cite[Chapter 6 p.\  2]{PloLN}.
In particular, we shall be concerned with the following three full sub-categories of $\omega {\bf ALG}$:
\begin{enumerate}
\item {\bf AlgLat}: the category of $\omega$-algebraic lattices 
\cite[Chapter 6 p.\  13]{PloLN}.
\item {\bf SDom}: the category of {\em Scott domains}, i.e. the consistently complete $\omega$-algebraic cpo's ({\it loc. cit.}).
(The name comes from the fact that this is exactly the category presented in \cite{Sco81,Sco82}.)
\item {\bf SFP}: the category of {\em strongly algebraic} cpo's 
\cite[Chapter 6 p.\  17]{PloLN}.
The name is an acronym for ``Sequences of Finite Posets''---in more standard terminology, these are the $\omega$-profinite cpo's.
This category was introduced in \cite{Plo76}.
\end{enumerate}
Each of these categories is a full sub-category of the next.

The justification for studying these categories comes from the fact that 
{\bf SFP} is closed under all the type constructions listed above, while 
{\bf SDom} is closed under all but the Plotkin powerdomain.
In particular, both are cartesian closed; indeed, {\bf SFP} is the 
{\em largest} cartesian closed full sub-category of $\omega {\bf ALG}$ 
\cite{Smy83a}, while {\bf SDom} is the largest ``basis elementary'' such sub-category \cite{Gun86}.
Moreover, both categories admit initial solutions of domain equations built from these constructions (obviously excluding the Plotkin powerdomain in the case of {\bf SDom}).
Almost all the domains needed in denotational semantics to date can be defined from these constructions by composition and recursion (some exceptions of three different kinds: \cite{Abr83}, \cite{Ole85}, \cite{Plo82}).
The reason for including {\bf AlgLat} is that it is a usefully simpler special case, which will be applicable to our work in Chapter~6.

Given an algebraic domain $D$, we shall write ${\cal K}(D)$ for its {\it basis}, i.e. the sub-poset of finite elements.
Now algebraic domains are {\em freely constructed} from their bases, i.e.
\[ D \cong {\sf Idl}({\cal K}(D)) \]
where {\sf Idl} is the ideal completion described in \cite[Chapter 6 p.\  5]{PloLN}.
Thus we can in fact completely describe such categories as {\bf SDom} and {\bf SFP} in an elementary fashion in terms of the bases; various ways of doing this for {\bf SDom} are presented in \cite{Sco81,Sco82}.

An important part of this programme is to describe the type constructions listed above in terms of their effect on the bases.
We shall fix some concrete definitions of the constructions for use in later chapters.
\begin{itemize}
\item ${\cal K}(A \times B) = {\cal K}(A) \times {\cal K}(B)$; the ordering is component-wise.
\item ${\cal K}(A \oplus B) = {\cal K}(A) \oplus {\cal K}(B)$, i.e.
\[ \{ \bot \} \cup (\{ 0 \} \times ({\cal K}(A) - \{ \bot_{A} \} )) \cup (\{ 1 \} \times ({\cal K}(B) - \{ \bot_{B} \} )) \]
with the ordering defined by
\begin{eqnarray*}
x \sqsubseteq y & \equiv & x = \bot \\
& & \mbox{or} \; x = (0, a) \: \& \: y = (0, b) \: \& \: a \sqsubseteq_{A} b \\
& & \mbox{or} \; x = (1, c) \: \& \: y = (1, d) \: \& \: c \sqsubseteq_{B} d .
\end{eqnarray*}
\item ${\cal K}((A)_{\bot}) = \{ \bot \} \cup ( \{ 0 \} \times {\cal K}(A))$, with the ordering defined by
\begin{eqnarray*}
x \sqsubseteq y & \equiv & x = \bot \\
& & \mbox{or} \; x = (0, a) \: \& \: y = (0, b) \: \& \: a \sqsubseteq_{A} b.
\end{eqnarray*}
\item ${\cal K}(P_{l}(A)) = \{ \converges_{{\cal K}(A)}(X) : X \in \wp_{\sf fne}({\cal K}(A)) \}$, with the subset ordering.
\item ${\cal K}(P_{u}(A)) = \{ \diverges_{{\cal K}(A)}(X) : X \in \wp_{\sf fne}({\cal K}(A)) \}$, with the superset ordering.
\item ${\cal K}(P_{p}(A)) = \{ {\sf Con}_{{\cal K}(A)}(X) : X \in \wp_{\sf fne}({\cal K}(A)) \}$, with the Egli-Milner ordering (which {\em is} a partial order on the convex-closed sets).
\end{itemize}
All these definitions are valid for {\em any} algebraic cpo.
Since $\omega {\bf ALG}$ is not cartesian closed, we must obviously describe the function space construction for one of its cartesian closed sub-categories.
As the description for {\bf SFP} is rather complicated (see \cite{Gun85}),  we shall give the simpler description for {\bf SDom}.
\begin{definition}
{\rm (i) (\cite[Chapter 6 p.\  1]{PloLN}). Let $A$, $B$ be algebraic domains. For $a \in {\cal K}(A)$, $b \in {\cal K}(B)$,
\[ [a, b] : A \rightarrow B \]
is the {\em one-step} function defined by
\[ [a, b] d = \left\{ \begin{array}{ll}
b & \mbox{if $a \sqsubseteq d$} \\
\bot & \mbox{otherwise}
\end{array} \right. \]
(ii) (\cite[Chapter 6 p.\  13]{PloLN}). $X \subseteq A$ is {\em consistent}:
\[ \bigtriangleup (X) \equiv \exists d \in A. \, \forall x \in X. \, x \sqsubseteq d. \]
We write $x \bigtriangleup y$ for $\bigtriangleup \{ x, y \}$.}
\end{definition}
Note that Plotkin writes $(a \Rightarrow b)$ for $[a, b]$, and $\diverges X$ for $\bigtriangleup (X)$.
\begin{proposition}
\label{funcon}
(\cite[Chapter 6 pp.\  14--15]{PloLN}).
Let $A$, $B$ be Scott domains, and $\{ a_{i} \}_{i \in I} \subseteq {\cal K}(A)$, $\{ b_{i} \}_{i \in I} \subseteq {\cal K}(B)$ for some finite set $I$.

\noindent (i) $\bigtriangleup \{ [a_{i}, b_{i}] : i \in I \}$ if and only if
\[ \forall J \subseteq I. \, \bigtriangleup \{ a_{j} : j \in J \} \; \Rightarrow \; \bigtriangleup \{ b_{j} : j \in J \} \]

\noindent (ii)  $\bigtriangleup \{ [a_{i}, b_{i}] : i \in I \}$ implies that $\bigsqcup \{ [a_{i}, b_{i}] : i \in I \}$ exists and is defined by
\[ ( \bigsqcup \{ [a_{i}, b_{i}] : i \in I \}) d = \bigsqcup \{ b_{i} : a_{i} \sqsubseteq d \} . \]
\end{proposition}
Now we finally get our description of the function space:
\begin{itemize}
\item For Scott domains $A$, $B$:
\begin{eqnarray*}
{\cal K}(A \rightarrow B) & = & \{ \bigsqcup \{ [a_{i}, b_{i}] : i \in I \} : \mbox{$I$ finite}, \\
& & \{ a_{i} \}_{i \in I} \subseteq {\cal K}(A), \: \{ b_{i} \}_{i \in I} \subseteq {\cal K}(B), \\
& &  \bigtriangleup \{ [a_{i}, b_{i}] : i \in I \} \} .
\end{eqnarray*}
\end{itemize}

