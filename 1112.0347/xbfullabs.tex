\section{Full Abstraction for SCCS}
So far, we have worked with abstract transition systems, in a syntax-free fashion.
This degree of abstraction carries a price; we lose compositionality.
Indeed, we need syntax to {\em define} compositionality.
Accordingly, in this Section we turn to a particular transition system specified by an algebraic syntax, namely Milner's SCCS \cite{Mil83}.
We equip our domain \Dom\ with a continuous algebraic structure corresponding to the signature of SCCS.
Our main result is that the resulting denotational semantics for SCCS is {\em fully abstract} \cite{Mil75,Plo77} with respect to bisimulation for finite terms, and with respect to the finitary preorder for recursive terms.
As a by-product we will show that \Dom\ is isomorphic to Hennessy's term model \cite{Hen81}, and hence obtain a complete axiomatisation of its equational theory as an immediate consequence of Hennessy's results.

Our choice of SCCS is for illustrative purposes, because it is simple and yet expressive.
Similar accounts could be given for CCS \cite{Mil80}, MEIJE \cite{AB84}, ACP \cite{BK84}, etc.
Note, however, that our semantics is fully abstract with respect to the {\em strong} congruence in Milner's terminology \cite{Mil83}, where all actions are observable.
A corresponding treatment of {\em observation equivalence} \cite{HM85}, where unobservable actions are factored out, is still an open problem as far as I know; some hints of a possible approach may be gleaned from \cite{Abr87b}.

We begin by recalling some basic definitions on SCCS from \cite{Mil83,Hen81}.
We assume familiarity with basic notions of universal algebra; see e.g. \cite{ADJ78,EM85}.

We fix a set of actions {\sf Act}, which we assume comes equipped with an {\em abelian monoid} structure comprising
\begin{itemize}
\item an associative, commutative binary operation which we denote by juxtaposition, e.g. $ab$
\item a unit 1.
\end{itemize}
The (one-sorted) signature $\Sigma$ of SCCS is then defined as follows:

\begin{definition} 
{\rm $\Sigma = \{ \Sigma_{n} \}_{n \in \omega}$, where $\Sigma_{n}$ is the set of operation symbols of arity $n$ in $\Sigma$.
\begin{eqnarray*}
\Sigma_{0} & \equiv & \{ \Oh , \Omega \} \\
\Sigma_{1} & \equiv & \{ a\_ : a \in {\sf Act} \} \cup \{ \_{\restriction} A : A \subseteq {\sf Act} \} \\
& & \mbox{} \cup \{ \_ [S] : S \; \mbox{is a monoid endomorphism on {\sf Act}} \}  \\
\Sigma_{2} & \equiv & \{ {+}, {\times} \} \\
\Sigma_{n} & \equiv & \varnothing , \;\; n > 2 .
\end{eqnarray*} }
\end{definition}

Thus our version of SCCS only has {\em finite} sums (in contrast with \cite{Mil83}), and has a constant for the undefined process as in \cite{Hen81}.

We define the subsignature $\Sigma' \subseteq \Sigma$ to be obtained by omitting the {\it restriction} operators $\_{\restriction} A$, 
the {\it relabelling} operators $\_ [S]$, and the {\it synchronous product} operator $\times$, leaving only the {\it nullary sum} $\Oh$, the {\it binary sum} $+$, {\it prefixing} $a\_$, and the {\it undefined} process $\Omega$.

We take the {\em finite processes} of SCCS to be the terms over the signature $\Sigma$, i.e. the elements of the term algebra $T_{\Sigma}$.
Evidently, we can take the elements of $T_{\Sigma'}$ as notations for the finite synchronisation trees ${\sf ST}_{\omega}$.

\begin{definition}[Operational Semantics]
\label{opdef}
{\rm We make $T_{\Sigma}$ into a transition system by defining the transition relation
and divergence predicate in a syntax-directed way, as the {\em least}
relations satisfying the following axioms and rules:}
\[ (D \Omega ) \;\; \Omega \diverges \]
\[ (D+L) \;\; \frac{t_{1} \diverges}{(t_{1} + t_{2}) \diverges} \;\;\;\;\;\;\;\; (D+R) \;\; \frac{t_{2} \diverges}{(t_{1} + t_{2}) \diverges} \]
\[ (D{\restriction} ) \;\; \frac{t \diverges}{(t {\restriction} A) \diverges} \;\;\;\;\;\;\;\; (DS) \;\; \frac{t \diverges}{t [S] \diverges} \]
\[ (D \times L) \;\; \frac{t_{1} \diverges}{t_{1} \times t_{2} \diverges} \;\;\;\;\;\;\;\; (D \times R) \;\; \frac{t_{2} \diverges}{t_{1} \times t_{2} \diverges} \]
\[ (Ta) \;\;  \labarrow{at}{a}{t} \]
\[ (T+L) \;\; \frac{\labarrow{t_{1}}{a}{t'_{1}}}{\labarrow{t_{1} + t_{2}}{a}{t'_{1}}} \;\;\;\;\;\;\;\; 
(T+R) \;\; \frac{\labarrow{t_{2}}{a}{t'_{2}}}{\labarrow{t_{1} + t_{2}}{a}{t'_{2}}} \]
\[ (T {\restriction} ) \;\; \frac{\labarrow{t}{a}{t'}, \; a \in A}{\labarrow{t {\restriction} A}{a}{t' {\restriction} A}} \;\;\;\;\;\;\;\; (TS) \;\; \frac{\labarrow{t}{a}{t'}}{\labarrow{t[S]}{S a}{t'[S]}} \]
\[ (T \times ) \;\; \frac{\labarrow{t_{1}}{a}{t'_{1}} \;\; \labarrow{t_{2}}{b}{t'_{2}}}{\labarrow{t_{1} \times t_{2}}{ab}{t'_{1} \times t'_{2}}} \] 
\end{definition}

For an illuminating discussion of the conceptual basis for these and related axioms, see \cite{Mil86}.

We now have a transition system $(T_{\Sigma}, {\sf Act}, {\rightarrow}, \diverges )$ implicitly defined by~\ref{opdef}.
The following proposition gives a more explicit description of this system.

\begin{proposition}
\label{tops}
For all $t, t_{1}, t_{2} \in T_{\Sigma}$:
\[ \begin{array}{rlcl}
(i) (a) & \Oh \converges & & (b) \;\; \nlabarrow{\Oh}{a}{} \\
(ii) (a) & \Omega \diverges & & (b) \;\; \nlabarrow{\Omega}{a}{} \\
(iii) (a) & at \converges \\
(b) & \labarrow{at_{1}}{b}{t_{2}} & \Longleftrightarrow & b = a \: \& \: t_{1} = t_{2} \\
(iv) (a) & (t_{1} + t_{2}) \diverges & \Longleftrightarrow & t_{1} \diverges \;  {\rm or} \; t_{2} \diverges \\
(b) & \labarrow{(t_{1} + t_{2})}{a}{t} & \Longleftrightarrow & \labarrow{t_{1}}{a}{t} \; {\rm or} \; \labarrow{t_{2}}{a}{t} \\
(v) (a) & (t {\restriction} A) \diverges & \Longleftrightarrow & t \diverges \\
(b) & \labarrow{t_{1} {\restriction} A}{a}{t_{2}} & \Longleftrightarrow & \exists t . \, \labarrow{t_{1}}{a}{t} \: \& \: t_{2} = t {\restriction} A \: \& \: a \in A \\
(vi) (a) & t[S] \diverges & \Longleftrightarrow & t \diverges \\
(b) & \labarrow{t_{1}[S]}{a}{t_{2}} & \Longleftrightarrow & \exists b, t . \, \labarrow{t_{1}}{b}{t} \: \& \: t_{2} = t[S] \: \& \:  a = Sb \\
(vii) (a) & (t_{1} \times t_{2}) \diverges & \Longleftrightarrow & t_{1} \diverges \;  {\rm or} \; t_{2} \diverges \\
(b) & \labarrow{t_{1} \times t_{2}}{a}{t} & \Longleftrightarrow & \exists t'_{1}, t'_{2}, b_{1}, b_{2} . \, \labarrow{t_{i}}{b_{i}}{t'_{i}} \; (i = 1, 2) \\
& & & \& \: t = t'_{1} \times t'_{2} \: \& \: a = b_{1}b_{2} .
\end{array} \]
\end{proposition}

\proof\ By induction on the length of proofs of $t \diverges$ and $\labarrow{t_{1}}{a}{t_{2}}$. \qed

Now given any $\Sigma$-algebra ${\cal A}$, by initiality of $T_{\Sigma}$ there is a unique $\Sigma$-homomorphism
\[ \lsem \cdot \rsem^{\cal A} : T_{\Sigma} \; \longrightarrow \; {\cal A} , \]
which is just another notation for a compositional denotational semantics as 
in \cite{MS76,Sto77,Gor79}.
Thus to form a denotational semantics $\lsem \cdot \rsem^{\cal D}$ based on our domain $\Dom$, it suffices to define each operation in $\Sigma$ as a function of the appropriate arity over $\Dom$.
We shall in fact define the operations so that they are {\em continuous} over $\Dom$.

\begin{definition} 
\label{sigd}
{\rm We specify a $\Sigma$-structure on $\Dom$:
\[ \begin{array}{rlcl}
(i) & \Oh^{\Dom} & \equiv & \emptyset \\
(ii) & \Omega^{\Dom} & \equiv & \lsing \bot \rsing \\
(iii) & {a\_}^{\Dom} & \equiv & \lambda d \in {\Dom}. \lsing \ltuple a, d \rtuple \rsing \\
(iv) & +^{\Dom} & \equiv & \uplus \\
\end{array} \]
Restriction:
\[ (v) \;\; (\_ {\restriction} A)^{\Dom}  \equiv  \mu \Phi \in [\Dom \rightarrow \Dom ] . \, \biguplus \circ P^{0} (g_{A} \Phi ) \]
where
\[ g_{A} : [ \Dom \rightarrow \Dom ] \rightarrow [ \sum_{a \in {\sf Act}} \Dom \rightarrow \Dom ] \]
is defined by
\[ \begin{array}{rcl}
g_{A} \Phi \bot & = & \lsing \bot \rsing \\
g_{A} \Phi \ltuple a, d \rtuple & = & \left\{ \begin{array}{ll} 
\lsing \ltuple a, \Phi d \rtuple \rsing & \mbox{if $a \in A$} \\
\emptyset & \mbox{otherwise}
\end{array}
\right. \\
\end{array} \]
(i.e. 
\[ g_{A} \Phi = {\coprod_{a \in A} \lambda d \in \Dom . \lsing \ltuple a, \Phi d \rtuple \rsing} \; {\textstyle \coprod} \; {\coprod_{a \in {\sf Act} - A} \lambda d \in \Dom . \emptyset } , \]
where $\coprod$ is ``source tupling'' \cite{ADJ85}).

\noindent Relabelling:
\[ (vi) \;\; (\_ [S])^{\Dom}  \equiv  \mu \Phi \in [\Dom \rightarrow \Dom ] . \,  P^{0} (g_{S} \Phi ) \]
where
\[ g_{S} : [ \Dom \rightarrow \Dom ] \rightarrow [ \sum_{a \in {\sf Act}} \Dom \rightarrow \sum_{a \in {\sf Act}} \Dom  ] \]
is defined by
\[ \begin{array}{rcl}
g_{S} \Phi \bot & = &  \bot  \\
g_{S} \Phi \ltuple a, d \rtuple & = & \ltuple Sa, \Phi d \rtuple
\end{array} \]
Product: 
\[ (vii) \;\; \times^{\Dom}  \equiv  \mu \Phi \in [\Dom^{2} \rightarrow \Dom ] . \,   (f \Phi )^{\dagger} \]
where
\[ f : [ \Dom^{2} \rightarrow \Dom ] \rightarrow [ ( \sum_{a \in {\sf Act}} \Dom )^{2} \rightarrow \sum_{a \in {\sf Act}} \Dom  ] \]
is defined by
\[ \begin{array}{rcl}
f \Phi (x, \bot ) = f \Phi (\bot , x)  & = &  \bot  \\
f \Phi (\ltuple a, d \rtuple  , \ltuple b, e \rtuple ) & = &  \ltuple ab, \Phi (d, e) \rtuple 
\end{array} \] }
\end{definition}

The only point which needs to be checked to ensure that this definition yields well-defined continuous functions is that $g_{A} \Phi$, $g_{S} \Phi$ and $f \Phi$ are (bi)strict and continuous, which is immediate from the definitions.
Note that restriction, relabelling and product are defined recursively, while sum and prefixing are interpreted by the basic operations derived from the domain equation for $\Dom$.
This corresponds to the fact that restriction, relabelling and product can be {\em eliminated} (for finite terms) in the equational theory of SCCS modulo bisimulation.

The continuous $\Sigma$-algebra defined by \ref{sigd} is denoted $\Dom_{\Sigma}$.
The following is an easy consequence of~\ref{sigd} and~\ref{feltp}.

\begin{proposition}
\label{surs}
The semantic function
\[ \lsem\cdot \rsem^{\Dom} : T_{\Sigma} \; \longrightarrow \; \Dom_{\Sigma} \]
cuts down to surjections
\[ T_{\Sigma} \twoheadrightarrow {\cal K}(\Dom ), \;\;\;\; T_{\Sigma'} \twoheadrightarrow {\cal K}(\Dom ) . \]
\end{proposition}
Thus the finite synchronisation trees provide a notation for the finite elements of $\Dom$.

We now relate our definitions of the SCCS operations on \Dom\ to the transition system view of $\Dom$.
\begin{proposition}
\label{dops}
For all $d, d_{1}, d_{2} \in {\cal K}(\Dom )$:
\[ \begin{array}{rlcl}
(i) (a) & \Oh^{\Dom} \converges & &  (b) \;\; \nlabarrow{\Oh^{\Dom}}{a}{} \\
(ii) (a) & \Omega^{\Dom} \diverges & & (b) \;\; \nlabarrow{\Omega^{\Dom}}{a}{} \\
(iii) (a) & a^{\Dom} d \converges & & \\
(b) & \labarrow{a^{\Dom} d_{1}}{b}{d_{2}} & \Longleftrightarrow & b = a \: \& \: d_{1} = d_{2} \\
(iv) (a) & (d_{1} +^{\Dom} d_{2}) \diverges & \Longleftrightarrow & d_{1} \diverges \; \mbox{or} \; d_{2} \diverges \\
(b) & \labarrow{d_{1} +^{\Dom} d_{2}}{a}{d} & \Longleftrightarrow & \labarrow{d_{1}}{a}{d} \; \mbox{or} \; \labarrow{d_{2}}{a}{d} \\
\end{array} \]
Restriction:
\[ \begin{array}{rlcl}
(v) (a) & (d {\restriction}^{\Dom} A) \diverges & \Longleftrightarrow & d \diverges \\
(b) & \labarrow{d_{1} {\restriction}^{\Dom} A}{a}{d_{2}} & \Longleftrightarrow & \exists e_{1}, e_{2} . \, \labarrow{d_{1}}{a}{e_{i}}, \; (i = 1, 2) \\
& & & \mbox{} \& \; e_{1} {\restriction}^{\Dom} A \sqsubseteq d_{2} \sqsubseteq e_{2} {\restriction}^{\Dom} A \\
& & &  \& \; a \in A 
\end{array} \]
Relabelling:
\[ \begin{array}{rlcl}
(vi) (a) & (d [S]^{\Dom}) \diverges & \Longleftrightarrow & d \diverges \\
(b) & \labarrow{d_{1} [S]^{\Dom}}{a}{d_{2}} & \Longleftrightarrow & \exists e_{1}, e_{2}, b_{1}, b_{2} . \, \labarrow{d_{1}}{a}{e_{i}}, \; (i = 1, 2) \\
& & & \mbox{} \& \; e_{1} [S]^{\Dom} \sqsubseteq d_{2} \sqsubseteq e_{2} [S]^{\Dom} \\
& & &  \& \; S b_{1} = a = S b_{2} 
\end{array} \]
Product:
\[ \begin{array}{rlcl}
(vii) (a) & (d_{1} \times^{\Dom} d_{2}) \diverges & \Longleftrightarrow & d_{1} \diverges \; \mbox{or} \; d_{2} \diverges \\
(b) & \labarrow{d_{1} \times^{\Dom} d_{2}}{a}{d} & \Longleftrightarrow & \exists u_{i}, v_{i}, b_{i}, c_{i} \; (i = 1, 2). \\
& & & \labarrow{d_{1}}{b_{i}}{u_{i}} \: \& \: \labarrow{d_{2}}{c_{i}}{v_{i}} \; (i = 1, 2) \\
& & & \mbox{} \& \; (u_{1} \times^{\Dom} v_{1}) \sqsubseteq d \sqsubseteq (u_{2} \times^{\Dom} v_{2}) \\
& & &  \& \; b_{i}c_{i} = a \; (i = 1, 2).
\end{array} \]
\end{proposition}

\proof\ We give two cases for illustration.

\noindent (v). We define
\begin{eqnarray*}
\Theta & \equiv & \{ \{ \ltuple a, d' {\restriction}^{\Dom} A \rtuple \} : \ltuple a, d' \rtuple \in d, a \in A \} \\
&  & \mbox{} \cup \{ \varnothing : d = \emptyset \; \mbox{or} \; \exists \ltuple a, d' \rtuple \in d. \, a \not\in A \} \\
&  & \mbox{} \cup \{ \{ \bot \} : \bot \in d \} .
\end{eqnarray*}
Now
\begin{eqnarray*}
d {\restriction}^{\Dom} A & = & Con( \bigcup \Theta^{\star})  \\
& = & Con((\bigcup \Theta )^{\star}) \;\; \mbox{by \cite{Plo76} p. 477} \\
& = & Con(\bigcup \Theta ) \;\; \mbox{since $d \in {\cal K}(\Dom )$} \\
& = & Con( \{ \ltuple a, d' {\restriction}^{\Dom} A \rtuple : \ltuple a, d' \rtuple \in d \: \& \: a \in A \} \\
&   & \mbox{} \cup \{ \bot : \bot \in d \} ), 
\end{eqnarray*}
and (v) is readily derived from this description.

\noindent (vii). Similarly to (v),
\begin{eqnarray*}
d_{1} \times^{\Dom} d_{2} & = & Con( \{ \ltuple b_{1}b_{2}, e_{1} \times^{\Dom} e_{2} \rtuple : \ltuple b_{i}, e_{i} \rtuple \in d_{i}, \; i = 1, 2 \} \\
& & \mbox{} \cup \{ \bot : \bot \in d_{1} \; \mbox{or} \; \bot \in d_{2} \} ). \;\;\; \qed
\end{eqnarray*}

\begin{proposition}
\label{sbis}
For all $t \in T_{\Sigma}$, $t \sim^{B} \lsem t \rsem^{\Dom}$.
\end{proposition}

\proof\ Firstly, we define a height function on $T_{\Sigma}$ in the obvious way:
\[ {\sf ht}( \sigma (t_{1}, \ldots , t_{n}) = \sup \; \{ {\sf ht}(t_{i} : 1 \leq i \leq n \} + 1 . \]
As an easy consequence of~\ref{tops}, we have:
\[ \labarrow{t}{a}{t'} \;\; \Longrightarrow \;\; {\sf ht}(t' ) < {\sf ht}(t) . \]
The proposition is proved by induction on ${\sf ht}(t)$, and cases on the construction of $t$.
The cases arising from operations in $\Sigma'$ are immediate in the light of the parallelism between~\ref{tops} and~\ref{dops}.
We give one of the remaining cases for illustration.

\noindent $t \equiv t_{1} {\restriction}^{\Dom} A$.
Firstly, 
\begin{Eqarray}
t \diverges & \Longleftrightarrow & t_{1} \diverges & \mbox{by \ref{tops}(v)} \\
& \Longleftrightarrow & \lsem t_{1} \rsem^{\Dom} \diverges & \mbox{by induction hypothesis} \\
& \Longleftrightarrow & ( \lsem t_{1} \rsem^{\Dom} {\restriction}^{\Dom} A) \diverges & \mbox{by \ref{dops}(v)} \\
& \Longleftrightarrow & \lsem t_{1} {\restriction}  A) \rsem^{\Dom} \diverges . &
\end{Eqarray}
Next,
\[ \begin{array}{llr}
\bullet & \labarrow{t}{a}{t'} & \\
\Longrightarrow & \labarrow{t_{1}}{a}{t'_{1}} \: \& \: t' = t'_{1} {\restriction}  A \: \& \: a \in A & \mbox{by \ref{tops}(v)} \\
\Longrightarrow & \exists d' . \, \labarrow{\lsem t_{1} \rsem^{\Dom}}{a}{d'} \; \& \; t'_{1} \preord^{B} d' & \mbox{ind. hyp. on $t_{1}$} \\
\Longrightarrow & t'_{1} {\restriction}  A \sim^{B} \lsem t'_{1} {\restriction}  A \rsem^{\Dom} & \mbox{ind. hyp. on $t'_{1} {\restriction}  A$} \\
& \mbox{} = \lsem t'_{1} \rsem^{\Dom}  {\restriction}^{\Dom}  A & \\
& \; \preord^{B} d' {\restriction}^{\Dom} A & \mbox{by \ref{ifs}} \\
& \mbox{(since ${\restriction}^{\Dom}$ is monotone)} & \\
\Longrightarrow & \exists u . \, \labarrow{\lsem t \rsem^{\Dom}}{a}{u} \; \& \; t' \preord^{B} u  & \mbox{by \ref{dops}(v).}
\end{array} \]
Similarly, we can show
\[ \labarrow{t}{a}{t'} \; \Rightarrow \; \exists u. \, \labarrow{\lsem t \rsem^{\Dom}}{a}{u} \: \& \: u \preord^{B} t' . \]
Again,
\[ \begin{array}{llr}
\bullet & \labarrow{\lsem t \rsem^{\Dom}}{a}{d} & \\
\Longrightarrow & \exists d_{1}, d_{2}. \: \labarrow{\lsem t_{1} \rsem^{\Dom}}{a}{d_{i}}, \; i = 1, 2  & \\
& \mbox{} \& \; d_{1} {\restriction}^{\Dom} A \sqsubseteq d \sqsubseteq d_{2} {\restriction}^{\Dom} A \\
& \mbox{} \& \; a \in A & \mbox{by \ref{dops}(v)} \\
\Longrightarrow & \exists t'_{1}, t'_{2}. \: \labarrow{t_{1}}{a}{t'_{i}}, \; i = 1, 2 & \\
& \mbox{} \& \; t'_{1} \preord^{B} d_{1}, \; d_{2} \preord^{B} t'_{2} & \mbox{by induction hypothesis} \\
\Longrightarrow & \labarrow{t}{a}{t'_{i} {\restriction} A}, \; i = 1, 2 & \\
& \mbox{} \& \; t'_{1} {\restriction} A \sim^{B} \lsem t'_{1} {\restriction} A \rsem^{\Dom} &  \mbox{by induction hypothesis} \\
& \mbox{} = \lsem t'_{1} \rsem^{\Dom} {\restriction}^{\Dom} A \; \preord^{B} \; d_{1} {\restriction}^{\Dom} A \; \preord^{B} d, & 
\end{array} \]
and similarly $d \preord^{B} t'_{2} {\restriction} A$.
Altogether, we have $t \sim^{B} \lsem t \rsem^{\Dom}$. \qed

As an immediate consequence of this Proposition and~\ref{ifs} we have
\begin{theorem}[Full Abstraction for Finite Terms]
\label{faft}
For all $t_{1}, t_{2} \in T_{\Sigma}$: 
\[ t_{1} \preord^{B} t_{2} \;\; \Longleftrightarrow \;\; \lsem t_{1} \rsem^{\Dom} \sqsubseteq \lsem t_{2} \rsem^{\Dom} . \]
\end{theorem}
As further consequences of \ref{faft} we have
\begin{itemize}
\item $\lsem \cdot \rsem^{\Dom}$ agrees with the syntax-free map $\lsem \cdot \rsem$ defined in Section 5.
Indeed, $t \sim^{B} \lsem t \rsem^{\Dom}$ implies $\Ellomega (\lsem t \rsem^{\Dom}) = \Ellomega (t) = \Ellomega (\lsem t \rsem )$, which implies $\lsem t \rsem^{\Dom} = \lsem t \rsem$.
\item $T_{\Sigma}$ is a finitary transition system, by \ref{ftsequiv}.
\end{itemize}
Moreover, we can derive two further characterisations of \Dom.
\begin{theorem}
(i) ${\cal K}(\Dom ) \; \cong \; (T_{\Sigma'}/{\sim^{B}}, {\preord^{B}}/{\sim^{B}})$, and therefore

\noindent (ii) $D \; \cong \; {\sf Idl} \: (T_{\Sigma'}/{\sim^{B}}, {\preord^{B}}/{\sim^{B}})$.
\end{theorem}

\proof\ Immediate from \ref{surs} and \ref{faft}. \qed

We recall the notion of {\em continuous $\Sigma$-algebra} \cite{ADJ78,Gue81}.
This is just a $\Sigma$-algebra whose carrier is a cpo, and whose operations are continuous.
A homomorphism of such algebras which is continuous on the carriers is 
a {\em continuous $\Sigma$-homomorphism}.
The category of these algebras and homomorphisms is denoted ${\bf CAlg}(\Sigma )$.
\begin{definition}
{\rm {\bf SCCS-Alg} is the full subcategory of ${\bf CAlg}(\Sigma )$ of those algebras ${\cal A}$ satisfying
\[ \forall t_{1}, t_{2} \in T_{\Sigma}. \: t_{1} \preord^{B} t_{2} \; \Longrightarrow \; \lsem t_{1} \rsem^{\cal A} \sqsubseteq \lsem t_{2} \rsem^{\cal A} . \] }
\end{definition}

\begin{theorem}
\label{binit}
$\Dom_{\Sigma}$ is initial in {\bf SCCS-Alg}.
\end{theorem}

\proof\ We begin by recalling a useful fact about continuous algebras (\cite{Gue81} Proposition 3.12).
Suppose $\cal A$ is a continuous algebra whose carrier $A$ is an algebraic domain, such that the finite elements ${\cal K}(A)$ form a $\Sigma$-subalgebra.
Then, given any monotonic $\Sigma$-homomorphism
\[ f : {\cal K}(A)  \longrightarrow  {\cal B} \]
to a continuous $\Sigma$-algebra $\cal B$, there is a unique extension
\[ \hat{f} : {\cal A}  \longrightarrow  {\cal B} \]
to a continuous $\Sigma$-homomorphism on $\cal A$.

By \ref{surs}, ${\cal K}(\Dom )$ is closed under the $\Sigma$-operations.
Hence it suffices to construct a unique monotone $\Sigma$-homomorphism
\[ f : {\cal K}(\Dom )  \longrightarrow  {\cal A} \]
to any $\cal A$ in {\bf SCCS-Alg}.
Given $d \in {\cal K}(\Dom )$, by \ref{surs} there is $t \in T_{\Sigma}$ with 
$\lsem t \rsem^{\Dom} = d$, and the only possible definition for $f$ giving a $\Sigma$-homomorphism is
\[ f : d \mapsto \lsem t \rsem^{\cal A} . \]
This establishes uniqueness.
For existence,
\begin{Eqarray}
\lsem t_{1} \rsem^{\Dom} = \lsem t_{2} \rsem^{\Dom} & \Longleftrightarrow & \lsem t_{1} \rsem^{\Dom} \sim^{B} \lsem t_{2} \rsem^{\Dom} &\mbox{by \ref{ifs}} \\
& \Longleftrightarrow & t_{1} \sim^{B} t_{2} & \mbox{by \ref{faft}} \\
& \Longrightarrow & \lsem t_{1} \rsem^{\cal A} = \lsem t_{2} \rsem^{\cal A} &
\end{Eqarray}
since $\cal A$ is in {\bf SCCS-Alg}, and so $f$ is well-defined.
Similarly,
\[ \lsem t_{1} \rsem^{\Dom} \sqsubseteq \lsem t_{2} \rsem^{\Dom} \; \Rightarrow \; t_{1} \preord^{B} t_{2} \; \Rightarrow \; \lsem t_{1} \rsem^{\cal A} \sqsubseteq \lsem t_{2} \rsem^{\cal A}, \]
and so $f$ is monotone. \qed

The purely algebraic part of SCCS which we have developed so far only allows the description of {\em finite} processes.
We now extend the calculus with recursion.
\begin{definition}
{\rm We fix a set of variables {\sf Var}, ranged over by $x, y, z$.
The syntax of {\em recursive terms} ${\rm REC}_{\Sigma}$, is then defined by
\[ t \;\; ::= \;\; \sigma (t_{1}, \ldots , t_{n}) \;\; (\sigma \in \Sigma_{n}) \; | \; x \; | \; {\sf rec} \: x.t \] }
\end{definition}
In an obvious way, we can take $T_{\Sigma}$ as a subset of ${\rm REC}_{\Sigma}$. Note that $ {\sf rec} \: x.t$ is a variable-binding construct.
The set of {\em closed recursive terms} is denoted ${\rm CREC}_{\Sigma}$.

We now extend the definition of the operational semantics to ${\rm CREC}_{\Sigma}$:
\[ (D{\sf rec}) \;\; \frac{t[{\Omega}/x] \diverges}{{\sf rec} \: x.t \diverges} 
\;\;\;\;\;\;\;\; (T {\sf rec}) \;\; \frac{\labarrow{t[{\sf rec} \: x.t/x]}{a}{t'}}{\labarrow{{\sf rec} \: x.t }{a}{t'}} \]

We thus obtain a transition system $({\rm CREC}_{\Sigma}, {\sf Act}, \rightarrow , \diverges )$.
It is not too hard to see that this system is weakly finite-branching, and therefore by~\ref{finots} satisfies (BN).
However, most of the other finiteness conditions on transition systems fail, as the following examples show.

\subsection*{Examples}
(1) {\bf Failure of sort-finiteness.} Assume {\sf Act} is infinite, in 
particular that $\{ a_{n} \}$ is a sequence of distinct actions, and that $S$ is a relabelling such that
\[ S a_{n} = a_{n + 1} \;\; (n \in \omega ). \]
Then
\[ {\sf rec} \: x. \: a_{0} \Oh + x[S] \]
has the behaviour described by the synchronisation tree
\[ {\sum_{n \in \omega} a_{n} \Oh } + \Omega . \]

\noindent (2) {\bf Failure of (FA), and ${\preord_{\omega}} \not= {\preord^{B}}$.} By the example following~\ref{finots}, it suffices to show that the synchronisation tree
\[ p \equiv {\sum_{n \in \omega} a^{n} \Oh } + \Omega \]
can be defined in SCCS to disprove (FA); while the same example shows that $\preord_{\omega} \not= \preord^{B}$, since
\[ p \sim_{\omega} p + a^{\omega}, \;\; p \nsim_{\omega + 1} p + a^{\omega} , \]
and we can define $a^{\omega} \equiv {\sf rec} \: x. \: ax$. 
But using {\em unguarded} recursion (cf. \cite{Mil83}), we can define
\[ p \equiv ({\sf rec} \: x. \: ( \Delta a + ( \Delta a \times x)))\restriction \{ a \} \]
where $\Delta a \equiv {\sf rec} \: y. \: a 1^{\omega} + 1 y$.

\noindent (3) ${\preord^{F}} \not= {\preord_{\omega}}$. Again, following the examples after~\ref{htl}, it suffices to show that the synchronisation trees
\begin{eqnarray*}
p & \equiv & a(\sum_{n \in \Nat} b_{n}\Oh ) + \Omega \\
q & \equiv & {\sum_{n \in \Nat} a({\sum_{m \in \Nat - \{ n \}} b_{m} \Oh} + \Omega )} + \Omega
\end{eqnarray*}
are definable in  SCCS.
Clearly $p$ is definable in the same way as Example~(1).
For $q$, we need some additional assumptions on {\sf Act}:
\begin{itemize}
\item There are $c, \{ c_{n} \} \in {\sf Act}$ such that,
for $k, m \in \Nat$:
\begin{eqnarray*}
c^{(k)} c_{m} & = & b_{m} \;\; (k \not= m) \\
c^{(m)} c_{m} & = & b_{m + 1} 
\end{eqnarray*}
where $c^{(k)} \equiv \underbrace{c \ldots c}_{k}$, i.e. the product in the monoid {\sf Act}.
\item There is a relabelling $S$ such that
\[ S c_{n} = c_{n + 1} \;\; (n \in \Nat ). \]
\end{itemize}
(To see that these requirements can be met, let {\sf Act} be the free abelian
monoid over the generators $0, a, b_{k}, c, c_{k}$ $(k \in \Nat)$ subject
to the relations
\[ 0x = x0 = 0, \;\;\;\; c^{(k)}c_{m} = b_{m} \;\;(k \not= m), \;\;\;\; c^{(m)}c_{m} = b_{m+1} \]
for $k, m \in \Nat$.
Let $S$ be the endomorphism induced by
\[ S 0 = S a = S b_{k} = S c = 0, \;\;\;\; S c_{k} = c_{k+1} , \]
which is well-defined since $S$ preserves the relations.)

Then we can define
\begin{eqnarray*}
q & \equiv & {\sf rec} \: x. \: ar + (1 c \Oh \times x) \\
r & \equiv & {\sf rec} \: y. \: c_{1} \Oh + x[S] ,
\end{eqnarray*}
and calculate:
\begin{eqnarray*}
r & = & {\sum_{n \in \Nat} c_{n} \Oh } + \Omega , \\
q & = & {\sum_{n \in \Nat} ( \prod_{i=1}^{n} 1 c \Oh \times ar) } + \Omega \\
& = & {\sum_{n \in \Nat} a(c^{(n)} \Oh \times \sum_{m \in \Nat} c_{m} \Oh + \Omega )} + \Omega \\
& = & {\sum_{n \in \Nat} a( \sum_{m \in \Nat} (c^{(n)} c_{m}) \Oh + \Omega )} + \Omega \\
& = &  {\sum_{n \in \Nat} a( \sum_{m \in \Nat - \{ n \}} b_{m} \Oh + \Omega )} + \Omega 
\end{eqnarray*}
as required.

By contrast with Example (3), Hennessy claims in~\cite{Hen81} Theorem~4.1 that ${\preord^{F}} = {\preord_{\omega}}$ for SCCS.
The defect in his argument occurs in the definition of $p^{(n)}$ at the start of section~4 of \cite{Hen81}; 
there appears to be an implicit assumption that SCCS is sort-finite.
Indeed, as an easy consequence of our work in the previous Section, we have
\begin{proposition}
In any sort-finite transition system satisfying (BN):
\[ {\preord^{F}} = {\preord_{\omega}} . \]
\end{proposition}

\proof\ Let $p, q \in {\rm Proc}$ in such a system.
\begin{Eqarray}
p \preord^{F} q & \Longrightarrow & \Ellomega (p) \subseteq \Ellomega (q) & \\
& \Longrightarrow & \Ell_{\bigvee \infty}(p) \subseteq \Ell_{\bigvee \infty}(q) & \mbox{(BN)} \\
& \Longrightarrow & {\rm HML}_{\omega}(p) \subseteq {\rm HML}_{\omega}(q) & \\
& \Longrightarrow & p \preord_{\omega} q & \mbox{sort-finiteness. \qed}
\end{Eqarray}

Nevertheless, Hennessy's results on full abstraction are valid when $\preord_{\omega}$ is replaced by $\preord^{F}$, and we shall make use of them shortly.

Firstly, we need to extend our denotational semantics $\lsem \cdot \rsem^{\Dom}$ to recursive terms.
This is done in the standard way; we introduce environments to deal with variables, and interpret recursion by least fixed points.
\begin{definition}
{\rm Denotational semantics of recursive terms:}
\[ {\sf Env} \equiv \Dom^{\sf Var} \]
\[ \lsem \cdot \rsem^{\Dom} : {\rm REC}_{\Sigma}  \longrightarrow {\sf Env} \longrightarrow \Dom \]
\[ \begin{array}{lll}
\lsem x \rsem^{\Dom} \rho & \equiv & \rho x \\
\lsem \sigma (t_{1}, \ldots , t_{n}) \rsem^{\Dom} \rho & \equiv & \sigma^{\Dom} (\lsem t_{1} \rsem^{\Dom} \rho , \ldots , \lsem t_{n} \rsem^{\Dom} \rho ) \\
\lsem {\sf rec} \: x. \, t \rsem^{\Dom} \rho & \equiv & \mu d \in \Dom . \: \lsem t \rsem^{\Dom} \rho [ x \mapsto d ] .
\end{array} \]
\end{definition}

We now want to extend our Full Abstraction Theorem to recursive terms.
We can use Hennessy's results in~\cite{Hen81} to get a cheap proof.
In that paper, Hennessy constructs a term model $\cal I$ with the following properties:
\begin{enumerate}
\item $\cal I$ is an algebraic continuous $\Sigma$-algebra all finite elements of which are definable in $T_{\Sigma}$.
\item $\cal I$ is fully abstract for recursive terms with repect to the finitary preorder; for all $t_{1}, t_{2} \in {\rm CREC}_{\Sigma}$:
\[ t_{1} \preord^{F} t_{2} \;\; \Longleftrightarrow \;\; \lsem t_{1} \rsem^{\cal I} \sqsubseteq  \lsem t_{2} \rsem^{\cal I}. \]
\end{enumerate}
Combining (1) and (2) with Theorem~\ref{binit}, we obtain
\begin{theorem}
\label{isoalg}
$\Dom_{\Sigma}$ and $\cal I$ are isomorphic as continuous $\Sigma$-algebras.
\end{theorem}
Let $h : \Dom_{\Sigma} \rightarrow {\cal I}$ be the isomorphism given by Theorem~\ref{isoalg}.
It is immediate that $h$ preserves denotations of terms in $T_{\Sigma}$:
\[ \forall t \in T_{\Sigma}. \: h(\lsem t \rsem^{\Dom}) = \lsem t \rsem^{\cal I} . \]
To extend this to recursive terms we need one further piece of machinery.
\begin{definition}
{\rm Let $\simeq$ be the least $\Sigma$-congruence over ${\rm REC}_{\Sigma}$ generated by
\[ {\sf rec} \: x. \, t \simeq t[{\sf rec} \: x. \, t/x] . \]
Let $t_{\Omega}$ be the term obtained from $t$ by replacing each subexpression of the form ${\sf rec} \: x. \, t'$ by $\Omega$.
The {\em syntactic approximants} of $t$ are defined by:
\[ SA(t) \equiv \{ t'_{\Omega} : t' \simeq t \} . \] } 
\end{definition}
Note that $SA(t) \subseteq T_{\Sigma}$ for all $t \in {\rm CREC}_{\Sigma}$.

Now the following is standard (cf. e.g. \cite{ADJ77}):
\begin{lemma}[Syntactic Approximation]
\label{sapprox}
For all $t \in {\rm CREC}_{\Sigma}$:
\[ \lsem t \rsem^{\Dom} = \bigsqcup \{ \lsem t' \rsem^{\Dom} : t' \in SA(t) \} . \]
\end{lemma}
Hennessy proves the corresponding result for $\lsem \cdot \rsem^{\cal I}$ as his Lemma~3.4.

\begin{proposition}
\label{hpres}
For all $t \in {\rm CREC}_{\Sigma}$:
\[ h(\lsem t \rsem^{\Dom}) = \lsem t \rsem^{\cal I} . \]
\end{proposition}

\proof\ 
\begin{Eqarray}
h(\lsem t \rsem^{\Dom}) & = & h( \bigsqcup \{ \lsem t' \rsem^{\Dom} : t' \in SA(t) \} ) & \mbox{by \ref{sapprox}} \\
& = & \bigsqcup \{ h( \lsem t' \rsem^{\Dom}) : t' \in SA(t) \}  & \mbox{$h$ is continuous} \\
& = & \bigsqcup \{ \lsem t' \rsem^{\cal I} : t' \in SA(t) \}  & \mbox{by \ref{isoalg}} \\
& = & \lsem t \rsem^{\cal I} . & \qed
\end{Eqarray}

\begin{theorem}[Full Abstraction for Recursive Terms]
For all $t_{1}, t_{2} \in {\rm CREC}_{\Sigma}$:
\[ t_{1} \preord^{F} t_{2} \;\; \Longleftrightarrow \;\;  \lsem t_{1} \rsem^{\Dom} \sqsubseteq \lsem t_{2} \rsem^{\Dom} . \]
\end{theorem}

\proof\ 
\begin{eqnarray*}
t_{1} \preord^{F} t_{2} & \Longleftrightarrow & \lsem t_{1} \rsem^{\cal I} \sqsubseteq \lsem t_{2} \rsem^{\cal I} \\
& \Longleftrightarrow & \lsem t_{1} \rsem^{\Dom} \sqsubseteq \lsem t_{2} \rsem^{\Dom} ,
\end{eqnarray*}
by~\ref{hpres} and since $h$ is an order-isomorphism. \qed

Since $\Dom$ is algebraic, this result extends to terms with variables in the obvious way.
It follows that the axiomatisation of the order and equality relations between 
terms of SCCS presented in \cite{Hen81} is sound and complete for $\Dom_{\Sigma}$.
