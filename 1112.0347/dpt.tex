\section{Domains as Propositional Theories}

We begin by introducing the first part of a meta-language for
denotational semantics, the {\em type expressions}, with syntax
\[ \sigma \;\; ::= \;\; {\bf 1} \; | \; \sigma \times \tau \; | \;
\sigma \rightarrow \tau \; | \; 
\sigma \oplus \tau \; | \; (\sigma )_{\bot} \; | \;
P_{u} \sigma \; | \; 
P_{l} \sigma \; | \; t  \; | \; 
{\sf rec} \, t.\sigma \]
where $t$ ranges over type variables, and $\sigma , \tau$ over
type expressions.

The standard way of interpreting these expressions is as objects
of  {\bf SDom} (more generally as cpo's, but {\bf SDom} is closed
under all the above constructions as a subcategory of  {\bf CPO}).
Thus for each type expression $\sigma$ we define a domain
${\cal D} ( \sigma ) = (D(\sigma ), \sqsubseteq_{\sigma} )$ in
{\bf SDom}; $\sigma \times \tau$ is interpreted as product,
$\sigma \rightarrow \tau$ as function space, $\sigma \oplus \tau$
as coalesced sum, $(\sigma )_{\bot}$ as lifting, $P_{u} \sigma$
and $P_{l} \sigma$ as the upper and lower (or Smyth and Hoare)
powerdomains, and ${\rm rec} \, t. \sigma$ as the solution of the
domain equation
\[t = \sigma (t),\]
i.e. as the initial fixpoint of an endofunctor over {\bf SDom}.
Other constructions (e.g. strict function space, smash product)
can be added to the list.

So far, all this is standard (\cite {PloLN,SP82}). 
Now we begin our alternative approach.
For each type expression $\sigma$, we shall define a propositional
theory ${\cal L}(\sigma ) = (L(\sigma ), \: \leq_{\sigma}, \: =_{\sigma})$,
where:
\begin{itemize}
\item $L(\sigma )$ is a set of formulae
\item $\leq_{\sigma}$, $=_{\sigma}$ are the relations of logical
{\em entailment} and {\em equivalence} between formulae.
\end{itemize}

${\cal L}(\sigma )$ is defined inductively via formation rules,
axioms and inference rules in the usual way.
\subsection*{Formation Rules}
\[ \bullet \;\; {\sl t, f} \in L(\sigma ) \;\;\;\;\;\;\;\;
\bullet \;\; \frac{\phi , \psi \in L(\sigma )}{\phi \wedge \psi , \phi \vee \psi
\in L(\sigma )}\]
\[\bullet \;\; \frac{\phi \in L(\sigma ), \; \psi \in L(\tau )}
{(\phi \times \psi ) \in L(\sigma \times \tau ), \;
(\phi \rightarrow \psi ) \in L(\sigma \rightarrow \tau ) } \]
\[ \bullet \;\; \frac{\phi \in L(\sigma ), \; \psi \in L(\tau )}
{(\phi \oplus \false ), \; (\false \oplus \psi ) \in L(\sigma \oplus \tau )} \;\;\;\;\;\;\;\; 
\bullet \;\; \frac{\phi \in L(\sigma )}{(\phi )_{\bot} \in L((\sigma )_{\bot})} \]
\[ \bullet \;\; \frac{\phi \in L(\sigma )}
{\Box \phi \in L(P_{u} \sigma ), \; 
\Diamond \phi \in L(P_{l} \sigma ) } \;\;\;\;\;\;\;\;
\bullet \;\; \frac{\phi \in L(\sigma [{\sf rec} \, t.\sigma / t])}
{\phi \in L({\sf rec} \, t.\sigma )} \]

We should think of $(\phi \rightarrow \psi )$, $\Box \phi$ etc.
as ``constructors'' or ``generators'', which build basic formulae
at complex types from arbitrary formulae at simpler types.
Note that no constructors are introduced for recursive types;
we are taking advantage of the observation, familiar from work on information
systems \cite{LW84}, that if we work with preorders it is easy to solve
domain equations up to {\em identity}.
\subsection*{Examples}
We define separated sum as a derived operation:
\[ \sigma + \tau \equiv (\sigma )_{\bot} \oplus (\tau )_{\bot} \]
Also, we define the Sierpinski space (two-point domain):
\[ \Oh \equiv ({\bf 1})_{\bot} \]
Now we construct a number of familiar semantic domains:
\begin{center}
\begin{tabular}{|l|c|l|} \hline
name &  expression &  description \\ \hline
{\sf B} & ${\bf 1} + {\bf 1}$ & flat domain of booleans \\
{\sf N} & ${\sf rec} \: t. \, \Oh \oplus t$ & flat domain of natural numbers \\
{\sf LN} & ${\sf rec} \: t. \, {\bf 1} + t$ & lazy natural numbers \\
{\sf List(N)} & ${\sf rec} \: t. \, {\bf 1} + ({\sf N} \times t)$ & lazy lists of eager numbers \\
{\sf CBN} & ${\sf rec} \: t. \, {\sf N} + (t \rightarrow t)$ & call-by-name untyped $\lambda$-calculus \\ \hline
\end{tabular}
\end{center}
Now we define some formulas in these types, to suggest how the expected
structure emerges from the formal definitions.
\begin{center}
\begin{tabular}{|l|c|l|}  \hline
name & formula & type \\ \hline
$\star$ & $(\true )_{\bot}$ & $\Oh$ \\
{\sf true} & $(\star \oplus \false )$ & {\sf B} \\
{\sf false} & $(\false \oplus \star )$ & {\sf B} \\
$\overline{0}$ & $(\star \oplus \false )$ & {\sf N} \\
$\overline{1}$ & $(\false \oplus \overline{0})$ & {\sf N} \\
$\overline{n+1}$ & $(\false \oplus \overline{n})$  & {\sf N} \\
{\sf nil} & $(\star \oplus \false )$ & {\sf List(N)} \\
$\overline{0} :: {\sf nil}$ & $(\false \oplus (\overline{0} \times {\sf nil}))$ & {\sf List(N)} \\
$\overline{0} :: \bot$ & $(\false \oplus (\overline{0} \times \true ))$ & {\sf List(N)} \\
{\sf parallel or} & $(({\sf true} \times \true ) \rightarrow {\sf true})$ & \\
& $\mbox{} \wedge ((\true \times {\sf true}) \rightarrow {\sf true})$ & \\
& $\mbox{} \wedge (({\sf false} \times {\sf false}) \rightarrow {\sf false})$ 
& $({\sf B} \times {\sf B}) \rightarrow {\sf B}$ \\ \hline
\end{tabular}
\end{center}
\subsection*{Auxiliary Predicates}
Before proceeding to the axiomatisation proper, we shall define some
auxiliary predicates on formulas. These will be used as side-conditions
on a number of axioms and rules (e.g. $(\rightarrow - \vee - R)$ below).
Thus it is important that they are recursive predicates, defined
syntactically on formulae. The main predicates we define are:
\begin{itemize}
\item {\sf PNF}($\phi$): $\phi$ is in {\em prime normal form},
defined by the condition that disjunctions only occur in $\phi$
immediately under $\Box$.
\end{itemize}
Then for $\phi$ in PNF, we shall define:
\begin{itemize}
\item {\sf C}($\phi$): $\phi$ is {\em consistent}, i.e. so that we have
\[ {\sf C}(\phi ) \; \Longleftrightarrow \; \neg (\phi \leq {\sl f})
\; \Longleftrightarrow \; \lsem \phi \rsem \neq \varnothing \]
(where $\lsem \cdot \rsem$ is the semantics to be introduced below).
\item {\sf T}($\phi$): $\phi$ requires {\em termination}, i.e. so that we have
\[ {\sf T}(\phi ) \; \Longleftrightarrow \; \neg ({\sl t} \leq \phi )
\; \Longleftrightarrow \; \bot \not\in \lsem \phi \rsem . \] 
\end{itemize}

Of these, the idea of formal consistency, and its definition for
function spaces, go back to \cite{Kre59}, and also play a major role in
\cite{Sco81,Sco82}.
The other predicates, as syntactic conditions on expressions, are
apparently new (and in the presence of the  type constructions
we are considering, specifically function space and coalesced sum,
the definitions of {\sf C} and {\sf T} are {\em mutually recursive}).
\[ \begin{array}{lll}
{\sf C}(\true ) & \equiv & {\sf true} \\
{\sf C}(\bigwedge_{i \in I}(\phi_{i} \times \psi_{i})) & \equiv & {\sf C}(\bigwedge_{i \in I}
\phi_{i}) \; \& \; {\sf C}(\bigwedge_{i \in I}\psi_{i}) \\
{\sf C}(\bigwedge_{i \in I}(\phi_{i} \rightarrow
\psi_{i})) & \equiv  & \forall J \subseteq I. \: 
{\sf C}
(\bigwedge_{j \in J}\phi_{j})
\; \Rightarrow \; {\sf C}(\bigwedge_{j \in J}\psi_{j}) \\
{\sf C}(\bigwedge_{i \in I}(\phi_{i} \oplus {\sl f}) & & \\
\mbox {} \wedge \bigwedge_{j \in J}({\sl f} \oplus \psi_{j})) & \equiv &
\neg ({\sf T}(\bigwedge_{i \in I}\phi_{i}) \; \& \; 
{\sf T}(\bigwedge_{j \in J}\psi_{j})) \\
& & \& \; {\sf C}(\bigwedge_{i \in I}\phi_{i}) \; \& \;
{\sf C}(\bigwedge_{j \in J}\psi_{j}) \\
{\sf C}(\bigwedge_{i \in I}(\phi_{i})_{\bot}) & \equiv & 
{\sf C}(\bigwedge_{i \in I}\phi_{i}) \\
{\sf C}(\bigwedge_{i \in I}\Diamond \phi_{i}) & \equiv & \forall i \in I. \:
{\sf C}(\phi_{i}) \\
{\sf C}(\bigwedge_{i \in I} \Box \bigvee_{j \in J_{i}} \phi_{ij}) & \equiv &
\exists f \in \prod_{i \in I} J_{i}. \: {\sf C}(\bigwedge_{i \in I} \phi_{i f(i)  } )
\end{array} \]
\[ \begin{array}{lll}
{\sf T}(\bigwedge_{i \in I}\phi_{i}) & \equiv & \exists i \in I. \: {\sf T}(\phi ) \\
{\sf T}(\phi \rightarrow \psi ) & \equiv & {\sf C}(\phi ) \; \& \; {\sf T}(\psi ) \\
{\sf T}(\phi \times \psi ) & \equiv & {\sf T}(\phi ) \; \mbox{or} \; {\sf T}(\psi ) \\
{\sf T}(\phi \oplus {\sl f}) & \equiv &
{\sf T}({\sl f} \oplus \phi ) \;\; \equiv \;\; {\sf T}(\phi ) \\
{\sf T}((\phi )_{\bot}) & \equiv & {\sf true} \\
{\sf T}(\Diamond \phi ) & \equiv & {\sf T}(\Box \phi ) \equiv T(\phi ) .
\end{array} \] 

Once we have defined {\sf C} and {\sf T}, we can introduce the following derived predicates:
\begin{eqnarray*}
{\sf CPNF}(\phi) & \equiv & {\sf PNF}(\phi ) \; \mbox{and for all sub-formulae
$\psi$ of $\phi$,} \\
& &  {\sf PNF}(\psi ) \; \Rightarrow \; {\sf C}(\psi ) . \\
{\sf CDNF}(\phi ) & \equiv & \phi = \bigvee_{i \in I} \phi_{i} \; \& \; \forall   i \in I. \, {\sf CPNF}(\phi_{i}) \\
\#(\phi ) & \equiv & \phi = \bigvee_{i \in I} \phi_{i} \; \& \; \forall i \in I. \, {\sf PNF}(\phi ) \: \& \: \neg {\sf C}(\phi ) \\ 
(\phi ) \converges  & \equiv & \phi = \bigvee_{i \in I} \phi_{i} \; \& \; \forall i \in I. \, {\sf PNF}(\phi ) \: \& \: {\sf T}(\phi ).
\end{eqnarray*}

Now we turn to the axiomatization.
The axioms of our logic are all ``polymorphic'' in character,
i.e. they arise from the type constructions uniformly over
the types to which the constructions are applied.
Thus we omit type subscripts.

The axioms fall into two main groups.
\subsection*{Logical Axioms}
These give each $\cal L (\sigma )$ the structure of a distributive
lattice.
\[ ({\leq}-{\rm ref}) \;\;\; \phi \leq \phi \;\;\;\;\;\;
({\leq}-{\rm trans}) \;\;\;  \frac{\phi \leq \psi , \; \psi \leq \chi}
{\phi \leq \chi} \]
\[ ({=}-I) \;\;\; \frac{\phi \leq \psi , \; \psi \leq \phi}{\phi = \psi} 
\;\;\;\;\;\;
({=}-E) \;\;\; \frac{\phi = \psi}{\phi \leq \psi , \; \psi \leq \phi} \]
\[ ({\true}-I) \;\;\; \phi \leq \true \;\;\;\;\;\;	
({\wedge}-I) \;\;\; \frac{\phi \leq \psi_{1} , \; \phi \leq \psi_{2}}
{\phi \leq \psi_1 \wedge \psi_2} \]
\[ ({\wedge}-E-L) \;\;\; \phi \wedge \psi \leq \phi \;\;\;\;\;\; 
({\wedge}-E-R) \;\;\; \phi \wedge \psi \leq \psi  \]
\[ ({\false}-E) \;\;\; \false \leq \phi \;\;\;\;\;\;
({\vee}-I) \;\;\; \frac{\phi_{1} \leq \psi , \; \phi_{2} \leq \psi}
{\phi_{1} \vee \phi_{2} \leq \psi} \]
\[ ({\vee}-E-L) \;\;\; \phi \leq \phi \vee \psi \;\;\;\;\;\;
({\vee}-E-R) \;\;\; \psi \leq \phi \vee \psi \]  
\[ ({\wedge}-{\rm dist}) \;\;\; \phi \wedge (\psi \vee \chi )  \leq  
(\phi \wedge \psi )
\vee (\psi \wedge \chi ) \]

\subsection*{Type-specific Axioms}
These articulate each type construction, by showing how its generators
interact with the logical structure.

\[ ({\times}-{\leq}) \;\;\; \frac{\phi \leq \phi' , \; \psi \leq \psi'}
{(\phi \times \psi ) \leq (\phi' \times \psi' )} \]
\[ ({\times}-{\wedge}) \;\;\; \bigwedge_{i \in I} (\phi_{i} \times \psi_{i}) = (\bigwedge_{i \in I} \phi_{i} \times \bigwedge_{i \in I} \psi_{i} ) \]
\[ ({\times}-{\vee}-L) \;\;\; ( \bigvee_{i \in I} \phi_{i} \times \psi ) = \bigvee_{i \in I} (\phi \times \psi ) \]
\[ ({\times}-{\vee}-R) \;\;\; (\phi \times \bigvee_{i \in I}\psi_{i}) = \bigvee_{i \in I} (\phi \times \psi_{i}) \]
\[ ({\rightarrow}-\leq ) \;\;\; \frac{\phi^{\prime} \leq \phi ,\; 
\psi \leq \psi^{\prime} }
{(\phi \rightarrow \psi )  \leq (\phi^{\prime} \rightarrow \psi^{\prime} )} \]
\[ ({\rightarrow}-\wedge ) \;\;\; 
(\phi \rightarrow \bigwedge_{i \in I} \psi_i ) = 
\bigwedge_{i \in I} (\phi \rightarrow \psi_i ) \]
\[ ({\rightarrow}-\vee-L) \;\;\; 
(\bigvee_{i \in I} \phi_{i} \rightarrow \psi ) =
\bigwedge_{i \in I} (\phi_{i} \rightarrow \psi ) \]
\[ ({\rightarrow}-\vee-R) \;\;\; 
(\phi \rightarrow \bigvee_{i \in I} \psi_{i}) =
\bigvee_{i \in I} (\phi \rightarrow \psi_{i})  \;\;\;\; ({\sf CPNF}(\phi )) \]
\[ ({\oplus}-{\leq}) \;\;\; \frac{\phi \leq \psi}{(\phi \oplus \false ) \leq (\psi \oplus \false ), \; (\false \oplus \phi ) \leq (\false \oplus \psi )} \]
\[ ({\oplus}-{\wedge}-L) \;\;\; (\bigwedge_{i \in I} \phi_{i} \oplus \false ) = 
\bigwedge_{i \in I} (\phi_{i} \oplus \false ) \]
\[ ({\oplus}-{\wedge}-R) \;\;\; (\false \oplus \bigwedge_{i \in I} \psi_{i}) =
\bigwedge_{i \in I} (\false \oplus \psi_{i}) \]
\[ ({\oplus}-{\vee}-R) \;\;\; (\bigvee_{i \in I} \phi_{i} \oplus \false ) =
\bigvee_{i \in I} (\phi_{i} \oplus \false ) \]
\[ ({\oplus}-{\vee}-L) \;\;\; (\false \oplus \bigvee_{i \in I} \psi_{i}) =
\bigvee_{i \in I} (\false \oplus \psi_{i})   \]
\[ ({( \cdot )_{\bot}}-{\leq}) \;\;\; \frac{\phi \leq \psi}{(\phi )_{\bot} \leq
(\psi )_{\bot}} \]
\[ ({( \cdot )_{\bot}}-{\wedge}) \;\;\; (\phi \wedge \psi )_{\bot} = (\phi )_{\bot} \wedge (\psi )_{\bot} \]
\[ ({( \cdot )_{\bot}}-{\vee}) \;\;\; (\bigvee_{i \in I} \phi_{i} )_{\bot} =
\bigvee_{i \in I} (\phi_{i})_{\bot} \]
\[ ({\Box}-{\leq}) \;\;\; \frac{\phi \leq \psi}{\Box \phi \leq \Box \psi} \]
\[ ({\Box}-{\wedge}) \;\;\; \Box \bigwedge_{i \in I} \phi_{i} =
\bigwedge_{i \in I} \Box \phi_{i} \]
\[ ({\Box}-{\sl f}) \;\;\; \Box {\sl f} = {\sl f} \]
\[ ({\Diamond}-{\leq}) \;\;\; \frac{\phi \leq \psi}{\Diamond \phi \leq \Diamond \psi} \]
\[ ({\Diamond}-{\vee}) \;\;\; \Diamond \bigvee_{i \in I} \phi_{i} =
\bigvee_{i \in I} \Diamond \phi_{i} \]
\[ ({\Diamond}-{\true}) \;\;\; \Diamond \true = \true \]
\[ (\#) \;\;\; \phi \leq \false \;\;\;\; (\# (\phi )) \]

The axiom $({\Box}-{\sl f})$ exemplifies the possibilities for
fine-tuning in our approach. It corresponds exactly to the
{\em omission} of the empty set from the upper powerdomain.

To make precise the sense in which this axiomatic presentation is
equivalent to the usual denotational construction of domains we define,
for each (closed) type expression $\sigma$, an interpretation function
\[ \lsem \cdot \rsem_{\sigma } : L(\sigma ) \longrightarrow K \Omega (\cal D 
(\sigma ) ) \]
by
\[ \begin{array}{lll}
\lsem \phi \wedge \psi \rsem_{\sigma} & = & \lsem \phi \rsem_{\sigma} \cap
\lsem \psi \rsem_{\sigma} \\
\lsem {\sl t} \rsem_{\sigma} & = & D(\sigma ) = 1_{K \Omega (\cal D (\sigma ))} 
\\
\lsem \phi \vee \psi \rsem_{\sigma} & = & \lsem \phi \rsem_{\sigma} \cup
\lsem \psi \rsem_{\sigma} \\
\lsem \false \rsem_{\sigma} & = & \varnothing = 0_{K \Omega (\cal D (\sigma ))}
\\
\lsem (\phi \times \psi ) \rsem_{\sigma \times \tau} & = & 
\{ \ltuple u, v \rtuple : u \in \lsem \phi \rsem_{\sigma}, \; v \in \lsem \psi
\rsem_{\tau} \} \\
\lsem (\phi \rightarrow \psi ) \rsem_{\sigma \rightarrow \tau} & = &
\{ f \in D(\sigma \rightarrow \tau ) : f(\lsem \phi \rsem_{\sigma })
\subseteq \lsem \psi \rsem_{\tau} \} \\
\lsem (\phi \oplus \false ) \rsem_{\sigma \oplus \tau} & = &
\{ \ltuple 0, u \rtuple : u \in \lsem \phi \rsem_{\sigma} - \{ \bot_{\sigma} \} \} \\
&  & \mbox{} \cup \{ \bot_{\sigma \oplus \tau} : \bot_{\sigma} \in \lsem \phi \rsem_{\sigma} \} \\
\lsem (\false \oplus \psi ) \rsem_{\sigma \oplus \tau} & = &
\{ \ltuple 1, v \rtuple : v \in \lsem \psi \rsem_{\tau} - \{ \bot_{\tau} \} \} \\
& & \mbox{} \cup \{ \bot_{\sigma \oplus \tau} : \bot_{\tau} \in \lsem \psi \rsem_{\tau} \} \\
\lsem (\phi )_{\bot} \rsem_{(\sigma )_{\bot}} & = & \{ \ltuple 0, u \rtuple :
u \in \lsem \phi \rsem_{\sigma} \} \\
\lsem \Box \phi \rsem_{P_{u} \sigma } & = & \{ S \in D(P_{u} \sigma ) :
S \subseteq \lsem \phi \rsem_{\sigma} \} \\
\lsem \Diamond \phi \rsem_{P_{l} \sigma } & = & \{ S \in D(P_{l} \sigma ) : 
S \cap \lsem \phi \rsem_{\sigma} \neq \varnothing \} \\
\lsem \phi \rsem_{{\sf rec} \: t. \, \sigma} & = & 
\{ {\foldalph}_{\sigma}(u) : u \in \lsem \phi \rsem_{\sigma [{\sf rec} \: t. \, {\sigma}/ t]} \}
\end{array} \] 
where ${\foldalph}_{\sigma} : \Dom (\sigma [{\sf rec} \: t. \, {\sigma}/t]) \cong \Dom ({\sf rec} \: t. \, \sigma )$\label{foldalph} is the isomorphism
arising from the initial solution to the domain equation $t = \sigma (t)$.

Then for $\phi , \psi \in L(\sigma )$, we define
\[ \cal D (\sigma ) \models \phi \leq \psi \; \equiv \; \lsem \phi 
\rsem_{\sigma}
\subseteq \lsem \psi \rsem_{\sigma}. \]

We now use the results of Chapter~3 to establish some fundamental properties
of our system of ``Domain Logic''.

Firstly, we note that operations on prelocales in the style of
Chapter~3 can be distilled from our definitions for product, lifting and
Hoare powerdomain.
The reader will find no difficulty in carrying out the same
programme for these constructions as that shown for function space,
Smyth powerdomain and coalesced sum in Chapter~3.
Now using~\ref{obs}, we see that, for each closed $\sigma$ and any $\rho \in {\sf LEnv}$:
\[ {\cal L} \lsem \sigma \rsem \rho = \Ell (\sigma ) . \]
The following results are then immediate consequences of our work
in Chapter~3.

\noindent {\bf Notation.} ${\sf PNF}(\sigma ) \equiv \{ \phi \in L(\sigma ) :
{\sf PNF}(\phi ) \}$, 
and similarly for ${\sf CPNF}(\sigma )$, ${\sf CDNF}(\sigma )$.
\begin{proposition}
\label{metap}
For all $\phi \in {\sf PNF}(\sigma )$:
\[ \begin{array}{rl}
(i) & \lsem \phi \rsem_{\sigma} \in {\sf pr}(K \Omega (\Dom (\sigma ))) \\
(ii) & {\sf C}(\phi ) \;\; \Longleftrightarrow \;\; \lsem \phi \rsem_{\sigma}
\not= \varnothing \\
(iii) & {\sf T}(\phi ) \;\; \Longleftrightarrow \;\; \bot_{\sigma} \not\in \lsem \phi \rsem .
\end{array} \]
\end{proposition}
\begin{lemma}[Normal Forms]
\label{dptnf}
For all $\phi \in L(\sigma )$, for some $\psi \in {\sf CDNF}(\sigma )$:
\[ \Ell (\sigma ) \vdash \phi = \psi . \]
\end{lemma}

Now we define a relation
\[ {\leftrightsquigarrow} \subseteq {\sf CPNF}(\sigma ) \times K( \Dom (\sigma )):\]
\[ \phi \leftrightsquigarrow u \equiv \lsem \phi \rsem_{\sigma} = \diverges u. \]
\begin{proposition}
\label{squig}
$\leftrightsquigarrow$ is a surjective total function.
\end{proposition}

Now we come to the main results of the section:
\begin{theorem}[Soundness and Completeness]
For all $\phi , \psi \in L(\sigma )$:
\[ {\cal L} (\sigma ) \vdash \phi \leq \psi \;\; \Longleftrightarrow
\;\; {\cal D} (\sigma ) \models \phi \leq \psi . \]
\end{theorem}
Now we define 
\[{\cal LA}(\sigma ) \; \equiv \; (L(\sigma )/{=}_{\sigma}, \:
\leq_{\sigma}/{=}_{\sigma}),\]
the {\em Lindenbaum algebra} of
${\cal L}(\sigma )$.
\begin{theorem}[Stone Duality]
\label{sdual}
${\cal LA}(\sigma )$ is the Stone dual of ${\cal D}(\sigma )$, i.e.
\[\begin{array}{rl}
(i)  &  {\cal D}(\sigma ) \; \cong \; {\sf Spec} \: {\cal LA}(\sigma ) \\
(ii) &  K\Omega ( {\cal D}(\sigma )) \; \cong \; {\cal LA}(\sigma ).
\end{array} \]
\end{theorem}
