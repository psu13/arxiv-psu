\section{Logical Semantics of Types}
We now build on the work of the previous sections to give a {\it logical semantics} for a language of type expressions, in which each type is interpreted as a propositional theory (domain prelocale).
\subsection*{Syntax of Type Expressions}
We define a set of type expressions {\sf TExp} by
\[ \sigma \;\; ::= \;\; \mbox{OP}(\sigma_{1}, \ldots \sigma_{n}) \; (\mbox{OP} \in \Sigma_{n}) \; | \; t \; | \; {\sf rec} \: t. \sigma \]
where $t$ ranges over a set of type variables {\sf TVar}, $\sigma$ over type expressions, and 
$\Sigma = \{\Sigma_{n}\}_{n \in \omega}$ is a ranked alphabet of type constructors. 
For each such constructor $\mbox{OP} \in \Sigma_{n}$, we assume we have an operation ${\rm op}^{{\cal L}} : {\bf DPL1}^{n} \rightarrow {\bf DPL1}$ which satisfies properties (T1) -- (T6) from the previous section with respect to a functor ${\rm op}^{{\cal D}} : {\bf SDom}^{n} \rightarrow {\bf SDom}$.
\subsection*{Logical Semantics of Type Expressions}
We define a semantic function
\[ {\cal L} : {\sf TExp} \longrightarrow {\sf LEnv} \longrightarrow {\bf DPL1} \]
where ${\sf LEnv}$ is the set of type environments
\[ {\sf TVar} \longrightarrow {\bf DPL1} \]
as follows:
\begin{eqnarray*}
{\cal L} \lsem \mbox{OP}(\sigma_{1}, \ldots ,\sigma_{n}) \rsem \rho & = & {\rm op}^{{\cal L}}({\cal L}\lsem \sigma_{1} \rsem \rho , \ldots ,{\cal L}\lsem \sigma_{n} \rsem \rho ) \\
{\cal L} \lsem t \rsem \rho & = &  \rho t \\
{\cal L} \lsem {\sf rec} \: t. \sigma \rsem \rho & = & {\sf fix}(F) = \bigsqcup_{k \in \omega}F^{k}({\bf 1}),
\end{eqnarray*} 
where $F : {\bf DPL1} \rightarrow {\bf DPL1}$ is defined by
\[ F(A) = {\cal L} \lsem \sigma \rsem \rho [t \mapsto A]. \]
We write ${\cal LA}(\sigma ) \rho$ for $\tilde{A}$, where $A = {\cal L} \lsem
\sigma \rsem \rho$.
\subsection*{Denotational Semantics of Type Expressions}
Similarly to the logical semantics, we define
\[ {\cal D} : {\sf TExp} \longrightarrow {\sf DEnv} \longrightarrow {\bf SDom} \]
where ${\sf DEnv} = {\sf TVar} \longrightarrow {\bf SDom}$. In this semantics, each $\mbox{OP} \in \Sigma_{n}$ is interpreted by the corresponding functor
\[ \mbox{op}^{{\cal D}} : ({\bf SDom^{E}})^{n} \longrightarrow {\bf SDom^{E}} \]
and ${\sf rec} \: t. \sigma$ as the inititial fixed point of the endofunctor 
${\bf SDom^{E}} \longrightarrow {\bf SDom^{E}}$ induced from  
$t \mapsto \sigma (t)$. 
See \cite[Chapter 5]{PloLN} and \cite{SP82,Nie84}.

\begin{theorem}[Stone Duality]
Let $\rho_{L} \in {\sf LEnv}$, $\rho_{D} \in {\sf DEnv}$ satisfy:
\[ \forall t \in {\sf TVar}. \, K \Omega (\rho_{D} t) \; \cong \; \rho_{L} t. \]
Then for any type expression $\sigma$, ${\cal LA} \lsem \sigma \rsem \rho_{L}$ is the Stone dual of ${\cal D} \lsem \sigma \rsem \rho_{D}$, i.e.
\[ \begin{array}{rl}
(i) & {\cal D} \lsem \sigma \rsem \rho_{D} \; \cong \; {\sf Spec} \; {\cal LA} \lsem \sigma \rsem \rho_{L} \\
(ii) & K \Omega ({\cal D} \lsem \sigma \rsem \rho_{D}) \; \cong \; 
{\cal LA} \lsem \sigma \rsem \rho_{L}. 
\end{array} \]
\end{theorem}

\proof\ Firstly, note that the two conclusions of the Theorem are equivalent, since Scott domains are coherent spaces. Thus it suffices to prove $(i)$.

It will be convenient to consider systems of simultaneous domain equations
\begin{equation}
\left. \begin{array}{rcl} 
\xi_{1} & = & \sigma_{1}(\xi_{1}, \ldots , \xi_{n})  \\
         & \vdots &  \label{systeq} \\
\xi_{n} & = & \sigma_{n}(\xi_{1}, \ldots , \xi_{n})
\end{array} \right\} 
\end{equation}
where each $\sigma_{i}$ is a type expression not containing any occurrences 
of ${\sf rec}$. 
It is standard that any $\sigma \in {\sf TExp}$ is equivalent to a system of 
equations of this form, in the sense that the denotation of $\sigma$ 
is isomorphic to a component of the solution of such a system. 
Thus what we shall show is that $\hat{A} \cong D$, where $A$ is the solution 
of~\ref{systeq} in {\bf DPL1} and $D$ is the solution in {\bf SDom}. 
To make this more precise, we need some definitions.

Firstly, we define a diagram $\Delta^{D}$ in $({\bf SDom}^{E})^{n}$ as follows:
\[ \Delta^{D} = (D_{n}, f_{n})_{n \in \omega} \]
where
\begin{eqnarray*}
D_{0} & = & ({\bf 1}^{\cal D}, \ldots , {\bf 1}^{\cal D}) \\
D_{k+1} & = & ({\cal D} \lsem \sigma_{1} \rsem \rho^{D} [\vec{\xi} \mapsto D_{k}], \ldots , {\cal D} \lsem \sigma_{n} \rsem \rho^{D} [\vec{\xi} \mapsto D_{k}])
\end{eqnarray*}
and $f_{k} : D_{k} \rightarrow D_{k+1}$ is defined as follows: $f_{0}$ is the unique morphism given by initiality of $D_{0}$ in $({\bf SDom}^{E})^{n}$;
\[ f_{k+1} =  ({\cal D}_{m} \lsem \sigma_{1} \rsem \rho_{m}^{D} [\vec{\xi} \mapsto f_{n}], \ldots , {\cal D}_{m} \lsem \sigma_{n} \rsem \rho_{m}^{D} [\vec{\xi} \mapsto f_{n}]) \]
where ${\cal D}_{m}$ gives the morphism part of the functor corresponding to $\sigma$, 
and $\rho_{m}^{D} t = {\sf id}_{\rho^{D} t}$. 
Now it is standard that the solution of \ref{systeq} in {\bf SDom} is given by 
\[ \lim_{\rightarrow} \Delta^{D}. \]

Similarly, we define a $\unlhd$--chain $\{ A_{n} \}$ in ${\bf DPL1}^{n}$ by 
\begin{eqnarray*}
A_{0} & = &  ({\bf 1}^{\cal L}, \ldots , {\bf 1}^{\cal L}) \\
A_{k+1} & = & ({\cal L} \lsem \sigma_{1} \rsem \rho^{L} [\vec{\xi} \mapsto A_{k}], \ldots , {\cal L} \lsem \sigma_{n} \rsem \rho^{L} [\vec{\xi} \mapsto A_{k}])
\end{eqnarray*}
and we let $\Delta^{L}$ be the diagram $(\hat{A}_{k}, e_{k})$ in $({\bf SDom}^{E})^{n}$, where $e_{k} : \hat{A}_{k} \rightarrow \hat{A}_{k+1}$ is the tuple of embeddings 
\[ e_{k, i} : \hat{A}_{k, i} \rightarrow \hat{A}_{k+1, i} \;\;\; (1 \leq i \leq n) \] 
induced by $A_{k, i} \unlhd A_{k+1, i}$. 
Now the solution of \ref{systeq} in {\bf DPL1} is given by 
\[ A_{\infty} = \bigsqcup_{k} A_{k} = (\bigsqcup_{k} A_{k, 1}, \ldots , \bigsqcup_{k} A_{k, n}). \]
It is easily verified that the cone 
$\mu : \Delta^{L} \rightarrow \hat{A}_{\infty}$ with $\mu_{k}$ the embedding induced by $A_{k} \unlhd A_{\infty}$ is colimiting in $({\bf SDom}^{E})^{n}$. 
Thus our task reduces to proving
\[ \lim_{\rightarrow} \Delta^{L} \; \cong \; \lim_{\rightarrow} \Delta^{D}, \]
for which it suffices to construct a natural isomorphism $\nu : \Delta^{L} \; \cong \; \Delta^{D}$.

We fix $\vec{\sigma} = (\sigma_{1}, \ldots , \sigma_{n})$ as the system of equations under consideration. For each $\vec{\tau} = (\tau_{1}, \ldots , \tau_{n})$ where each $\tau_{i}$ contains no occurrences of ${\sf rec}$, and $k \in \omega$, we shall define:
\begin{itemize}
\item objects $D_{\vec{\tau}, k}$ and morphisms 
\[ f_{\vec{\tau}, k} : D_{\vec{\tau}, k} \rightarrow D_{\vec{\tau}, k+1} \]
in $({\bf SDom}^{E})^{n}$;
\item objects $A_{\vec{\tau}, k}$ in ${\bf DPL1}^{n}$ and morphisms
\[ e_{\vec{\tau}, k} : \hat{A}_{\vec{\tau}, k} \rightarrow \hat{A}_{\vec{\tau}, k+1} \]
\item morphisms $\nu_{\vec{\tau}, k} : \hat{A}_{\vec{\tau}, k} \rightarrow D_{\vec{\tau}, k}$.
\end{itemize}
\[ D_{\vec{\tau}, 0} = ({\bf 1}^{\cal D}, \ldots , {\bf 1}^{\cal D}); \;\;\;  A_{\vec{\tau}, 0} = ({\bf 1}^{\cal L}, \ldots , {\bf 1}^{\cal L}) \]
\[ D_{\vec{\tau}, k+1} = ({\cal D} \lsem \tau_{1} \rsem \rho^{D} [\vec{\xi} \mapsto D_{\vec{\sigma}, k}], \ldots , {\cal D} \lsem \tau_{n} \rsem \rho^{D} [\vec{\xi} \mapsto D_{\vec{\sigma}, k}]) \]
\[ A_{\vec{\tau}, k+1} = ({\cal L} \lsem \tau_{1} \rsem \rho^{L} [\vec{\xi} \mapsto A_{\vec{\sigma}, k}], \ldots , {\cal L} \lsem \tau_{n} \rsem \rho^{L} [\vec{\xi} \mapsto A_{\vec{\sigma}, k}]) \] 
$f_{\vec{\tau}, 0}$ is the unique morphism given by initiality.
\[ f_{\vec{\tau}, k+1} = ({\cal D}_{m} \lsem \tau_{1} \rsem \rho^{D} [\vec{\xi} \mapsto f_{\vec{\sigma}, k}], \ldots , {\cal D}_{m} \lsem \tau_{n} \rsem \rho^{D} [\vec{\xi} \mapsto f_{\vec{\sigma}, k}]) \]
$e_{\vec{\tau}, k+1}$ is the embedding induced by
\[ A_{\vec{\tau}, k} \trianglelefteq A_{\vec{\tau}, k+1} \]
which holds since $A_{\vec{\sigma}, k} \trianglelefteq A_{\vec{\sigma}, k+1}$ by the usual argument.
$\nu_{\vec{\tau}, 0}$ is the unique isomorphism arising from $\hat{{\bf 1}}^{\cal L} \cong {\bf 1}^{\cal D}$.
\[ \nu_{\vec{\tau}, k+1} = (\nu_{\tau_{1}, k+1}, \ldots , \nu_{\tau_{n}, k+1}), \]
where $\nu_{\tau , k+1}$ is defined by induction on $\tau$:
\[ \nu_{\xi_{i}, k+1} = \nu_{\sigma_{i}, k} \]
\[ \nu_{t, k+1} = {\hat{\rho}}^{L} t \cong {\rho}^{D} t, \]
the isomorphism given in the hypothesis of the theorem.
For $\tau = \mbox{OP}(\theta_{1}, \ldots , \theta_{m})$,
\[ \nu_{\tau , k+1} = \mbox{op}^{\cal D}(\nu_{\theta_{1}, k+1}, \ldots , 
\nu_{\theta_{m}, k+1}) \circ \eta_{\tau , k+1}, \]
where $\eta_{\tau , k+1} : \hat{A}_{\tau , k+1} \cong \mbox{op}^{\cal D}(\hat{A}_{\theta_{1}, k+1}, \ldots , \hat{A}_{\theta_{m}, k+1})$ is the isomorphism given by property (T6)(B) for $\mbox{OP}$.

Note that
\[ {\Delta}^{D} = (D_{\vec{\sigma}, k}, f_{\vec{\sigma}, k})_{k \in \omega}, \]
\[ {\Delta}^{L} = (\hat{A}_{\vec{\sigma}, k}, e_{\vec{\sigma}, k})_{k \in \omega}, \]
and so, defining $\nu : \Delta^{L} \rightarrow \Delta^{D}$ by $\nu_{k} \equiv \nu_{\vec{\sigma}, k}$, it remains to verify that for all $k$:
\begin{itemize}
\item $\nu_{k}$ is an isomorphism
\item $\nu_{k+1} \circ e_{k} = f_{k} \circ \nu_{k}$.
\end{itemize}
We argue by induction on $k$. 
The basis follows from the fact that $\hat{{\bf 1}}^{\cal L} \cong {\bf 1}^{\cal D}$, and the initiality of $({\bf 1}^{\cal D}, \ldots , {\bf 1}^{\cal D})$ in $({\bf SDom}^{E})^{n}$. For the inductive step, we assume:
\[ (i) \;\; \nu_{k} = \nu_{\vec{\sigma}, k} \; \mbox{is an isomorphism} \]
\[ (ii) \;\; \nu_{k+1} \circ e_{k} = \nu_{\vec{\sigma}, k+1} \circ e_{\vec{\sigma}, k} = f_{\vec{\sigma}, k} \circ \nu_{\vec{\sigma}, k} = f_{k} \circ \nu_{k} \]
and prove that for all $\tau$ with no occurrences of ${\sf rec}$,
\[ (iii) \;\; \nu_{\tau , k+1} \; \mbox{is an isomorphism} \]
\[ (iv) \;\; \nu_{\tau , k+2} \circ e_{\tau , k+1} = f_{\tau , k+1} \circ \nu_{\tau , k+1} \]
(where $(e_{\tau , k+1}, \ldots , e_{\tau , k+1}) = e_{(\tau , \ldots , \tau ), k+1}$,
and similarly for $f_{\tau , k+1}$).
Taking $\tau = \sigma_{i}$, $1 \leq i \leq n$ in $(iii)$ and $(iv)$ then yields
\[ (v) \;\; \nu_{k+1} = \nu_{\vec{\sigma}, k+1} \; \mbox{is an isomorphism} \]
and
\begin{eqnarray*}
 (vi) \;\; \nu_{k+2} \circ e_{k+1} = \nu_{\vec{\sigma}, k+2} \circ e_{\vec{\sigma}, k+1} & = & f_{\vec{\sigma}, k+1} \circ \nu_{\vec{\sigma}, k+1}\\
& = & f_{k+1} \circ \nu_{k+1}, 
\end{eqnarray*}
as required. We prove $(iii)$ and $(iv)$ by induction on $\tau$.

\noindent Case 1: $\tau = \xi_{i}$. In this case, $(iii)$ just says that $\nu_{\sigma_{i}, k}$ is an isomorphism, and $(iv)$ that
\[ \nu_{\sigma_{i}, k+1} \circ e_{\sigma_{i}, k} = f_{\sigma_{i}, k} \circ \nu_{\sigma_{i}, k}, \]
and we can use our outer induction hypothesis on $k$.

\noindent Case 2: $\tau = t$. In this case, $\tau$ denotes a constant functor, and
\[ f_{\tau , k+1} = {\sl id}_{D_{\tau , k+1}}, \]
\[ e_{\tau , k+1} = {\sl id}_{\hat{A}_{\tau , k+1}}, \]
\[ \nu_{\tau , k+1} = \nu_{\tau , k+2} = 
(\hat{\rho}^{L} t \cong \rho^{D} t), \]
so $(iii)$ and $(iv)$ hold trivially.

\noindent Case 3: $\tau = \mbox{OP}(\theta_{1}, \ldots , \theta_{m})$. Applying our inner induction hypothesis to each $\theta_{i}$, we have
\[ (vii) \;\; \nu_{\theta_{i}, k+1} \; \mbox{is an isomorphism} \]
\[ (viii) \;\; \nu_{\theta{i}, k+2} \circ e_{\theta{i}, k+1} = f_{\theta_{i}, k+1} \circ \nu_{\theta_{i}, k+1}. \]
By definition,
\[ \nu_{\tau , k+1} = \mbox{op}^{\cal D}(\nu_{\theta_{1}, k+1}, \ldots , \nu_{\theta_{m}, k+1}) \circ \eta_{\tau , k+1}. \]
Since $\mbox{op}^{\cal D}$ is a functor, by $(vii)$ $\mbox{op}^{\cal D}(\nu_{\theta_{1}, k+1}, \ldots , \nu_{\theta_{m}, k+1})$ is an isomorphism; while $\eta_{\tau , k+1}$ is given as an isomorphism by (T6)(B). 
This proves~$(iii)$. Finally,
\[ \begin{array}{clr}
& \nu_{\tau , k+2} \circ e_{\tau , k+1} & \\
= & \mbox{op}^{\cal D}(\nu_{\theta_{1}, k+2}, \ldots , \nu_{\theta_{m}, k+2}) \circ \eta_{\tau , k+2} \circ e_{\tau , k+1} & \\
= & \mbox{op}^{\cal D}(\nu_{\theta_{1}, k+2}, \ldots , \nu_{\theta_{m}, k+2}) \circ \mbox{op}^{\cal D}(e_{\theta_{1}, k+1}, \ldots , e_{\theta_{m}, k+1}) \circ \eta_{\tau , k+1} \\
  & \mbox{by (T6)(B)} \\
= & \mbox{op}^{\cal D}(\nu_{\theta_{1}, k+2} \circ e_{\theta_{1}, k+1}, \ldots , \nu_{\theta_{m}, k+2} \circ e_{\theta_{m}, k+1}) \circ \eta_{\tau , k+1} & \\ 
= & \mbox{op}^{\cal D}(f_{\theta_{1}, k+2} \circ \nu_{\theta_{1}, k+1}, \ldots , f_{\theta_{m}, k+2} \circ \nu_{\theta_{m}, k+1}) \circ \eta_{\tau , k+1} \\
  & \mbox{by $(viii)$}  \\ 
= & \mbox{op}^{\cal D}(f_{\theta_{1}, k+2}, \ldots , f_{\theta_{m}, k+2}) \circ \mbox{op}^{\cal D}(\nu_{\theta_{1}, k+1}, \ldots , \nu_{\theta_{m}, k+1}) \circ \eta_{\tau , k+1} \\
= & f_{\tau , k+2} \circ \nu_{\tau , k+1}, &
\end{array} \] 
which proves $(iv)$. \qed

We finish with an observation that will be useful in the next Chapter. 
In our definitions of the constructions $A \rightarrow B$ etc. in section 4, 
we used the ``semantic'' predicates $pr$, $con$, $t$ at the argument types $A$, $B$. 
Now suppose we are forming a theory as the denotation of a type expression, 
e.g. ${\cal L} \lsem \sigma \rightarrow \tau \rsem \rho$; 
the arguments are $A = \lsem \sigma \rsem \rho$, $B = \lsem \tau \rsem \rho$. 
Then it makes sense to use the {\it syntactic} predicates ${\sf PNF}(A)$, 
${\sf CON}(A)$, 
${\sf T}(A)$ etc. in our definition of 
\[ A \rightarrow B = {\cal L} \lsem \sigma \rightarrow \tau \rsem \rho. \] 
Using properties (T1), (T2) and (T8) for each type construction, 
it is straightforward to prove the

\begin{observation}
\label{obs}
For all $\sigma$, $\rho$ the same theory is obtained as 
${\cal L} \lsem \sigma \rsem \rho$ whether syntactic or semantic predicates are used in each application of a type construction. \qed
\end{observation}
