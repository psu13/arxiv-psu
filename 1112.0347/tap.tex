\maketitle
\chapter*{Abstract}
\addtocounter{page}{1}
The mathematical framework of Stone duality is used to synthesize a number of hitherto separate developments in Theoretical Computer Science:
\begin{itemize}
\item Domain Theory, the mathematical theory of computation introduced by Scott as a foundation for denotational semantics.
\item The theory of concurrency and systems behaviour developed by Milner, Hennessy {\it et al.} based on operational semantics.
\item Logics of programs.
\end{itemize}
Stone duality provides a junction between semantics (spaces of points = denotations of computational processes) and  logics (lattices of {\em properties} of 
processes).
Moreover, the underlying logic is {\em geometric}, which can be computationally
interpreted as the logic of {\em observable} properties---i.e. properties which
can be determined to hold of a process on the basis of a finite amount
of information about its execution.

These ideas lead to the following  programme:
\begin{enumerate}
\item A metalanguage is introduced, comprising
\begin{itemize}
\item types = universes of discourse for various computational situations.
\item terms = programs = syntactic intensions for models or points.
\end{itemize}
\item A standard denotational interpretation of the metalanguage is given, assigning domains to types and domain elements to terms.
\item The metalanguage is also given a {\em logical} interpretation, in which 
types are interpreted as  propositional theories and terms are interpreted 
{\it via} a program logic, which axiomatizes the properties they satisfy.
\item The two interpretations are related by showing that they are Stone duals of each other.
Hence, semantics and logic are guaranteed to be in harmony with each other, and in fact each determines the other up to isomorphism.
\item 
This opens the way to a whole range of applications.
Given a denotational description of a computational situation in our
meta-language, we can turn the handle to obtain a logic for that situation.
\end{enumerate}
\section*{Organization}
Chapter~1 is an introduction and overview.
Chapter~2 gives some background on domains and locales.
Chapters~3 and~4 are concerned with~1--4 above.
Chapters~5 and~6 each develop a major case study along the lines suggested by~5, in the areas of concurrency and $\lambda$-calculus respectively.
Finally, Chapter~7  discusses directions for further research.
