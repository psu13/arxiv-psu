\section{Variations}

Throughout this Chapter, we have focussed on the lazy $\lambda$-calculus. 
We round off our treatment by briefly considering the varieties of function space.

\subsection*{1. The Scott function space}
$[D \rightarrow E]$, the standard function space of all continuous functions from $D$ to $E$, 
which we treated in Chapters 3 and 4. 
In terms of our domain logic $\cal L$, we can obtain this construction by adding the axiom
\[ (1) \;\; \true \leq ( \true \rightarrow \true ) . \]
Note that with (1), $\cal L$ collapses to a single equivalence class (corresponding to the trivial one-point solution of $D = [ D \rightarrow D]$). 
For this reason, Coppo {\it et al.} have to introduce atoms in their work on Extended Applicative Type Structures \cite{CDHL84}.

\subsection*{2. The strict function space}
$[D \rightarrow_{\bot} E]$, all {\em strict} continuous functions. 
This satisfies (1), and also
\[ (2) \;\; ( \true \rightarrow_{\bot} \phi ) \leq \false \;\; (\phi \converges ) . \]

\subsection*{3. The lazy function space}
$[D \rightarrow E]_{\bot}$, which satisfies neither (1) nor (2). 
This has of course been our object of study in this Chapter.

\subsection*{4. The Landin-Plotkin function space}
$[D \rightarrow_{\bot} E]_{\bot}$, the lifted strict function space. 
This satisfies (2) but not (1). 
The reason for our nomenclature is that this construction in the category of 
domains and strict continuous functions corresponds to Plotkin's 
$[D \rightharpoonup E]$ construction in his (equivalent) category of 
predomains and partial functions \cite{Plo85}.
Moreover, this may be regarded as the formalisation of Landin's applicative-order $\lambda$-calculus, 
with abstraction used to protect expressions from evaluation, as illustrated extensively in \cite{Lan64,Lan65,Bur75}.

The intriguing point about these four constructions is that (1) and (2) are {\em mathematically} natural, 
yielding cartesian closure and monoidal closure in e.g. {\bf CPO} and ${\bf CPO}_{\bot}$ respectively (the latter being analogous to partial functions over sets); 
while (3) and (4) are {\em computationally} natural, as argued extensively for (3) in this Chapter, 
and as demonstrated convincingly for (4) by Plotkin in his work on predomains \cite{Plo85}. 
Much current work is aimed at providing good categorical descriptions of generalisations of (4) \cite{Ros86,RR87,Mog86,Mog87a,Mog87b}; it remains to be seen if a similar programme can be carried out for (3).
