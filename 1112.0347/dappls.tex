\section{Applications: The Logic of a Domain Equation}
A denotational analysis of a computational situation results in the
description of a domain which provides an appropriate semantic universe
for this situation.
Canonically, domains are specified by type expressions in a metalanguage.
We can then use our approach to ``turn the handle'', and generate 
a logic for this situation in a quite mechanical way.

We shall now go on to develop two case studies of this kind, in the areas of concurrency (Chapter~5) and the $\lambda$-calculus (Chapter~6).
