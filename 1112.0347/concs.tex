\chapter{Further Directions}
Our development of the research programme adumbrated in Chapter~1 has been fairly extensive, but certainly not complete.
There are many possibilities for extension and generalisation of our results.
In this Chapter, we shall try to pick out some of the most promising topics for future research.
\begin{enumerate}
\item A first, very basic extension would be to rework the material of Chapters~3 and~4 for {\bf SFP} rather than {\bf SDom}.
In terms of the meta-language, the extension would be to incorporate the Plotkin powerdomain and the associated term constructions.
Our treatment of the Plotkin powerdomain in a specific instance in Chapter~5 should convey the general flavour of what is involved.
The extension to {\bf SFP} is conceptually straightforward; we remain within the sphere of coherent spaces.
However, there are some technical intricacies which arise with the meta-predicates, 
to do with the fact that the identification of primes is more subtle in the 
{\bf SFP} case; this should be clear from our work on normal forms in Chapter~5 section~4.
These intricacies are negotiable, and indeed I claim that all our work in this 
thesis {\em does} carry over (a detailed account, taking Chapters~3 and~4 of  
the present thesis as its starting point, is being worked out by a student of 
Glynn Winskel's \cite{Zha87}).
\item All our work in this thesis has been based on Domain Theory, simply because this is the best established and most successful foundation for denotational semantics, and a wealth of applications are ready to hand.
However, our programme is really much more general than this.
{\em Any} category of topological spaces in which a denotational metalanguage can be interpreted, and for which a suitable Stone duality exists, could serve as the setting for the same kind of exercise as we carried out in Chapter~4.
As one example of this: the main alternatives to domains in denotational 
semantics over the past few years have been {\em compact ultrametric spaces} 
\cite{Niv81,deBZ82,Mat85}.
These spaces in their metric topologies are Stone spaces, and indeed the category of compact ultrametric spaces and continuous maps is {\em equivalent} to the category of second-countable Stone spaces \cite{Abr85?}.
A restricted denotational metalanguage comprising product, (disjoint) sum and powerdomain (the Vietoris construction \cite{Joh85,Smy83}, which in this context is induced by the Hausdorff metric \cite{Niv81,deBZ82,Mat85}), can be interpreted in {\bf Stone}, together with the corresponding sub-language of terms (with {\em guarded} recursion, leading to {\em contracting} maps, and hence unique fixpoints \cite{Niv81,deBZ82,Mat85}).
Under the classical Stone duality as expounded in Chapter~1, the corresponding logical structures are Boolean algebras, and a {\em classical} logic can be presented for this metalanguage in entirely analogous fashion to that of Chapter~4.
Since the meta-language is rich enough to express a domain equation for 
synchronisation trees, a case study along the same lines as that of Chapter~5 can be carried through.
Moreover, there is a satisfying relationship between the Stone space of 
synchronisation trees (which is the metric topology on the ultrametric 
space constructed in \cite{deBZ82}), and the corresponding domain studied in 
Chapter~5; namely, the former is the {\em subspace of maximal elements} of the latter. 
This is in fact an instance of a general relationship, as set out in \cite{Abr85?}.
The important point here is that our programme is just as applicable to the metric-space  approach to denotational semantics as to the domain-theoretic approach.
\item A further kind of generalisation would be to structures other than topological spaces.
Many Stone-type dualities in such alternative contexts are known; e.g. Stone-Gelfand-Naimark duality for $C^{\star}$-algebras, Pontrjagin duality for topological groups, Gabriel-Ulmer duality for locally finitely presented categories, etc. \cite{Joh82}.
Particularly promising for Computer Science applications are the measure-theoretic dualities studied by Kozen \cite{Koz83} as a basis for the semantics and logic of probabilistic programs.
A very interesting feature of these dualities is that whereas the purely 
topological dualities have the Sierpinski space $\Oh$ as their 
``schizophrenic object'' (see \cite[Chapter 6]{Joh82}), i.e. the fundamental relationship $P \models \phi$ takes values in $\{ 0, 1 \}$, the measure-theoretic dualities take their ``characters'' in the reals; satisfaction of a measurable function by a measure is expressed by {\em integration} \cite{Koz83}.
The richer mathematical structure of these dualities should deepen our understanding of the framework.
Furthermore, there are intriguing connections with Lawvere's concept of ``generalised logics'' \cite{Law73}.
\item The logics of compact-open sets considered in this thesis have been very weak in expressive power, and are clearly inadequate as a specification formalism.
For example, we cannot specify such properties of a stream computation as ``emits an infinite sequence of ones''.
Thus we need a language, with an accompanying semantic framework, which permits us to go beyond compact-open sets.
A first step would be to allow the expression of more general open sets, e.g. by means of  a least fixed point operator on formulae $\mu p . \phi$, permitting the finite description of infinite disjunctions $\bigvee_{i \in \omega} \phi^{i}(\false )$.
This would have the advantage of not requiring any major extension of our semantics, but would still not be sufficiently expressive for specification purposes, as the above example shows.
What is needed is the ability to express infinite {\em conjunctions}, e.g. by {\em greatest} fixpoints $\nu p . \phi$, corresponding to $\bigwedge_{i \in \omega} \phi^{i}(\true )$.
Such an extension of our logic would necessarily take us beyond open sets.
An important topic for further investigation is whether such an extension can be smoothly engineered and given a good conceptual foundation.

Another reason for extending the logic is the tempting proximity of locale theory to topos theory.
Could this be the basis of the junction between topos theory and Computer Science which many researchers have looked for but none has yet convincingly demonstrated?
We must leave this point unresolved.
If there {\it is} a natural extension of our work to the level of topos theory, we have not (yet) succeeded in finding it.
\item Another variation is to change the {\em morphisms} under consideration.
Stone dualities relating to the various powerdomain constructions
(i.e. dualities for {\em multi-functions} rather than functions) are
interesting for a number of reasons: they generalise
{\em predicate transformers} in the sense of Dijkstra \cite{Dij76,Smy83};
dualities for the Vietoris construction provide a natural setting
for intuitionistic modal logic, with interesting differences to
the approach recently taken by Plotkin and Stirling;
while there are some remarkable {\em self-dualities} arising from the
Smyth powerdomain \cite{Vic87c}.
These turn out, quite unexpectedly, to provide a model for
Girard's classical linear logic \cite{Gir87}; more speculatively,
they also suggest the
possibility of a homogeneous logical framework in which programs and
properties are interchangeable.
This may turn out to provide the basis for a unified and systematic
treatment of a number of existing  {\it ad hoc} formalisms \cite{GS86,Win85}.
\item Turning now to the first of our case studies, a number of interesting further developments suggest themselves.
Firstly, from the results of Chapter~5, we can define a fully abstract denotational semantics for SCCS in our denotational metalanguage, and faithfully interpret Hennessy-Milner logic into our domain logic.
Thus we should {\em automatically} get a compositional proof theory for HML.
It would be particularly worthwhile to demonstrate this in detail, as the construction of compositional proof systems for HML by Stirling \cite{Sti87} and Winskel \cite{Win85} is one of the most impressive examples to date of the exercise of {\it ad hoc} ingenuity in the design of program logics.

Other useful extensions of our work would be to equivalences other then bisimulation (hard); and to countable non-determinism, using Plotkin's powerdomain for countable non-determinism \cite{Plo82}.
An interesting point about this construction is that  we lack a good representation for it, and a logical description might help.
\item Our development of the lazy $\lambda$-calculus represents no more than a beginning.
An extensive study is being undertaken by Luke Ong; anyone interested in pursuing the subject further is strongly recommended to read his forthcoming thesis 
(Imperial College, University of London; expected 1988).
\item Some more general points concerning the two case studies.
Firstly, the operational models we study---labelled transition systems in Chapter~5 and lambda transition systems in Chapter~6---are {\em almost} derived in a systematic way from our domain equations.
Namely, a labelled transition system is a map
\[ {\rm Proc} \longrightarrow \wp (({\sf Act} \times {\rm Proc} ) \cup \{ \bot \} ) \]
i.e. a coalgebra of the functor (on {\bf Set})
\[ X \mapsto \wp (({\sf Act} \times X ) \cup \{ \bot \} ) . \]
Similarly, an applicative transition system is a coalgebra of the {\bf Set}-functor
\[ X \mapsto (X \rightarrow X) \cup \{ \bot \} . \]
Since ${\sf Act} \times {\cal D} \cup \{ \bot \}$ can be put in natural 
bijection with $\sum_{a \in {\sf Act}} {\cal D}$, and $( {\cal D} \rightarrow {\cal D} ) \cup \{ \bot \}$ with $( {\cal D} \rightarrow {\cal D} )_{\bot}$, we see that our domain equations give rise to essentially the {\em same} functors, but over domains rather than sets.
Moreover, because of the limit-colimit coincidence in Domain theory \cite{SP82}, we can take the {\em initial solution} of a domain equation (with respect to embeddings) as the {\em final coalgebra} (with respect to  projections).
Thus our results can in some sense be seen as concerning the interpretation and ``best approximation'' of {\bf Set}-based structures in topological ones.
Clearly some general theory is called for here.
\item Finally, one of our aims in Chapters~5 and~6 was to place the study of functional languages and concurrency on as similar a footing as possible.
Much remains to be done here, although we hope to have made a useful first step.
\end{enumerate}
