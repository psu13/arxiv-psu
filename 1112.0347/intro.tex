\tableofcontents
\chapter{Introduction}
The main aim of this thesis is to synthesize a number of hitherto separate developments in Theoretical Computer Science and Logic:
\begin{itemize}
\item Domain Theory, the mathematical theory of computation introduced by Scott as a foundation for denotational semantics.
\item The theory of concurrency and systems behaviour developed by Milner, Hennessy {\it et al.} based on operational semantics.
\item Logics of programs.
\item Locale Theory.
\end{itemize}
The key to our synthesis is the mathematical theory of Stone duality, which provides a junction between semantics (topological spaces) and the {\em logic of observable properties} (locales).
As a worked example, we show how Domain Theory can be construed as a logic of observable properties;
and explore some applications to the study of programming languages.
\section{Background}
Domain Theory has been extensively studied since it was introduced by Scott 
\cite{Sco70}, both as regards the basic mathematical theory \cite{PloLN}, 
and the applications, particularly in denotational semantics 
\cite{MS76}, \cite{Sto77}, \cite{Gor79}, \cite{Sch86}, and more recently in static program analysis 
\cite{Myc81}, \cite{Nie84}, \cite{AH87}.
In the course of this development, a number of new perspectives have emerged.
\subsection*{Syntax vs. Semantics}
Domain theory was originally presented as a model theory for computation,
and this aspect was emphasised in \cite{Sco70,Sco80b}.
However, the effective character of domain constructions was immediately 
evident, and made fully explicit in \cite{EC76,Sco76,Smy77,Kan79}.
Moreover, in recent presentations of domains via neighbourhood systems and information systems \cite{Sco81,Sco82}, Scott has shown how the theory can be based on elementary, and finitary, set-theoretic representations, which in the case of information systems are deliberately suggestive of proof theory.

A further step towards explicitly syntactic presentations of domain theory was taken by Martin-L\"{o}f, in his Domain Interpretation of Intuitionistic Type Theory \cite{M-L83}.
His formulation also traces a line of descent from Kreisel's definition of the continuous functionals \cite{Kre59}, via \cite{M-L70,Ers72}.

The general tendency of these developments is to suggest that domains may as well be viewed in terms of {\em theories} as of {\em models}.
Our work should not only confirm this suggestion, but also show how it may be put to use.
\subsection*{Points vs. Properties}
An important recent development in mathematics has been the rise of {\it locale theory}, or ``topology without points'' \cite{Joh82},
in which the open-set lattices rather than the spaces of points become the primary objects of study.
That these mathematical developments have direct bearing on Computer Science was emphasised by Smyth in \cite{Smy83}.
If we think of the open sets as {\em properties} or {\em propositions},
we can think of spaces as logical {\em theories}; continuous maps act on these theories under inverse image as {\em predicate transformers} in the sense of Dijkstra \cite{Dij76}, or modal operators as studied in {\em dynamic logic} \cite{Pra79,Har79}.

There is also an important theme in Computer Science which emerges as confluent with these mathematical developments; namely, the use of notions of {\em observation} and {\em experiment} as a basis for the behavioural semantics of systems.
This plays a major role in the work of Milner, Hennessy {\it et al.} on concurrent systems \cite{Mil80,HM85,Win80}, and also in the theory of higher-order functional languages, e.g. \cite{Plo77,Mil77,BC85,BCL85}.
The leading idea here is to take some notion of {\em observable event} or {\em experiment} as an ``information quantum'', and to construct the meaning of a system out of its information quanta.
This corresponds to the leading idea of locale theory, that ``points'' are nothing but constructions out of properties.
By exploiting this correspondence, we may hope to obtain a {\it rapprochement} between domain theory and denotational semantics, on the one hand, and operationally formulated notions such as {\em observation equivalence} \cite{HM85} on the other.
\subsection*{Denotational vs. Axiomatic}
Another area in programming language theory which has received intensive 
development over the past 15 years has been {\em logics of programs}, 
e.g. Hoare logic \cite{Hoa69,deB80}, dynamic logic \cite{Pra79,Har79}, temporal logic \cite{Pnu77}, etc.
However, to date there has not been a satisfactory integration of this work with domain theory.
For example, dynamic logic deals with sets and relations, which from the perspective of domain theory corresponds only to an extremely naive and restricted fragment of programming language semantics.
One would like to see a dynamic logic of {\em domains} and {\em continuous functions}, which would encompass higher-order functions, quasi-infinite (or ``lazy'') data structures, self-application, non-determinism, and all the other computational phenomena for which domain theory provides a mathematical foundation.

The key mathematical idea which forms the basis of our attempt to draw all these diverse strands together is {\em Stone Duality}, which we now briefly review; a fuller discussion will be found in Chapter 2.
\section{Overview: Stone Duality}
The classic Stone Representation Theorem for Boolean algebras \cite{Sto36} 
is aimed at solving the following problem: 
\begin{quote}
show that every (abstract) Boolean algebra can be represented as a field of 
sets, in which the operations of meet, join and complement are represented by 
intersection, union and set complement.
\end{quote}

Stone's solution to the problem begins with observation that for any topological space $X$, the lattice ${\sf Clop} \; X$ of clopen subsets of $X$ forms a field of sets.
His radical step was to construct, from any Boolean lagebra $B$, a topological space ${\sf Spec} \; B$.
To understand the construction, think of $B$ as (the Lindenbaum algebra of) a classical propositional theory.
The elements of $B$ are thus to be thought of as (equivalence classes of) formulae, and the operations as logical conjunction, disjunction and negation.
Now a {\em model} of $B$ is an assignment of ``truth-values'' 0 or 1 to elements of $B$, in a manner consistent with the logical structure; e.g. so that $\neg b$ is assigned 1 if and only if $b$ is assigned 0.
In short, a model is a Boolean algebra homomorphism $f : B \rightarrow {\bf 2}$, where ${\bf 2} = \{ 0, 1 \}$ is the two-element lattice.
Identifying such an $f$ with $f^{-1}(1) \subseteq B$, which as is well-known is an {\em ultrafilter} over $B$ (see e.g. \cite{Joh82}), we can take ${\sf Spec} \; B$ as the set of ultrafilters over $B$, with the topology generated by
\[ U_{a} \equiv \{ x \in {\sf Spec} \; B : a \in x \} \;\;\; ( a \in B) . \]
The spaces arising as ${\sf Spec} \; B$ for Boolean algebras $B$ in this way were characterised by Stone as the totally disconnected compact Hausdorff spaces (subsequently named {\it Stone spaces} in his honour).
Moreover, we have the isomorphisms
\begin{equation}
B \cong {\sf Clop} \; {\sf Spec} \; B 
\end{equation}
\[ b \mapsto \{ x \in {\sf Spec} \; B : b \in x \} \]
\begin{equation}
S \cong {\sf Spec} \; {\sf Clop} \; S
\end{equation}
\[ s \mapsto \{ U \in {\sf Clop} \; S : s \in U \} . \]
The first of these isomorphisms solves the representation problem, and comprises Stone's Theorem in its classical form.
But we can go further; these correspondences also extend (contravariantly) to morphisms:
\[ \frac{S \stackrel{f}{\longrightarrow} T}{{\sf Clop} \; S 
\stackrel{f^{-1}}{\longleftarrow} {\sf Clop} \; T} \;\;\;\;\;\;
\frac{A \stackrel{h^{\star}}{\longleftarrow} B}{{\sf Spec} \; A \stackrel{h}{\longrightarrow} {\sf Spec} \; B} \]
where
\[ h : x \mapsto \{ b \in B : h^{\star} b \in x \} . \]
In modern terminology, this yields a {\em duality} (= contravariant equivalence of categories):
\[ {\bf Stone} \simeq {\bf Bool}^{\sf op} . \]
This is the prototype for a whole family of ``Stone-type duality theorems'', and leads to locale theory, as ``pointless topology'' or junior-grade (propositional) topos theory.
(An excellent reference for these topics is \cite{Joh82}).

But what has all this to do with Computer Science?
Two interpretations of Stone duality can be found in the existing literature from mathematics and logic:
\begin{itemize}
\item The topological view: Points vs. Open sets.
\item The logical view: Models vs. Formulas.
\end{itemize}
We wish to add a third interpretation:
\begin{itemize}
\item The Computer Science view: (Denotations of) computational processes vs. (extensions of) specifications.
\end{itemize}
The importance of Stone duality for Computer Science is that {\em it provides the right framework for understanding the relationship between denotational semantics and program logic}.
The fundamental logical relationship of program development is
\[ P \models \phi \]
to be read ``$P$ satisfies $\phi$'', where $P$ is a program (a syntactic description of a computational process), and $\phi$ is a formula (a syntactic description of a property of computations).
Thus $P$ is the ``how'' and $\phi$ the ``what'' in the dichotomy standardly used to explain the distinction between programs and specifications.
We can easily describe the main formal activities of the program development process in terms of this relation:
\begin{itemize}
\item {\em Program specification} is the task of defining (a list of) properties $\phi$ to be satisfied by the program.
\item {\em Program synthesis} is the task of finding $P$ given (a list of) $\phi$.
\item {\em Program verification} is the task of proving that $P \models \phi$.
\end{itemize}
The two sides of Stone duality---the spatial and the logical or localic---yield alternative but equivalent perspectives on this fundamental relationship:
\begin{itemize}
\item The spatial side of the duality, where points are taken as primary, 
properties are constructed as (open) sets of points, and the fundamental 
relationship is interpreted as $s \in U$ ($s$ a point, $U$ a property), 
corresponds to {\em denotational semantics}, where the data domains 
(i.e. the {\em types}) of a programming language are interpreted as spaces 
of points, and programs are given denotations as points in these spaces; 
this denotational perspective yields a topological interpretation of program logic.
\item The logical or localic side of the duality, where properties,  
as elements of an abstract (logical) lattice, are taken as primary, 
and points are constructed as sets (prime filters) of properties, 
with the fundamental relationship interpreted as $a \in x$ ($a$ a property, 
$x$ a point), corresponds to program logic, and yields a {\em logical interpretation of denotational semantics}.
The idea is that the structure of the open-set lattices and prime filters are presented {\em syntactically}, via axioms and inference rules, as a formal system.
\end{itemize}
We extract the following concrete research programme from these general perspectives on Stone duality:
\begin{enumerate}
\item A metalanguage is introduced, comprising
\begin{itemize}
\item types = data domains = universes of discourse for various computational situations.
\item terms = programs = syntactic intensions for models or points.
\end{itemize}
\item A standard denotational interpretation of the metalanguage, assigning domains to types and domain elements to terms, can be given using the spatial side of Stone duality.
\item The metalanguage is also given a {\em logical} interpretation, in which the localic side of the duality is presented as a formal system with axioms and inference rules.
Each type is interpreted as a propositional theory; and terms are interpreted by axiomatising the satisfaction relation $P \models \phi$.
This gives a program logic.
\item The denotational semantics from~2 and the program logic from~3 are related by showing that they are Stone duals of each other---a strengthened form of the logician's ``Soundness and Completeness''.
As a consequence of this, semantics and logic are guaranteed to be in harmony with each other, and in fact each determines the other up to isomorphism.
\item The framework developed in~1--4 is very {\em general}.
The metalanguage can be used to describe a wide variety of computational situations, following the ideas of ``classical'' denotational semantics.
Given such a description, we can turn the handle to obtain a logic for that situation.
This offers two exciting prospects: of replacing {\it ad hoc} ingenuity in the design of program logics to match a given semantics by the routine application of systematic general theory; and of bringing hitherto divergent fields of programming language theory (e.g. $\lambda$-calculus and concurrency) within the scope of a single unified framework.
\end{enumerate}
The main objective of this thesis is to elaborate the programme outlined in~1--5.
Chapter~2 is devoted to filling in some background on domains and locales.
Then Chapters~3 and~4 are concerned with~1--4 above.
Chapters~5 and~6 each develop a major case study along the lines suggested by~5, in the areas of concurrency and $\lambda$-calculus respectively.
Finally, Chapter~7  discusses directions for further research.
