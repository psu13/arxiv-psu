\section{A Cpo of Pre-locales}
In this section, we follow the ideas of Larsen and Winskel \cite{LW84}, and 
define a (large) cpo of domain pre-locales, in such a way that type constructions can be represented as continuous functions over this cpo, and the process of solving recursive domain equations reduced to taking least fixed points of such functions.

\begin{definition} 
{\rm Let $A$, $B$ be domain prelocales. Then we define $A \Subset B$ iff
\begin{itemize}
\item $|A| \subseteq |B|$
\item $(|A|, 0_{A}, \vee_{A}, 1_{A}, \wedge_{A})$ is a subalgebra of $(|B|, 0_{B}, \vee_{B}, 1_{B}, \wedge_{B})$
\item $\leq_{A} \; \subseteq \; \leq_{B}$
\end{itemize}}
\end{definition}
Although this inclusion relation is simple, it is too weak, and has only been introduced for organisational purposes. What we need is
\begin{definition} 
{\rm $A \trianglelefteq B$ iff}
\[(s1) \;\;\; A \Subset B \]
\[(s2) \;\;\; \forall a, b \in |A|. \, a \leq_{B} b \; \Rightarrow a \leq_{A} b \]
\[(s3) \;\;\; pr(A) \subseteq pr(B) \]
\end{definition}
Note that apart from $(s3)$ this is just the usual notion of {\it submodel} 
(cf. e.g. \cite{CK73}).
\begin{proposition}
The class of domain prelocales under $\trianglelefteq$ is an $\omega$-chain complete partial order.
\end{proposition}

\proof\ The verification that $\trianglelefteq$ is a partial order is routine. Let $\{A_{n}\}$ be a $\trianglelefteq$-chain. Set
\[A_{\infty} \equiv (\bigcup_{n \in \omega}A_{n}, \bigcup_{n \in \omega}\leq_{A_{n}}, \ldots etc.). \]
We check that $A_{\infty}$ is a well-defined domain prelocale, for in that case it is clearly the least upper bound of the chain. We verify $(d1)$ for illustration.

Given $a \in |A_{\infty}|$, for some $n$, $a \in |A_{n}|$,  hence
\[a =_{A_{n}} \bigvee_{i \in I}a_{i}, \;\; (a_{i} \in pr(A_{n}), i \in I).\]
Clearly $a =_{A_{\infty}} \bigvee_{i \in I}a_{i}$; furthermore, $pr(A_{n}) \subseteq pr(A_{\infty})$. To see this, suppose $b \in pr(A_{n})$ and $b \leq_{A_{\infty}} c \vee d$. For some $m \geq n$, $\{a, b, c\} \subseteq |A_{m}|$, and so $b \leq_{A_{m}} c \vee d$. Since $A_{n} \trianglelefteq A_{m}$, $pr(A_{n}) \subseteq pr(A_{m})$, and so $b \leq_{A_{m}} c$ or $b \leq_{A_{m}} d$, which implies $b \leq_{A_{\infty}} c$ or $b \leq_{A_{\infty}} d$, as required. \qed

The class of domain prelocales is not a cpo under $\trianglelefteq$; it does not have a least element. However, we can easily remedy this deficiency.
\begin{definition} 
{\rm {\bf 1} is the domain prelocale defined as follows. 
The carrier $|{\bf 1}|$ is defined inductively by
\begin{itemize}
\item $\true , \false \in |{\bf 1}|$
\item $a, b \in |{\bf 1}| \; \Rightarrow \; a \wedge b, a \vee b \in |{\bf 1}|$
\end{itemize}
The operations are defined ``freely'' in the obvious way:
\[0_{{\bf 1}} \equiv \false , \;\; 1_{{\bf 1}} \equiv \true , \;\; a \vee_{{\bf 1}} b \equiv a \vee b , \;\; a \wedge_{{\bf 1}} b \equiv a \wedge b \]
Finally, $\leq_{{\bf 1}}$, $=_{{\bf 1}}$ are defined inductively as the least relations satisfying $(p1)$--$(p4)$.
It is easy to see that $\tilde{{\bf 1}}$ is the two-point lattice; hence {\bf 1} is a domain prelocale.}
\end{definition}
Now let {\bf DPL1} be the class of domain prelocales $A$ such that ${\bf 1} \trianglelefteq A$. Clearly {\bf DPL1} is still chain-complete. Thus we have
\begin{proposition}
{\bf DPL1} is a large cpo with least element {\bf 1}. \qed
\end{proposition}
{\bf DPL1} also determines a full subcategory of {\bf DPL}. To see that we are not losing anything in passing from {\bf DPL} to {\bf DPL1}, we note
\begin{proposition}
{\bf DPL1} is equivalent to {\bf DPL}. \qed
\end{proposition}

We now relate this partial order of prelocales to the category of domains and 
embeddings used in the standard category-theoretic treatment of the solution 
of domain equations \cite{SP82}. 
Recall that an {\it embedding-projection pair} between domains $D$, $E$ 
is a pair of continuous functions $e : D \rightarrow E$, $p : E \rightarrow D$ satisfying
\[p \circ e = {\sf id}_{D} \]
\[ e \circ p \sqsubseteq {\sf id}_{E}. \]
Each of these functions uniquely determines the other, since $e$ is left adjoint to $p$. We write $e^{R}$ for the projection determined by $e$.
\begin{proposition}
If $A \trianglelefteq B$, then $e : \hat{A} \rightarrow \hat{B}$ is an 
embedding, where 
\[ e : x \mapsto \diverges_{B}(x). \]
($\hat{A}$, $\hat{B}$ are defined as in the proof of Theorem~\ref{domtheq}).
\end{proposition}

\proof\ We define $p : \hat{B} \rightarrow \hat{A}$ by
\[ p(y) = y \cap |A|. \]
Since $A$ is a sublattice of $B$, $p$ is well defined and continuous (it is the surjection corresponding under Stone duality to the inclusion of $A$ in $B$). We check that $e$ is well defined, specifically that $e(x)$ is prime, $x \in \hat{A}$. Suppose $b \vee c \in e(x)$. Then for some $a \in x$, $a \leq_{B} b \vee c$. By $(d1)$,
\[a =_{A} \bigvee_{i \in I}a_{i}, \;\; (a_{i} \in pr(A), i \in I). \]
Since $x$ is a prime proper filter, $a_{i} \in x$ for some $i \in I$. Since $A \trianglelefteq B$, $a_{i} \in pr(B)$, and so
\begin{eqnarray*}
a_{i} \leq_{B} a \leq_{B} b \vee c & \Rightarrow & a_{i} \leq_{B} b \;
{\rm or} \; a_{i} \leq_{B} c \\
& \Rightarrow & b \in e(x) \; {\rm or} \; c \in e(x).
\end{eqnarray*} 
Moreover,
\[p \circ e(x) = \diverges_{B}(x) \cap |A| = x \]
\[e \circ p(y) = \diverges_{B}(y \cap |A|) \subseteq \diverges_{B}(y) = y. \]
Finally, $e$ preserves all joins since it is a left adjoint; in particular, it is continuous. \qed

Now given a (unary) type construction $T$, we will seek to represent it as a function
\[f_{T} : {\bf DPL1} \rightarrow {\bf DPL1} \]
which is $\trianglelefteq$-monotonic and chain continuous. We can then construct the initial solution of the domain equation
\[D = T(D) \]
as the least fixpoint of the function $f_{T}$, given in the usual way as
\[\bigsqcup_{n \in \omega}f_{T}^{(n)}({\bf 1}). \]

More generally, we can consider systems of domain equations by using powers of {\bf DPL1}; while $T$ can be built up by composition from various primitive operations. As long as each basic type construction is $\trianglelefteq$-monotonic and continuous, this approach will work.

The task of verifying continuity is eased by the following observation, 
adapted from \cite{LW84}.
\begin{proposition}
Suppose $f : {\bf DPL1} \rightarrow {\bf DPL1}$ is $\trianglelefteq$-monotonic and continuous on carriers, i.e. given a chain
$\{A_{n}\}_{n \in \omega}$,
\[ |f(\bigsqcup_{n \in \omega}A_{n}| = \bigcup_{n \in \omega}|f(A_{n})|,\]
then $f$ is continuous.
\end{proposition}

\proof\ Firstly, note that $A \trianglelefteq B$ and $|A| = |B|$ implies $A = B$. Now given a chain $\{A_{n}\}$, let
\[B \equiv \bigsqcup_{n}f(A_{n}), \]
\[C \equiv f(\bigsqcup_{n}A_{n}). \]
By monotonicity of $f$, $B \trianglelefteq C$, while by continuity on carriers, $|B| = |C|$. Hence $B = C$, and $f$ is continuous. \qed
