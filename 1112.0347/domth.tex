\chapter{Domains and Theories}
\section{Introduction}
In this Chapter, we lay some of the foundations for the domain 
logic to be presented in Chapter 4. 
In section 2, a category of domain prelocales (coherent propositional theories) and approximable mappings is defined, 
and proved equivalent to {\bf SDom}. 
This is the category in which, implicitly, all the work of Chapter 4 is set. 
In section 3, following the ideas of a number of authors, particularly 
Larsen and Winskel in \cite{LW84}, 
a large cpo of domain prelocales is defined, and used to reduce the solution 
of domain equations to taking least fixpoints of continuous functions over this cpo. 
In section 4, a number of type constructions are defined as operations over domain prelocales. 
We prove in detail that these operations are naturally isomorphic to the corresponding constructions on domains. 
In section 5 a semantics for a language of recursive type expressions is given, in which each type is interpreted as a logical theory. 
This is related to a standard semantics in which types denote domains by 
showing that for each type its interpretation in the logical semantics is 
the Stone dual of its denotation in the standard semantics.

{\bf Important Notational Convention.}
Throughout this Chapter and the next, we shall use $I$, $J$, $K$, $L$
to range over {\em finite} index sets.
\section{A Category of Pre-Locales}
\begin{definition} 
{\rm A {\em coherent prelocale} is a structure
\[A = (|A|, \leq_{A}, =_{A}, 0_{A}, \vee_{A}, 1_{A}, \wedge_{A}) \]
where \begin{itemize}
\item $|A|$ is a set, the {\it carrier}
\item $\leq_{A}$, $=_{A}$ are binary relations over $|A|$
\item $0_{A}$, $1_{A}$ are constants, i.e. elements of $|A|$
\item $\vee_{A}$, $\wedge_{A}$ are binary operations over $|A|$
\end{itemize}
subject to the following axioms (subscripts omitted):}
\[(p1) \;\;\; a \leq a \;\;\; \frac{a \leq b \;\; b \leq c}{a \leq c} \;\;\;\;     \frac{a \leq b \;\; b \leq a}{a = b} \;\;\;\; \frac{a = b}{a \leq b \;\; b \leq a} \]
\[(p2) \;\;\; 0 \leq a \;\;\;\; \frac{a \leq c \;\; b \leq c}{a \vee b \leq c} \;\;\;\; a \leq a \vee b \;\;\;\; b \leq a \vee b \]
\[(p3) \;\;\; a \leq 1 \;\;\;\; \frac{a \leq b \;\; a \leq c}{a \leq b \wedge c} \;\;\;\; a \wedge b \leq a \;\;\;\; a \wedge b \leq b \]
\[(p4) \;\;\; a \wedge (b \vee c) \leq (a \wedge b) \vee (a \wedge c) \]
\end{definition}
Evidently, the quotient structure
\[ \tilde{A} = (|A|/{=_{A}}, {\leq} /{=_{A}}) \]
is a distributive lattice.
\begin{definition} 
{\rm Given a prelocale A, we define}
\[\begin{array}{rrcl}
(i) & pr(A) & \equiv & \{a \in |A| : \forall b, c \in |A|.\, a \leq b \vee c \Rightarrow a \leq b \; {\rm or} \; a \leq c \} \\
(ii) & con(A) & \equiv & \{a \in |A| : \neg (a =_{A} 0) \} \\
(iii) & cpr(A) & \equiv & con(A) \cap pr(A) \\
(iv) & t(A) & \equiv & \{a \in |A| : \neg (a =_{A} 1) \}
\end{array} \]
\end{definition}
\begin{definition} 
{\rm A {\it domain prelocale} is a coherent prelocale $A$ which satisfies 
the following additional axioms:}
\[ (d1) \;\;\; \forall a \in |A| . \, \exists b_{1}, \ldots b_{n} \in pr(A). \, a =_{A} \bigvee_{i=1}^{n}b_{i} \]
\[ (d2) \;\;\; 1_{A} \in cpr(A) \]
\[ (d3) \;\;\; a, b \in pr(A) \; \Rightarrow \; a \wedge b \in pr(A) \]
\end{definition}
We now introduce a notion of morphism for domain prelocales, based on Scott's {\it approximable mappings} \cite{Sco81,Sco82}.
\begin{definition} 
{\rm Let $A$, $B$, be domain prelocales. An {\it approximable mapping} $R : A \rightarrow B$ is a relation $R \subseteq |A| \times |B|$ satisfying}
\[ (r1) \;\;\; a R 1 \]
\[ (r2) \;\;\; a R b \: \& \: a R c \; \Rightarrow \; a R (b \wedge c) \]
\[ (r3) \;\;\; 0 R b\]
\[ (r4) \;\;\; a R c \: \& \: b R c \; \Rightarrow \; (a \vee b) R c \]
\[ (r5) \;\;\; a \leq a' R b' \leq c \; \Rightarrow \; a R b \]
\[ (r6) \;\;\; a R 0 \; \Rightarrow \; a =_{A} 0 \]
\[ (r7) \;\;\; a \in pr(A) \: \& \: a R (b \vee c) \; \Rightarrow \; a R b \: 
\; {\rm or} \; \: a R c. \]
\end{definition}

Approximable mappimgs are closed under relational composition. 
We verify the least trivial closure condition, $(r7)$. 
Suppose $R : A \rightarrow B$, $S : B \rightarrow C$, $a \in pr(A)$ and 
$a (R \circ S) b \vee c$. 
For some $d \in |B|$, $a R d$ and $d S b \vee c$. By $(d1)$,
\[ d =_{B} \bigvee_{i \in I}d_{i} \;\;(d_{i} \in pr(B), i \in I). \]
If $I = \varnothing$, $d =_{B} 0_{B}$, hence by $(r3)$ $d R b$, and so $a (R \circ S) b$. 
Otherwise, by $(r7)$, $a R d_{i}$ for some $i \in I$. Now
\[ d_{i} \leq \bigvee_{i \in I}d_{i} S (b \vee c) \]
\[ \Rightarrow \;\; d_{i} S (b \vee c) \;\;\; (r5) \]
\[ \Rightarrow \;\; d_{i} S b \; {\rm or} \; d_{i} S c \;\;\; (r7) \]
\[ \Rightarrow \;\; a (R \circ S) b \; {\rm or} \; a (R \circ S) c\]
as required.
Identities with respect to this composition are given by
\[ a \: {\rm id}_{A} \: b \; \equiv \; a \leq_{A} b. \]
Hence we can define a category {\bf DPL} of domain prelocales and approximable mappings.

\begin{definition} 
{\rm A {\em pre-isomorphism} $\varphi : A \simeq B$ of domain prelocales is a surjective function
\[ \varphi : |A| \rightarrow |B| \]
satisfying}
\[ \forall a, b \in |A|. \, a \leq_{A} b \; \Leftrightarrow \; \varphi (a) \leq_{B} \varphi (b). \]
\end{definition}
\begin{proposition}
If $\varphi : A \simeq B$ is a preisomorphism, the relation
\[ a R_{\varphi} b \; \equiv \; \varphi (a) \leq_{B} b \]
is an isomorphism in {\bf DPL}. \qed
\end{proposition}
\begin{theorem} \label{domtheq}
{\bf DPL} is equivalent to {\bf SDom}.
\end{theorem}

\proof\ We define functors
\[ F : {\bf SDom} \rightarrow {\bf DPL} \]
\[ G : {\bf DPL} \rightarrow {\bf SDom} \]
as follows:
\[ F(D) = (K \Omega (D), \subseteq , =, \varnothing , \cup , D, \cap ) \]
i.e. the distributive lattice of compact-open subsets of $D$;
\[ F(f) = R_{f}, \]
where
\[ a R_{f} b \; \equiv \; a \subseteq f^{- 1}(b). \]
The verification that F is well-defined is routine. Note that:
\[ \bullet \;\; pr(F(D)) = \{\diverges  u : u \in K(D) \} \cup \{\varnothing \} \]
\[ \bullet \;\; a \in con(F(D)) \; \Leftrightarrow \; a \not= \varnothing \]
\[ \bullet \;\; {\diverges  u} \cap {\diverges  v} \in con(F(D)) \; \Leftrightarrow \; u \consistent\ v \]
To verify $(r7)$ for $R_{f}$, note that, for $u \in K(D)$:
\begin{eqnarray*}
\diverges  u \subseteq f^{-1}(b \cup c)  & \Leftrightarrow &  u \in f^{-1}(b \cup c) \\
                     & \Leftrightarrow & f(u) \in b \cup c \\
                     & \Leftrightarrow & f(u) \in b \; {\rm or} \; f(u) \in c \\
                     & \Leftrightarrow & \diverges  u \subseteq  f^{-1}(b) 
\; {\rm or} \; \diverges  u \subseteq f^{-1}(c). 
\end{eqnarray*} 

\[ G(A) \equiv \hat{A}, \]
where $\hat{A}$ is the set of prime proper filters of $A$, i.e. sets $x \subseteq |A| - \{0_{A}\}$ closed under finite conjunction and entailment and satisfying
\[ a \vee b \in x \; \Rightarrow \; a \in x \; {\rm or} \; b \in x. \]
$\hat{A}$ is a partial order under set inclusion; or, equivalently, (via the specialisation order) a topological space with basic opens
\[ U_{a} \equiv \{x \in \hat{A} : a \in x \} \;\; (a \in |A|). \]
Note that, with either structure,
\[ \hat{A} \; \cong \; {\rm Spec} \: \tilde{A}. \]

\[ G(R) = f_{R}, \]
where
\[ f_{R}(x) = \{b \: | \: \exists a \in x. \, a R b \}. \]
We check that $G$ is well defined. By $(d2)$, the filter generated by 1
is prime, hence a least element for $\hat{A}$; while it is easy to see that $\hat{A}$ is closed under unions of directed families. 
Thus $\hat{A}$ is a cpo. 
Moreover, the principal filters $\diverges  (a)$ with $a \in cpr(A)$ are prime, and (using $(d1)$) form a basis of finite elements.
Finally, by $(d3)$ this basis is closed under consistent finite  joins. Thus $\hat{A}$ is a Scott domain.

Now we check that $f_{R}$ is well defined and continuous. Given $x \in \hat{A}$, it is easy to see that $f_{R}(x)$ is a filter. To check that it is prime, suppose $b \vee c \in f_{R}(x)$. 
Then for some $a \in x$, we must have $a R (b \vee c)$. 
By $(d1)$,
\[ a =_{A} \bigvee_{i \in I}a_{i}, \;\;\;(a_{i} \in cpr(A), i \in I). \]
Since $x$ is a proper filter, $a \not= 0$, hence $I \not= \varnothing$. 
Then since $x$ is prime, for some $i \in I$ $a_{i} \in x$. Now by $(r7)$,
\[ a_{i} R (b \vee c) \; \Rightarrow \; a_{i} R b \; {\rm or} \; a_{i} R c \]
and so $b \in f_{R}(x)$ or $c \in f_{R}(x)$.
Since directed joins in $\hat{A}$ are just unions, continuity of $f_{R}$ is trivial.

The remainder of the verification that G is a functor is routine.

We now define natural transformations
\[ \eta : I_{{\bf SDom}} \rightarrow GF \]
\[ \epsilon : I_{{\bf DPL}} \rightarrow FG \]

\[ \eta D(d) = \{U \in K\Omega (D) : d \in U\} \]
\[ \epsilon A = R_{\varphi A}, \]
where $\varphi A : A \simeq K\Omega (\hat{A})$ is the pre-isomorphism defined by
\[ \varphi A(a) = \{x \in \hat{A} : a \in x\}. \]
Note that $\eta$, $\varphi$ are the natural isomorphisms in the Stone duality for distributive lattices. This shows that the components of $\eta$, $\epsilon$ are isomorphisms, while naturality is easily checked to extend to our setting.

Altogether, we have shown that 
\[ (F, G, \eta , \epsilon ) : {\bf SDom} \simeq {\bf DPL} \]
is an equivalence of categories. \qed
