\section{Programs as Morphisms: Exogenous Logic}
We now introduce a second extension of our denotational meta-language,
which provides a syntax of terms denoting {\em morphisms between},
rather than elements of, domains.
This is an extended version of the algebraic meta-language for cartesian closed
categories \cite{Poi86,LS86}, just as the language of the previous section was an extended typed
$\lambda$-calculus.
Terms are sorted on {\em morphism types} $(\sigma , \tau )$, with notation
$f : (\sigma , \tau )$.
We shall give the formation rules in ``polymorphic'' style, with type
subscripts omitted.
\subsection*{Syntax of morphism terms}
\[ \indic {\sf id} : (\sigma , \sigma ) \;\;\;\;\;\; 
\indic \frac{f : (\sigma , \tau ) \;\;\; g : (\tau , \upsilon )}{f ; g : (\sigma , \upsilon )} \]
\[ \indic 1 : (\sigma , {\bf 1}) \]
\[ \indic \frac{f : (\upsilon , \sigma ) \;\;\; g : (\upsilon , \tau )}{\ltuple f, g \rtuple : (\upsilon , \sigma \times \tau )} \;\;\;\;\;\;
   \indic {\sf p} : (\sigma \times \tau , \sigma ) \;\;\;\;\;\;
\indic {\sf q} : (\sigma \times \tau , \tau ) \]
\[ \indic \frac{f : (\sigma \times \tau , \upsilon )}{\curry{f} : (\sigma , \tau \rightarrow \upsilon )} \;\;\;\;\;\;
\indic {\sf Ap} : ((\sigma \rightarrow \tau ) \times \sigma , \tau ) \]
\[ \indic {\sf l} : (\sigma , \sigma \oplus \tau ) \;\;\;\;\;\;
   \indic {\sf r} : (\tau , \sigma \oplus \tau ) \;\;\;\;\;\;
   \indic \frac{f : (\sigma , \upsilon ) \;\;\; g : \tau , \upsilon )}{\stup{f}{g} : (\sigma \oplus \tau , \upsilon )} \]
\[ \indic {\sf up} : (\sigma , (\sigma )_{\bot}) \;\;\;\;\;\;
 \indic \frac{f : (\sigma , \tau )}{{\sf lift}(f) : ((\sigma)_{\bot}, \tau )} \;\;\;\;\;\;
\indic \frac{f : (\sigma , \tau )}{{\sf strict}(f) : (\sigma , \tau )} \]
\[ \indic \lsing \cdot \rsing_{l} : (\sigma , P_{l} \sigma ) \;\;\;\;\;\;
    \indic \lsing \cdot \rsing_{u} : (\sigma , P_{u} \sigma ) \]
\[ \indic \frac{f : (\sigma , P_{l} \tau )}{f^{\dagger}_{l} : (P_{l} \sigma , P_{l} \tau )} \;\;\;\;\;\;
   \indic \frac{f : (\sigma , P_{u} \tau )}{f^{\dagger}_{u} : (P_{u} \sigma , P_{u} \tau )} \]
\[ \indic +_{l} : (P_{l} \sigma \times P_{l} \sigma , P_{l} \sigma ) \;\;\;\;\;\;
   \indic +_{u} : (P_{u} \sigma \times P_{u} \sigma , P_{u} \sigma ) \]
\[ \indic \otimes_{l} : (P_{l} \sigma \times P_{l} \tau , P_{l} (\sigma \times \tau )) \;\;\;\;\;\;
   \indic \otimes_{u} : (P_{u} \sigma \times P_{u} \tau , P_{u} (\sigma \times \tau )) \]
\[ \indic {\sf fold} : (\sigma [{\sf rec} \: t. \, \sigma /t], {\sf rec} \: t. \, \sigma ) \;\;\;\;\;\;
\indic {\sf unfold} : ( {\sf rec} \: t. \, \sigma , \sigma [{\sf rec} \: t. \, \sigma /t]) \]
\[ \indic {\sf Y} : (\sigma \rightarrow \sigma , \sigma ) \]
We now form an exogenous logic $\DDL$ (for {\em dynamic domain logic}, because of the evident analogy with dynamic logic \cite{Pra79,Har79}).
$\DDL$ is an extension of ${\cal L}$, the basic domain logic described in Section~2.
\subsection*{Formation Rules}
We define the set of formulas ${\rm DDL}(\sigma )$ for each type $\sigma$.
\[ \indic L(\sigma ) \subseteq {\rm DDL}(\sigma ) \;\;\;\;\;\;
\indic \frac{f : (\sigma , \tau ) \;\;\; \psi \in {\rm DDL}(\tau )}{[f]\psi \in {\rm DDL}(\sigma )} \]
\[ \indic \true , \false \in {\rm DDL}(\sigma ) \;\;\;\;\;\;
\indic \frac{\phi , \psi \in {\rm DDL}(\sigma )}{\phi \wedge \psi , \phi \vee \psi \in {\rm DDL}(\sigma )} \]
\subsection*{Axiomatization}
The following axioms and rules are added to those of $\cal L$.
\[ \indic \frac{\phi \leq \psi}{[f]\phi \leq [f]\psi} \;\;\;\;\;\;
   \indic [f]\bigwedge_{i \in I} \phi_{i} = \bigwedge_{i \in I} [f] \phi_{i} \;\;\;\;\;\;
   \indic [f]\bigvee_{i \in I} \phi_{i} = \bigvee_{i \in I} [f] \phi_{i} \]
\[ \indic [{\sf id}]\phi = \phi \;\;\;\;\;\; \indic [f ; g] \phi = [f][g]\phi \]
\[ \indic [\ltuple f, g \rtuple ] (\phi \times \psi ) = [f]\phi \wedge [g]\psi \]
\[ \indic [{\sf p}]\phi = (\phi \times \true ) \;\;\;\;\;\;
\indic [{\sf q}]\psi = (\true \times \psi ) \]
\[ \indic \frac{(\phi \times \psi ) \leq [f]\theta}{\phi \leq [\curry{f}](\psi \rightarrow \theta )} \;\;\;\;\;\;
\indic (\phi \rightarrow \psi ) \times \phi \leq [{\sf Ap}]\psi \]
\[ \indic [{\sf l}](\phi \oplus \false ) = \phi \;\;\;\;\;\;
\indic [{\sf l}](\false \oplus \psi ) = \false \;\; (\psi \converges ) \]
\[ \indic [{\sf r}](\phi \oplus \false ) = \false \;\; (\phi \converges ) \;\;\;\;\;\;
\indic [{\sf r}](\false \oplus \psi ) = \psi \]
\[ \indic [\stup{f}{g}]\phi = ([{\sf strict}(f)]\phi \oplus \false ) \vee (\false \oplus [{\sf strict}(g)]\phi ) \]
\[ \indic \frac{\phi \leq [f]\psi}{\phi \leq [{\sf strict}(f)]\psi} \;\; (\phi \converges ) \]
\[ \indic [{\sf up}](\phi )_{\bot} = \phi \;\;\;\;\;\;
\indic [{\sf lift}(f)]\phi = ([f]\phi )_{\bot} \;\; (\phi \converges ) \]
\[ \indic [\lsing \cdot \rsing_{l}] \Diamond \phi = \phi \;\;\;\;\;\;
\indic [\lsing \cdot \rsing_{u}] \Box \phi = \phi \]
\[ \indic \frac{\phi \leq [f]\Diamond \psi}{\Diamond \phi \leq [f^{\dagger}_{l}]\Diamond \psi} \;\;\;\;\;\;
\indic \frac{\phi \leq [f]\Box \psi}{\Box \phi \leq [f^{\dagger}_{u}]\Box \psi} \]
\[ \indic [+_{l}]\Diamond \phi = (\Diamond \phi \times \true ) \vee (\true \times \Diamond \phi ) \;\;\;\;\;\;
\indic [+_{u}]\Box \phi = (\Box \phi \times \Box \phi )  \]
\[ \indic [\otimes_{l}]\Diamond (\phi \times \psi ) = (\Diamond \phi \times \Diamond \psi ) \;\;\;\;\;\;
\indic [\otimes_{u}]\Box (\phi \times \psi ) = (\Box \phi \times \Box \psi ) \]
\[ \indic [{\sf fold}]\phi = \phi \;\;\;\;\;\; 
\indic [{\sf unfold}]\phi = \phi \;\;\;\;\;\; 
\indic \frac{\phi \leq [{\sf Y}]\psi}{\phi \wedge (\psi \rightarrow \theta ) \leq [{\sf Y}]\theta} \]
At this point, we could proceed to give a direct treatment of the semantics and meta-theory of $\DDL$, just as we did for the endogenous logic in Section~3.
This would ignore the salient fact that  our morphism term language and the
typed $\lambda$-calculus presented in Section~3 are essentially 
{\em equivalent}.
Instead, we shall give a translation of morphism terms into $\lambda$-terms.
The idea is that a morphism term $f : (\sigma , \tau )$ is translated into
a $\lambda$-term $\trans{f} : \sigma \rightarrow \tau$.
\subsection*{Translation}
\[ \begin{array}{rcl}
\trans{{\sf id}} & = & \lambda x.x  \\
\trans{f;g} & = & \lambda x. \trans{g}(\trans{f} x) \\
\trans{1} & = & \lambda x. \star  \\
\trans{\ltuple f, g \rtuple} & = & \lambda x. (\trans{f} x, \trans{g} x) \\
\trans{{\sf p}} & = & \lambda z. \mylet{z}{x}{y}{x} \\
\trans{{\sf q}} & = & \lambda z. \mylet{z}{x}{y}{y}  \\
\trans{\curry{f}} & = & \lambda x. \lambda y. \trans{f} (x, y)  \\
\trans{{\sf Ap}} & = & \lambda f. \lambda x. f x \\
\trans{{\sf l}} & = & \lambda x. \imath (x) \\
\trans{{\sf r}} & = & \lambda y. \jmath (y) \\
\trans{\stup{f}{g}} & = & \lambda z. \mycases{z}{x}{\trans{f} x}{y}{\trans{g} y} \\
\trans{{\sf strict}(f)} & = & \lambda z. \mycases{\imath (\trans{f} x)}{x}{\trans{f} x}{y}{y} \\
\trans{{\sf up}} & = & \lambda x. {\sf up}(x) \\
\trans{{\sf lift}(f)} & = & \lambda y. \lift{y}{x}{\trans{f} x} \\
\end{array} \]
\[ \begin{array}{rcl}
\trans{\lsing \cdot \rsing_{l}} & = & \lambda x. \lsing x \rsing_{l} \\
\trans{\lsing \cdot \rsing_{u}} & = & \lambda x. \lsing x \rsing_{u} \\
\trans{f^{\dagger}_{l}} & = & \lambda z. \lextend{z}{x}{\trans{f} x} \\
\trans{f^{\dagger}_{u}} & = & \lambda z. \uextend{z}{x}{\trans{f} x} \\
\trans{+_{l}} & = & \lambda z. \mylet{z}{x}{y}{x \uplus_{l} y} \\
\trans{+_{u}} & = & \lambda z. \mylet{z}{x}{y}{x \uplus_{u} y} \\
\trans{\otimes_{l}} & = & \lambda z. \mylet{z}{x}{y}{x \otimes_{l} y} \\
\trans{\otimes_{u}} & = & \lambda z. \mylet{z}{x}{y}{x \otimes_{u} y} \\
\trans{{\sf fold}} & = & \lambda x. {\sf fold}(x) \\
\trans{{\sf unfold}} & = & \lambda x. {\sf unfold}(x) \\
\trans{{\sf Y}} & = & \lambda f. \mu x. f x 
\end{array} \]
\subsection*{Semantics}
Let ${\cal M}(\sigma , \tau )$ be the set of morphism terms of sort $(\sigma , \tau )$.
Since 
\[ {\bf SDom}({\cal D}(\sigma ), {\cal D}(\tau )) \cong {\cal D}(\sigma \rightarrow \tau ) \]
by cartesian closure, we can get a semantics
\[ \lsem \cdot \rsem_{\sigma \tau} : {\cal M}(\sigma , \tau ) \longrightarrow {\bf SDom}({\cal D}(\sigma ), {\cal D}(\tau )) \]
for morphism terms from the above translation.
We use this to extend our semantics for $\cal L$ from Section~2 to $\DDL$:
\[ \lsem [f]\phi \rsem = (\lsem f \rsem )^{-1} (\lsem \phi \rsem ) \]
(the other clauses being handled in the obvious way).
Note that the denotations of formulas in $\DDL$ are still {\em open} sets (continuity!),
but need no longer be compact-open, since compactness is not preserved under inverse image in general.

This semantics yields a notion of validity for $\DDL$ assertions:
\[ \models \phi \leq \psi \;\; \equiv \;\; \lsem \phi \rsem \subseteq \lsem \psi \rsem . \]
\begin{theorem} $\DDL$ is sound:
\[ \DDL \vdash \phi \leq \psi \;\; \Longrightarrow \;\; \models \phi \leq \psi \]
\end{theorem}

\proof\ The usual routine induction on the length of proofs.
We give a few cases for illustration.

Left injection.
\begin{eqnarray*}
(i) \;\; \lsem [{\sf l}]( \phi  \oplus \false ) \rsem & = & ( \lsem {\sf l} \rsem )^{-1}(\lsem ( \phi \oplus \false ) \rsem ) \\
& = & \{ d : \ltuple 0, d \rtuple \in \lsem (\phi \oplus \false ) \rsem \} \cup \{ \bot : \bot \in \lsem ( \phi \oplus \false ) \rsem \} \\
& = & \lsem \phi \rsem .
\end{eqnarray*}
\[ (ii) \;\; \psi \converges \Rightarrow \bot \not\in \lsem \psi \rsem \Rightarrow
(\lsem {\sf l} \rsem )^{-1} (\lsem ( \false \oplus \psi ) \rsem ) 
= \varnothing . \] 

Strictification. Note that
\[ \lsem {\sf strict}(f) \rsem d = \left\{ \begin{array}{ll}
\bot , & d = \bot \\
f d  & \mbox{otherwise}
\end{array}
\right. \]
Now,
\[ \phi \converges \Rightarrow \bot \not\in \lsem \phi \rsem \Rightarrow
\forall d \in \lsem \phi \rsem. \: \lsem {\sf strict}(f) \rsem d = fd, \]
which implies
\[ \lsem \phi \rsem \subseteq \lsem [f]\psi \rsem \; \Leftrightarrow \; \lsem
\phi \rsem \subseteq \lsem [{\sf strict}(f)]\psi \rsem . \]

Union.
\begin{eqnarray*}
(i) \;\; \lsem [+_{l}] \Diamond \phi \rsem & = &
\{ (X, Y) : (X \cup Y) \cap \lsem \phi \rsem \not= \varnothing \} \\
& = & \{ (X, Y) : X \cap \lsem \phi \rsem \not= \varnothing \; \mbox{or} \;
Y \cap \lsem \phi \rsem \not= \varnothing \} \\
& = & \{ (X, Z) : X \cap \lsem \phi \rsem \not= \varnothing \} \\
&   & \mbox{} \cup \{ (Z, Y) : Y \cap \lsem \phi \rsem \not= \varnothing \} \\
& = & \lsem (\Diamond \phi \times \true ) \vee (\true \times \Diamond \phi ) \rsem
\end{eqnarray*}
\begin{eqnarray*}
(ii) \;\; \lsem [+_{u}]\Diamond \phi \rsem & = &
\{ (X, Y) : X \cup Y \subseteq \lsem \phi \rsem \} \\
& = & \{ (X, Y) : X \subseteq \lsem \phi \rsem \: \& \: Y \subseteq \lsem \phi \rsem \} \\
& = & \lsem (\Box \phi \times \Box \phi ) \rsem .
\end{eqnarray*}

Recursion.
\[ \begin{array}{ll}
\bullet & \lsem \phi \rsem \subseteq \lsem [{\sf Y}]\psi \rsem \\
\Rightarrow & \forall f \in \lsem \phi \rsem . \: {\sf Y}f \in \lsem \psi \rsem \\
\Rightarrow & \forall f \in \lsem \phi \rsem \cap \lsem (\psi \rightarrow \theta ) \rsem . \: {\sf Y}f = f ({\sf Y} f) \in \lsem \theta \rsem . \;\;\; \qed
\end{array}  \]

Next, we turn to what can be proved in the way of completeness.
A {\em Hoare triple} in $\DDL$ is a formula $\phi \leq [f]\psi$
such that $\phi$ and $\psi$ are formulas of $\cal L$, i.e. do not contain any program modalities.
\begin{theorem}[Completeness For Hoare Triples]
\label{Htrip}
Let $\phi \leq [f]\psi$ be a Hoare triple. Then
\[ \DDL \vdash \phi \leq [f]\psi \;\; \Longleftrightarrow \;\; \models \phi \leq [f]\psi. \]
\end{theorem}
This result can either be proved directly, in similar fashion to Theorem~\ref{endogth}; or it can be reduced to that result, since
\[ \models \phi \leq [f]\psi \;\; \Longleftrightarrow \;\;
\trans{f}, \Gamma_{\true} \models (\phi \rightarrow \psi ) \;\; 
\Longleftrightarrow \;\; \trans{f}, \Gamma_{\true} \vdash (\phi \rightarrow \psi ) \]
(where $\Gamma_{\true}$ is the constant map $x \mapsto \true$).
It thus suffices to prove:
\[ \trans{f}, \Gamma_{\true} \vdash (\phi \rightarrow \psi ) \;\; \Longrightarrow \;\; \DDL \vdash \phi \leq [f]\psi . \]
In either approach, the argument is a straightforward variation on our work in
section~3, which we omit since it adds nothing new.

Finally, we come to a limitative result, which differentiates $\DDL$ from the
endogenous logic of Section~3, and shows that the restricted form of \ref{Htrip} is necessary.
The result is of course not ``surprising'', since 
$\DDL$ is semantically more expressive than the endogenous logic,  allowing
the description of non-compact open sets.
\begin{theorem}
The validity problem for $\DDL$ is $\Pi^{0}_{2}$-complete.
\end{theorem}

\proof\ 
We will need some notions on effectively given domains; see \cite[Chapter 7]{PloLN}.
Firstly, each type expression in our meta-language has an effectively given 
domain as its denotation (since effectively given domains are closed under recursive definitions and all our type constructions \cite[Chapter 7 pp.\ 16, 21, Chapter 8 pp.\ 16, 54]{PloLN}).
Similarly, each term $f : (\sigma , \tau )$ denotes a computable morphism from
${\cal D}(\sigma )$ to ${\cal D}(\tau )$.
Moreover, each $\phi \in {\cal L}(\sigma )$ denotes a compact-open, and hence
computable open set in ${\cal D}(\sigma )$; and computable open sets are closed
under inverse images of computable maps \cite[Chapter 7 p.\ 9]{PloLN},
and under finite unions and intersections \cite[Chapter 7 p.\ 7]{PloLN}.
Thus each formula of $\DDL$ denotes a computable open set, and the problem
of deciding the validity of the assertion $\phi \leq \psi$ can be reduced
to that of deciding the inclusion of r.e. sets $\lsem \phi \rsem \subseteq \lsem \psi \rsem$, which as is well-known \cite[IV.1.6]{Soa87} is  $\Pi^{0}_{2}$.

To complete the argument, we take a standard $\Pi_{2}^{0}$-complete problem,
and reduce it to validity in $\DDL$.
The problem we choose is
\[ {\sf Tot} = \{ x : W_{x} = \Nat \} \]
i.e. the set of codes of {\em total} recursive functions \cite[IV.3.2]{Soa87}.
To perform the reduction, we proceed as follows:
\begin{itemize}
\item The type ${\Nat}_{\bot} \equiv {\sf rec} \: t. \, ({\bf 1})_{\bot} \oplus t$ is
used to model the flat domain of natural numbers.
\item We can show that every partial recursive function $\varphi : \Nat \rightarrow \Nat$, thought of as a strict continuous function of type 
${\Nat}_{\bot} \rightarrow {\Nat}_{\bot}$, can be defined by a morphism term.
This is quite standard: the numerals are constructed from the injections,
lifting, 
and {\sf fold} and {\sf unfold}; the conditional and basic predicates from source tupling; and primitive recursion from general recursion ({\sf Y}) and conditional.
We omit the details.
\item In particular, we can define a morphism term 
$N : ({\Nat}_{\bot}, {\Nat}_{\bot})$ such that:
\[ \lsem N \rsem d = \left\{ \begin{array}{ll}
\bot , & d = \bot \\
0  & \mbox{otherwise}
\end{array}
\right. \]
\item Now given a partial recursive function $\varphi$, represented
by a morphism term $f$, the totality of $\varphi$ is equivalent to the $\DDL$-validity of
\[ N \leq [f][N]\bar{0}  \]
where $\bar{0} \equiv ((\true )_{\bot} \oplus \false)$ (so $\lsem \bar{0} \rsem = \{ 0 \}$). \qed
\end{itemize}
