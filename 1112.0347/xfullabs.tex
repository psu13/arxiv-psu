\section{Lambda Transition Systems considered as Programming Languages}
The classical discussion of full abstraction in the  $\lambda$-calculus 
\cite{Plo77,Mil77} is set in the typed $\lambda$-calculus with ground data. 
As remarked in the Introduction, this material has not to date been 
transferred successfully to the pure untyped $\lambda$-calculus. 
To see why this is so, let us recall some basic notions from \cite{Plo77,Mil77}.

Firstly, there is a natural notion of {\em program}, namely closed term of 
ground type. Programs either diverge, or yield a ground constant as result. 
This provides a natural notion of observable behaviour for programs, and 
hence an operational order on them. This is extended to arbitrary terms 
via ground contexts; in other words, the point of view is taken that only 
program behaviour is directly observable, and the meaning of a higher-type 
term lies in the observable behaviour of the programs into which it can be 
embedded. 
Thus both the presence of ground data, and the fact that terms are typed, 
enter into the basic definitions of the theory.

By contrast, we have a notion of atomic observation for the lazy 
$\lambda$-calculus in the absence of types or ground data, namely convergence 
to weak head normal form. This leads to the applicative bisimulation relation, 
and hence to a natural operational ordering. 
We can thus develop a theory of full abstraction in the pure untyped 
$\lambda$-calculus. 
Our results will correspond recognisably to those in \cite{Plo77}, although 
the technical details contain many differences. One feature of 
our development is that we work axiomatically with classes of lts under 
various hypotheses, rather than with particular languages. 
(Note that operational transition systems and ``programming languages'' such as $\lambda \ell$ actually {\em are} lts under our definitions.)
\begin{definition}
{\rm Let $\cal A$ be an lts. $\cal D$ is {\em fully abstract} for $\cal A$ if ${\Im}({\cal A}) = {\Im}({\cal D})$. }
\end{definition}
This definition is consistent with that in \cite{Plo77,Mil77}, provided 
we accept the applicative bisimulation ordering on $\cal A$ as 
the appropriate operational preorder. The argument for doing so 
is made highly plausible by Proposition~\ref{cont}, which characterises 
applicative bisimulation as a contextual preorder analogous to those used 
in \cite{Plo77,Mil77}. We shall prove \ref{cont} later in this section.

We now turn to the question of conditions under which $\cal D$ is 
fully abstract for $\cal A$. 
As emerges from \cite{Plo77,Mil77}, this is essentially a question of definability.
\begin{definition}
{\rm An ats $\cal A$ is $\cal L$-{\em expressive} if for all $\phi \in {\cal L}$, for some $a \in {\cal A}$:
\[ {\cal L}(a) = {\uparrow} \phi \; \equiv \; \{ \psi \in {\cal L} \; : \; {\cal L} \: \vdash \: \phi \leq \psi \} . \]
}
\end{definition}
In the light of Stone Duality, $\cal L$-expressiveness can be read as: ``all finite elements of $\cal D$ are definable in $\cal A$''.
\begin{definition}
\label{Cdef}
{\rm Let $\cal A$ be an ats.
\begin{itemize}
\item {\em Convergence testing} is definable in $\cal A$ if for some $c \in  A$, $\cal A$ satisfies:
\begin{itemize}
\item $c {\Converges}$
\item $x {\Diverges} \; \Rightarrow \; c x {\Diverges}$
\item $x {\Converges} \; \Rightarrow \; cx = {\bf I}$.
\end{itemize}
In this case, we use {\sf C} as a constant to denote $c$.
\item {\em Parallel convergence} is definable in $\cal A$ if for some $p \in  A$, $\cal A$ satisfies:
\begin{itemize}
\item $p {\Converges} , \;\; p x {\Converges}$
\item $x {\Converges} \; \Rightarrow \; p x y {\Converges}$
\item $y {\Converges} \; \Rightarrow \; p x y {\Converges}$
\item $x {\Diverges} \: \& \: y {\Diverges} \; \Rightarrow \; p x y {\Diverges}$ .
\end{itemize}
In this case, we use {\sf P} to denote such a $p$.
\end{itemize}}
\end{definition}
Note that if {\sf C} is definable, it is unique (up to bisimulation); this is not so for {\sf P}.

The notion of parallel convergence is reminiscent of Plotkin's parallel or, 
and will play a similar role in our theory. 
(A sharper comparison will be made later in this section.) 
The notion of convergence testing is less expected. 
We can think of the combinator {\sf C} as a sort of ``1-strict'' version of {\bf K}:
\[ {\sf C} x y = {\bf K} x y = y \;\;\;\; {\rm if} \; x {\Converges} \]
\[ {\sf C} x y {\Diverges} \;\;\;\; {\rm if} \; x {\Diverges} . \]
This 1-strictness allows us to test, sequentially, a number of expressions 
for convergence. 
Under the hypothesis that {\sf C} is definable, we can give a very satisfactory picture of the relationship between all these notions.
\begin{theorem}[Full Abstraction]
Let $\cal A$ be a sensible, approximable lts in which {\sf C} is definable. 
The following conditions are equivalent:
\begin{center}
\begin{tabular}{rl}
(i) &  Parallel convergence is definable in $\cal A$. \\ 
(ii) &  $\cal A$ is $\cal L$-expressive. \\
(iii) &  $\cal A$ is $\cal L$-complete. \\
(iv)  & ${\sf t}_{\cal A}$ is a combinatory embedding with ${\sl K}({\cal D}) \subseteq {\sl Im} \; {\sf t}_{\cal A}$. \\ 
(v)  & $\cal D$ is fully abstract for $\cal A$. 
\end{tabular}
\end{center}
\end{theorem}

\proof\ We shall prove a sequence of implications to establish the theorem, indicating in each case which hypotheses on $\cal A$ are used.

\noindent $(i) \; \Longrightarrow \; (ii)$ ($\cal A$ sensible, {\sf C} definable).

Since $\cal A$ is sensible, $\BOmega$ diverges in $\cal A$.
\\
{\bf Notation.} Given a set {\sf Con} of constants, $\BLambda ({\sf Con})$ is the set of $\lambda$-terms over Con.

For each $\phi \in N{\cal L}$ we shall define terms $M_{\phi}, T_{\phi} \in \BLambda(\{{\sf P, C}\})$ such that:
\[ \bullet \;\; M_{\phi} \; \models_{\cal A} \; \psi \;\; \Longleftrightarrow \;\; {\cal L} \; \vdash \; \phi \leq \psi \]
\[ \bullet \;\; \forall a \in A. \, \left\{ \begin{array}{ll}
T_{\phi} a \Converges & \mbox{if $a \; \models_{\cal A} \; \phi$,} \\
T_{\phi} a \Diverges & \mbox{otherwise.}
\end{array} \right. \]
The definition is by induction on the complexity of
\[ \phi \; \equiv \; \bigwedge_{i \in I} (\phi_{i, 1} \rightarrow \cdots (\phi_{i, k_{i}} \rightarrow \lambda )_{\bot} \cdots )_{\bot} . \]
If $I = \varnothing$, $M_{\phi} \equiv \BOmega$. Otherwise, we define $M_{\phi} \; \equiv \; M(\phi , k)$, where $k = {\rm max} \: \{ k_{i} \: | \: i \in I \}$:
\begin{eqnarray*}
M( \phi , 0 ) & \equiv & {\bf K} \BOmega \\
M( \phi , i+1 ) & \equiv & \lambda x_{j} . \, {\sf C} N M(\phi , i ) 
\end{eqnarray*}
where
\begin{eqnarray*}
j & \equiv & k - i \\
N & \equiv & \sum \{ N_{i} : j \leq k_{i} \} \\
N_{i} & \equiv & {\sf C} (T_{\phi_{i, 1}} x_{1}) ({\sf C}(T_{\phi_{i, 2}} x_{2})( \ldots ({\sf C} ( T_{\phi_{i, j}} x_{j} )) \ldots )) \\
\sum \varnothing & \equiv & \BOmega \\
\sum \{ N \} \cup \Theta & \equiv & {\sf P} N ( \sum \Theta ) .
\end{eqnarray*}
\begin{eqnarray*}
T_{\phi} & \equiv & \lambda x . \, \prod \{ x M_{\phi_{i, 1}} \ldots M_{\phi_{i, k_{i}}} : i \in I \} \\
\prod \varnothing & \equiv & {\bf K} \BOmega \\
\prod \{ N \} \cup \Theta & \equiv & {\sf C} N ( \prod \Theta ) .
\end{eqnarray*}
We must show that these definitions have the required properties. Firstly, we prove for all $\phi \in N{\cal L}$:
\[ (1) \;\; M_{\phi} \; \models_{\cal A} \; \phi \]
\[ (2) \;\; a \; \models_{\cal A} \; \phi \;\; \Rightarrow \;\; T_{\phi} a \Converges \]
by induction on $\phi$:
\[ \begin{array}{clr}
\bullet & \forall i \in I. \, a_{j} \; \models_{\cal A} \; \phi_{i, j} \;\; (1 \leq j \leq k_{i}) & \\
\Rightarrow & M_{\phi} a_{1} \ldots a_{k_{i}} {\Converges} & \mbox{by induction hypothesis (2),} \\
\therefore & M_{\phi} \; \vdash_{\cal A} \; \phi . & 
\end{array} \]
\[ \begin{array}{clr}
\bullet & a \; \models_{\cal A} \; \phi & 
\mbox{by induction hypothesis (1)} \\
\Rightarrow & T_{\phi} a \Converges . &
\end{array} \]
We complete the argument by proving, for all $\phi , \psi \in N{\cal L}$:
\[ \begin{array}{cccc}
(3)  & M_{\phi} \; \models_{\cal A} \; \psi & \Rightarrow & {\cal L} \; \vdash \; \phi \leq \psi \\
(4)  & M_{\psi} \; \models_{\cal A} \; \phi & \Rightarrow & {\cal L} \; \vdash \; \psi \leq \phi \\
(5) & T_{\phi} M_{\psi} \Converges & \Rightarrow &  M_{\psi} \; \models_{\cal A} \; \phi \\
(6) & T_{\psi} M_{\phi} \Converges & \Rightarrow &  M_{\phi} \; \models_{\cal A} \; \psi .
\end{array} \]
The proof is by induction on $n + m$, where $n, m$ are the number of sub-formulae of $\phi , \psi$ respectively. Let
\[  \phi \; \equiv \; \bigwedge_{i \in I} (\phi_{i, 1} \rightarrow \cdots (\phi_{i, k_{i}} \rightarrow \lambda )_{\bot} \cdots )_{\bot} , \]
\[  \psi \; \equiv \; \bigwedge_{j \in J} (\psi_{j, 1} \rightarrow \cdots (\psi_{j, k_{j}} \rightarrow \lambda )_{\bot} \cdots )_{\bot} . \] 
(3):
\[ \begin{array}{clr}
\bullet & M_{\phi} \; \models_{\cal A} \; \psi & \\ 
\Rightarrow & \forall j \in J . \, M_{\phi}  M_{\psi_{j, 1}} \ldots M_{\psi_{j, k_{j}}} {\Converges} & \mbox{by (1) ,} \\
\Rightarrow & \forall j \in J . \, \exists i \in I . \, k_{j} \leq k_{i} \: \& \: T_{\phi_{i, l}} M_{\psi_{j, l}} \Converges , \;\; 1 \leq l \leq k_{j} & \\
\Rightarrow &  M_{\psi_{j, l}} \; \models_{\cal A} \; \phi_{i, l} , \;\; 1 \leq l \leq k_{j} & \mbox{ind. hyp. (5)} \\
\Rightarrow & {\cal L} \; \vdash \; \psi_{j, l} \leq \phi_{i, l} , \;\; 1 \leq l \leq k_{j} & \mbox{ind. hyp. (4)} \\
\Rightarrow & {\cal L} \; \vdash \; \phi \leq \psi . &
\end{array} \]

\noindent (4): Symmetrical to (3).

\noindent (5):
\[  \begin{array}{clr}
\bullet & T_{\phi} M_{\psi} {\Converges} & \\ 
\Rightarrow & \forall i \in I . \, M_{\psi} M_{\phi_{i, 1}} \ldots M_{\phi_{i, k_{i}}} {\Converges} & \\
\Rightarrow & \forall i \in I . \, \exists j \in J . \, k_{i} \leq k_{j} \: \& \: T_{\psi_{j, l}} M_{\phi_{i, l}} \Converges , \;\; 1 \leq l \leq k_{i} & \\
\Rightarrow &  M_{\phi_{i, l}} \; \models_{\cal A} \; \psi_{j, l} , \;\; 1 \leq l \leq k_{i} & \mbox{ind. hyp. (6)} \\
\Rightarrow & {\cal L} \; \vdash \; \phi_{i, l} \leq \psi_{j, l} , \;\; 1 \leq l \leq k_{i} & \mbox{ind. hyp. (3)} \\
\Rightarrow & {\cal L} \; \vdash \; \psi \leq \phi  & \\ 
\Rightarrow &  M_{\psi} \; \models_{\cal A} \; \phi & \mbox{by (1). } 
\end{array} \]

\noindent (6): Symmetrical to (5).

\noindent $(ii) \; \Longrightarrow \; (iii)$ ($\cal A$ approximable). 

\noindent {\bf Notation.} For each $\phi \in {\cal L}$, $a_{\phi} \in A$ is the element representing $\phi$. Given $\Gamma : {\sf Var} \rightarrow {\cal L}$, $\rho_{\Gamma} \in Env({\cal A})$ is defined by
\[ \rho_{\Gamma} x = a_{\Gamma x} . \]
Finally, $\Gamma_{\true} : {\sf Var} \rightarrow {\cal L}$ is the constant map $x \mapsto \true$.

We begin with some preliminary results.
\[ (1) \;\; {\cal A} \models \phi \leq \psi \;\; \Longleftrightarrow \;\; {\cal L} \vdash \phi \leq \psi . \]
One half is the Soundness Theorem for $\cal L$. For the converse, note that
\begin{eqnarray*}
{\cal A} \models \phi \leq \psi & \Rightarrow & a_{\phi} \models_{\cal A} \psi \\
& \Rightarrow & {\cal L} \vdash \phi \leq \psi . 
\end{eqnarray*}
\[ (2) \;\; \forall \psi \in N{\cal L} . \, \psi \neq \true \: \& \: ab \models_{\cal A} \psi \; \Rightarrow \; \exists \phi . \, a \models_{\cal A} (\phi \rightarrow \psi )_{\bot} \: \& \: b \models_{\cal A} \phi . \]
This is shown by induction on $\psi$.
\[ \begin{array}{ll}
\bullet & a b \models_{\cal A} \bigwedge_{i \in I} \psi_{i} \;\; (I \neq \varnothing )  \\
\Rightarrow & \forall i \in I . \, a b \models_{\cal A}  \psi_{i} \\
\Rightarrow & \forall i \in I . \, \exists \phi_{i} . \, a \models_{\cal A} (\phi_{i} \rightarrow \psi_{i})_{\bot} \: \& \: b \models_{\cal A} \phi_{i} \;\;\; \mbox{by ind. hyp.} \\
\Rightarrow & \forall i \in I . \, a \models_{\cal A} (\bigwedge_{i \in I} \phi_{i} \rightarrow \psi_{i})_{\bot} \: \& \: b \models_{\cal A} \bigwedge_{i \in I} \phi_{i} \\
\Rightarrow & a \models_{\cal A} (\bigwedge_{i \in I} \phi_{i} \rightarrow \bigwedge_{i \in I} \psi_{i})_{\bot} \: \& \: b \models_{\cal A} \bigwedge_{i \in I} \phi_{i} .
\end{array} \]
\[ \begin{array}{ll}
\bullet & a b \models_{\cal A} (\psi_{1} \rightarrow \cdots ( \psi_{k} \rightarrow \lambda )_{\bot} \cdots )_{\bot} \\
\Rightarrow & a b a_{\psi_{1}} \ldots a_{\psi_{k}} {\Converges} \\
\Rightarrow & \exists \phi , \phi_{1} , \ldots , \phi_{k} . \, b \models_{\cal A} \phi \: \& \: a_{\psi_{i}} \models_{\cal A} \phi_{i} \; (1 \leq i \leq k) \\ 
& \mbox{} \& \: a \models_{\cal A} (\phi \rightarrow ( \phi_{1} \rightarrow \cdots (\phi_{k} \rightarrow \lambda )_{\bot} \cdots )_{\bot}, \\
& \mbox{since {\cal A} is approximable} \\
\Rightarrow & {\cal L} \vdash \psi_{i} \leq \phi_{i} \; (1 \leq i \leq k) \\
\Rightarrow & {\cal L} \vdash (\phi \rightarrow (\phi_{1} \rightarrow \cdots (\phi_{k} \rightarrow \lambda )_{\bot} \cdots )_{\bot} \\
& \mbox{} \leq (\phi \rightarrow (\psi_{1} \rightarrow \cdots (\psi_{k} \rightarrow \lambda )_{\bot} \cdots )_{\bot} \\
\Rightarrow & a \models_{\cal A} (\phi \rightarrow \psi )_{\bot} \: \& \: b \models_{\cal A} \phi .
\end{array} \]

\[ (3) \;\; \forall M \in \BLambda . \, M, \Gamma \models_{\cal A} \phi \; \Longleftrightarrow \; M, \rho_{\Gamma} \models_{\cal A} \phi . \]
The right to left implication is clear, since $\rho_{\Gamma} \models_{\cal A} \Gamma$. We prove the converse by induction on $M$.
\begin{eqnarray*}
x, \Gamma \models_{\cal A} \phi & \Longleftrightarrow & {\cal A} \models \Gamma x \leq \phi \\
& \Longleftrightarrow & {\cal L} \vdash \Gamma x \leq \phi \;\; {\rm by (1)} \\
& \Longleftrightarrow & a_{\Gamma x} \models_{\cal A} \phi \\
& \Longleftrightarrow & x, \rho_{\Gamma} \models_{\cal A} \phi .
\end{eqnarray*} 

The case for $\lambda x . M$ is proved by induction on $\phi$. We show the non-trivial case.
\[ \begin{array}{llr}
\bullet & \lambda x . M, \rho_{\Gamma} \models_{\cal A} (\phi \rightarrow \psi )_{\bot} & \\
\Longrightarrow & M, \rho_{\Gamma}[x \mapsto a_{\phi}] \models_{\cal A} \psi & \\
\Longrightarrow & M, \Gamma [x \mapsto \phi ] \models_{\cal A} \psi &  \mbox{by (outer) induction hypothesis} \\
\Longrightarrow & \lambda x . M, \Gamma \models_{\cal A} (\phi \rightarrow \psi )_{\bot} . &
\end{array} \]

\[ \begin{array}{llr}
\bullet & MN, \rho_{\Gamma} \models_{\cal A} \psi & \\
\Longrightarrow & \lsem M \rsem^{\cal A}_{\rho_{\Gamma}} \lsem N \rsem^{\cal A}_{\rho_{\Gamma}} \models_{\cal A} \psi & \\
\Longrightarrow & \exists \phi . \, \lsem M \rsem^{\cal A}_{\rho_{\Gamma}} \models_{\cal A} (\phi \rightarrow \psi )_{\bot} \: \& \: \lsem N \rsem^{\cal A}_{\rho_{\Gamma}} \models_{\cal A} \phi & \mbox{by (2)} \\
\Longrightarrow & M, \Gamma \models_{\cal A} (\phi \rightarrow \psi )_{\bot} \: \& \: N, \Gamma \models_{\cal A} \phi & \mbox{ind. hyp.} \\
\Longrightarrow & M N , \Gamma \models_{\cal A} \psi . &
\end{array} \]

\noindent (4):
\[ \begin{array}{rrcl}
(i) & x, \Gamma [x \mapsto \phi ] \models_{\cal A} \psi & \Longleftrightarrow & {\cal L} \vdash \phi \leq \psi \\
(ii) & \lambda x . M , \Gamma \models_{\cal A} (\phi \rightarrow \psi )_{\bot} & \Longleftrightarrow & M, \Gamma [ x \mapsto \phi ] \models_{\cal A} \psi \\
(iii) & MN, \Gamma \models_{\cal A} \psi & \Longleftrightarrow & \exists \phi . \, M, \Gamma \models_{\cal A} (\phi \rightarrow \psi )_{\bot} \\
& & & \mbox{} \& \: N, \Gamma \models_{\cal A} \phi .
\end{array} \]

$4(i)$ is proved using (1).

$4(ii)$:
\[ \begin{array}{ll}
\bullet & \lambda x . M , \Gamma \models_{\cal A} (\phi \rightarrow \psi )_{\bot} \\
\Rightarrow & \forall \rho , a . \, \rho \models_{\cal A} \Gamma \: \& \: a \models_{\cal A} \phi \; \Rightarrow \; \lsem \lambda x . M \rsem^{\cal A}_{\rho} . a \models_{\cal A} \psi \\
\Rightarrow & \forall \rho . \, \rho \models_{\cal A} \Gamma [ x \mapsto \phi ] \; \Rightarrow \; M, \rho \models_{\cal A} \psi \\ 
& \;\; \mbox{since} \; \lsem \lambda x \appl\ M \rsem^{\cal A}_{\rho} . a = \lsem  M \rsem^{\cal A}_{\rho [ x \mapsto a ]} , \\
\Rightarrow & M, \Gamma [ x \mapsto \phi ] \models_{\cal A} \psi .
\end{array} \]
The converse follows from the soundness of $\cal L$.

$4(iii)$:
\begin{Eqarray}
MN, \Gamma \models_{\cal A} \psi & \Longleftrightarrow & MN, \rho_{\Gamma} \models_{\cal A} \psi & \mbox{by (3)} \\
& \Longleftrightarrow & \lsem M \rsem^{\cal A}_{\rho_{\Gamma}} \lsem N \rsem^{\cal A}_{\rho_{\Gamma}} \models_{\cal A} \psi & \\
& \Longleftrightarrow & \exists \phi . \, \lsem M \rsem^{\cal A}_{\rho_{\Gamma}} \models_{\cal A} (\phi \rightarrow \psi )_{\bot} \: \& \: \lsem N \rsem^{\cal A}_{\rho_{\Gamma}} \models_{\cal A} \phi & \mbox{by (2)} \\
& \Longleftrightarrow & \exists \phi . \, M, \Gamma \models_{\cal A} (\phi \rightarrow \psi )_{\bot} \: \& \: N, \Gamma \models_{\cal A} \phi & \mbox{by (3)}
\end{Eqarray}

We can now prove
\[ M, \Gamma \models_{\cal A} \phi \; \Rightarrow \; M, \Gamma \vdash \phi \]
by induction on $M$, using (4).

\noindent $(iii) \; \Longrightarrow \; (i)$.

Firstly, note that $(iii)$ implies
\[ {\cal A} \models \phi \leq \psi \; \Longleftrightarrow \; {\cal L} \vdash \phi \leq \psi . \]
One half is the Soundness Theorem. For the converse, suppose ${\cal A} \models \phi \leq \psi$ and ${\cal L} \nvdash \phi \leq \psi$. Then ${\bf I} \models_{\cal A} (\phi \rightarrow \psi )_{\bot}$ but ${\bf I} \nvdash (\phi \rightarrow \psi )_{\bot}$, and so $\cal A$ is not $\cal L$-complete.

Now suppose that {\sf P} is not definable in $\cal A$, and consider
\[ \phi \equiv (\lambda \rightarrow (\true \rightarrow \lambda )_{\bot})_{\bot} \wedge (\true \rightarrow ( \lambda \rightarrow \lambda )_{\bot})_{\bot} , \]
\[ \psi \equiv (\true \rightarrow (\true \rightarrow \lambda )_{\bot})_{\bot} . \]
Clearly, ${\cal L} \nvdash \phi \leq \psi$. 
However, for $a \in {\cal A}$, if $a \models_{\cal A} \phi$, then $x \Converges$ or $y \Converges$ implies $a x y \Converges$; 
since {\sf P} is not definable in $\cal A$, and in particular, $a$ does not define {\sf P}, we must have $a x y \Converges$ even if $x \Diverges$ and $y \Diverges$, and hence $a \models_{\cal A} \psi$. 
Thus ${\cal A} \models \phi \leq \psi$ and so by our opening remark, $\cal A$ is not $\cal L$-complete.

\noindent $(ii) \; \Longrightarrow \; (iv)$ ($\cal A$ approximable).

Clearly ${\sf Im} \; t_{\cal A} \supseteq  {\cal K}(D)$, by 5.14(ii). 
Also, since $\cal A$ is approximable, we can apply the Characterisation Theorem 
to deduce that $t_{\cal A}$ is injective (modulo bisimulation). 
To show that $t_{\cal A}$ is a combinatory morphism, we argue as in 
\ref{isocalg}. 
Application is preserved by $t_{\cal A}$ using (2) from the proof of 
$(ii)  \Rightarrow  (iii)$ and \ref{isocalg}. 
The proof is completed by showing that $t_{\cal A}$ preserves denotations of 
$\lambda$-terms, i.e.
\[ \forall M \in \BLambda , \rho \in Env({\cal A}) . \, t_{\cal A} (\lsem M \rsem^{\cal A}_{\rho}) = \lsem M \rsem^{D}_{t_{\cal A} \circ \rho } . \]
The proof is by induction on $M$. 
Since it is very similar to the corresponding part of the proof of \ref{isocalg}, we omit it. 
The only non-trivial point is that in the case for abstraction we need:
\[ \forall a \in A. \, a \models_{\cal A} \phi \; \Longrightarrow \; M, \rho [x \mapsto a ] \models_{\cal A} \psi \]
if and only if
\[ M, \rho [ x \mapsto a_{\phi}] \models_{\cal A} \psi , \]
which is proved similarly to (3) in $(ii) \; \Rightarrow \; (iii)$.

\noindent $(iv) \; \Longrightarrow \; (v)$.

Assuming $(iv)$, $\cal A$ is isomorphic (modulo bisimulation) to a substructure of $D$. 
Since formulas in {\sf HF} are (equivalent to) universal ($\Pi^{0}_{1}$) 
sentences, this yields ${\Im}(D) \subseteq {\Im}({\cal A})$. 
Since ${\cal K}(D) \subseteq {\sf Im}\: t_{\cal A}$, to prove the converse it is sufficient to show, for $H \in {\sf HF}$:

\[ D, \rho \nvDash H \;\; \Longrightarrow \;\; \exists \rho_{0} : {\sf Var} \rightarrow {\cal K}(D). \, D, \rho \nvDash H . \]
Let $H \equiv P \Rightarrow F$, where $P \equiv \bigwedge_{i \in I}M_{i} \Converges \wedge \bigwedge_{j \in J}N_{j} \Diverges$. There are four cases, corresponding to the form of $F$.

Case 1: $F \equiv M \sqsubseteq N$. 
$D, \rho \nvDash P \Rightarrow F$ implies $D, \rho \models P$ and 
$D, \rho \nvDash M \sqsubseteq N$. Since $D$ is algebraic, 
$D, \rho \nvDash M \sqsubseteq N$ implies that for some $b \in {\cal K}(D)$, 
$b \sqsubseteq \lsem M \rsem^{D}_{\rho}$ and  $b \not\sqsubseteq \lsem N \rsem^{D}_{\rho}$. 
Since the expression  $\lsem M \rsem^{D}_{\rho}$ is continuous in $\rho$, 
$b \sqsubseteq \lsem M \rsem^{D}_{\rho}$ implies that for some 
$\rho_{1} : {\sf Var} \rightarrow {\cal K}(D)$, $\rho_{1} \sqsubseteq \rho$ 
and $b \sqsubseteq \lsem M \rsem^{D}_{\rho_{1}}$. 
For all $\rho'$ with $\rho_{1} \sqsubseteq \rho' \sqsubseteq \rho$, 
$\lsem N \rsem^{D}_{\rho'} \sqsubseteq \lsem N \rsem^{D}_{\rho}$, and hence 
$b \not\sqsubseteq \lsem N \rsem^{D}_{\rho'}$. 
Again, since $D$ is algebraic,
\[ D, \rho \models M_{i} \Converges \;\; \Longrightarrow \;\; \exists \rho_{i} : {\sf Var} \rightarrow {\cal K}(D) . \, \rho_{i} \sqsubseteq \rho \: \& \: D, \rho_{i} \models M_{i} \Converges . \]
Now let $\rho_{0} \equiv \bigsqcup_{i \in I}{\rho_{i} \sqcup \rho_{1}}$. 
This is well-defined since $D$ is a lattice. 
Moreover, $\rho_{0} \sqsubseteq \rho$, and $\rho_{0} : {\sf Var} \rightarrow {\cal K}(D)$. 
Since $\rho_{0} \sqsupseteq \rho_{i} \; (i \in I)$, $D, \rho_{0} \models M_{i} \Converges$; 
while since $\rho_{0} \sqsubseteq \rho$, $D, \rho_{0} \models N_{j} \Diverges \; (j \in J)$. 
Since $\rho_{1} \sqsubseteq \rho_{0} \sqsubseteq \rho$, 
$b \sqsubseteq \lsem M \rsem^{D}_{\rho_{0}}$ and  
$b \not\sqsubseteq \lsem N \rsem^{D}_{\rho_{0}}$, and so 
$D, \rho_{0} \nvDash M \sqsubseteq N$. 
Thus  $D, \rho_{0} \nvDash P \Rightarrow F$, as required.

The remaining cases are proved similarly.

\noindent $(v) \; \Longrightarrow \; (i)$ ($\cal A$ sensible).

Consider the formula
\[ H \equiv x \BOmega ({\bf K} \BOmega ) \Converges \wedge x ( {\bf K} \BOmega ) \BOmega \Converges \; \Rightarrow \; x \BOmega \BOmega \Converges . \]
It is easy to see that ${\cal A} \models H$ iff {\sf P} is not definable in $\cal A$. 
Since {\sf P} is definable in $D$, the result follows. \qed

We now turn to the question of when the bisimulation preorder on an lts can be characterised by means of a contextual equivalence, as in \cite{Bar,Plo77,Mil77}.

\begin{definition}
{\rm Let $\cal A$ be an lts, $X, Y \subseteq A$. Then {\em $X$ separates $Y$} if:}
\[ \begin{array}{l}
\forall M, N \in {\BLambda}^{0}(Y). \, {\cal A} \nvDash M \sqsubseteq N \; \Longrightarrow \\ 
\;\; \exists P_{1}, \ldots , P_{k} \in {\BLambda}^{0}(X) . \,
{\cal A} \models M P_{1} \ldots P_{k} \Converges \: \& \: {\cal A} \models N P_{1} \ldots P_{k} \Diverges . 
\end{array} \]
\end{definition}

In particular, if $X$ separates $A$ we say that it is a {\em separating set}. For example, $A$ is always a separating set.

\begin{proposition}
\label{sepl}
Let $\cal A$ be an approximable lts, and suppose $X$ separates $Y$. Then
\[ \forall  M, N \in {\BLambda}^{0}(Y). \, {\cal A} \models M \sqsubseteq N \;\; \Longleftrightarrow \] 
\[ \forall C[\cdot ] \in {\BLambda}^{0}(X). \,
{\cal A} \models C[M] \Converges \; \Rightarrow \; {\cal A} \models C[N] \Converges . \] 
\end{proposition}

\proof\ Suppose ${\cal A} \nvDash M \sqsubseteq N$. 
Then since $X$ separates $Y$, for some $P_{1}, \ldots , P_{k} \in {\BLambda}^{0}(X)$, 
${\cal A} \models M P_{1} \ldots P_{k} \Converges$ and 
${\cal A} \models N P_{1} \ldots P_{k} \Diverges$. 
Let $C[\cdot ] \equiv [\cdot ]P_{1} \cdots P_{k}$. 
For the converse, suppose ${\cal A} \models M \sqsubseteq N$ and ${\cal A} \models C{M} \Converges$. 
Since ${\cal A}$ is approximable and ${\cal A} \models C[M] = \lambda x . C[x] M$, 
for some $\phi$ $\lambda x . C[x] \models_{\cal A} (\phi \rightarrow \lambda )_{\bot}$ and $M \models_{\cal A} \phi$. 
Since ${\cal A} \models M \sqsubseteq N$, by the Characterisation Theorem $N \models_{\cal A} \phi$, and so ${\cal A} \models C[N] \Converges$. \qed

As a first application of this Proposition, we have:

\begin{proposition}
Let $\cal A$ be a sensible, approximable lts in which {\sf C} and {\sf P} are definable. Then $\{ {\sf C}, {\sf P} \}$ is a separating set.
\end{proposition}

\proof\ By the Full Abstraction Theorem, for each $\phi \in {\cal L}$ there is $M_{\phi} \in {\BLambda}^{0}(\{{\sf C}, {\sf P} \})$ such that
\[ M_{\phi} \models_{\cal A} \psi \; \Longleftrightarrow \; {\cal L} \vdash \phi \leq \psi . \]
Now
\[ \begin{array}{ll}
\bullet & {\cal A} \nvDash M \sqsubseteq N  \\ 
\Longrightarrow & \exists \phi . \, M \models_{\cal A} \phi \: \& \: N \nvDash \phi , \;\; 
\mbox{since {\cal A} is approximable} \\
\Longrightarrow & \exists \phi_{1} , \ldots , \phi_{k} . \, M \models_{\cal A} (\phi_{1} 
\rightarrow \cdots (\phi_{k} \rightarrow \lambda )_{\bot} \cdots )_{\bot}  \\ 
& \;\; \& \: N \nvDash_{\cal A} (\phi_{1} \rightarrow \cdots (\phi_{k} \rightarrow \lambda )_{\bot} \cdots )_{\bot}  \\
\Longrightarrow & M M_{\phi_{1}} \ldots M_{\phi_{k}} \Converges \: \& \: N M_{\phi_{1}} \ldots M_{\phi_{k}} \Diverges . \;\;\; \qed
\end{array} \]

The hypothesis of approximability has played a major part in out work. We now give a useful sufficient condition.

\begin{definition}
{\rm Let $\cal A$ be an lts, $X \subseteq A$. Then $\cal A$ is $X$-{\em sensible} if}
\[ \forall M \in {\BLambda}^{0}(X). \, {\cal A} \models M \Converges \; \Rightarrow \; D \models M \Converges . \]
\end{definition}

Here $\lsem M \rsem^{D}$ is the denotation in $D$ obtained by mapping each $a \in X$ to $t_{\cal A}(a)$. Note that if we extend our endogenous program logic to terms in ${\BLambda}^{0}(X)$, with axioms
\[ a, \Gamma \vdash \phi \;\; ( \phi \in {\cal L}(a)) , \]
then the Soundness and Completeness Theorems for $D$ still hold, by a straightforward extension of the arguments used above.

\begin{proposition}
\label{approxl}
Let $\cal A$ be an $X$-sensible lts. Then $\cal A$ is $X$-approximable, i.e.
\[ \forall M, N_{1} , \ldots , N_{k} \in {\BLambda}^{0}(X) . \, {\cal A} \models M N_{1} \ldots N_{k} \Converges \; \Rightarrow \; \exists \phi_{1} , \ldots , \phi_{k} . \]
\[ M \models_{\cal A} (\phi_{1} \rightarrow \cdots (\phi_{k} \rightarrow \lambda )_{\bot} \cdots )_{\bot} \: \& \: N_{i} \models_{\cal A} \phi_{i}, \; 1 \leq i \leq k . \]
\end{proposition}

\proof 
\[ \begin{array}{ll}
\bullet & {\cal A} \models M N_{1} \ldots N_{k} \Converges \\
\Rightarrow & D \models M  N_{1} \ldots N_{k} \Converges \\
\Rightarrow &  \exists \phi_{1} , \ldots , \phi_{k} . \,  M \models_{\cal D} (\phi_{1} \rightarrow \cdots (\phi_{k} \rightarrow \lambda )_{\bot} \cdots )_{\bot} \\
& \;\; \& \: N_{i} \models_{\cal D} \phi_{i}, \; 1 \leq i \leq k , \; \mbox{since {D} is approximable} \\
\Rightarrow & \exists \phi_{1} , \ldots , \phi_{k} . \,  M \vdash (\phi_{1} \rightarrow \cdots (\phi_{k} \rightarrow \lambda )_{\bot} \cdots )_{\bot} \\
& \;\; \& \: N_{i} \vdash \phi_{i}, \; 1 \leq i \leq k , \; \mbox{by extended Completenss} \\
\Rightarrow & \exists \phi_{1} , \ldots , \phi_{k} . \,  M \models_{\cal A} (\phi_{1} \rightarrow \cdots (\phi_{k} \rightarrow \lambda )_{\bot} \cdots )_{\bot} \\
& \;\; \& \: N_{i} \models_{\cal A} \phi_{i}, \; 1 \leq i \leq k , \; \mbox{by extended Soundness} . \;\;\; \qed
\end{array} \]

In particular, if $X$ generates $\cal A$ and $\cal A$ is $X$-sensible, then $\cal A$ is approximable. We now turn to a number of applications of these ideas to syntactically presented lts, i.e. ``programming languages''.

Firstly, we consider the lts $\ell = ({\BLambda}^{0}, eval)$ defined in section 3 (and studied previously in section 2). 
Since $\ell$ is $\varnothing$-sensible by \ref{ltsprop}, and it is generated by $\varnothing$, it is approximable by \ref{approxl}. 
Since $\varnothing$ is a separating set for ${\BLambda}^{0}$, we can apply \ref{sepl} to obtain Theorem \ref{cont}.

Next, we consider extensions of $\ell$.

\begin{definition}
{\rm (i) $\ell_{\sf C}$ is the extension of $\ell$ defined by
\[ \ell_{\sf C} = (\BLambda (\{{\sf C} \}), \_ \Converges \_ ) \]
where $\Converges$ is the extension of the relation defined in \ref{convdef} with the following rules:
\[ \bullet \; {\sf C} \Converges {\sf C} \;\;\;\; \bullet \; \frac{M \Converges}{{\sf C} M \Converges {\bf I}} \]
(ii) $\ell_{\sf P}$ is the extension $(\BLambda (\{{\sf C} \}), \_ \Converges \_ )$ of $\ell$ with the rules}
\[ \bullet \; {\sf P} \Converges {\sf P} \;\;\;\; \bullet \; {\sf P} M \Converges {\sf P} M \;\;\;\; \bullet \; \frac{M \Converges}{{\sf P} M N \Converges {\bf I}} \;\;\;\; \bullet \; \frac{N \Converges}{{\sf P} M N \Converges {\bf I}} \]
\end{definition}

It is easy to see that the relation $\_ \Converges \_$ as defined in both $\ell_{\sf C}$ and $\ell_{\sf P}$ is a partial function. 
Moreover, with these definitions the {\sf C} and {\sf P} combinators have the 
properties required by \ref{Cdef}; while {\sf C} is definable in $\ell_{\sf P}$, by
\[ {\sf C} M \equiv {\sf P} M M . \]

Since $\ell_{\sf C}$ is generated by $\{ {\sf C } \}$, and $\ell_{\sf P}$ by  $\{ {\sf P } \}$, these are separating sets. 
Thus to apply Theorem \ref{sepl}, we need only check that  $\ell_{\sf C}$ is {\sf C}-sensible, and  $\ell_{\sf P}$ {\sf P}-sensible. 

To do this for $\ell_{\sf C}$, we proceed as follows. Define
\[ c \equiv \{ (\lambda \rightarrow ( \phi \rightarrow \phi )_{\bot} )_{\bot} \: | \: \phi \in {\cal L} \}^{\dag} \in {\sf Filt} \: {\cal L}. \]
Then it is easy to see that $c \subseteq t_{\cal A}({\sf C})$, and by monotonicity and the Soundness Theorem,
\[ \lsem M[c / {\sf C} ] \rsem^{D} \subseteq \lsem M \rsem^{D} \]
for $M \in \BLambda^{0}(\{ {\sf C} \} )$. Thus
\[ (\star) \;\; D \models M [ c / {\sf C} ] \Converges \; \Longrightarrow \; D \models M \Converges . \]
Now we prove
\[ \begin{array}{cl}
(\star \star ) & \forall M, N \in {\BLambda}^{0}(\{ {\sf C} \} ) . \\ 
& M \Converges N \;\; \Longrightarrow \;\;
\lsem M [ c / {\sf C} ] \rsem^{D} = \lsem N [ c / {\sf C} ] \rsem^{D} \: \& \: D \models N [ c / {\sf C} ] \Converges , 
\end{array} \]
which by $(\star )$ yields   $\ell_{\sf C} \models M \Converges \; \Rightarrow \; D \models M \Converges$, as required. $(\star \star )$ is proved by a straightforward induction on the length of the proof that $M \Converges N$.

The argument for $\ell_{\sf P}$ is similar, using
\[ p \equiv \{ (\lambda \rightarrow (\true \rightarrow (\phi \rightarrow \phi )_{\bot} )_{\bot} )_{\bot} \wedge (\true \rightarrow (\lambda \rightarrow (\psi \rightarrow \psi )_{\bot} )_{\bot} )_{\bot} : \phi , \psi \in {\cal L} \}^{\dag} . \]

Altogether, we have shown
\begin{theorem}[Contextual Equivalence]
(i) $\forall M, N \in \BLambda^{0}(\{ {\sf C} \} )$:
\[ \ell_{\sf C} \models M \sqsubseteq N \; \Longleftrightarrow \; \forall C[\cdot ] \in {\BLambda}^{0}(\{ {\sf C} \} ). \, \ell_{\sf C} \models C[M] \Converges \; \Rightarrow \; \ell_{\sf C} \models C[N] \Converges . \]
(ii) $\forall M, N \in \BLambda^{0}(\{ {\sf P} \} )$:
\[ \ell_{\sf P} \models M \sqsubseteq N \; \Longleftrightarrow \; \forall C[\cdot ] \in {\BLambda}^{0}(\{ {\sf P} \} ). \, \ell_{\sf P} \models C[M] \Converges \; \Rightarrow \; \ell_{\sf P} \models C[N] \Converges . \]
\end{theorem}

As a further application of these ideas, we have
\begin{proposition}[Soundness of D]
If $\cal A$ is $X$-sensible, and $X$ separates $X$ in $\cal A$, then:
\[ {\Im}^{0}(D, X) \subseteq {\Im}^{0}({\cal A}, X) . \]
\end{proposition}

\proof 
\[ \begin{array}{ll}
\bullet & D \models M \sqsubseteq N \\
\Longrightarrow & \forall C[ \cdot ] \in {\BLambda}^{0}(X). \, D \models C[M] \sqsubseteq C[N] \\
\Longrightarrow & D \models C[M] \Converges \; \Rightarrow \; D \models C[N] \Converges \\
\Longrightarrow & {\cal A} \models C[M] \Converges \; \Rightarrow \; {\cal A} \models C[N] \Converges \\
\Longrightarrow & {\cal A} \models M \sqsubseteq N .
\end{array} \]
The argument for formulae of other forms is similar. \qed

As an immediate corollary of this Proposition,
\begin{proposition}
The denotational semantics of each of our languages is {\em sound} with respect to
the operational semantics:
\[ \begin{array}{rl}
(i) & {\Im}^{0}(D) \subseteq {\Im}^{0}(\ell ) \\
(ii) &  {\Im}^{0}(D, \{ {\sf C} \} ) \subseteq {\Im}^{0}(\ell_{\sf C}, \{ {\sf C} \} ) \\
(iii) & {\Im}^{0}(D, \{ {\sf P} \} ) \subseteq {\Im}^{0}(\ell_{\sf P}, \{ {\sf P} \} ) . 
\end{array} \]
\end{proposition}

We now turn to the question of full abstraction for these languages. Since, as we have seen, $\ell_{\sf P}$ is {\sf P}-sensible, and hence sensible and approximable, and {\sf C} and {\sf P} are definable, we can apply the Full Abstraction Theorem to obtain
\begin{proposition}
D is fully abstract for $\ell_{\sf P}$.
\end{proposition}

We now use the sequential nature of $\ell$ and $\ell_{\sf C}$ to obtain negative full abstraction results for these languages. This will require a few preliminary notions.

\begin{definition}
{\rm The {\em one-step reduction} relation $>$ over terms in $\BLambda$ is the least satisfying the following axioms and rules:
\[ 
\bullet \;\; (\lambda x . M)N > M[N/x] \;\;\;\;
\bullet \;\; \frac{M > M'}{MN > M' N} 
\]
This is then extended to $\BLambda ( \{ {\sf C} \} )$ with the additional rules
\[ \bullet \;\; {\sf C} ( \lambda x . M ) > {\bf I} 
\;\;\;\; \bullet \;\; {\sf CC} > {\bf I} 
\;\;\;\; \bullet \;\; \frac{M > M'}{{\sf C} M > {\sf C} M'} \]
We then define
\[ \begin{array}{crcl}

\bullet & \gg & \equiv & \mbox{the reflexive, transitive closure of $>$} \\
\bullet & M \diverges & \equiv & \exists \{ M_{n} \} . \, M = M_{0} \: \& \: \forall n . \, M_{n} > M_{n + 1} \\
\bullet & M {\not>} & \equiv & M \not\in {\sf dom} {>} \\
\bullet & M {\converges} & \equiv & M \gg N \: \& \: N \not>.
\end{array} \] }
\end{definition}

It is clear that $>$ is a partial function. Note that these relations are being defined over {\em all} terms, not just closed ones. For closed terms, these new notions are related to the evaluation predicate $\_ \Converges \_$ as follows:

\begin{proposition}
\label{one1}
For $M, N \in \BLambda^{0} \; ( \BLambda^{0} ( \{ {\sf C} \} )$:
\[ \begin{array}{rrcl}
(i) & M \Converges N & \Longleftrightarrow & M \converges N \\
(ii) & M \Diverges & \Longrightarrow & M \diverges .
\end{array} \] 
\end{proposition}

We omit the straightforward proof. The following proposition is basic; it says that ``reduction commutes with substitution''.

\begin{proposition}
\label{one2}
$ M \gg N \; \Rightarrow \; M [ P / x ] \gg N [ P / x ]$ .
\end{proposition}

\proof\ Clearly, it is sufficient to show:
\[ M > N \; \Rightarrow \; M [ P / x ] > N [ P / x ] . \]
This is proved by induction on $M$, and cases on why $M > N$. We give one case for illustration:
\[ M \equiv (\lambda y . M_{1})M_{2} > N \equiv M_{1} [ M_{2} / y] . \]
We assume $x \not= y$; the other sub-case is simpler.
\begin{Eqarray}
M [ P / x ] & = & (\lambda y . M_{1} [ P / x ] ) M_{2} [ P / x ] & \\
& > &  M_{1} [ P / x ] [ M_{2} [ P / x ] / y ] & \\
& = & M_{1} [ M_{2} / y] [ P / x ] &  \mbox{by \cite[2.1.16]{Bar}} \\
& = & N [ P / x ] .  & \qed 
\end{Eqarray}

Now we come to the basic sequentiality property of $\ell$ from which various non-definability results can be deduced.

\begin{proposition} 
\label{thrcase}
For $M \in \BLambda$, exactly one of the following holds:
\[ \begin{array}{rl}
(i) & M \diverges \\
(ii) & M \gg \lambda x . N \\
(iii) & M \gg x N_{1} \ldots N_{k} \; (k \geq 0 ) .
\end{array} \]
\end{proposition}

\proof\ Since $>$ is a partial function, the computation sequence beginning with $M$ is uniquely determined. Either it is infinite, yielding $(i)$; or it terminates in a term $N$ with $ N \not>$, which must be in one of the forms $(ii)$ or $(iii)$. \qed

As a consequence of this proposition, we obtain

\begin{theorem}
\label{Cundef}
{\sf C} is not definable in $\ell$. Moreover, $D$ is not fully abstract for $\ell$.
\end{theorem}

\proof\ We shall show that $\ell$ satisfies
\[ (\star ) \;\; x = {\bf I} \;\; \mbox{or} \;\; [x \BOmega \Converges \;\; \Longleftrightarrow \;\; x ({\bf K} \BOmega ) \Converges ] . \]
Indeed, consider any term $M \in {\BLambda}^{0}$. 
Either $M \Diverges$, in which case $M \BOmega \Diverges $ and $M ( {\bf K} \BOmega ) \Diverges $, or $M \Converges$. 
In the latter case, by $(\Converges \eta )$ we have $\lambda \ell \models M = \lambda x . M x$. 
Thus without loss of generality we may take $M$ to be of the form $\lambda x . M'$, with $FV(M) \subseteq \{ x \}$. 
Now applying the three previous propositions to $M'$, we see that in case $(i)$ of \ref{thrcase}, $(\lambda x . M' ) \BOmega \Diverges$ and 
$(\lambda x . M' ) ({\bf K} \BOmega ) \Diverges$; 
in case $(ii)$, $(\lambda x . M' ) \BOmega \Converges$ and $(\lambda x . M' ) ({\bf K} \BOmega ) \Converges$; 
finally in case $(iii)$, if $k = 0$, $\lambda x . M' = {\bf I}$; 
while if $k > 0$, $(\lambda x . M' ) \BOmega \Diverges$ and $(\lambda x . M' ) ({\bf K} \BOmega ) \Diverges$. 
Since ${\sf C} \not= {\bf I}$, ${\sf C} \BOmega \Diverges$ and ${\sf C} ( {\bf K} \BOmega ) \Converges$, this shows that {\sf C} is not definable. 
Moreover, $(\star )$ implies
\[ (\star \star ) \;\;  x \BOmega \Diverges \: \& \: x ( {\bf K} \BOmega ) \Converges \; \Rightarrow \; x = {\bf I} \]
which is not satisfied by $D$, since {\sf C} is definable in $D$, and taking $x = {\sf C}$ refutes $(\star \star )$; hence $D$ is not fully abstract for $\ell$. \qed

Note that since {\sf C} is not definable in $\ell$, we could not apply the Full Abstraction Theorem. 
By contrast, to show that $D$ is not fully abstract for $\ell_{\sf C}$, it suffices to show that {\sf P} is not definable. 
For this purpose, we prove a result analogous to \ref{thrcase}.

\begin{proposition} 
\label{frcase}
For $M \in \BLambda ( \{ {\sf C} \} )$, exactly one of the following conditions holds:
\[ \begin{array}{rl}
(i) & M \diverges \\
(ii) & M \gg \lambda x . N \\
(iii) & M \gg {\sf C} \\
(iv) & M \gg \underbrace{{\sf C}({\sf C} \ldots ({\sf C}}_{n} x N_{1} \ldots N_{k} ) \ldots ) P_{1} \ldots P_{m} \;\; (n, k, m \geq 0)
\end{array} \]
\end{proposition}

\proof\ Similar to \ref{thrcase}. \qed

\begin{theorem}
\label{Pundef}
{\sf P} is not definable in $\ell_{\sf C}$; hence $D$ is not fully abstract for $\ell_{\sf C}$.
\end{theorem}

\proof\ We show that $\ell_{\sf C}$ satisfies
\[ x ({\bf K} \BOmega ) \BOmega \Converges \: \& \: x \BOmega ( {\bf K} \BOmega ) \Converges \; \Rightarrow \; x \BOmega \BOmega \Converges , \]
and hence, as in the proof of the Full Abstraction Theorem, {\sf P} 
is not definable in $\ell_{\sf C}$. 
As in the proof of \ref{Cundef}, without loss of generality we consider closed terms of the form $\lambda y_{1} . \lambda y_{2} . M$. 
Assume $(\lambda y_{1} . \lambda y_{2} . M) ({\bf K} \BOmega ) \BOmega \Converges$ 
and  $(\lambda y_{1} . \lambda y_{2} . M) \BOmega ( {\bf K} \BOmega ) \Converges$. 
Applying \ref{frcase}, we see that case $(i)$ is impossible; 
cases $(ii)$ and $(iii)$ imply that $(\lambda y_{1} . \lambda y_{2} . M) \BOmega \BOmega  \Converges$; 
while in case $(iv)$, if $x = y_{1}$, then  $(\lambda y_{1} . \lambda y_{2} . M) \BOmega ( {\bf K} \BOmega ) \Diverges$, {\em contra hypothesis}; 
and if $x = y_{2}$, $(\lambda y_{1} . \lambda y_{2} . M) ({\bf K} \BOmega ) \BOmega \Diverges$, 
also {\em contra hypothesis}. 
Thus case $(iv)$ is impossible, and the proof is complete. \qed

For our final non-definability result, we shall consider a different style of extension of $\ell$, to incorporate {\em ground data}. 
We shall consider the simplest possible such extension, where a single atom is added. This corresponds to the domain equation
\[ D_{\star} = {\bf 1} + [D_{\star} \rightarrow D_{\star}] \]

(where $+$ is separated sum), which is indeed an extension of our original domain, in the sense that $D$ is a retract of $D_{\star}$. 
$D_{\star}$ is still a Scott domain (indeed, a coherent algebraic cpo), but it is no longer a lattice; 
we have introduced {\em inconsistency} via the sum.

This extension is reflected on the syntactic level by two constants, $\star$ and {\sf C}. We define
\[ \ell_{\star} = (\BLambda^{0} ( \{ \star , {\sf C} \} ), \_ \Converges \_ ) \]
with $\_ \Converges \_$ extending the definition for $\ell$ as follows:
\[ \bullet \;\; \star \Converges \star \]
\[ \bullet \;\; {\sf C} \Converges {\sf C} \]
\[ \bullet \;\; \frac{M \Converges \lambda x . N}{{\sf C} M \Converges {\sf T}} \;\; ({\sf T} \equiv \lambda x . \lambda y . x ) \]
\[ \bullet \;\; \frac{M \Converges {\sf C}}{{\sf C} M \Converges {\sf T}} \]
\[ \bullet \;\; \frac{M \Converges \star}{{\sf C} M \Converges {\sf F}} \;\; ({\sf F} \equiv \lambda x . \lambda y . y ) \]

We see that the {\sf C} combinator introduced here is a natural generalisation 
(not strictly an extension) of the {\sf C} defined previously in the pure case. 
Of course, {\sf C} corresponds to {\em case selection}, which in the unary case --- lifting being unary separated sum --- is just convergence testing.

A theory can be developed for $\ell_{\star}$ which runs parallel to what we have done for the pure lazy $\lambda$-calculus. 
Some of the technical details are more complicated because of the presence of inconsistency, but the ideas and results are essentially the same. 
Our reasons for mentioning this extension are twofold:

\begin{enumerate}

\item To show how the ideas we have developed can be put in a broader context. 
In particular, with the extension to $\ell_{\star}$  the reader should be able to see, at least in outline, 
how our work can be applied to systems such as Martin-L\"{o}f's Type Theory 
under its Domain Interpretation \cite{Cha83}, and (the analogues of) our 
results in this section can be used to settle most of the questions and conjectures raised in \cite{Cha83}.

\item To prove an interesting result which clarifies a point about which there seems to be some confusion in the literature; 
namely, {\em what is parallel or}? 
\end{enumerate}

The {\it locus classicus} for parallel or in the setting of typed $\lambda$-calculus is \cite{Plo77}. 
But what of untyped $\lambda$-calculus? 
In \cite[p.\  375]{Bar}, we find the following definition: 
\[ F M N = \left\{ \begin{array}{ll}
{\bf I} & \mbox{if {\mit M} or {\mit N} is solvable,} \\
{\rm unsolvable} & \mbox{otherwise}
\end{array}
\right. \]
which (modulo the difference between the standard and lazy theories) corresponds to our parallel convergence combinator {\sf P}. 
The point we wish to make is this: in the pure $\lambda$-calculus, where (in domain terms) there are no inconsistent data values 
(since everything is a function), i.e. we have a lattice, 
parallel convergence does indeed play the role of parallel or, as the Full Abstraction Theorem shows. 
However, when we introduce ground data, and hence inconsistency, a distinction reappears between parallel convergence and parallel or, 
and it is definitely {\em wrong} to conflate them. 
To substantiate this claim, we shall prove the following result: even if parallel convergence is added to $\ell_{\star}$, parallel or is still not definable. 
This result is also of interest from the point of view of the fine structure of definability; 
it shows that parallelism is not all or nothing even in the simple, deterministic setting of $\ell_{\star}$.

\begin{definition}
{\rm $\ell_{\star {\sf P}}$ is the extension of $\ell_{\star}$ with a constant {\sf P} and the rules}
\[ \bullet \;\; {\sf P} \Converges {\sf P} \;\;\;\;
\bullet \;\; {\sf P} M \Converges {\sf P} M \;\;\;\;
\bullet \;\; \frac{M \Converges}{{\sf P} M N \Converges {\bf I}} \;\;\;\; 
\bullet \;\; \frac{N \Converges}{{\sf P} M N \Converges {\bf I}} \]
\end{definition}

\begin{definition}
{\rm Let $\ell'$ be an extension of $\ell_{\star}$. 
We say that {\em parallel or is definable in $\ell'$} if for some term $M$
\[ \begin{array}{rl}
(i) & M ({\bf K} \BOmega ) \BOmega ,  M \BOmega ({\bf K} \BOmega ) \;\; 
\mbox{converge to abstractions} \\
(ii) & M \star \star \Converges \star .
\end{array} \] }
\end{definition}

\begin{theorem}
Parallel or is not definable in $\ell_{\star {\sf P}}$.
\end{theorem}

\proof\ We proceed along similar lines to our previous non-definability results. Firstly, we extend our definition of $>$ as follows:
\[ \bullet \;\; {\sf constructor}(M) \equiv M \; \mbox{is an abstraction, {\sf P}, {\sf C} or $\star$} \]
\[ \bullet \;\; {\sf constructor}(M) \: \& \: M \not= \star \; \Rightarrow \; {\sf C} M > {\sf T} \]
\[ \bullet \;\; {\sf C} \star > {\sf F} \]
\[ \bullet \;\; \frac{M > M'}{{\sf C} M > {\sf C} M'} \]
\[ \bullet \;\; {\sf constructor}(M) \; \mbox{or} \; {\sf constructor}(N) \; \Rightarrow \; {\sf P} M N > {\bf I} \]
\[ \bullet \;\; \frac{M > M' \;\; N > N'}{{\sf P} M N > {\sf P} M' N'} \]

With these extensions, $>$ is still a partial function, and \ref{one1}, \ref{one2} still hold. For each $M \in \BLambda ( \{ \star , {\sf C} , {\sf P} \} )$, one of the following two disjoint conditions must hold:
\[ \begin{array}{cl}
\bullet & M \diverges \\
\bullet & M \gg N \: \& \: N \not> .
\end{array} \]

We now define $\cal T$ to be the set of all terms $M$ in $\BLambda ( \{ \star , {\sf C} , {\sf P} , \bot \} )$, where $\bot$ is a new constant, such that:
\[ \begin{array}{cl}
\bullet & FV(M) \subseteq \{ y_{1}, y_{2} \} \\
\bullet & M \; \mbox{contains no {$>$}-redex.}
\end{array} \]
Note that $\cal T$ is closed under sub-terms.

\subsection*{Lemma A}
For all $M \in {\cal T}$:
\[ \begin{array}{c}
M[ {\bf K} \BOmega / y_{1} , \BOmega / y_{2} ] \converges a \; \& \; 
M[ \BOmega / y_{1} , {\bf K} \BOmega / y_{2} ] \converges b \; \& \; 
M[ \star / y_{1} , \star / y_{2} ] \converges c \\
\Rightarrow \; a = b = c = \star \; \mbox{or} \;\star \not\in \{ a, b, c \} .
\end{array} \] 

\proof\ By induction on $M$. Since terms in $\cal T$ contain no $>$-redexes, $M$ must have one of the following forms:
\[ \begin{array}{rl}
(i) & x N_{1} \ldots N_{k} \;\; ( x \in \{ y_{1}, y_{2} \} , k \geq 0 ) \\
(ii) & \star N_{1} \ldots N_{k} \;\; (k \geq 0 ) \\
(iii) & \lambda x . N \\
(iv) & {\sf C} \;\; (v) \; {\sf P} \;\; (vi) \; {\sf P} N \\
(vii) & {\sf C} N N_{1} \ldots N_{k} \;\; (k \geq 0 ) \\
(viii) & {\sf P} M_{1} M_{2} N_{1} \ldots N_{k} \;\; (k \geq 0 ) \\
(ix) & \bot N_{1} \ldots N_{k} \;\; (k \geq 0 )
\end{array} \]

Most of these cases can be disposed of directly; we deal with the two which use the induction hypothesis.

$(vii)$. Firstly, we can apply the induction hypothesis to $N$ to conclude that $N[c_{1} / y_{1}, c_{2} / y_{2}]$ 
converges to the same result (i.e. either an abstraction or $\star$) 
for all three argument combinations $c_{1}, c_{2}$; 
we can then apply the induction hypothesis to either $N_{1} N_{3} \ldots N_{k}$ or $N_{2} N_{3} \ldots N_{k}$.

$(viii)$. Under the hypothesis of the Lemma, we must have 
\[ ({\sf P} M_{1} M_{2})[c_{1} / y_{1}, c_{2} / y_{2}] \Converges {\bf I} \]
for all three argument combinations $c_{1}, c_{2}$; hence we can apply the induction hypothesis to $N_{1} \ldots N_{k}$. \qed

\subsection*{Lemma B}
Let $M \in \BLambda\ ( \{ \star , {\sf C} , {\sf P} \} )$, with $FV(M) \subseteq \{ y_{1}, y_{2} \}$. 
Then for some $M' \in {\cal T}$, for all $P, Q \in {\BLambda}^{0} ( \{ \star , {\sf C} , {\sf P} \} )$:
\[ M [ P / y_{1} , Q / y_{2} ] \converges {\star} \;\; \Longleftrightarrow \;\;  M' [ P / y_{1} , Q / y_{2} ] \converges {\star} . \]

\proof\ Given $M$, we obtain $M'$ as follows; working in an inside-out fashion, we replace each sub-term $N$ by:
\[ \left\{ \begin{array}{ll}
N' & \mbox{if $N \converges N'$} \\
\bot & \mbox{if $N \diverges$.} \;\;\; \qed
\end{array} \right. \]

Now suppose that we are given a putative term in ${\BLambda}^{0} ( \{ \star , {\sf C} , {\sf P} \} )$ defining parallel or. 
As in the proof of \ref{Pundef}, we may take this term to have the form $\lambda y_{1} . \lambda y_{2} . M$. 
Applying Lemma B, we can obtain $M' \in {\cal T}$ from $M$; but then applying Lemma A, we see that $\lambda y_{1} . \lambda y_{2} . M'$ 
cannot define parallel or. 
Applying Lemma B again, we conclude that $\lambda y_{1} . \lambda y_{2} . M$ cannot define parallel or either. \qed
