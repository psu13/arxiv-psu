\section{Locales}
Our reference for locale theory and Stone duality will be \cite{Joh82}.
Since locale theory is not yet a staple of Computer Science, we shall briefly review some of the basic ideas.

Classically, the study of general topology is based on the category {\bf Top} of topological spaces and continuous maps.
However, in recent years mathematicicans influenced by categorical and constructive ideas have advocated that attention be shifted to the open-set lattices as the primary objects of study.
Given a space $X$, we write $\Omega (X)$ for the lattice of open subsets of $X$ ordered by inclusion.
Since $\Omega (X)$ is closed under arbitrary unions and finite intersections, it is a complete lattice satisfying the infinite distributive law
\[ a \wedge \bigvee S = \bigvee \{ a \wedge s : s \in S \} . \]
(By the Adjoint Functor Theorem, in any complete lattice this law is equivalent to the existence of a right adjoint to conjunction, i.e. to the fact that implication can be defined in a canonical way.)
Such a lattice is a {\em complete Heyting algebra}, i.e. the Lindenbaum algebra of an {\em intuitionistic} theory.
The continuous functions  between topological spaces preserve unions and intersections, and hence  all joins and finite meets of open sets, under inverse image; thus we get a functor
\[ \Omega : {\bf Top} \rightarrow {\bf Loc} \]
where {\bf Loc}, the category of {\em locales}, is the opposite of {\bf Frm}, the category of {\em frames}, which has complete Heyting algebras as objects, and maps preserving all joins and finite meets as morphisms.
Note that {\bf Frm} is a concrete category of structured sets and structure-preserving maps, and consequently convenient to deal with (for example, it is monadic over {\bf Set}).
Thus we study {\bf Loc} {\it via} {\bf Frm}; but it is {\bf Loc} which is the proposed alternative or replacement for {\bf Top}, and hence the ultimate object of study.

{\bf Notation.} Given a morphism $f : A \rightarrow B$ in {\bf Loc}, we write $f^{\star}$ for the corresponding morphism $B \rightarrow A$ in {\bf Frm}.

Now we can define a functor
\[ {\sf Pt} : {\bf Loc} \rightarrow {\bf Top} \]
as follows (for motivation, see our discussion of Stone's original construction in Chapter~1):
${\sf Pt} (A)$ is the set of all frame morphisms $f : A \rightarrow {\bf 2}$, where {\bf 2} is the two-point lattice.
Any such $f$ can be identified with the set $F = f^{-1}(1)$, which satisfies:
\[ 1 \in F \]
\[ a, b \in F \; \Rightarrow \; a \wedge b \in F \]
\[ a \in F, a \leq b \; \Rightarrow \; b \in F \]
\[ \bigvee_{i \in I} a_{i} \in F \; \Rightarrow \; \exists i \in I. \, a_{i} \in F . \]
Such a subset is called a {\em completely prime filter}.
Conversely, any completely prime filter $F$ determines a frame homomorphism 
$\chi_{F} : A \rightarrow {\bf 2}$.
Thus we can identify ${\sf Pt}(A)$ with the completely prime filters over $A$.
The topology on ${\sf Pt}(A)$ is given by the sets $U_{a}$ ($a \in A$):
\[ U_{a} \equiv \{ x \in {\sf Pt}(A) : a \in F \} . \]
Clearly, 
\[ {\sf Pt}(A) = U_{1}, \;\; U_{a} \cap U_{b} = U_{a \wedge b}, \;\; \bigcup_{i \in I} U_{a_{i}} = U_{\bigvee_{i \in I} a_{i}}, \]
so this is a topology.
{\sf Pt} is extended to morphisms by:
\[ \frac{A \stackrel{f^{\star}}{\longleftarrow} B}{{\sf Pt}(A) \stackrel{{\sf Pt}(f)}{\longrightarrow} {\sf Pt}(B)} \]
\[ {\sf Pt}(f) x = \{ b : f^{\star} b \in x \} . \]

We now define, for each $X$ in {\bf Top} and $A$ in {\bf Loc}:
\[ \eta_{X} : X \rightarrow {\sf Pt}( \Omega (X)) \]
\[ \eta_{X}(x) = \{ U : x \in U \} \]
\[ \epsilon_{A} : \Omega ({\sf Pt}(A)) \rightarrow A \]
\[ \epsilon_{A}^{\star}(a) = \{ x : a \in x \} . \]
Now we have
\begin{theorem}
\label{t.1}
(\cite[II.2.4]{Joh82}). $(\Omega , {\sf Pt}, \eta , \epsilon ) : {\bf Top} 
\rightharpoonup {\bf Loc}$ defines an adjunction between {\bf Top} and 
{\bf Loc}; moreover (\cite[II.2.7]{Joh82}), this cuts down to an equivalence between the full sub-categories {\bf Sob} of sober spaces and {\bf SLoc} of spatial locales.
\end{theorem}
The equivalence between {\bf Sob} and {\bf SLoc} (and therefore the {\em duality} or contravariant equivalence between {\bf Sob} and {\bf SFrm}) may be taken as the most general purely topological version of Stone duality.
For our purposes, some  dualities arising as restrictions of this one are of interest.
\begin{definition}
{\rm A space $X$ is {\em coherent} if the compact-open subsets of $X$ 
(notation: $K \Omega (X)$) form a basis closed under finite intersections, i.e. for which $K \Omega (X))$ is a distributive sub-lattice of $\Omega (X)$.
}
\end{definition}
\begin{theorem}
\label{t.2}
(i) (\cite[II.2.11]{Joh82}). The forgetful functor from {\bf Frm} to {\bf DLat}, the category of distributive lattices, has as left adjoint the functor {\sf Idl}, which takes a distributive lattice to its ideal completion.

\noindent (ii) (\cite[II.3.4]{Joh82}). Given a distributive lattice $A$, define ${\sf Spec} \; A$ as the set of prime filters over $A$ (i.e. sets of the form $f^{-1}(1)$ for lattice homomorphisms $f : A \rightarrow {\bf 2}$), with topology generated by
\[ U_{a} \equiv \{ x \in {\sf Spec} \; A : a \in x \} \;\;\; (a \in A). \]
Then ${\sf Spec} \; A \cong {\sf Pt}({\sf Idl}(A))$.

\noindent (iii) (\cite[II.3.3]{Joh82}). The duality of Theorem~\ref{t.1} cuts down to a duality
\[ {\bf CohSp} \simeq {\bf CohLoc} \simeq {\bf DLat}^{\sf op} \]
where {\bf CohSp} is the category of coherent $T_{0}$ spaces, and continuous maps which preserve compact-open subsets under inverse image; 
and ${\bf CohLoc}^{\sf op}$ is the image of {\bf DLat} under the functor {\sf Idl}.
\end{theorem}

The logical significance of the coherent case is that finitary syntax---specifically finite disjunctions---suffices.
The original Stone duality theorem discussed in Chapter~1 is obtained as the further  restriction of this duality to coherent Hausdorff spaces (which turns out to be another description of the Stone spaces) and Boolean algebras, i.e. complemented distributive lattices.
Note that under the compact Hausdorff condition, {\em all} continuous maps satisfy the special property in part (iii) of the Theorem.

As a further special case of Stone duality, we note:
\begin{theorem}
\label{msl}
(i) The forgetful functor from distributive lattices to the category {\bf MSL} of meet-semilattices has a left adjoint {\sf L}, where ${\sf L}(A) = \{ \converges(X) : X \in \wp_{\sf f}(A) \}$, ordered by inclusion. 
(Notice that this is the same construction as for the lower powerdomain; this fact {\em is} significant, but not in the scope of this thesis.)

\noindent (ii) For any meet-semilattice $A$, define ${\sf Filt}(A)$ as 
the set of all filters over $A$, with topology defined exactly as for ${\sf Spec}(A)$. Then
\[ {\sf Filt}(A) \cong {\sf Spec}({\sf L}(A)) \cong {\sf Pt}({\sf Idl}({\sf L}(A))). \]

\noindent (iii) The duality of Theorem~\ref{t.2} cuts down to a duality
\[ {\bf CohAlgLat} \simeq {\bf MSL}^{\sf op} \]
where {\bf CohAlgLat} is the full sub-category of {\bf CohSp} of algebraic 
lattices with the Scott topology (to be defined in the next section).
\end{theorem}

An extensive treatment of locale theory and Stone-type dualities can be found in \cite{Joh82}.
Our purpose in the remainder of this section is to give some conceptual perspectives on the theory.

Firstly, a {\em logical} perspective.
As already mentioned, locales are the Lindenbaum algebras of intuitionistic theories, more particularly of {\em propositional geometric theories}, i.e. the logic of finite conjunctions and infinite conjunctions.
The morphisms preserve this geometric structure, but are {\em not} required to preserve the additional ``logical'' structure of implication and negation (which can be defined in any complete Heyting algebra).
Thus from a logical point of view, locale theory is propositional geometric logic.
Moreover, Stone duality also has a logical interpretation.
The {\em points} of a space correspond to {\em models} in the logical sense; the {\em theory} of a model is the completely prime filter of opens it satisfies, where the satisfaction relation is just
\[ x \models a \equiv x \in a \]
in terms of spaces, (i.e. with $x \in X$ and $a \in \Omega (X)$), and
\[ x \models a \equiv a \in x \]
in terms of locales (i.e. with $x \in {\sf Pt}(A)$ and $a \in A$).
Spatiality of a class of locales is then a statement of {\em Completeness}: every consistent theory has a model.

Secondly, a {\em computational} perspective.
If we view the points of a space as the denotations of computational processes 
(programs, systems), then the elements of the corresponding locale can be 
seen as {\em properties} of computational processes.
More than this, these properties can in turn be thought of as computationally meaningful; we propose that they be interpreted as {\em observable properties}.
Intuitively, we say that a property is observable if we can tell whether or not it holds of a process on the basis of only a finite amount of information about that process\footnote{This is really only one facet of observability. Another
is {\em extensionality}, i.e. that we regard a process as a black box with
some specified interface to its environment, and only take what is observable
via this interface into account in determining the meaning of the process.
Extensionality in this sense is obviously {\em relative} to our choice of 
interface;
it is orthogonal to the notion being discussed in the main text.}.
Note that this is really {\em semi}-observability, since if the property is {\em not} satisfied, we do not expect that this is finitely observable.
This intuition of observability  motivates the asymmetry between conjunction and disjunction in geometric logic and topology.
Infinite disjunctions of observable properties are still observable---to see that $\bigvee_{i \in I} a_{i}$ holds of a process, we need only observe that {\em one} of the $a_{i}$ holds---while infinite conjunctions clearly do not preserve finite observability in general.
More precisely, consider Sierpinski space $\Oh$.
We can regard this space as representing the possible outcomes of an experiment to determine whether a property is satisfied; the topology is motivated by semi-observability, so an observable property on a space $X$ should be a {\em continuous} function to $\Oh$.
In fact, we have
\[ \Omega (X) \cong (X \rightarrow \Oh ) \]
where $(X \rightarrow \Oh )$ is the continuous function space, ordered pointwise (thinking of $\Oh$ as {\bf 2}).
Now for infinite $I$, $I$-ary disjunction, viewed as a function
\[ \Oh^{I} \rightarrow \Oh \]
is continuous, while $I$-ary conjunction is not.
Similarly, implication and negation, taken as functions
\[ \Rightarrow : \Oh^{2} \rightarrow \Oh , \;\;\; \neg : \Oh \rightarrow \Oh \]
are not continuous.
Thus from this perspective,
\begin{center}
geometric logic = observational logic.
\end{center}

These ideas follow those proposed by Smyth in his pioneering paper \cite{Smy83}, but with some differences.
In \cite{Smy83}, Smyth interprets ``open set'' as {\em semi-decidable} property; this represents an ultimate commitment to interpret our mathematics in some effective universe.
My preference is to do Theoretical Computer Science in as ontologically or foundationally {\em neutral} a manner as possible.
The distinction between semi-observability and semi-decidability is analogous 
to the distinction between the computational motivation for the basic axioms 
of domain theory in terms of ``physical feasibility'' given in 
\cite[Chapter 1]{PloLN}, without any appeal to notions of recursion theory; and a commitment to only considering computable elements and morphisms of effectively given domains, as advocated in \cite{Kan79}.
It should also be said that the link between observables and open sets in domain theory was clearly (though briefly!) stated in \cite[Chapter 8 p.\  16]{PloLN}, and used there to motivate the definition of the Plotkin powerdomain.

A final perpective is {\em algebraic}.
The category {\bf Frm} is algebraic over {\bf Set} (\cite[II.1.2]{Joh82}); thus working with locales, we can view topology as a species of (infinitary) algebra.
In particular,  constructions of universal objects of various kinds by ``generators and relations'' are possible.
Two highly relevant examples in the locale theory literature are \cite{Joh85} and \cite{Hyl81}. 
This provides a link with the information systems approach to domain theory as in \cite{Sco82,LW84}.
Some of our work in Chapters~3 and~4 can be seen as a systematization of these ideas in an explicitly syntactic framework.

\section{Domains and Locales}
We now turn to the connections between domains and locales.
Firstly, it is standard that domains can be viewed topologically.
\begin{definition}
{\rm (\cite[Chapter 1 p.\  16]{PloLN}). Given a poset $P$, the {\em Scott topology} on $P$ has as open sets those $U \subseteq P$ satisfying
\begin{enumerate}
\item $U$ is upper-closed, i.e. $U = \diverges (U)$.
\item $U$ is inaccessible by $\omega$-chains, i.e.
\[ \bigsqcup_{n \in \omega} x_{n} \in U \; \Rightarrow \;  \exists n. \, x_{n} \in U. \]
\end{enumerate}
We write $\sigma (D)$ for the Scott topology on a domain $D$.}
\end{definition}
\begin{proposition}
\label{cop}
(i) ({\it loc. cit.}) Let $D$, $E$ be cpo's; a function $f : D \rightarrow E$ 
is continuous in the cpo sense iff it is continuous with respect to the Scott topology.

\noindent (ii) (\cite[Chapter 6 p.\  3]{PloLN}). 
For algebraic domains $D$, the Scott topology has a particularly simple form: namely all sets of the form
\[ \bigcup_{i \in I} \diverges (b_{i}) \;\;\;\; (b_{i} \in {\cal K}(D), i \in I) \]
Moreover, the compact-open sets are just those of this form with $I$ {\em finite}.
\end{proposition}
Given a space $X$, we define the {\it specialisation order} on $X$ by
\[ x \leq_{\sf spec} y \equiv \forall U \in \Omega (X). \, x \in U \; \Rightarrow \; y \in U. \]
\begin{proposition}
(\cite[Chapter 1 p.\  16]{PloLN}). Let $D$ be a cpo. The specialisation order on the space $(D, \sigma (D))$ coincides with the original ordering on $D$.
\end{proposition}
Thus we may regard domains indifferently as posets or as spaces with the Scott topology, justifying some earlier abuses of notation.

We now relate domains to coherent spaces.
\begin{theorem}[The $2/3$ SFP Theorem]
(\cite[Chapter 8 p.\  41]{PloLN}). An algebraic cpo is coherent as a space iff it is ``$2/3$ SFP'' in the terminology of (\it loc. cit.).
Since coherent spaces are sober (\cite{Joh82} II.3.4), any such domain $D$ satisfies
\[ D \cong {\sf Spec}(K \Omega (D)). \]
\end{theorem}
We shall refer to such domains as {\em coherent algebraic}.
Thus {\bf SDom} and {\bf SFP} are categories of coherent spaces, and we need only consider the lattices of compact-open sets on the logical side of the duality.

We conclude with some observations which show how the finite elements in a coherent algebraic domain play an ambiguous role as both points and properties.
Firstly, we have
\[ D \cong {\sf Idl}({\cal K}(D)) \]
so the finite elements determine the structure of $D$ on the spatial side.
We can also recover the finite elements in purely lattice-theoretic terms from $A = K \Omega (D)$.
Say that $a \in A$ is {\it consistent} if $a \not= 0$, and {\it prime} if $a \leq {b \vee c}$ implies $a \leq b$ or $a \leq c$. (We should probably say coprime rather than prime, but as we will have no need for the dual concept, we will use the shorter term.)
Writing $cpr(A)$ for the set of consistent primes of $A$, we have
\begin{equation}
{\cal K}(D) = (cpr(A))^{\sf op}, \;\;\; A \cong {\sf L}(({\cal K}(D))^{\sf op}) . 
\end{equation}
(The fact that the latter construction produces a distributive lattice even 
though ${\cal K}(D)$ is not a meet-semilattice follows from the MUB axioms 
characterizing the coherent algebraic domains \cite[Chapter 8 p.\  41]{PloLN}.)

\begin{theorem}
Let $A$ be a distributive lattice. ${\sf Spec}(A)$ is coherent algebraic iff the following conditions are satisfied:
\[ \begin{array}{rl}
(1) & 1_{A} \in cpr(A) \\
(2) & \forall a \in A. \, \exists b_{1}, \ldots , b_{n} \in cpr(A). \, a = \bigvee_{i=1}^{n} b_{i}. 
\end{array} \]
\end{theorem}
Of these, (1) ensures the existence of a bottom point, and (2) says ``there are enough primes''.
This result will be proved as part of our work in the next Chapter.

