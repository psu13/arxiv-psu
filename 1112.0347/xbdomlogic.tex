\section{A Domain Logic for Transition Systems} 
We now introduce our domain logic in an infintary version \Ellinfty, with
a finitary subset $\Ellomega$. 
We show how \Ellinfty\ can be interpreted in any transition system,
present a proof system, and establish its soundness. 
We then turn to \Ellomega\ , and prove the main result of the section:
\Ellomega\ is the Stone dual of \Dom. That is, \Dom\ is isomorphic to
the spectral space of \Ellomega, while \Ellomega\ is isomorphic to the
lattice of compact-open subsets of \Dom. 
This duality will be crucial to our work in the next section. 

\begin{definition} 
{\rm The language \Ellinfty\ has two {\em sorts}: $\pi$ (process) and
$\kappa$ (capability). We write $\Ell_{\infty \pi}$ ($\Ell_{\infty
\kappa}$) for the class of formulae of sort $\pi$ ($\kappa$), which are
defined inductively as follows:} 
\[ \bullet \;\;\; \frac{\{\phi_{i} \in \Ell_{\infty \sigma} \}_{i \in
I}}{\bigvee_{i \in I}\phi_{i} , \bigwedge_{i \in I}\phi_{i} \in \Ell_{\infty
\sigma}} \;\;\; (\sigma \in \{ \pi , \kappa \}) \] 
\[ \bullet \;\;\; \frac{a \in {\sf Act}, \;\; \phi \in \Ell_{\infty \pi}}{a(\phi ) \in
\Ell_{\infty \kappa}} \]
\[ \bullet \;\;\; \frac{\phi \in \Ell_{\infty \kappa}}{\Box \phi , \Diamond
\phi \in \Ell_{\infty \pi}} . \] 
\end{definition} 

\noindent {\bf Notation.} We write $\true \; \equiv \;  \bigwedge_{i \in
\varnothing} \phi_{i}$, $\false \; \equiv \; \bigvee_{i \in \varnothing}
\phi_{i}$. 

The sublanguage of \Ellinfty\ obtained by the restriction to {\em finite}
conjunctions and disjunctions is denoted \Ellomega\ . {\em Height}, {\em
modal depth} and {\em sort} are defined for \Ell\ in entirely analogous
fashion to HML. 
For example: 
\[ \begin{array}{llllll} 
\bullet & {\sf md}(\bigwedge_{i \in I}\phi_{i}) & \equiv & {\sf md}(\bigwedge_{i \in
I}\phi_{i}) & \equiv & \sup \; \{ {\sf md}(\phi_{i} : i \in I \} \\ 
\bullet & {\sf md}(a(\phi )) & \equiv & {\sf md}(\phi ) & & \\ 
\bullet & {\sf md}(\Box \phi ) & \equiv & {\sf md}(\Diamond \phi ) & \equiv & {\sf md}(\phi
) + 1 . 
\end{array} \] 
For each $A \subseteq {\sf Act}$ and ordinal $\lambda$: 
\[ {\Ell}^{(A, \lambda )}_{\infty} \; \equiv \; \{ \phi \in \Ellinfty :
{\sf sort}(\phi ) \subseteq A \: \& \: {\sf md}(\phi ) \leq \lambda \} . \] 

It should be clear how the form of our language is derived from the type
expression 
\[ {\sf rec} \: t. \, P^{0}[\sum_{a \in {\sf Act}} t] . \] 
The two-sorted structure of \Ell\ corresponds to the type constructions
$P^0$~($\pi$) and $\sum_{a \in {\sf Act}}$~($\kappa$). 
The recursion in the type expression is mirrored by the mutual recursion
between the two sorts. 
Note that the Plotkin powerdomain is built from the combination of the
{\em must} modality $\Box$ of the Smyth powerdomain and the {\em may}
modality $\Diamond$ of the Hoare powerdomain ({\it cf.} \cite{Abr83a,Win83}). 
\subsection*{Interpretation of $\cal L$\/  in transition systems} 
Given a transition system $({\rm Proc}, {\sf Act}, {\rightarrow} , \diverges )$, we
define 
\[ {\rm Cap} \; \equiv \; \{ \bot \} \cup {({\sf Act} \times {\rm Proc})} \] 
\[ C : {\rm Proc} \rightarrow \wp ({\rm Cap}) \] 
\[ C(p) = \{ \bot : p \diverges \} \cup \{ \ltuple a, q \rtuple : \labarrow{p}{a}{q} \} . \] 
$C(p)$ is the set of {\em capabilities} of $p$. 
We can now define satisfaction relations 
\[ {\models_{\pi}} \subseteq {\rm Proc} \times \Ell_{\infty \pi} , \] 
\[ {\models_{\kappa}} \subseteq {\rm Proc} \times \Ell_{\infty \kappa} : \] 
For $\sigma  \in \{ \pi , \kappa \}$:
\[ \begin{array}{lll}
w \models_{\sigma} \bigwedge_{i \in I} \phi_{i} & \equiv & \forall i \in I .
\, w \models_{\sigma} \phi_{i} \\
w \models_{\sigma} \bigvee_{i \in I} \phi_{i} & \equiv & \exists i \in I . \,
w \models_{\sigma} \phi_{i} \\
p \models_{\pi} \Box \phi & \equiv & \forall c \in C(p) . \, c
\models_{\kappa} \phi \\
p \models_{\pi} \Diamond \phi & \equiv & \exists c \in C(p) \cup \{ \bot \}
. \, c \models_{\kappa} \phi \\
c \models_{\kappa} a(\phi ) & \equiv & c = {<} a, q {>} \: \& \: q
\models_{\pi} \phi .
\end{array} \]
The {\em assertions} over \Ell\ have the form
\[ \phi \leq_{\sigma} \psi , \;\; \phi =_{\sigma} \psi \;\;\;\; (\sigma \in \{
\pi , \kappa \} , \phi , \psi \in \Ell_{\infty \sigma}) . \]
The satisfaction relation between transition systems and assertions is
defined by:
\begin{eqnarray*}
{\cal T} \models \phi \leq_{\sigma} \psi & \equiv & \forall w \in
S_{\sigma} . \, w \models_{\sigma} \phi \;\; \Longrightarrow \;\; w
\models_{\sigma} \psi \\
{\cal T} \models \phi =_{\sigma} \psi & \equiv & \forall w \in S_{\sigma}
. \, w \models_{\sigma} \phi \;\; \Longleftrightarrow \;\; w \models_{\sigma}
\psi .
\end{eqnarray*}
\[ ( \sigma \in \{ \pi , \kappa \} , S_{\pi} = {\rm Proc}, S_{\kappa} = {\rm Cap} ). \]
This is extended to a class of transition systems {\bf C} by:
\[ {\bf C} \models A \;\; \equiv \;\; \forall {\cal T} \in {\bf C} . \, {\cal T}
\models A . \]
If {\bf C} is the class of all transition systems, we simply write $\models
A$.

\subsection*{A Proof System For $\Ellinfty$}
Firstly, we define a predicate $( \cdot ){\converges}$ on \Ellinfty\ :
\[ \begin{array}{lll}
(\bigwedge_{i \in I} \phi_{i} ) {\converges} & \equiv & \exists i \in I . \,
\phi_{i} {\converges} \\
(\bigwedge_{i \in I} \phi_{i} ) {\converges} & \equiv & \forall i \in I . \,
\phi_{i} {\converges} \\
a(\phi ) {\converges} & \equiv & {\sf true} \\
(\Box \phi ) {\converges} & \equiv & \phi {\converges} \\
(\Diamond \phi ) {\converges} & \equiv & \phi {\converges} .
\end{array} \]
Intuitively, $\phi {\converges}$ means that at least the completely
undefined process does {\em not} satisfy $\phi$ (i.e. $\phi \not= \true$).
We will use it to restrict one of our axiom schemes.

We now present a proof system for assertions over \Ellinfty\ .
Sort subscripts are omitted.

\subsection*{Logical Axioms}
Exactly as in Chapter 4, except that the restriction to finite index sets on
conjunctions and disjunctions is lifted.

\subsection*{Modal Axioms}
\begin{center}
\[ (a-{\leq}) \;\;\;  \frac{\phi \leq \psi}{a(\phi ) \leq a(\psi )} \]
\[ (a-{\wedge})(i) \;\;\; a(\bigwedge_{i \in I} \phi_{i} ) = \bigwedge_{i \in
I} a(\phi_{i} ) \;\;\;\; (I \not= \varnothing ) \]
\[ (a-{\wedge})(ii) \;\;\; a(\phi ) \wedge b(\psi ) = \false \;\;\;\; (a \not=
b) \]
\[ (a-{\vee}) \;\;\; a(\bigvee_{i \in I}\phi_{i} ) = \bigvee_{i \in I}
a(\phi_{i})  \]
\[ ({\Box}-{\leq}) \;\;\; \frac{\phi \leq \psi}{\Box \phi \leq \Box \psi} \]
\[ ({\Box}-{\wedge}) \;\;\; \Box \bigwedge_{i \in I} \phi_{i} =
\bigwedge_{i \in I} \Box \phi_{i} \]
\[ ({\Diamond}-{\leq}) \;\;\;  \frac{\phi \leq \psi}{\Diamond \phi \leq
\Diamond \psi} \]
\[ ({\Diamond}-{\vee}) \;\;\; \Diamond \bigvee_{i \in I} \phi_{i} =
\bigvee_{i \in I} \Diamond \phi_{i}  \]
\[ ({\Box}-{\vee}) \;\;\; \Box (\phi \vee \psi ) \leq \Box \phi \vee
\Diamond \psi  \]
\[ ({\Diamond}-{\wedge}) \;\;\;  \Box \phi \wedge \Diamond \psi \leq
\Diamond ( \phi \wedge \psi ) \;\;\;\; ( \psi {\converges}) \]
\[ ({\Diamond}-{\true}) \;\;\; \Diamond \true\ = \true . \]
\end{center}

The form of our axiomatisation follows the same pattern as that of
Chapter 4, of (the general approach exemplified by) which it is of course a
special case.
The first group of axioms and rules give the logical structure of
entailment, conjunction and disjunction.
They give (the Lindenbaum algebra of) \Ellinfty\ the structure of a (large)
{\em completely distributive lattice} \cite{Joh82}.
We then articulate the modal structure by showing how the 
constructors interact with the logical structure.
The axioms for the $a( \cdot )$ constructor correspond to those for coalesced
sum given in Chapter 4; the fact that {\em separated} sum is intended here
is reflected by the side-condition on $(a-{\wedge})(i)$.
The axioms for $\Box$ and $\Diamond$ individually correspond to those
presented for the upper and lower powerdomains in Chapter 4; however,
these two modalities interact in the Plotkin powerdomain, resulting in its
greater complexity; these interactions are expressed in logical terms by
$({\Box}-{\vee})$ and $({\Diamond}-{\wedge})$.
Our surgery on the ordering to keep a least element while adding the empty
set is reflected by the presence of $({\Diamond}-\true )$ and the side
condition on $({\Diamond}-{\wedge})$.

We write $\Ell  \vdash  A$ or just $\vdash  A$ if an assertion $A$ is
derivable from the above rules and axioms.
It will be convenient to have equational versions of $({\Box}-{\vee})$ and
$({\Diamond}-{\wedge})$, which can be obtained as theorems of \Ell\ :
\[ \begin{array}{clr}
(D1) & \vdash  \Box ( \phi \vee \psi ) = \Box \phi \vee ( \Box (\phi \vee
\psi ) \wedge \Diamond \psi )  & \\
(D2) & \vdash  \Box \phi \wedge \Diamond \psi = \Box \phi \wedge
\Diamond (\phi \wedge \psi )  & (\psi {\converges}) .
\end{array} \]

We now turn to the question of soundness for our system.
As a first step, we show that our auxiliary predicate $(){\converges}$
works as intended.

\begin{proposition}
\label{auxp}
(i)  $\forall \phi \in \Ell_{\infty \kappa}. \: \phi \converges \;\;
\Longleftrightarrow \;\;  \bot \nvDash_{\kappa} \phi$. 

\noindent (ii) $\forall \phi \in \Ell_{\infty \pi}. \: \phi \converges \;\;
\Longleftrightarrow \;\;  p \models_{\pi} \phi \; \Rightarrow \; C(p) \not=
\{ \bot \}$.
\end{proposition}

\proof\ We prove (i) and (ii) simultaneously by induction on $\phi$.
We consider the two non-trivial cases:

\noindent $\Box \phi$: Assume $(\Box \phi )\converges  \equiv  \phi
\converges$, and $p \models_{\pi} \Box \phi$.
$C(p) = \{ \bot \}$ would then imply $\bot \models_{\kappa} \phi$, but
this is impossible by the induction hypothesis.
For the converse, suppose $(\Box \phi ) \diverges$, i.e. $\phi \diverges$.
Then by induction hypothesis, $\bot \models_{\kappa} \phi$, and hence
$\Omega\ \models_{\pi} \Box \phi$ with $C(\Omega ) = \{ \bot \}$.

\noindent $\Diamond \phi$: Assume $\phi \converges$ and $p
\models_{\pi} \Diamond \phi$.
Then $\bot \nvDash_{\kappa} \phi$, and so there must be $c \in C(p) - \{
\bot \}$ with $c \models_{\kappa} \phi$.
The converse is proved by the same argument as for $\Box \phi$. \qed

\begin{theorem}[Soundness of $\Ell$]
$\vdash  A \;\; \Longrightarrow \;\; \models  A$.
\end{theorem}

\proof\ By a routine induction over proofs.
For illustration, we consider $({\Diamond}-{\wedge})$.
Assume $\psi \converges$ and $p \models_{\pi} \Box \phi \wedge
\Diamond \psi$.
Then $p \models_{\pi} \Diamond \psi$, and so by~\ref{auxp}, $C(p) \not= \{ \bot
\}$ and $\bot \nvDash_{\kappa} \psi$, and there must be $c \in C(p) - \{
\bot \}$ such that $c \models_{\kappa} \psi$.
But then $p \models_{\pi} \Box \phi$ implies that $c \models_{\kappa}
\phi$, and so $p \models_{\pi} \Diamond ( \phi \wedge \psi )$ as required.
\qed

We now turn to the finitary logic $\Ellomega$.
Henceforth we assume that ${\sf Act}$ is countable.
It is then clear that $\Ellomega$ can be made into a countable set by a
suitable choice of canonical representatives of logical equivalence
classes.

Recall that ${\sf Spec} \; \Ellomega$ is the set of {\em prime filters} over
$\Ell_{\omega \pi}$, i.e. subsets $x \subseteq \Ell_{\omega \pi}$
satisfying
\[ \begin{array}{cc}
\bullet & \phi \in x \: \& \: \vdash  \phi \leq \psi \;\; \Rightarrow \;\;
\psi \in x \\
\bullet & \true\ \in x \\
\bullet & \phi , \psi \in x \;\; \Rightarrow \;\; \phi \wedge \psi \in x \\
\bullet & \false\ \not\in x \\
\bullet & \phi \vee \psi \in x \;\; \Rightarrow \;\; \phi \in x \; \mbox{or}
\; \psi \in x .
\end{array} \]

${\sf Spec} \; \Ellomega$ is topologised by taking as basic opens
\[ U_{\phi} \; \equiv \;  \{ x \in {\sf Spec} \; \Ellomega\ : \phi \in x \}
\;\;\; (\phi \in \Ell_{\omega \pi}) , \]
or, equivalently in our context, by taking the Scott topology over the
specialisation order on ${\sf Spec} \; \Ellomega$, which is simply set
inclusion.

Our aim is to prove the following fundamental result, which ahows that
the logic $\Ellomega$ does indeed correspond exactly to the domain 
\Dom\ :

\begin{theorem}[Stone Duality]
\label{bdual}
\Dom\ and \Ellomega\ are Stone duals, i.e.
\[ \begin{array}{rl}
(i) & \Dom\ \; \cong \; {\sf Spec} \; \Ellomega \\
(ii) & K \Omega (\Dom\ ) \; \cong \; (\Ell_{\omega \pi}/{=_{\pi}},
{\leq_{\pi}}/{=_{\pi}}) .
\end{array} \]
\end{theorem}





Here $K \Omega (D)$ is the lattice of compact-open subsets of $\Dom$, while 
\[ ( {\Ell}_{\omega \pi}/{=_{\pi}}, \leq_{\pi}/{=_{\pi}}) \]
is the {\em Lindebaum algebra} of $\Ellomega$. 
Since $\Dom$ is coherent, (i) and (ii) are indeed equivalent (\cite{Joh82}).

The Stone Duality Theorem is entirely analogous to Theorem \ref{sdual}, and our proof strategy is identical.
However, some of the technical details are more complex; in particular, the syntactic identification of primes is less obvious than for Scott domains, since primes are no longer preserved under meets.

We begin by defining a normal form for $\Ellomega$.
\begin{definition}
{\rm (i) $\phi$ is in {\em strong disjunctive normal form} (SDNF) if it has the form $\bigvee_{i \in I} \phi_{i}$, where each $\phi_{i}$ is in {\em prime normal form} (PNF).

\noindent (ii) $\phi$ is in PNF if it has one of the forms
\begin{itemize}
\item $\bigwedge_{i \in I} \Diamond a_{i} (\phi_{i})$, where each $\phi_{i}$ is in PNF.
\item $\Box \bigvee_{i \in I} a_{i} ( \phi_{i}) \; \wedge \; \bigwedge_{j \in J} \Diamond b_{j}(\psi_{j})$, where
\begin{enumerate}
\item Each $\phi_{i}$ and $\psi_{j}$ is in PNF.
\item $\forall i \in I. \, \exists j \in J. \, \vdash  b_{j}(\psi_{j}) \leq 
a_{i}(\phi_{i})$.
\item $\forall j \in J. \, \exists i \in I. \, \vdash  b_{j}(\psi_{j}) \leq 
a_{i}(\phi_{i})$.
\end{enumerate}
\end{itemize}}
\end{definition}
We call (2) and (3) the {\em convexity conditions} (note the resemblance to the Egli--Milner ordering).

The combinatorics are concentrated in the following 
\begin{theorem}[SDNF]
\label{SDNF}
For every $\phi \in \Ell_{\omega \pi}$, there is (effectively) a $\psi$ in SDNF such that 
\[ \vdash  \phi =_{\pi} \psi . \]
\end{theorem}

\proof\ By induction on ${\sf md}(\phi )$.
The idea is to form a sequence of ``transformations''
\[ \phi \equiv \phi_{0} \rightsquigarrow \phi_{1} \rightsquigarrow \cdots 
\rightsquigarrow \phi_{n} \]
such that
\[ \begin{array}{ll}
(1) & \vdash  \phi_{i} = \phi_{i+1} \;\;\; (0 \leq i < n) \\
(2) & {\sf md}(\phi_{i+1}) \leq {\sf md}(\phi_{i}) \;\;\; (0 \leq i < n) \\
(3) & \phi_{n} \; \mbox{is in SDNF.}
\end{array} \]
(Condition (2) is needed to keep the induction going.)
To keep the notation bearable, we shall omit indices in conjunctions and disjunctions, writing e.g. $\bigvee \{ \phi \}$.

Firstly, using the distributive lattice laws we can transform $\phi_{0}$ into
\begin{equation}  
\label{SDNF1}
\bigvee \{ \bigwedge \{ \Box \bigwedge \{ \bigvee \{ a(\phi ) \} \} \} \; \wedge \; \bigwedge \{ \Diamond \bigwedge \{ \bigvee \{ b(\psi ) \} \} \} \}
\end{equation}
Using $({\Box}-{\wedge})$ in the outwards direction for each $\Box$-conjunct in \ref{SDNF1}, and the distributive law and then $({\Diamond}-{\vee})$, followed by the distributive law again, in each $\Diamond$-conjunct, we otain
\begin{equation} 
\label{SDNF2}
\bigvee \{ \bigwedge \{ \Box \bigvee \{ a(\phi ) \} \} \; \wedge \; \bigwedge \{ \Diamond \bigwedge \{ b(\psi ) \} \} \}
\end{equation}
Now for each non-empty conjunction
\[ \bigwedge \{ \Box \bigvee \{ a(\phi ) \} \} \]
in \ref{SDNF2}, we can use $({\Box}-{\wedge})$, the distributive law, and $(a-{\wedge})$ $(i)$ or $(ii)$; similarly, inside each $\Diamond \bigwedge \{ b(\psi ) \}$ we can use $({\Diamond}-{\true})$ if the conjunction is empty, and otherwise $(b-{\wedge})$ $(i)$ or $(ii)$ (with further applications of $({\Diamond}-{\vee})$ and the distributive laws as in the previous step if $(b-{\wedge})(ii)$ is applicable), to obtain
\begin{equation} 
\label{SDNF3}
\bigvee \{ \theta \}
\end{equation}
where each $\theta$ is in one of the forms
\begin{equation} 
\label{SDNF4}
\bigwedge \{ \Diamond b(\psi ) \}
\end{equation}
or
\begin{equation} 
\label{SDNF5}
\Box \bigvee \{ a(\phi ) \} \; \wedge \; \bigwedge \{ \Diamond b(\psi ) \}
\end{equation}
Since we have not increased modal depth in obtaining \ref{SDNF3}, we can apply the inductive hypothesis to each $\phi$ and $\psi$ to obtain $\bigvee \{ \phi' \}$, $\bigvee \{ \psi' \}$ with each $\phi'$ and $\psi'$ in PNF.
Using  $(a-{\vee})$, $({\Diamond}-{\vee})$ and the distributive laws, we can thus obtain a formula of the same form as \ref{SDNF3}, in which each $\phi$ and $\psi$ in \ref{SDNF4} and \ref{SDNF5} is in PNF.

At this point, our formula \ref{SDNF3} can only fail to be in SDNF because of disjuncts \ref{SDNF5} which do not satisfy the convexity conditions
\begin{itemize}
\item For each $a(\phi )$, for some $b(\psi )$: $\vdash  b(\psi ) \leq a(\phi )$.
\item For each $b(\psi )$, for some $a(\phi )$: $\vdash   b(\psi ) \leq a(\phi )$.
\end{itemize}

Our strategy is to remove any failures of these two conditions, using our derived equations $(D1)$ and $(D2)$ respectively. We begin with the first condition.
We argue by induction on $(m, n)$ in the lexicographic ordering on $\omega \times \omega$, where:
\begin{itemize}
\item $m$ is the maximum number of $a(\phi )$ occurring in one of the disjuncts \ref{SDNF5} of our formula \ref{SDNF3} such that there is no $b(\psi )$ with $\vdash  b(\psi ) \leq a(\phi )$.
\item $n$ is the number of disjuncts attaining this maximum.
\end{itemize}
If $m = 0$, there is nothing to prove.
Otherwise, choose such an $a(\phi )$ in one of the maximal disjuncts.
We can apply $(D1)$ to 
\[ \Box \bigvee \{ a' (\phi' ) \} \; \vee \; a(\phi ) \]
to obtain
\begin{equation} 
\label{SDNF6}
\Box \bigvee \{ a' ( \phi' ) \} \;\; \vee \;\; [ \Box ( \bigvee \{ a' ( \phi' ) \} \vee a(\phi )) \wedge \Diamond a(\phi ) ]
\end{equation}
We can then use the distributive law to obtain a new formula of the form \ref{SDNF3} to which the inner induction hypothesis can be applied, since the first disjunct in \ref{SDNF6} has jettisoned $a(\phi )$, while the second disjunct evidently contains a $\Diamond b(\psi )$ such that $\vdash  b(\psi ) \leq a(\phi )$, namely $a(\phi )$ itself.

The final stage is to remove failures of the second condition.
We argue by induction in the same way as for the previous stage.
Suppose we are given a $b(\psi )$ in \ref{SDNF5} with no $a(\phi )$ such that $\vdash  b(\psi ) \leq a(\phi )$.
Firstly, we note that
$\psi \diverges$  implies $\vdash  \psi = \true$, 
which is easily proved by induction on $\psi$.
Hence if $\psi \diverges$, we can use $({\Diamond}-{\true})$ to eliminate the conjunct $\Diamond b(\psi )$.
Otherwise, we can use $(D2)$ to obtain
\begin{equation} 
\label{SDNF7}
\Box \bigvee \{ a(\phi ) \} \; \wedge \; \Diamond [ b(\psi ) \wedge \bigvee \{ a(\phi ) \} ] \; \wedge \; \bigwedge \{ \Diamond b' (\psi' ) \}
\end{equation}
Now we can use the distributive law inside the second main conjunct in \ref{SDNF7}, followed by $(a-{\wedge})$, $({\Diamond}-{\vee})$, and the distributive law again.
In this way, the disjunct \ref{SDNF7} of our main formula is replaced by the disjunction of all those formulae
\begin{equation} 
\label{SDNF8}
\Box \bigvee \{ a(\phi ) \} \; \wedge \; \Diamond b(\phi' \wedge \psi ) \; \wedge \; \bigwedge \{ \Diamond b' ( \psi' ) \}
\end{equation}
for $a'(\phi' ) \in \{ a(\phi ) \}$ with $a' = b$.
For each such $\phi' \wedge \psi$, we can apply the outer induction hypothesis to obtain $\bigvee \{ \theta' \}$ with each $\theta'$ in PNF.
Applying $(b-{\vee})$, $({\Diamond}-{\vee})$ and the distributive laws as before, we obtain disjuncts of the form
\begin{equation} 
\label{SDNF9}
\Box \bigvee \{ a(\phi ) \} \; \wedge \; \Diamond b(\theta' ) \; \wedge \;  \bigwedge \{ \Diamond b' ( \psi' ) \}
\end{equation}
Since
\[ \vdash \; \theta' \leq \bigvee \{ \theta' \} = \phi' \wedge \psi \leq \phi' , \]
we can apply the inner induction hypothesis to \ref{SDNF9}.
This completes the process of transforming $\phi$ into SDNF. \qed

We shall now prove that formulae in PNF denote primes in $K \Omega (\Dom )$.
\begin{proposition}
\label{psoun}
For all $\phi$ in PNF there exsists $k(\phi ) \in {\cal K}(\Dom )$ such that:
\[ \forall d \in \Dom . \; d \models \phi \;\; \Longleftrightarrow \;\; k(\phi ) \sqsubseteq d . \]
\end{proposition}

\proof\ We define $k(\phi )$ (which must clearly be unique) by induction on $\phi$:
\[ \bullet \;\; k(\bigwedge_{i \in I} \Diamond a_{i}(\phi_{i} )) \; \equiv \; \biguplus_{i \in I} \lsing \ltuple a_{i} , k(\phi_{i}) \rtuple \rsing \uplus \lsing \bot \rsing \]
\[ \bullet \;\; k(\Box \bigvee_{i \in I} a_{i}(\phi_{i}) \; \wedge \; \bigwedge_{j \in J} \Diamond b_{j}(\psi_{j})) \; \equiv \]
\[  \;\;\;\; \biguplus_{i \in I} \lsing \ltuple a_{i}, k(\phi_{i} ) \rtuple \rsing \; \uplus \; \biguplus_{j \in J} \lsing \ltuple b_{j}, k(\psi_{j}) \rtuple \rsing . \]

We shall prove the proposition by induction on $\phi$.
Note that in the statement of the proposition, we are viewing $\Dom$ as a transition system, according to~\ref{ifs}.
With our convention of eliding the isomorphisms between $\Dom$ and $P^{0} [ \sum_{a \in {\sf Act}} \Dom ]$, we have:
$d = C(d)$, $(d \in \Dom )$. 

\noindent Case 1:  $\phi \equiv \bigwedge_{i \in I} \Diamond a_{i}(\phi_{i} )$.
\[ \begin{array}{ll}
\bullet & d \models \bigwedge_{i \in I} \Diamond a_{i}(\phi_{i}) \\
\Longleftrightarrow & \forall i \in I . \, \exists \ltuple a_{i}, d_{i} \rtuple \in d . \, d_{i} \models \phi_{i} \\
\Longleftrightarrow & \forall i \in I . \, \exists \ltuple a_{i}, d_{i} \rtuple \in d . \, k(\phi_{i}) \sqsubseteq d_{i} \;\; \mbox{by induction hypothesis} \\
\Longleftrightarrow & k(\phi ) \sqsubseteq d .
\end{array} \]

\noindent Case 2: $\phi  \equiv  \Box \bigvee_{i \in I} a_{i}(\phi_{i})  \wedge  \bigwedge_{j \in J} \Diamond b_{j}(\psi_{j})$.
Let $\Phi = \{ a_{i}(\phi_{i}) : i \in I \} \cup \{ b_{j}(\psi_{j}) : j \in J \}$.
\[ \begin{array}{ll}
\bullet & d \models \phi \\
\Longleftrightarrow & \forall \ltuple a, d' \rtuple \in d. \, \exists i \in I . \, a = a_{i} \: \& \: d' \models \phi_{i} \\
& \& \: \bot \not\in d \: \& \: \forall j \in J . \, \exists \ltuple b_{j}, d_{j} \rtuple \in d . \, d_{j} \models \psi_{j} \\
\Longleftrightarrow & \forall \ltuple a, d' \rtuple \in d. \, \exists  a(\theta ) \in \Phi . \, d' \models \theta \\
& \& \: \bot \not\in d \: \& \: \forall a(\theta ) \in \Phi . \, \exists \ltuple a, d' \rtuple \in d . \, d \models \theta \\
& \mbox{by the convexity conditions and the Soundness Theorem,} \\
\Longleftrightarrow & k(\phi ) \sqsubseteq d, \; \mbox{by induction hypothesis.} \;\;\; \qed
\end{array} \]

\begin{theorem}[Prime Completeness]
For all $\phi$, $\phi'$ in PNF:
\[ \Dom  \models  \phi \leq \phi' \;\; \Longrightarrow \;\; \Ell  \vdash  \phi \leq \phi' . \]
\end{theorem}

\proof\ By 4.7,
\[ \Dom  \models  \phi \leq \phi' \;\; \Longleftrightarrow \;\; k(\phi' ) \sqsubseteq k(\phi ) . \]
Suppose then that $k(\phi' ) \sqsubseteq k(\phi )$.
We argue by induction on $\phi$.
There are a number of cases, according to the forms of $\phi$ and $\phi'$.
We consider the case
\begin{eqnarray*}
\phi & \equiv & \Box \bigvee_{i \in I} a_{i}(\phi_{i}) \; \wedge \; \bigwedge_{j \in J} \Diamond b_{j}(\psi_{j}) , \\
\phi' & \equiv & \Box \bigvee_{i' \in I'} a_{i'}(\phi_{i'}) \; \wedge \; \bigwedge_{j' \in J'} \Diamond b_{j'}(\psi_{j'}) .
\end{eqnarray*}
\[ \begin{array}{ll}
\bullet & k(\phi' ) \sqsubseteq k(\phi ) \\
\Longleftrightarrow & \forall j' \in J' . \, \exists j \in J . \, b_{j} = b_{j'} \: \& \: k(\psi_{j'}) \sqsubseteq k(\psi_{j}) \\
& \mbox{}\& \; \forall i \in I . \, \exists i' \in I' . \, a_{i} = a_{i'} \: \& \: k(\phi_{i'}) \sqsubseteq k(\phi_{i}), \\
& \mbox{by the convexity conditions, Soundness, and \ref{psoun}} \\
\Longrightarrow & \forall j' \in J' . \, \exists j \in J . \, \vdash  b_{j}(\psi_{j}) \leq b_{j'}(\psi_{j'}) \\
& \mbox{} \& \; \forall i \in I . \, \exists i' \in I' . \, \vdash \: a_{i}(\phi_{i}) \leq a_{i'}(\phi_{i'}), \\
& \mbox{by the induction hypothesis,} \\
\Longrightarrow & \vdash \: \phi \leq \phi' . \;\;\; \qed
\end{array} \]

We can now use the same arguments as in Chapter~3 T7 to prove

\begin{theorem}[Completeness]
For all $\phi , \psi \in \Ellomega$:
\[ \Dom  \models  \phi \leq \psi \;\; \Longrightarrow \;\; \Ellomega  \vdash  \phi \leq \psi . \]
\end{theorem}

We now establish a converse to \ref{psoun}.
\begin{theorem}[Definability]
For all $d \in {\cal K}(\Dom )$, for some $\phi$ in PNF, $k(\phi ) = d$.
\end{theorem}

\proof\ We define $\phi (d)$ by induction on the construction of $d$ according to~\ref{felts}:
\begin{eqnarray*}
\phi (\biguplus_{i \in I} \lsing \ltuple a_{i}, d_{i} \rtuple \rsing \; \uplus \; \lsing \bot \rsing ) & \equiv & \bigwedge_{i \in I} \Diamond a_{i}(\phi (d_{i} )) \\
\phi ( \biguplus_{i \in I} \lsing \ltuple a_{i}, d_{i} \rtuple \rsing ) & \equiv & \Box \bigvee_{i \in I} a_{i} (\phi (d_{i} )) \; \wedge \; \bigwedge_{i \in I} \Diamond a_{i}(\phi (d_{i})).
\end{eqnarray*} 
Note in particular that $\phi (\emptyset ) = \Box \false$.
It is easily verified that $\phi (d)$ is in PNF and that $k(\phi (d)) = d$. \qed

The Duality Theorem is an immediate consequence of Soundness, Completeness and Definability, just as in Chapter~3 T8.

Combining Soundness and Completeness we obtain
\begin{theorem}[Completeness for $\Ellomega$]
Let {\bf C} be any class of transition systems containing $\Dom$.
Then for $\phi , \psi \in \Ellomega$:
\[ {\bf C}  \models \phi \leq \psi \;\; \Longleftrightarrow \;\; \Dom  \models  \phi \leq \psi \;\; \Longleftrightarrow \;\; \Ell  \vdash  \phi \leq \psi . \]
\end{theorem}


