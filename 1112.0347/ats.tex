\section{Applicative Transition Systems}
The theory $\lambda \ell$ defined in the previous section was derived from 
a particular operational model, the transition system $(\Lambda^{0}, \Converges )$. What is the general concept of which this is an example?
\begin{definition}
{\rm A {\em quasi-applicative transition system} is a structure $(A, ev)$ where}
\[ ev : A \rightharpoonup (A \rightarrow A) . \]
\end{definition}
{\bf Notations:}
\[ \begin{array}{rrcl}
(i) & a \Converges f & \equiv & a \in {\sf dom} \, ev \; \& \; ev(a) = f \\
(ii) & a \Converges & \equiv & a \in {\sf dom} \, ev \\
(iii) & a \Diverges & \equiv & a \not\in {\sf dom} \, ev
\end{array} \]
\begin{definition}[Applicative Bisimulation]
\label{apsim}
{\rm Let $(A, ev)$ be a quasi-ats. We define
\[ F : Rel(A) \rightarrow Rel(A) \]
by
\[ F(R) = \{ (a,b) : a \Converges f \;\; \Longrightarrow \;\; b \Converges g \; \& \; \forall c \in A . \, f(c) R g(c) \} . \]
Then $R \in Rel(A)$ is an {\em applicative bisimulation} iff $R \subseteq F(R)$; and ${\preord^{B}} \in Rel(A)$ is defined by
\[ a \preord^{B} b \; \equiv \; a R b \; \mbox{for some applicative bisimulation} \; R. \]}
\end{definition}
Thus ${\preord^{B}} = \bigcup \{ R \in Rel(A) : R \subseteq F(R) \}$, and hence is the maximal fixpoint of the monotone function $F$. 
Since the relation $\Converges$ is a partial function, it is easily shown that the closure ordinal of $F$ is $\leq \omega$, and we can thus describe $\preord^{B}$ more explicitly as follows:
\[ \bullet \;\; a \preord^{B} b \;\; \equiv \;\; \forall k \in \omega . \, a \preord_{k} b \]
\[ \bullet \;\; a \preord_{0} b \;\; {\rm always} \]
\[ \bullet \;\; a \preord_{k+1} b \;\; \equiv \;\; a \Converges f \;\; \Longrightarrow \;\; b \Converges g \; \& \; \forall c \in A. \, f(c) \preord_{k} g(c) \]
\[ \bullet \;\; a \bisim^{B} b \;\; \equiv \;\; a \preord^{B} b \; \& \; b \preord^{B} a . \]
It is easily seen that $\preord^{B}$, and also each $\preord_{k}$, is a preorder; $\bisim^{B}$ is therefore an equivalence.

We now come to our main definition.
\begin{definition}
{\rm An {\em applicative transition system} (ats)  is a quasi-ats $(A, ev)$ satisfying:}
\[ \forall a, b, c \in A. \, a \Converges f \; \& \; b \preord^{B} c \; \Rightarrow \; f(b) \preord^{B} f(c) . \]
\end{definition}
An ats has a well-defined quotient $(A/{\bisim^{B}}, ev/{\bisim^{B}})$, where
\[ ev/{\bisim^{B}} ([a]) = \left\{ \begin{array}{ll}
[b] \mapsto [f(b)], & a \Converges f \\
\mbox{undefined} & \mbox{otherwise.}
\end{array}
\right. \]

The reader should now refresh her memory of such notions as {\em applicative structure, combinatory algebra {\rm and} lambda model} from \cite[Chapter 5]{Bar}.
\begin{definition}
{\rm A {\em quasi-applicative structure with divergence} is a structure $(A, \appl\ , \Diverges )$ such that $(A, \appl\ )$ is an applicative structure, and $\Diverges \subseteq A$ is a divergence predicate satisfying}
\[ x \Diverges \;\; \Longrightarrow \;\; (x\appl\ y)\Diverges . \]
\end{definition}
Given $(A, \appl\ , \Diverges )$, we can define
\[ a \preord^{A} b \;\; \equiv \;\; a \Converges \;\; \Longrightarrow \;\; 
b \Converges \: \& \: \forall c \in A. \, a\appl\ c \preord^{A} b\appl\ c \]
as the maximal fixpoint of a monotone function along identical lines to \ref{apsim}.

Applicative transition systems and applicative structures with divergence are not quite equivalent, but are sufficiently so for our purposes:
\begin{proposition}
\label{appstruct}
Given an ats ${\cal B} = (A, ev)$, we define ${\cal A} = (A, \appl\ , \Diverges)$ by
\[ a\appl\ b \; \equiv \; \left\{ \begin{array}{ll}
a, & a \Diverges \\
f(b) & a \Converges f .
\end{array} \right. \]
Then 
\[ a \preord^{A} b \;\; \Longleftrightarrow \;\; a \preord^{B} b , \]
and moreover we can recover ${\cal B}$ from ${\cal A}$ by
\[ ev(a) =  \left\{ \begin{array}{ll}
b \mapsto a\appl\ b, & a \Converges \\
\mbox{undefined} & \mbox{otherwise.}
\end{array} \right. \]
Furthermore, $\appl\ $ is compatible with $\preord^{B}$, i.e.
\[ a_{i} \preord^{B} b_{i} \; (i = 1,2) \; \Rightarrow \; a_{1}\appl\ a_{2} \preord^{B} b_{1}\appl\ b_{2} . \;\;\; \qed \]
\end{proposition}
We now turn to a language for talking about these structures.
\begin{definition}
{\rm We assume a fixed set of variables {\sf Var}. Given an applicative structure ${\cal A} = (A, \appl\ )$, we define $CL({\cal A})$, the {\em combinatory terms over ${\cal A}$}, by
\[ \bullet \;\; {\sf Var} \subseteq CL({\cal A}) \]
\[ \bullet \;\; \{ c_{a} : a \in A \} \subseteq CL({\cal A}) \]
\[ \bullet \;\; M, N \in CL({\cal A}) \; \Rightarrow \; MN \in CL({\cal A}) . \]
Let $Env({\cal A}) \; \equiv \; {\sf Var} \rightarrow A$. Then the {\em interpretation function}
\[ \lsem \rsem^{{\cal A}} : CL({\cal A}) \rightarrow Env({\cal A}) \rightarrow A \]
is defined by:}
\begin{eqnarray*}
\lsem x \rsem^{{\cal A}}_{\rho} & = & \rho x \\
\lsem c_{a} \rsem^{{\cal A}}_{\rho} & = & a \\
\lsem MN \rsem^{{\cal A}}_{\rho} & = & (\lsem M \rsem^{{\cal A}}_{\rho})\appl\ (\lsem N \rsem^{{\cal A}}_{\rho}) .
\end{eqnarray*}
\end{definition}
Given an ats ${\cal A} = (A, ev)$, with derived applicative structure $(A, \appl\ )$, the satisfaction relation between ${\cal A}$ and atomic formulae over $CL({\cal A})$, of the forms
\[ M \sqsubseteq N, \;\; M = N, \;\; M \Converges \;\; M \Diverges \]
is defined by:
\begin{eqnarray*}
{\cal A}, \rho \models M \sqsubseteq N & \equiv & \lsem M \rsem^{{\cal A}}_{\rho} \preord^{B} \lsem N \rsem^{{\cal A}}_{\rho} \\
{\cal A}, \rho \models  M = N & \equiv & \lsem M \rsem^{{\cal A}}_{\rho} \bisim^{B} \lsem N \rsem^{{\cal A}}_{\rho} \\
{\cal A}, \rho \models M \Converges & \equiv & \lsem M \rsem^{{\cal A}}_{\rho} \Converges \\
{\cal A}, \rho \models M \Diverges & \equiv & \lsem M \rsem^{{\cal A}}_{\rho} \Diverges 
\end{eqnarray*}
while
\[ {\cal A} \models \phi \; \equiv \; \forall \rho \in Env({\cal A}). \, {\cal A}, \rho \models \phi . \]
This is extended to first-order formulae in the usual way.

Note that equality in $CL({\cal A})$ is being interpreted by bisimulation in ${\cal A}$. 
We could have retained the standard notion of interpretation as in \cite{Bar} by working in the quotient structure $(A/{\bisim^{B}}, \appl\  /{\bisim^{B}})$. 
This is equivalent, in the sense that the same sentences are satisfied.
\begin{definition}
{\rm A {\em lambda transition system} (lts) is a structure $(A, ev, k, s)$, where:
\begin{itemize}
\item $(A, ev)$ is an ats
\item $k, s \in A$, and $A$ satisfies the following axioms 
(writing {\bf K, S} for $c_{k}, c_{s}$):
\[ \bullet \;\; {\bf K} \Converges , \;\;\; {\bf K} x \Converges \]
\[ \bullet \;\; {\bf K} x y = x \]
\[ \bullet \;\; {\bf S} \Converges , \;\;\; {\bf S} x \Converges , \;\;\; {\bf S} x y \Converges \]
\[ \bullet \;\; {\bf S} x y z = (x z)(y z) \]
\end{itemize}}
\end{definition}

We now check that these definitions do indeed capture our original example.
\subsection*{Example}
{\rm We define $\ell = (\Lambda^{0}, ev)$, where
\[ ev(M) = \left\{ \begin{array}{ll}
P \mapsto N[P/x] , & M \Converges \lambda x . N \\
\mbox{undefined} & \mbox{otherwise.}
\end{array} \right. \]
$\ell$ is indeed an ats by $\ref{lazycong}(iv)$. Moreover, it is an lts via the definitions
\[ k \; \equiv \; \lambda x . \lambda y . x \]
\[ s \; \equiv \; \lambda x . \lambda y . \lambda z . (x z) (y z) . \]

We now see how to interpret $\lambda$-terms in any lts.
\begin{definition}
{\rm Given an lts ${\cal A}$, we define $\Lambda ({\cal A})$, the $\lambda$-terms over ${\cal A}$, by the same clauses as for $CL({\cal A})$, plus the additional one:
\[ \bullet \;\; x \in {\sf Var}, M \in \Lambda ({\cal A}) \; \Rightarrow \; \lambda x . M \in \Lambda({\cal A}) . \]}
\end{definition}
We define a translation
\[ ( \cdot )_{CL} : \Lambda ({\cal A}) \rightarrow CL({\cal A}) \]
by
\begin{eqnarray*}
(x)_{CL} & \equiv & x \\
(c_{a})_{CL} & \equiv & c_{a} \\
(MN)_{CL} & \equiv & (M)_{CL}(N)_{CL} \\
(\lambda x . M)_{CL} & \equiv & \lambda^{\ast} x . (M)_{CL}
\end{eqnarray*}
where
\begin{eqnarray*}
\lambda^{\ast} x . x & \equiv & {\bf I} \; ( \equiv {\bf SKK}) \\
\lambda^{\ast} x . M & \equiv & {\bf K} M \;\; (x \not\in FV(M)) \\
\lambda^{\ast} x . MN & \equiv & {\bf S} (\lambda^{\ast} x . M) (\lambda^{\ast} x . N) .
\end{eqnarray*}
We now extend $\lsem \cdot \rsem$ to $\Lambda ({\cal A})$ by:
\[ \lsem M \rsem^{{\cal A}}_{\rho} \; \equiv \; \lsem (M)_{CL}\rsem^{{\cal A}}_{\rho} . \]
\begin{definition}
{\rm We define two sets of formulae over $\Lambda$:
\begin{itemize}
\item {\em Atomic formulae:}
\begin{eqnarray*}
{\sf AF} & \equiv & \{ M \sqsubseteq N, \: M = N, \: M \Diverges , \: N \Diverges \; | \;   M, N \in \Lambda \}
\end{eqnarray*} 
\item {\em Conditional formulae:}
\begin{eqnarray*}
{\sf CF} & \equiv & \{ \bigwedge_{i \in I} M_{i} \Converges \wedge \bigwedge_{j \in J} N_{j} \Diverges \Rightarrow F : F \in {\sf AF}, M_{i}, N_{i} \in \Lambda, \\
& &  I, J \; {\rm finite} \}
\end{eqnarray*} 
\end{itemize}
Note that, taking $I = J = \varnothing$, ${\sf AF} \subseteq {\sf CF}$. 
Now given an lts ${\cal A}$, ${\Im}({\cal A})$, the {\em theory} of ${\cal A}$, is defined by
\[  {\Im}({\cal A}) \; \equiv \; \{ C \in {\sf CF} : {\cal A} \models C \} . \]
We also write ${\Im}^{0}({\cal A})$ for the restriction of 
${\Im}({\cal A})$ to closed formulae; 
and given a set {\sf Con} of constants and an interpretation 
${\sf Con} \rightarrow A$, we write ${\Im}({\cal A}, {\sf Con})$ 
for the theory of conditional formulae built from terms in $\Lambda ({\sf Con})$.}
\end{definition}

\noindent {\bf Example (continued)}. 
We set $\lambda \ell = {\Im}(\ell )$. 
This is consistent with our usage in the previous section. 
We saw there that $\lambda \ell$ satisfied much stronger properties than the 
simple combinatory algebra axioms in our definition of lts. 
It might be expected that these would fail for general lts; 
but this is to overlook the powerful extensionality principle built into our 
definition of the theory of an ats through the applicative bisimulation 
relation.
\begin{proposition}
\label{extprop}
Let ${\cal A}$ be an ats. The axiom scheme of {\rm conditional extensionality} over $CL({\cal A})$:
\[ (\Converges {\rm ext}) \;\;\; M\Converges \: \& \: N \Converges \; \Rightarrow \; ([\forall x . M x = N x] \; \Rightarrow \; M = N) \]
\begin{flushright}
$(x \not\in FV(M) \cup FV(N))$
\end{flushright}
is valid in ${\cal A}$.
\end{proposition}

\proof\ Let $\rho \in Env({\cal A})$.
\[ {\cal A}, \rho \; \models \; M \Converges \: \& \: N \Converges \: \& \: \forall x . \, M x = N x \]
\[ \Rightarrow \;\; \lsem M \rsem^{{\cal A}}_{\rho} \Converges \: \& \:  
\lsem N \rsem^{{\cal A}}_{\rho} \Converges \: \& \: \forall a \in A . \, \lsem M \rsem^{{\cal A}}_{\rho} \appl\  a = \lsem N \rsem^{{\cal A}}_{\rho} \appl\  a \]
\begin{flushright}
since $x \not\in FV(M) \cup FV(N)$
\end{flushright}
\[ \Rightarrow \;\; \lsem M \rsem^{{\cal A}}_{\rho} \bisim^{A} \lsem N \rsem^{{\cal A}}_{\rho} \]
\[ \Rightarrow \;\; \lsem M \rsem^{{\cal A}}_{\rho} \bisim^{B} \lsem N \rsem^{{\cal A}}_{\rho} \]
\[ \Rightarrow \;\; {\cal A}, {\rho} \; \models \; M = N . \;\;\; \qed \]
Using this Proposition, we can now generalise most of \ref{lazyprop} to an arbitrary lts.
\begin{theorem}
\label{ltsprop}
Let ${\cal A} = (A, ev, k, s)$ be an lts. Then \\
(i) $(A, ., k, s)$ is a lambda model, and hence $\lambda \subseteq {\Im}({\cal A})$. \\
(ii) ${\cal A}$ satisfies the conditional $\eta$ axiom scheme:
\[ (\Converges \eta ) \;\; M \Converges \; \Rightarrow \; \lambda x . M x = M \;\;\;\; (x \not\in FV(M)) \]
(iii) For all $M \in \Lambda^{0}$:
\[ \lambda \ell \; \vdash \; M \Converges \;\; \Rightarrow \;\; {\cal A} \; \models \; M \Converges \]
(iv) ${\cal A} \; \models \; x \sqsubseteq {\bf YK}$. \\
(v) $\sqsubseteq$ is a precongruence in ${\Im}({\cal A})$.
\end{theorem}

\proof\ (i). Firstly, by the very definition of lts, ${\cal A}$ is a combinatory algebra. We now use the following result due to Meyer and Scott, cited from \cite[Theorem 5.6.3, p.\ 117]{Bar}:
\begin{itemize}
\item Let ${\cal M}$ be a combinatory algebra. Define
\[ {\bf 1} \: \equiv \: {\bf 1}_{1} \: \equiv \: {\bf S(KI)}, \]
\[ {\bf 1}_{k+1} \: \equiv \: {\bf S(K} {\bf 1}_{{\rm k}}) . \]
Then ${\cal M}$ is a lambda model iff it satisfies
\[ \begin{array}{rl}
\mbox{(I)} & \forall x . \, a x = b x \; \Rightarrow \; {\bf 1} a = {\bf 1} b \\
\mbox{(II)} & {\bf 1}_{2} {\bf K} = {\bf K} \\
\mbox{(III)} & {\bf 1}_{3} {\bf S} = {\bf S} .
\end{array} \]
\end{itemize}
Thus it is sufficient to check that ${\cal A}$ satisfies (I)--(III). 
For (I), note firstly that ${\cal A} \; \models \; {\bf 1} a 
\Converges 
x \: \& \: {\bf 1} b \Converges$ by the convergence axioms for an lts. Hence we can apply \ref{extprop} to obtain
\[ {\cal A} \; \models \; [\forall x . \, {\bf 1} a x = {\bf 1} b x] \; \Rightarrow \; {\bf 1} a = {\bf 1} b . \]
We now assume $\forall x . \, a x =  b x$ and prove $\forall x . \, {\bf 1} a x = {\bf 1} b x$:
\begin{eqnarray*}
{\bf 1} a x & = & {\bf S(KI)} a x \\
& = & {\bf (KI)} x (a x) \\
& = & {\bf (KI)} x (b x) \\
& = & {\bf S (KI)} b x \\
& = & {\bf 1} b x .
\end{eqnarray*}
(II) and (III) are proved similarly.

\noindent (ii). Let $\rho \in Env({\cal A})$, and assume ${\cal A}, \rho \; \models \; M \Converges$. We must prove that
\[ {\cal A}, \rho \; \models \; \lambda x . M x = M . \]
Firstly, note that for any abstraction $\lambda z . P$,
\[ {\cal A} \; \models \; \lambda z . P \Converges \]
by the definition of $\lambda^{\ast} z . P$ and the convergence axioms for an lts. Thus since $x \not\in FV(M)$, we can apply $(\Converges {\rm ext})$ to obtain
\[ {\cal A}, \rho \; \models \; [ \forall x . \, (\lambda x . M x) x = M x ] \; \rightarrow \; \lambda x . M x = M . \]
It is thus sufficient to show
\[ {\cal A} \; \models \; (\lambda x . M x) x = M x . \]
But this is just an instance of $(\beta )$, which ${\cal A}$ satisfies by (i).

\noindent (iii). We calculate:
\begin{eqnarray*}
\lambda \ell \; \vdash \; M \Converges & \Rightarrow & M \Converges \lambda x . N \\
& \Rightarrow & \lambda \; \vdash \; M = \lambda x . N \\
& \Rightarrow & {\cal A} \; \models \; M = \lambda x . N \\
& \Rightarrow & {\cal A} \; \models \; M \Converges , 
\end{eqnarray*}
since ${\cal A} \; \models \; \lambda x . N \Converges$, as noted in (ii).

\noindent (iv). By (i) and (iii),
\[ {\cal A} \; \models \; {\bf YK} \Converges \: \& \: \forall x . \, {\bf (YK)} x = {\bf YK} . \]
Hence we can use the same argument as in \ref{lazyprop}(iv) to prove that
\[ {\cal A} \; \models \; x \sqsubseteq {\bf YK} . \]

\noindent (v). This assertion amounts to the same list of properties as Proposition \ref{lazycong}, 
but with respect to ${\Im}({\cal A})$. 
The only difference in the proof is that \ref{lazycong}(vii) follows immediately from \ref{appstruct} and the fact that ${\cal A}$ is an ats, and can then be used to prove \ref{lazycong}(iv) by induction on $P$. \qed

Part (iii) of the Theorem tells us that all the closed terms which we expect to converge must do so in any lts. What of the converse? For example, do we have
\[ {\cal A} \; \models \; {\bf \Omega} \Diverges \]
in every lts? This is evidently not the case, since we have not imposed any axioms which require {\em anything} to be divergent.

\begin{observation}
Let ${\cal A} = (A, ev)$ be an ats in which $ev$ is {\em total}, i.e. ${\sf dom} \; ev = A$. Then ${\Im}({\cal A})$ is {\em inconsistent}, in the sense that
\[ {\cal A} \; \models \; x = y . \]
\end{observation}
This is of course because the distinctions made by applicative bisimulation are based on divergence.

In the light of this observation and \ref{ltsprop}, it is natural to make the following definition in analogy with that in \cite{Bar}:
\begin{definition}
{\rm An lts ${\cal A}$ is {\em sensible} if the converse to \ref{ltsprop}(iii) holds, i.e. for all $M \in \Lambda^{0}$:
\[ {\cal A} \; \models \; M \Converges \;\; \Longleftrightarrow \;\; 
\lambda \ell \; \vdash \; M \Converges \;\; \Longleftrightarrow \;\; \exists x, N. \, \; \lambda \; \vdash \; M = \lambda x . N . \]
(The second equivalence is justified by an appeal to the Standardisation Theorem \cite{Bar}.)}
\end{definition}

 
