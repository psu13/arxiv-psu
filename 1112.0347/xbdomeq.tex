\section{A Domain Equation for Synchronisation Trees}

In this section, we shall define a domain of synchronisation trees, and establish some of its basic properties.
Since our definitions will use the Plotkin powerdomain, we need to work in a category which is closed under this construction.
This means that we cannot use {\bf SDom}, as we did in the previous two Chapters.
Instead, we will use {\bf SFP}.
The only facts about {\bf SFP} which we will need are that it is a category of algebraic domains closed under the following type constructions:
\subsection*{Separated Sum}
Let $A$ be a countable set, and $\{ D_{a} \}_{a \in A}$ an $A$-indexed family of domains. Then $\sum_{a \in A} D_{a}$ is formed by taking the disjoint union of the $D_{a}$ and adjoining a bottom element.
We shall write elements of the disjoint union as $\ltuple a, d \rtuple$ ($a \in A$, $d \in D_{a}$).
Note that the ordering is defined so that
\[ \ltuple a, d \rtuple \sqsubseteq \ltuple a' , d' \rtuple \;\; \Longleftrightarrow \;\; a = a' \: \& \: d \sqsubseteq_{D_{a}} d' . \]
\begin{itemize}
\item For each $a \in A$, the function
\[ D_{a} \rightarrow \sum_{a \in A} D_{a} \]
\[ d \mapsto \ltuple a, d \rtuple \]
is continuous.

\item Separated sum is functorial; given a family
\[ f_{a} : D_{a} \rightarrow E_{a} \;\; (a \in A) , \]
\[ \sum_{a \in A} f_{a} : \sum_{a \in A} D_{a} \rightarrow \sum_{a \in A} E_{a} \]
is defined by:
\[ \begin{array}{lll}
(\sum_{a \in A} f_{a}) \bot & = & \bot \\
(\sum_{a \in A} f_{a}) \ltuple a, d \rtuple & = & \ltuple a, f_{a} d \rtuple .
\end{array} \]
\end{itemize}

\subsection*{The Plotkin Powerdomain}
We write $P[D]$ for the Plotkin powerdomain over $D$.
Although this construction is best {\em characterised} abstractly, as in 
\cite{HP79}, for purposes of comparison with more concrete operational notions a good representation is invaluable.
This is provided in \cite{Plo76,PloLN}.
\begin{definition}
{\rm For an algebraic domain $D$ the {\em Lawson topology} on $D$ is generated by the sub-basic sets
\[ \diverges b , \;\; D - \diverges b \]
for finite $b \in D$ (so the Lawson topology refines the Scott topology). 
We will write the closure operator associated with the Lawson topology as $Cl$. (NB: in \cite{Plo76}, the Lawson topology is called the Cantor topology).}
\end{definition}

\begin{definition}
{\rm For $X \subseteq D$,
\[ \begin{array}{rlcl}
(i) & Con(X) & \equiv & \{ d : \exists d_{1}, d_{2} \in X . \, d_{1} \sqsubseteq d \sqsubseteq d_{2} \} \\
(ii) & X^{\star} & \equiv & Con \circ Cl .
\end{array} \]
$X$ is said to be
\begin{itemize}
\item {\em Lawson-closed} if $X = Cl \; X$
\item {\em Convex-closed} if $X = Con \; X$
\item {\em Closed} if $X = X^{\star}$.
\end{itemize}}
\end{definition}

\begin{definition}
{\rm The {\em Egli-Milner order}. For $X, Y \subseteq D$:}
\begin{eqnarray*}
X \sqsubseteq_{EM} Y & \equiv & \forall x \in X . \, \exists y \in Y . \, x \sqsubseteq y \; \& \; \forall y \in Y . \, \exists x \in X . \, x \sqsubseteq y .
\end{eqnarray*}
\end{definition}

The representation of the Plotkin powerdomain can now be defined as follows:
\begin{eqnarray*}
P[D] & \equiv & ( \{ X \subseteq D : X \not= \varnothing , X = X^{\star} \} , \sqsubseteq_{EM} ) .
\end{eqnarray*}

There are also a number of (continuous) operations associated with the Plotkin powerdomain, which we shall describe in terms of our representation of $P[D]$.

\begin{itemize}

\item Firstly, $P$ is {\em functorial}: given $f \: D \rightarrow E$,
\[ P f : P[D] \rightarrow P[E] \]
is defined by
\begin{eqnarray*}
P f (X) & \equiv & \{ f(x) | x \in X \}^{\star} .
\end{eqnarray*}

\item {\em Singleton:}
\[ \lsing . \rsing : D \rightarrow P[D] \]
is defined by
\begin{eqnarray*}
\lsing d \rsing & \equiv & \{ d \}^{\star} = \{ d \} .
\end{eqnarray*}

\item {\em Union:}
\[ \uplus : P[D]^{2} \rightarrow P[D] \]
is defined by
\begin{eqnarray*}
X \uplus Y & \equiv & (X \uplus Y)^{\star} = Con(X \cup Y) .
\end{eqnarray*}

\item {\em Big Union:}
\[ \biguplus : P[P[D]] \rightarrow P[D] \]
is defined by
\begin{eqnarray*}
\biguplus ( \Theta ) & \equiv & ( \bigcup \Theta )^{\star} = Con ( \bigcup \Theta ) .
\end{eqnarray*}

\item {\em Tensor Product} \cite{HP79}. We will only need the following: given
\[ f : D^{n} \rightarrow D \]
the {\em multilinear extension}
\[ f^{\dagger} \: P[D]^{n} \rightarrow P[D] \]
is defined by
\begin{eqnarray*}
f^{\dagger} (X_{1}, \ldots , X_{n} ) & \equiv & \{ f(x_{1}, \ldots , x_{n} ) : x_{i} \in X_{i} \}^{\star} .
\end{eqnarray*}
(Note that for $n = 1$, $f^{\dagger} = P f$.) This extension has the property
\begin{eqnarray*}
f^{\dagger} (X_{1}, \ldots , X_{i} \uplus X'_{i}, \ldots , X_{n}) & = & f^{\dagger} (X_{1}, \ldots , X_{i}, \ldots , X_{n}) \\
& & \mbox{} \uplus f^{\dagger} (X_{1}, \ldots ,  X'_{i}, \ldots , X_{n})  
\end{eqnarray*}
for $(1 \leq i \leq n)$.
\end{itemize}

\subsection*{Adjoining the empty set}

To the best of my knowledge, the only significant precursor of our work in this Chapter is \cite{MM79}.
The main reason that something like our present programme could not have been carried through in their framework is that, because of a technical problem, they used the Smyth rather than the Plotkin powerdomain.
This rules out any hope of gaining a correspondence with bisimulation.
The technical problem is that of adjoining the empty set to the
powerdomain to model the convergent process with no actions (NIL in CCS
\cite{Mil80}, $\Oh$ in SCCS \cite{Mil83}, STOP in CSP \cite{Hoa85},
$\delta$ in ACP \cite{BK84}, etc.).
If we add the empty set to our representation of $P[D]$, it is not related to
anything except itself under $\sqsubseteq_{EM}$; in category-theoretic
terms, the problem is the non-existence of a certain free construction
(\cite{PloLN} ).
Fortunately, we do not need these non-existent solutions.
We shall adjoin the empty set to the Plotkin powerdomain in a way which
has two advantages:
\begin{enumerate}
\item There is no theoretical overhead, since it is definable as a derived
operation from standard type constructions.
\item It works, i.e. is exactly suited to our semantic purposes, as the
results to follow will show.
\end{enumerate}


For motivation, consider a transition system $({\rm Proc}, {\sf Act}, \rightarrow ,
\diverges )$ and processes $p, r \in {\rm Proc}$ such that
\[ \begin{array}{rl}
(i) & p\diverges , \; r \converges \\
(ii) & p \nrightarrow , \; r \nrightarrow .
\end{array} \]
Then it is easy to see that, for all $q \in {\rm Proc}$:
\[ \begin{array}{rlcl}
(i) & r \preord^{B} q & \Longleftrightarrow & r \sim^{B} q \\
(ii) & q \preord^{B} r & \Longleftrightarrow & q \nrightarrow \\
& & \Longleftrightarrow & q \sim^{B} p \; \mbox{or} \; q \sim^{B} r .
\end{array} \]

This suggests the following
\begin{definition}
{\rm $P^{0}[D]$, the Plotkin powerdomain with empty set.

\noindent Representation of $P^{0}[D]$:
\begin{center}
\begin{tabular}{ll}
{\bf Elements} & $\{ X \subseteq D : X = X^{\star} \} = P[D] \cup \{
\varnothing \}$. \\
{\bf Ordering} & $X \sqsubseteq Y \; \equiv \; X = \{ \bot \} \; \mbox{or} \;
X \sqsubseteq _{EM} Y$.
\end{tabular}
\end{center}}
\end{definition}

\begin{observation}
$P^{0}[D] \; \cong \; ({\bf 1})_{\bot} \oplus P[D]$.
\end{observation}

In principle, we could work throughout with 3.5 as the {\em definition} of
$P^{0}[D]$; in practice, it is much more convenient to work with the
representation given by 3.4.
This requires that we extend our definitions of the powerdomain
operations to work on $P^{0}[D]$.
In fact, all of the definitions following 3.3 still make sense for $P^{0}[D]$.
It is easily checked that $\uplus$, $\biguplus$ and $\lsing \cdot \rsing$
are continuous on $P^{0}[D]$.
For $P^{0} f$ and $f^{\dagger}$ a technical point arises, which is not
specific to 3.4, but stems from the use of coalesced sum in 3.5.
As is well known,  coalesced sum is functorial only on the category of
{\em strict} functions.
Hence we can only use $P^{0} f$ if $f$ is strict, and $f^{\dagger}$ if $f$ is
strict in each argument separately.
With these provisos, the extended operations are continuous.

{\bf Notation}. We use $\emptyset$ to denote the empty set in $P^{0}[D]$;
if $I$ is a finite index set, we write
\[ \biguplus_{i \in I} X_{i} \]
meaning the iterated use of $\uplus$ (which is associative, commutative
and idempotent on $P^{0}[D]$, just as it is on $P[D]$) if $I \not=
\varnothing$, and $\emptyset$ otherwise.
Also, we write
\[ \lsing d : A \rsing \]
where $d \in D$ and $A$ is some sentence, meaning $\lsing d \rsing$ if
$A$ is true, and $\emptyset$ otherwise.

We are now ready for the main definition of the section.
\begin{definition}
{\rm Let ${\sf Act}$ be a {\em countable} set of actions.
Then $\Dom ({\sf Act})$, the domain of synchronisation trees over ${\sf Act}$ (we
henceforth omit the parameter ${\sf Act}$), is defined to be the initial solution
of the domain equation}
\begin{equation}
\Dom \; \cong \; P^{0}[ \sum_{a \in {\sf Act}} \Dom ]  . \label {domeq} 
\end{equation}
\end{definition}
Here the sum $\sum_{a \in {\sf Act}} \Dom$ is the ``copower'' of ${\sf Act}$ copies of 
 $\Dom$.
The equation is essentially that of \cite{MM79}, minus the value passing
and with a different powerdomain.

How can we relate this domain equation to the formalism of Chapter~4?
Suppose we extend the metalanguage of types introduced there with a
constructor $P_{p}( \cdot )$ for the Plotkin powerdomain.
Then we can write
\[ \Dom \; \equiv \; {\sf rec} \: t . ({\bf 1})_{\bot} \oplus P_{p}[\sum_{a \in {\sf Act}}
t] \]
using 3.5 to eliminate $P^0$.
This is not yet a valid type expression because of the sum
\begin{equation}
\sum_{a \in {\sf Act}} t \label{sumeq}
\end{equation}
Let us take the main case of interest, where ${\sf Act}$ is countably infinite,
say ${\sf Act} = \{ a_{n} \}_{n \in \omega}$.
Then we can replace~\ref{sumeq} by the recursive expression
\begin{equation}
{\sf rec} \: u . (t)_{\bot} \oplus u  \label{recsum} 
\end{equation}
yielding the overall expression
\begin{equation}
\Dom \; \equiv \; {\sf rec} \: t. ({\bf 1})_{\bot} \oplus P_{p}[{\sf rec} \: u . (t)_{\bot}
\oplus u  ]  \label{typeexp} 
\end{equation}
the intention being that the $i$'th summand as we unfold~\ref{recsum}
corresponds to $a_{i} \in {\sf Act}$.

The reader will by now probably appreciate our efforts to streamline the
presentation.
Nevertheless, we regard the ``closed form'' expression \ref{typeexp} as
fundamental, and the logic we shall introduce in the next section could be
derived mechanically from it in the manner detailed in Chapter 4.

In the remainder of this section, we shall apply some standard
domain-theoretic methods to elucidate the structure of $\Dom$.

{\bf Notation.} We write $\bot$ for the bottom element of
$\sum_{a \in {\sf Act}} \Dom$; $\lsing \bot \rsing$ is then the bottom element
of $ P^{0}[ \sum_{a \in {\sf Act}} \Dom ]$.

How can we unpack the structure of $\Dom$ from the domain equation
\ref{domeq}?
This is best done in two parts:
\begin{enumerate}
\item A {\em specified} isomorphism pair
\[ \Dom \begin{array}{c}
\eta \\
\rightleftarrows \\
\theta
\end{array} P^{0}[ \sum_{a \in {\sf Act}} \Dom ] . \]
In fact, we shall elide $\eta$ and $\theta$, and treat \ref{domeq} as an
identity; this is only a notational convenience, and the reader can put
$\eta$ and $\theta$ back without encountering any difficulties.

\item {\em Initiality}. The categorical framework is clumsy to work with
for our purposes.
Instead, we will use an ``intrinsic'' (or in the terminology of
\cite{SP82} a ``local'' or ``{\bf O}-notion'') formulation.
\end{enumerate}

\begin{definition}
{\rm We define a sequence of functions
\[ \pi_{k} : \Dom \rightarrow \Dom \]
as follows:
\[ \begin{array}{lll}
\pi_{0} & \equiv & \lambda x \in \Dom . \lsing \bot \rsing \\
\pi_{k+1} & \equiv & P^{0} \sum_{a \in {\sf Act}} \pi_{k} .
\end{array} \] }
\end{definition}
Note that $\sum_{a \in {\sf Act}}$ always produces a strict function, so this is
well-defined.

Now the following proposition is standard (\cite[Chapter 5 Theorem 3]{PloLN}):
\begin{proposition}
\label{icolim}
\Dom\ is the ``internal colimit'' of the $\pi_{k}$:
\[ \begin{array}{rl}
(i) & \mbox{Each $\pi_{k}$ is continuous and $\pi_{k} \sqsubseteq
\pi_{k+1}$} \\
(ii) & \bigsqcup_{k} \pi_{k} = {\sf id}_{\Dom} \\
(iii) & \pi_{k} \circ \pi_{k} = \pi_{k} \\
(iv) & \forall d_{1}, d_{2} \in \Dom . \, d_{1} \sqsubseteq d_{2} \;\;
\Longleftrightarrow \;\; \forall k . \, \pi_{k} d_{1} \sqsubseteq \pi_{k} d_{2} .
\end{array} \]
\end{proposition}
In particular, we will use part $(iv)$ of this Proposition as the
cutting edge of initiality.

Next, it will be useful to have an explicit description of the finite
elements of $\Dom$, which, as already noted, is in {\bf SFP}, and hence
algebraic.

\begin{definition}
\label{felts}
{\rm $K(\Dom ) \subseteq \Dom$ is defined inductively as follows:}
\begin{itemize}
\item $\emptyset \in K(\Dom )$
\item $\lsing \bot \rsing \in K(\Dom )$
\item $a \in {\sf Act}, d \in K(\Dom ) \; \Rightarrow \; \lsing {<}a, d{>} \rsing
\in K(\Dom )$
\item $d_{1}, d_{2} \in K(\Dom ) \; \Rightarrow \; d_{1} \uplus d_{2} \in
K(\Dom )$.
\end{itemize}
\end{definition}
The following is again standard:
\begin{proposition}
\label{feltp}
$ K(\Dom )$ is exactly the set of finite elements of $\Dom$.
\end{proposition}

Finally, we consider $\Dom$ as a {\em transition system} $(\Dom , {\sf Act},
\rightarrow , \diverges )$ defined by:
\[ \begin{array}{clcl}
\bullet & \labarrow{d}{a}{d'}  & \equiv & {<}a, d' {>} \in d \\
\bullet & d \diverges & \equiv & \bot \in d .
\end{array} \]

\begin{proposition}
\label{ifs}
$\Dom$ is ``internally fully abstract'', i.e.
\[ \forall d_{1}, d_{2} \in \Dom\ . \, d_{1} {\preord}^{B} d_{2} \;\;
\Longleftrightarrow \;\; d_{1} \sqsubseteq d_{2} . \]
\end{proposition}

\proof\ We shall prove
\[ (1) \;\; \forall k . \; d_{1} \preord_{k} d_{2} \;\; \Longrightarrow \;\; \pi_{k}
d_{1} \sqsubseteq \pi_{k} d_{2} \]
and
\[ (2) \;\; {\sqsubseteq} \subseteq {\preord^{B}} . \]
Clearly (1) implies
\[ (3) \;\; {\preord_{\omega}} \subseteq {\sqsubseteq}  \]
by \ref{icolim}$(iv)$, and since
\[ (4) \;\; {\preord^{B}} \subseteq {\preord_{\omega}} , \]
we obtain ${\preord^{B}} = {\sqsubseteq}$, as required.

\noindent (1). By induction on $k$. The basis is trivial. 
For the inductive step, assume $d \preord_{k+1} e$.
Now $d = \emptyset$ and $d \preord_{k+1} e$ implies $e = \emptyset$,
while $d = \lsing \bot \rsing$ implies $d \sqsubseteq e$, so we may
assume $d \not= \emptyset \not= e$, and it suffices to prove $d
\sqsubseteq_{EM} e$.

From the definitions we have $\pi_{k+1} d = X^{\star}$, where
\[ X = \{ {<}a, \pi_{k} d' {>} : {<}a, d' {>} \in d \} \cup \{ \bot : \bot \in d \} , 
\]
and similarly $\pi_{k+1} e = Y^{\star}$. Now
\[ \begin{array}{ll}
\bullet & {<}a, \pi_{k} d' {>} \in X \\
\Longrightarrow & \labarrow{d}{a}{d'} \\
\Longrightarrow & \exists e' . \, \labarrow{e}{a}{e'} \: \& \: d' \preord_{k} e' \\
\Longrightarrow & \exists e' . \, {<}a, e' {>} \in e \: \& \: \pi_{k} d' \sqsubseteq
\pi_{k} e' 
\;\; \mbox{by induction hypothesis} \\
\Longrightarrow & \exists {<}a, \pi_{k} e' {>} \in Y . \, {<}a, \pi_{k} d' {>}
\sqsubseteq {<}a, \pi_{k} e' {>} .
\end{array} \]
Again,
\[ \begin{array}{ll}
\bullet & \bot \not\in X \\
\Longrightarrow & \bot \not\in d \\
\Longrightarrow & \bot \not\in e \: \& \: [ \labarrow{e}{a}{e'}  \; \Rightarrow \;
\exists d' . \, \labarrow{d}{a}{d'} \: \& \: d' \preord_{k} e' ] \\
\Longrightarrow & \bot \not\in Y \: \& \: \forall {<}a, \pi_{k} e' {>} \in Y . \,
\exists {<}a, \pi_{k} d' {>} \in X . \, \pi_{k} d' \sqsubseteq \pi_{k} e' 
\end{array} \]
by the induction hypothesis again,
and we have shown $X \sqsubseteq_{EM} Y$, which implies $X^{\star}
\sqsubseteq_{EM} Y^{\star}$, as required.

\noindent (2). It suffices to show that $\sqsubseteq$ is a prebisimulation.
This is a simple calculation:
\[ \begin{array}{ll}
\bullet & d \sqsubseteq e \\
\Longrightarrow &  \forall {<}a, d' {>} \in d . \, \exists {<}a,
e' {>} \in e . \, d' \sqsubseteq e' \\
& \& \: \bot \not\in d \; \Rightarrow \; \bot \not\in e 
\; \& \; [ \forall {<}a, e' {>} \in e . \, \exists {<}a, d' {>} \in d . \, d'
\sqsubseteq e'] \\
\Longrightarrow & \forall a \in {\sf Act} . \, \labarrow{d}{a}{d'} \; \Rightarrow \;
\exists e' . \, \labarrow{e}{a}{e'} \: \& \: d' \sqsubseteq e' \\
& \& \: d \converges \; \Rightarrow \; e \converges 
\; \& \; [ \labarrow{e}{a}{e'} \; \Rightarrow \; \exists d' . \,
\labarrow{d}{a}{d'} \: \& \: d' \sqsubseteq e' ] . \;\;\; \qed
\end{array} \]

We finish with some examples to illustrate the richness of $\Dom$ as a
transition system.

\subsection*{Examples}

(1). $\Dom$ is not sort-finite.
\begin{eqnarray*}
d_{0} & \equiv & \lsing {<} a_{0} , \lsing \bot \rsing {>}  \rsing \\
d_{1} & \equiv & \lsing {<}a_{0} , \lsing {<} a_{1}, \lsing \bot \rsing {>}
\rsing {>} \rsing \\
& \vdots & \\
{\sf sort}( \bigsqcup d_{k} ) & = &  \{ a_{0}, a_{1} , \ldots \} 
\end{eqnarray*}

\noindent (2). $\Dom$ is not weakly image-finite.
\begin{eqnarray*}
c_{k} & \equiv & {\sum_{i \leq k} a^{i} \Oh } + a^{k} \Omega \;\;\; (k \in
\omega ) \\
\bigsqcup c_{k} & = & { \sum_{k \in \omega} a^{k} \Oh } + a^{\omega} .
\end{eqnarray*}


