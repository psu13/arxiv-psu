\section{Constructions}
In this section, we fill in the programme outlined in the previous section by defining a number of type constructions as $\trianglelefteq$-monotonic and continuous functions over ${\bf DPL1}$. These definitions will follow a common pattern. We take a binary type construction $T(A,B)$ for illustration. Specific to each such construction will be a set of {\it generators} $G(T(A,B))$. Then the carrier $|T(A,B)|$ is defined inductively by
\[ \bullet \;\;\; G(T(A,B)) \subseteq |T(A,B)| \]
\[ \bullet \;\;\; \true , \false \in |T(A,B)| \]
\[ \bullet \;\;\; \frac{a, b \in |T(A,B)|}{a \wedge b, a \vee b \in |T(A,B)|} \]
The operations $0, 1, \wedge , \vee$ are then defined ``freely'' in the obvious way, i.e.
\[ 0_{T(A,B)} \equiv \false , \;\; a \vee_{T(A,B)} b \equiv a \vee b, \;\; 1_{T(A,B)} \equiv \true , \;\; a \wedge_{T(A,B)} b \equiv a \wedge b. \]
Finally, the relations $\leq_{T(A,B)}$, $=_{T(A,B)}$ are defined inductively as the least satisfying $(p1)$--$(p4)$ plus specific axioms on the generators. 
(Note that our definition of {\bf 1} in the previous section is the special case of this scheme where the set of generators is empty.)

As an essential part of the machinery for defining the type constructions, we shall introduce a number of meta-predicates over the carriers $|T(A,B)|$ of the constructed prelocales. These will be used as side-conditions on a number of axiom-schemes and rules. They will serve as ``syntactic'' analogues of the ``semantic'' predicates $con$, $pr$, $t$ introduced previously. 
The same predicates will be defined for each contruction:
\begin{itemize}
\item ${\sf PNF}$, prime normal form.
\item ${\sf CON}$, ${\sf T}$, defined over elements of the form $\bigwedge_{i \in I}a_{i}$, 
with each $a_{i}$ in ${\sf PNF}$. 
${\sf CON}$ is {\it consistency} (i.e. ${\sf CON}(a)$ means $a \not= 0$), 
and ${\sf T}$ is {\it termination} (i.e. ${\sf T}(a)$ means $a \not= 1$).
\item ${\sf CPNF}$, consistent prime normal forms, where ${\sf CPNF}(a)$ implies ${\sf PNF}(a)$ and ${\sf CON}(a)$.
\end{itemize}
Given these definitions, three further predicates are defined as follows:
\begin{itemize}
\item ${\sf CDNF}$, consistent disjunctive normal form:
\[ {\sf CDNF}(a) \; \equiv \; a = \bigvee_{i \in I} a_{i} \: \& \: \forall i \in I . \, {\sf CPNF}(a_{i}) \]
\end{itemize}
\[ \bullet \;\;\; a \converges \; \equiv \; a = \bigvee_{i \in I}a_{i} \;\& \; \forall i \in I . \, {\sf PNF}(a_{i}) \: \& \: {\sf T}(a_{i}) \]
\[ \bullet \;\;\; \#(a) \; \equiv \; a = \bigvee_{i \in I}a_{i} \; \& \; \forall i \in I . \, {\sf PNF}(a_{i}) \: \& \: \neg {\sf CON}(a_{i}). \]

It will follow from our general scheme of definition and the way that the generators are defined that the following points are immediate, for $A, A', B, B'$ in {\bf DPL1} with $A \trianglelefteq A'$ and $B \trianglelefteq B'$:
\begin{itemize}
\item $T(A,B)$ satisfies $(p1)$--$(p4)$
\item ${\bf 1} \trianglelefteq T(A,B)$
\item $T(A,B) \Subset T(A',B')$
\item $T$ is continuous on carriers.
\end{itemize}
We are left to focus our attention on proving that:
\begin{itemize}
\item $T(A,B)$ satisfies $(d1)$--$(d3)$
\item conditions $(s2)$ and $(s3)$ for $T(A,B) \trianglelefteq T(A',B')$ are satisfied.
\end{itemize}

Our method of establishing this for each $T$ is uniform, and goes via another essential verification, namely that $T$ does indeed correspond to the intended construction over domains. We define a semantic function
\[ \lsem \cdot \rsem_{T(A,B)} : |T(A,B)| \rightarrow K\Omega (F_{T}(\hat{A}, \hat{B})) \]
where $F_{T}$ is the functor over {\bf SDom} corresponding to $T$, and show 
that $\lsem \cdot \rsem_{T(A,B)}$ is a (pre)isomorphism; and moreover natural with respect to embeddings induced by $\trianglelefteq$. 
This allows us to read off the required ``proof-theoretic'' facts about $T$ from the known ``model-theoretic'' ones about $F_{T}$. 
Moreover, we can derive ``soundness and completeness'' theorems as byproducts.

For each type construction $T$, we prove the following sequence of results:

{\bf T1: Adequacy of Metapredicates.} {\it For each $a \in {\sf PNF}(T(A,B))$:}
\[\begin{array}{rl}
(i)   & \lsem a \rsem_{T(A,B)} \in pr(K\Omega (F_{T}(\hat{A},\hat{B}))) \\
(ii)  & {\sf CON}(a) \; \Longleftrightarrow \; \lsem a \rsem_{T(A,B)} \neq \varnothing \\
(iii) & {\sf T}(a) \; \Longleftrightarrow \; \bot_{F_{T}(\hat{A},\hat{B})} \not\in \lsem a \rsem_{T(A,B)}. 
\end{array} \]

{\bf T2: Normal Forms.} 
\[ \forall a \in |T(A,B)|. \, \exists b \in {\sf CDNF}(T(A,B)) . \, a =_{T(A,B)} b. \]

{\bf T3: Soundness.} {\it For all $a, b \in |T(A,B)|$:}
\[ a \leq_{T(A,B)} b \; \Rightarrow \; \lsem a \rsem_{T(A,B)} \subseteq \lsem b \rsem_{T(A,B)}. \]

{\bf T4: Prime Completeness.} {\it For all $a, b \in {\sf CPNF}(T(A,B))$:}
\[ \lsem a \rsem_{T(A,B)} \subseteq \lsem b \rsem_{T(A,B)} \; \Rightarrow \; a \leq_{T(A,B)} b. \]

{\bf T5: Definability.} 
\[ \forall u \in K(F_{T}(\hat{A},\hat{B})). \, \exists a \in {\sf CPNF}(T(A,B)). \, \lsem a \rsem_{T(A,B)} = \diverges (u). \]

{\bf T6: Naturality.} {\it Given $A \trianglelefteq A'$, $B \trianglelefteq B'$ in {\bf DPL1}, let $e_{1} : \hat{A} \rightarrow \hat{A'}$, $e_{2} : \hat{B} \rightarrow \hat{B'}$ be the corresponding embeddings. Given an embedding $e : D \rightarrow E$, let $e^{\dag} : K\Omega (D) \rightarrow K\Omega (E)$ be defined by
\[ e^{\dag}(\diverges X) = \diverges \{ e(x) : x \in X\} \]
which is well defined since embeddings map finite elements to finite elements. 
Let 
\[ \eta_{T(A,B)} : \hat{C} \rightarrow F_{T}(\hat{A},\hat{B}) \]
be the adjoint of $\lsem \cdot \rsem_{T(A, B)}$, where $C = T(A, B)$. Then:
\[ \begin{array}{lrcl}
(A) & (F_{T}(e_{1}, e_{2}))^{\dag} \circ \lsem \cdot \rsem_{T(A, B)} & = & 
\lsem \cdot \rsem_{T(A' , B' )} \\
(B) & F_{T}(e_{1}, e_{2}) \circ \eta_{T(A, B)} & = & \eta_{T(A' , B' )} 
\circ {\converges}_{T(A' , B' )} (\cdot )
\end{array} \]
(These equations make sense since $T(A, B) \Subset T(A' , B' )$ by assumption.)}

All the desired properties of our constructions can easily be derived from these results.

{\bf T7: Completeness.} {\it For $a, b \in |T(A,B)|$:}
\[ \lsem a \rsem_{T(A,B)} \subseteq \lsem b \rsem_{T(A,B)} \; \Rightarrow \; a \leq_{T(A,B)} b. \]

\proof\ By (T2),
\[ a =_{T(A,B)} \bigvee_{i \in I}a_{i}, \;\; b=_{T(A,B)} \bigvee_{j \in J}b_{j}, \]
with $a_{i}, b_{j} \in {\sf CPNF}(T(A,B))$ ($i \in I, j \in J$). By (T3),
\[ \lsem a \rsem_{T(A,B)} = \lsem \bigvee_{i \in I}a_{i} \rsem_{T(A,B)}, \;\;\; \lsem b \rsem_{T(A,B)} = \lsem \bigvee_{j \in J}b_{j} \rsem_{T(A,B)}. \]
By (T1),
\[ \lsem a_{i} \rsem_{T(A,B)}  =   \diverges (u_{i}),  
\lsem b_{j} 
\rsem_{T(A,B)} =  
\diverges (v_{j}) \]
\[ u_{i}, v_{j} \in K(F_{T}(\hat{A},\hat{B})) \;\;\; (i \in I, j \in J). \] 
Now,
\[\begin{array}{lll}
            & \lsem a \rsem_{T(A,B)} \subseteq \lsem b \rsem_{T(A,B)}   & \\
\Longrightarrow & \bigcup_{i \in I}\diverges (u_{i}) \subseteq \bigcup_{j \in J} \diverges (v_{j})  & \\
\Longrightarrow & \forall i \in I. \, \exists j \in J. \, \diverges (u_{i}) \subseteq \diverges (v_{j}) & \\
\Longrightarrow & \forall i \in I. \, \exists j \in J. \, a_{i} \leq_{T(A,B)} b_{j} &  \mbox{by (T4)} \\
\Longrightarrow & \bigvee_{i \in I}a_{i} \leq_{T(A,B)} \bigvee_{j \in J}b_{j} & \mbox{by (p2)} \\
\Longrightarrow & a \leq_{T(A,B)} b & \mbox{by (p1).} \; \qed 
\end{array} \]

{\bf (T8): Stone Duality.} {\it $T(A,B)$ is the Stone dual of $F_{T}(\hat{A},
\hat{B})$, i.e.}
\[\begin{array}{rl}
(i)   & F_{T}(\hat{A},\hat{B}) \; \cong \;  \hat{C} \;\;\; (C = T(A,B)) \\
(ii)  & \lsem \cdot \rsem : |T(A,B)| \rightarrow K\Omega (F_{T}(\hat{A},\hat{B})) 
\; \mbox{is a pre-isomorphism.}
\end{array} \]

\proof\ $(i)$ and $(ii)$ are equivalent since Scott domains are coherent.
$(ii)$ is an immediate consequence of (T3), (T5) and (T7). \qed

{\bf (T9).} {\it $T$ is a well defined, $\trianglelefteq$-monotonic and continuous operation on {\bf DPL1}.}

\proof\ T(A,B) is a domain prelocale by (T8), 
since $K\Omega (F_{T}(\hat{A},\hat{B}))$ is. 
Given $A \trianglelefteq A'$, $B \trianglelefteq B'$, 
$T(A,B) \trianglelefteq T(A',B')$ follows from (T6)(A) and the following 
general properties of $e^{\dag}$ for embeddings $e : D \rightarrow E$:
\begin{enumerate}
\item $e^{\dag}$ is an order-mono, i.e. for $U, V \in K\Omega (D)$:
\[ U \subseteq V \; \Longleftrightarrow \; e^{\dag}(U) \subseteq e^{\dag}(V) \]
\item $e^{\dag}$ preserves primes.
\end{enumerate}
To prove (1), we take $U = \diverges  X$, $V = \diverges  Y$, and calculate:
\begin{eqnarray*}
\diverges  X \subseteq \diverges  Y & \Longleftrightarrow & X \sqsubseteq_{u} Y \\
& \Longleftrightarrow & e(X) \sqsubseteq_{u} e(Y) \;\;\; \mbox{{\mit e} 
is an order-mono} \\
& \Longleftrightarrow & \diverges  e(X) \subseteq \diverges  e(Y) \\
& \Longleftrightarrow & e^{\dag}(U) \subseteq e^{\dag}(V).
\end{eqnarray*}
For (2), we recall that $U \in pr(K\Omega (D))$ implies $U = \varnothing$ or $U = \diverges (u)$ for some $u \in K(D)$. But $e^{\dag}(\varnothing) = \varnothing$, $e^{\dag}(\diverges (u)) = \diverges (e(u))$.

By the remarks at the beginning of the section, the proof is now complete. \qed

{\bf Notation.} Given a domain prelocale $A$, we write
\[ \lsem \cdot \rsem_{A} : |A| \rightarrow K\Omega (\hat{A}) \]
for the pre-isomorphism $\varphi A$ defined in the proof of Theorem~\ref{domtheq}.

We note a further trivial but useful fact about direct images of embeddings for future use.
\begin{proposition}
\label{embim}
If $A \trianglelefteq B$, and $e : \hat{A} \rightarrow \hat{B}$ is the induced embedding, then
\[ e^{\dag} \circ \lsem \cdot \rsem_{A} = \lsem \cdot \rsem_{B}. \; \qed \]
\end{proposition}

\begin{definition} 
{\rm The {\em function space} construction $A \rightarrow B$.

\noindent (i) The generators:
\[ G(A \rightarrow B) \; \equiv \; \{ (a \rightarrow b) : a \in |A|, b \in |B| \}. \]
This fixes $|A \rightarrow B|$ according to the general scheme described above.

\noindent (ii) The metapredicates:
\begin{eqnarray*}
{\sf PNF}(A \rightarrow B) & \equiv & \{\bigwedge_{i \in I}(a_{i} \rightarrow b_{i}) : a_{i} \in pr(A), b_{i} \in pr(B), i \in I \} \\
{\sf CON}(\bigwedge_{i \in I}(a_{i} \rightarrow b_{i})) & \equiv & \forall J \subseteq I. \\ 
& & \bigwedge_{j \in J}a_{j} \in con(A) \; \Longrightarrow \bigwedge_{j \in J}b_{j} \in con(B) \\
{\sf T}(\bigwedge_{i \in I}(a_{i} \rightarrow b_{i})) & \equiv & \exists i \in I. \, a_{i} \in con(A) \& b_{i} \in t(B) \\
{\sf CPNF}(\bigwedge_{i \in I}(a_{i} \rightarrow b_{i})) & \equiv & {\sf CON}(\bigwedge_{i \in I}(a_{i} \rightarrow b_{i})) \\ 
& & \& \; \forall i \in I. \, a_{i} \in con(A) \:
\& \: b_{i} \in con(B)
\end{eqnarray*}
The predicates ${\sf CDNF}$, $\#(.)$, $\underline{\ }\converges$ are then defined according to our general scheme.

\noindent (iii) The relations $\leq_{A \rightarrow B}$, $=_{A \rightarrow B}$ are then defined inductively by the following axioms and rules in addition to $(p1)$--$(p4)$ (subscripts omitted).
\[ (\rightarrow - \leq ) \;\;\; \frac{a' \leq a, \; b \leq b'}{(a \rightarrow b) \leq (a' \rightarrow b' )} \]
\[ (\rightarrow - \wedge) \;\;\; (a \rightarrow \bigwedge_{i \in I}b_{i}) = \bigwedge_{\ \in I}(a \rightarrow b_{i}) \]
\[ (\rightarrow - \vee - L) \;\;\; (\bigvee_{i \in I}a_{i} \rightarrow b) = \bigwedge_{i \in I}(a_{i} \rightarrow b) \]
\[ (\rightarrow - \vee - R) \;\;\;  (a \rightarrow \bigvee_{i \in I}b_{i}) = \bigvee_{i \in I}(a \rightarrow b_{i}) \;\;\; (a \in cpr(A)) \]
\[ (\#) \;\;\; a \leq 0 \;\;\; (\#(a)) \] 

\noindent (iv) The semantic function
\[ \lsem \cdot \rsem_{A \rightarrow B} : |A \rightarrow B| \longrightarrow K\Omega ([\hat{A} \rightarrow \hat{B}]) \]
is defined by
\[ \lsem (a \rightarrow b) \rsem_{A \rightarrow B} = (\lsem a \rsem_{A}, \lsem b \rsem_{B}) \]
where for spaces $X$, $Y$ and subsets $U \in K\Omega (X)$, $V \in K\Omega (Y)$,
\[ (U,V) \; \equiv \; \{ f : X \rightarrow Y \; | \; f \; {\rm continuous,} \; f(U) \subseteq V \} \]
is a sub-basic open set in the compact-open topology. The further clauses
\[ \lsem \bigwedge_{i \in I}a_{i} \rsem = \bigcap_{i \in I} \lsem a_{i} \rsem \]
\[ \lsem \bigvee_{i \in I}a_{i} \rsem = \bigcup_{i \in I} \lsem a_{i} \rsem \]
will apply to all type constructions.}
\end{definition}

We will now establish that the function space construction satisfies (T1)--(T6) in a sequence of propositions.

\begin{proposition}[T1]
\label{funT1}
For all $a \in {\sf PNF}(A \rightarrow B)$:
\[ \begin{array}{rl}
(i)   & \lsem a \rsem_{A \rightarrow B} \in pr(K \Omega ([\hat{A} \rightarrow \hat{B}])) \\
(ii)  & {\sf CON}(a) \; \Longleftrightarrow \; \lsem a \rsem_{A \rightarrow B} \neq \varnothing \\
(iii) & {\sf T}(a) \; \Longleftrightarrow \; \bot \not\in \lsem a \rsem_{A \rightarrow B}.
\end{array} \]
\end{proposition}

\proof\ (i) Let $a \in pr(A)$, $b \in pr(B)$. If $a \not\in con(A)$,
\[ \lsem (a \rightarrow b) \rsem_{A \rightarrow B} = [\hat{A} \rightarrow \hat{B}] = 1_{K\Omega ([\hat{A} \rightarrow \hat{B}])}; \]
while if $a \in con(A)$, $b \not\in con(B)$, 
\[ \lsem (a \rightarrow b) \rsem_{A \rightarrow B} = \varnothing . \]
Otherwise, $a \in con(A)$ and $b \in con(B)$. 
Let $u = \diverges (a)$, $v = \diverges (b)$. 
Then $u \in K(\hat{A})$, $v \in K(\hat{B})$, and so
\begin{eqnarray*}
\lsem (a \rightarrow b) \rsem_{A \rightarrow B} & = & (\lsem a \rsem_{A}, \lsem b \rsem_{B}) \\
& = & (\diverges u, \diverges v) \\
& = & \diverges [u, v],
\end{eqnarray*} 
where $[u, v]$ is the step function in $[\hat{A} \rightarrow \hat{B}]$. Similarly, for $a_{i} \in cpr(A)$, $b_{i} \in cpr(B)$:
\begin{eqnarray*}
\lsem \bigwedge_{i \in I}(a_{i} \rightarrow b_{i}) \rsem_{A \rightarrow B} & = & \bigcap_{i \in I} \diverges [u_{i}, v_{i}] \\
& = & \left\{ \begin{array}{ll}
               \diverges ( \bigsqcup_{i \in I}[u_{i}, v_{i}]) & \mbox{if 
               $\consistent \{[u_{i}, v_{i}] : i \in I \}$} \\
               \varnothing & \mbox{otherwise.}
               \end{array}
      \right.
\end{eqnarray*}
(ii) Let $a = \bigwedge_{i \in I}(a_{i} \rightarrow b_{i})$. We use the notation of (i). Suppose ${\sf CON}(a)$. Then for $i \in I$,
\[ b_{i} \not\in con(B) \; \Longrightarrow \; a_{i} \not\in con(A) \; \Longrightarrow \; \lsem (a_{i} \rightarrow b_{i}) \rsem_{A \rightarrow B} = 1_{K\Omega ([\hat{A} \rightarrow \hat{B}])}, \]
and so
\begin{eqnarray*}
\lsem a \rsem_{A \rightarrow B} & = & \lsem \bigwedge 
\{ (a_{j} \rightarrow b_{j}) : a_{j} \in cpr(A), b_{j} \in cpr(B) \} \rsem_{A \rightarrow B} \\
& = & \diverges (\bigsqcup \{ [u_{j}, v_{j}] : a_{j} \in cpr(A), b_{j} \in cpr(B) \} ),
\end{eqnarray*} 
which is well-defined by \ref{funcon}.
For the converse, suppose $\neg {\sf CON}(a)$. Then for some $J \subseteq I$,
$\bigwedge_{j \in J}a_{j} \in con(A)$ and $\bigwedge_{j \in J}b_{j} \not\in con(B)$. But then we have
\[ \lsem a \rsem_{A \rightarrow B} \subseteq \lsem 
( \bigwedge_{j \in J}a_{j} \rightarrow \bigwedge_{j \in J}b_{j} ) 
\rsem_{A \rightarrow B} = \varnothing . \]
(iii) With notation as in (ii),
\[ \bot \not\in \lsem a \rsem_{A \rightarrow B} \; \Longleftrightarrow \; \exists i \in I. \, \bot \not\in \lsem (a_{i} \rightarrow b_{i}) \rsem_{A \rightarrow B}. \]
Now if $a_{i} \not\in con(A)$, 
\[ \bot \in 1_{K\Omega ([\hat{A} \rightarrow \hat{B}])} = \lsem (a_{i} \rightarrow b_{i})\rsem_{A \rightarrow B}; \]
while if $a_{i} \in con(A)$, $b_{i} \not\in con(B)$, then
\[ \bot \not\in \varnothing = \lsem (a_{i} \rightarrow b_{i})\rsem_{A \rightarrow B}. \]
Finally, if $a_{i} \in con(A)$ and $b_{i} \in con(B)$, then $\lsem (a_{i} \rightarrow b_{i})\rsem_{A \rightarrow B} = \diverges [u_{i}, v_{i}]$, and
\[ \bot \not\in \lsem (a_{i} \rightarrow b_{i})\rsem_{A \rightarrow B} \;
\Longleftrightarrow \; v_{i} \neq \bot \; \Longleftrightarrow \; b_{i} \in t(B). \]
\[ \mbox{Thus } \bot \not\in \lsem (a_{i} \rightarrow b_{i})\rsem_{A \rightarrow B} \; \Longleftrightarrow \; a_{i} \in con(A) \: \& \: b_{i} \in t(B).  \;\;\; \qed \]

As corollaries we have:
\[ \begin{array}{rl}
\mbox{(iv)} & {\sf CPNF}(\bigwedge_{i \in I}(a_{i} \rightarrow b_{i})) \; \Longrightarrow \; \lsem \bigwedge_{i \in I}(a_{i} \rightarrow b_{i})\rsem_{A \rightarrow B} = \diverges (\bigsqcup_{i \in I}[u_{i}, v_{i}]), \\
&  \mbox{where } \diverges u_{i} = \lsem a_{i} \rsem_{A}, \diverges v_{i} = \lsem b_{i} \rsem_{B}, i \in I. \\
\mbox{(v)} & \#(a) \; \Longleftrightarrow \; \lsem a \rsem_{A \rightarrow B} = \varnothing . \\
\mbox{(vi)} & a \converges \; \Longleftrightarrow \; \bot \not\in \lsem a \rsem_{A \rightarrow B}.
\end{array} \]

\begin{proposition}[T2]
$\forall a \in |A \rightarrow B|. \, \exists b \in {\sf CDNF}(A \rightarrow B). \, a =_{A \rightarrow B} b.$ 
\end{proposition}

\proof\ Using the distributive lattice laws, $a$ can be put in the form
\[ \bigvee_{i \in I} \bigwedge_{j \in J_{i}} (a_{ij} \rightarrow b_{ij}). \]
By $(d1)$, each $a_{ij}$ is equal to
\[ \bigvee_{k \in K_{ij}}c_{k}, \;\;\; (c_{k} \in pr(A), k \in K_{ij}), \]
and each $b_{ij}$ is equal to 
\[ \bigvee_{l \in L_{ij}}d_{l}, \;\;\; (d_{l} \in pr(B), l \in L_{ij}). \]
Moreover, we may assume that $c_{k} \in con(A)$ for all $k \in K_{ij}$, since otherwise
\[ \bigvee_{k \in K_{ij}}c_{k} =_{A} \bigvee_{k' \in K_{ij} - \{k\}}c_{k'}, \]
and so any inconsistent disjuncts can be deleted; and similarly for the $d_{l}$. Now
\begin{eqnarray*}
(\bigvee_{k \in K_{ij}}c_{k} \rightarrow \bigvee_{l \in L_{ij}}d_{l}) & =_{A \rightarrow B} &
\bigwedge_{k \in K_{ij}}(c_{k} \rightarrow \bigvee_{l \in L_{ij}}d_{l}) \;\;\; {\rm by } \; (\rightarrow - \vee -L) \\
& =_{A \rightarrow B} & \bigwedge_{k \in K_{ij}} \bigvee_{l \in L_{ij}} (c_{k} \rightarrow d_{l}) \;\;\; {\rm by } \; (\rightarrow - \vee - R).
\end{eqnarray*}
Using the distributive lattice laws again, we obtain the required normal form. \qed
\begin{proposition}[T3]
$\forall a, b \in |A \rightarrow B|. \, a \leq_{A \rightarrow B} \; \Rightarrow \; \lsem a \rsem_{A \rightarrow B} \subseteq \lsem b \rsem_{A \rightarrow B}.$ 
\end{proposition}

\proof\ $\lsem \rsem_{A \rightarrow B}$ preserves meets and joins by definition, and $(p1)$--$(p4)$ are valid in any distributive lattice. Moreover, given any spaces $X$, $Y$ and subsets $U \subseteq X$, $V \subseteq Y$,
\[ U' \subseteq U, V \subseteq V' \; \Longleftrightarrow \; (U, V) \subseteq (U', V') \]
\[ (U, \bigcap_{i \in I}V_{i}) = \bigcap_{i \in I}(U, V_{i}) \]
\[ (\bigcup_{i \in I}U_{i}, V) = \bigcap_{i \in I}(U_{i}, V) \]
are simple set-theoretic calculations. The soundness of ($\rightarrow$-$\#$) 
follows from Corollary (v) to Proposition~\ref{funT1}. 
Finally, suppose $a \in cpr(A)$. 
Then $\lsem a \rsem_{A} = \diverges  u$ with $u \in K(\hat{A})$, and
\begin{eqnarray*}
\lsem (a \rightarrow \bigvee_{i \in I}b_{i}) \rsem_{A \rightarrow B} & = & (\diverges u, \bigcup_{i \in I}\lsem b_{i} \rsem_{B}) \\
& = & \{ f : f(u) \in \bigcup_{i \in I}\lsem b_{i} \rsem_{B} \} \;\;\; \mbox{by monotonicity}\\
& = & \bigcup_{i \in I}\{ f : f(u) \in \lsem b_{i} \rsem_{B} \} \\
& = & \bigcup_{i \in I}(\diverges u, \lsem b_{i} \rsem_{B}) \\
& = & \lsem \bigvee_{i \in I}(a \rightarrow b_{i}) \rsem_{A \rightarrow B}
\end{eqnarray*} 
and so $(\rightarrow - \vee - R)$ is sound. \qed

\begin{proposition}[T4]
For $\bigwedge_{i \in I}(a_{i} \rightarrow b_{i})$, $\bigwedge_{j \in J}(a_{j} \rightarrow b_{j})$ in ${\sf CPNF}(A \rightarrow B)$:
\[ \lsem \bigwedge_{i \in I}(a_{i} \rightarrow b_{i}) \rsem_{A \rightarrow B} \subseteq \lsem \bigwedge_{j \in J}(a_{j} \rightarrow b_{j})\rsem_{A \rightarrow B} \]
implies
\[ \bigwedge_{i \in I}(a_{i} \rightarrow b_{i}) \leq_{A \rightarrow B} \bigwedge_{j \in J}(a_{j} \rightarrow b_{j}). \]
\end{proposition}

\proof\ By Corollary (iv) to Proposition~\ref{funT1},
\[  \lsem \bigwedge_{i \in I}(a_{i} \rightarrow b_{i}) \rsem_{A \rightarrow B} = \diverges \bigsqcup_{i \in I}[u_{i}, v_{i}], \]
\[  \lsem \bigwedge_{j \in J}(a_{j} \rightarrow b_{j}) \rsem_{A \rightarrow B} = \diverges \bigsqcup_{j \in J}[u_{j}, v_{j}], \]
where
\[ \diverges u_{i} = \lsem a_{i} \rsem_{A}, \ldots \; etc. \]
Now,
\[ \lsem \bigwedge_{i \in I}(a_{i} \rightarrow b_{i}) \rsem_{A \rightarrow B} \subseteq  \lsem \bigwedge_{j \in J}(a_{j} \rightarrow b_{j}) \rsem_{A \rightarrow B} \]
\[ \Longleftrightarrow \;\; \bigsqcup_{j \in J}[u_{j},v_{j}] \sqsubseteq 
\bigsqcup_{i \in I}[u_{i}, v_{i}] \]
\[ \Longleftrightarrow \;\; \forall j \in J. \, v_{j} \sqsubseteq \bigsqcup \{ v_{i} : u_{i} \sqsubseteq u_{j} \} \]
\[ \Longleftrightarrow \;\; \forall j \in J. \, \lsem \bigwedge \{ b_{i} : \lsem a_{j} \rsem_{A} \subseteq \lsem a_{i} \rsem_{A} \} \rsem_{B} \subseteq \lsem b_{j} \rsem_{B} \]
\[ \Longleftrightarrow \;\; \forall j \in J. \, \bigwedge \{b_{i} : a_{j} \leq_{A} a_{i} \} \leq_{B} b_{j} \;\;\; (*). \]
Thus, for all $j \in J$:
\begin{Eqarray}
\bigwedge_{i \in I}(a_{i} \rightarrow b_{i}) & \leq_{A \rightarrow B} & \bigwedge \{ (a_{i} \rightarrow b_{i}) : a_{j} \leq_{A} a_{i} \} & \mbox{by (p3)} \\
& \leq_{A \rightarrow B} & \bigwedge \{ (a_{j} \rightarrow b_{i}) : a_{j} \leq_{A} a_{i} \} & \mbox{by $(\rightarrow - \leq )$} \\
& =_{A \rightarrow B} & (a_{j} \rightarrow \bigwedge \{ b_{i} : a_{j} \leq_{A} a_{i} \}) & \mbox{by $(\rightarrow - \wedge )$} \\
& \leq_{A \rightarrow B} & (a_{j} \rightarrow b_{j}) & \mbox{by (*)}
\end{Eqarray}
and so by $(p2)$
\[ \bigwedge_{i \in I}(a_{i} \rightarrow b_{i}) \leq_{A \rightarrow B} \bigwedge_{j \in J}(a_{j} \rightarrow b_{j}). \;\;\; \qed \]

\begin{proposition}[T5] 
$\forall U \in K\Omega ([\hat{A} \rightarrow \hat{B}]). \, \exists a \in |A \rightarrow B|. \, \lsem a \rsem_{A \rightarrow B} = U.$ 
\end{proposition}

\proof\ Directly from Propositions~\ref{cop} and~\ref{funT1}. \qed

\begin{proposition}[T6]
Given $A \trianglelefteq A'$, $B \trianglelefteq B'$, let $e_{1} : \hat{A} \rightarrow \hat{A'}$, $e_{2} : \hat{B} \rightarrow \hat{B'}$ be the corresponding embeddings. Then
\[ (A) \;\;\; (e_{1} \rightarrow e_{2})^{\dag} \circ \lsem \cdot \rsem_{A \rightarrow B} = \lsem \cdot \rsem_{A' \rightarrow B'} \]
\[ (B) \;\;\; (e_{1} \rightarrow e_{2}) \circ \eta_{A \rightarrow B} =  \eta_{A' \rightarrow B'} \circ \converges (\cdot ). \]
\end{proposition}

\proof\ Firstly, we recall the definition of $e_{1} \rightarrow e_{2}$:
\[ (e_{1} \rightarrow e_{2})(f) = e_{2} \circ f \circ e_{1}^{R}, \]
where $e_{1}^{R}$ is the right adjoint of $e_{1}$, i.e. the corresponding projection. Now in fact we can eliminate the use of the projection in describing $(e_{1} \rightarrow e_{2})^{\dag}$, since we have
\[ (e_{1} \rightarrow e_{2})(\bigsqcup_{i \in I}[u_{i}, v_{i}]) = \bigsqcup_{i \in I}[e_{1}(u_{i}), e_{2}(v_{i})]. \]
Indeed,
\[ \begin{array}{cl}
  & (e_{1} \rightarrow e_{2})(\bigsqcup_{i \in I}[u_{i}, v_{i}])(d) \\ 
= & e_{2} \circ \bigsqcup_{i \in I}[u_{i}, v_{i}] \circ e_{1}^{R} (d) \\
= & e_{2}(\bigsqcup_{i \in I}\{ v_{i} : u_{i} \sqsubseteq e_{1}^{R}(d) \}) \\
= & e_{2}(\bigsqcup_{i \in I}\{ v_{i} : e_{1}(u_{i}) \sqsubseteq d \} ) \\
= & \bigsqcup_{i \in I}\{ e_{2}(v_{i}) : e_{1}(u_{i}) \sqsubseteq d \} \\
  & \mbox{($e_{2}$ preserves joins since it is a left adjoint)} \\
= & (\bigsqcup_{i \in I}[e_{1}(u_{i}), e_{2}(v_{i})])(d).
\end{array} \] 
Now for (A), given
\[ a =_{A \rightarrow B} \bigvee_{i \in I} \bigwedge_{j \in J_{i}}(a_{ij} \rightarrow b_{ij}) \in {\sf CDNF}(A \rightarrow B), \]
we calculate
\begin{eqnarray*}
(e_{1} \rightarrow e_{2})^{\dag} \lsem a \rsem_{A \rightarrow B} & = & \bigcup_{i \in I} \bigcap_{j \in J_{i}} (e_{1}^{\dag} \lsem a_{ij} \rsem_{A}, e_{2}^{\dag}\lsem b_{ij} \rsem_{B}) \\
& = & \bigcup_{i \in I} \bigcap_{j \in J_{i}} (\lsem a_{ij} \rsem_{A'}, \lsem b_{ij} \rsem_{B'}) \;\;\; \mbox{by~\ref{embim}} \\
& = & \lsem a \rsem_{A' \rightarrow B'}.
\end{eqnarray*} 
Similarly for (B) we have:
\[ \begin{array}{cl}
  & (e_{1} \rightarrow e_{2}) \circ \eta_{A \rightarrow B}(x) \\ 
= & \bigsqcup \{ [u, v] : \exists (a \rightarrow b) \in x. \, \diverges u = \lsem a \rsem_{A} \: \& \: \diverges v = \lsem b \rsem_{B} \} \\
= & \bigsqcup \{ [u, v] : \exists (a \rightarrow b) \in x. \, \diverges u = \lsem a \rsem_{A'} \: \& \: \diverges v = \lsem b \rsem_{B'} \} \\
= & \eta_{A' \rightarrow B'}( \converges (x)). \;\;\; \qed
\end{array} \] 

To illustrate the uniformity in our treatment of all the type constructions, we shall deal with two more: the upper or Smyth powerdomain, and the coalesced sum.

\begin{definition} 
{\rm The {\it upper powerdomain} $P_{u}(A)$.

\noindent (i) The generators:
\[ G(P_{u}(A)) \; \equiv \; \{ \Box a | a \in |A| \]

\noindent (ii) Metapredicates:
\begin{eqnarray*}
{\sf PNF}(P_{u}(A)) & \equiv & \{ \Box \bigvee_{i \in I}a_{i} : a_{i} \in pr(A), i \in I \} \\
{\sf CON}(t) & & \\
{\sf CON}(\bigwedge_{i \in I} \Box \bigvee_{j \in J_{i}} a_{ij}) & \equiv & \exists 
f \in \prod_{i \in I}J_{i}. \, \bigwedge_{i \in I}a_{i, f(i)} \in con(A) \\
{\sf T}(\bigwedge_{i \in I} \Box \bigvee_{j \in J_{i}} a_{ij}) & \equiv & \exists i \in I. \, \forall j \in J_{i}. \, a_{ij} \in t(A) \\
{\sf CPNF}( \Box \bigvee_{i \in I}a_{i}) & \equiv & {\sf CON}(\Box \bigvee_{i \in I}a_{i}) 
\: \& \: I \neq \varnothing \\
& & \& \: \forall i \in I. \, a_{i} \in con(A)
\end{eqnarray*} 

\noindent (iii) Axioms in addition to $(p1)$ -- $(p4)$:
\[ (\Box - \leq) \;\;\; \frac{a \leq b}{\Box a \leq \Box b} \]
\[ (\Box - \wedge) \;\;\; \Box \bigwedge_{i \in I}a_{i} = \bigwedge_{i \in I} \Box a_{i} \]
\[ (\Box - 0) \;\;\; \Box 0 = 0 \]

\noindent (iv) The semantic function:
\[ \lsem \cdot \rsem_{P_{u}(A)} : | P_{u}(A) | \longrightarrow K\Omega (P_{u}(\hat{A})) \]
\[ \lsem \Box a \rsem_{P_{u}(A)} = \{ S \in P_{u}(\hat{A}) : S \subseteq \lsem a \rsem_{A} \} \]
(The further clauses are the standard ones described in the definition of function space.)}
\end{definition}
\begin{proposition}[T1]
\label{pdomT1}
For all $a, \{ a_{i}\}_{i \in I} \in {\sf PNF}(P_{u}(A))$:
\[ \begin{array}{rl}
(i)   & \lsem a \rsem_{P_{u}(A)} \in pr(K\Omega (P_{u}(A))) \\
(ii)  & {\sf CON}(\bigwedge_{i \in I}a_{i}) \; \Longleftrightarrow \; \lsem \bigwedge_{i \in I}a_{i} \rsem_{P_{u}(A)} \neq \varnothing \\
(iii) & {\sf T}(\bigwedge_{i \in I}a_{i}) \; \Longleftrightarrow \; \bot \not\in \lsem \bigwedge_{i \in I}a_{i} \rsem_{P_{u}(A)}  \\
\end{array} \]
\end{proposition}

\proof\ $(i)$. Let $\Box \bigvee_{i \in I}a_{i} \in {\sf PNF}(P_{u}(A))$. Then either $\bigvee_{i \in I}a_{i} \not\in con(A)$, and
\[ \lsem \Box \bigvee_{i \in I}a_{i} \rsem_{P_{u}(A)} = \varnothing \in  pr(K\Omega (P_{u}(A))); \]
or for some $X \subseteq_{\sf f} {\cal K}(\hat{A})$, $X \neq \varnothing$ and
\[ \lsem \bigvee_{i \in I}a_{i} \rsem_{A} = \diverges_{\hat{A}} X. \]
In the latter case,
\begin{eqnarray*}
\lsem \Box \bigvee_{i \in I}a_{i} \rsem_{P_{u}(A)} & = & \{ S \in P_{u}(\hat{A}) : S \subseteq \lsem \bigvee_{i \in I}a_{i} \rsem_{A} \} \\
& = & \{ S \in P_{u}(\hat{A}) : \diverges_{\hat{A}} X \sqsubseteq_{u} S \} \\
& = & \diverges_{P_{u}(\hat{A})}(\lsem \bigvee_{i \in I}a_{i} \rsem_{A}).
\end{eqnarray*} 
(ii) Firstly,
\[ \lsem \bigwedge_{i \in I} \Box \bigvee_{j \in J_{i}} a_{ij}  \rsem_{P_{u}(A)} = \lsem \Box \bigvee_{f \in \prod_{i \in I}J_{i}} \bigwedge_{i \in I}a_{i, f(i)} \rsem_{P_{u}(A)}, \]
by $(\Box - \wedge )$ (see the proof of (T3)) and distributivity. Now by (i),
\[ \lsem \Box \bigvee_{f \in \prod_{i \in I}J_{i}} \bigwedge_{i \in I}a_{i, f(i)} \rsem_{P_{u}(A)} \neq \varnothing \]
\[ \Longleftrightarrow \;\; \lsem \bigvee_{f \in \prod_{i \in I}J_{i}} \bigwedge_{i \in I}a_{i, f(i)} \rsem_{A} \neq \varnothing \]
\[ \Longleftrightarrow \;\; \exists f \in \prod_{i \in I}J_{i}. \, \bigwedge_{i \in I}a_{i, f(i)} \in con(A). \]
(iii) This follows from the fact that
\[ \bot \not\in \lsem \Box a \rsem_{P_{u}(A)} \; \Longleftrightarrow \; \bot \not\in \lsem a \rsem_{A}. \;\;\; \qed \]

\begin{proposition}[T2] 
\label{pdomT2}
$\forall a \in |P_{u}(A)|. \, \exists b \in {\sf CDNF}(P_{u}(A)). \, a =_{P_{u}(A)} b.$
\end{proposition}

\proof\ We can use the distributive lattice laws to put $a$ in the form
\[ \bigvee_{i \in I} \bigwedge_{j \in J_{i}} \Box a_{ij}. \]
By $(d1)$, each $a_{ij}$ can be written as
\[ \bigvee_{k \in K_{ij}}b_{k}, \]
where each $b_{k} \in cpr(A)$. We can now use $(\Box - \wedge )$ and the distributive laws to obtain an expression of the form
\[ \bigvee_{i' \in I'} \Box \bigvee_{l \in L_{i'}} c_{l}, \]
where each $c_{l} \in cpr(A)$. Moreover disjuncts with $L_{i'} = \varnothing$ can be deleted using $(\Box - 0)$. This yields the required normal form. \qed

\begin{proposition}[T3]
For all $a, b \in |P_{u}(A)|$:
\[ a \leq_{P_{u}(A)} b \; \Longrightarrow \; \lsem a \rsem_{P_{u}(A)} \subseteq \lsem b \rsem_{P_{u}(A)}. \]
\end{proposition}

\proof\ Given $U \in K\Omega (\hat{A}))$, define
\[ \Box U \; \equiv \; \{ S \in P_{u}(\hat{A}) : S \subseteq U \}. \]
Then
\[ U \subseteq V \; \Longrightarrow \Box U \subseteq \Box V, \]
\[ \Box \bigcap_{i \in I}U_{i} = \bigcap_{i \in I} \Box U_{i} \]
are simple set calculations, which validate $(\Box - \leq )$ and 
$(\Box - \wedge )$. $(\Box - 0)$ is valid because the empty set is 
excluded from $P_{u}(\hat{A})$. 
(In fact, dropping $(\Box - 0)$ exactly corresponds to retaining the empty set). \qed

\begin{proposition}[T4]
For all $\Box a, \Box b \in {\sf CPNF}(P_{u}(A))$:
\[ \lsem \Box a \rsem_{P_{u}(A)} \subseteq \lsem \Box b \rsem_{P_{u}(A)} \; \Longrightarrow \; \Box a \leq_{P_{u}(A)} \Box b. \]
\end{proposition}

\proof\ Using the description of $\lsem \Box a \rsem_{P_{u}(A)}$, $\lsem \Box b \rsem_{P_{u}(A)}$ from the proof of Proposition~\ref{pdomT1}(i),
\[ \lsem \Box a \rsem_{P_{u}(A)} \subseteq \lsem \Box b \rsem_{P_{u}(A)} \]
\[ \Longrightarrow \;\; \lsem a \rsem_{A} \subseteq \lsem b \rsem_{A} \]
\[ \Longrightarrow \;\; a \leq_{A} b \]
\[ \Longrightarrow \;\; \Box a \leq_{P_{u}(A)} \Box b \;\;\; (\Box - \leq). \;\;\; \qed \]

\begin{proposition}[T6(A)]
Let $A \trianglelefteq B$, with $e : \hat{A} \rightarrow \hat{B}$ the corresponding projection. Then
\[ (P_{u}(e))^{\dag} \circ \lsem \cdot \rsem_{P_{u}(A)} = \lsem \cdot \rsem_{P_{u}(B)}. \]
\end{proposition}

\proof\ From the proof of Proposition~\ref{pdomT1}(i), for $a \in con(A)$:
\[ (*) \;\;\;\; \lsem \Box a \rsem_{P_{u}(A)} = \diverges_{P_{u}(A)} \lsem a \rsem_{P_{u}(A)}, \]
while for $a \in con(A)$ we have, directly from the definitions,
\[ (**) \;\;\; P_{u}(e)(\lsem a \rsem_{A}) = e^{\dag}(\lsem a \rsem_{A}). \]
Now given $a \in |P_{u}(A)|$, by \ref{pdomT2}
\[ a =_{P_{u}(A)} \bigvee_{i \in I} \Box a_{i}, \;\;\; (a_{i} \in con(A), i \in I), \]
and we can calculate:
\begin{Eqarray}
P_{u}(e)^{\dag}(\lsem a \rsem_{P_{u}(A)}) & = & \bigcup_{i \in I}P_{u}(e)^{\dag}( \lsem \Box a_{i} \rsem_{P_{u}(A)}) & \\
& = & \bigcup_{i \in I}P_{u}(e)^{\dag}( \diverges_{P_{u}(\hat{A})} \lsem a_{i} \rsem_{A}) & (*) \\
& = & \bigcup_{i \in I} \diverges_{P_{u}(\hat{B})} (P_{u}(e) \lsem a_{i} \rsem_{A}) & \\
& = & \bigcup_{i \in I} \diverges_{P_{u}(\hat{B})} (e^{\dag} \lsem a_{i} \rsem_{A}) & (**) \\
& = & \bigcup_{i \in I} \diverges_{P_{u}(\hat{B})} (\lsem a_{i} \rsem_{B}) & \ref{embim} \\
& = & \bigcup_{i \in I} \lsem \Box a_{i} \rsem_{P_{u}(B)} & (*) \\
& = & \lsem a \rsem_{P_{u}(B)}. & \qed 
\end{Eqarray}

\begin{definition} 
{\rm The {\it coalesced sum.}

\noindent (i) The generators:
\[ G(A \oplus B) \; \equiv \; \{ (a \oplus \false ) : a \in |A| \} \cup \{ (\false \oplus b) : b \in |B| \}. \]

\noindent (ii) Metapredicates:
\[ {\sf PNF}(A \oplus B) \; \equiv \; \{ (a \oplus \false ) : a \in pr(A) \} \cup \{ (\false \oplus b) : b \in pr(B) \} \cup \{ \true \} \]
\[ {\sf CON}( \true ) \]
\begin{eqnarray*}
{\sf CON}( \bigwedge_{i \in I}(a_{i} \oplus \false ) \wedge \bigwedge_{j \in J}(\false \oplus b_{j})) & \equiv & 
\neg (\bigwedge_{i \in I}a_{i} \in t(A) 
\: \& \: \bigwedge_{j \in J}b_{j} \in t(B)) \\
& & \& \: \bigwedge_{i \in I}a_{i} \in con(A) \\ 
& & \& \: \bigwedge_{j \in J}b_{j} \in con(B)  
\end{eqnarray*}
\[ {\sf T}( \bigwedge_{i \in I}(a_{i} \oplus \false ) \wedge \bigwedge_{j \in J}(\false \oplus b_{j})) \; \equiv \; \exists i \in I. \, a_{i} \in t(A) \; {\rm or} \; \exists j \in J. \, b_{j} \in t(B) \]
\[ {\sf CPNF}(a) \; \equiv \; {\sf CON}(a) \]

\noindent (iii) Axioms:
\[ (\oplus - \leq ) \;\;\; \frac{a \leq b}{(a \oplus \false ) \leq (b \oplus \false )} \;\;\;\;\; \frac{a \leq b}{(\false \oplus a) \leq (\false \oplus b)} \]
\[ (\oplus - \wedge ) \;\;\; \bigwedge_{i \in I}(a_{i} \oplus \false ) = (\bigwedge_{i \in I}a_{i} \oplus \false ) \;\;\;\;\; \bigwedge_{i \in I}(\false \oplus a_{i} ) = (\false \oplus \bigwedge_{i \in I}a_{i}) \]
\[ (\oplus - \vee ) \;\;\; \bigvee_{i \in I}(a_{i} \oplus \false ) = (\bigvee_{i \in I}a_{i} \oplus \false ) \;\;\;\;\; \bigvee_{i \in I}(\false \oplus a_{i} ) = (\false \oplus \bigvee_{i \in I}a_{i}) \]
\[ (\oplus - \#) \;\;\; a \leq \false \;\;\;\;\; (\#(a)) \]

\noindent (iv) Semantic function:}
\[ \lsem \cdot \rsem_{A \oplus B} : |A \oplus B| \longrightarrow K\Omega (\hat{A} \oplus \hat{B}) \]
\begin{eqnarray*}
\lsem (a \oplus \false ) \rsem_{A \oplus B} & = &  \{ <0, d> : d \in \lsem a \rsem_{A}, d \neq \bot \} \\
& & \mbox{} \cup  \{ x \in \hat{A} \oplus \hat{B} : \bot \in \lsem a \rsem_{A} \}
\end{eqnarray*} 
\begin{eqnarray*}
\lsem (\false \oplus b) \rsem_{A \oplus B} & = & \{ <1, d> : d \in \lsem b \rsem_{B}, d \neq \bot \} \\
& & \mbox{} \cup  \{ x \in \hat{A} \oplus \hat{B} : \bot \in \lsem b \rsem_{B} \}
\end{eqnarray*} 
\end{definition}

\begin{proposition}[T1]
For all $c, \{ c_{i} \}_{i \in I} \in {\sf PNF}(A \oplus B)$:
\[ \begin{array}{rl}
(i)   & \lsem c \rsem_{A \oplus B} \in pr(K\Omega (\hat{A} \oplus \hat{B})) \\
(ii)  & {\sf CON}(\bigwedge_{i \in I}c_{i}) \; \Longleftrightarrow \; \lsem \bigwedge_{i \in I}c_{i} \rsem_{A \oplus B} \neq \varnothing \\
(iii) & {\sf T}(\bigwedge_{i \in I}c_{i}) \; \Longleftrightarrow \; \bot \not\in \lsem \bigwedge_{i \in I}c_{i} \rsem_{A \oplus B}.
\end{array} \]
\end{proposition}

\proof\ (i) If $c = (a \oplus \false )$, $a \in pr(A)$, we can distinguish three cases: \\
(1): $a \not\in con(A)$. In this case, 
\[ \lsem c \rsem_{A \oplus B} = \varnothing . \]
(2): $\lsem a \rsem_{A} = 1_{K\Omega (\hat{A})} = \diverges (\bot )$. In this case,
\[ \lsem c \rsem_{A \oplus B} = \diverges (\bot ) \in pr(K\Omega (\hat{A} \oplus \hat{B})). \]
(3): $a \in con(A)$, $\bot \not\in \lsem a \rsem_{A}$. In this case, for some $u \in K(\hat{A})$, $u \neq \bot$, $\lsem a \rsem_{A} = \diverges u$. Then
\begin{eqnarray*}
\lsem c \rsem_{A \oplus B} & = & \{ <0, d> : u \sqsubseteq d \} \\
& = & \diverges_{\hat{A} \oplus \hat{B}}(<0, u>).
\end{eqnarray*} 
The case for $c = (\false \oplus b)$ is similar.

(ii), (iii). Straightforward. \qed

\begin{proposition}[T2]
$\forall a \in |A \oplus B|. \, \exists b \in {\sf CDNF}(A \oplus B). \, a =_{A \oplus B} b.$
\end{proposition}

\proof\ We can use the distributive lattice laws to put $a$ in the form
\[ \bigvee_{i \in I}(\bigwedge_{j \in J_{i}}(a_{ij} \oplus \false ) 
\wedge \bigwedge_{k \in K_{i}}(\false \oplus b_{ik})). \]
Moreover, we can write each $a_{ij}$ as $\bigvee_{l \in L_{ij}}c_{l}$, $b_{ik}$ as $\bigvee_{m \in M_{ik}}d_{m}$, with $c_{l} \in cpr(A)$, $d_{m} \in cpr(B)$. Using $(\oplus - \vee )$, we obtain
\[ \bigvee_{i \in I'}(\bigwedge_{j \in {J_{i}}'}(a_{ij} \oplus \false ) \wedge \bigwedge_{k \in {K_{i}}'}(\false \oplus b_{ik})) \]
with $a_{ij} \in cpr(A)$, $b_{ik} \in cpr(B)$. Now using $(\oplus - \wedge )$, we obtain
\[ \bigvee_{i \in I'}((\bigwedge_{j \in {J_{i}}'}a_{ij} \oplus \false ) \wedge (\false \oplus \bigwedge_{k \in {K_{i}}'} b_{ik})). \]
For each $i \in I'$, if both 
\[ \bigwedge_{j \in {J_{i}}'}a_{ij} \in t(A) \] 
and 
\[ \bigwedge_{k \in {K_{i}}'}b_{ik} \in t(B), \] 
we may delete the $i$'th disjunct by $(\oplus - \# )$. If either 
\[ \bigwedge_{j \in {J_{i}}'}a_{ij} \not\in con(A) \]
or 
\[ \bigwedge_{k \in {K_{i}}'}b_{ik} \not\in con(B), \] 
we can delete the $i$'th disjunct by $(\oplus - \vee )$. Otherwise, either 
\[ \bigwedge_{j \in {J_{i}}'}a_{ij} =_{A} 1_{A} \]
or 
\[ \bigwedge_{k \in {K_{i}}'}b_{ik} =_{B} 1_{B}, \] 
and we can delete one of these conjuncts by $(\oplus - \wedge )$. In this way we obtain an expression of the form
\[ \bigvee \{ (a \oplus \false ) \} \vee \bigvee \{ (\false \oplus b) \} , \]
with each $a \in cpr(A)$, $b \in cpr(B)$, as required. \qed

\begin{proposition}[T4]
For all $c, d \in {\sf CPNF}(A \oplus B)$:
\[ \lsem c \rsem_{A \oplus B} \subseteq \lsem d \rsem_{A \oplus B} \; \Longrightarrow \; c \leq_{A \oplus B} d. \]
\end{proposition}

\proof\ Take $c = (a \oplus \false )$. We consider two subcases. \\
(1): $d = (b \oplus \false )$.
\begin{eqnarray*}
\lsem c \rsem_{A \oplus B} \subseteq \lsem d \rsem_{A \oplus B} & \Longrightarrow  & \lsem a \rsem_{A} \subseteq \lsem b \rsem_{A} \\
& \Longrightarrow & a \leq_{A} b \\
& \Longrightarrow & (a \oplus \false ) \leq_{A \oplus B} (b \oplus \false ) \;\;\; {\rm by} \; (\oplus - \leq ). 
\end{eqnarray*} 
(2): $d = (\false \oplus b)$.
\begin{eqnarray*}
\lsem c \rsem_{A \oplus B} \subseteq \lsem d \rsem_{A \oplus B} & \Longrightarrow  & \bot \in \lsem b \rsem_{B} \\
& \Longrightarrow & \true \leq_{B} b \\
& \Longrightarrow & c \leq_{A \oplus B} \true \\
& &  =_{A \oplus B} (\false \oplus \true ) \;\;\; (\oplus - \wedge) \\
& &  \leq_{A \oplus B} (\false \oplus b) \;\;\; (\oplus - \leq ).
\end{eqnarray*} 
The case for $c = (\false \oplus a)$ is similar. \qed
