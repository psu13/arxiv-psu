\chapter{Domain Theory In Logical Form}
\section{Introduction}
In this Chapter we shall complete the core of our 
research programme, as set out in Chapter~1.
We shall introduce a meta-language for denotational semantics,
give it a logical interpretation {\it via} the localic side of Stone duality,
and relate this logical interpretation to the standard denotational one
by showing that they are Stone duals of each other.

Denotational semantics is always based, more or less explicitly,
on a typed functional meta-language.
The types are interpreted as topological spaces
(usually domains in the sense of Scott \cite{Sco81,Sco82}, but
sometimes metric spaces, as in \cite{deBZ82,Niv81}), while the
terms denote elements of or functions between these spaces.
A {\em program logic} comprises an assertion language of formulas
for expressing properties of programs, and an interface between these
properties and the programs themselves.
Two main types of interface can be identified \cite{Pnu77}:
\begin{description}
\item[Endogenous logic] In this style, formulas describe properties
pertaining to the ``world'' of a single program.
Notation: \[ P \models \phi \]
where $P$ is a program and $\phi$ is a formula. Examples:
temporal logic as used e.g. in \cite{Pnu77}; Hennessy-Milner logic \cite{HM85};
type inference \cite{DM82}.
\item[Exogenous logic] Here, programs are embedded in formulas as
{\em modal operators}. Notation: \[ [P]\phi \]
where $P$ is now a program denoting a function or relation.
Examples: dynamic logic \cite{Har79,Pra79}, including as special cases
Hoare logic \cite{Hoa69}, since ``Hoare triples''
$\{ \phi \} P \{ \psi \}$ can be represented by
\[ \phi \rightarrow [P] \psi , \]
and Dijkstra's wlp-calculus \cite{Dij76}, since $wlp( P, \psi )$
can be represented as $[P] \psi$. (Total correctness assertions can also
be catered for; see \cite{Har79}.)
\end{description}

Extensionally, formulas denote sets of points in our denotational
domains, i.e. $\phi$ is a syntactic description of
$\{ x : x \: {\rm satisfies} \: \phi \}$.
Then $P \models \phi$ can be interpreted as $x \in U$, where
$x$ is the point denoted by $P$, and $U$ is the set denoted by 
$\phi$.
Similarly, $[M] \phi$ can be interpreted as $f^{-1}(U)$, where $f$
is the function denoted by $M$ (and elaborations of this when
$M$ denotes a relation or multifunction).
In this way, we can give a topological interpretation of program
logic.

But this is not all: duality cuts both ways.
We can also use it to give a {\em logical interpretation of
denotational semantics}.
Rather than starting with the denotational domains as spaces of points,
and then interpreting formulas as sets of points, we can give an
axiomatic presentation of the topologies on our spaces,
viewed as abstract lattices (logical theories), and then reconstruct
the points from the properties they satisfy.
In other words, we can present denotational semantics in axiomatic
form, as a logic of programs.
This has a number of attractions:
\begin{itemize}
\item It unifies semantics and program logic in a general and systematic
setting.
\item It extends the scope of program logic to the entire range of
denotational semantics -- higher-order functions, recursive types,
powerdomains etc.
\item The syntactic presentation of recursive types, powerdomains etc.
makes these constructions more ``visible'' and easier to calculate with. 
\item The construction of ``points'', i.e. denotations of computational
processes, from the properties they satisfy is very compatible
with work currently being done in a mainly operational setting in
concurrency \cite {HM85,Win80} and elsewhere \cite{BC85}, and offers
a promising approach to unification of this work with denotational semantics.
\end{itemize}

The setting we shall take for our work in this Chapter is {\bf SDom}, the category of Scott domains.
The significance of this as far as the meta-language is concerned is that 
we omit
the Plotkin powerdomain construction.
However, this construction will be treated, in the context of a particular domain equation,
in Chapter 5.
Our reason for not including the Plotkin powerdomain, and extending the duality
to {\bf SFP}, is that this creates some additional technical complications,
though certainly not insuperable ones; lack of time and energy supervened.
For further discussion, see Chapter 7.

The remainder of the Chapter is organised as follows.
In section~2, we interpret the types of our denotational meta-language
as propositional theories.
We can then apply the results of Chapter~3 to show that each such theory is the Stone dual of the domain obtained as the denotation of the type in the standard interpretation.
In section~3, we extend the meta-language to include typed terms, i.e. {\em functional programs}.
We extend our logic to an axiomatisation of the satisfaction relation $P \models \phi$ ($P$ a term, $\phi$ a formula of the logic introduced in section 2),
and prove that this axiomatisation is sound and complete with respect to the
spatial interpretation $x \in U$, where $x$ is the point denoted by $P$, and $U$ the open set denoted by $\phi$.
In section~4, we consider an alternative formulation of the meta-language, 
in which terms are formed at the morphism level rather than the element level;
the comparison between these formulations extends the standard one between
$\lambda$-calculus (element level) and cartesian closed categories (morphism level).
We find a pleasing correspondence between the two known, but hitherto quite unrelated, dichotomies:
\begin{center}
\begin{tabular}{ccc}
cartesian closed categories & & exogenous logic \\
{\it vs.} & ${\Large \sim}$ & {\it vs.} \\
$\lambda$-calculus & & endogenous logic.
\end{tabular}
\end{center}
Our axiomatisation of the morphism-level language comprises an extended and 
generalised {\em dynamic logic} \cite{Pra79,Har79}.
We prove a restricted Completeness Theorem for this axiomatisation,
and show that the general validity problem for this logic is undecidable.
Finally, in section~5 we indicate how the results of this Chapter pave the way
for a whole class of applications, and set the scene for the two case studies
to be described in Chapters~5 and~6.

