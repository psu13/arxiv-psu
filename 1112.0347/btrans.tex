\section{Transition Systems and Related Notions}

We begin with the basic notion of a labelled transition system (with divergence), which abstracts from the operational semantics of many concurrent calculi.

\begin{definition}
{\rm A {\em transition system} is a structure
\[ ({\rm Proc}, {\sf Act}, \rightarrow, \diverges ) \]
where:
\begin{itemize}

\item ${\rm Proc}$ is a set of {\em processes} or {\em agents}.

\item ${\sf Act}$ is a set of atomic {\em actions} or {\em experiments}.

\item ${\rightarrow} \subseteq {\rm Proc} \times {\sf Act} \times {\rm Proc}$ (notation: $p \stackrel{a}{\rightarrow} q$).

\item $\diverges \subseteq {\rm Proc}$ (notation: $p \diverges$).
\end{itemize}}
\end{definition}

We write
\[ p \converges \equiv \neg (p \diverges ) . \]
We read $p \stackrel{a}{\rightarrow} q$ as ``$p$ has the capability to do $a$ and become (i.e. change state to) $q$''; $p \diverges$ as ``$p$ may diverge''; and $p \converges$ as ``$p$ definitely converges''.
We define
\[ {\sf sort}(p) \equiv \{ a \in {\sf Act} \: | \: \exists q, r . \, p \rightarrow^{\star} q \stackrel{a}{\rightarrow} r \} \]
where $p \rightarrow q \equiv \exists a \in {\sf Act} . \, p \stackrel{a}{\rightarrow} q$, and $\rightarrow^{\star}$ is the reflexive, transitive closure of $\rightarrow$.

We now define a number of finiteness conditions on transition systems:
\begin{center}
\begin{tabular}{ll}
{\bf image-finiteness} & $\forall p \in {\rm Proc} , a \in {\sf Act} . \, \{ q \: | \: p \stackrel{a}{\rightarrow} q \}$ is finite. \\
{\bf sort-finiteness} & $\forall p \in {\rm Proc} . \, {\sf sort}(p)$ is finite. \\
{\bf finite-branching} & $\forall p \in {\rm Proc} . \, \{ q \: | \: p \rightarrow q \}$ is finite. \\
{\bf initials-finiteness} & $\forall p \in {\rm Proc} . \, \{ a \in {\sf Act} \: | \: \exists q . \, p \stackrel{a}{\rightarrow} q \}$ is finite.
\end{tabular}
\end{center}

Each of these properties has a weak form, obtained by making it conditional on convergence. For example:
\begin{center}
\begin{tabular}{ll}
{\bf weak image-finiteness} &
$\forall p \in {\rm Proc} , a \in {\sf Act} . \, p \converges \; \Rightarrow \; \{ q \: | \: p \stackrel{a}{\rightarrow} q \}$ is finite. 
\end{tabular}
\end{center}

We now introduce a particularly useful source of examples for transition systems, the {\em synchronisation trees}.
Given a set ${\sf Act}$ of actions, ${\sf ST}_{\infty}({\sf Act})$, the synchronisation trees over ${\sf Act}$, are defined as the (proper) class of infinitary terms generated by the following inductive definition:
\begin{equation}
\frac{ \{ a_{i} \in {\sf Act}, t_{i} \in {\sf ST}_{\infty}({\sf Act}) \}_{i \in I}}{\sum_{i \in I} a_{i}t_{i} \; [ + \Omega ] \in {\sf ST}_{\infty}({\sf Act})} \label{st}
\end{equation}
where $[ + \Omega ]$ means optional inclusion of $\Omega$ as a summand (i.e. there are really two clauses in this definition). We write
\begin{eqnarray*}
\Oh & \equiv & \sum_{i \in \varnothing} a_{i}t_{i} \\
\Omega & \equiv & \sum_{i \in \varnothing} a_{i}t_{i} + \Omega .
\end{eqnarray*}

The subclass of terms formed using only finite sums is denoted 
${\sf ST}_{\omega}({\sf Act})$. 
Given a synchronisation tree $t$ formed according to \ref{st}, we stipulate:
\begin{itemize}
\item $t \diverges$ iff $\Omega$ is included as a summand.
\item $t \stackrel{a_{i}}{\rightarrow} t_{i}$ for each summand $a_{i}t_{i}$ $(i \in I)$.
\end{itemize}

This defines a (large) transition system $( {\sf ST}_{\infty}({\sf Act}) , {\sf Act}, \rightarrow , \diverges )$; restriction to a subset of synchronisation trees yields a small transition system. 
In particular, by choosing a canonical system of representatives for ${\sf ST}_{\omega}({\sf Act})$ which is closed under subtrees 
we obtain a countable transition system of finite synchronisation trees, which by abuse of notation we refer to also as ${\sf ST}_{\omega}({\sf Act})$.

We are now ready to introduce the main concept we will study.

\begin{definition}
{\rm (\cite{Par81,Mil80,Mil81})
A relation $R \subseteq {\rm Proc} \times {\rm Proc}$ is a {\em prebisimulation} if, for all $p, q \in {\rm Proc}$:
\[ \begin{array}{lrl}
p R q & \Longrightarrow & \forall a \in {\sf Act} . \\
& \bullet & p \stackrel{a}{\rightarrow} p' \;\; \Longrightarrow \;\; \exists q' . \, q \stackrel{a}{\rightarrow} q' \: \& \: p' R q' \\
& \bullet & p \converges \;\; \Longrightarrow \;\; q \converges \; \& \; [q \stackrel{a}{\rightarrow} q' \; \Rightarrow \; \exists p' . \, p \stackrel{a}{\rightarrow} p' \: \& \: p' R q'] .
\end{array} \]
We write
\[ p \preord^{B} q \equiv \exists R . \, R \; \mbox{is a prebisimulation and} \; p R q . \] }
\end{definition}

For an alternative description of $\preord^{B}$, let $Rel({\rm Proc})$ be the set of all binary relations over ${\rm Proc}$; this is a complete lattice under set inclusion. Now define
\[ F : Rel({\rm Proc}) \rightarrow Rel({\rm Proc}) \]
\[ \begin{array}{lrl}
F(R) & = & \{ (p, q) \: | \: \forall a \in {\sf Act} . \\
& & \bullet \; p \stackrel{a}{\rightarrow} p' \; \Rightarrow \; \exists q' . \, q \stackrel{a}{\rightarrow} q' \: \& \: p' R q' \\
& & \bullet \; p \converges \; \Rightarrow \; q \converges \: \& \:  [q \stackrel{a}{\rightarrow} q' \; \Rightarrow \; \exists p' . \, p \stackrel{a}{\rightarrow} p' \: \& \: p' R q'] \} .
\end{array} \]
Clearly, $R$ is a prebisimulation iff $R \subseteq F(R)$, i.e. $R$ is a {\em pre-fixed point of $F$}. Since $F$ is monotone, by Tarski's Theorem it has a maximal fixpoint, given by $\bigcup \{ R \: | \: R \subseteq F(R) \}$, i.e. $\preord^{B}$. Thus $\preord^{B}$ is itself a prebisimulation, and evidently the largest one. Moreover, it is reflexive and transitive; the corresponding equivalence is denoted $\sim^{B}$.

We can also describe $\preord^{B}$ more explicitly, in terms of iterations of $F$. We define relations $\preord_{\alpha}$, $(\alpha \in {\sf Ord})$ (the class of ordinals), by the following ordinal recursion:
\begin{itemize}
\item $p \preord_{0} q$ always (i.e. $\preord_{0} = {\rm Proc} \times {\rm Proc}$, the top element in the lattice $Rel({\rm Proc})$).
\item $p \preord_{\alpha + 1} q$ iff
\[ \begin{array}{l}
\forall a \in {\sf Act} . \\
\bullet \; p \stackrel{a}{\rightarrow} p' \;\; \Longrightarrow \;\; \exists q' . \, q \stackrel{a}{\rightarrow} q' \: \& \: p' \preord_{\alpha} q' \\
\bullet \; p \converges \;\; \Longrightarrow \;\; q \converges \; \& \;  [q \stackrel{a}{\rightarrow} q' \; \Rightarrow \; \exists p' . \, p \stackrel{a}{\rightarrow} p' \: \& \: p' \preord_{\alpha} q' ]  .
\end{array} \]
(i.e. $\preord_{\alpha + 1} = F(\preord_{\alpha})$).

\item For limit $\lambda$, $p \preord_{\lambda} q$ iff $\forall \alpha < \lambda . \, p \preord_{\alpha} q$ (i.e. $\preord_{\lambda} = \bigcap_{\alpha < \lambda} \preord_{\alpha}$).
\end{itemize}

This sequence of relations is decreasing, and bounded below by $\preord^{B}$; i.e. for all $\alpha$
\[ {\preord_{\alpha}} \supseteq {\preord_{\alpha + 1}} \supseteq {\preord^{B}} . \]
For any (small) transition system the sequence is eventually stationary; for some $\lambda$, for all $\alpha > \lambda$, $\preord_{\alpha} = \preord_{\lambda}$.
The least ordinal $\lambda$ for which this holds is called the {\em closure ordinal} \cite{Mos74}; and we have ${\preord_{\lambda}} = {\preord^{B}}$.
Note that each $\preord_{\alpha}$ is relexive and transitive.

The relations $\preord^{B}$ and $\sim^{B}$ have been defined in the context of a given transition system. However, we frequently want to use them to compare processes from different transition systems.
This is easily accomplished by forming the disjoint union of the two systems, and then using $\preord^{B}$ as defined above. In the sequel, we will do this without further comment.

We now introduce a program logic due to Hennessy and Milner \cite{HM85}. The idea is to obtain a characterisation of $\preord^{B}$ in terms of a suitable notion of {\em property} of process; $p \preord^{B} q$ iff every property satisfied by $p$ is satisfied by $q$.

\begin{definition}
{\rm Given a set of actions ${\sf Act}$, the language ${\rm HML}_{\infty}({\sf Act})$ (we henceforth elide the parameter ${\sf Act}$) is defined by the following inductive clauses:
\[ \frac{a \in {\sf Act} , \: \phi \in {\rm HML}_{\infty}}{[a] \phi ,  \ltuple a \rtuple \phi \in {\rm HML}_{\infty}} \]
\[ \frac{\phi_{i} \in {\rm HML}_{\infty} \: (i \in I)}{\bigwedge_{i \in I} \phi_{i} , \bigvee_{i \in I} \phi_{i} \in {\rm HML}_{\infty}} \]}
\end{definition}
In particular, we write:
\begin{eqnarray*}
\true & \equiv & \bigwedge_{i \in \varnothing} \phi_{i} \\
\false & \equiv & \bigvee_{i \in \varnothing} \phi_{i} .
\end{eqnarray*}
We use the subscript $\infty$ to indicate the presence of infinite conjunctions and disjunctions.
We write ${\rm HML}_{\omega}$ for the sublanguage obtained by restricting the formation rules to finite conjunctions and disjunctions.

We now define a satisfaction relation ${\models} \subseteq {\rm Proc} \times {\rm HML}_{\infty}$.
\[ \begin{array}{lcl}
p \models \bigwedge_{i \in I} \phi_{i} & \equiv & \forall i \in I . \, p \models \phi_{i} \\
p \models \bigvee_{i \in I} \phi_{i} & \equiv & \exists i \in I . \, p \models \phi_{i} \\
p \models {\ltuple a \rtuple} \phi & \equiv & \exists q . \,  \labarrow{p}{a}{q} \; \& \; q \models \phi \\
p \models {[ a ]} \phi & \equiv & \forall q . \,  \labarrow{p}{a}{q} \;\; \Longrightarrow \;\; q \models \phi . \\
\end{array} \]
We write
\begin{eqnarray*}
{\rm HML}_{\infty}(p) & \equiv & \{ \phi \in {\rm HML}_{\infty} : p \models \phi \}
\end{eqnarray*}
plus obvious variations on this notation.

We define two useful assignments of ordinals to formulas in ${\rm HML}_{\infty}$, the {\em modal depth}:
\[ \begin{array}{lllll}
{\sf md} (\bigwedge_{i \in I}\phi_{i} ) & \equiv & {\sf md} (\bigvee_{i \in I}\phi_{i} ) & \equiv & \sup \{ {\sf md}  ( \phi_{i} ) : i \in I \} \\
{\sf md} ( [ a ] \phi ) & \equiv & {\sf md}  ( \ltuple a \rtuple \phi ) & \equiv & {\sf md} ( \phi ) + 1 
\end{array} \]
and the {\em height}:
\[ \begin{array}{lllll}
{\sf ht}(\bigwedge_{i \in I}\phi_{i} ) & \equiv & {\sf ht} (\bigvee_{i \in I}\phi_{i} ) & \equiv & \sup \{ {\sf ht} ( \phi_{i} ) : i \in I \} + 1 \\
{\sf ht}( [ a ] \phi ) & \equiv & {\sf ht} ( \ltuple a \rtuple \phi ) & \equiv & {\sf ht}( \phi ) + 1 .
\end{array} \]

We define ${\sf sort}( \phi )$ to be the set of action symbols which occur in $\phi$.

Now given a set $A \subseteq {\sf Act}$ and an ordinal $\lambda$, we define a sublanguage of ${\rm HML}_{\infty}$:
\[ {\rm HML}^{(A, \lambda )}_{\infty} = \{ \phi \in {\rm HML}_{\infty} : {\sf sort}( \phi ) \subseteq A \; \& \; {\sf md} ( \phi ) \leq \lambda \} . \]

We are now ready to prove a generalised and strengthened version of the Modal Characterisation Theorem \cite{Mil81,Mil85,HM85}.

\begin{theorem}[Modal Characterisation Theorem]
\label{mct}
Suppose that $A \subseteq {\sf Act}$ satisfies
\[ {\sf sort}(p) \cup {\sf sort}(q) \subseteq A \not= \varnothing ; \]
then 
\[ p \preord_{\lambda} q \;\; \Longleftrightarrow \;\; {\rm HML}^{(A, \lambda )}_{\infty} (p) \subseteq  {\rm HML}^{(A, \lambda )}_{\infty} (q). \]
As an immediate consequence we obtain
\[  p \preord^{B} q \;\; \Longleftrightarrow \;\; {\rm HML}_{\infty} (p) \subseteq  {\rm HML}_{\infty} (q). \]
\end{theorem}

\proof\ The left-to-right implication is proved by induction on $\lambda$. 
The cases for $\lambda = 0$, $\lambda$ a limit ordinal are trivial. 
For $\lambda = \alpha + 1$, we argue by induction on ${\sf ht}(\phi )$. 
The cases for $\bigwedge_{i \in I} \phi_{i}$, $\bigvee_{i \in I} \phi_{i}$ are trivial. 
Suppose $p \models {\ltuple a \rtuple} \phi$. 
Then for some $p'$, $\labarrow{p}{a}{p'}$ and $p \models \phi$. 
Since $p \preord_{\lambda} q$, for some $q'$, $\labarrow{q}{a}{q'}$ and $p' \preord_{\alpha} q'$. 
By the outer induction hypothesis, $q' \models \phi$, hence $q \models {\ltuple a \rtuple} \phi$, as required. 
The case for ${[a]} \phi$ is similar.

For the converse, we argue by induction on $\lambda$. 
Suppose $p \npreord_{\lambda} q$: we must find $\phi \in  {\rm HML}^{(A, \lambda )}_{\infty} (p) -  {\rm HML}^{(A, \lambda )}_{\infty} (q)$. \\
Case 1: $\labarrow{p}{a}{p'}$ and for all $q'$, $\labarrow{q}{a}{q'}$ implies $p' \npreord_{\alpha} q'$ for some $\alpha < \lambda$. 
By induction hypothesis, for each such $q'$ there is $\phi \in  {\rm HML}^{(A, \alpha )}_{\infty} (p') -  {\rm HML}^{(A, \alpha )}_{\infty} (q')$. 
Now take
\[ \phi = {\ltuple a \rtuple} \bigwedge \{ \phi_{q'} : \labarrow{q}{a}{q'} \} . \]
Case 2: $p \converges$ and $p \diverges$. Take $\phi \equiv {[a]} \true$, for any $a \in A$. \\
Case 3: $p \converges$, $q \converges$, $\labarrow{q}{a}{q'}$, and for all $p'$, $\labarrow{p}{a}{p'}$ implies $p' \npreord_{\alpha} q'$ for some $\alpha < \lambda$. 
Defining $\phi_{p'}$ analogously to Case 1,
\[ \phi = {[a]} \bigvee \{ \phi_{p'} : \labarrow{p}{a}{p'} \} . \;\;\; \qed \]
The reader familiar with infinitary logic will recognise the strong similarity between this result and Karp's Theorem \cite{Barw75}.
Similar remarks apply to ``Master Formula Theorems'' as in \cite{Rou85}, {\em vis a vis}  the
Scott Isomorphism Theorem \cite{Barw75}.

Note that, if $A$ is a finite set and $\lambda$ a finite ordinal, then (up to logical equivalence) ${\rm HML}^{(A, \lambda )}_{\infty}$ is finite. 
It follows easily from this observation that each formula in ${\rm HML}^{(A, \lambda )}_{\infty}$ is equivalent to one in ${\rm HML}^{(A, \lambda )}_{\omega}$. 
Hence as a Corollary to the Characterisation Theorem we obtain
\begin{theorem}
\cite{Abr87b} If the transition system is sort-finite, then
\[ p \preord_{\omega} q \;\; \Longleftrightarrow \;\; {\rm HML}_{\omega}(p) \subseteq {\rm HML}_{\omega}(q) . \]
\end{theorem}

Moreover, we have the following result from \cite{HM85}:
\begin{theorem}
If the transition system is image-finite, then
\[ \begin{array}{rl}
(i) & {\preord_{\omega}} = {\preord^{B}} \\
(ii) & p \preord_{\omega} q \;\; \Longleftrightarrow \;\; {\rm HML}_{\omega}(p) \subseteq {\rm HML}_{\omega}(q) . 
\end{array} \]
\end{theorem}

Unfortunately, if unguarded recursion is allowed in any of the standard concurrent calculi (SCCS, CCS, CSP, etc.) 
they are neither image-finite nor sort-finite (though sort-finiteness may be regained e.g. for CCS by imposing fairly mild restrictions on the relabelling operators). 
Thus these two Theorems cannot be applied. 
To see how weak finitary Hennessy-Milner logic is when the set of actions is finite, consider the following

\noindent {\bf Example.}
\begin{eqnarray*}
p & \equiv & a \Oh + \Omega \\
q & \equiv & \sum_{n \in \omega} a b_{n} \Oh + \Omega
\end{eqnarray*}
where we assume $b_{m} \not= b_{n}$ for $m \not= n$. Now $p \npreord_{2} q$, but we have

\begin{proposition}
\label{bex}
${\rm HML}_{\omega}(p) \subseteq {\rm HML}_{\omega}(q).$
\end{proposition}

In order to prove this Proposition we need a lemma.

\begin{lemma}
Every formula in ${\rm HML}_{\omega}(\Oh )$ is satisfied by cofinitely many of the $b_{n}\Oh$.
\end{lemma}

\proof\ By induction on formulas in  ${\rm HML}_{\omega}(\Oh )$. 
For conjunctions and disjunctions, the intersection and union of finitely many cofinite sets are cofinite. 
(It is the case for conjunction which necessitates the strength of statement of the Lemma).
The case for ${\ltuple b \rtuple}\phi$ is vacuous. 
For ${[b]}\phi$, cofinitely many (in fact, all but at most one) of the $b_{n}\Oh$ do not have a $b$-action, hence satisfy ${[b]}\phi$. \qed

The Proposition can now be proved by induction on formulas in ${\rm HML}_{\omega}$. 
The only non-trivial case is ${\ltuple a \rtuple}\phi$, which follows from the Lemma.

The deficiency of Hennessy-Milner logic illustrated by this example is disturbing, 
because processes generated by a finitary calculus (including $p$ and $q$ above) should be adequately modelled by a finitary semantics and logic. 
This suggests that Hennessy-Milner logic is not quite right as it stands.

