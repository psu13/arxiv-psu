\section{Applications of the Domain Logic}
We shall now use domain logic to study bisimulation.
Our results in this section can be grouped under four main headings:
\begin{enumerate}
\item Comparisons with Hennessy-Milner logic 
\item Characterisation Theorems
\item Finitary Transition Systems
\item Universal Semantics
\end{enumerate}
Of these, (1) and (2) will confirm the appropriateness of our definitions, while (3) and (4) will represent a distinctive payoff for our approach.

\subsection*{Comparison with Hennessy-Milner logic}
We begin with some technicalities on normal forms.
\begin{definition}
{\rm We define a class of normal forms ${\sf N} \Ellinfty \subseteq \Ell_{\infty \pi}$ inductively as follows:}
\[ \bullet \;\; \frac{ \{\phi_{i} \in {\sf N} \Ellinfty \}_{i \in I}}{\bigwedge_{i \in I} \phi_{i}, \bigvee_{i \in I} \phi_{i} \in {\sf N} \Ellinfty} \]
\[ \bullet \;\; \frac{\phi \in {\sf N} \Ellinfty\ , \;\; a \in {\sf Act}}{\Diamond a(\phi ) \in {\sf N} \Ellinfty} \]
\[ \bullet \;\; \frac{ \{ \phi_{i} \in {\sf N} \Ellinfty \}_{i \in I}, \;\; \{ a_{i} \in {\sf Act} \}_{i \in I} \;\; \{ i \not= j \; \Rightarrow \; a_{i} \not= a_{j} \}_{i, j \in I}}
{\Box \bigvee_{i \in I} a_{i} ( \phi_{i}) \in {\sf N} \Ellinfty} \]
\end{definition}

\begin{lemma}[Normal Forms]
\label{hmnf}
For all $\phi \in \Ell_{\infty \pi}$, for some $\psi \in {\sf N} \Ellinfty$:
\[ \Ellinfty  \vdash  \phi = \psi . \]
\end{lemma}

\proof\ By induction on ${\sf md}(\phi )$.
We consider the two non-trivial cases.

\noindent $\Diamond \phi$: In this case, using the distributive lattice laws there is $\phi'$ of the form
\[ \bigvee_{i \in I} \bigwedge_{j \in J_{i}} a_{ij}(\phi_{ij}) \]
such that $\vdash  \phi = \phi'$, and ${\sf md}(\phi' ) \leq {\sf md}(\phi)$.
By the induction hypothesis, for each $\phi_{ij}$ there is $\phi'_{ij} \in {\sf N} \Ellinfty$ such that $\vdash  \phi_{ij} = \phi'_{ij}$.
Using $(a-{\leq})$ and $({\Diamond}-{\leq})$, we have
\begin{equation} 
\label{nf1}
\vdash  \Diamond \phi = \Diamond \bigvee_{i \in I} \bigwedge_{j \in J_{i}} a_{ij}(\phi_{ij}) .
\end{equation}
Now for each $i \in I$, there are three cases:
\begin{enumerate}
\item $J_{i} = \varnothing$. In this case, $\vdash \Diamond \phi = \Diamond \true$, and we can use $({\Diamond}-\true\ )$ to obtain a normal form.
\item $\exists j_{1}, j_{2} \in J_{i}$, $a_{j_{1}} \not= a_{j_{2}}$.
In this case, we can use $(a-{\wedge})$ to delete the $i$'th disjunct in the RHS of~\ref{nf1}.
\item $\{ a_{ij} : j \in J_{i} \} = \{ a \}$, for some $a \in {\sf Act}$. In this case, we can use $(a-{\wedge})(i)$.
\end{enumerate}
In this way, we obtain {\em either}
\[ \vdash \Diamond \phi = \true , \]
if case (1) is ever applicable, {\em or}
\[ \vdash \Diamond \phi = \Diamond \bigvee_{i' \in I'} a_{i'}(\psi_{i'}) \;\;\; (\psi_{i'} \in {\sf N} \Ellinfty ). \]
In the latter case, we can apply $({\Diamond}-{\vee})$ to get a normal form.

\noindent $\Box \phi$: Similarly to the previous case, we have
\[ \vdash \Box \phi = \Box \bigwedge_{i \in I} \bigvee_{j \in J_{i}} a_{ij} (\phi_{ij}) \;\;\; (\phi_{ij} \in {\sf N} \Ellinfty ). \]
We can then use $({\Box}-{\wedge})$ to get
\[ \vdash \Box \phi =  \bigwedge_{i \in I} \Box \bigvee_{j \in J_{i}} a_{ij} (\phi_{ij}) . \]
Now if we partition each $J_{i}$ by $\sim_{i}$, with
\[ j \sim_{i} k \;\; \Longleftrightarrow \;\; a_{ij} = a_{ik} \;\;\; (j, k \in J_{i}),  \]
we have
\[ \vdash \Box \phi =  \bigwedge_{i \in I} \Box \bigvee_{[j] \in J_{i}/{\sim_{i}}} ( \bigvee_{k \in [j]} a_{ij} (\phi_{ik}))  \]
using the lattice laws; we can then apply $(a-{\vee})$ to get a normal form. \qed

\begin{definition}
\label{tfuns}
{\rm We define translation functions}
\[ ( \cdot )^{\ast} : {\rm HML}_{\infty} \longrightarrow {\sf N} \Ellinfty\ , \]
\[ ( \cdot )^{\dagger} : {\sf N} \Ellinfty  \longrightarrow {\rm HML}_{\infty} . \]
\[ \begin{array}{lll}
( \bigwedge_{i \in I} \phi_{i})^{\ast} & = & \bigwedge_{i \in I} (\phi_{i})^{\ast} \\
( \bigvee_{i \in I} \phi_{i})^{\ast} & = & \bigvee_{i \in I} (\phi_{i})^{\ast} \\
( \ltuple a \rtuple \phi )^{\ast} & = & \Diamond a(\phi^{\ast}) \\
( [ a ] \phi )^{\ast} & = & \Box  a((\phi )^{\ast}) \vee \bigvee \{ b(\true ) : b \in {\sf Act} - \{ a \} \} ) \\
( \bigwedge_{i \in I} \phi_{i})^{\dagger} & = & \bigwedge_{i \in I} (\phi_{i})^{\dagger} \\
( \bigvee_{i \in I} \phi_{i})^{\dagger} & = & \bigvee_{i \in I} (\phi_{i})^{\dagger} \\
( \Diamond a(\phi ))^{\dagger} & = &  \ltuple a \rtuple (\phi )^{\dagger} \\
( \Box \bigvee_{i \in I} a_{i}(\phi_{i}))^{\dagger} & = & \bigwedge_{i \in I} [ a_{i} ] (\phi_{i})^{\dagger} \; \wedge \; \bigwedge \{ [ b ] \false : b \in {\sf Act} - \{ a_{i} : i \in I \} \}
\end{array} \]
\end{definition}
The following is easily verified.
\begin{proposition}
\label{faitht}
For all $\phi \in {\rm HML}_{\infty}, \psi \in {\sf N} \Ellinfty$:
\[ \begin{array}{rrcl}
(i) & {\sf md}(\phi ) & = & {\sf md}(\phi^{\ast}) \\
(ii) & {\sf md}(\psi ) & = & {\sf md}(\psi^{\dagger}) \\
(iii) & p \: \models \: \phi & \Longleftrightarrow & p \: \models \phi^{\ast} \\
(iv) &  p \: \models \: \psi & \Longleftrightarrow & p \: \models \psi^{\dagger} .
\end{array} \]
\end{proposition}

As an immediate consequence of this Proposition together with~\ref{hmnf}, we have

\begin{theorem}[Comparison Theorem (Infinitary Case)]
For $p, q \in {\rm Proc}$ in any transition system,  $A \subseteq {\sf Act}$ and $\lambda \in {\sf Ord}$:
\[ \Ell^{(A, \lambda )}_{\infty}(p) \subseteq \Ell^{(A, \lambda )}_{\infty}(q) \;\; \Longleftrightarrow \;\; {\rm HML}^{(A, \lambda )}_{\infty}(p) \subseteq {\rm HML}^{(A, \lambda )}_{\infty}(q) . \]
\end{theorem}

Thus in the infinitary case, $\Ellinfty$ determines the same preorder on processes as ${\rm HML}_{\infty}$.
However, when {\sf Act} is infinite this does {\em not} cut down to a corresponding result for the finitary case, since our
translation functions introduce infinite disjunctions in translating $[a]$, and infinite conjunctions in translating $\Box$, even for finite formulas.
Our general considerations on observability in Chapter 2 suggest that the introduction of infinite conjunctions is more serious, and indicates a weakness of expressive power in ${\rm HML}_{\infty}$ as an ``observational logic''.
This is in keeping with our remarks at the end of Section~2.
In fact, our translation functions suggest an appropriate way of extending ${\rm HML}_{\infty}$ so as to render it equivalent to $\Ellomega$.
This will be the content of a second Comparison Theorem which we will prove later in this section, when we have some additional machinery at our disposal.

\subsection*{Characterisation Theorems}
Combining the Comparison Theorem with the Modal Characterisation Theorem~\ref{mct}, we have:
\begin{theorem}[Characterisation Theorem for \Ellinfty]
\label{linfct}
With notation as in the previous Theorem,
\[ p {\preord}_{\lambda} q \;\; \Longleftrightarrow \;\; {\Ell}^{({\sf Act}, \lambda )}_{\infty}(p) \subseteq {\Ell}^{({\sf Act}, \lambda )}_{\infty}(q) \]
and therefore
\[ p \preord^{B} q \;\; \Longleftrightarrow \;\; {\Ellinfty}(p) \subseteq {\Ellinfty}(q) . \]
\end{theorem}

We now turn to the question of finding a Characterisation Theorem for $\Ellomega$.
Intuitively, \Ellomega\ represents finitely observable properties of processes, hence should correspond to the ``finitely observable part'' of bisimulation.
If we accept the finite synchronisation trees ${\sf ST}_{\omega}$ as a suitable notion of {\em finite process}, we can use them to determine the {\em algebraic} part of the bisimulation preorder, in the sense e.g. of \cite{Gue81}.
\begin{definition}
{\rm The {\em finitary preorder} $\preord^{F}$ is defined on any transition system by:}
\[ p \preord^{F} q \; \equiv \; \forall t \in {\sf ST}_{\omega}. \: t \preord^{B} p \; \Rightarrow \; t \preord^{B} q . \]
\end{definition}
Our aim is to prove
\begin{theorem}[Characterisation Theorem for $\Ellomega$]
\label{lct}
With notation as in the previous Theorem,
\[ p \preord^{F} q \;\; \Longleftrightarrow \;\; \Ellomega (p) \subseteq \Ellomega (q). \]
\end{theorem}

We will need a few auxiliary results which also have some independent interest.
\begin{definition}
{\rm The {\em height} of a synchronisation tree is defined by:}
\[ {\sf ht}(\sum_{i \in I} a_{i}t_{i} \; [+ \Omega ] ) = \sup \: \{ {\sf ht}(t_{i}) : i \in I \} + 1 \]
\end{definition}

\begin{lemma}
\label{htl}
For any synchronisation tree $T \in {\sf ST}_{\infty}$, ${\sf ht}(T) < \lambda$ implies
\[ T \preord^{B} p \;\; \Longleftrightarrow \;\; T \preord_{\lambda} p . \]
\end{lemma}

\proof\ The left-to-right implication is immediate; the converse is an easy induction on ${\sf ht}(T)$. \qed

In particular, we see that for a finite synchronisation tree $t \in {\sf ST}_{\omega}$, $t \preord^{B} p \; \Leftrightarrow \; t\preord_{\omega} p$.  
Thus we have the inclusions
\[ {\preord^{B}} \subseteq {\preord_{\omega}} \subseteq {\preord^{F}} . \]
In general, these inclusions are strict.
\subsection*{Examples}
(1) ${\preord^{B}} \not= {\preord_{\omega}}$.
\[ p \equiv a^{\omega} + \Omega , \;\;\;\; q \equiv \sum_{k \in \omega} a^{k} \Oh\ + \Omega \]
Then $p \preord_{\omega} q$, but $p \npreord_{\omega + 1} q$.

\noindent (2) ${\preord_{\omega}} \not= {\preord^{F}}$.
\begin{eqnarray*}
p & \equiv & a( \sum_{n \in \omega} b_{n} \Oh + \Omega ) + \Omega \\
q & \equiv & \sum_{n \in \omega} a(\sum_{m \in \omega - \{ n \} } b_{n} \Oh + \Omega ) + \Omega
\end{eqnarray*}
Then $p \preord^{F} q$, but $p \npreord_{2} q$.

These examples gain in significance because all the processes involved can be defined in finitary calculi, in particular SCCS, as we shall see in the next section.

\begin{lemma}[Sort Lemma]
\label{stl}
In any transition system, let $p, q \in {\rm Proc}$, ${\sf sort}(p) \subseteq A \subseteq {\sf Act}$, $\lambda \in {\sf Ord}$. Then
\[ p \npreord_{\lambda} q \;\; \Longrightarrow \;\; {\Ell}^{(A, \lambda )}_{\infty}(p) \not\subseteq {\Ell}^{(A, \lambda )}_{\infty}(q) . \]
\end{lemma}

\proof\ By induction on $\lambda$.
We assume $ p \npreord_{\lambda} q$, and must construct $\phi \in {\Ell}^{(A, \lambda )}_{\infty}(p) - {\Ell}^{(A, \lambda )}_{\infty}(q)$.
There are three cases.

\noindent (1) $\labarrow{p}{a}{p'}$ and for all $q'$, $\labarrow{q}{a}{q'}$ implies $p' \npreord_{\alpha} q'$ for some $\alpha < \lambda$.
By induction hypothesis, for each such $q$ there is $\phi_{q'} \in {\Ell}^{(A, \alpha )}_{\infty}(p') - {\Ell}^{(A, \alpha )}_{\infty}(q')$.
Now define
\[ \phi \; \equiv \; \Diamond a ( \bigwedge \{ \phi_{q'} : \labarrow{q}{a}{q'} \} ) . \]

\noindent (2) $p \converges$ and $q \diverges$.
Let $\phi \equiv \Box \bigvee \{ a(\true ) : \exists p' . \, \labarrow{p}{a}{p'} \}$.

\noindent (3) $p \converges$, $q \converges$, $\labarrow{q}{a}{q'}$, and for all $p'$, $\labarrow{p}{a}{p'}$ implies $p' \npreord_{\alpha} q'$ for some $\alpha < \lambda$.
Define $\phi_{p'}$ similarly to case (1). Then we define
\[ \phi \; \equiv \; \Box ( \bigvee \{ a(\phi_{p'}) : \labarrow{p}{a}{p'} \} \; \vee \; \bigvee \{ b(\true ) : b \not= a \: \& \: \exists r . \, \labarrow{p}{a}{r} \} ) . \;\;\; \qed \]
Note that this result is stronger than the Modal Characterisation Theorem~\ref{mct} for Hennessy-Milner logic, since we only require ${\sf sort}(p) \subseteq A$.
This is significant in the light of the example at the end of Section 2.

\begin{proposition}
\label{finbis}
For all $t \in {\sf ST}_{\omega}$:
\[ t \preord^{B} p \;\; \Longleftrightarrow \;\; {\Ellomega}(t) \subseteq {\Ellomega}(p) . \]
\end{proposition}

\proof\ Combining \ref{htl} and \ref{stl}, we see that
\[ t \preord^{B} p \;\; \Longleftrightarrow \;\; {\Ell}^{(A, k )}_{\infty}(t)  \subseteq {\Ell}^{(A, k )}_{\infty}(p) , \]
where $A = {\sf sort}(t)$ and $k = {\sf ht}(t)$.
Since $A$ and $k$ are both finite, $ {\Ell}^{(A, k )}_{\infty}$ is finite up to logical equivalence (i.e. the Lindenbaum algenbra is finite).
Thus each formula in $ {\Ell}^{(A, k )}_{\infty}$ is equivalent to one in $\Ellomega$, and the proposition is proved. \qed

We need one more auxiliary result, which will in fact be a consequence of our work on SCCS in the next section.
Firstly, we define a map from prime normal forms to finite synchronisation trees
\[ {\sf st} : \mbox{PNF} \rightarrow {\sf ST}_{\omega} \]
as follows:
\[ \begin{array}{lll}
{\sf st}( \bigwedge_{i \in I} \Diamond a_{i}(\phi_{i})) & \equiv & {\sum_{i \in I} a_{i} {\sf st}(\phi_{i}) } + \Omega \\
{\sf st}( \Box \bigvee_{i \in I} a_{i}(\phi_{i}) \; \wedge \; \bigwedge_{j \in J} \Diamond b_{j}(\psi_{i})) & \equiv & {\sum_{i \in I} a_{i} {\sf st}(\phi_{i}) } + { \sum_{j \in J} b_{j} {\sf st}(\psi_{j}) }  .
\end{array} \]
Now analogously to~\ref{psoun} we have
\begin{proposition} 
\label{stprop}
For all $\phi$ in PNF, and $p \in {\rm Proc}$ in any transition system:
\[ p  \models  \phi \;\; \Longleftrightarrow \;\; {\sf st}(\phi ) \preord^{B} p . \]
\end{proposition}
The proof is entirely analogous to \ref{psoun}.

We can now prove~\ref{lct}.
Firstly, ${\Ellomega}(p) \subseteq {\Ellomega}(q)$ implies $p \preord^{F} q$, by \ref{finbis}.
For the converse, assume $p \preord^{F} q$ and $p \models \phi$, $(\phi \in \Ellomega\ )$.
By the SDNF Theorem~\ref{SDNF},
\[ \begin{array}{llr}
\bullet & \vdash \phi = \bigvee_{i \in I} \phi_{i} & (\phi_{i} \in \mbox{PNF}) \\
\Longrightarrow & \exists i \in I . \, p \models \phi_{i} & \\
\Longrightarrow & {\sf st}(\phi_{i}) \preord^{B} p & \ref{stprop} \\
\Longrightarrow & {\sf st}(\phi_{i}) \preord^{B} q & p \preord^{F} q \\
\Longrightarrow & q \models \phi_{i} & \ref{stprop} \\
\Longrightarrow & q \models \phi . & \qed\ 
\end{array} \]
\subsection*{Finitary Transition Systems}

We now embark on our next topic.
The various finiteness conditions on transition systems defined in section~2 reflect attempts to capture features of finitary processes.
Nowever, none of these conditions seems to capture exactly the right class of systems unless we make some unwelcome assumptions such as that the set of actions is finite.
We shall adopt what seems to be a novel approach, of using our program logic to axiomatize a class of systems which we propose as the finitary ones.
Our axiomatisation consists of two schemes over $\Ellinfty$.

\noindent {\bf Notation.} ${\sf Fin}(I)$ is the set of finite subsets of $I$.
\begin{itemize}
\item The axiom scheme of {\em bounded non-determinacy}:
\[ \mbox{(BN)} \;\; \Box \bigvee_{i \in I} \phi_{i} \leq \bigvee_{J \in {\sf Fin}(I)} \Box \bigvee_{j \in J} \phi_{j} \;\;\; (\phi_{i} \in \Ellomega\ ). \]
\item The axiom scheme of {\em finite approximability}:
\[ \mbox{(FA)} \;\; \bigwedge_{J \in {\sf Fin}(I)} \Box \bigwedge_{j \in J} \phi_{j}  \leq \Diamond \bigwedge_{i \in I} \phi_{i} \;\;\; (\phi_{i} \in \Ellomega\ ). \]
\end{itemize}

Note that these axioms are duals.
Since the opposite entailments are theorems of {\Ellinfty}, we shall in fact use (BN) and (FA) to denote the corresponding {\em equations}.
The axioms could equivalently be formulated as: $\Box$~preserves directed joins, $\Diamond$~preserves filtered meets.

What are the intuitions behind these axioms?
(BN) is (thinking of each process as the set of its capabilities and each $\phi_{i}$ as an open set) exactly a statement of {\em compactness}; the link between compactness and the computational notion of bounded non-determinacy is well-known from the literature on powerdomains \cite{PloLN,Smy83}.

The axiom of finite approximability is less familiar from either the topological or the computer science literature.
It is best understood as a logical (or localic) expression of the idea that only {\em closed} sets are taken as elements of a finitary powerdomain construction (or, better put, that from the point of view of finite observability we cannot distinguish between a set and its closure).
The best way to get a more precise understanding is probably to read the proof of the next Theorem.

The duality between the two axioms is reminiscent of the discussion of finite {\em breadth} (BN) and finite {\em length} (FA) limitations of testing in \cite{Abr83a}.

\begin{definition}
\label{fts}
{\rm A transition system is {\em finitary} if it satisfies (all instances of) (BN) and (FA).
The class of finitary transition systems is denoted {\bf FTS}.}
\end{definition}

As a first step, we shall give a substantive example of a finitary transition system.
As we will see, it is actually the best possible example.
\begin{theorem}
\Dom\ is a finitary transition system.
\end{theorem}

\proof\ By the Duality Theorem~ \ref{bdual}, we have a map
\[ \lsem \cdot \rsem : \Ell_{\omega \pi} \longrightarrow K \Omega (\Dom ) \]
\[ \lsem \phi \rsem  \equiv  \{ d \in \Dom : d \models \phi \} . \]
Now for $d \in \Dom$,
\[ d \models \Box \bigvee_{i \in I} \phi_{i}  \;\; \Longrightarrow \;\; d \models \bigvee_{J \in {\sf Fin}(I)} \Box \bigvee_{j \in J} \phi_{j}  \]
is just the statement
\[ d \subseteq \bigcup_{i \in I} O_{i} \;\; \Longrightarrow \;\; \exists J \in {\sf Fin}(I) . \, d \subseteq \bigcup_{j \in J} O_{j} , \]
where $O_{i} = \lsem \phi_{i} \rsem$, i.e. that $d$ is compact as a subset of $\sum_{a \in {\sf Act}} \Dom$.
Since $d \in \Dom \cong P^{0} [ \sum_{a \in {\sf Act}} \Dom ]$, and elements of the Plotkin powerdomain are Scott-compact subsets of the base domain (\cite{PloLN}), this proves that \Dom\ satisfies (BN).

Next we show that \Dom\ satisfies (FA).
Since there are only countably many distinct formulae in $\Ellomega$, it suffices to prove the following:
\begin{itemize}
\item Given a sequence $\{ U_{n} \}$ of compact-open subsets of $\Dom$, 
with $U_{n} \supseteq U_{n + 1} \; (n \in \omega )$, and an element $d \in \Dom$ such that $d \cap U_{n} \not= \varnothing$ $(n \in \omega)$, then $d \cap \bigcap_{n \in \omega} U_{n} \not= \varnothing$.
\end{itemize}
(The alternative case for $d \models U_{n}$, namely $\bot \in U_{n}$ for all $n$, is trivial.)

Since each $U_{n}$ is compact-open, it has the form $\diverges B_{n}$, where $B_{n}$ is a finite subset of ${\cal K}(\Dom\ )$.
Also, $B_{n} \sqsubseteq_{u} B_{n + 1}$, where
\[ X \sqsubseteq_{u} Y \; \equiv \; \forall y \in Y. \, \exists x \in X . \, x \sqsubseteq y \;\;\; (X, Y \subseteq \Dom\ ) . \]
Now define
\[ C_{n} \; \equiv \; \{ b \in B_{n} : \exists x \in d . \, b \sqsubseteq x \} \;\;\; (n \in \omega ) . \]
Since $d \cap U_{n} \not= \varnothing$, $C_{n} \not= \varnothing$ for all $n$.
Also, $C_{n} \sqsubseteq_{u} C_{n + 1}$.
Thus by K\"{o}nig's Lemma in the form given e.g. in~\cite{Niv81}, there is a sequence $\{ c_{n} \}$ with $c_{n} \sqsubseteq c_{n+1}$ and $c_{n} \in C_{n}$.
Now define
\[ e_{n} \; \equiv \; \lsing c_{n} \rsing \uplus \lsing \bot \rsing \;\;\; (n \in \omega ). \]
Clearly $e_{n} \sqsubseteq e_{n + 1}$ and $e_{n} \sqsubseteq d$ for all $n$, whence $\bigsqcup e_{n} \sqsubseteq d$.
But $\bigsqcup c_{n} \in \bigsqcup e_{n}$ (using the description of least 
upper bounds of chains in the Plotkin powerdomain given in \cite[Theorem 8]{Plo76}), 
and so for some $x \in d$, $\bigsqcup c_{n} \sqsubseteq x$.
Since $\bigsqcup c_{n} \in U_{n}$ for all $n$, $d \cap \bigcap_{n \in \omega} U_{n} \not= \varnothing$, and the proof is complete. \qed

We now draw some striking consequences from the finitary axioms.
\begin{definition}
{\rm A formula $\phi \in \Ellinfty$ is in {\em finitary normal form} if it has the form}
\[ \bigwedge_{i \in I} \bigvee_{j \in J_{i}} \phi_{ij} \;\;\; (\phi_{ij} \in \Ellomega ). \]
\end{definition}

\begin{lemma} \label{fnf}
For each $\phi \in \Ellinfty$, for some finitary normal form $\psi$:
\[ \mbox{(BN) + (FA)} \vdash \phi = \psi . \]
\end{lemma}

\proof\ An easy induction on ${\sf ht}(\phi)$. \qed

\begin{proposition}
\label{p18}
In any finitary transition system ${\cal T}$, for all $p, q \in {\rm Proc}$:
\[ \Ellinfty (p) \subseteq \Ellinfty (q) \;\; \Longleftrightarrow \;\; \Ellomega (p) \subseteq \Ellomega (q) . \]
\end{proposition}

\proof\ The left to right implication is immediate.
For the converse, suppose $ \Ellomega (p) \subseteq \Ellomega (q)$, and $p \models \phi$, $(\phi \in \Ellinfty )$.
By~\ref{fnf},
\[ \mbox{(BN) + (FA)} \vdash \phi = \bigwedge_{i \in I} \bigvee_{j \in J_{i}} \phi_{ij} \;\;\;  (\phi_{ij} \in \Ellomega ) \]
hence since ${\cal T} \models \mbox{(BN) + (FA)}$, ${\cal T} \models \phi = \bigwedge_{i \in I} \bigvee_{j \in J_{i}} \phi_{ij}$, and
\[ \begin{array}{ll}
\bullet & p \models \bigwedge_{i \in I} \bigvee_{j \in J_{i}} \phi_{ij} \\
\Longrightarrow & \forall i \in I . \, \exists j \in J_{i} . \, p \models \phi_{ij} \\
\Longrightarrow &  \forall i \in I . \, \exists j \in J_{i} . \, q \models \phi_{ij} \\
\Longrightarrow & q \models \bigwedge_{i \in I} \bigvee_{j \in J_{i}} \phi_{ij} \\
\Longrightarrow & q \models \phi . \;\;\; \qed
\end{array} \]

\begin{theorem}[Finitary Characterisation Theorem]
\label{fct}
With notation as in the previous Proposition:
\[ p \preord^{B} q \;\; \Longleftrightarrow \;\; p \preord_{\omega} q \;\; \Longleftrightarrow \;\; p \preord^{F} q \;\; \Longleftrightarrow \;\; \Ellomega (p) \subseteq \Ellomega (q) . \]
\end{theorem}

\proof\ Combine Theorems~\ref{linfct}, \ref{lct} and \ref{p18}. \qed

In order to continue our study of finitary transition systems, we need to introduce some notions from our final topic of this section.

\subsection*{Universal Semantics}

Given any transition system and $p \in {\rm Proc}$, it is easy to see that $\Ellomega (p) \subseteq \Ellomega$ satisfies the axioms of a prime filter; hence we have a map
\[ \Ellomega (\cdot ) : {\rm Proc} \longrightarrow {\sf Spec} \; \Ellomega\ . \]
If we compose this with the isomorphism $ {\sf Spec} \; \Ellomega \cong \Dom$ from the Duality Theorem~\ref{bdual}, we get a map
\[ \lsem \cdot \rsem : {\rm Proc} \longrightarrow \Dom \]
which takes each process to an element of our domain.
This map can be regarded as a {\em syntax-free denotational semantics}; it is {\em universal} since it is defined on every transition system.

\begin{theorem}[Universal Semantics] 
\label{usem}
For any transition system ${\cal T}$ with $p, q \in {\rm Proc}$:
\[  \begin{array}{rl}
(i)  & p \preord^{F} q \;\; \Longleftrightarrow \;\; \lsem p \rsem \sqsubseteq \lsem q \rsem \\
(ii) & p \sim^{F} \lsem p \rsem .
\end{array} \]
If ${\cal T}$ is finitary, then:
\[ \begin{array}{rl}
(iii)  & p \preord^{B} q \;\; \Longleftrightarrow \;\; \lsem p \rsem \sqsubseteq \lsem q \rsem \\
(iv) & p \sim^{B} \lsem p \rsem .
\end{array} \] 
\end{theorem}

\proof\ Clearly (i) follows from (ii), and (iii) from (iv).
Now $\Ellomega (p) = \Ellomega (\lsem p \rsem )$; 
and so (ii) follows from \ref{lct}; while (iv) follows from \ref{fct}. \qed

We can think of \ref{usem} as a {\em full abstraction theorem} 
\cite{Mil75,Plo77,Mil77} for our semantics; it says that every transition system (finitary transition system) can be embedded in \Dom\ with as much identification as possible modulo the finitary equivalence (bisimulation).

Since \Dom\ can itself be viewed as a transition system, we can tie things up even more neatly.
Let {\bf TS} be the category with objects the transition systems, and morphisms ${\cal T}_{1} \rightarrow {\cal T}_{2}$ maps
\[ f : {\rm Proc}_{1} \rightarrow {\rm Proc}_{2} \]
for which
\[ \Ellomega (p) = \Ellomega (f(p)) \;\;\; (p \in {\rm Proc}_{1}). \]
It is clear that for such $f$
\[ p \preord^{F} q \;\; \Longleftrightarrow \;\; f(p) \preord^{F} f(q) , \]
and if ${\cal T}_{1}$ and  ${\cal T}_{2}$ are finitary,
\[ p \preord^{B} q \;\; \Longleftrightarrow \;\; f(p) \preord^{B} f(q) . \]
Now we have

\begin{theorem}[Final Algebra Theorem]
\label{falg}
\Dom\ is final in {\bf TS}, and also in the subcategory {\bf FTS} of finitary transition systems.
\end{theorem}

\proof\ All we need to show is that the semantic map $\lsem \cdot \rsem$ is the unique morphism from a transition system to $\Dom$.
But for $d_{1}, d_{2} \in \Dom$,
\begin{Eqarray}
\Ellomega (d_{1}) \subseteq \Ellomega (d_{2}) & \Longleftrightarrow & K \Omega (d_{1}) \subseteq K \Omega (d_{2}) & \mbox{by~\ref{bdual}} \\
& \Longleftrightarrow & d_{1} \sqsubseteq d_{2} & \mbox{since \Dom\ is coherent,} \\
\end{Eqarray}
which gives uniqueness. \qed

\subsection*{Finitary Transition Systems Resumed}
Firstly, some conditions equivalent to finitariness.
\begin{proposition}
\label{ftsequiv}
For any transition system ${\cal T}$, the following conditions are equivalent:
\begin{center}
\begin{tabular}{rl}
(i) & ${\cal T}$ is finitary \\
(ii) & $\forall p \in {\rm Proc} . \, p \sim^{B} \lsem p \rsem$ \\
(iii) & ${\preord^{B}} = {\preord^{F}}$ in the combined system ${\cal T} + \Dom$ (disjoint union).
\end{tabular}
\end{center}
\end{proposition}

\proof\  $(i) \; \Longrightarrow \; (ii)$ is~\ref{usem} (iv); $(ii) \; \Longrightarrow \; (iii)$ since \Dom\ is finitary.

\noindent $(ii) \; \Longrightarrow \; (i)$. Suppose that ${\cal T}$ is not finitary, in particular that (BN) fails; i.e. that for some $p \in {\rm Proc}$,
\[ p \models \Box \bigvee_{i \in I} \phi_{i} \;\;\; ( \phi_{i} \in \Ellomega ) \]
and $\forall J \in {\sf Fin}(I). \, p \nvDash \bigvee_{j \in J} \phi_{j}$.
Since $\Ellomega (p) = \Ellomega ( \lsem p \rsem )$, and each $\bigvee_{j \in J} \phi_{j} \in \Ellomega$, 
$\lsem p \rsem \nvDash \bigvee_{j \in J} \phi_{j}$ for all $J \in {\sf Fin}(I)$;
hence since $\lsem p \rsem \in \Dom$ and $\Dom$ is finitary, $\lsem p \rsem \nvDash \Box \bigvee_{i \in I} \phi_{i}$.
Thus $\Ellinfty (\lsem p \rsem ) \not= \Ellinfty (p)$, and so by~\ref{linfct} $p \sim^{B} \lsem p \rsem$.
The case when (FA) fails is similar. 

\noindent $(iii) \; \Longrightarrow \; (ii)$. Suppose for some $p$, $p \nsim^{B} \lsem p \rsem$.
Then since $p \sim^{F} \lsem p \rsem$ by~\ref{usem} (ii), ${\preord^{B}} \not= {\preord^{F}}$. \qed

Note that in part (iii) of this Proposition we have ``added in'' \Dom\ to the given transition system ${\cal T}$.
This is to overcome the problem that there may not be enough processes in ${\cal T}$ alone to cause $\preord^{B} = \preord^{F}$ to fail.

Now we relate some of the finitariness conditions of Section~2 to our axioms.

\begin{proposition}
\label{finots}
(i) Weakly finite branching is equivalent to weakly image finite plus weakly initials finite.

\noindent (ii) Weakly finite branching implies (BN).

\noindent (iii) (BN) implies weakly initials finite.

\noindent (iv) (BN) + (FA) do {\em not} imply weakly image finite.
\end{proposition}

\proof\ (i). Easy.

\noindent (ii). Suppose $p \models \Box \bigvee_{i \in I} \phi_{i}$.
$(\bigvee_{i \in I} \phi_{i}) \diverges  \Leftrightarrow  \exists i \in I. \, \phi_{i} \diverges$, in which case $\vdash \phi_{i} = \true$, and the conclusion is trivial.
Otherwise, $p \converges$, and so $C(p)$ is finite, say 
\[ C(p) = \{ \ltuple a_{1}, p_{1} \rtuple\ , \ldots , \ltuple a_{n}, p_{n} \rtuple \} . \]
Then for each $k$ with $1 \leq k \leq n$, $\ltuple a_{k}, p_{k} \rtuple \models \phi_{i_{k}}$ for some $i_{k} \in I$, and so $p \models \Box \bigvee_{j \in J} \phi_{j}$, where $J = \{ i_{1}, \ldots , i_{n} \}$.

\noindent (iii). Assume (BN) and $p \converges$.
Then $p \models \Box \bigvee_{a \in {\sf Act}} a(\true\ )$, and so by (BN)
\[ p \models \bigvee_{J \in {\sf Fin}({\sf Act})} \Box \bigvee_{a \in J} a(\true\ ) , \]
which says exactly that $p$ has a finite set of initial actions.

\noindent (iv). $\sum_{n \in \omega} a^{n} + a^{\omega}$ is in $\Dom$. \qed

All the usual finitary  calculi are weakly finite branching, and so satisfy (BN).
However, in general these calculi do {\em not} satisfy (FA) (analogously to the fact that generating trees over domains do not yield closed sets, although they always yield compact ones; cf. \cite{PloLN}).
As a standard counterexample, define
\[ \begin{array}{lll}
p & \equiv & \sum_{n \in \omega} a^{n} \Oh + \Omega \\
\phi_{0} & \equiv & \true\  \\
\phi_{k + 1} & \equiv & a(\Diamond \phi_{k}) .
\end{array} \]
Then for all $J \in {\sf Fin}(\omega )$, $p \models \Diamond \bigwedge_{j \in J} \phi_{j}$, but $p \nvDash \Diamond \bigwedge_{i \in \omega} \phi_{i}$.

Thus if $p$ can be defined in our calculus, it does not satisfy (FA).
Since $p$ {\em can} be defined in CCS, SCCS (see next section), etc., these calculi are not finitary transition systems according to Definition~\ref{fts}.
However, we can take the view that if we only take account of {\em observable} 
information via the semantics $\lsem \cdot \rsem$, we have collapsed the 
given system into a finitary one which will actually, 
by Theorems~\ref{usem} and~\ref{falg}, be isomorphic to a subsystem (or, topologically, a subspace) of $\Dom$.

\subsection*{Comparison Theorems Resumed}
We now return to the question of finding a suitable correspondence between the finitary parts of HML and $\Ell$.
As confirmation of our claim that ${\rm HML}_{\omega}$ is unsatisfactory, we have:

\noindent {\bf Observation.} ${\rm HML}_{\omega}$ does not characterise $\preord^{F}$.

In fact, \ref{bex} provides a counter-example since, with the notation used there, $p \npreord^{F} q$ while $ {\rm HML}_{\omega}(p) \subseteq {\rm HML}_{\omega}(q)$.

We can get an idea of how to extend $ {\rm HML}_{\omega}$ by inspection of the translation functions~\ref{tfuns}.
Although $( \cdot )^{\dagger}$ introduces infinitary conjunctions, these are of a special kind, for which a finitary counterpart can be found.

\begin{definition} 
{\rm ${\rm HML}^{+}$ is the extension of ${\rm HML}_{\omega}$ with additional atomic fomulae of the form
\[ {\sf init}(A) \;\;\; (A \in {\sf Fin}({\sf Act})). \]
The definition of the satisfaction relation is extended by
\[ p \models {\sf init}(A)  \equiv  p \converges \; \& \; \{ a \in {\sf Act} : \exists q . \, \labarrow{p}{a}{q} \} \subseteq A . \] }
\end{definition}

We can now modify the translation function $( \cdot )^{\dagger}$ as follows:
\[ ( \Box \bigvee_{i \in I} a_{i} ( \phi_{i} ))^{\dagger} \; \equiv \; \bigwedge_{i \in I} [a_{i}] ( \phi_{i})^{\dagger} \; \wedge \; {\sf init}(\{ a_{i} : i \in I \} ). \]
Proposition~\ref{faitht} clearly still holds with this modification, and $( \cdot )^{\dagger}$ now cuts down to a function
\[ {\sf N} \Ellomega \longrightarrow {\rm HML}^{+}. \]
There is still a mismatch in the other direction, since $( \cdot )^{\ast}$ introduces infinite disjunctions.
To overcome this, we have to make the assumption that the transition system satisfies (BN)---a mild one, as~\ref{finots} and the ensuing discussion shows.

Let $\Ell_{\bigvee \infty}$ be the sublanguage of \Ellinfty\ obtained by the restriction to finite {\em conjunctions} (but with infinite disjunctions still allowed).
\begin{proposition} 
In any transition system satisfying (BN), for all $p, q \in {\rm Proc}$:
\[ \Ell_{\bigvee \infty}(p) \subseteq \Ell_{\bigvee \infty}(q) \;\; \Longleftrightarrow \;\; \Ellomega(p) \subseteq \Ellomega(q) . \]
\end{proposition}

\proof\ Just like~\ref{p18}. \qed

Clearly, $( \cdot )^{\ast}$, extended by the clause
\[ ({\sf init}(A))^{\ast} \; \equiv \; \Box \bigvee \{ a(\true\ : a \in A \} \]
cuts down to a function
\[ {\rm HML}^{+}  \longrightarrow  {\sf N} \Ell_{\bigvee \infty} . \]
We thus arrive at our
\begin{theorem}[Comparison Theorem (Finitary Case)]
With notation as in the previous Proposition:
\[ {\rm HML}^{+}(p) \subseteq {\rm HML}^{+}(q) \;\; \Longleftrightarrow \;\; \Ellomega (p) \subseteq \Ellomega (q) . \]
\end{theorem}
