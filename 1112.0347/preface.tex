\chapter*{Preface}
\section*{Acknowledgements}
My warmest thanks to the many people who have helped me along the way:
\begin{itemize}
\item To my colleagues at Queen Mary College (1978--83) for five
very happy and productive years.
\item To my supervisor, Richard Bornat, who gave me so much of his
time during my two years as a full-time Research Student, and also
gave me confidence in the worth of my ideas.
\item To Tom Maibaum for our regular meetings to work on semantics
in 1982--3; these were a life-line when my theoretical work had
previously been done in a vacuum.
\item To my colleagues in the Theory and Formal Methods Group
in the Department of Computing, Imperial College: 
Mark Dawson, Dov Gabbay, Chris Hankin, Yves Lafont, Tom Maibaum, Luke Ong, Iain Phillips, 
Martin Sadler, Mike Smyth,  Richard Sykes, 
Paul Taylor and Steve Vickers,
for creating such a stimulating and inspiring environment in which to
work.
\item To Axel Poign\'{e}, who has just returned to Germany to take
up a post at GMD, for being the most inspiring of colleagues,
whose interest in and encouragement of my work has meant
a great deal to me.
\item To Mark Dawson, for unfailingly finding elegant solutions to all
my computing problems.
\item To my hosts for two very enjoyable visits when much of the
work reported in Chapters 5 and 6 was done: the Programming
Methodology Group, Chalmers Technical University, G\"{o}teborg, Sweden,
March 1984; and Professor Raymond Boute and the Functional Languages 
and Architectures Group, University of Nijmegen, the Netherlands, 
March--April and
August, 1986.
\item To a number of colleagues for conversations, lectures and writings
which have provided inspiration and stimulus to this work: Henk Barendregt,
Peter Dybjer, Matthew Hennessy,  Per Martin-L\"{o}f, 
Robin Milner, 
Gordon Plotkin, Jan Smith, 
Mike Smyth,
Colin Stirling
and  Glynn Winskel.
Glynn's persistent enthusiasm for and encouragement of this work have
meant a great deal.
\end{itemize}
The ideas of  Mike Smyth, Gordon Plotkin and Per Martin-L\"{o}f have
been of particular importance to me in my work on this thesis.
Equally important has been the paradigm of how to do Computer Science
which I like many others have found in the work of Robin Milner and 
Gordon  Plotkin.
I thank them all for their inspiration and example.

I thank the Science and Engineering Research Council for supporting
my work, firstly with a Research Studentship and then with a number
of Research Grants.
Thanks also to the Alvey Programme for funding such ``long-term''
research, and in particular for providing the equipment on which
this document was produced (by me).

Finally, I thank my family for their love and support and, over the
past few months, their forbearance.
\section*{Chronology}
It may be worthwhile to make a few remarks about the chronology of the
work reported in this thesis, as a number of manuscripts describing
different versions of some of the material have been in circulation
over the past few years.
My first version of ``Domain Logic'' was worked out in October and
November of 1983, and presented to the Logic Programming Seminar
at Imperial (the invitation was never repeated), and again
at a seminar at Manchester arranged by Peter Aczel the following
February.
The slides of the talk, under the title ``Intuitionistic Logic of
Computable Functions'', were copied to a few researchers.
The main results of Chapter 6 were obtained, in the setting of
Martin-L\"{o}f's Domain Interpretation of his Type Theory,
during and shortly after a visit to Chalmers in March 1984.
A draft paper was begun in 1984 but never completed;
it formed the basis of a talk given at the CMU Seminar on Concurrency
in July 1984.
The outline of Chapter 5 was developed, with the benefit of
many discussions with Axel Poign\'{e}, in October and November 1984.
Thus the main ideas of the thesis had been formulated, admittedly in
rather inchoate form, by the end of 1984.
The following year was mainly taken up with other things;
but a manuscript on ``Domain Theory in Logical Form'',
essentially the skeleton of the present Chapter 4, minus the endogenous
logic, was written in December 1985, and circulated among a few researchers.
A manuscript on ``A Domain Equation for Bisimulation'' was written
during a visit to the University of Nijmegen in March--April 1986,
and another on ``Finitary Transition Systems'' soon afterwards.
A talk on ``The Lazy $\lambda$-Calculus'' was given at Nijmegen in
August 1986.
Chapters 3, 5 and 6 were written in September--December 1986, together
with a skeletal version of Chapter 4, which was presented at the
Second Symposium on Logic in Computer Science at Cornell, June 1987
\cite{Abr87a}.
