\section{A Domain Equation for Applicative Bisimulation}
We now embark on the same programme as in the previous Chapter; 
to obtain a domain-theoretic analysis of our computational notions, 
based on a suitable domain equation. 
What this should be is readily elicited from the definition of ats. 
The structure map
\[ ev : A \rightharpoonup (A \rightarrow A) \]
is {\em partial}; the standard approach to partial maps in domain theory 
({\em pace} Plotkin's recent work on predomains \cite{Plo85}) is to make 
them into total ones by sending undefined arguments to a ``bottom'' element, 
i.e. changing the type of $ev$ to
\[ A \rightarrow (A \rightarrow A)_{\bot} . \]
This suggests the domain equation
\[ D = (D \rightarrow D)_{\bot} \]
i.e. the denotation of the type expression 
${\sf rec} \, t. (t \rightarrow t)_{\bot}$. 
This equation is composed from the function space and lifting constructions. 
Since {\bf SDom} is closed under these constructions, $D$ is a Scott domain. 
Indeed, by the same reasoning it is an algebraic lattice. 
The crucial point is that this equation has a 
{\em non-trivial initial solution}, and thus there is a good candidate 
for a canonical model. 
To see this, consider the ``approximants'' $D_{k}$, with 
$D_{0} \equiv {\bf 1}$, $D_{k+1} \equiv (D_{k} \rightarrow D_{k})_{\bot}$. 
Then
\begin{eqnarray*}
D_{1} & = &  ({\bf 1} \rightarrow {\bf 1})_{\bot} \cong ({\bf 1})_{\bot} \cong \sierp \\
D_{2} & \cong & (\sierp \rightarrow \sierp )_{\bot}, \;\;\; \mbox{with four elements} \\
& \vdots & 
\end{eqnarray*} 
etc. 
We now unpack the structure of $D$. Our treatment will be rather cursory, 
as it proceeds along similar lines to our work in the previous Chapter. 
Firstly, there is an isomorphism pair 
\[ {\sf unfold} : D \rightarrow (D \rightarrow D)_{\bot}, \]
\[ {\sf fold} : (D \rightarrow D)_{\bot} \rightarrow D . \]
Next, we recall the categorical description of lifting, as the left adjoint 
to the forgetful functor
\[ U : {\bf Dom}_{\bot} \rightarrow {\bf Dom} \]
where ${\bf Dom}_{\bot}$ is the sub-category of strict functions. 
Thus we have:
\begin{itemize}
\item A natural transformation ${\sf up} : I_{\bf Dom} \rightarrow U \circ 
( \cdot )_{\bot}$.
\item For each continuous map $f : D \rightarrow U E$ its adjoint
\[ {\sf lift}(f) : (D)_{\bot} \rightarrow_{\bot} E . \]
\end{itemize}
Concretely, we can take
\begin{eqnarray*}
(D)_{\bot} & \equiv & \{ \bot \} \; \cup \; \{ {<}0, d{>} \: | \: d \in D \} \\
x \sqsubseteq y & \equiv & x = {\bot} \\
& & \mbox{or} \; x = {<}0, d{>} \: \& \: y = {<}0, d'{>} \: \& \: d \sqsubseteq_{D} d' \\
{\sf up}_{D}(d) & \equiv & {<}0, d{>} \\
{\sf lift}(f)(\bot ) & \equiv & \bot_{E} \\
{\sf lift}(f) {<}0, d{>} & \equiv & f(d) .
\end{eqnarray*}
We can now define
\[ ev : D \rightharpoonup (D \rightarrow D) \]
by
\[ ev(d) = \left\{ \begin{array}{ll}
f, & {\sf unfold}(d) = {<}0, f{>} \\
{\rm undefined} & {\sf unfold}(d) = \bot .
\end{array} \right. \]
Thus $(D, ev)$ is a quasi-ats, and we write $d \Converges f$, $d \Diverges$ etc. Note that we can recover $d$ from $ev(d)$ by
\[ d = \left\{ \begin{array}{ll}
{\sf fold}({<}0, f{>}),  & d \Converges f \\
\bot_{D} & d \Diverges .
\end{array} \right. \]
The final ingredient in the definition of $D$ is initiality. 
The only direct consequence of this which we will use is contained in
\begin{theorem}
$D$ is internally fully abstract, i.e.
\[ \forall d, d' \in D . \, d \sqsubseteq d' \;\; \Longleftrightarrow \;\; d \preord^{B} d' . \]
\end{theorem}

\proof\ Unpacking the definitions, we see that for all $d, d' \in D$:
\[ d \sqsubseteq d' \;\; \Longleftrightarrow \;\; d \Converges f \; \Rightarrow \; d' \Converges g \: \& \: \forall d'' \in D . \, f(d'' ) \sqsubseteq g(d'' ) . \]
Thus the domain ordering is an applicative bisimulation, and so is included in $\sqsubseteq^{B}$. For the converse, we need some additional notions. We define $d_{k}$, $f_{k}$ for $d \in D$, $f \in [D \rightarrow D]$, $k \in \omega$ by:
\[ d_{0} \Diverges \]
\[ d \Diverges \; \Rightarrow \; d_{k} \Diverges \]
\[ d \Converges f \; \Rightarrow \; d_{k+1} \Converges f_{k} \]
\[ f_{k} : d \mapsto (f d)_{k} . \]
We can use standard techniques to prove, from the initiality of $D$:
\[ \bullet \;\; \forall d \in D . \, d = \bigsqcup_{k \in \omega} d_{k} . \]
The proof is completed with a routine induction to show that:
\[ \forall k \in \omega . \, d \preord_{k} d' \; \Rightarrow \; d_{k} \sqsubseteq d'_{k} . \;\;\; \qed \]
As an immediate corollary of this result, we see that $D$ is an ats. We thus have an interpretation function
\[ \lsem \cdot \rsem^{D} : CL(D) \rightarrow Env(D) \rightarrow \rightarrow D . \]
We extend this to $\Lambda (D)$ by:
\[ \lsem \lambda x . M \rsem^{D}_{\rho} = {\sf fold}({\sf up}(\lambda d \in D . \lsem M \rsem^{D}_{\rho [ x \mapsto d ]} )) . \]
Note that the application  induced from $(D, ev)$ can be described by
\[ d \appl\  d' = {\sf lift}(Ap) \: {\sf unfold}(d) \: d' \]
where
\[ Ap : [D \rightarrow D] \rightarrow D \rightarrow D \]
is the standard application function; and is therefore continuous. This together with standard arguments about environment semantics guarantees that our extension of $\lsem \rsem^{D}$ is well-defined. Note also that $\lsem \lambda x . M \rsem^{D}_{\rho} \neq \bot_{D}$, as expected.

We can now define
\[ k \equiv \lsem \lambda x . \lambda y . x \rsem^{D}_{\rho} , \]
\[ s \equiv \lsem \lambda x . \lambda y . \lambda z . (x z) (y z) \rsem^{D}_{\rho} \]
for $D$. It is straightforward to verify
\begin{proposition}
$D$ is an lts. \qed
\end{proposition}

Thus far, we have merely used our domain equation to construct a particular lts $D$. However, its ``categorical'' or ``absolute'' nature should lead us to suspect that we can use $D$ to study the whole class of lts. The medium we will use for this purpose is once again a suitable domain logic.
