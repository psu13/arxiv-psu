\chapter{Applications to Concurrency:  
A Domain Equation for Bisimulation}
\section{Introduction}
Our aim in this Chapter is to treat some basic topics in the theory
of concurrency from the point of view of domain logic. 
This will serve as a major case study for the general theory developed in the previous two Chapters; and will also weave another of the strands mentioned in Chapter~1 into our narrative. 
Our aim is not only to exemplify the general theory, but to {\em apply} it in order to shed some new light on concurrency.
In particular, we shall study {\em bisimulation} \cite{Par81,Mil83,HM85}. 
This notion has emerged as one of the more stable and mathematically natural concepts to have been formulated in the study of concurrency over the past decade.
It is commonly accepted as the {\em finest} extensional or behavioural equivalence on processes one would want to impose.
To date, bisimulation has been studied almost exclusively from the operational and logical points of view.
Our aim is to show that this notion can be captured elegantly in the setting of domain theory, using Plotkin's powerdomain construction \cite{Plo76}.
Moreover, we shall make extensive use of the logical form of domain theory developed in the previous Chapter.
Thus our motivation can be summarised as follows:
\begin{itemize}

\item To show that more can be done in the sphere of concurrency using domain-theoretic and denotational methods than seems to be commonly realised.

\item To analyze the apparently {\it ad hoc} and ``application oriented'' notions of bisimulation over labelled transition systems and Hennessy-Milner logic by means of the general, mathematically basic, and ``reusable'' notions of domain theory, specifically type constructions and the solution of recursive domain equations.

\item To form part of our general programme of connecting
\begin{enumerate}

\item Domain theory and operational notions of observability

\item Denotational semantics and program logics.
\end{enumerate}

This programme is made systematic by using the information conveyed in the syntactic description of domains by type expressions.
It can be argued that a full domain-theoretic analysis of some computational situation is only obtained when we have written down an explicit type expression, rather than using some {\it ad hoc} construction of a cpo.
At any rate, the benefits which flow from having such a description are very considerable.
Using the ideas developed in the previous Chapter, we can derive a propositional theory from the type expression, and use this to explore the ``observational logic'' of the computational situation.
\end{itemize}

We now summarise the further contents of the Chapter.
After reviewing some basic notions on transition systems etc., we introduce a domain of synchronisation trees defined by means of a domain equation (recursive type expression).
Then we present a domain logic for transition systems, which is derived from this domain equation in the sense of Chapter 3.
The main result of section 4 is that the finitary part of this logic is the Stone dual of our domain of synchronisation trees.

In section 5, we present a number of applications of this logic.
It is shown to be equivalent to Hennessy-Milner logic in the infinitary case, and hence to characterise bisimulation.
In the finitary case, it more powerful than Hennessy-Milner logic, and we obtain a more satisfactory characterisation result for it; namely, it is shown to characterise the ``finitary part'' of bisimulation for {\em all} transition systems.

We also develop an extension of Hennessy-Milner logic which is equivalent to the finitary domain logic.
The infinitary domain logic is then used to {\em axiomatize} a suitable notion of ``finitary transition system''.
These systems are shown indeed to be finitary in a strong sense --- their bisimulation preorders are algebraic.
Finally, the domain of synchronisation trees (i.e. the spectral space of the logic) is shown to be finitary {\it qua} transition system, and moreover to be {\em final} in a suitable category of such systems.
This yields a syntax-free ``universal semantics'' for transition systems, which is fully abstract with respect to bisimulation.

In section 6, we give a conventional (syntax-directed) denotational semantics for the concurrent calculus SCCS \cite{Mil83}, based on our domain of synchronisation trees.
A full abstraction result is proved for this semantics; as a by-product, our domain is shown to be isomorphic to Hennessy's term model \cite{Hen81}.
