\section{A Domain Logic for Applicative Transition Systems}
\begin{definition}
{\rm The syntax of our domain logic ${\cal L}$ is defined by}
\[ \phi \;\; ::= \;\; {\sl t} \; | \; \phi \wedge \psi \; | \; (\phi \rightarrow \psi)_{\bot} \]
\end{definition}
\begin{definition}[Semantics of ${\cal L}$]
{\rm Given a quasi ats ${\cal A}$, we define the satisfaction relation ${\models_{{\cal A}}} \subseteq {{\cal A} \times {\cal L}}$:}
\[ a \; \models_{{\cal A}} \; {\sl t}  \;\; {\rm always} \]
\[ a \; \models_{{\cal A}} \; \phi \wedge \psi \; \; \equiv \;\;   a \; \models_{{\cal A}} \; \phi \; \& \;  a \; \models_{{\cal A}} \; \psi \]
\[ a \; \models_{{\cal A}} \; (\phi \rightarrow \psi )_{\bot} \;\; \equiv \;\; a \Converges f \; \& \; \forall b \in A . \, b \; \models_{{\cal A}} \; \phi \; \Rightarrow \; f(b) \; \models_{{\cal A}} \; \psi . \]
\end{definition}
{\bf Notation:}
\begin{eqnarray*}
{\cal L}(a) & \equiv &  \{ \phi \in {\cal L} \; : \; a \; \models_{{\cal A}} \; \phi \} \\
{\cal A} \; \models \; \phi \leq \psi  & \equiv & \forall a \in A . \, a \; \models_{{\cal A}} \; \phi \;\; \Longrightarrow \;\; a \; \models_{{\cal A}} \; \psi \\
{\cal A} \; \models \; \phi = \psi   & \equiv & \forall a \in A . \, a \; \models_{{\cal A}} \; \phi \;\; \Longleftrightarrow \;\; a \; \models_{{\cal A}} \; \psi \\
\models \; \phi \leq \psi  & \equiv & \forall {\cal A} . \, {\cal A} \; \models \; \phi \leq \psi \\
\lambda  & \equiv & ({\sl t} \rightarrow {\sl t})_{\bot} \\
a \sqsubseteq^{\cal L} b & \equiv & {\cal L}(a) \subseteq {\cal L}(b) .
\end{eqnarray*}
Note that: $\forall a \in A . \, a \Converges \;\; \Longleftrightarrow \;\; a \models_{{\cal A}} \lambda$.
\begin{lemma}
Let ${\cal A}$ be a quasi ats. Then
\[ \forall a, b \in A . \, a \sqsubseteq^{B} b \;\; \Longrightarrow \;\; a \sqsubseteq^{{\cal L}} b . \]
\end{lemma}

\proof\ We assume $a \sqsubseteq^{B} b$ and prove $\forall \phi \in {\cal L} . \,  a \; \models_{{\cal A}} \; \phi \; \Rightarrow \;  b \; \models_{{\cal A}} \; \phi$ by induction on $\phi$. The non-trivial case is $(\phi \rightarrow \psi )_{\bot}$.
\[ \begin{array}{clr}
\bullet & a \; \models_{{\cal A}} \; (\phi \rightarrow \psi )_{\bot} & \\
\Longrightarrow & a \Converges f & \\ 
\Longrightarrow & b \Converges g \: \& \: \forall c . \, f(c) \sqsubseteq^{B} g(c) & \\ 
\Longrightarrow & \forall c . \,  c \; \models_{{\cal A}} \; \phi \; 
\Longrightarrow \;  f(c) \sqsubseteq^{B} g(c) \: \& \: f(c) \; \models_{{\cal A}} \; \psi & \\ 
\Longrightarrow & \forall c . \,  c \; \models_{{\cal A}} \; \phi \; \Rightarrow \;  g(c) \; \models_{{\cal A}} \; \psi & \mbox{ind. hyp.} \\ 
\Longrightarrow & b \; \models_{{\cal A}} \; (\phi \rightarrow \psi )_{\bot} . \;\;\; \qed 
\end{array} \]
To get a converse to this result, we need a condition on ${\cal A}$.
\begin{definition}
{\rm A quasi ats {\cal A} is {\em approximable} iff}
\[ \forall a, b_{1}, \ldots , b_{n} \in A . \, a b_{1} \ldots b_{n} \Converges \; \Rightarrow \; \exists \phi_{1} , \cdots , \phi_{n} . \]
\[ \;\; a \; \models_{{\cal A}} \; (\phi_{1} \rightarrow \cdots ( \phi_{n} \rightarrow \lambda )_{\bot} \cdots )_{\bot} \;\; \& \;\; b_{i} \; \models_{\cal A} \; \phi_{i} , \;\; 1 \leq i \leq n . \]
\end{definition}
This is a natural condition, which says that convergence of a function application is caused by some finite amount of information (observable properties) of its arguments.

As expected, we have
\begin{theorem}[Characterisation Theorem]
Let ${\cal A}$ be an approximable quasi ats. Then
\[ {\preord^{B}} = {\preord^{\cal L}} . \]
\end{theorem}

\proof\ By 5.3, ${\preord^{B}} \subseteq {\preord^{\cal L}}$. For the converse, suppose $a \notpreord^{B} b$. Then for some $k$, $a \notpreord^{B}_{k} b$, and so for some $c_{1}, \cdots , c_{k} \in A$:
\[ a c_{1} \cdots c_{k} \Converges \; \& \; b c_{1} \cdots c_{k} \Diverges . \]
By approximability, for some $\phi_{1} , \cdots , \phi_{k} \in {\cal L}$,
\[ a \; \models_{{\cal A}} \; (\phi_{1} \rightarrow \cdots ( \phi_{k} \rightarrow \lambda )_{\bot} \cdots )_{\bot} \; \& \; b_{i} \; \models_{\cal A} \; \phi_{i} , \;\; 1 \leq i \leq k . \]
Clearly $b \; \nvDash_{{\cal A}} \; (\phi_{1} \rightarrow \cdots ( \phi_{k} \rightarrow \lambda )_{\bot} \cdots )_{\bot}$, and so $a \notpreord^{\cal L} b$. \qed

As a further consequence of approximability, we have:
\begin{proposition}
An approximable quasi ats is an ats.
\end{proposition}

\proof\ Suppose $a \Converges f$ and $b \preord^{B} c$. We must show $f(b) \preord^{B} f(c)$. It is sufficient to show that for all $k \in \omega$, $d_{1}, \ldots , d_{k} \in A$:
\[ f(b) d_{1} \ldots d_{k} \Converges \;\; \Rightarrow \;\; f(c) d_{1} \ldots d_{k} \Converges . \]
Now $f(b) d_{1} \ldots d_{k} \Converges$ implies $a b d_{1} \ldots d_{k} \Converges$; hence by approximability, for some $\phi, \phi_{1}, \ldots \phi_{k} \in {\cal L}$:
\[ a \; \models_{\cal A} \; (\phi_{1} \rightarrow \cdots ( \phi_{k} \rightarrow \lambda )_{\bot} \cdots )_{\bot} \]
and
\[ b \; \models_{\cal A} \; \phi , \;\;  b_{i} \; \models_{\cal A} \; \phi_{i} , \;\; 1 \leq i \leq k . \]
By 5.5, $c \; \models_{\cal A} \phi$, and so $a b d_{1} \ldots d_{k} \; \models_{\cal A} \; \lambda$, and $f(c) d_{1} \ldots d_{k} \Converges$ as required. \qed

We now introduce a proof system for assertions of the form $\phi \leq \psi$, $\phi = \psi$ ($\phi , \psi \in {\cal L})$.
\subsection*{Proof System For $\cal L$}
\[ ({\rm REF}) \;\;\; \phi \leq \phi \]
\[ ({\rm TRANS}) \;\;\; \frac{\phi \leq \psi \;\; \psi \leq \xi}{\phi \leq \xi} \]
\[ (=-I) \;\;\; \frac{\phi \leq \psi \;\; \psi \leq \phi}{\phi = \psi} \]
\[ (=-E) \;\;\; \frac{\phi = \psi}{\phi \leq \psi \;\; \psi \leq \phi} \]
\[(\true\ -I) \;\;\; \phi \leq \true \]
\[ (\wedge -I) \;\;\; \frac{\phi \leq \phi_{1} \;\; \phi \leq \psi_{2}}{\phi \leq \phi_{1} \wedge \phi_{2}} \]
\[ (\wedge -E) \;\;\; \phi \wedge \psi \leq \phi \;\;\;\; \phi \wedge \psi \leq \psi \]
\[ ((\rightarrow )_{\bot}-\leq ) \;\;\; \frac{\phi_{2} \leq \phi_{1} \;\; \psi_{1} \leq \psi_{2}}{(\phi_{1} \rightarrow \psi_{1})_{\bot} \leq (\phi_{2} \rightarrow \psi_{2})_{\bot}} \]
\[ ((\rightarrow )_{\bot}-\wedge ) \;\;\; (\phi \rightarrow \psi_{1} \wedge \psi_{2} )_{\bot} = (\phi \rightarrow \psi_{1})_{\bot} \wedge (\phi \rightarrow \psi_{2})_{\bot} \]
\[ ((\rightarrow )_{\bot}-\true ) \;\;\; (\phi \rightarrow \true )_{\bot} \leq (\true \rightarrow \true )_{\bot} . \]
We write ${\cal L} \; \vdash \; A$ or just $\vdash \; A$ to indicate that an assertion $A$ is derivable from these axioms and rules.
Note that the converse of $((\rightarrow )_{\bot}-\true )$ is derivable from 
$(\true -I)$ and $((\rightarrow )_{\bot}-\leq )$; 
by abuse of notation we refer to the corresponding equation by the same name.

\begin{theorem}[Soundness Theorem]
\label{lsoun}
$\vdash \; \phi \leq \psi \;\; \Longrightarrow \;\; \models \; \phi \leq \psi$.
\end{theorem}

\proof\ By a routine induction on the length of proofs. \qed

So far, our logic has been presented in a syntax-free fashion so far as the elements of the ats are concerned. Now suppose we have an lts ${\cal A}$. $\lambda$-terms can be interpreted in $\cal A$, and for $M \in \BLambda^{0}$, $\rho \in Env(\cal A )$, we can define:
\[ M, \, \rho \; \models_{\cal A} \; \phi \;\; \equiv \;\; \lsem M \rsem^{\cal A}_{\rho} \; \models_{\cal A} \; \phi . \]
We can extend this to arbitrary terms $M \in \BLambda$ in the presence of {\em assumptions} $\Gamma :  {\sf Var} \rightarrow {\cal L}$ on the variables:
\[ M, \, \Gamma \; \models_{\cal A} \; \phi \;\; \equiv \;\; \forall \rho \in Env({\cal A}). \, \rho \; \models_{\cal A} \; \Gamma \;\; \Rightarrow \;\; \lsem M \rsem^{\cal A}_{\rho} \; \models_{\cal A} \; \phi  \]
where
\[ \rho \; \models_{\cal A} \; \Gamma \;\; \equiv \;\; \forall x \in {\sf Var}. \, \rho x \; \models_{\cal A} \; \Gamma x . \]
We write
\[ M, \, \Gamma \; \models \; \phi \;\; \equiv \;\; \forall {\cal A} . \, M, \, \Gamma \; \models_{\cal A} \; \phi . \]

We now introduce a proof system for assertions of the form $M, \, \Gamma \; \vdash \; \phi$.
\subsection*{Proof System For Program Logic}
\[ (TR) \;\;\; M, \, \Gamma \vdash \; \true \]
\[ (AND) \;\;\; \frac{M, \, \Gamma \; \vdash \; \phi \;\; M, \, \Gamma \; \vdash \; \psi}{M, \, \Gamma \; \vdash \; \phi \wedge \psi} \]
\[ (LEQ) \;\;\; \frac{\Gamma \leq \Delta \;\; M, \, \Delta \; \vdash \; \phi \;\; \phi \leq \psi}{M, \, \Gamma \; \vdash \; \psi} \]
\[ (VAR) \;\;\; x, \, \Gamma [ x \mapsto \phi ] \; \vdash \; \phi \]
\[ (ABS) \;\;\; \frac{M, \, \Gamma [ x \mapsto \phi ] \; \vdash \; \psi}{\lambda x . M, \, \Gamma \; \vdash \; (\phi \rightarrow \psi )_{\bot}} \]
\[ (APP) \;\;\; \frac{M, \, \Gamma \; \vdash \; ( \phi \rightarrow \psi )_{\bot} \;\; N, \, \Gamma \; \vdash \; \phi}{MN, \, \Gamma \; \vdash \; \psi} . \]
\begin{theorem}[Soundness of Program Logic]
For all $M$, $\Gamma$, $\phi$:
\[ M, \, \Gamma \: \vdash \: \phi \;\; \Longrightarrow \;\; M, \, \Gamma \: 
\models \: \phi . \;\;\; \qed \]
\end{theorem}
The proof is again routine. 
Note the striking similarity of our program logic with type inference, in particular with the intersection type discipline and Extended Applicative Type Structures of \cite{CDHL84}. 
The crucial {\em difference} lies in the entailment relation $\leq$, and in particular the fact that their axiom (in our notation)
\[ \true \leq (\true \rightarrow \true )_{\bot} \]
is {\em not} a theorem in our logic; instead, we have the weaker $((\rightarrow )_{\bot})$. This reflects a different notion of ``function space''; we discuss this further in section 7.

We now come to the expected connection between the domain logic $\cal L$ 
and the domain $D$. 
Once again, the connecting link is the domain equation used to define $D$, 
and from which $\cal L$ is derived. 
Since this equation corresponds to the type expression 
$\sigma \; \equiv \; {\sf rec} \, t. (t \rightarrow t)_{\bot}$, 
it falls within the scope of the general theory developed in Chapter 4. 
The logic $\cal L$ presented in this section is a streamlined version of 
${\cal L}(\sigma )$ as defined in Chapter 4. 
Once we have shown that $\cal L$ is equivalent to  ${\cal L}(\sigma )$, we 
can apply the results of Chapter 4 to obtain the desired relationships between 
${\cal L} \simeq {\cal L}(\sigma )$ and $D \simeq D(\sigma )$.

Firstly, note that $\cal L$ as presented contains no disjunctive structure, 
while the constructs $\rightarrow$, $(\cdot )_{\bot}$ appearing in $\sigma$ 
generate no inconsistencies according to the definition of {\sf C} in Chapter 4. 
Thus (the Lindenbaum algebra of) ${\cal L}_{\wedge}(\sigma )$, 
the purely conjunctive part of ${\cal L}(\sigma )$, is a meet-semilattice, 
and applying Theorem~\ref{msl}, we obtain
\[ {\sf Spec} \; ({\cal L}(\sigma )/{=_{\sigma}}, {\leq_{\sigma}}/{=_{\sigma}}) \; \cong \; {\sf Filt}({\cal L}_{\wedge}(\sigma )/{=_{\sigma}}, {\leq_{\sigma}}/{=_{\sigma}}) . \]
It remains to show that $\cal L$ is pre-isomorphic to 
${\cal L}_{\wedge}(\sigma )$. 
We can describe the syntax of ${\cal L}_{\wedge}(\sigma )$ as follows:
\begin{itemize}
\item  $L_{\wedge}(\sigma )$:
\[ \phi \;\; ::= \;\; \true \; | \; \phi \wedge \psi \; | \; (\phi )_{\bot} \;\; (\phi \in L(\sigma \rightarrow \sigma )) \]
\item $L_{\wedge}(\sigma \rightarrow \sigma )$:
\[ \phi \;\; ::= \;\; \true \; | \; \phi \wedge \psi \; | \; (\phi \rightarrow \psi) \;\; (\phi , \psi \in L(\sigma )). \] 
\end{itemize}
Using $(()_{\bot}-\wedge )$ and $(\rightarrow -\true )$ (i.e. the nullary instances of $(\rightarrow - \wedge )$) from Chapter 4, we obtain the following normal forms for $L_{\wedge}(\sigma )$:
\[ \phi \;\; ::= \;\; \true \; | \; \phi \wedge \psi \; | \; (\phi \rightarrow \psi )_{\bot} . \]
In this way we see that $L \subseteq L_{\wedge}(\sigma )$, and that each $\phi \in L_{\wedge}(\sigma )$ is equivalent to one in $L$. 
Moreover, the axioms and rules of $\cal L$ are easily seen to be derivable in ${\cal L}_{\wedge}(\sigma )$. 
For example, $((\rightarrow )_{\bot}-\true )$ is derivable, since
\[ {\cal L}_{\wedge}(\sigma ) \; \vdash \; (\phi \rightarrow \psi )_{\bot} = (\true )_{\bot} = (\true \rightarrow \true )_{\bot} . \]
It remains to show the converse, i.e. that for $\phi , \psi \in {\cal L}$:
\[ {\cal L}_{\wedge}(\sigma ) \; \vdash \; \phi \leq \psi \;\; \Longrightarrow \;\; {\cal L} \; \vdash \; \phi \leq \psi . \]
For this purpose, we use $((\rightarrow )_{\bot}-\wedge )$ and $((\rightarrow )_{\bot}-\true )$ to get normal forms for $\cal L$.
\begin{lemma}[Normal Forms]
Every formula in $\cal L$ is equivalent to one in $N{\cal L}$, where:
\[ \bullet \;\; N{\cal L} = \{ \bigwedge_{i \in I} \phi_{i} : I \; {\rm finite}, \; \phi_{i} \in SN{\cal L}, \; i \in I \}\]
\[ \bullet \;\; SN{\cal L} = \{ (\phi_{1} \rightarrow \cdots (\phi_{k} \rightarrow \lambda )_{\bot} \cdots )_{\bot} : k \geq 0, \phi_{i} \in N{\cal L}, \; 1 \leq i \leq k \} . \;\; \qed \]
\end{lemma}
Now by the semantic arguments of Chapter 3, we have
\begin{lemma}
For $\phi$, $\psi$ with
\[ \phi \; \equiv \; \bigwedge_{i \in I} (\phi_{i} \rightarrow \phi'_{i})_{\bot}, \]
\[ \psi \; \equiv \; \bigwedge_{j \in J} (\psi_{j} \rightarrow \psi'_{j})_{\bot} : \]
\[ {\cal L}(\sigma ) \; \vdash \; \phi \leq \psi \;\; \Longleftrightarrow \;\; \forall j \in J.\, 
{\cal L}(\sigma ) \; \vdash \; \bigwedge \{ \phi'_{i} \; : \; {\cal L}(\sigma ) \; \vdash \; \psi_{j} \leq \phi_{i} \} \leq \psi'_{j} . \] 
\end{lemma}
\begin{proposition}
For $\phi , \psi \in N{\cal L}$, if ${\cal L}(\sigma ) \; \vdash \; \phi \leq \psi$ then there is a proof of $\phi \leq \psi$ using only the meet-semilattice laws and the derived rule $((\rightarrow )_{\bot})$.
\end{proposition}

\proof\ By induction on the complexity of $\phi$ and $\psi$, and the preceding Lemma. \qed

We have thus shown that 
\[ {\cal L}(\sigma ) \; \cong \; {\cal L}_{\wedge}(\sigma ) \; \cong \;
{\cal L} , \]
and we can apply the Duality Theorem of Chapter 4 to obtain
\begin{theorem}[Stone Duality]
$\cal L$ is the Stone dual of $\cal D$:
\[ \begin{array}{rl}
(i) & {\cal D} \; \cong \; {\sf Filt} \: {\cal L} \\
(ii) & (K({\cal D}))^{op} \; \cong \; (L/{=}, {\leq}/{=}). 
\end{array} \]
\end{theorem}
\begin{corollary}
${\cal D} \; \models \; \phi \leq \psi \;\; \Longleftrightarrow \;\; {\cal L} \; \vdash \; \phi \leq \psi$. 
\end{corollary}

We can now deal with the program logic over $\lambda$-terms in a similar fashion. The denotational semantics for $\BLambda$ in $\cal D$ given in the precious section can be used to define a translation map
\[ ( \cdot )^{\ast} : \BLambda \rightarrow \BLambda (\sigma ) . \]
The logic presented in this section is equivalent to the endogenous logic of Chapter 4 in the sense that
\[ M,\, \Gamma \; \vdash \; \phi \;\; \Longleftrightarrow \;\; M^{\ast}, \, \Gamma \; \vdash \; \phi \]
where $M \in \BLambda$, $\Gamma : {\sf Var} \rightarrow L$, $\phi \in L \subseteq L(\sigma )$. We omit the details, which by now should be routine. As a consequence of this result, we can apply the Completeness Theorem for Endogenous Logic from Chapter 4, to obtain:
\begin{theorem}
$\cal D$ is $\cal L$-complete, i.e. for all $M \in \BLambda$, $\Gamma : {\sf Var} \rightarrow L$, $\phi \in L \subseteq L(\sigma )$:
\[ M, \, \Gamma \; \vdash \; \phi \;\; \Longleftrightarrow \;\; M, \, \Gamma \; \models_{\cal L} \; \phi . \]
\end{theorem}

In the previous section, we defined an lts over $\cal D$; 
and we have now shown that $\cal D$ is isomorphic to ${\sf Filt} \: {\cal L}$. 
We can in fact describe the lts structure over ${\sf Filt} \: {\cal L}$ directly; 
and this will show how $\cal D$, defined by a domain equation reminiscent of 
the $D_{\infty}$ construction, can also be viewed as a graph model or 
``PSE algebra'' in the terminology of \cite{Lon83}.

\noindent {\bf Notation.} For $X \subseteq L$, $X^{\dag}$ is the filter generated by $X$. This can be defined inductively by:
\begin{itemize}
\item $X \subseteq X^{\dag}$
\item $\true \in X^{\dag}$
\item $\phi , \psi \in X^{\dag} \; \Rightarrow \; \phi \wedge \psi \in X^{\dag}$
\item $\phi \in X^{\dag}, \; {\cal L} \: \vdash \: \phi \leq \psi \;\; \Rightarrow \;\; \psi \in X^{\dag}$ .
\end{itemize}
\begin{definition}
{\rm The quasi-applicative structure with divergence 
\[ ({\sf Filt} \: {\cal L}, \appl\ , \Diverges ) \]
is defined as follows:
\[ \bullet \;\; x \Diverges \;\; \equiv \;\; x = \{ \true \} \]
\[ \bullet \;\; x \appl\  y \;\; \equiv \;\; \{ \psi : \exists \phi . \, (\phi \rightarrow \psi )_{\bot} \in x \: \& \: \phi \in y \} \cup \{ \true \} . 
\]}
\end{definition}
It is easily verified that in this structure
\[ x \preord^{B} y \;\; \Longleftrightarrow \;\; x \subseteq y , \]
and hence that application is monotone in each argument, and ${\sf Filt} \: {\cal L}$ is an ats. Thus we have an interpretation function
\[ \lsem \cdot \rsem^{{\sf Filt} \: {\cal L}} : CL({\sf Filt} \: {\cal L}) \rightarrow Env({\sf Filt} \: {\cal L}) \rightarrow {\sf Filt} \: {\cal L} \]
which is extended to $\BLambda({\sf Filt} \: {\cal L})$ by
\[ \lsem \lambda x . M \rsem^{{\sf Filt} \: {\cal L}}_{\rho} = \{ (\phi \rightarrow \psi )_{\bot} : \psi \in \lsem M \rsem^{{\sf Filt} \: {\cal L}}_{\rho [ x \mapsto \diverges \psi ]} \}^{\dag} . \]
We then define
\begin{definition}
\begin{eqnarray*}
s & \equiv & \lsem \lambda x . \lambda y . \lambda z . (xz) (yz) \rsem^{{\sf Filt} \: {\cal L}} \\
k & \equiv & \lsem \lambda x . \lambda y . x \rsem^{{\sf Filt} \: {\cal L}} . 
\end{eqnarray*}
\end{definition}
\begin{proposition}
\label{isocalg}
${\sf Filt} \: {\cal L}$ is an lts. Moreover, ${\sf Filt} \: {\cal L}$ and $\cal D$ are isomorphic as combinatory algebras.
\end{proposition}

\proof\ It is sufficient to show that the isomorphism of the Duality Theorem 
preserves application, divergence and the denotation of $\lambda$-terms, 
since it then preserves $s$ and $k$ and so is a combinatory isomorphism, and 
${\sf Filt} \: {\cal L}$ is an lts, since $\cal D$ is.

Firstly, we show that application is preserved, i.e. for 
$d_{1}, d_{2} \in {\cal D}$:
\[ (\star ) \;\; {\cal L}(d_{1} \appl\ d_{2}) = {\cal L}(d_{1}) \appl\ {\cal L}(d_{2}) \]
The right to left inclusion follows by the same argument as the soundness of 
$(APP)$ in \ref{lsoun}. 
For the converse, suppose $\psi \in {\cal L}(d_{1} \appl\  d_{2})$, 
${\cal L} \; \nvdash \; \psi = \true$. 
By the Duality Theorem, each $\psi$ in ${\cal L}$ corresponds to a unique 
$c \in K({\cal D}$ with ${\cal L}(c) = {\diverges} \psi$. 
Since application is continuous in $\cal D$, $c \sqsubseteq d_{1}\appl\ d_{2}$, 
$c \neq \bot$ implies that for some $b \in K({\cal D})$, 
${\sf fold}({<}0, [b,c]{>}) \sqsubseteq d_{1}$ and 
$b \sqsubseteq d_{2}$. Let ${\cal L}(b) = {\diverges} \phi$, 
then $(\phi \rightarrow \psi )_{\bot} \in {\cal L}(d_{1})$ and 
$\phi \in {\cal L}(d_{2})$, as required.

Next, we show that denotations of $\lambda$-terms are preserved, i.e. for all 
$M \in \BLambda$, $\rho \in Env({\cal D})$:
\[ (\star \star ) \;\; {\cal L}( \lsem M \rsem^{\cal D}_{\rho}) = \lsem M \rsem^{{\sf Filt} \: {\cal L}}_{{\cal L} \circ \rho} . \]
This is proved by induction on $M$. The case when $M$ is a variable is trivial; 
the case for application uses $(\star )$. For abstraction, 
we argue by structural induction over ${\cal L}$. 
We show the non-trivial case. 
Let $\phi$, $b$ be paired in the isomorphism of the Duality Theorem. Then
\[ \begin{array}{clr}
& \lambda x . M , \, \rho \; \models_{\cal D} \; (\phi \rightarrow \psi )_{\bot} & \\
\Longleftrightarrow &  M , \, \rho [ x \mapsto b ] \; \models_{\cal D} \; \psi & \\
\Longleftrightarrow &  M , \, {\cal L}() \circ (\rho [ x \mapsto b ]) \; \models_{{\sf Filt} \: {\cal L}} \; \psi & \mbox{ind. hyp.} \\
\Longleftrightarrow &  M , \, ({\cal L}() \circ \rho ) [ x \mapsto \diverges \phi ] \; \models_{{\sf Filt} \: {\cal L}} \; \psi  & \\
\Longleftrightarrow & \lambda x . M , \, {\cal L}() \circ \rho \; \models_{{\sf Filt} \: {\cal L}} \; (\phi \rightarrow \psi )_{\bot} . &
\end{array} \]

Finally, divergence is trivially preserved, since the only divergent 
elements in 
$\cal D$, ${\sf Filt} \: {\cal L}$ are $\bot$, $\{ \true \}$, are these are in 
bi-unique correspondence under the isomorphism of the Duality Theorem. \qed

We can now proceed in exact analogy to Chapter 5, and use Stone Duality to convert the Characterisation Theorem into a Final Algebra Theorem.
\begin{definition}
{\rm We define a number of categories of transition systems:
\begin{description}
\item[ATS] Objects: applicative transition systems; morphisms ${\cal A} \rightarrow {\cal B}$: maps $f : A \rightarrow B$ satisfying
\[ a \; \models_{\cal A} \; \phi \;\; \Longleftrightarrow \;\; f(a) \; \models_{\cal B} \; \phi . \]
\item[LTS] The subcategory of {\bf ATS} of lts and morphisms which preserve application, $s$ and $k$.
\item[CLTS] The full subcategory of {\bf LTS} of those $\cal A$ satisfying {\em continuity}:
\[ \psi \neq \true , \; a b \; \models_{\cal A} \; \psi \;\; \Longrightarrow \;\; \exists \phi . \, a \; \models_{\cal A} \; (\phi \rightarrow \psi )_{\bot} \: \& 
\: b \; \models_{\cal A} \; \phi , \]
and also
\[ {\cal L}(s) = \lsem s \rsem^{{\sf Filt} \: {\cal L}} , \;\; {\cal L}(k) = \lsem k \rsem^{{\sf Filt} \: {\cal L}} . \]
Note that continuity implies approximability.
\end{description}}
\end{definition}
\begin{theorem}[Final Algebra]
(i) $\cal D$ is final in {\bf ATS}. \\
(ii) Let $\cal A$ be an approximable lts. The map
\[ {\sf t}_{\cal A} : {\cal A} \rightarrow {\cal D} \]
from (i) is an {\bf LTS} morphism iff $\cal A$ is continuous. \\
(iii) $\cal D$ is final in {\bf CLTS}.
\end{theorem}

\proof\  (i). Given $\cal A$ in {\bf ATS}, define
\[ {\sf t}_{\cal A} : {\cal A} \rightarrow {\cal D} \]
by
\[ {\sf t}_{\cal A} \;\; \equiv \;\; {\cal A} \stackrel{{\cal L}()}{\rightarrow} {\sf Filt} \: {\cal L} \stackrel{\eta}{\rightarrow} {\cal D} \]
where $\eta$ is the isomorphism from the Stone Duality Theorem. For $a \in A$,
\[ {\cal L}(a) = {\cal L} \circ \eta \circ {\cal L}(a) = {\cal L} \circ {\sf t}_{\cal A}(a) , \]
and so ${\sf t}_{\cal A}$ is an {\bf ATS} morphism; moreover, it is unique, since for $d, d' \in D$:
\[ {\cal L}(d) = {\cal L}(d') \; \Rightarrow \; {\cal K}(d) = {\cal K}(d') \; \Rightarrow \; d = d' . \]
(ii). That ${\cal L} ()$ is a combinatory morphism iff $\cal A$ is in {\bf CLTS} is an immediate consequence of the definitions; the result then follows from the fact that $\eta$ is a combinatory isomorphism. \\
(iii). Immediate from (ii). \qed

Note that if $\cal A$ is approximable, we have:
\[ a \preord^{B} b \;\; \Longleftrightarrow \;\; {\sf t}_{\cal A}(a) \preord^{B} {\sf t}_{\cal A}(b) . \]
Thus we can regard the Final Algebra Theorem as giving a syntax-free fully abstract semantics for approximable ats. 
However, from the point of view of applications to programming language semantics, this is not very useful.  
In the next section, we shall study full abstraction in a syntax-directed framework, using our domain logic as a tool.

