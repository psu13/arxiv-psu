\begin{center}
 {\bf Prologue}
\end{center}
\vspace{0.5cm}
The implementation of spontaneous symmetry breaking in the framework
of gauge theories in the 1960s triggered the breakthrough in the
construction of the standard electroweak theory, as it still persists
today. The idea of driving the spontaneous breakdown of a gauge
symmetry by a self-interacting scalar field, which thereby lends mass
to gauge bosons, is known as the {\it Higgs mechanism} and goes back
to the early work of 
%Peter Higgs et al.~\cite
\Brefs{Higgs:1964ia,Higgs:1964pj,Higgs:1966ev,Englert:1964et,Guralnik:1964eu}. 
The postulate of a
new scalar neutral boson, known as the {\it Higgs particle}, comes as
a phenomenological imprint of this mechanism. Since the birth of this
idea, the Higgs boson has successfully escaped detection in spite
of tremendous search activities at the high-energy colliders
LEP and Tevatron, leaving open the crucial question whether the
Higgs mechanism is just a theoretical idea or a `true model'
for electroweak symmetry breaking. The experiments at the Large Hadron
Collider (LHC) will answer this question, either positively upon detecting 
the Higgs boson, or negatively by ruling out the existence of a particle with
properties attributed to the Higgs boson within the Standard Model.
In this sense the outcome of the Higgs search at the LHC will
either carve our present understanding of electroweak interactions
in stone or will be the beginning of a theoretical revolution.

%This is the first Report of the international noisy mathrock band Grumpf
%Quartet to be released in 2011. 10 tracks of powerful and schizo rich and
%inventive physics. 
