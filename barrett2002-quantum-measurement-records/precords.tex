\documentclass [12pt]{article}
%\usepackage{amsmath,amsthm,amscd,amssymb}
\usepackage[noBBpl,sc]{mathpazo}
\usepackage[papersize={6.9in, 10.0in}, left=.5in, right=.5in, top=.7in, bottom=.9in]{geometry}
\linespread{1.05}
\sloppy
\raggedbottom
\pagestyle{plain}

% these include amsmath and that can cause trouble in older docs.
\input{../helpers/cmrsum}
\makeatletter

\DeclareFontFamily{OMX}{MnSymbolE}{}
\DeclareSymbolFont{largesymbolsX}{OMX}{MnSymbolE}{m}{n}
\DeclareFontShape{OMX}{MnSymbolE}{m}{n}{
    <-6>  MnSymbolE5
   <6-7>  MnSymbolE6
   <7-8>  MnSymbolE7
   <8-9>  MnSymbolE8
   <9-10> MnSymbolE9
  <10-12> MnSymbolE10
  <12->   MnSymbolE12}{}

\DeclareMathSymbol{\downbrace}    {\mathord}{largesymbolsX}{'251}
\DeclareMathSymbol{\downbraceg}   {\mathord}{largesymbolsX}{'252}
\DeclareMathSymbol{\downbracegg}  {\mathord}{largesymbolsX}{'253}
\DeclareMathSymbol{\downbraceggg} {\mathord}{largesymbolsX}{'254}
\DeclareMathSymbol{\downbracegggg}{\mathord}{largesymbolsX}{'255}
\DeclareMathSymbol{\upbrace}      {\mathord}{largesymbolsX}{'256}
\DeclareMathSymbol{\upbraceg}     {\mathord}{largesymbolsX}{'257}
\DeclareMathSymbol{\upbracegg}    {\mathord}{largesymbolsX}{'260}
\DeclareMathSymbol{\upbraceggg}   {\mathord}{largesymbolsX}{'261}
\DeclareMathSymbol{\upbracegggg}  {\mathord}{largesymbolsX}{'262}
\DeclareMathSymbol{\braceld}      {\mathord}{largesymbolsX}{'263}
\DeclareMathSymbol{\bracelu}      {\mathord}{largesymbolsX}{'264}
\DeclareMathSymbol{\bracerd}      {\mathord}{largesymbolsX}{'265}
\DeclareMathSymbol{\braceru}      {\mathord}{largesymbolsX}{'266}
\DeclareMathSymbol{\bracemd}      {\mathord}{largesymbolsX}{'267}
\DeclareMathSymbol{\bracemu}      {\mathord}{largesymbolsX}{'270}
\DeclareMathSymbol{\bracemid}     {\mathord}{largesymbolsX}{'271}

\def\horiz@expandable#1#2#3#4#5#6#7#8{%
  \@mathmeasure\z@#7{#8}%
  \@tempdima=\wd\z@
  \@mathmeasure\z@#7{#1}%
  \ifdim\noexpand\wd\z@>\@tempdima
    $\m@th#7#1$%
  \else
    \@mathmeasure\z@#7{#2}%
    \ifdim\noexpand\wd\z@>\@tempdima
      $\m@th#7#2$%
    \else
      \@mathmeasure\z@#7{#3}%
      \ifdim\noexpand\wd\z@>\@tempdima
        $\m@th#7#3$%
      \else
        \@mathmeasure\z@#7{#4}%
        \ifdim\noexpand\wd\z@>\@tempdima
          $\m@th#7#4$%
        \else
          \@mathmeasure\z@#7{#5}%
          \ifdim\noexpand\wd\z@>\@tempdima
            $\m@th#7#5$%
          \else
           #6#7%
          \fi
        \fi
      \fi
    \fi
  \fi}

\def\overbrace@expandable#1#2#3{\vbox{\m@th\ialign{##\crcr
  #1#2{#3}\crcr\noalign{\kern2\p@\nointerlineskip}%
  $\m@th\hfil#2#3\hfil$\crcr}}}
\def\underbrace@expandable#1#2#3{\vtop{\m@th\ialign{##\crcr
  $\m@th\hfil#2#3\hfil$\crcr
  \noalign{\kern2\p@\nointerlineskip}%
  #1#2{#3}\crcr}}}

\def\overbrace@#1#2#3{\vbox{\m@th\ialign{##\crcr
  #1#2\crcr\noalign{\kern2\p@\nointerlineskip}%
  $\m@th\hfil#2#3\hfil$\crcr}}}
\def\underbrace@#1#2#3{\vtop{\m@th\ialign{##\crcr
  $\m@th\hfil#2#3\hfil$\crcr
  \noalign{\kern2\p@\nointerlineskip}%
  #1#2\crcr}}}

\def\bracefill@#1#2#3#4#5{$\m@th#5#1\leaders\hbox{$#4$}\hfill#2\leaders\hbox{$#4$}\hfill#3$}

\def\downbracefill@{\bracefill@\braceld\bracemd\bracerd\bracemid}
\def\upbracefill@{\bracefill@\bracelu\bracemu\braceru\bracemid}

\DeclareRobustCommand{\downbracefill}{\downbracefill@\textstyle}
\DeclareRobustCommand{\upbracefill}{\upbracefill@\textstyle}

\def\upbrace@expandable{%
  \horiz@expandable
    \upbrace
    \upbraceg
    \upbracegg
    \upbraceggg
    \upbracegggg
    \upbracefill@}
\def\downbrace@expandable{%
  \horiz@expandable
    \downbrace
    \downbraceg
    \downbracegg
    \downbraceggg
    \downbracegggg
    \downbracefill@}

\DeclareRobustCommand{\overbrace}[1]{\mathop{\mathpalette{\overbrace@expandable\downbrace@expandable}{#1}}\limits}
\DeclareRobustCommand{\underbrace}[1]{\mathop{\mathpalette{\underbrace@expandable\upbrace@expandable}{#1}}\limits}

\makeatother


\usepackage[small]{titlesec}
\usepackage{cite}
\usepackage{microtype}

% hyperref last because otherwise some things go wrong.
\usepackage[colorlinks=true
,breaklinks=true
,urlcolor=blue
,anchorcolor=blue
,citecolor=blue
,filecolor=blue
,linkcolor=blue
,menucolor=blue
,linktocpage=true]{hyperref}
\hypersetup{
bookmarksopen=true,
bookmarksnumbered=true,
bookmarksopenlevel=10
}

% make sure there is enough TOC for reasonable pdf bookmarks.
\setcounter{tocdepth}{3}

%\usepackage[dotinlabels]{titletoc}
%\titlelabel{{\thetitle}.\quad}
%\titleformat{\section}[block]
  {\fillast\medskip}
  {{\thesection. }}
  {1ex minus .1ex}
  {\scshape}
 
\titleformat*{\subsection}{\itshape}
\titleformat*{\subsubsection}{\itshape}

\setcounter{tocdepth}{2}

\titlecontents{section}
              [2.3em] 
              {\bigskip}
              {{\contentslabel{2.3em}}\large\scshape}
              {\hspace*{-2.3em}}
              {\titlerule*[1pc]{}\contentspage}
              
\titlecontents{subsection}
              [4.7em] 
              {}
              {{\contentslabel{2.3em}}}
              {\hspace*{-2.3em}}
              {\titlerule*[.5pc]{}\contentspage}

% hopefully not used.           
\titlecontents{subsubsection}
              [7.9em]
              {}
              {{\contentslabel{3.3em}}}
              {\hspace*{-3.3em}}
              {\titlerule*[.5pc]{}\contentspage}
%\makeatletter
\renewcommand\tableofcontents{%
    \section*{\contentsname
        \@mkboth{%
           \MakeLowercase\contentsname}{\MakeLowercase\contentsname}}%
    \@starttoc{toc}%
    }
\def\@oddhead{{\scshape\rightmark}\hfil{\small\scshape\thepage}}%
\def\sectionmark#1{%
      \markright{\MakeLowercase{%
        \ifnum \c@secnumdepth >\m@ne
          \thesection\quad
        \fi
        #1}}}
        
\makeatother



%\makeatletter

 \def\small{%
  \@setfontsize\small\@xipt{13pt}%
  \abovedisplayskip 8\p@ \@plus3\p@ \@minus6\p@
  \belowdisplayskip \abovedisplayskip
  \abovedisplayshortskip \z@ \@plus3\p@
  \belowdisplayshortskip 6.5\p@ \@plus3.5\p@ \@minus3\p@
  \def\@listi{%
    \leftmargin\leftmargini
    \topsep 9\p@ \@plus3\p@ \@minus5\p@
    \parsep 4.5\p@ \@plus2\p@ \@minus\p@
    \itemsep \parsep
  }%
}%
 \def\footnotesize{%
  \@setfontsize\footnotesize\@xpt{12pt}%
  \abovedisplayskip 10\p@ \@plus2\p@ \@minus5\p@
  \belowdisplayskip \abovedisplayskip
  \abovedisplayshortskip \z@ \@plus3\p@
  \belowdisplayshortskip 6\p@ \@plus3\p@ \@minus3\p@
  \def\@listi{%
    \leftmargin\leftmargini
    \topsep 6\p@ \@plus2\p@ \@minus2\p@
    \parsep 3\p@ \@plus2\p@ \@minus\p@
    \itemsep \parsep
  }%
}%
\def\open@column@one#1{%
 \ltxgrid@info@sw{\class@info{\string\open@column@one\string#1}}{}%
 \unvbox\pagesofar
  \gdef\thepagegrid{one}%
 \global\pagegrid@col#1%
 \global\pagegrid@cur\@ne
 \global\count\footins\@m
 \set@column@hsize\pagegrid@col
 \set@colht
}%

\def\frontmatter@abstractheading{%
\bigskip
 \begingroup
  \centering\large
  \abstractname
  \par\bigskip
 \endgroup
}%

\makeatother

%\DeclareSymbolFont{CMlargesymbols}{OMX}{cmex}{m}{n}
%\DeclareMathSymbol{\sum}{\mathop}{CMlargesymbols}{"50}
%\pdfbookmark[1]{Introduction}{Introduction}

\begin{document}

\title{On the Nature of Measurement Records in Relativistic
Quantum Field Theory}

\author{Jeffrey A. Barrett}

\date{14 August 2000}

\maketitle


\begin{abstract}
A resolution of the quantum measurement problem would require one to
explain how it is that we end up with determinate records at the end of
our measurements.  Metaphysical commitments typically do real work in
such an explanation.  Indeed, one should not be satisfied with one's
metaphysical commitments unless one can provide some account of
determinate measurement records.  I will explain some of the problems
in getting determinate records in relativistic quantum field theory and
pay particular attention to the relationship between the measurement
problem and a generalized version of Malament's theorem.
\end{abstract}


\section{Introduction}

Does relativistic quantum field theory tell us that the world is made
of fields or particles or something else?  One difficulty in answering
this is that physical theories typically do not pin down a
single preferred ontology.  This can be seen in classical
mechanics where we are some 350 years on, and we do not have anything like
a canonical metaphysics for the theory.  Are the fundamental entities of classical
mechanics point particles or are they extended objects?  Does the
theory tell us that there is an absolute sustantival space or are
positions only relative to other objects?  Of course, part of the
problem here is that it is not entirely clear what classical mechanics is.
But even if one does the reconstruction work that it would take to
get a sharp formal theory, one can always provide alternative
metaphysical interpretations.  This can be seen as an aspect of a
general underdetermination problem: not only are physical theories
typically underdetermined by empirical evidence, but one's ontological
commitments are typically underdetermined by physical
theories one adopts.

If our physical theories are in fact always subject to interpretation, then
one might take the debate over the {\em proper\/} ontology of relativistic quantum
field theory to be futile.  While there is something right in this
reaction, metaphysical considerations have in the past proven important
to understanding and to clearly formulating physical theories, and we could
certainly use all the clarity we can get in finding a satisfactory formulation
of relativistic quantum field theory.  If one could cook up a satisfactory
ontology for some formulation of relativistic quantum field theory, then
it would mean that that formulation of the theory {\em could\/} be understood
as descriptive of the physical world, and in the context of relativistic
quantum field theory, this would be something new.  What is required here
is not just showing that the particular theory is logically consistent by
providing a model; what we want is to show that the theory could be
descriptive of our physical world.

One of the features of our world is that we have determinate measurement
records.  We perform experiments, record the results, then compare these
results against the predictions of our physical theories.  Measurement
records then should somehow show up in the ontology
that we associate with our best physical theory.  Indeed, if not
for the existence of such records, it would be difficult
to account for the possibility of empirical science all.

I mention this aspect of our world because the existence of determinate
records is something that is difficult to get in nonrelativistic quantum
mechanics and more difficult to get in relativistic quantum
mechanics.  The problem of getting determinate measurement records is the
quantum measurement problem.

Metaphysics typically does real work in solutions to the quantum
measurement problem by providing the raw material for explaining
how it is that we have determinate measurement records.  We see this in
solutions to the quantum measurement problem in nonrelativistic quantum mechanics.  In
Bohm's theory it is the always determinate particle positions that
provide determinate measurement records.  In many-world interpretations
it is the determinate facts in the world inhabited by a particular
observer that determines the content of that observer's records.

The point here is just that in quantum mechanics one's metaphysical
commitments must be sensitive to how one goes about solving the measurement
problem.  Indeed, it seems to me that no metaphysics for relativistic
quantum field theory can be considered satisfactory unless determinate
measurement records somehow show up in one's description of the world.
Put another way, one must have a solution to the quantum
measurement problem before one can trust any specific interpretation
of relativistic quantum field theory.


\section{The Measurement Problem}

The measurement problem arises in nonrelativistic quantum mechanics when
one tries to explain how it is that we get determinate measurement records.
If the deterministic unitary dynamics (the time-dependent Schr\"odinger
equation in nonrelativistic quantum mechanics) described all physical
interactions, then a measurement would typically result in an entangled
superposition of one's measuring apparatus recording mutually contradictory
outcomes.  If one has a good measuring apparatus that starts ready to
make a measurement, the linear dynamics predicts one would typically
end up with something like:
\begin{equation}
\sum a_i |p_i \rangle_S |\mbox{``$p_i$''} \rangle_M
\end{equation}
This is a state where the measured system $S$ having property $p_1$ and
the measuring apparatus $M$ recording that the measured
system has property $p_1$) is superposed with (the measured
system $S$ having property $p_2$ and the measuring apparatus $M$
recording that the measured system has property $p_2$) etc.  And
this clearly does not describe the measuring apparatus $M$ as
recording any particular determinate measurement record.\footnote{See Barrett (1999)
for a detailed account of what it means to have a good measuring device
and why it would necessarily end up in this sort of entangled state.} 

This indeterminacy problem is solved on the standard von Neumann-Dirac
formulation of nonrelativistic quantum mechanics by stipulating that the
state of the measured system randomly collapses to an eigenstate of the observable
being measured whenever one makes a measurement, where the probability
of collapse to the state $|p_k  \rangle_S |\mbox{``$p_k$''}  \rangle_M$ is
$|a_k|^2$.  It is this collapse of the state that generates a determinate
measurement record ($|p_k  \rangle_S |\mbox{``$p_k$''}  \rangle_M$ is a state
where $S$ determinately has property $p_k$ and $M$ determinately records
that $S$ has property $p_k$).  But it is notoriously difficult to provide
an account of how and when collapses occur that does not look blatantly
ad hoc and even harder to provide and account that is consistent with the
demands of relativity.\footnote{For two related attempts to get
a collapse story that satisfies the demands of relativity see Aharonov and
Albert (1980) and Fleming (1988, 1996).}

If there are is no collapse of the quantum mechanical state on measurement,
then one might try adding something to the usual quantum-mechanical state
that represents the values of the determinate physical records.  This
so-called hidden variable would determine the value of one's
determinate measurement record even when the usual quantum-mechanical
state represents an entangled superposition of incompatible
records.  But it is also unclear how to describe the
evolution of this extra component of the physical state in a way
that is compatible with relativity.\footnote{Much of the literature on this
topic is concerned with either trying to find a version of Bohm's theory
that is compatible with relativity or trying to explain why strict
compatibility between the two theories is not really necessary.}

It is orthodox dogma that it is only possible to reconcile quantum mechanics and
relativity in the context of a quantum {\em field\/} theory, where the fundamental
entities are fields rather than particles.\footnote{This is the position expressed,
for example, by Steven Weinberg (1987, 78--9).  See David Malament (1996, 1--9)}  While
there may be other reasons for believing that we need a field theory in order
to reconcile quantum mechanics and relativity (and we will consider one
of these shortly), relativistic quantum field theory
does nothing to solve the quantum measurement problem and it is easy to see why.

In relativistic quantum field theory one starts by adopting an
appropriate relativistic generalization of the unitary dynamics.  The
relativistic dynamics describes the relations that must hold between quantum-mechanical field
states in neighboring space-time regions.  By knowing how the field states in different
space-time regions are related, one can make statistical predictions concerning
expected correlations between measurements performed on the various field quantities.
But relativistic quantum field theory provides no account whatsoever for
how determinate measurement records might be generated.

The problem here is analogous to the problem that arises in nonrelativistic quantum
mechanics.  If the possible determinate measurement records are supposed to be 
represented by the elements of some set of orthogonal field configurations,
then there typically are no determinate measurement records since (given the 
unitary dynamics) the state of the field in a given space-time region
will typically be an entangled superposition of different elements of
the orthogonal set of field configurations.  An appropriate
collapse of the field would generate a determinate local field configuration
which might in turn represent a determinate measurement result, but such an evolution
of the state would violate the relativistic unitary dynamics.  And, as it is usually
presented, relativistic quantum field theory has nothing to say about the conditions under which a such a
collapse might occur, nor does it have
anything to say about how such an evolution might be made compatible with relativity.  One
might try adding a new physical parameter that 
represents the values of one's determinate measurement records
to the usual quantum mechanical state.  But relativistic quantum field theory has nothing to 
say about how to do this or about how one might then give a relativity-compatible
dynamics for the new physical parameter.\footnote{That one can predict statistical
correlations between measurement results but cannot explain the determinate measurement
results has led some (see Rovelli (1997) and Mermin (1998) for example), to conclude that relativistic quantum
field theory (and quantum mechanics more generally) predicts statistical correlations without there being
anything that is in fact statistically correlated---``correlations without
correlata.''  The natural objection is that the very notion of there being
statistical correlations between measurement records presumably requires that
there be determinate measurement records.}

So relativistic quantum field theory does nothing to solve the quantum
measurement problem.  Indeed, because of the additional relativistic
constraints, accounting for determinate measurement records is
more difficult than ever.

In what follows, I will explain another sense in which the metaphysics
of relativistic quantum mechanics must be sensitive to measurement considerations
and why we are far from having a clear account of measurement in
relativistic quantum mechanics.


\section{Malament's Theorem}

David Malament (1996) presented his local entities no-go theorem in defense
of the dogma that  a field ontology (rather than a particle ontology) is
appropriate to relativistic quantum mechanics.  The theorem follows from four
apparently weak conditions that most physicists would expect to be satisfied
by the structure one would use to represent the state of a single particle
in relativistic quantum mechanics.  If these conditions are satisfied, then the
theorem entails that the probability of finding the particle
in any closed spatial region must be zero, and this presumably violates
the assumption that there is a (detectable) particle at all.  Malament thus concludes that
a particle ontology is inappropriate for relativistic quantum mechanics.

A version of Malament's theorem can be proven that applies equally well
to point particles or extended objects.  I will describe this version
of the theorem without proof.\footnote{The proof of this version of the theorem
is essentially the same as the proof in Malament (1996).  The only difference
is the physical interpretation of $P_\Delta$.  Malament's theorem relies
on a lemma by Borchers (1967).}  The statement of the theorem below
and its physical interpretation follows Malament (1996) with a few supporting
comments.


Let $M$ be Minkowski space-time, and let ${\cal H}$ be a Hilbert space where a ray in
${\cal H}$ represents the pure state of the object $S$.  Let $P_\Delta$ be the projection
operator on ${\cal H}$ that represents the proposition that the object $S$ would be
detected to be {\em entirely\/} within spatial set $\Delta$ if a detection experiment were performed.
Relativistic quantum mechanics presumably requires one to satisfy at least
the following four conditions.

(1)  {\em Dynamics Translation Covariance Condition}:  For all
vectors $a$ in $M$ and for all spatial sets $\Delta$
\begin{equation}
P_{\Delta+a}= U(a)P_\Delta U(-a)
\end{equation}
where $a \mapsto U(a)$ is a strongly continuous, unitary representation in ${\cal H}$
of the translation group in $M$ and $\Delta+a$ is the set that results
from translating $\Delta$ by the vector $a$.

This condition stipulates that the dynamics is represented by a family of unitary
operators.  More specifically, it says that
the projection operator that represents the proposition that the object would be
detected within spatial region $\Delta+a$ can be obtained by a unitary transformation that
depends only on $a$ of the projection operator that represents the proposition that the
object would be detected within region $\Delta$.  Note that if this condition is universally
satisfied, then there can be no collapse of the quantum-mechanical state.

(2) {\em Finite Energy Condition}:  For all future-directed time-like vectors $a$
in $M$, if $H(a)$ is the unique
self-adjoint operator satisfying
\begin{equation}
U(t a)=\exp{-i t H(a)},
\end{equation}
then the spectrum of H(a) is bounded below.

$H(a)$ is the Hamiltonian of the system $S$.  It represents the energy properties
of the system and determines the unitary dynamics (by the relation above).
Supposing that the spectrum of the Hamiltonian is bounded below amounts to supposing
that $S$ has a finite (energy) ground state.

(3)  {\em Hyperplane Localizability Condition}:  If $\Delta_1$ and $\Delta_2$
are disjoint spatial sets in the same hyperplane,
\begin{equation}
P_{\Delta_1}P_{\Delta_2} = P_{\Delta_2}P_{\Delta_1} = \mbox{\bf 0}
\end{equation}
where $\mbox{\bf 0}$ is the zero operator on {\cal H}.

This condition is supposed to capture the intuition that a single object $S$
cannot be entirely within any two disjoint regions at the same time (relative to any inertial frame).  This
is presumably part of what it would mean to say that there is {\em just one\/}
spatially extended object.

(4)  {\em General Locality Condition}:  If $\Delta_1$ and $\Delta_2$ are any two disjoint spatial
sets that are spacelike related (perhaps not on the same hyperplane!),
\begin{equation}
P_{\Delta_1}P_{\Delta_2} = P_{\Delta_2}P_{\Delta_1}.
\end{equation}

Relativity together with what it means to be an object presumably requires
that if an object were detected to be entirely within one spatial region, then since
an object cannot travel faster than light, it
could not also be detected to be entirely within a disjoint,
space-like related region in any inertial frame.  If this is right, then one would
expect the following to hold

($\diamondsuit$)  {\em Relativistic Object Condition\/}:  For {\em any\/} two spacelike
related spatial regions $\Delta_1$ and $\Delta_2$ (not just
any two in the same hyperplane!)
\begin{equation}
P_{\Delta_1}P_{\Delta_2} = P_{\Delta_2}P_{\Delta_1} = \mbox{\bf 0}.
\end{equation}

Condition ($\diamondsuit$) is strictly stronger than the conjunction of
conditions (3) and (4).  The idea behind condition (4) is that even if it {\em were\/}
possible to detect $S$ to be entirely within two disjoint spacelike related spatial regions and
if condition (3) were still satisfied (because the two detectors were in different inertial
frames and $\Delta_1$ and $\Delta_2$ were consequently not in the same hyperplane),
then the probability of detecting the object to be entirely within $\Delta_1$ should at
least be statistically independent of the probability of detecting it to be entirely within
$\Delta_2$.  That is, proving the theorem from conditions (3) and (4) rather than the strictly
stronger (but very plausible!) condition ($\diamondsuit$) allows for the possibility that
particle detection in a particular space-time region might be hyperplane dependent.  While
this is certainly something that Malament would want to allow for (since he was responding to
Fleming's hyperplane-dependent formulation of quantum mechanics), it is probably not a
possibility that most physicists would worry about much.  If this is right, then one might
be perfectly happy replacing conditions (3) and (4) by condition ($\diamondsuit$).


The theorem is that if conditions (1)--(4) are satisfied (or conditions (1), (2), and
($\diamondsuit$)), then $P_\Delta=\mbox{\bf 0}$ for all compact closed spatial sets
$\Delta$.  This means that the only extended object possible (or, perhaps better,
the only {\em detectable\/} extended object possible) is one with infinite
extension.  And this conclusion is taken to favor a field ontology.  It may also
have curious implications for the nature of one's measurement records in relativistic
quantum mechanics.  Or it may be that getting determinate measurement
records in relativistic quantum mechanics requires one to violate one or
more of the four conditions that make the theorem possible.



\section{Measurement Records}


In the broadest sense, a good measurement consists in correlating the state
of a record with the physical property being measured.  The goal is to produce a
detectable, reliable, and stable record.  It might be made in terms of ink marks on paper,
the final position of the pointer on a measuring device, the bio-chemical state
of an observer's brain, or the arrangement of megaliths on the Salisbury Plain,
but whatever the medium, useful measurement records must be detectable (so that
one can know the value of the record), reliable (so that one can correctly infer
the value of the physical property that one wanted to measure), and stable (so
that one can make reliable inferences concerning physical states at different
times).  Such measurement records provide the evidence on which empirical science
is grounded.

Consider the following simple experiment where I test my one-handed
typing skills.  This experiment involves, as all do, making a
measurement.

\vspace{.5in}
\noindent
The time it took me to type this sentence one-handed (because I am holding
a stopwatch in the other hand) up to the following colon: {\bf 41.29 seconds}.

\vspace{.5in}
\noindent
I am indeed a slow typist, but that is not the point.  The point is that I
measured then recorded how long it took me to type the above sentence fragment
one-handed; and because I have a determinate, detectable, reliable, and stable record
token, I know how long it took to type the sentence fragment, and you do
too if you have interacted with the above token of the measurement record in an
appropriate way.

Setting aside the question of exactly what it might mean for a measurement record
to be reliable and stable, let's consider the detectability condition.  For a record
token to be detectable, it must presumably be the sort of thing one can
{\em find}.  And in order to be the sort of thing one can find, the
presence or absence of a detectable record token $R$ must presumably be something
that can be represented in quantum mechanics as a projection operator on
a finite spatial region.  That is, there must be a projection
operator $R_\Delta$ that represents the proposition that there is an $R$-record
in region $\Delta$.  This is apparently just part of
what it means for a record to be detectable in relativistic quantum mechanics.

Now consider the bold-faced typing-speed record token above.  It is detectable.
Not only can you find and read it, but you can find and read it in
a finite time.  If we rule out superluminal effects, then it seems that
the {\em detected record token\/} must occupy a finite spatial region.  Call
this spatial region $S$.  Given the way that observables are represented
in relativistic quantum mechanics, this means that
there must be a projection operator $R_S$ that represents the proposition
that there is a token of the {\bf 41.29 seconds} record in region $S$.

The problem with this is that Malament's theorem tells us that there
can be no such record-detection operator.  More specifically, it tells us that
$R_\Delta=\mbox{\bf 0}$ for all closed sets $\Delta$, which means that the
probability of finding the record token in the spatial set $S$ is zero.
 Indeed, the probability of finding the (above!?) record token anywhere is
zero.  But how can this be if there is in fact a detectable
record token at all?  And if there is no detectable record token, then how
can you and I know the result of my typing-speed measurement
as we both presumably do?


A natural reaction would be to deny the
assumption that a detectable record token is a detectable
entity that occupies a finite spatial region and insist that
in relativistic quantum field theory, as one would expect,
all determinate record tokens are represented in the determinate
configuration of some unbounded field.\footnote{I would like to thank
Rob Clifton for his defense of this eminently reasonable line of
argument in conversation.}  After
all, this is presumably how records would have to be represented
in {\em any\/} field theory.  Couldn't a determinate measurement
record be represented, say, in the local configuration of an unbounded
field?  Sure, but there are a couple of problems one would
still have to solve in order to have a satisfactory account of
determinate measurement records.

One problem, of course, is the old one.  Given the unitary dynamics and
the standard interpretation of states, relativistic quantum
field theory would typically not predict a
determinate local field configuration in a spacetime
region.  But let's set the traditional measurement problem
aside for a moment and suppose that we can somehow cook
up a formulation of the theory where one typically
does have a determinate local field configurations at the end of
a measurement.  

If one could somehow get determinate local field configurations that
are appropriately correlated,
then one could explain how it is possible for me to know my typing
speed by stipulating that my mental state supervenes on the
determinate value of some field quantity in a some spatial region
region that, in turn, is reliably correlated with my typing speed.
So not only is it possible for a local field configuration to represent
a determinate measurement result, but one can explain how it is
possible for an observer to know the value of the record by stipulating
an appropriate supervenience relation between mental and physical
states.  What more could one want?

It seems to me that one should ultimately want to explain how
our actual measurements might yield determinate records.  But
to do this, one needs an account of measurement
records that makes sense of the experiments that
we in fact perform, and the problem is that our records seem to be
spacio-temporal things.  They seem to be the sort of things
that have a location; the sort of things that one can find, lose,
and move from one place to another.  Indeed, we use their
spacio-temporal properties to individuate our records.  In order
to know how fast I typed the sentence, I must be able to find the
right record, and this (apparently) amounts to looking for it
in the right place.  It seems then that we know where our
records are, and this is good because, given the way that we individuate
our records, one must know where a record is in order to read it and to
know what one is reading!  This is just a point about our 
experimental practice and conventions.


So it seems that our actual records are in fact
detectable in particular spacetime regions.  But if this is right,
then there must be detection-of-a-record-at-a-location operators
($R_\Delta$ that represent the proposition that there is a record in region
$\Delta$).  And if these are subject to Malament's theorem, then we have a
puzzle: there apparently cannot be detectable records of just the sort
that we take ourselves to have.

This is particularly puzzling when one considers the sort of 
records that are supposed to provide the empirical support for relativistic
quantum field theory itself.  These records are supposed to include
such things as photographs of the trajectories of fundamental particles,
but if there are no detectable spacio-temporal
entities, then how could there be a photographic trajectory record with
a detectable shape?  The shape of the trajectory is supposed to represent
all of the empirical evidence that one has, but it seems, at least at first 
pass, that there can be no detectable entities that have determinate
shapes given Malament's theorem.


While Malament's theorem arguably does nothing to prohibit an entity from
having a determinate position, it does seem to prohibit anything from having a
{\em detectable} position.  But detectable positions are just what our records
apparently have: they are typically individuated by position,
so one must be able to find a record at a location to read it and to
know what one is reading, and, given our practice and conventions,
the records themselves are typically supposed to be
made in terms of the detectable position or shape of something.


One might argue that one does not need to know
where a record is in order to set up the appropriate correlations
in order to read a record or that one
can know where the record is and thus set up the appropriate correlations
to read the record without the position of the record itself being detectable.
And while one can easily see how each of these lines of argument would
go, it seems to me that is our actual practice that ultimately renders
such arguments implausible.  If I forget what my typing speed was, then I need to
find a stable reliable record, and, given the way that I
recorded it and the way that I individuate my records, in order to find one, I
must do a series position detection observations: Only if I can
find {\em where\/} the record token is, can I
then determine {\em what\/} it is.

The situation is made more puzzling by the fact that
we are also used to treating observers themselves as localizable
entities in order to get specific empirical predictions out
of our physical theories.\footnote{Consider,
for example, Galileo comparing the motions of the planets against
theoretical predictions.  That he, the observer, has a
specifiable relative position is needed for the theory
to make any empirical predictions, and without comparing
such predictions against what he actually sees, he would
never be able to judge the empirical merits of
the theory.}  The location an observer occupies provides
the observer with the spacio-temporal perspective that we use to explain why the
world appears the way it does to {\em that\/} observer and not
the way it might to another.  We also use
the fact that an observer occupies a location to explain why
his empirical knowledge has spacio-temporal constraints.\footnote{If
{\em I\/} am represented in the configuration of an {\em unbounded\/} relativistic
field, then why don't I know what is happening around $\alpha$-Centauri
right now (in my inertial frame---whatever that might be if I have no
fully determinate position!)?  After all, on this representation of me, I would be
there now.  Or, for that matter, why would I not
know what will happen here two minutes from now?}

If detectable spacio-temporal objects are incompatible
with relativistic quantum mechanics, then the challenge is to
explain why it seems that we and those physical objects to which we have
the most direct epistemic access (our measurement records) are just
such objects.\footnote{Note that the problem of
explaining how we could have the records we have without there being detectable
spacio-temporal objects is more basic than the problem of explaining why
it appears that there are detectable particles or other extended objects
since the only way that we know of other spacio-temporal
objects is via our records of them (in terms of patches of photographic
pigment, or patterns of neurons firing on one's retina, etc.).}  As far as I
can tell, it is possible that all observers and their records are somehow
represented in field configurations; it is just unclear how the
making, finding, and reading of such records is supposed to work
in relativistic quantum field theory.  Perhaps one could argue that
observers and their records have only approximate
positions and that this is enough for us to individuate them (and make
sense of what it means for theory to be empirically adequate for
a given observer), then argue that there is nothing analogous
to Malament's theorem in relativistic quantum mechanics that prevents
there from being detectable entities with only approximately determinate
positions.  Our standard talk of detectable
localized objects might then be translated into the physics of
such quasi-detectable, quasi-localized objects.  But this would require some
careful explaining.

But it may well be that none of this matters after all.  The real problem,
the one on which the solution to the others must hang, is the one we set aside
at the beginning of this section.

While theorems like Malament's might be relevant
to what metaphysical morals one should draw from relativistic quantum mechanics,
whether such theorems hold or not is itself contingent on how one goes about
solving the quantum measurement problem.  A collapse formulation of quantum
mechanics would, for example, typically violate condition~(1): The dynamics
translation covariance condition is an assumption concerning how
physical states in different space-time regions are related, and
it is incompatible with a collapse of the quantum mechanical state
on measurement.  But if we might have to violate the apparently
weak and obvious assumptions that go into proving Malament's theorem in
order to get a satisfactory solution to the measurement problem, then 
all bets are off concerning the applicability of the theorem to the
detectable entities that inhabit our world.\footnote{The very possibility of
having to violate such conditions might be taken to illustrate how difficult it is
to solve the measurement problem and satisfy relativistic constraints.}

The upshot is that we are very nearly back where we started: one cannot
trust any specific metaphysical conclusions one draws from relativistic quantum
field theory without a solution to the quantum measurement problem, and
we have every reason to suppose that the constraints imposed by relativity
will make finding a satisfactory solution more difficult than ever.


\section{Conclusion}



An adequate resolution of the quantum measurement problem would explain
how it is that we have the determinate measurement records that
we take ourselves to have.  It has proven difficult to find a satisfactory
resolution of the measurement problem in 
the context of nonrelativistic quantum mechanics, and relativistic
quantum mechanics does nothing to make the task any easier.  Indeed,
the constraints imposed by relativity make explaining how we end up
with the determinate, detectable, physical records all the more difficult.


Since one's ontological commitments typically do real work in proposed resolutions
to the quantum measurement problem in nonrelativistic quantum mechanics,
it would be a mistake to try to draw any conclusions
concerning the proper ontology of relativistic quantum field theory
without a particular resolution to the measurement problem in mind.  This point
is clearly made by the fact that one cannot even know whether the so-called
local entities no-go theorems are relevant to one's theory if one does not
know what to do about the quantum measurement problem.\footnote{I would like to thank
David Malament and Rob Clifton for valuable comments on a earlier draft of this paper.}





\newpage
\begin{center}
\Large
REFERENCES
\normalsize
\end{center}
\vspace{.5in}

\vspace{5mm}
\noindent
Aharonov, Y.\ and Albert, D.: 1980 ``States and Observables in Relativistic Quantum
Field Theory,'' {\em Physical Review\/} D 21:3316--24.

\vspace{5mm}
1981 ``Can We Make Sense out of the Measurement Process in Reltivistic Quantum
Mechanics,'' {\em Physical Review\/} D 24:359--70.

\vspace{5mm}
\noindent
Barrett, J.\ A.: 1999 {\em The Quantum Mechanics of Minds and Worlds},
Oxford: Oxford University Press.

\vspace{5mm}
1996, `Empirical Adequacy and the Availability of Reliable
Records in Quantum Mechanics,' {\em Philosophy of Science\/} 63:
49--64.

\vspace{5mm}
\noindent
Borchers, H.\ J.: 1967 ``A Remark on Theorem by B.\ Misra,''  {\em Communications
in Mathematical Physics\/} 4:315--323.

\vspace{5mm}
\noindent
Dickson, M.\ and R.\ Clifton: 1998 ``Lorentz-Invariance in Modal Interprerations,''
in D.\ Dieks and P.\ Vermaas (eds.) {\em The Modal Interpretation of Quantum
Mechanics}, Dordrecht: Kluwer, 9--48.

\vspace{5mm}
\noindent
Fleming, G.: 1988 ``Hyperplane-Dependent Quantized Fields and Lorentz Invariance,''
in H.\ R.\ Brown and R.\ Harr\'e (eds.\ ) {\em Philosophical Foundations of Quantum Field
Theory\/}, Oxford: Clarendon Press, 93--115. 

\vspace{5mm}
\noindent
Malament, D.\ B.: 1996 ``In Defense of Dogma: Why There Cannot be a Relativistic
Quantum Mechanics of (Localizable) Particles,'' in R.\ Clifton (ed.),
{\em Perspectives on Quantum Reality}, The Netherlands: Kluwer Academic Publishers,
1--10.

\vspace{5mm}
\noindent
Mermin, D.: 1998 ``What is Quantum Mechanics Trying to Tell Us?'' xxx.lanl.gov archive
preprint quant-ph/9801057 v2 (2 Sep 1998). 

\vspace{5mm}
\noindent
Rovelli, C.: 1997 ``Relational Quantum Mechanics'' xxx.lanl.gov preprint
quant-ph/9609002 v2 (24 Feb 1997).


\vspace{5mm}
\noindent
Weinberg, S.: 1987 {\em Elementary Particles and the Laws of Physics, The 1986
Dirac Memorial Lectures}, Cambridge: Cambridge University Press.



\end{document}



