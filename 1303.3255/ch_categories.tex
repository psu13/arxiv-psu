
%%%%%%%%%%%%%%%%%%%%%%%%%%%%%%%%%%%%%%%%%%%%%%%%%%%%%%%%%%%%%
%%%%%%%%%%%%%%%%%%%%%%%%%%%%%%%%%%%%%%%%%%%%%%%%%%%%%%%%%%%%%
%
%  %%%%%%%  %    %  %%%%%%  %%%%%%  %%%%%%%  %%%%%%%  %%%%%
%  %        %    %  %    %  %     %    %     %        %    %
%  %        %    %  %%%%%%  %     %    %     %        %    %
%  %        %%%%%%  %    %  %%%%%%     %     %%%%%    %%%%%
%  %        %    %  %    %  %          %     %        %    %
%  %%%%%%%  %    %  %    %  %          %     %%%%%%%  %     %
%
%%%%%%%%%%%%%%%%%%%%%%%%%%%%%%%%%%%%%%%%%%%%%%%%%%%%%%%%%%%%%
%%%%%%%%%%%%%%%%%%%%%%%%%%%%%%%%%%%%%%%%%%%%%%%%%%%%%%%%%%%%%

\chapter{A Primer on Category Theory}% (fold)
\label{sec:categories}

\begin{quote}
{\em ``A healthy new seed was planted some twenty odd years ago in the well fertilized soil of the mathematical periodical literature --- the notion of a category. It sprouted, took root, flowered, attracted bees, and by now the landscape is dotted with its progeny. It is a beneficent plant: mathematical gardeners have come to appreciate its usefulness in holding down the topsoil and preventing dust storms; indeed, some half dozen books have appeared within the past dozen years putting it to this use. It is a beautiful plant too, whose rapid proliferation has produced many unique and exotic variants; but, perhaps because of its increasingly multiform variety, the book extolling all its loveliness has not yet been written.''}
\begin{flushright} --- F.E.J. Linton~\cite{linton1965} \end{flushright}
\end{quote}

Categories emerged out of the study of functors, which were originally conceived as a principled way of assigning algebraic invariants to topological spaces. Thus, category theory is part and parcel of the study of algebraic topology. However, from its conception in Samuel Eilenberg and Saunders Mac Lane's 1945 paper on a ``General Theory of Natural Equivalence''~\cite{em-cat45}, it was realized that the language of categories provides a way of identifying formal similarities throughout mathematics. The success of this perspective is largely due to the fact that category theory --- as opposed to set theory --- emphasizes understanding the relationships between objects rather than the objects themselves.

In this section, we provide a brief review of the parts of category theory needed to understand the abstract definitions of a sheaf and cosheaf in Chapter~\ref{sec:abstract_sheaves}. Most importantly, the reader should be able to do the following before moving onto that section:
\begin{itemize}
	\item Think of the set of open sets of a topological space $X$ as a category.
	\item Understand how to summarize the behavior of various functors via limits and colimits.
\end{itemize}

We have tried to provide a self-contained introduction to category theory, but the reader is urged to consult Mac Lane's ``Categories for the Working Mathematician''~\cite{catwork} for a book that very well may be the book anticipated by the quote above.

\section{Categories}

One should visualize categories as graphs with objects corresponding to vertices and maps as edges between vertices, subject to relations that specify when following one sequence of edges is equivalent to another sequence. One can think of some of the axioms of a category as gluing in triangles and tetrahedra to witness these relations.
\[
	\xymatrix{ & & \\ \bullet \ar[rr] & & \bullet }  \qquad
	\xymatrix{ & \bullet \ar[dr] & \\ \bullet \ar[ru] \ar[rr] & & \bullet } \qquad
	\xymatrix{ & \bullet \ar[dr] \ar[r] & \bullet \\ \bullet \ar[ru] \ar@{.>}[rru] \ar[rr] & & \bullet \ar[u] }
\]

\begin{defn}[Category]\index{category}
A \textbf{category} $\cat$ consists of a class of objects denoted $\obj(\cat)$ and a set of morphisms $\Hom_{\cat}(a,b)$ between any two objects $a,b\in\obj(\cat)$. An individual morphism $f:a\to b$ is also called an arrow since it points (maps) from $a$ to $b$. We require that the following axioms hold:
\begin{itemize}
	\item Two morphisms $f\in \Hom_{\cat}(a,b)$ and $g\in\Hom_{\cat}(b,c)$ can be composed to get another morphism $g\circ f\in \Hom_{\cat}(a,c)$.
	\item Composition is associative, i.e. if $h\in\Hom(c,d)$, then $(h\circ g)\circ f=h \circ (g\circ f)$.
	\item For each object $x$ there is an identity morphism $\id_x\in\Hom_{\cat}(x,x)$ that satisfies $f\circ \id_a=f$ and $\id_b\circ f=f$.
\end{itemize}
When the category $\cat$ is understood, we will sometimes write $\Hom(a,b)$ to mean $\Hom_{\cat}(a,b)$.
\end{defn}

One can usually ignore the technicality that the collection of objects forms a class rather than a set. A class is a collection of sets that one can refuse to quantify over in a logical sense. This prohibits Russell-type paradoxes gotten by considering the category of all categories that do not contain themselves. Colloquially, one says a proper class is ``bigger'' than a set. In order to avoid certain machinery that accompanies the use of classes, we will often consider categories that are ``small'' in a precise sense.\footnote{The machinery we are referring to is that of Grothendieck universes.}

\begin{defn}[Small and Finite Categories]\index{category!small}\index{category!finite}
	A category is \textbf{small} if its class of objects is actually a set. A category is \textbf{finite} if its set of objects has finite cardinality.
\end{defn}

\begin{ex}[Discrete Category]\index{category!discrete}
	Any set $X$ can be regarded as a \textbf{discrete category} $\bar{X}$ with only the identity morphism $\id_x$ sitting over each object. There are no non-identity morphisms.
\end{ex}

Recall that a \textbf{relation}\index{relation} $R$ on a set $X$ is a subset of the product set $X\times X$. If two elements are related by $R$, one writes $x R y$ to mean that $(x,y)\in R$. We now give an example of some relations on a set that endow that set with the structure of a category.

\begin{ex}[Posets and Preorders]
	A \textbf{preordered set}\index{preordered set} is a set $X$ along with a relation $\leq$ that satisfies the following two axioms:
\begin{description}
 \item[Reflexivity] --- $x\leq x$ for all $x\in X$
 \item[Transitivity] --- $x\leq y$ and $y\leq z$ implies $x\leq z$
\end{description}
	A \textbf{partially ordered set}, or \textbf{poset}\index{poset} for short, is a preordered set that additionally satisfies the following third axiom:
	\begin{description}
		\item[Anti-Symmetry] --- $x\leq y$ and $y\leq x$ implies $x=y$
	\end{description}
Any preordered set $(X,\leq)$ defines a category by letting the objects be the elements of $X$ and by declaring each Hom set $\Hom(x,y)$ to either have a unique morphism if $x\leq y$ or to be empty if $x\nleq y$. 
\end{ex}

We now reach our example of fundamental importance.

\begin{ex}[Open Set Category]\index{category!of open sets}
 The \textbf{open set category} associated to a topological space $X$, denoted $\Open(X)$, has as objects the open sets of $X$ and a unique morphism $U\to V$ for each pair related by inclusion $U\subseteq V$.
\end{ex}

 There is an example very closely aligned with the category of open sets that is allegedly due to Raoul Bott, who gave it as an example of a topological category~\cite{bott-mexico,bott-legacy}.

\begin{ex}[Pointed Open Set Category]\index{category!of points and open sets}
The \textbf{pointed open set category} $\Open^*(X)$ associated to a topological space $X$ has pairs $(U,x)$, where $U$ is an open set and $x$ is a point in $U$, for objects and a unique morphism $(U,x)\to (V,y)$ if $U\subset V$ and $x=y$.
\end{ex}

This pointed open set category takes us nicely over to a category whose objects are points of a topological space. First, we introduce some terminology.

\begin{defn}[Groupoid]\label{defn:groupoid}\index{category!groupoid}\index{groupoid}
A \textbf{groupoid} is a category where every morphisms is invertible. In other words if $\bsG$ is a groupoid, then for every pair of objects $x,y\in\obj(\bsG)$ and every morphism $\alpha\in\Hom_{\bsG}(x,y)$ there exists a morphism $\beta\in \Hom_{\bsG}(y,x)$ such that $\alpha\circ\beta=\id_y$ and $\beta\circ\alpha=\id_x$.
\end{defn}

\begin{exr}
Let $\bsG$ be a groupoid with only one object. Show that the structure axioms of a category along with the property of being a groupoid guarantees that $\bG$ is a group. Observe that this gives us a way of treating every group as a category, where multiplication in the group corresponds to composition of morphisms.
\end{exr}

\begin{defn}[The Fundamental Groupoid]\label{defn:fundamental_groupoid}\index{category!fundamental groupoid $\pijuan(X)$}\index{fundamental groupoid $\pijuan(X)$}
Let $X$ be a topological space. The \textbf{fundamental groupoid} $\pijuan(X)$ has points $x\in X$ for objects and homotopy classes of paths relative endpoints for morphisms. Specifically,
\[
	\Hom_{\pijuan(X)}(x,y):=\{\gamma:[0,1]\to X\,|\,\gamma(0)=x, \gamma(1)=y\}/\sim
\]
where $\gamma\sim\gamma'$ if there exists a third continuous map $h:[0,1]^2\to X$ such that $h(0,t)=\gamma(t)$, $h(1,t)=\gamma'(t)$, $h(s,0)=x$ and $h(s,1)=y$.
\end{defn} 

\begin{rmk}[Poincar\'e $\infty$-Groupoid]\index{fundamental groupoid $\pijuan(X)$!Poincar\'e $\infty$-groupoid}
	To a topological space $X$, one can consider a generalization of the fundamental groupoid, called the \textbf{Poincar\'e $\infty$-groupoid} $\piinfty(X)$, which has an object for each point of $X$, a morphism for every path $\gamma:[0,1]\to X$ , a ``2-morphism'' for every continuous map $\sigma:\Delta^2\to X$, and so on for higher $\Delta^n$. The 2-morphisms should be regarded as providing a homotopy between $\sigma|_{0,2}$ and $\sigma|_{1,2}\circ\sigma|_{0,1}$, i.e. a morphism between morphisms. Here $\sigma|_{i,j}$ is the restriction of the map $\sigma$ to the edge going from vertex $i$ to $j$. As stated, this is an example of an $\infty$-category, which is currently vying to replace ordinary category theory as the foundation for mathematics~\cite{lurie-htt}.  
\end{rmk}

The above examples of categories are quite small when compared to the categories that Eilenberg and Mac Lane first introduced. The categories considered there correspond to \textbf{data types}\index{category!data}\index{data types} and we will usually refer to them with the letter $\dat$. For this paper $\dat$ will usually mean one of the following: 
\begin{description}
	\item[$\Set$] --- the category whose objects are sets and whose morphisms are all set maps (multi-valued maps are prohibited as are partially defined maps)
	\item[$\Ab$] --- the category whose objects are abelian groups and whose morphisms are group homomorphisms
	\item[$\Vect$] --- the category whose objects are vector spaces and whose morphisms are linear transformations
	\item[$\vect$] --- the category whose objects are \emph{finite-dimensional} vector spaces and linear transformations
	\item[$\Top$] --- the category whose objects are topological spaces and whose morphisms are continuous maps
\end{description}
The category $\vect$ is an example of a subcategory, which we now define.
\begin{defn}[Subcategories]\index{category!subcategory}
	Let $\cat$ be a category. A \textbf{subcategory} $\bat$ of $\cat$ consists of a subcollection of objects from $\cat$ and a choice of subset of the morphism set $\Hom_{\cat}(x,y)$ for each pair $x,y\in\obj(\bat)$. We require that these morphism sets have the identity and be closed under composition so as to guarantee that $\bat$ is a category. We say that a subcategory is \textbf{full} if $\Hom_{\bat}(x,y)=\Hom_{\cat}(x,y)$. 
\end{defn}

Categories have a built-in notion of directionality. For example, in $\Set$ every object $X$ has a unique map from the empty set $\emptyset$, but there are no maps to the empty set. We can abstract out this property, so as to make it apply in other situations.

\begin{defn}[Initial and Terminal Objects]\index{initial object}\index{terminal object}
	An object $x\in\obj(\cat)$ is said to be \textbf{initial} if for any other object $y\in\obj(\cat)$ there is a unique morphism from $x$ to $y$. Dually, an object $y$ is said to be \textbf{terminal} if for any object $x$ there is a unique morphism from $x$ to $y$. 
\end{defn}

As already mentioned, in $\Set$ the empty set is initial, but it is not terminal. On the contrary, the terminal object is the one point set $\{\star\}$ since there is only one constant map. Similarly, for $\Open(X)$ the empty set is initial, but the whole space $X$ is terminal. In $\Vect$ the initial and terminal objects coincide with the zero vector space. In some sense, the difference between the initial and terminal objects in a category measure how different it is from its reflection. We now say what we mean by a category's reflection. 

\begin{ex}[Opposite Category]\index{category!opposite}
 For any category $\cat$ there is an \textbf{opposite category} $\cat^{op}$ where all the arrows have been turned around, i.e. $\Hom_{\cat^{op}}(x,y)=\Hom_{\cat}(y,x)$.
\end{ex}
\begin{rmk}[Duality and Terminology]
Because one can always perform a general categorical construction in $\cat$ or $\cat^{op}$ every concept is really two concepts. As we shall see, this causes a proliferation of ideas and is sometimes referred to as the \textbf{mirror principle}. The way this affects terminology is that a construction that is dualized is named by placing a ``co'' in front of the name of the un-dualized construction. Thus, as we will see shortly, there are limits and colimits, products and coproducts, equalizers and coequalizers, among other things.
\end{rmk}

Now we introduce the fundamental device that assigns objects and morphisms in one category to objects and morphisms in another category. Historically, this device was introduced first and categories were summoned into existence to provide a domain and range for this assignment. 

\begin{defn}[Functor]\index{functor}
	A \textbf{functor} $F:\cat\to\dat$ consists of the following data: To each object $a\in\cat$ an object $F(a)\in\dat$ is associated, i.e. $a\rightsquigarrow F(a)$. To each morphism $f:a\to b$ a morphism $F(f):F(a)\to F(b)$ is likewise associated. We require that the functor respect composition and preserve identity morphisms, i.e. $F(f\circ g)=F(f)\circ F(g)$ and $F(\id_a)=\id_{F(a)}$. For such a functor $F$, we say $\cat$ is the \textbf{domain} and $\dat$ is the \textbf{codomain} of $F$.
\end{defn}
\begin{rmk}
We can phrase the definition of a functor differently by saying that we have a \emph{function} $F:\obj(\cat)\to\obj(\dat)$ and \emph{functions} $F(a,b):\Hom_{\cat}(a,b)\to\Hom_{\dat}(F(a),F(b))$ for every pair of objects $a,b\in\obj(\cat)$. We require that these functions preserve identities and composition. When $F(a,b):\Hom_{\cat}(a,b)\to\Hom_{\dat}(F(a),F(b))$ is injective for every pair of objects we say $F$ is \textbf{faithful}. When $F(a,b)$ is surjective for every pair of objects we say $F$ is \textbf{full}. When a functor is both full and faithful, we say it is \textbf{fully faithful}.
\end{rmk}

\begin{exr}
Check that the definition of a subcategory guarantees that the inclusion $\bat\hookrightarrow \cat$ is a functor.
\end{exr}

An example familiar to every topologist is that of homology and cohomology with field coefficients. In every non-negative degree $i$, these invariants define functors
\[
	H_i(-;k):\Top\to\Vect \qquad \mathrm{and} \qquad H^i(-;k):\Top^{op}\to\Vect
\]
respectively. Here we have used the opposite category as an alternative way of saying cohomology is \textbf{contravariant}.
     
Historically, there was a plethora of different homology theories --- simplicial, singular, \v{C}ech, Vietoris, Alexander, et al --- and every time one was introduced a long repetition of the basic properties of that homology theory ensued. Understanding the precise relationships between these motivated the notion of a map between functors, which led in turn to the Eilenberg-Steenrod axioms~\cite[p.335]{maclane-concepts}.

\begin{defn}[Natural Transformation]\index{natural transformation}
	Given two functors $F,G:\cat\to\dat$ a \textbf{natural transformation}, sometimes written $\eta:F \Rightarrow G$, consists of the following information: to each object $a\in\cat$, a morphism $\eta(a):F(a)\to G(a)$ is assigned such that for every morphism $f:a\to b$ in $\cat$ the following diagram \textbf{commutes}:
	\begin{displaymath}
		\xymatrix{F(a) \ar[r]^{\eta(a)} \ar[d]_{F(f)} & G(a)\ar[d]^{G(f)}\\F(b) \ar[r]^{\eta(b)} & G(b)}
	\end{displaymath}
	By commutes, we mean $G(f)\circ \eta(a)=\eta(b)\circ F(f)$.
\end{defn}

\begin{defn}\index{naturally isomorphic}
	Two functors $F,G:\cat\to\dat$ are said to be \textbf{naturally isomorphic} if there is a natural transformation $\eta:F\Rightarrow G$ such that for every object $a\in \cat$ the morphism $\eta(a)$ is an isomorphism, i.e. it is invertible. These inverse maps $\eta(a)^{-1}$ define an inverse natural transformation $\eta^{-1}:G\Rightarrow F$.
\end{defn}

Functors and natural transformations assemble themselves into a category in their own right. Since an arrow is an arrow by any other symbol, we will sometimes use the notation $F\to G$ to denote a natural transformation, instead of $F\Rightarrow G$. In the functor category, we will see that naturally isomorphic functors are isomorphic objects. This demonstrates again the linguistic efficiency of category theory.

\begin{ex}[Functor Category]\index{category!of functors}
 $\Fun(\cat,\dat)$ denotes the category whose objects are functors from $C$ to $D$ and whose morphisms are natural transformations.
\end{ex}

Certain functors deserve special attention. These are the ones that allow us to identify two different categories. One approach to identifying categories is to say that two categories $\cat$ and $\dat$ are \textbf{isomorphic} if there are functors $F:\cat\to\dat$ and $G:\dat\to\cat$ such that $G\circ F=\id_{\cat}$ and $F\circ G=\id_{\dat}$. This definition is so restrictive that it rarely occurs. Thus, we have a looser notion that includes isomorphism as a special case. Instead of asking that $F\circ G$ be equal to $\id_{\dat}$, we only require that they be isomorphic as objects in $\Fun(\dat,\dat)$ and similarly for $G\circ F$ and $\id_{\cat}$ in $\Fun(\cat,\cat)$. The reader should compare this with the notion of homotopy equivalence. 

\begin{defn}\index{equivalence!adjoint}
A pair of functors $F:\cat\to\dat$ and $G:\dat\to\cat$ together define an \textbf{adjoint equivalence} of categories if there are two natural isomorphisms of functors $\epsilon:F\circ G\to \id_{\dat}$ and $\eta:\id_{\cat}\to G\circ F$.
\end{defn}
We will see that this notion of a equivalence is a special instance of an adjunction, which is taken up in Section~\ref{sec:adjunctions}

Equivalence can also be phrased in a way that doesn't require us to construct $G$ as a ``weak inverse'' of $F$.
\begin{defn}[Fully Faithful and Essentially Surjective]\index{equivalence!fully faithful and essentially surjective}
	A functor $F:\cat\to\dat$ induces an \textbf{equivalence} of categories if it is bijective on $\Hom$ sets (fully faithful) and is \textbf{essentially surjective}. This last property means that for every object $d\in\dat$ there is an object $c\in\cat$ such that $F(c)$ is isomorphic to $d$, i.e. $F$ is bijective on isomorphism classes of $\cat$ and $\dat$.
\end{defn}

The notion of equivalence allows us to find compressed presentations of a category.

\begin{defn}[Skeletal Subcategory]\index{category!skeletal subcategory}\index{skeletal subcategory}
	Suppose $\cat$ is a category, then a subcategory $\covS$ is \textbf{skeletal} if the inclusion functor is an equivalence, and no two objects of $\covS$ are isomorphic.
	
	If $\cat$ is small, then we can describe explicitly how to construct a skeletal subcategory $\covS$. On the objects of $\cat$ we define an equivalence relation that says $x\sim x'$ if and only if $x$ and $x'$ are isomorphic. To define a skeletal subcategory we pick one object $x\in\bar{x}$ from each equivalence class and define the morphisms to be $\Hom_{\covS}(\bar{x},\bar{y}):=\Hom_{\cat}(x,y)$.
\end{defn}

\begin{exr}[Fundamental groupoid]
Suppose $X$ is a path connected space. Show that for any point $x_0\in X$, the fundamental group $\pijuan(X,x_0)$ is a skeletal subcategory of $\pijuan(X)$.
\end{exr}

Finally, let's analyze how working in the opposite category impacts functors and natural transformations. Observe, first and foremost, that formality allows us to take a functor $F:\cat\to\dat$ and define a functor $F^{op}:\cat^{op}\to\dat^{op}$. Moreover, a natural transformation $\eta:F\Rightarrow G$ translates to a natural transformation $\eta^{op}:G^{op}\Rightarrow F^{op}$. This observation allows us to state the equalities
\[
	\Fun(\cat^{op},\dat^{op})=\Fun(\cat,\dat)^{op} \qquad \mathrm{or} \qquad \Fun(\cat^{op},\dat^{op})^{op}=\Fun(\cat,\dat)
\]
since $(\cat^{op})^{op}$ is isomorphic to $\cat$ (not just equivalent). See the wonderful work ``Abstract and Concrete Categories: The Joy of Cats''~\cite{joyofcats} for more on duality and category theory more generally.

\section{Diagrams and Representations}
\label{subsec:reps}
Categories and functors allow us to develop an \emph{algebra of shape}, the shapes being modeled on the domain category of a functor. For example, we will be interested in studying data arranged in the following forms:
\[
	\xymatrix{\bullet \ar[r] \ar[d] & \bullet \\ \bullet &} \qquad \qquad
	\xymatrix{ \bullet \ar@<-.5ex>[rd] \ar@<.5ex>[rd] & \bullet \\ \bullet & \bullet} \qquad \qquad
	\xymatrix{& \bullet\ar[d] \\ \bullet \ar[r] &\bullet}
\]
If we imagine the identity arrows in a category as being the vertices themselves, and thus not drawn independently of the objects, each of these shapes gives an example of a finite category.

\begin{defn}[Diagram]\index{diagram}
	Suppose $\Iat$ is a small category and $\cat$ is an arbitrary category. A \textbf{diagram} is simply a functor $F:\Iat\to \cat$.
\end{defn}

\begin{ex}[Constant Diagram]
	For any category $\Iat$ there is always a diagram for each object $O\in \cat$, called the \textbf{constant diagram}, $\mathrm{const}_O:\Iat\to\cat$ where $\mathrm{const}_O(x)=\mathrm{const}_O(y)=O$ for all objects $x,y\in \Iat$. Every morphism in $\Iat$ goes to the identity morphism.
\end{ex}

\begin{defn}[Representation]\index{representation}
	A \textbf{representation} of a category $\cat$ is a functor $F:\cat\to\Vect$. 
\end{defn}

One should note that this definition generalizes the notion of a representation of a group. Every group, say $\ZZ$ for example, can be considered as a small category with a single object $\star$ and $\Hom(\star,\star)=\ZZ$. A representation of $\ZZ$ then corresponds to picking a vector space $V$ and assigning an endomorphism of $V$ for each element of $\ZZ$, i.e. it is a functor.
\[
	\xymatrix{\star \ar@{~>}[r] \ar[d]_g & V \ar[d]^{\rho(g)} \\
	\star \ar@{~>}[r] & V}
\]
Maps of representations correspond precisely with natural transformations of such functors. Isomorphic representations are naturally isomorphic functors.\footnote{Confusingly, the term ``equivalent representations'' is often used.} These basic notions carry over to the representation theory of arbitrary categories, which allows us to compare different situations in one language.

\section{Cones and Limits}
\label{subsec:limits}
The next two sections are devoted to studying one way (and a dual way) of summarizing a functor's behavior. This gives a way of compressing the data of a functor into a single object. These concepts are fundamental to the study of sheaves and cosheaves.

\begin{defn}[Cone]\index{cones}
	Suppose $F:\Iat\to\cat$ is a diagram. A \textbf{cone} on $F$ is a natural transformation from a constant diagram to $F$. Specifically, it is a choice of object $L\in\cat$ and a collection of morphisms $\psi_{x}:L\to F(x)$, one for each $x$, such that if $g:x\to y$ is a morphism in $\Iat$, then $F(g)\circ 	\psi_x=\psi_y$, i.e. the following diagram commutes:
\[
\xymatrix{F(x) \ar[rr]^{F(g)} & & F(y)\\ & L \ar[ul]^{\psi_x} \ar[ru]_{\psi_y} &}
\]
In other words, $\psi_y=F(g)\circ \psi_x$.
\end{defn}

\begin{defn}\index{cones!category of}\index{category!of cones}
	The collection of cones on a diagram $F$ form a category, which we will call $\Cone(F)$. The objects are cones $(L,\psi_x)$ and a morphism between two cones $(L',\psi_x')$ and $(L,\psi_x)$ consists of a map $u:L'\to L$ such that $\psi_x'=\psi_x\circ u$ for all $x$
\end{defn}

A limit is simply a distinguished or universal object in the category of cones on $F$.

\begin{defn}[Limit]\index{limit}
	The \textbf{limit} of a diagram $F:\Iat\to\cat$, denoted $\varprojlim F$ is the terminal object in $\Cone(F)$. This means that a limit is an object $\varprojlim F\in\cat$ along with a collection of morphisms $\psi_x:L\to F(x)$ that commute with arrows in the diagram such that whenever there is another object $L'$ and morphisms $\psi'_x$ that also commute there then exists is a unique morphism $u:L'\to \varprojlim F$ that additionally commutes with everything in sight, i.e. $\psi_x'=\psi_x\circ u$ for all $x$.
	\[
	\xymatrix{F(x) \ar[rr]^{F(g)} & & F(y)\\ & \varprojlim F \ar[ul]_{\psi_x} \ar[ru]^{\psi_y} & \\ & L' \ar[uul]^{\psi'_x} \ar@{.>}[u]^{\exists!}_u  \ar[uur]_{\psi'_y}  &}
	\]
\end{defn}


\begin{rmk}[Glossary]\index{limit!synonyms for}
	Quite confusingly, the following terms are synonyms for limits: inverse limits, projective limits, left roots, $\lim$ and $\varprojlim$ are all common.
\end{rmk}

We now consider some examples of limits over discrete categories.

\begin{ex}[Products]\label{ex:products}\index{product}
	Consider the following index category and diagram:
	\[
		\xymatrix{\bullet & \bullet} \qquad \qquad \qquad \xymatrix{F(i) & F(j)}
	\]
	The limit of this diagram is called the \textbf{product} and is usually written
	\[
		F(i) \prod F(j).
	\]
	More generally, we define the product to be the limit of any diagram $F:\Iat\to\cat$ indexed by a discrete category and write $\prod_i F(i)$. Sometimes one writes $\times_i F(i)$ for the product.
\end{ex}

We give an unusual example of a product that will prepare the reader for thinking about the category of open sets.

\begin{ex}[Open Sets: Limits are Intersections]\index{limit!is an intersection}
	Suppose $\Lambdat=\{1,\ldots, n\}$ is a finite discrete category, i.e. it has $n$ objects and the only morphisms are the identity morphisms. Now let $X$ be a topological space and let $\cat=\Open(X)$ be the category of open sets in $X$. This is a category that has an object for each open set and a single morphism $U\to V$ if $U\subset V$. A functor $F:\Lambdat\to\Open(X)$ is nothing more than a choice of $n$ not necessarily distinct open sets. A cone to $F$ is an open set that includes into all the open sets picked out by $F$. The limit of $F$ is the largest possible open set that includes into all the open sets picked out by $F$, i.e. $$\varprojlim F=\cap_{i=1}^n F(i).$$
\end{ex}

\begin{ex}
	Consider the following small category $\Iat$ along with some representation $F:\Iat\to\Vect$.
	\[
	\xymatrix{\bullet \ar[r] \ar[d] & \bullet \\ \bullet &} \qquad \qquad \qquad
	\xymatrix{U \ar[r]^A \ar[d]_B & V \\ W &}
	\]
	By thinking about the definition, one can see that
	\[
		\varprojlim F\cong U.
	\]
\end{ex}

\begin{ex}[Pullbacks]\label{ex:pullback}\index{pullback}	
	Consider the category $\Jat=\Iat^{op}$ and a representation $F:\Jat\to\Vect$.
	\[
	\xymatrix{& \bullet\ar[d] \\ \bullet \ar[r] &\bullet}
	\qquad \qquad \qquad
	\xymatrix{ & V\ar[d]^A \\ W \ar[r]_B & U}
	\]
	With some thought one can describe the limit set-theoretically as
	\[
		\varprojlim F\cong \{(v,w)\in V\times W | Av=Bw\},
	\]
	which is called the \textbf{pullback}. If $U=0$, then we re-obtain the product of $V$ and $W$ and one usually writes $V\times W$. 
\end{ex}

\begin{ex}[Equalizers and Kernels]\label{ex:equalizer}\index{equalizer}\index{kernel}
	Consider the following category $\Kat$ and an arbitrary functor $F:\Kat\to\dat$.
	\[
	 \xymatrix{ \bullet \ar@<-.5ex>[r] \ar@<.5ex>[r] & \bullet}
	\qquad \qquad \qquad
	 \xymatrix{ X \ar@<-.5ex>[r]_{g} \ar@<.5ex>[r]^{f} & Y}
	\]
	The limit of this diagram, which is also called the \textbf{equalizer}, is an object $E$ along with a map $h$ that satisfies $f\circ h=g\circ h$.
	\[
	 \xymatrix{E \ar[r]^h & X \ar@<-.5ex>[r]_g \ar@<.5ex>[r]^f & Y}
	\]
	If $\dat=\Vect$ and one sets $g=0$, then the equalizer is the \textbf{kernel}. Thus, if one wants to mimic kernels in data types lacking of zero maps and objects, equalizers can be substituted.
\end{ex}

Finally, we finish with an example from representation theory.

\begin{ex}[Invariants]\label{ex:invariants}\index{invariants}
	Suppose that $V$ is a vector space with an endomorphism $T:V\to V$, i.e. a $k[x]$-module. Just as a group can be viewed as a category with one object, a ring can be viewed as a category with multiplication corresponding to composition of morphisms and addition corresponding to addition of morphisms, thus such a category has extra structure. Thus the $k[x]$-module determined by $V$ and $T$ is equivalent to a functor $k[x]\to\Vect$ that sends the unique object $\star$ to $V$ and sends $x$ to $T$. The limit of such a functor is called the \textbf{invariants} of the action, i.e.
	\[
		I=\{v\in V \, | \, T(v)=v\}.
	\]
\end{ex}

\section{Co-Cones and Colimits}
\label{subsec:colimits}
Here we invoke the mirror principle to dualize the theory of cones and limits. In accordance with usual terminology, we refer to these as \emph{co}cones and \emph{co}limits.

\begin{defn}[Co-Cone]\index{cocones}
	Given a diagram $F:\Iat\to \cat$, a \textbf{cocone} is a natural transformation from $F$ to a constant diagram. In other words, it consists of an object $C\in\cat$ along with a collection of maps $\phi_x:F(x)\to C$ such that these maps commute with the ones internal to the diagram.
	\[
		\xymatrix{ & C & \\ F(x) \ar[rr]_{F(g)} \ar[ru]^{\phi_x} & & F(y) \ar[ul]_{\phi_y} }
	\]
\end{defn}

Similarly, there is a category of cocones\index{cocones!category of}\index{category!of cocones} to a diagram $F$, denoted $\CoCone(F)$. A colimit is a distinguished object in this category.

\begin{defn}[Colimit]\index{colimit}
	The \textbf{colimit} of a diagram $F$ is the initial object in the category $\CoCone(F)$. One should practice dualizing the explicit description of the limit in order to understand the following diagram: 
	\[
		\xymatrix{
		& C' & \\
		& \varinjlim F \ar@{.>}[u]^{\exists!}_u & \\
		F(x) \ar[ur]_{\phi_x} \ar[uur]^{\phi'_x} \ar[rr]_{F(g)} & & F(y) \ar[ul]^{\phi_y} \ar[uul]_{\phi'_y}
		}
	\]
\end{defn}

\begin{rmk}[Glossary]\index{colimit!synonyms for}
	The following terms are synonyms for colimits: direct limits, inductive/injective limits, right roots, $\colim$ and $\varinjlim$ are all used.
\end{rmk}

To better understand the similarities and differences between limits and colimits, let us re-examine the same examples in the previous section.

\begin{ex}[Coproducts]\label{ex:coproducts}\index{coproduct}
	Consider the following index category and diagram:
	\[
		\xymatrix{\bullet & \bullet} \qquad \qquad \qquad \xymatrix{F(i) & F(j)}
	\]
	The colimit of this diagram is called the \textbf{coproduct} and is usually written
	\[
		F(i) \coprod F(j).
	\]
	More generally, we define the product to be the limit of any diagram $F:\Iat\to\cat$ indexed by a discrete category and write $\coprod_i F(i)$. Alternative notations for the coproduct, depending usually on whether the target category is $\Set,\Vect$, $\Ab$ or $\Top$ include
	\[
		\bigoplus_i F(i) \qquad \mathrm{and} \qquad \sum_i F(i) \qquad \mathrm{and} \qquad \bigsqcup_i F(i).
	\]
\end{ex}

\begin{ex}[Open Sets: Colimits are Unions]\index{colimit!is a union}
	Suppose $\Lambdat=\{1,\ldots, n\}$ is a finite discrete category. Let $\cat=\Open(X)$ be the category of open sets in $X$. A functor $F:\Lambdat\to\Open(X)$ is a choice of $n$ not necessarily distinct open sets. A cocone to $F$ is an open set that contains all the open sets picked out by $F$. The colimit of $F$ is the smallest possible open set containing all the open sets picked out by $F$, i.e. the union: $$\varinjlim F=\cup_{i=1}^n F(i)$$
	One should note that since the arbitrary union of open sets is still open one could have worked over a larger indexing category $\Lambdat$.
\end{ex}

\begin{ex}[Pushouts]\label{ex:pushout}\index{pushout}
	Consider the following small category $\Iat$ and a representation $F:\Iat\to\Vect$.
	\[
	\xymatrix{\bullet \ar[r] \ar[d] & \bullet \\ \bullet &}
	\qquad \qquad \qquad
	\xymatrix{U \ar[r]^A \ar[d]_B & V \\ W &}
	\]
	Contrary to the case of the limit, this one requires a bit more thought. Let's start with something that is \emph{not} a cocone, but is nevertheless naturally built out of pieces of the diagram.
	\[
	\xymatrix{U \ar[r]^A \ar[d]_B \ar[rd]^{B\oplus A} & V \ar[d]^{\iota_V} \\ W \ar[r]_{\iota_W} & W\oplus V}
	\]
	This is not a cocone because the diagram does not commute since $(Bu,0)\neq (Bu,Au)\neq (0,Au)$. We can force commutativity by forcing the equivalence relation $[(Bu,0)]\sim [(0,Au)]$ or equivalently $[(Bu,-Au)]\sim [(0,0)]$. We thus conclude that
	\[
	\varinjlim F=W\oplus V/\im(B\oplus -A) \quad \phi_U=q\circ \iota_WB=q\circ\iota_VA \quad \phi_W=q\circ \iota_W \quad \phi_V=q\circ \iota_W
	\]
	where $q$ is the quotient map. One should note that this is clearly dual to the limit computation in \ref{ex:pullback} with the added complication that whereas the limit is a sub-object, the colimit is a quotient object.
	
	Like before, if $U=0$ then the pushout reduces to the coproduct of $V$ and $W$ and one writes it as $V\oplus W$.
\end{ex}

\begin{ex}
	Consider the example $\Jat=\Iat^{op}$ and corresponding representation $F:\Jat\to\Vect$.
	\[
	\xymatrix{& \bullet\ar[d] \\ \bullet \ar[r] &\bullet}
	\qquad \qquad \qquad
	\xymatrix{ & V\ar[d]^A \\ W \ar[r]_B & U}
	\]
	One can see that
	\[
		\varinjlim F\cong U.
	\]
\end{ex}

\begin{ex}[Coequalizers and Cokernels]\label{ex:coequalizer}\index{cokernel}\index{coequalizer}
	Consider the same category $\Kat$ as before and a functor $F:\Kat\to\dat$.
	\[
	 \xymatrix{ \bullet \ar@<-.5ex>[r] \ar@<.5ex>[r] & \bullet}
	\qquad \qquad \qquad
	 \xymatrix{ X \ar@<-.5ex>[r]_g \ar@<.5ex>[r]^f & Y}
	\]
	The colimit, which is called the \textbf{coequalizer}, is an object $E$ and map $h$ such that $h \circ f=h\circ g$.
	\[
	 \xymatrix{ X \ar@<-.5ex>[r]_g \ar@<.5ex>[r]^f & Y \ar[r]^h & E}
	\]
	If $\dat=\Vect$ and one sets $g=0$, then the coequalizer is the \textbf{cokernel}. Thus if one wants to mimic cokernels in data types lacking of zero maps and objects, coequalizers can be substituted.
\end{ex}

\begin{ex}[Co-invariants]\index{coinvariants}
	As described in Example~\ref{ex:invariants}, a vector space $V$ with an endomorphism $T$ is equivalent to a functor $k[x]\to\Vect$. The colimit of this functor is called the \textbf{coinvariants} of $T$, i.e.
	\[
		C=V/<Tv-v>.
	\]
\end{ex}

\section{Adjunctions}
\label{sec:adjunctions}

Adjunctions allow us to derive interesting relationships with almost no effort; they are in essence dualities. For the individual interested in using category theory to model the world, facile manipulations of adjunctions is essential. One often can transform a complicated problem into a simpler one via an adjunction, thereby gaining a computational payoff at the cost of abstraction. This is why using adjunctions between the functors defined in Section~\ref{subsec:map_functors} is one of the key technical skills every sheaf theorist must master. Adjunctions also have played an essential role in the development of sheaf theory. Finding an adjoint to the functor $f_!$ was one of the primary reasons that the notion of a derived category was invented. Only by enlarging the domain could a new, adjoint functor $f^!$ be defined. Here we introduce the general theory.

\begin{defn}\index{adjoint}
 Suppose $F:\cat\to\dat$ and $G:\dat\to\cat$ are functors. We say that $(F,G)$ is an \textbf{adjoint pair} or that \textbf{$F$ is left adjoint to $G$} (or equivalently $G$ is right adjoint to $F$) if we have a natural transformation $\eta:\id_{\cat}\to G\circ F$ and a natural transformation to $\epsilon:F\circ G\to \id_{\dat}$ such that
\[
	\xymatrix{G\ar[r]^{\eta G} & GFG \ar[r]^{G\epsilon} & G},\qquad \xymatrix{F\ar[r]^{F\eta} & FGF \ar[r]^{\epsilon F}& F}
\]
We call $\eta$ the \textbf{unit} of the adjunction and $\epsilon$ the \textbf{counit} of the adjunction.
\end{defn}

There are about a half-dozen different, but equivalent, ways of defining an adjunction; see~\cite[p.81]{catwork} for a list. One can just specify $\eta$ and ask that it is universal,\footnote{In other words, initial in a particular comma category; see~\cite[p.56]{catwork}} i.e. for each $x\in\cat$ and for every $y\in\dat$ there is a map $\eta_x:x\to GF(x)$ such that if we have $f:x\to G(y)$, then there exists a unique map $f':F(x)\to y$ with $G(f')\circ \eta_x=f$. 
\[
\xymatrix{x \ar[r]^{\eta_x} \ar[dr]_{f} & GF(x) \ar@{.>}[d] \\ & G(y)}
\]
Of course we could have just defined $\epsilon$ and asked that it is universal in a dual sense.\footnote{It is final in a different comma category.} The point is this: an adjunction is equivalent to specifying for every $x\in\cat$ and $y\in\dat$ a natural bijection $\varphi_{x,y}$
\[
	\Hom_{\dat}(F(x),y)\cong\Hom_{\cat}(x,G(y)).
\]

The following theorem gives us an abstract criterion for determining when a functor has an adjoint.

\begin{thm}[Freyd's Adjoint Functor Theorem]\label{thm:freyd}\index{adjoint!functor theorem, Freyd's}
	Let $\dat$ be a complete category and $G:\dat\to\cat$ a functor, then $G$ has a left adjoint $F$ if and only if $G$ preserves all limits and satisfies the \textbf{solution set condition}. This condition states that for each object $x\in\cat$ there is a set $\Iat$ and an $\Iat$-indexed family of arrows $f_i:c\to G(a_i)$ such that every arrow $f:x\to G(a)$ can be factored as $x\to G(a_i)\to G(a)$, where the first map is $f_i:x\to G(a_i)$ and the second is $G$ applied to to some $t:a_i\to a$.
\end{thm}

The solution set condition holds nearly all the time, so in practice one only needs to check that $G$ preserves limits, in which case $G$ is a right adjoint (has a left adjoint). Dually, for a functor to be a left adjoint it needs to preserve colimits.