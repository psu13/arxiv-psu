\makefrontmatter

\tocundopartskip

\ctparttext{This part serves multiple groups of people and can be used in different ways:
\begin{itemize}
\item For those who are category theory neophytes, a reading of Chapter~\ref{sec:categories} is advised, after which they should move on to Part~\ref{part:easy_sheaves}, with particular emphasis on the beginning of Chapter~\ref{sec:cell_sheaves} and Chapter~\ref{sec:homology}.

\item Chapter~\ref{sec:abstract_sheaves} is designed for those who want a general definition of sheaves and cosheaves on a topological space. After looking at the definition, one should proceed as quickly as possible to Chapter~\ref{sec:examples} to get some simple examples. 

\item Section~\ref{subsec:cover_structure} is meant for people who have always found the expression of the sheaf axiom as an exact sequence a little opaque. Such people are usually frustrated by the notation used in \v{C}ech homology, which is the subject of Section~\ref{subsec:cech}. 

\item Sections~\ref{subsec:generalities} and~\ref{subsec:cosheafification} are for those who think of cosheaves simply as sheaves valued in the opposite category.
\end{itemize}
}
\part{A Mathematical Introduction}
\label{part:abstract_intro}

\subincludefrom{.}{ch_categories}
\subincludefrom{.}{ch_abstract_sheaves}
\subincludefrom{.}{ch_examples}

\ctparttext{In this part we emphasize that cellular sheaves and cosheaves are nothing more than linear algebra parametrized by a cell complex. The use of the term ``sheaf'' is justified by the Alexandrov topology, which makes functors out of posets into sheaves or cosheaves. Explicit proofs are presented since the primary reference of Shepard~\cite{shepard} is unpublished and not easily accessed. Cellular sheaf cohomology and cosheaf homology are presented computationally in Chapter~\ref{sec:homology} and then put on the firm foundation of derived categories in Chapter~\ref{sec:derived}. The novel contributions from this part, aside from working explicitly with cosheaves, are the introduction of ``barcodes'' to interpret cellular sheaf cohomology and cosheaf homology and exploiting the existence of enough projectives for cellular sheaves to define sheaf \emph{homology}.}
\part{Linear Algebra over Cell Complexes}
\label{part:easy_sheaves}

\subincludefrom{.}{ch_cell_sheaves}
\subincludefrom{.}{ch_maps}
\subincludefrom{.}{ch_homology}
\subincludefrom{.}{ch_derived}

\ctparttext{This part constitutes a first application of cellular sheaves and cosheaves to problems in science and engineering. 

Chapter~\ref{sec:barcodes} begins with a short introduction to persistent homology, which we reformulate using sheaves and cosheaves. The advantage of this reformulation is the ability to distribute homology computations and aggregate efficiently, as noted in Section~\ref{subsec:local-to-global}, which is part of joint work~\cite{DMT_sheaves} where discrete Morse theory is adapted to compute cellular sheaf cohomology. A theorem that uses the apparatus of spectral sequences to connect level set and sub-level set persistence is proved in Section~\ref{subsec:level2sub}. A motivating example for multi-dimensional persistence and an introduction of ``generalized barcodes'' concludes the chapter.

Chapter~\ref{sec:nc_coding} reviews an application of sheaves to network coding introduced first in~\cite{GH-ncs}. However, here the language of cellular sheaves is employed and the barcode method is used to visualize the flow of data. A duality theorem for network coding sheaves is proved in two different ways.

Chapter~\ref{sec:sensors} casts various sensor network problems in the language of sheaves. Any attempt to use level set persistence to study the intruder problem is proven to be a ``no-go'' using the machinery of cosheaves. The main contribution of the chapter is the introduction of a linearized model for multi-modal sensing in Section~\ref{subsec:multi_modal}. It was this model that first motivated the author to take up the theory of indecomposables as a way of interpreting sheaf cohomology.
}

\part{Applications to Science and Engineering}
\label{part:applications}

\subincludefrom{.}{ch_TDA}
\subincludefrom{.}{ch_nc_coding}
\subincludefrom{.}{ch_sensors}

\ctparttext{This part represents the mathematical heart of the thesis, although many of its results were motivated by the applications considered in Part~\ref{part:applications}.

Chapter~\ref{subsec:strat} is by far the most technically demanding part of the thesis. It takes up and proves an equivalence between constructible cosheaves and representations of MacPherson's entrance path category, which hinges on a proof of the Van Kampen theorem for this category. The full machinery of stratification theory is then used to construct representations of the (definable) entrance path category from a stratified (definable) map. This part also rests on proving a codimension-criterion under which Thom's condition $a_f$ always holds.

Chapter~\ref{sec:duality} proves that Verdier duality is rightly conceived as an exchange of sheaves and cosheaves. An explicit formula for the derived equivalence of cellular sheaves and cosheaves is presented.

Chapter~\ref{sec:valuations} uses the formula of Chapter~\ref{sec:duality} to prove that compactly supported cellular sheaf cohomology can be viewed as taking a (derived) coend with the image of the constant sheaf under this formula.

Chapter~\ref{sec:graded} proves that the derived category of cellular sheaves over a one-dimensional base space is equivalent to a graded category. This formalizes the intuition of why spectral sequences over graphs always collapse on the $E_2$ page.

Chapter~\ref{sec:metric} introduces the interleaving distance for sheaves defined on a metric space. Although officially an extended pseudo-metric on the category of pre-sheaves, we prove it is an extended metric on the category of sheaves. One of the most fundamental properties of this extended metric is that global sections places sheaves into distinct connected components. To illustrate the theory more concretely, we take up an explicit description of the space of constructible sheaves over the real line.
}
\part{Novel Mathematical Contributions}
\label{part:math_contributions}

\subincludefrom{.}{ch_strat}
\subincludefrom{.}{ch_duality}
\subincludefrom{.}{ch_valuations}
\subincludefrom{.}{ch_graded}
\subincludefrom{.}{ch_metric}

% \appendixpreamble{blah blah}
% \appendix

\clearpage

\backmatter

\bibliographystyle{alpha}
\bibliography{THESIS}

\printindex
