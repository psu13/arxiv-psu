

%%%%%%%%%%%%%%%%%%%%%%%%%%%%%%%%%%%%%%%%%%%%%%%%%%%%%%%%%%%%%
%%%%%%%%%%%%%%%%%%%%%%%%%%%%%%%%%%%%%%%%%%%%%%%%%%%%%%%%%%%%%
%
%  %%%%%%%  %    %  %%%%%%  %%%%%%  %%%%%%%  %%%%%%%  %%%%%
%  %        %    %  %    %  %     %    %     %        %    %
%  %        %    %  %%%%%%  %     %    %     %        %    %
%  %        %%%%%%  %    %  %%%%%%     %     %%%%%    %%%%%
%  %        %    %  %    %  %          %     %        %    %
%  %%%%%%%  %    %  %    %  %          %     %%%%%%%  %     %
%
%%%%%%%%%%%%%%%%%%%%%%%%%%%%%%%%%%%%%%%%%%%%%%%%%%%%%%%%%%%%%
%%%%%%%%%%%%%%%%%%%%%%%%%%%%%%%%%%%%%%%%%%%%%%%%%%%%%%%%%%%%%

\chapter{Cosheaves as Valuations on Sheaves}
\label{sec:valuations}

\begin{quote}
{\em``Speech is the twin of my vision....it is unequal to measure itself.''}
\begin{flushright} --- Walt Whitman's Song of Myself [25] \end{flushright}
\end{quote}


The development of cosheaves as a theory is largely fragmented. Researchers at different points in time have found a use for it here and there, at the service of different purposes and interests. The more strongly categorical and logical community have done some considerable work understanding the relationship between the topos of sheaves and cosheaves. One insight that seems very worthwhile is that cosheaves act on sheaves in a natural way. Although one can use a little bit of category theory to draw this conclusion, we use this to give some surprising reformulations of classical sheaf theory. Namely, the primary observation of this section is that the action of taking compactly supported cohomology of a sheaf can be interpreted as an action of a very particular cosheaf on the category of all sheaves.

\section{Left and Right Modules and Tensor Products}
\index{module!left and right}
Suppose $R$ is a ring with unit $1_R$. One can think of $R$ as a category with a single object $\star$ whose set of morphisms
\[
	\Hom_R(\star,\star)\cong R
\]
has the structure of an abelian group. The multiplication in the ring plays the role of a composition so $r\cdot s=r\circ s$. The abelian group structure, which corresponds to the ability to add morphisms $r+s$, reflects the fact that rings have an underlying abelian group structure. One says that $R$ is a pre-additive category, or is a category enriched in $\Ab$ --- the category of abelian groups.

An additive functor $B:R\to\Ab$ is a functor that preserves the abelian group structure, so it picks out a single abelian group, which we also call $B$, and satisfies the relation $(r+s)\cdot B=r\cdot B + s\cdot B$ and $(rs)\cdot B=r\cdot (s\cdot B)$ so such a functor is precisely the data of a \textbf{left $R$-module}. Dually, a contravariant functor $A:R^{op}\to \Ab$ prescribes the data of a \textbf{right $R$-module}. Taking the tensor product over $\ZZ$ of $A$ and $B$ allows us to define a bi-module
\[
	A\otimes B :R^{op}\times R \to \Ab \qquad (\star,\star) \mapsto A\otimes_{\ZZ} B.
\]
The latter is the group freely generated by pairs of elements from $A$ and $B$ modulo the usual relations $(a+a')\otimes b=a\otimes b + a'\otimes b$ and $a\otimes (b+b')= a\otimes b + a\otimes b'$. However, in the presence of the action of a ring $R$, there is another tensor product $A\otimes_R B$ that further quotients $A\otimes_{\ZZ} B$ by the relation $(a\cdot r)\otimes b=a\otimes (r\cdot b)$. Said using diagrams, we require that for each $r$, the following diagram commutes.
\[
	\xymatrix{A\otimes B \ar[r]^{1_A\otimes B(r)} \ar[d]_{A(r)\otimes 1_B} & A\otimes B \ar[d] \\
	A\otimes B \ar[r] & A\otimes_R B}
\]

In other words there is a coequalizer
\[
 \xymatrix{ A\otimes_{\ZZ} R \otimes_{\ZZ} B \ar@<-.5ex>[rr]_{(a,r,b)\mapsto (ar,b)} \ar@<.5ex>[rr]^{(a,r,b)\mapsto (a,rb)} & & A\otimes_{\ZZ} B \ar[r] & A\otimes_{R} B}
\]
that realizes the tensor product using purely categorical operations. This allows us to work in a greater degree of generality by making use of a special type of colimit called a \textbf{coend}, that generalizes the tensor product described above.

\begin{defn}[Tensoring Sheaves with Cosheaves]\index{coend}\index{tensoring sheaves with cosheaves}
	Let $X$ be a topological space and let $\hG$ and $F$ be a pre-cosheaf and a pre-sheaf respectively, both valued in $\Vect$. Note that for every pair of objects $U\to V$ in $\Open(X)$ we have a diagram
	\[
		\xymatrix{ & \hG(V)\otimes F(V) \\ \hG(U)\otimes F(V) \ar[ru]^{r^G_{V,U}\otimes \id} \ar[rd]_{\id\otimes\rho_{U,V}^F} & \\ & \hG(U)\otimes F(U)}
	\]
	which is the building block in defining the \textbf{coend} or \textbf{tensor product over $X$}
	\[
	\bigoplus_{U\to V}\hG(U)\otimes F(V) \rightrightarrows \bigoplus_W \hG(W)\otimes F(W) \rightarrow \int^{\Open(X)} \hG(W)\otimes F(W)=:\hG\otimes_X F.
	\]
\end{defn}

We illustrate this definition with an immediate example. 
\begin{ex}[Stalks and Skyscraper Cosheaf]
Recall that we defined the \emph{skyscraper cosheaf} at $x$ to be the cosheaf
\[
	\skycshv_x(U)=\left\{ \begin{array}{ll} k & \textrm{if $x\in U$}\\
	0 & \textrm{other wise}\end{array} \right.
\]
With some thought one can show that the tensor product of any pre-sheaf $F$ with the cosheaf $\skycshv_x$ yields
\[
	\skycshv_x\otimes_X F \cong F_x
\]
by treating $F$ as a variable which can range over all pre-sheaves, one gets, in particular, a functor
\[
	\skycshv_x\otimes_X - :\Shv(X) \to \Vect \qquad F \rightsquigarrow F_x.
\]
\end{ex}

The previous example demonstrates an important observation: 
\emph{The operation of taking stalks is equivalent to the process of tensoring with the skyscraper cosheaf.} 

To see how far this observation can be generalized, note that if we fix $\hG$ and let $F$ vary then we get a functor
\[
\hG\otimes_X -:\Shv(X)\to\Vect
\]
that is defined in terms of colimits and is thus co-continuous (it sends colimits to colimits). Now we are free to take an arbitrary cosheaf and let it act on sheaves. The ``one obvious choice'' of taking stalks at a point is run over by a veritable slew of valuations, one for each cosheaf. Moreover, it is clear that this description extends to a pairing between the symmetric monoidal categories $\Coshv(X)$ and $\Shv(X)$, i.e.
\[
	-\otimes_X-:\Coshv(X)\times \Shv(X)\to\Vect \qquad (\hG,F)\mapsto \hG\otimes_X F:=\int^{\Open(X)}\hG(U)\otimes F(U),
\]
although we haven't used the sheaf or cosheaf axiom anywhere, so the pairing is actually valid for pre-sheaves and pre-cosheaves.

\section{Compactly-Supported Cohomology}

Although the idea of using coends to tensor together sheaves and cosheaves has been independently re-discovered many times, cf. Jean-Pierre Schneider's 1987 work~\cite{schneiders-ch}, it has not been used to do any serious work. This is a shame in light of the following 1985 theorem of A.M. Pitts~\cite{pitts}.
\begin{thm}\index{Pitt's theorem}
	Let $X$ be any topological space. Every colimit-preserving functor on sheaves arises by tensoring with a cosheaf, i.e. $$\Coshv(X;\Set)\cong\Fun^{co-cts}(\Shv(X;\Set),\Set).$$
\end{thm}
This theorem is also stated in Marta Bunge and Jonathan Funk's 2006 book ``Singular Coverings of Toposes''~\cite{bungefunk} as theorem 1.4.3, which further surveys some of Lawvere's philosophy of distributions on topoi. The topos community deserves commendation for keeping the study of cosheaves alive during the past few decades, but so far work in the enriched and computable setting of vector spaces is largely missing.

We attempt to partly remedy this gap by establishing a connection between the tensor operation and the cohomology of sheaves. However, instead of establishing an enriched version of Pitt's theorem,\footnote{We delay this for another paper.} we will use it as a guide. For example, in classical sheaf theory, compactly supported cohomology is gotten by taking the constant map $p:X\to\star$ and associating to it the pushforward with compact supports functor $p_!:\Shv(X)\to\Shv(\star)\cong\Vect$. Of course, just applying $p_!$ defines only compactly supported zeroth cohomology of a sheaf $H_c^0(X;F)$. To get the higher compactly supported cohomology groups one takes an injective resolution and applies $p_!$ to the resolution. The result will be a complex of vector spaces, whose cohomology in turn produces the desired groups:
\[
	F\to I^{\bullet} \rightsquigarrow Rp_!:=p_! I^{\bullet} \qquad H^i(p_! I^{\bullet}):= H^i_c(X;F).
\] 

Historically the first fundamental duality result in sheaf theory was the statement that $Rp_!$ admits a right adjoint on the level of the derived category. This adjunction is sometimes called \textbf{global Verdier duality}:
\[
	\Hom(Rp_!F,G)\cong \Hom(F,p^!G).
\]
By applying the fact that left adjoints are co-continuous one is led to believe, in light of Pitt's theorem, that there should be a cosheaf that realizes the operation of taking derived pushforward with compact supports.

In light of the derived equivalence between cellular sheaves and cosheaves established in this paper, we provide an explicit description of the complex of cosheaves that realizes the derived pushforward.

In preparation, one should note that there are several cosheaves that realize the operation of taking stalks at a point $x$ in the cellular world. One is
\[
  \hat{\delta}_{\sigma}(\tau)=\left\{ \begin{array}{ll} k & \textrm{if $\sigma=\tau$}\\
	0 & \textrm{o.w.}\end{array} \right.
\]
The other is the correct formulation of $\skycshv_{\sigma}$ when using the Alexandrov topology
\[
  [\hat{\sigma}](\tau)=\left\{ \begin{array}{ll} k & \textrm{if $\tau\leq\sigma$}\\
	0 & \textrm{o.w.}\end{array} \right.
\]
Recall that this is also the elementary projective cosheaf concentrated on $\sigma$ with value $k$.

Observe that the first cosheaf returns the value $F(\sigma)$ because every other cell is tensored with zero. The second cosheaf works by restricting the non-zero values of $F$ to the closure of the cell $\sigma$, but this restricted diagram has a terminal object given by $F(\sigma)$, so the colimit returns $F(\sigma)$ as well.

This allows us to state the main theorem of this section.

\begin{thm}\index{cohomology!of a cellular sheaf!compactly-supported}\index{sheaf!cellular!compactly supported cohomology of}
	Let $X$ be a cell complex, then the operation $Rp_!:\Shv(X)\to\Vect$ on cellular sheaves is equivalent to tensoring with the image of the constant sheaf through the derived equivalence, i.e.
	\[
		\equivp(k_X)= \bigoplus_{v\in X} [\hat{v}] \to \bigoplus_{e\in X} [\hat{e}] \to \bigoplus_{\sigma\in X} [\hat{\sigma}] \to \cdots.
	\]
\end{thm}
\begin{proof}
	The proof is immediate given the previous description of taking stalks, i.e. one can check directly the formula
	\[
		\equivp(k_X)\otimes_X F \cong \bigoplus_{v\in X} F(v) \to \bigoplus_{e\in X} F(e) \to \bigoplus_{\sigma\in X} F(\sigma) \to \cdots
	\]
	whose cohomology is by definition the compactly supported cohomology of a cellular sheaf $F$.
\end{proof}

This perspective is especially satisfying for the following reason: it makes transparent how the underlying topology of the space $X$ is coupled with the cohomology of a sheaf $F$. Compactly supported sheaf cohomology arises by tensoring with the complex of cosheaves that computes the Borel-Moore homology of the underlying space.

\section{Sheaf Homology and Future Directions}
\index{sheaf!homology, cellular}
The perspective of tensoring sheaves and cosheaves together offers numerous directions for further research both in pure and applied sheaf theory. Just the heuristic that
\begin{quote}
\begin{center}
	\emph{each cosheaf determines a (co-)continuous valuation on the category of sheaves,}
\end{center}
\end{quote}
is suggestive of the idea that if we are going to use sheaves to model the world, then cosheaves should allow us to weight different models of the world.

After having recovered some classical operations on sheaves, we are left with many more to consider. For example the constant cosheaf $\hat{k}_X$ should act on sheaves by returning its colimit, i.e. zeroth sheaf homology
\[
	\hat{k}_X\otimes_X - :\Shv(X) \to \Vect \qquad F \rightsquigarrow H_0(X;F)=p_{\dd} F.
\]
By taking a projective resolution of the constant cosheaf once and for all, one then gets for free a way of computing \textbf{higher sheaf homology}. This yet-to-be-explored theory has only recently found its use in applications, e.g. the work of Sanjeevi Krishnan on max-flow min-cut.

Additionally, the decategorification of the pairing of the categories of constructible sheaves and cosheaves provides an alternative approach to the study of Euler integration and leads in a natural way to the study of \textbf{higher Euler calculus} through higher K-theory. More directly the operation of pairing sheaves and cosheaves is reminiscent of a convolution operation. This area is under active research in collaboration with Aaron Royer.
\index{coend!higher Euler calculus}

