
%%%%%%%%%%%%%%%%%%%%%%%%%%%%%%%%%%%%%%%%%%%%%%%%%%%%%%%%%%%%%
%%%%%%%%%%%%%%%%%%%%%%%%%%%%%%%%%%%%%%%%%%%%%%%%%%%%%%%%%%%%%
%
%  %%%%%%%  %    %  %%%%%%  %%%%%%  %%%%%%%  %%%%%%%  %%%%%
%  %        %    %  %    %  %     %    %     %        %    %
%  %        %    %  %%%%%%  %     %    %     %        %    %
%  %        %%%%%%  %    %  %%%%%%     %     %%%%%    %%%%%
%  %        %    %  %    %  %          %     %        %    %
%  %%%%%%%  %    %  %    %  %          %     %%%%%%%  %     %
%
%%%%%%%%%%%%%%%%%%%%%%%%%%%%%%%%%%%%%%%%%%%%%%%%%%%%%%%%%%%%%
%%%%%%%%%%%%%%%%%%%%%%%%%%%%%%%%%%%%%%%%%%%%%%%%%%%%%%%%%%%%%

\chapter{Duality: Exchange of Sheaves and Cosheaves}
\label{sec:duality}

\begin{quote}
{\em ``Quod superius sicut quod inferius, et quod inferius sicut quod superius, ad perpetranda miracula rei unius.''}
\begin{flushright} --- Hermes Trismegistus~\cite[p. xxix]{vaughan1888magical} \end{flushright}

\end{quote}

In this section, we are concerned with the derived equivalence of cellular sheaves and cosheaves. In Section \ref{subsec:closures}, we introduce the functor that establishes this equivalence and try to motivate it topologically via taking the ``closure'' of the data over an open cell. In the case when $X$ is a manifold, Theorem \ref{thm:mfld_duality} gives us a duality result for data that relates sheaf cohomology with our new theory of sheaf homology. Finally, the equivalence is proved in Section \ref{subsec:derived_equiv}.

\section{Taking Closures and Classical Dualities Re-Obtained}
\label{subsec:closures}

In this section we are going to explain the all-important Poincar\'e-Verdier duality as an exchange of sheaves and cosheaves. To introduce this duality, we explain an odd, but clean way of going from a cellular sheaf to a complex of cellular cosheaves. This is meant to express the idea that duality is an exchange of open and closed cells.

Suppose we start with a sheaf $F$ on the unit interval $X=[0,1]$ stratified with end points $x=0$, $y=1$, and $a=(0,1)$. Such a sheaf is just a diagram of vector spaces of the form
\[
	\xymatrix{& F(a) & \\ F(x) \ar[ur]^{\rho_{a,x}} & & \ar[lu]_{\rho_{a,y}} F(y) .}
\]
Now we are going to extend the value of the sheaf on a cell $\sigma$ to its closure $\bar{\sigma}$ by defining $\hF(\tau)=F(\sigma)$ for every cell $\tau\leq \sigma$ and using the identity maps from $\sigma$ to its faces. This in effect smears the value of the sheaf on an open cell onto all of its faces. However, what should we do to the values of the sheaf already stored on a face $\tau$? This is where we use the different slots in a complex of vector space to store independently the values:
\[
\xymatrix{F(a) & \ar[l]_{\id} F(a) \ar[r]^{\id} & F(a) \\ F(x) \ar[u]^{\rho_{a,x}} & \ar[l] 0 \ar[u] \ar[r] & \ar[u]_{\rho_{a,y}} F(y) .}
\]
For dimension reasons, it should be clear that this smearing operation defines an assignment of chain complexes to each open cell with chain maps extending to the faces:
\[
	\xymatrix{& \ar[ld]_{r^{\bullet}_{x,a}} \equivp(F)(a) \ar[rd]^{r^{\bullet}_{y,a}} & \\ \equivp(F)(x) & & \equivp(F)(y) .}
\]
This motivates the following general definition of a functor $\equivp$: to a cellular sheaf $F\in \Shv(X)$ we associate the following cosheaf of chain complexes $\equivp(F)$
\[
	\equivp(F): \qquad \sigma \qquad \rightsquigarrow \qquad F(\sigma) \to \bigoplus_{\sigma\leq_1\tau} F(\tau) \to \bigoplus_{\sigma\leq_2\gamma} F(\gamma) \to \cdots.
\]
where $F(\sigma)$ is placed in cohomological degree $\dim|\sigma|$ or homological degree $-\dim|\sigma|$
However, in order for this to be a chain complex, following two arrows in sequence should give zero. In order to guarantee this we need to use the fact that $X$ is a cell complex, and as such for any pair of cells $\sigma\leq_2\gamma$ differing in dimension by two, there are precisely two ways $\tau_1,\tau_2$ of going between $\sigma$ and $\gamma$. Using the signed incidence relations $[\sigma:\tau_i]$ and the restrictions maps internal to $F$ allows us to define the differentials in this complex by $d^{i+1}:=\oplus [\tau:\gamma]\rho^F_{\gamma,\tau}$. Now let's consider a cell $\lambda$ that is a codimension one face of $\sigma$, then the extension map $r^{\bullet}_{\lambda,\sigma}$ is defined to be the chain map
\[
	\xymatrix{0 \ar[r] \ar[d]_{r^{i-1}_{\lambda,\tau}} & F(\sigma) \ar[r]^{d^i} \ar[d]_{r^{i}_{\lambda,\tau}} & \bigoplus_{\sigma\leq_1\tau} F(\tau) \ar[r]^{d^{i+1}} \ar[d]_{r^{i+1}_{\lambda,\tau}} & \bigoplus_{\sigma\leq_2\gamma} F(\gamma) \ar[d]_{r^{i+2}_{\lambda,\tau}} \\
	F(\lambda) \ar[r]_{d^{i-1}} & \bigoplus_{\lambda\leq_1\sigma} F(\sigma) \ar[r]_{d^i} & \bigoplus_{\lambda\leq_2\tau} F(\tau) \ar[r]_{d^{i+1}} & \bigoplus_{\lambda\leq_3\gamma} F(\gamma) }.
\]
The reason it is a chain map is clear from the fact that if $\lambda\leq\sigma$ then $U_{\sigma}\subset U_{\lambda}$ and so the chain complex $\equivp(F)(\sigma)$ simply includes term by term into the chain complex $\equivp(F)(\lambda)$.

Although the idea of a cosheaf of chain complexes is perhaps easier to visualize, for actual algebraic manipulation, one uses a chain complex of cosheaves to express the same idea in a different way. 

\begin{defn}[Poincar\'e-Verdier Equivalence Functor]\index{duality!Poincar\'e-Verdier equivalence functor}
	Let $X$ be a cell complex and let $\Shv(X)$ and $\Coshv(X)$ denote the categories of cellular sheaves and cosheaves respectively. We define the \textbf{Poincar\'e-Verdier Equivalence Functor} $\equivp:D^b(\Shv(X))\to D^b(\Coshv(X))$ by the following formula: to a sheaf $F\in \Shv(X)$ we associate the following complex of projective co-sheaves, the cohomological degree corresponding to the dimension of the cell:
	\[
		\xymatrix{\cdots \ar[r] & \bigoplus_{\sigma^i\in X}[\hat{\sigma}^i]^{F(\sigma^i)} \ar[r]^-{[\sigma:\gamma]\rho^F} & \bigoplus_{\gamma^{i+1}\in X}[\hat{\gamma}^{i+1}]^{F(\gamma^{i+1})} \ar[r]^-{[\gamma:\tau]\rho^F} & \bigoplus_{\tau^{i+2}\in X}[\hat{\tau}^{i+2}]^{F(\tau^{i+2})} \ar[r] & \cdots}
	\]
	Here $\sigma^i$ denotes the $i$-cells and $[\sigma^i:\gamma^{i+1}]=\{0,\pm 1\}$ records whether the cells are incident and whether orientations agree or disagree. The maps in between are to be understood as the matrix $\oplus [\sigma^i:\gamma^{i+1}]\rho^F_{\sigma,\gamma}$.

	For a complex of sheaves
	\[
	\xymatrix{F^i \ar[d] & \rightsquigarrow & \cdots \ar[r] & \bigoplus_{\gamma^{j+1}\in X}[\hat{\gamma}^{j+1}]^{F^i(\gamma^{j+1})} \ar[r]_-{[\gamma:\tau]\rho^{F^i}} \ar[d] & \bigoplus_{\tau^{j+2}\in X}[\hat{\tau}^{j+2}]^{F^i(\tau^{j+2})} \ar[r] \ar[d] & \cdots \\
	F^{i+1} \ar[d] & \rightsquigarrow & \cdots \ar[r] & \bigoplus_{\gamma^{j+1}\in X}[\hat{\gamma}^{j+1}]^{F^{i+1}(\gamma^{j+1})} \ar[r]_-{[\gamma:\tau]\rho^{F^{i+1}}} \ar[d] & \bigoplus_{\tau^{j+2}\in X}[\hat{\tau}^{j+2}]^{F^{i+1}(\tau^{j+2})} \ar[r] \ar[d] & \cdots \\
	F^{i+2} & \rightsquigarrow & \cdots \ar[r] & \bigoplus_{\gamma^{j+1}\in X}[\hat{\gamma}^{j+1}]^{F^{i+2}(\gamma^{j+1})} \ar[r]_-{[\gamma:\tau]\rho^{F^{i+2}}}  & \bigoplus_{\tau^{j+2}\in X}[\hat{\tau}^{j+2}]^{F^{i+2}(\tau^{j+2})} \ar[r]  & \cdots}
	\]
	where we then pass to the totalization.
\end{defn}
	
Before discussing why this functor is an equivalence, let us deduce a few computational consequences of this functor.
\begin{thm}
	If $F$ is a cell sheaf on a cell complex $X$, then
	\[
		H^i_c(X;F)\cong H_{-i}(X;\equivp(F)).
	\]
\end{thm}
\begin{proof}
First we apply the equivalence functor $\equivp$ to $F$
\[
	\xymatrix{0 \ar[r] & \bigoplus_{v\in X}[\hat{v}]^{F(v)} \ar[r]_-{[v:e]\rho{e,v}} & \bigoplus_{e\in X}[\hat{e}]^{F(e)} \ar[r]_-{[e:\sigma]\rho_{\sigma,e}} & \bigoplus_{\sigma\in X}[\hat{\sigma}]^{F(\sigma)} \ar[r] & \cdots}
\]
Taking colimits (pushing forward to a point) term by term produces the complex of vector spaces
\[
	\xymatrix{0 \ar[r] & \bigoplus_{v\in X}F(v) \ar[r]_-{[v:e]\rho{e,v}} & \bigoplus_{e\in X}F(e) \ar[r]_-{[e:\sigma]\rho_{\sigma,e}} & \bigoplus_{\sigma\in X}F(\sigma) \ar[r] & \cdots}
\]
which the reader should recognize as being the computational formula for computing compactly supported sheaf cohomology.
\end{proof}

Now let us give a simple proof of the standard Poincar\'e duality statement on a manifold $X$ with coefficients in an arbitrary cell sheaf $F$, except this time the sheaf homology groups are used. 

\begin{thm}\label{thm:mfld_duality}\index{duality!over a manifold}
	Suppose $F$ is a cell sheaf on a cell complex $X$ that happens to be a compact manifold (so it has a dual cell structure $\hat{X}$), then
	\[
		H^i(X;F)\cong H_{n-i}(X;F).
	\]
	Where the group on the right is not just notational, but it indicates the left-derived functors of $p_{\dd}$ on sheaves.
\end{thm}

\begin{proof}
	We repeat the first step of the proof of the previous theorem. By feeding $F$ through the equivalence $\equivp$ we get a complex of cosheaves. Pushing forward to a point yields a complex whose (co)homology is the compactly supported cohomology of the sheaf $F$. Now we recognize that the formula yields a formula for the Borel-Moore homology for the cosheaf naturally defined on the dual cell structure.
	\[
		\xymatrix{0 \ar[r] & \ar@{~>}[d] \bigoplus_{v\in X}F(v) \ar[r]_-{[v:e]\rho{e,v}} & \ar@{~>}[d] \bigoplus_{e\in X}F(e) \ar[r]_-{[e:\sigma]\rho_{\sigma,e}} & \ar@{~>}[d] \bigoplus_{\sigma\in X}F(\sigma) \ar[r] & \cdots \\
		0 \ar[r] & \bigoplus_{\tilde{v}\in \tilde{X}}\hF(\tilde{v}) \ar[r]_-{[\tilde{v}:\tilde{e}]\rho{\tilde{e},\tilde{v}}} & \bigoplus_{\tilde{e}\in \tilde{X}}\hF(\tilde{e}) \ar[r]_-{[\tilde{e}:\tilde{\sigma}]\rho_{\tilde{\sigma},\tilde{e}}} & \bigoplus_{\tilde{\sigma}\in \tilde{X}}\hF(\tilde{\sigma}) \ar[r] & \cdots}
	\]
	Taking the homology of the bottom row is the usual formula for the Borel-Moore homology of a cellular cosheaf except the top dimensional cells are place in degree 0, the $n-1$ cells in degree -1, and so on. Everything being shifted by $n=\dim X$ we get the isomorphism
	\[
		H_{-i}(X;\equivp(F))\cong H^{BM}_{n-i}(\tilde{X};\hF).
	\]
	However, we already observed in Theorem~\ref{thm:mfld_sheaf_cosheaf} that the diagrams $\hF$ on $\tilde{X}$ and $F$ on $X$ are the same in every possible way, so in particular sheaf homology of $F$ must coincide with cosheaf homology of $\hF$. Thus using compactness to drop the Borel-Moore label and chaining together the previous theorem we get
	\[
		H^i(X;F)\cong H_{-i}(X;\equivp(F))\cong H_{n-i}(\tilde{X};\hF)\cong H_{n-i}(X;F).
	\]
\end{proof}

\section{Derived Equivalence of Sheaves and Cosheaves}
\label{subsec:derived_equiv}

Historically, the derived equivalence of cellular sheaves and cosheaves appears in a few places and is re-discovered again and again. In chronological order, the first published proof appears to be in the 1998 paper of Peter Schneider in ``Verdier Duality on the Building''~\cite{schneider-vd}, which is a follow-up of a longer paper connecting sheaves, buildings and representation theory~\cite{schneider-rep}. Unfortunately, Schneider uses the term ``local coefficient systems'' to mean what we mean by cellular cosheaves. At around the same time Maxim Vybornov made explicit mention of the relationship between sheaves and cosheaves, relating them through Koszul duality~\cite{vybornov-triang}, but it took up until 2005 for Kohji Yanagawa to explicitly state that Vybornov's work implied the derived equivalence of sheaves and cosheaves~\cite{yanagawa}.

However, the perspective presented here was arrived at independently of the above work. In early March 2012, Bob MacPherson gave a lecture (which the author attended) where he conjectured that the derived category of cellular sheaves and cosheaves should be equivalent. Within a few weeks the author produced a proof. After some truly insightful comments from David Lipsky, the equivalence was refined to its current form.

Although the ideas were foreshadowed by many sources, the use of stalk (co)sheaves appears to be a novel way of arguing.

\begin{thm}[Equivalence]\label{thm:equivalence}\index{duality!derived equivalence}\index{derived equivalence!of cellular sheaves and cosheaves}\index{sheaf!cellular!derived equivalence with cosheaves}\index{cosheaf!cellular!derived equivalence with sheaves}
	$\equivp:D^b(\Shv(X))\to D^b(\Coshv(X))$ is an equivalence.
\end{thm}
\begin{proof}
First let us point out that the functor $\equivp$ really is a functor. Indeed if $\alpha:F\to G$ is a map of sheaves then we have maps $\alpha(\sigma):F(\sigma)\to G(\sigma)$ that commute with the respective restriction maps $\rho^F$ and $\rho^G$. As a result, we get maps $[\hat{\sigma}]^{F(\sigma)}\to [\hat{\sigma}]^{G(\sigma)}$. Moreover, these maps respect the differentials in $\equivp(F)$ and $\equivp(G)$, so we get a chain map. It is clearly additive, i.e. for maps $\alpha,\beta:F\to G$ $\equivp(\alpha+\beta)=P(\alpha)+P(\beta)$. This implies that $\equivp$ preserves homotopies.

It is also clear that $\equivp$ preserves quasi-isomorphisms. Note that a sequence of cellular sheaves $A^{\bullet}$ is exact if and only if $A^{\bullet}(\sigma)$ is an exact sequence of vector spaces for every $\sigma\in X$. This implies that $\equivp(A^{\bullet})$ is a double-complex with exact rows. By standard results surrounding the theory of spectral sequences or by the acyclic assembly lemma (\cite{weibel} Lem. 2.7.3) we get that the totalization is exact.

Let us understand what this functor does to an elementary injective sheaf $[\sigma]^V$. Applying the definition we can see that
\[
\xymatrix{
 \equivp: [\sigma]^V &\rightsquigarrow& \bigoplus_{\tau^0\subset\sigma}[\hat{\tau}^0]^V \ar[r] &\cdots \ar[r] & \bigoplus_{\tau^i\subset\sigma}[\hat{\tau}^i]^V \ar[r] &\cdots \ar[r] & [\hat{\sigma}]^V
}
\]
which is nothing other than the projective cosheaf resolution of the skyscraper (or stalk) cosheaf $\hat{S}_{\sigma}^V$ supported on $\sigma$, i.e.
\[
	\hat{S}_\sigma^V(\tau)=\left\{\begin{array}{ll} V &\sigma=\tau\\ 0 & \mathrm{o.w.}\end{array}\right.
\]
Consequently, there is a quasi-isomorphism $q:\equivp([\sigma]^V)\to \hat{S}_{\sigma}^V[-\dim\sigma]$ where $\hat{S}_{\sigma}^V$ is placed in degree equal to the dimension of $\sigma$ assuming that $[\sigma]^V$ is initially in degree 0. By abusing notation slightly and letting $\equivpc$ send cosheaves to sheaves, we see that
\[
 \equivpc(q):\equivpc\equivp([\sigma]^V)\to \equivpc(S_{\sigma}^V)=[\sigma]^V
\]
and thus we can define a natural transformation from $\equivpc\equivp$ to $\id_{D^b(\Shv)}$ when restricted to elementary injectives. However, by Lemma \ref{lem:inj} we know that every injective looks like such a sum, so this works for injective sheaves concentrated in a single degree. However, it is clear that $\equivp$ sends a complex of injectives, before taking the totalization of the double complex to the projective resolutions of a complex of skyscraper cosheaves. Applying $\equivp$ to the quasi-isomorphism relating the double complex of projective cosheaves to the complex of skyscrapers, extends the natural transformation to the whole derived category. However, since $\equivp$ preserves quasi-isomorphisms, this natural transformation is in fact an equivalence. This shows $\equivpc\equivp\cong\id_{D^b(\Shv)}$. Repeating the argument starting from co-sheaves shows that
\[
 \equivp:D^b(\Shv(X))\leftrightarrow D^b(\Coshv(X)):\equivpc
\]
is an adjoint equivalence of categories.
\end{proof}

The above proof should be taken as the primary duality result from which other dualities spring. This was not always appreciated and the author's first attack on the proof was to chain together two well-known dualities, which we review in the next two sections.

\subsection{Linear Duality}
There is an endofunctor on the category of finite dimensional vector spaces $\vect$ given by sending a vector space to its dual $V\rightsquigarrow V^*$. This functor has the effect of taking a cellular sheaf $(F,\rho)$ to a cellular co-sheaf $(F^*,\rho^*)$, since the restriction maps get dualized into extension maps. It is contravariant since a sheaf morphism $F\to G$ gets sent to a co-sheaf morphism in the opposite direction $F^*\leftarrow G^*$ as one can easily check. We can promote this functor to the derived category, using a subscript $f$ to remind the reader when we restrict to the finite dimensional full subcategories.

\begin{defn}[Linear Duals]\index{duality!linear}\index{linear duality}
	Define $\hat{V}:D^b(\Shv(X))^{op}\to D^b(\Coshv(X))$ as follows
	\begin{itemize}
		\item[-] $\hat{V}(F^{\bullet})=(F^*)^{-\bullet}$, i.e. take a sheaf in slot $i$, dualize its internal restriction maps $\rho_{\sigma,\tau}^{F^i}$ to extension maps $r^{F^{i*}}_{\tau,\sigma}$ to obtain a co-sheaf and then put it in slot $-i$.
		\item[-] $\hat{V}$ sends differentials between sheaves $d^i$ to their adjoints in negative degree $\partial^{-i-1}:=(d^i)^*$
		\[\star(\xymatrix{\cdots\ar[r] & F^{i} \ar[r]^{d^i} & F^{i+1} \ar[r] & \cdots)}=\xymatrix{\cdots \ar[r] & [(F^{i+1})^*]^{-i-1} \ar[r]_{\partial^{-i-1}}& [(F^{i})^*]^{-i} \ar[r] & \cdots }\]
		We'll adopt the convention that lowering the index increases the degree $\partial^{-i-1}\to \partial_{i}$.
	\end{itemize}
	We will reserve the right to abuse notation and let $V$ map from co-sheaves to sheaves in the obvious manner, i.e. $V:D^b(\Coshv(X))^{op}\to D^b(\Shv(X))$ or formally equivalent $V:D^b(\Coshv(X))\to D^b(\Shv(X))^{op}$.
\end{defn}

\begin{lem}
	$\hat{V}_f:D^b(\Shv_f(X))\to D^b(\Coshv_f(X))^{op}$ is an equivalence of categories.
\end{lem}
\begin{proof}
	It is clear that if $\alpha:I^{\bullet}\to J^{\bullet}$ is a map in the category of complexes of sheaves homotopic to zero $\alpha\simeq 0$, i.e. there exists a map $h:I^{\bullet}\to J^{\bullet-1}$, written $h:I\to J[-1]$ such that $\alpha^n-0^n=d_J^{n-1}h^n+h^{n+1}d_I^n$. Writing out how $\hat{V}$ acts carefully we see that $\hat{V}(\alpha):\hat{V}(J)\to \hat{V}(I)$ and $\hat{V}(h):V(J[-1])=\hat{V}(J)[+1]\to \hat{V}(I)$ defines a homotopy between $\hat{V}(\alpha)$ and $\hat{V}(0)=0$ by setting $(h^*)^{\bullet}=\hat{V}(h)^{\bullet -1}$.

	$\hat{V}$ thus sends $K^b(Inj-S_f)^{op}$ to $K^b(Proj-C_f)$ and composed twice $V\hat{V}:K^b(Inj-S_f)\to K^b(Inj-S_f)$ is naturally isomorphic to the identity functor, so it is an equivalence. We can repeat the arguments for co-sheaves and use formality to put the $^{op}$ where we want.
\end{proof}

\subsection{Verdier Dual Anti-Involution}

\begin{defn}[Verdier Dual]\index{Verdier duality}\index{duality!Verdier}
The Verdier dual functor $D:D(\Shv_f(X))\to D(\Shv_f(X))^{op}$ is defined as $D:=\mathcal{H}om(-,\omega^{\bullet}_X)$. Recall that $\mathcal{H}om(F,G)$ is a sheaf whose value on a cell $\sigma$ is given by $\Hom(F|_{st(\sigma)},G|_{st(\sigma)})$, i.e. natural transformations between the restrictions to the star of $\sigma$.

The complex of injective sheaves $\omega_X^{\bullet}$ is called the dualizing complex of $X$. It has in negative degree $\omega^{-i}_X$ the sum over the one-dimensional elementary injectives concentrated on $i$-cells $[\gamma^i]$. The maps between use the orientations on cells to guarantee it is a complex.
\[
\xymatrix{\cdots \ar[r] & \oplus_{|\tau|=i+1}[\tau] \ar[r]^{\oplus[\gamma:\tau]} & \oplus_{|\gamma|=i}[\gamma] \ar[r]^{\oplus[\sigma:\gamma]} & \oplus_{|\sigma|=i-1}[\sigma] \ar[r] &\cdots }
\]
The Verdier dual of $F$ is the complex of sheaves $D^{\bullet}F:=\mathcal{H}om(F,\omega^{\bullet}_X)$. Written out explicitly it is
\[
\xymatrix{\cdots \ar[r] & \oplus_{|\tau|=i+1}[\tau]^{F(\tau)^*} \ar[r]_{\oplus[\gamma:\tau]\rho^*} & \oplus_{|\gamma|=i}[\gamma]^{F(\gamma)^*} \ar[r]_{\oplus[\sigma:\gamma]\rho^*} & \oplus_{|\sigma|=i-1}[\sigma]^{F(\sigma)^*} \ar[r] &\cdots }
\]
\end{defn}

\begin{prop}
	The functor $\equivp:D^b(\Shv_f(X))\to D^b(\Coshv_f(X))$ composed with linear duality $V:D^b(\Coshv_f(X))\to D^b(\Shv_f(X))^{op}$ gives the Verdier dual anti-equivalence, i.e. $D\cong V\equivp$.
\end{prop}
\begin{proof}
Just check by hand.
\end{proof}

\begin{rmk}
We could have used well-known facts about Verdier duality to prove a weaker version of our main theorem by restricting to finitely generated stalks.
\end{rmk}
