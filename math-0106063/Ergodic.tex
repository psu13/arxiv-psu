%!TEX root = IntroArithGrps.tex

\mychapter{Ergodic Theory} \label{ErgodicChap}

\prereqs{none.}

Ergodic Theory is the study of measure-theoretic aspects of group actions. 
%(Because ergodic theory is essentially synonymous with probability theory, one can also say that it is the study of probabilistic aspects of group actions.) 
%(Classical Ergodic Theory considers only actions of $\integer$ or~$\real$.)
Topologists and geometers may be more comfortable in the category of continuous functions, but important results in \cref{MargulisSuperChap,NormalSubgroupChap} will be proved by using measurable properties of actions of~$\Gamma$, so we will introduce some of the basic ideas. 

\section{Terminology}

The reader is invited to skim through this section, and refer back as necessary.

\begin{assump} \ 
\noprelistbreak
\begin{enumerate}
\item All measures are assumed to be \defit[sigma-finite measure@$\sigma$-finite measure]{$\sigma$-finite}. That is, if $\mu$ is a measure on a measure space~$X$, then we always assume that $X$ is the union of countably many subsets of finite measure.
\item We have no need for abstract measure spaces, so all measures are assumed to be \defit[measure!Borel]{Borel}. That is, when we say $\mu$ is a measure on a measure space~$X$, we are assuming that $X$ is a Borel subset of a complete, separable, metrizable space, and the implied $\sigma$-algebra on~$X$ consists of the subsets of~$X$ that are equal to a Borel set, modulo a set of measure~$0$.
\end{enumerate}
\end{assump}

\begin{defns}
Let $\mu$ be a measure on a measure space~$X$.
\noprelistbreak
	\begin{enumerate}
	\item We say $\mu$ is a \defit[measure!finite]{finite measure} if $\mu(X) < \infty$.
	\item A subset $A$ of~$X$ is:
		\begin{itemize}
		\item \defit[null set]{null} if $\mu(A) = 0$,
		\item \defit[conull set]{conull} if the complement of~$A$ is null.
		\end{itemize}
	\item We often abbreviate ``almost everywhere'' to ``a.e\zz.''
	\item \defit[essentially]{Essentially} is a synonym for ``almost everywhere\zz.'' For example, a function~$f$ is \emph{essentially constant} iff $f$~is constant (a.e.).
%	\item For any measurable function $\varphi \colon X \to Y$, the measure $\varphi_*\mu$ on~$Y$ is defined by
%		\nindex{$\varphi_*\mu$ = push-forward of measure~$\mu$}
%		$$ (\varphi_*\mu)(A) = \mu \bigl( \varphi^{-1}(A) \bigr) .$$
	\item Two measures $\mu$ and~$\nu$ on~$X$ are in the same \defit[measure!class]{measure class}
		%(or $\nu$ is \defit[equivalent!measure]{equivalent} to~$\mu$) 
if they have exactly the same null sets:
		$$ \mu(A) = 0 \iff \nu(A) = 0 .$$
	(This defines an equivalence relation.) Note that if $\nu = f \mu$, for some real-valued, measurable function~$f$, such that $f(x) \neq 0$ for a.e.\ $x \in X$, then $\mu$ and~$\nu$ are in the same measure class \csee{fmuClassOfMuEx}. 
The Radon-Nikodym Theorem \pref{RadonNikodym} implies that the converse is true.
	\end{enumerate}
\end{defns}

\begin{defns}
Suppose $H$ is a Lie group~$H$ that acts continuously on a metrizable space~$X$, $\mu$ is a measure on~$X$, and $A$~is a subset of~$X$.
	\begin{enumerate}
	\item The set $A$ is \defit[invariant!set]{invariant} (or, more precisely, \defit[invariant!set]{$H$-invariant}) if $hA = A$ for all $h \in H$.
	\item The measure~$\mu$ is \defit[invariant!measure]{invariant} (or, more precisely, \defit[invariant!measure]{$H$-invariant}) if $h_*\mu = \mu$ for all $h \in H$. (Recall that the push-forward $h_*\mu$ is defined in \pref{PushForwardDefn}.)
	\item The measure~$\mu$ is \defit[quasi-!invariant measure]{quasi-invariant} if $h_*\mu$ is in the same measure class as~$\mu$, for all $h \in H$.
	\item A (measurable) function $f$ on~$X$ is \defit[essentially!$H$-invariant]{essentially $H$-invariant} if, for every $h \in H$, we have
	$$ \text{$f(hx) = f(x)$ for a.e.\ $x \in X$.} $$
	\end{enumerate}
\end{defns}

The Lebesgue measure on a manifold is not unique, but it determines a well-defined measure class, which is invariant under any smooth action:

\begin{lem}[\csee{LebesgueMeasClassEx}]
If $X$ is a manifold, and $H$ acts on~$X$ by diffeomorphisms, then Lebesgue measure provides a measure on~$X$ that is quasi-invariant for~$H$.
\end{lem}

\begin{exercises}

\item \label{fmuClassOfMuEx}
Suppose $\mu$ is a measure on a measure space~$X$, and $f$~is a real-valued, measurable function on~$X$, such that $f \ge 0$ for a.e.~$x$. Show that $f \mu$ is in the measure class of~$\mu$ iff $f(x) \neq 0$ for a.e.\ $x \in X$.

\item Suppose a Lie group~$H$ acts continuously on a metrizable space~$X$, and $\mu$ is a measure on~$X$. Show that $\mu$ is quasi-invariant iff the collection of null sets is $H$-invariant. (This means that if $A$~is a null set, and $h \in H$, then $h(A)$ is a null set.)

\item \label{Diffble(null)=nullEx}
Suppose 
	\begin{itemize}
	\item $A$ is a null set in~$\real^n$ (with respect to Lebesgue measure),
	and
	\item $f$ is a diffeomorphism of some open subset~$\open$\, of~$\real^n$.
	\end{itemize}
Show that $f(A \cap \open\,)$ is a null set.
\hint{Change of variables.}

\item \label{LebesgueMeasClassEx}
Suppose $X$ is a (second countable) smooth, $n$-dimensional manifold. This means that $X$ can be covered by coordinate patches $(X_i,\varphi_i)$ (where $\varphi_i \colon X_i \to \real^n$, and the overlap maps are smooth).
	\begin{enumerate}
	\item Show there exists a partition $X = \bigcup_{i=1}^\infty \hat X_i$ into measurable subsets, such that $\hat X_i \subseteq X_i$ for each~$i$.
	\item Define a measure $\mu$ on~$X$ by $\mu(X) = \lambda \bigl( \varphi_i(A \cap \hat X_i) \bigr)$, where $\lambda$ is the Lebesgue measure on~$\real^n$. This measure may depend on the choice of $X_i$, $\varphi_i$, and~$\hat X_i$, but show that the measure class of~$\mu$ is independent of these choices. 
	\end{enumerate}
\hint{\Cref{Diffble(null)=nullEx}.}

\end{exercises}



\section{Ergodicity} \label{ErgodicitySect}

Suppose $H$ acts on a topological space~$X$. If $H$ has a dense orbit on~$X$, then it is easy to see that every continuous, $H$-invariant function is constant \csee{DenseOrb->FuncConstEx}. Ergodicity is the much stronger condition that every \emph{measurable} $H$-invariant function is constant (a.e.):

\begin{defn} \label{ErgodicDefn}
Suppose $H$ acts on~$X$ with a quasi-invariant measure~$\mu$.
We say the action of~$H$ is \defit{ergodic} (or that $\mu$~is an \defit{ergodic} measure for~$H$) if every $H$-invariant, real valued, measurable function on~$X$ is essentially constant.
\end{defn}

It is easy to see that transitive actions are ergodic \csee{transitive->erg}. But non-transitive actions can also be ergodic:

\begin{eg}[(Irrational rotation of the circle)] \label{IrratRotErg}
For any $\alpha \in \real$, we may define a homeomorphism~$T_\alpha$ of the circle $\torus = \real/\integer$ by
	$$ T_\alpha(x) = x + \alpha \pmod{\integer} .$$
By considering Fourier series, it is not difficult to show that if $\alpha$~is irrational, then every $T_\alpha$-invariant function in $\LL2(\torus)$ is essentially constant \csee{IrratRotErgEx}. This implies that the $\integer$-action generated by~$T_\alpha$ is ergodic \csee{ErgCheckLp}.
\end{eg}

\Cref{IrratRotErg}  is a special case of the following general result:

\begin{prop} \label{Dense->Erg}
If $H$ is any dense subgroup of a Lie group~$L$, then the natural action of~$H$ on~$L$ by left translation is ergodic\/ \textup(with respect to the Haar measure on~$L$\textup).
\end{prop}

\begin{proof}
For any measurable $f \colon L \to \real$, its \defit[stabilizer!essential]{essential stabilizer} in~$L$ is defined to be:
	$$ \Stab_L(f) = \{\, g \in L \mid \text{$f(gx) = f(x)$ \ for a.e.\ $x \in L$} \,\} .$$
It is not difficult to show that $\Stab_L(f)$ is closed \csee{EssStabClosed}.
(It is also a subgroup of~$L$, but we do not need this fact.)  Hence, if $\Stab_L(f)$ contains a dense subgroup~$H$, then it must be all of~$L$. This implies that $f$ is constant (a.e.) \csee{EssInvt->EssConst}.
\end{proof}

It was mentioned above that transitive actions are ergodic; therefore, $G$ is ergodic on $G/\Gamma$.
What is not obvious, and leads to important applications for arithmetic groups, is that most subgroups of~$G$ are also ergodic on $G/\Gamma$:

\begin{thm}[(\thmindex{Moore Ergodicity}Moore Ergodicity Theorem, see \cref{MooreErgLatticeEx})] \label{MooreErgodicity}
If 
	\begin{itemize}
	\item $H$ is any noncompact, closed subgroup of~$G$, 
	and
	\item $\Gamma$~is irreducible,
	\end{itemize}
then $H$ is ergodic on $G/\Gamma$.
\end{thm}

If $H$ is ergodic on $G/\Gamma$, then $\Gamma$ is ergodic on $G/H$ \csee{MackeySwitchErgodic}. Hence:

\begin{cor} \label{GammaErgOnG/H}
If $H$ and\/~$\Gamma$ are as in \cref{MooreErgodicity},
then\/ $\Gamma$ is ergodic on $G/H$.
\end{cor}


\begin{exercises}

\item \label{EssStabClosed}
Show $\Stab_L(f)$ is closed, for every Lie group~$L$ and measurable $f \colon L \to \real$.
\hint{If $f$~is bounded, then, for any $\varphi \in C_c(L)$ and $\{g_n\} \subseteq \Stab_L(f)$, we have
	$ \int_L {}^g \! f \cdot \varphi \,d \mu
	= \int_L f \cdot {}^{g^{-1}} \! \varphi \,d \mu
	= \lim \int_L f \cdot {}^{g_n^{-1}} \! \varphi \,d \mu
	= \lim \int_L {}^{g_n} \! f \cdot  \varphi \,d \mu
%	= \lim_{n \to \infty} \int_L f \cdot  \varphi \,d \mu
	= \int_L f \cdot  \varphi \,d \mu
	$.} %since $\varphi$~is uniformly continuous.
%There is a weak topology on $\LL\infty(L, \mu)$, defined by 
%	$$ f_n \to f \quad \Leftrightarrow \quad
%	\int_L f_n \varphi \,d\mu \to \int_L f\varphi \,d\mu
%	\ \text{for all $\varphi \in \LL1(L,\mu)$} .$$
%Since functions in $\LL1$ can be approximated by uniformly continuous functions, it is not difficult to see that the action of~$L$ on  $\LL\infty(L, \mu)$ is continuous (with respect to this weak topology). This implies that stabilizers are closed.}

\item \label{DenseOrb->FuncConstEx}
Suppose $H$ acts on a topological space~$X$, and has a dense orbit. Show that every real-valued, continuous, $H$-invariant function on~$X$ is constant.

\item \label{transitive->erg}
Show that $H$ is ergodic on $H/H_1$, for every closed subgroup~$H_1$ of~$H$.
\hint{Every $H$-invariant function is constant, not merely essentially constant.}

\item \label{EssInvt->EssConst}
Suppose $H$ is ergodic on~$X$, and $f \colon X \to \real$ is measurable and essentially $H$-invariant. Show that $f$ is essentially constant.

\item Our definition of ergodicity is not the usual one, but it is equivalent:
show that $H$ is ergodic on~$X$ iff every $H$-invariant measurable subset of~$X$ is either null or conull. 
\hint{The characteristic function of an invariant set is an invariant function. Conversely, the sub-level sets of an invariant function are invariant sets.}

\item \label{IrratRotErgEx}
In the notation of \cref{IrratRotErg}, show (without using \cref{Dense->Erg}):
	\begin{enumerate}
	\item If $\alpha$~is irrational, then every $T_\alpha$-invariant function in $\LL2(\torus)$ is essentially constant.
	\item If $\alpha$~is rational, then there exist $T_\alpha$-invariant functions in $\LL2(\torus)$ that are not essentially constant.
	\end{enumerate}
\hint{Any $f \in \LL2(\torus)$ can be written as a unique Fourier series:
	$ f = \sum_{n=-\infty}^\infty a_n e^{in\theta}$.
If $f$ is invariant and $\alpha$~is irrational, then uniqueness implies $a_n = 0$ for $n \neq 0$.}

\item \label{ErgCheckLp}
Suppose $\mu$ is an $H$-invariant, \emph{finite} measure on~$X$. For all $p \in [1,\infty]$, show that $H$ is ergodic iff every $H$-invariant element of $\LL{p}(X,\mu)$ is essentially constant.

\item Let $H = \integer$ act on $X = \real$ by translation, and let $\mu$ be Lebesgue measure. Show:
	\begin{enumerate}
	\item $H$ is not ergodic on~$X$,
	and
	\item for every $p \in [1,\infty)$, every $H$-invariant element of $\LL{p}(X,\mu)$ is essentially constant.
	\end{enumerate}
Why is this not a counterexample to \cref{ErgCheckLp}?

\item Let $H$ be a dense subgroup of~$L$. Show that if $L$ is ergodic on~$X$, then $H$ is also ergodic on~$X$.
\hint{\cref{EssStabClosed}.}

\item Show that if $H$ acts continuously on~$X$, and $\mu$~is a quasi-invariant measure on~$X$, then the support of~$\mu$ is an $H$-invariant subset of~$X$.

\item \label{AEOrbitDense}
Ergodicity implies that a.e.\ orbit is dense in the support of~$\mu$. More precisely, show that if $H$ is ergodic on~$X$, and the support of~$\mu$ is all of~$X$ (in other words, no open subset of~$X$ has measure~$0$), then a.e.\ $H$-orbit in~$X$ is dense. (That is, for a.e.\ $x \in X$, the orbit $Hx$ of~$x$ is dense in~$X$.
\hint{The characteristic function of the closure of any orbit is invariant.}

\item \label{MackeySwitchErgodic}
Suppose $H$ is a closed subgroup of~$G$. Show that $H$ is ergodic on $G/\Gamma$ iff $\Gamma$ is ergodic on $G/H$.
\hint{$H \times \Gamma$ acts on~$G$ (by letting $H$ act on the left and $\Gamma$~act on the right). Show $H$ is ergodic on $G/\Gamma$ iff $H \times \Gamma$ is ergodic on~$G$ iff $\Gamma$ is ergodic on $G/H$.}

\item The Moore Ergodicity Theorem has a converse: Assume $G$ is not compact, and show that if $H$ is any compact subgroup of~$G$, then $H$~is not ergodic on $G/\Gamma$.
\hint{$\Gamma$ acts properly discontinuously on $G/H$, so the orbits are not dense.}

\item Show that if $n \ge 2$, then
	\begin{enumerate}
	\item the natural action of $\SL(n,\integer)$ on~$\real^n$ is ergodic,
	and
	\item the $\SL(n,\integer)$-orbit of a.e.\ vector in~$\real^n$ is dense in $\real^n$.
	\end{enumerate}
\hint{Identify $\real^n$ with a homogeneous space of $G = \SL(n,\real)$ (a.e.), by noting that $G$ is transitive on the nonzero vectors of~$\real^n$.}

\item
Let 	
	\begin{itemize}
	\item $G = \SL(3,\real)$,
	\item $\Gamma$ be a lattice in~$G$,
	and
	\item $P = \begin{Smallbmatrix} \upast&& \\ \upast&\upast& \\ \upast&\upast&\upast\end{Smallbmatrix} \subset G$.
	\end{itemize}
Show:
\begin{enumerate}
\item The natural action of~$\Gamma$ on the homogeneous space $G/P$ is ergodic.
\item The diagonal action of~$\Gamma$ on $(G/P)^2 = (G/P) \times (G/P)$ is ergodic.
\item The diagonal action of~$\Gamma$ on $(G/P)^3 = (G/P) \times (G/P) \times (G/P)$ is \emph{not} ergodic.
\end{enumerate}
\hint{$G$ is transitive on a conull subset of $(G/P)^k$, for $k \le 3$. What is the stabilizer of a generic point in each of these spaces?}

 \item \label{AEOrbitDenseInG/Gamma}
 Assume $\Gamma$ is irreducible, and let $H$ be a closed, noncompact subgroup of~$G$. Show, for a.e.\ $x \in G/\Gamma$, that $Hx$ is dense in $G/\Gamma$.

\item \label{RErgodic->ZErgodic}
Suppose $H$ acts ergodically on~$X$, with invariant measure~$\mu$. 
Show that if $\mu(X) < \infty$ and $H \iso \real$, then some cyclic subgroup of~$H$ is ergodic on~$X$.
\hint{For each $t \in \real$, choose a nonzero, $h^t$-invariant function $f_t \in L^2(X)$, such that $f_t \perp 1$. The projection of $f_r$ to the space of $h^s$-invariant functions is invariant under both $h^r$ and~$h^s$. 
Therefore, if $r$ and~$s$ are linearly independent over~$\rational$, then $f_r \perp f_s$. This is impossible, because $\LL2(X,\mu)$ is separable.}

\end{exercises}



\section{Consequences of a finite invariant measure}

Measure-theoretic techniques are especially powerful when the action has an invariant  measure that is finite. One example of this is the Poincar\'e Recurrence Theorem \pref{PoincareRecurThm}. Here is another.

We know that almost every orbit of an ergodic action is dense \csee{AEOrbitDense}.
For the case of a $\integer$-action with a finite, invariant measure, the orbits are not only dense, but uniformly distributed:

\begin{defn}
Let 
	\begin{itemize}
	\item $\mu$ be a finite measure on a topological space~$X$,
		% such that $\mu(X) = 1$,
	and
	\item $T$ be a homeomorphism of~$X$. 
	\end{itemize}
The $\langle T \rangle$-orbit of a point $x$ in~$X$ is \defit{uniformly distributed} with respect to~$\mu$ if
	$$ \lim_{n \to \infty} \frac{1}{n} \sum_{k=1}^n f \bigl( T^k(x) \bigr) 
	= \frac{1}{\mu(X)} \int_X f \, d\mu ,$$
for every bounded, continuous function~$f$ on~$X$.
\end{defn}

\begin{thm}[{(\thmindex{Ergodic!Pointwise}{Pointwise Ergodic Theorem})}] 
	%\index{Pointwise Ergodic Theorem|indsee{Ergodic Theorem, Pointwise}}
\label{PointwiseErgThm}
Suppose 
	\begin{itemize}
	\item $\mu$ is a finite measure on a second countable, metrizable space~$X$,
	and
	\item $T$ is an ergodic, measure-preserving homeomorphism of~$X$.
	\end{itemize}
Then a.e.\ $\langle T \rangle$-orbit in~$X$ is uniformly distributed\/ \textup(with respect to~$\mu$\textup).
\end{thm}

It is tricky to show that $\lim_{n \to \infty} \frac{1}{n} \sum_{k=1}^n f \bigl( T^k(x) \bigr)$ converges %to $\int_X f \, d\mu$ 
pointwise \csee{PtwiseErgZ}.
Convergence in norm is much easier \csee{MeanErgThmEx}. 

\begin{rem} \label{PtwiseErgAmenRem}
Although the Pointwise Ergodic Theorem was stated only for actions of a cyclic group, it generalizes very nicely to the ergodic actions of any \term{amenable group}. (The values of~$f$ are averaged over an appropriate \term[Folner@F\o lner!set]{F\o lner set} in the amenable group.) See \cref{PointwiseErgThmForREx} for actions of~$\real$.
\end{rem}


\begin{exercises}

\item Suppose the $\langle T \rangle$-orbit of~$x$ is uniformly distributed with respect to a finite measure~$\mu$ on~$X$. Show that if the support of~$\mu$ is all of~$X$, then the $\langle T \rangle$-orbit of~$x$ is dense in~$X$.

\item \label{IterateUnitaryEx}
Suppose 
	\begin{itemize}
	\item $U$ is a unitary operator on a \term{Hilbert space}~$\Hilbert$, 
	\item $v \in \Hilbert$,
	and
	\item $\langle v \mid w \rangle = 0$, for every vector~$w$ that is fixed by~$U$.
	\end{itemize}
Show $\frac{1}{n} \sum_{k=1}^n U^k v \to 0$ as $n \to \infty$.
\hint{Apply the {Spectral Theorem} to diagonalize the unitary operator~$U$.}

\item \label{MeanErgThmEx}
(\thmindex{Ergodic!Mean}{Mean Ergodic Theorem})
Assume the setting of the Pointwise Ergodic Theorem \pref{PointwiseErgThm}.
Show that if $f \in \LL2(X,\mu)$, then
	$$\frac{1}{n} \sum_{k=1}^n f \bigl( T^k(x) \bigr) \to \frac{1}{\mu(X)} \int_X f \, d\mu \text{\quad in $\LL2$} .$$
That is, show
	$$ \lim_{n \to \infty} \left\| \ \frac{1}{n} \sum_{k=1}^n f \bigl( T^k(x) \bigr) 
	\  - \  \frac{1}{\mu(X)} \int_X f \, d\mu \ \right\|_2 = 0 .$$
\emph{Do not assume \cref{PointwiseErgThm}.}
\hint{\Cref{IterateUnitaryEx}.}

\item \label{PtwiseErgZMaxl}
 Assume $X$, $\mu$, and~$T$ are as in \cref{PointwiseErgThm}, and that $\mu(X) = 1$. For $f \in L^1(X,\mu)$, define
	$$ S_n(x) =  f(x) + f \bigl( T(x) \bigr) + \cdots + f \bigl( T^{n-1}(x) \bigr) .$$
  Prove the \thmindex{Ergodic!Maximal}\textit{\textbf{Maximal Ergodic
Theorem}}: for every $\alpha \in \real$, if we let
 $$ E_\alpha = \bigset{ x \in X }{ \sup_n \frac{S_n(x)}{n} > \alpha} ,$$
 then $\int_E f \, d\mu \ge \alpha \, \mu(E)$.
 \hint{Assume $\alpha = 0$. Let $S^+_n(x) = \max_{0\le k\le n} S_k(x)$,
and $E_n = \{\, x \mid S^+_n > 0\,\}$, so $E = \cup_n E_n$. For $x \in
E_n$, we have $f(x) \ge S^+_n(x) - S^+_n \bigl( T(x) \bigr)$, so
$\int_{E_n} f\, d\mu \ge 0$.}

\item \label{PtwiseErgZ}
 Prove the Pointwise Ergodic Theorem \pref{PointwiseErgThm}.
 \hint{Either $\{\, x \mid \limsup S_n(x)/n >
\alpha \,\}$ or its complement must have measure~$0$. If it is the complement, then \cref{PtwiseErgZMaxl} implies $\int_X f \, d\mu \ge \alpha$.}

\item \label{PointwiseErgBddL1Ex}
Assume the setting of the Pointwise Ergodic Theorem \pref{PointwiseErgThm}. 
For every bounded $\varphi \in \LL1(X,\mu)$, show, for a.e.\ $x \in X$, that 
	$$ \lim_{n \to \infty} \frac{1}{n} \sum_{k=1}^n \varphi \bigl( T^k(x) \bigr) 
	= \frac{1}{\mu(X)} \int_X \varphi \, d\mu .$$
\hint{You may assume the Pointwise Ergodic Theorem. Use {Lusin's Theorem}~\pref{LusinsThm} to approximate~$\varphi$ by a continuous function.}

\item \harder 
Remove the assumption that $\varphi$ is bounded in \cref{PointwiseErgBddL1Ex}.

\item \label{PointwiseErgThmForREx}
Suppose 
	\begin{itemize}
	\item $\{a^t\}$ is a (continuous) $1$-parameter group of homeomorphisms of a topological space~$X$,
and
	\item $\mu$ is an ergodic, $a^t$-invariant, finite measure on~$X$.
	\end{itemize}
For every bounded, continuous function~$f$ on~$X$, show that
	$$ \text{$\displaystyle \lim_{T \to \infty} \frac{1}{T} \int_0^t f(a^t x) \, dt = \frac{1}{\mu(X)} \int_X f\, d\mu$ \quad  for a.e.~$x \in X$} .$$
\hint{Apply \cref{PointwiseErgThm} to $\overline{f}(x) = \int_0^1 f(a^t x)\, dt$ if $a^1$ is ergodic \ccf{RErgodic->ZErgodic}.}

\end{exercises}






\section{Ergodic decomposition} \label{ErgDecompSect}

In this \lcnamecref{ErgDecompSect}, we briefly explain that every group action (with a quasi-invariant measure) can be decomposed into ergodic actions.

\begin{eg}[(Irrational rotation of the plane)]
For any irrational $\alpha \in \real$, define a homeomorphism~$T_\alpha$ of~$\complex$ by
	$ T_\alpha(z) = e^{2\pi i \alpha} x $.
Then $|T_\alpha(z)| = |z|$, so each circle centered at the origin is invariant under~$T_\alpha\mk$. The restriction of $T_\alpha$ to any such circle is an irrational rotation of the circle, so it is ergodic \csee{IrratRotErg}. Thus, the entire action can be decomposed as a union of ergodic actions. 
\end{eg}

A similar decomposition is always possible, as long as we work with nice spaces:

\begin{defn}
A topological space~$X$ is \defit[Polish topological space]{Polish} if it is homeomorphic to a complete, separable metric space.
\end{defn}

\begin{thm}[(\thmindex{Ergodic!decomposition}Ergodic decomposition)] \label{ErgodicDecomp}
Suppose a Lie group~$H$ acts continuously on a Polish space~$X$, and $\mu$ is a quasi-invariant, finite measure on~$X$. Then there exist
	\begin{itemize}
	\item a measurable function $\zeta \colon X \to [0,1]$, 
	and
	\item a finite measure~$\mu_z$ on $\zeta^{-1}(z)$, for each $z \in [0,1]$, 
	\end{itemize}
such that $\mu = \int_{[0,1]} \mu_z \, d \nu(z)$, where $\nu = \zeta_* \mu$. For $f \in C_c(X)$, this means
	$$ \int_X f \, d\mu = \int_Z \, \int_{\zeta^{-1}(z)} f \, d\mu_z \, d\nu(z) .$$
Furthermore, for a.e.\ $z \in [0,1]$, 
	\begin{enumerate}
	\item $\zeta^{-1}(z)$ is $H$-invariant,
	and
	\item $\mu_z$ is quasi-invariant and ergodic for the action of~$H$.
	\end{enumerate}
\end{thm}

\begin{rem}
The ergodic decomposition is unique (a.e.). More precisely, if $\zeta'$ and~$\mu_z'$ also satisfy the conclusions, then there is a measurable bijection $\pi \colon [0,1] \to [0,1]$, such that 
	\begin{enumerate}
	\item $\zeta' = \pi \circ \zeta$ a.e.,
	and
	\item $\mu_{\pi(z)}' = \mu_z$ for a.e.~$z$.
	\end{enumerate}
\end{rem}

\begin{defn}
In the notation of \cref{ErgodicDecomp}, each set $\zeta^{-1}(z)$ is called an \defit[ergodic!component]{ergodic component} of the action.
\end{defn}


The remainder of this \lcnamecref{ErgDecompSect} sketches two different proofs of \cref{ErgodicDecomp}.


\subsection{First proof}
The main problem is to find the function~$\zeta$, because the following general Fubini-like theorem will then provide the required decomposition of~$\mu$ into an integral of measures~$\mu_z$ on the fibers of~$\zeta$. (In Probability Theory, each $\mu_z$ is called a \defit{conditional measure} of~$\mu$.)

\begin{prop} %[(\thmindex{Rokhlin's}Rokhlin)]
\label{RokhlinDecompMeas}
Suppose
	\begin{itemize}
	\item $X$ and~$Z$ are Polish spaces,
	\item $\zeta \colon X \to Z$ is a Borel measurable function,
	and
	\item $\mu$ is a probability measure on~$X$.
	\end{itemize}
Then there is a Borel map $\lambda \colon Z \to \Prob(X)$, such that 
 \begin{enumerate}
 \item $\mu = \int_Z \lambda_z \, d\nu(z)$, where $\nu = \zeta_*\mu$,
 and
 \item $\lambda_z \bigl( \zeta^{-1}(z) \bigr) = 1$, for all $z \in Z$.
 \end{enumerate}
 Furthermore, $\lambda$ is unique\/ \textup(a.e.\textup).
 \end{prop}

The map $\zeta$ is a bit difficult to pin down, since it is not completely well-defined --- it can be changed on an arbitrary set of measure zero. We circumvent this difficulty by looking not at the value of~$\zeta$ on individual points (which is not entirely well-defined), but at its effect on an algebra of functions (which is completely well defined). 

\begin{defns} \label{BoolDefn}
Assume $\mu$ is a finite measure on a Polish space~$X$.
	\begin{enumerate}
	\item Let 
	\nindex{$\Bool(X)$ = $\{ \text{Borel subsets} \}$, modulo sets of measure~$0$}%
	$\Bool(X)$ be the collection of all Borel subsets of~$X$, where two sets are identified if they differ by a set of measure~$0$. This is a $\sigma$-algebra.
%	Alternatively, by identifying a set with its characteristic function, we can think of $\Bool(X)$ as the set of $\{0,1\}$-valued measurable functions, considered as a subset of $\LL\infty(X, \mu)$ (which means that two functions are identified if they are equal almost everywhere). 
	\item $\Bool(X)$ is a complete, separable metric space, with respect to the metric $d(A,B) = \mu(A \symmdiff B)$, where 
	\nindex{$A \symmdiff B$ = symmetric difference of $A$ and~$B$}%
	$A \symmdiff B = (A \smallsetminus B) \cup (B \smallsetminus A)$ is the \defit[symmetric!difference]{symmetric difference} of $A$ and~$B$.
%	Being a subset of the unit ball in $\LL\infty(X,\mu)$, $\Bool(X)$ inherits a weak$^*$ topology.  The Banach-Alaoglu Theorem \pref{BanachAlaogluThm} implies that $\Bool(X)$ is compact.
 	\item If a Lie group~$H$ acts continuously on~$X$, we let $\Bool(X)^H$ be the set of $H$-invariant elements of~$\Bool(X)$. This is a sub-$\sigma$-algebra of $\Bool(X)$.
	\end{enumerate}
\end{defns}

%Note that $\Bool(X)$ is a \defit{Boolean algebra}. (This means that it is closed under the set-theoretic operations of union and complement.). Also, $\Bool(X)^H$ is a topologically closed Boolean subalgebra \csee{B(X)HClosedBoolean}. 
The map~$\zeta$ is constructed by the following result:
 
 \begin{lem}
 Suppose 
 	\begin{itemize}
	\item $\mu$ is a finite measure on a Polish space~$X$,
	and
	\item $\Bool$ is a sub-$\sigma$-algebra of $\Bool(X)$.
	\end{itemize}
Then there is a Borel map $\zeta \colon X \to [0,1]$, such that 
	$$ \Bool = \{\, \zeta^{-1}(E) \mid \text{$E$ is a Borel subset of\/ $[0,1]$} \,\} .$$
\end{lem}

\begin{proof}[Idea of proof]
Let 
	\begin{itemize}
	\item $\{E_n\}$ be a countable, dense subset of~$\Bool$,
	\item $\chi_n$ be the characteristic function of~$E_n$, for each~$n$,
	and
	\item $\displaystyle \zeta(x) = \sum_{n = 1}^\infty \frac{\chi_n(x)}{3^n}$.
	\end{itemize}

It is clear from the definition of~$\zeta$ that if $I$ is any open interval in $[0,1]$, then $\zeta^{-1}(I)$ is a Boolean combination of elements of~$\{E_n\}$; therefore, it belongs to~$\Bool$. Since $\Bool$ is a $\sigma$-algebra, this implies that $\zeta^{-1}(E) \in \Bool$ for every Borel subset~$E$ of $[0,1]$.

Conversely, it is clear from the definition of~$\zeta$ that each $E_n$ is the inverse image of a (one-point) Borel subset of $[0,1]$. Since $\{E_n\}$ generates $\Bool$ as a $\sigma$-algebra \csee{DenseGensSigmaAlg}, this implies that every element of~$\Bool$ is the inverse image of a Borel subset of $[0,1]$.
\end{proof}

We will use the following very useful fact:

\begin{thm}[(\thmindex{von\,Neumann Selection}von\,Neumann Selection Theorem)] \label{vonNeumannSelectionThm}
  Let
  \begin{itemize}
  \item $X$ and~$Y$ be Polish spaces,
  \item $\mu$ be a finite measure on~$X$,
  \item $\mathcal{F}$ be a Borel subset of $X \times Y$,
  and
  \item $X_{\mathcal{F}}$ be the projection of~$\mathcal{F}$ to~$X$.
  \end{itemize}
  Then there is a Borel function $\Phi \colon X \to Y$,
  such that $\bigl( x, \Phi(x) \bigr) \in \mathcal{F}$, for a.e.\ $x \in X_{\mathcal{F}}$.
  \end{thm}

\begin{proof}[The first proof of \cref{ErgodicDecomp}]
We wish to show that $\mu_z$ is ergodic (a.e.). If not, then there is a set~$E$ of positive measure in $[0,1]$, such that, for each $z \in E$, the action of~$H$ on $\bigl( \zeta^{-1}(z), \mu_z \bigr)$ is not ergodic. This means there exists an $H$-invariant, measurable, $\{0,1\}$-valued function $\varphi_z \in \LL\infty \bigl( \zeta^{-1}(z), \mu_z \bigr)$ that is \textbf{not} constant (a.e.).
There are technical problems that we will ignore, but, roughly speaking, the  von\,Neumann Selection Theorem \pref{vonNeumannSelectionThm} implies that the selection of $\varphi_z$ can be done measurably, so we have a Borel subset $A$ of~$X$, defined by
	$$ A = \{\, x \in X \mid \text{$\zeta(x) \in E$ and $\varphi_z(x) = 1 $} \,\} .$$

Since $\varphi_z$ is not constant on the fiber $\zeta^{-1}(z)$, we know that $A$ is not of the form $\zeta^{-1}(E)$, for any Borel subset~$E$ of $[0,1]$. On the other hand, we have $A \in \Bool(X)^H$ (since each $\varphi_z$ is $H$-invariant). This contradicts the choice of the function~$\zeta$.
\end{proof}


\subsection{Second proof} \label{ErgDecompPfChoquet}
We now describe a different approach. Instead of obtaining the decomposition of~$\mu$ from the map~$\zeta$, we reverse the argument, and obtain the map~$\zeta$ from a direct-integral decomposition of~$\mu$. 
For simplicity, however, we assume that the space we are acting on is compact. We consider only invariant measures, instead of quasi-invariant measures, so we do not have to keep track of Radon-Nikodym derivatives.

\begin{defns}
Suppose $C$ is a convex subset of a vector space~$V$.
	\begin{enumerate}
	\item A point $c \in C$ is an \defit{extreme point} of~$C$ if there do not exist $c_0,c_1 \in C \smallsetminus \{c\}$ and $t \in (0,1)$, such that $c = t c_0 + (1-t)c_1$.
	\item Let \nindex{$\ext C$ = $\{\text{extreme points of $C$}\}$}$\ext C$ be the set of extreme points of~$C$.
	\end{enumerate}
\end{defns}

\begin{eg}
Suppose $H$ acts continuously on a compact, separable metric space~$X$, and let 
	$$\Prob(X)^H = \{\, \mu \in \Prob(X) \mid \text{$\mu$ is $H$-invariant} \,\} . $$ 
This is a closed, convex subset of $\Prob(X)$, so $\Prob(X)^H$ is a compact, convex subset of $C(X)^*$ (with the weak$^*$ topology). The extreme points of this set are precisely the $H$-invariant probability measures that are ergodic \csee{Erg<>Extreme}.
\end{eg}

The well-known \thmindex{Krein-Milman}Krein-Milman Theorem states that every compact, convex set~$C$ is the closure of the convex hull of the set of extreme points of~$C$. (So, in particular, if $C$ is nonempty, then there exists an extreme point.) We will use the following strengthening of this fact:

\begin{thm}[{(\thmindex{Choquet's}Choquet's Theorem)}] \label{ChoquetsThm}
Suppose 
\noprelistbreak
	\begin{itemize}
	\item $\LocConvex$ is a locally convex topological vector space over~$\real$,
	\item $C$ is a metrizable, compact, convex subset of~$\LocConvex$, 
	and 
	\item $c_0 \in C$.
	\end{itemize}
Then there is a probability measure $\nu$ on $\ext C$, such that
	$$ c_0 = \int_{\ext C} x \, d\nu(x) .$$
\end{thm}

We will also need a corresponding uniqueness result.

\begin{defns}[(Choquet)]
Suppose $\LocConvex$ and $C$ are as in the statement of \cref{ChoquetsThm}. 
\noprelistbreak
	\begin{enumerate}
	\item Let \nindex{$\Sigma C = \{\, t c \mid t \in [0,\infty), \ c \in C \,\}$}%
	$\Sigma C = \{\, t c \mid t \in [0,\infty), \ c \in C \,\} \subseteq \LocConvex$.
	\item Define a partial order $\le$ on~$\Sigma C$ by $a \le b$ iff $b - a \in \Sigma C$.
	\item Two elements $a_1,a_2 \in \Sigma C$ have a \defit{least upper bound} if there exists $b \in \Sigma C$, such that
		\begin{itemize}
		\item $a_i \le b$ for $i = 1,2$,
		and
		\item for all $c \in \Sigma C$, such that $a_i \le c$ for $i = 1,2$, we have $b \le c$.
		\end{itemize}
	\end{enumerate}
\end{defns}

\begin{eg} \label{ProbUB}
Any two elements of $\Sigma \Prob(X)$ have a least upper bound \csee{ProbLUB}, so the same is true for $\Prob(X)^H$.
\end{eg}

\begin{thm}[(Choquet)]
Suppose $\LocConvex$, $C$, and $c_0$ are as in the statement of \cref{ChoquetsThm}. If every two elements of~$\Sigma C$ have a least upper bound, then the measure~$\nu$ provided by \cref{ChoquetsThm} is unique.
\end{thm}

\begin{proof}[The second proof of \cref{ErgodicDecomp}]
Assume, for simplicity, that $\mu$ is $H$-invariant, and that $X$ is compact. By normalizing, we may assume $\mu(X) = 1$, so $\mu \in \Prob(X)$. Then Choquet's Theorem \pref{ChoquetsThm} provides a probability measure~$\nu$, such that 
	$$ \mu = \int_{\ext \Prob(X)^H} \sigma \, d\nu(\sigma) .$$
By identifying $\ext \Prob(X)$ with a Borel subset of $[0,1]$, we may rewrite this as:
	$$ \mu = \int_{[0,1]} \mu_z \, d\nu(z) ,$$
where $\nu$ is a probability measure on $[0,1]$. Furthermore, \cref{Erg<>Extreme} tells us that each $\sigma \in \ext \Prob(X)^H$ is ergodic, so $\mu_z$ is an ergodic $H$-invariant measure for a.e.~$z$.

All that remains is to define a map $\zeta \colon X \to [0,1]$, such that $\mu_z$ is concentrated on $\zeta^{-1}(z)$.
For each Borel subset~$E$ of $\ext \Prob(X)^H$, let $\mu_E = \int_E \sigma \, d\nu(\sigma)$. Then $\mu_E$ is absolutely continuous with respect to~$\mu$, so we may write $\mu_E = f_E \mu$, for some measurable $f_E \colon X \to [0,\infty)$. Then 
	$$\psi(E) = \{\, x \in X \mid f_E(x) \neq 0 \,\}$$
is a well-defined element of $\Bool(X)$. Therefore, we have defined a map $\psi \colon \Bool \bigl( [0,1] \bigr) \to \Bool(X)$, and it can be verified that this is a homomorphism of $\sigma$-algebras. Hence, there is a measurable function $\zeta \colon X \to [0,1]$, such that $\psi(E) = \zeta^{-1}(E)$, for all~$E$ \csee{BoolPtwise}.
By using the uniqueness of~$\nu$, it can be shown that $\mu_z \bigl( \zeta^{-1}(z) \bigr) = 1$ for a.e.~$z$.
\end{proof}

\begin{exercises}

\item \label{sigma<>closed}
Let $\Bool$ be a (nonempty) subset of $\Bool(X)$ that is closed under complements and finite unions. Show that $\Bool$ is closed under countable unions (so $\Bool$ is a sub-$\sigma$-algebra of $\Bool(X)$) if and only if $\Bool$ is a closed set with respect to the topology determined by the metric on $\Bool(X)$.
\hint{($\Leftarrow$)~If $E = \bigcup_{i=1}^\infty E_i$, then $\bigcup_{i=1}^n E_i \to E$ in the topology on $\Bool(X)$.
\par
($\Rightarrow$)~If $d(E_i,E) < 2^{-i}$, then $E = \bigcap_{n = 1}^\infty \bigcup_{i=n}^\infty E_i$ (up to a set of measure~$0$.}

\item \label{DenseGensSigmaAlg}
Show that if $\mathcal{E}$ is dense in a sub-$\sigma$-algebra~$\Bool$ of $\Bool(X)$, then $\mathcal{E}$ is not contained in any proper sub-$\sigma$-algebra of~$\Bool$.
\hint{If $E_n \to E$, then $\bigcup_{n=1}^\infty (E_n \cap E) = E$ (up to a set of measure~$0$).}

\item \label{Erg<>Extreme}
Prove that a measure $\mu \in \Prob(X)^H$ is ergodic iff it is an extreme point.
\hint{If $E$ is an $H$-invariant set, then $\mu$ is a convex combination of the restrictions to $E$ and its complement.
Conversely, if $\mu = t\, \mu_1 + (1-t)\,\mu_2$, then the Radon-Nikodym Theorem implies $\mu_1 = f \, \mu$ for some ($H$-invariant) function~$f$.}

\item \label{ProbLUB}
Show that any two elements of $\Sigma \Prob(X)$ or $\Sigma \Prob(X)^*$ have a least upper bound.
\hint{Write $\nu = f \, \mu + \nu_s$ (uniquely), where $\nu_s$ is concentrated on a set of measure~$0$.}

\item \label{BoolPtwise}
Suppose $\psi \colon \Bool(Z) \to \Bool(X)$ is a function that respects complements and countable unions (and $\psi(\emptyset) = \emptyset$). Show there is a Borel function $\zeta \colon X \to Z$, such that $\psi(E) = \zeta^{-1}(E)$, for every Borel subset~$E$ of~$Z$.
\hint{Assume, for simplicity, that $Z = \bigl\{\, \sum a_k 3^{-k} \mid a_k \in \{0,1\} \, \bigr\} \subset [0,1]$. Then $\zeta = \sum \chi_{E_k} 3^{-k}$ for an appropriate collection $\{E_k\}$ of Borel subsets of~$X$.}

\end{exercises}




\section{Mixing} \label{MixingSect}

It is sometimes important to know that a product of group actions is ergodic. To discuss this issue (and related matters), let us fix some notation.

\begin{notation}
Throughout this \lcnamecref{MixingSect}, we assume:
	\begin{enumerate}
	\item $X_i$ is a Polish space, for every~$i$,
	\item $H$ is a Lie group that acts continuously on~$X_i$, for each~$i$,
	and
	\item $\mu_i$ is an $H$-invariant probability measure on~$X_i$, for each~$i$.
	\end{enumerate}
Furthermore, we use $X$ and~$\mu$ as abbreviations for $X_0$ and~$\mu_0$, respectively.
\end{notation}

\begin{defns} \ 
	\begin{enumerate}
	\item The \defit[product!action]{product action} of~$H$ on $X_1 \times X_2$ is the $H$-action defined by $h(x_1,x_2) = (hx_1, hx_2)$. The product measure $\mu_1 \times \mu_2$ is an invariant measure for this action.
	\item The action on~$X$ is said to be \defit[mixing!weak or weakly]{weak mixing} (or \defit[mixing!weak or weakly]{weakly mixing}) if the product action on $X \times X$ is ergodic.
	\end{enumerate}
\end{defns}

We have the following very useful characterizations of weakly mixing actions:

\begin{thm} \label{WMIff}
The action of~$H$ on~$X$ is weak mixing if and only if the \textup(one-dimensional\/\textup) space of constant functions is the only nontrivial, finite-dimensional, $H$-invariant subspace of $\LL2(X,\mu)$.
\end{thm}

\begin{proof}
We prove the contrapositive of each direction.

($\Rightarrow$)
Suppose $V$ is a nontrivial, finite-dimensional, $H$-invariant subspace of $\LL2(X)$. If the functions in~$V$ are not all constant (a.e.), then we may assume (by passing to a subspace) that $V \perp 1$. Choose an orthonormal basis $\{\varphi_1,\ldots,\varphi_r\}$ of~$V$, and define $\varphi \colon X \times X \to \complex$ by
	\begin{align} \label{WMIffPf-DefinePhi}
	 \varphi(x,y) = \sum_{i=1}^r \varphi_i(x) \, \overline{\varphi_i(y)} 
	 . \end{align}
Then $\varphi$ is an $H$-invariant function that is not constant \csee{WMIffPfDefinePhiEx}, so $H$ is not ergodic on $X \times X$.

($\Leftarrow$) Suppose $\varphi$ is a nonconstant, $H$-invariant, bounded function on $X \times X$. We may assume $\varphi(x,y) = \overline{\varphi(y,x)}$ by replacing $\varphi$ with either $\varphi(x,y) + \overline{\varphi(y,x)}$ or $\sqrt{-1} \, \bigl( \varphi(x,y) - \overline{\varphi(y,x)} \bigr)$. Therefore, we have a compact, self-adjoint operator on $\LL2(X,\mu)$, defined by
	$$ (T \psi)(x) = \int_X \varphi(x,y) \, \psi(y) \, d\mu(y) .$$
The Spectral Theorem \pref{SpectralThmCpct} implies $T$ has an eigenspace~$V$ that is finite-dimensional (and contains a nonconstant function). This eigenspace is $H$-invariant, since $T$~commutes with~$H$ (because $\varphi$ is $H$-invariant).
\end{proof}

We often have the following stronger condition:

\begin{defn} \label{StrongMixingDefn}
The action of $H$ on~$X$ is said to be \defit{mixing} (or, alternatively, \defit[mixing!strongly]{strongly mixing}) if $H$~is noncompact, and, for all $\varphi,\psi \in \LL2(X,\mu)$, such that $\varphi \perp 1$, we have
	$$ \lim_{h \to \infty} \langle h \varphi \mid \psi \rangle = 0 .$$
(We are using $1$ to denote the constant function of value~$1$, and, as usual, $(h \varphi)(x) = \varphi(h^{-1} x)$ \csee{RepL2}.)
\end{defn}

\begin{prop}[\csee{MixingIffEx}] \label{MixingIff}
If $H$ is not compact, then the following are equivalent:
	\begin{enumerate}
	\item \label{MixingIff-mixing}
	The action of $H$ on~$X$ is mixing.
	\item \label{MixingIff-funcs}
	For all $\varphi,\psi \in \LL2(X,\mu)$, we have
	$$ \lim_{h \to \infty} \langle h \varphi \mid \psi \rangle = \langle \varphi \mid 1 \rangle \, \langle 1 \mid \psi \rangle .$$
	\item \label{MixingIff-sets}
	For all Borel subsets $A$ and~$B$ of~$X$, we have
		$$ \lim_{h \to \infty} \mu(hA \cap B) = \mu(A) \, \mu(B) .$$ 
	\end{enumerate}
\end{prop}

\begin{rem}
Condition~\pref{MixingIff-sets} is the motivation for the choice of the term ``mixing:'' as $h \to \infty$, the space~$X$ is getting so stirred up (or well-mixed) that $hA$ is becoming uniformly distributed throughout the entire space.
\end{rem} 

When $G$ is simple, decay of matrix coefficients \pref{DecayMatCoeffSimple} implies that every action of~$G$ (with finite invariant measure) is mixing. In fact, we can say more.

\begin{defn}
Generalizing \cref{StrongMixingDefn}, we say that the action of~$H$ on~$X$ is \defit[mixing!of order~$r$]{mixing of order~$r$} if $H$ is not compact, and, for all Borel subsets $A_1,\ldots,A_r$ of~$X$, we have
	$$ \lim_{h_i^{-1} h_j \to \infty} \mu \left( \bigcap_{i=1}^r h_i A_i \right)
		= \mu(A_1) \, \mu(A_2) \, \cdots \, \mu(A_r) .$$
In particular, 
	\begin{itemize}
	\item every action of~$H$ is mixing of order~$1$ (if $H$ is not compact),
	and
	\item ``mixing'' is the same as ``mixing of order~$2$\zz.''
	\end{itemize}
\end{defn}

\begin{warn}
Some authors use a different numbering, for which this is ``mixing of order $r-1$\zz,'' instead of ``mixing of order $r$\zz.'' 
\end{warn}

Ledrappier constructed an action of $\integer^2$ that is mixing of order~$2$, but not of order~$3$. However, there are no such examples for semisimple groups:

\begin{thm} \label{GMixing}
Every mixing action of~$G$ \textup(with finite invariant measure\/\textup) is mixing of all orders.
\end{thm}

In the special case where $H = \integer$, we mention the following additional characterizations, some of which are weaker versions of \fullcref{MixingIff}{sets}:

\begin{thm} \label{WeakMixingZ}
If $H = \langle T \rangle$ is an infinite cyclic group, then the following are equivalent:
	\begin{enumerate}
	
	\item \label{WeakMixingZ-wm}
	$H$~is weak mixing on~$X$.

	\item \label{WeakMixingZ-eig}
	Every eigenfunction of~$T$ in $\LL2(X,\mu)$ is constant\/ \textup(a.e.\textup). That is, if $f \in \LL2(X,\mu)$, and there is some\/ $\lambda \in \complex$, such that $f(Tx) = \lambda \, f(x)$ a.e., then $f$~is constant\/ \textup(a.e.\textup).
			
	\item \label{WeakMixingZ-spectrum}
	The spectral measure of~$T$ has no atoms other than~$1$, and the eigenvalue~$1$ is simple\/ \textup(that is, the corresponding eigenspace is\/ $1$-dimensional\/\textup). 
%That is, if we let $\nu$ be the spectral measure of the unitary operator~$T$ on $\LL2(X,\mu)$ \csee{SpectralMeasDefn}, then $\nu(\{z\}) = 0$, for every singleton set~$\{z\}$ in~$\torus$.

	\item \label{WeakMixingZ-XxAny}
	The action of~$H$ on $X \times X_1$ is ergodic, whenever the action of~$H$ on~$X_1$ is ergodic.

%	\item \label{WeakMixingZ-IndepSets}
%The weakly independent sets are dense in $\Bool(X)$, where a set~$A$ is said to be \defit[weakly independent set]{\emph{weakly independent}} if there exist $n_1 < n_2 < \cdots$, such that 
%		$$ \text{$\mu \bigl( T^{n_i} A \cap T^{n_j} A \bigr) = \mu(A)^2$ whenever $i \neq j$} . $$
		
%	\item \label{WeakMixingZ-fulldensity}
%For any Borel subsets $A$ and~$B$ of~$X$, there is a subset $\mathcal{N}$ of full density in~$\integer^+$, such that 
%	$$ \mu(T^k A \cap B) \longrightarrow \mu(A) \, \mu(B)  
%	\qquad \text{as $k \to \infty$ with $k \in \mathcal{N}$}
%	.$$
%To say $\mathcal{N}$ has \defit[full density]{\emph{full density}} means 
%	$$ \lim_{n \to \infty} \frac{\# \bigl( \mathcal{N} \cap \{1,2,3,\ldots,n\} \bigr)}{n} = 1 .$$

	\item \label{WeakMixingZ-AbsVal}
For all Borel subsets $A$ and~$B$ of~$X$, 
		$$ \lim_{n \to \infty} \frac{1}{n} \sum_{k=1}^n 
			\bigl| \mu( T^k A \cap B) - \mu(A) \, \mu(B) \bigr| = 0 .$$

	\item \label{WeakMixingZ-fulldensitySameset}
%The quantifiers in \pref{WeakMixingZ-fulldensity} can be reversed:
There is a subset $\mathcal{N}$ of full density in~$\integer^+$, such that, for all Borel subsets $A$ and~$B$ of~$X$, we have
	$$ \mu(T^k A \cap B) \longrightarrow \mu(A) \, \mu(B)  
	\qquad \text{as $k \to \infty$ with $k \in \mathcal{N}$}
	.$$
To say $\mathcal{N}$ has \defit[full!density]{\emph{full density}} means 
	$$ \lim_{n \to \infty} \frac{\# \bigl( \mathcal{N} \cap \{1,2,3,\ldots,n\} \bigr)}{n} = 1 .$$
		
	\item \label{WeakMixingZ-multi}
If $A_0,A_1,\dots,A_r$ are any Borel subsets of~$X$, then there is a subset $\mathcal{N}$ of full density in~$\integer^+$, such that
		$$ \lim_{\begin{smallmatrix}k \to \infty \\ k \in \mathcal{N}\end{smallmatrix}} 
			\mu \bigl(A_0 \cap T^kA_1 \cap T^{2k} A_2 \cap \cdots \cap T^{rk} A_r \bigr)
		= \mu(A_0) \, \mu(A_1) \cdots \mu(A_r)
		%\quad \text{as $k \to \infty$ with $k \in \mathcal{N}$}
		.$$

	\end{enumerate}
\end{thm}

\begin{proof}[Sketch of proof]
($\ref{WeakMixingZ-wm} \Leftrightarrow \ref{WeakMixingZ-eig}$) By \cref{WMIff}, it suffices to observe that every finite-dimensional representation contains an irreducible subrepresentation, and that the irreducible representations of~$\integer$ (or, more generally, of any abelian group) are one-dimensional.

($\ref{WeakMixingZ-eig} \Leftrightarrow \ref{WeakMixingZ-spectrum}$)
These are two different ways of saying the same thing.

($\ref{WeakMixingZ-spectrum} \Rightarrow \ref{WeakMixingZ-XxAny}$)
Since $\LL2(X \times X_1) \iso \LL2(X) \otimes \LL2(X_1)$, the spectral measure~$\nu$ of $\LL2(X \times X_1)$ is the product $\nu_1 \times \nu_2$ of the spectral measures of $\LL2(X)$ and $\LL2(X_1)$. Therefore, any point mass in $\nu$ is obtained by pairing a point mass in~$\nu_1$ with a point mass in~$\nu_2$.

($\ref{WeakMixingZ-XxAny} \Rightarrow \ref{WeakMixingZ-wm}$)
Take $X_1 = X$.

($\ref{WeakMixingZ-wm} \Rightarrow \ref{WeakMixingZ-AbsVal}$)
For simplicity, let $a = \mu(A)$ and $b = \mu(B)$.
By \cref{MeanErgThmEx} (the Mean Ergodic Theorem), % @@@
ergodicity on~$X$ implies
	$$  \frac{1}{n} \sum_{k=1}^n \mu(T^k A \cap B) \stackrel{n \to \infty}{\longrightarrow} ab .$$
For the same reason, ergodicity on $X \times X$ implies
	\begin{align*}
	\sum_{k=1}^n (\mu \times \mu) \bigl (T^k A \times T^k A) \cap (B \times B) \bigr)
	\stackrel{n \to \infty}{\longrightarrow} 
	(\mu \times \mu)(A \times A) \cdot (\mu \times \mu) ( B \times B) 
	. \end{align*}
By simplifying both sides, we see that
	$$ \frac{1}{n} \sum_{k=1}^n \mu(T^k A \cap B)^2 
	\stackrel{n \to \infty}{\longrightarrow} 
	a^2 b^2 .$$
Therefore, simple algebra yields
	$$  \frac{1}{n} \sum_{k=1}^n \bigl| \mu(T^k A \cap B) - \mu(A) \, \mu(B) \bigr|^2
	\ \stackrel{n \to \infty}{\longrightarrow}  \ 
	a^2 \,b^2 - 2 (ab)(ab) + (ab)^2
	= 0
	. $$
\Cref{CesaroIff} implies that we have the same limit without squaring the absolute value.

($\ref{WeakMixingZ-AbsVal} \Rightarrow \ref{WeakMixingZ-eig}$)
Approximating by linear combinations of characteristic functions implies
	$$ \lim_{n \to \infty} \frac{1}{n} \sum_{k=1}^n \bigl| \langle T^k \varphi \mid \varphi \rangle  \bigr|
	= 0 
	\qquad \text{for all $\varphi \perp 1$} 
	. $$
However, if $\varphi$ is an eigenfunction for an eigenvalue $\neq 1$, then it is easy to see that the limit is nonzero (either directly, or by applying \cref{CesaroIff}).

($\ref{WeakMixingZ-AbsVal} \Leftrightarrow \ref{WeakMixingZ-fulldensitySameset}$)
\Cref{CesaroIff} implies that the two assertions are equivalent, up to reversing the order of the quantifiers in~\pref{WeakMixingZ-fulldensitySameset}. To reverse the quantifiers, note that a variant of Cantor diagonalization provides a set~$\mathcal{N}$ of full density that works for all $A$ and~$B$ in a countable dense subset of $\Bool(X)$.

 ($\ref{WeakMixingZ-multi} \Rightarrow \ref{WeakMixingZ-fulldensitySameset}$)
Take $r = 1$.

 ($\ref{WeakMixingZ-fulldensitySameset} \Rightarrow \ref{WeakMixingZ-multi}$)
The proof proceeds by induction on~$r$ (with \pref{WeakMixingZ-fulldensitySameset} as the starting point), and is nontrivial. We have no need for this result, so we omit the proof. % @@@
\end{proof}


\begin{exercises}

\item Show (directly from the definitions) that if the action of~$H$ on~$X$ is weak mixing, then it is ergodic.

\item \label{WMIffPfDefinePhiEx}
Let $\varphi \colon X \times X \to \complex$ be as in \pref{WMIffPf-DefinePhi} of the proof of \cref{WMIff}.
	\begin{enumerate}
	\item \label{WMIffPfDefinePhiEx-invt}
	Show $\varphi$ is $H$-invariant (a.e.).
	\item \label{WMIffPfDefinePhiEx-NotConst}
	Show $\varphi$ is not constant (a.e.). 
	\end{enumerate}
\hint{\pref{WMIffPfDefinePhiEx-invt}~Write $h \, \varphi_i = \sum_{i,j} h_{i,j} \varphi_j$, and observe that $[h_{i,j}]$ is a unitary matrix.
\\ 
\pref{WMIffPfDefinePhiEx-NotConst}~$\varphi(x,x) > 0$, but $\int \varphi(x,y) \, d\mu(y) = 0$.}

\item \label{MixingIffEx}
Prove \cref{MixingIff}.
\hint{($\ref{MixingIff-mixing} \Rightarrow \ref{MixingIff-funcs}$)
For $c = \langle \varphi \mid 1 \rangle$, we have $\langle (\varphi - c) \mid 1 \rangle = 0$. Now calculate $\lim_{h \to \infty} \langle h (\varphi - c) \mid \psi \rangle$ in two ways.
%		= \lim_{h \to \infty} \langle h \varphi \mid \psi \rangle 
%			- c \langle 1 \mid \psi \rangle .$$
\\ ($\ref{MixingIff-funcs} \Rightarrow \ref{MixingIff-sets}$)
Let $\varphi$ and~$\psi$ be the characteristic functions of $A$ and~$B$.
\\ ($\ref{MixingIff-sets} \Rightarrow \ref{MixingIff-funcs}$)
Approximate $\varphi$ and~$\psi$ by linear combinations of characteristic functions.}

\item \label{CesaroIff}
For every bounded sequence $\{a_k\} \subset [0,\infty)$, show
	$$ \lim_{n \to \infty} \frac{1}{n} \sum_{k=1}^n a_k = 0
	\ \Leftrightarrow \ 
	\text{$a_k \to 0$ as $k \to \infty$ in some set of full density.} $$
\hint{($\Rightarrow$)~For each $m > 0$, the set $A_m = \{\, k \mid a_k > 1/m\,\}$ has density~$0$, so there exists $N_m > N_{m-1}$, such that, for all $n \ge N_m$, we have
	$$N_{m-1} + \#\bigl( A_m \cap \{1,2,\ldots,n\} \bigr) < n/m .$$
Let $\mathcal{N}$ be the complement of $\bigcup_m \bigl( A_m \cap [N_m, N_{m+1}) \bigr)$.}

\end{exercises}


\begin{notes}

The focus of classical Ergodic Theory is on actions of $\integer$ and~$\real$ (or other abelian groups). A few of the many introductory books on this subject are \cite{EinsiedlerWard-ErgThyViewNumThy, Halmos, KatokHasselblatt-intro, Walters}. They include proofs of the Poincar\'e Recurrence Theorem \pref{PoincareRecurThm} and the Pointwise Ergodic Theorem \pref{PointwiseErgThm}.

Some basic results on the Ergodic Theory of noncommutative groups can be found in \cite[\S2.1]{ZimmerBook}.

The Moore Ergodicity Theorem \pref{MooreErgodicity} is due to C.\,C.\,Moore \cite{Moore-ergodicity}. 

See \cite{Lindenstrauss-Ptwise} for a very nice version of the Pointwise Ergodic Theorem that applies to all amenable groups (\cref{{PtwiseErgAmenRem}}).

See \cite[Thm.~1.1 (and Thm.~5.2)]{GreschonigSchmidt} for a proof of the ergodic decomposition \pref{ErgodicDecomp}, using Choquet's Theorem as in \cref{ErgDecompPfChoquet}.
\Cref{RokhlinDecompMeas} can be found in \cite[\S3]{Rohlin-FundIdeas}.
See \cite[\S5.5]{Srivastava-BorelSets} for a proof of \cref{vonNeumannSelectionThm}.
%A version of \cref{vonNeumannSelectionThm} appears in \cite[Thm.~3.4.3, p.~77]{Arveson-InvitationCstar}.

See \cite[Prop.~1, pp.~157--158]{Moore-ergodicity} for a proof of \cref{MixingIff}.

The standard texts on ergodic theory only prove \cref{MixingIff} for the special case $H = \integer$, but the same arguments apply in general.

\Cref{GMixing} is due to S.\,Mozes \cite{Mozes-MixingAllOrders}. Ledrappier's counterexample for $\integer^2$ is in 
\cite{Ledrappier-ChampMarkovien}.

\Cref{WeakMixingZ} is in the standard texts on ergodic theory, except for Part~\pref{WeakMixingZ-multi}, which is a ``\thmindex{multiple recurrence}multiple recurrence theorem'' that plays a key role in Furstenberg's proof of \thmindex{Szemeredi's}Szemeredi's theorem that there are arbitrarily long arithmetic progressions in every set of positive density in~$\integer^+$. For a proof of ($\ref{WeakMixingZ-fulldensitySameset} \Rightarrow \ref{WeakMixingZ-multi}$), see 
\cite[Prop.~7.13, p.~191]{EinsiedlerWard-ErgThyViewNumThy} 
or 
\cite[Thm.~4.10]{Furstenberg-RecErgThyCombNumThy}.

A proof of \cref{CesaroIff} is in \cite[Lem.~2.41, p.~54]{EinsiedlerWard-ErgThyViewNumThy}.
\end{notes}


\begin{references}{99}

%\bibitem{Arveson-InvitationCstar}
%  W.\,Arveson,
%  \emph{An Invitation to $C^*$-Algebras},
%  Springer, New York, 1976.
%  \MR{0512360}

\bibitem{EinsiedlerWard-ErgThyViewNumThy}
M.\,Einsiedler and T.\,Ward:
\emph{Ergodic Theory With a View Towards Number Theory}.
Springer, London, 2011. 
ISBN 978-0-85729-020-5,
\MR{2723325}

\bibitem{Furstenberg-RecErgThyCombNumThy}
H.\,Furstenberg:
\emph{Recurrence in Ergodic Theory and Combinatorial Number Theory}. 
Princeton University Press, Princeton, N.J., 1981.
ISBN 0-691-08269-3,
\MR{0603625}

\bibitem{GreschonigSchmidt}
  G.\,Greschonig and K.\,Schmidt,
  Ergodic decomposition of quasi-invariant probability measures,
  \emph{Colloq. Math.} 84/85 (2000), part~2, 495--514.
  \MR{1784210},
  \maynewline
  \url{http://eudml.org/doc/210829}
%  \url{http://pldml.icm.edu.pl/pldml/element/bwmeta1.element.bwnjournal-article-cmv84i2p495bwm}

\bibitem{Halmos}
 P.\,R.\,Halmos:
 \emphit{Lectures on Ergodic Theory.}
 Chelsea, New York, 1960.
 \MR{0111817}
% Zbl 0096.09004%,  Zbl 0073.09302

\bibitem{KatokHasselblatt-intro}
 A.\,Katok and B.\,Hasselblatt:
 \emphit{Introduction to the Modern Theory of Dynamical Systems.}
% With a supplementary chapter by Katok and Leonardo Mendoza.
% Encyclopedia of Mathematics and its Applications, 54. 
 Cambridge University Press, Cambridge, 1995.
 ISBN 0-521-34187-6,
 \MR{1326374}

\bibitem{Ledrappier-ChampMarkovien}
F.\,Ledrappier:
Un champ markovien peut être d'entropie nulle et mélangeant,
\emph{C. R. Acad. Sci. Paris Sér. A-B} 287 (1978), no.~7, A561--A563. 
\MR{0512106} % no URL available @@@

 \bibitem{Lindenstrauss-Ptwise}
E.\,Lindenstrauss:
Pointwise theorems for amenable groups,
\emph{Invent. Math.} 146 (2001), no.~2, 259--295. 
\MR{1865397},
\maynewline
\url{http://dx.doi.org/10.1007/s002220100162}

%\bibitem{MargulisBook}
% G.\,A.\,Margulis:
% \emph{Discrete Subgroups of Semisimple Lie Groups.}
% Springer, New York, 1991.
%ISBN 3-540-12179-X,
%\MR{1090825}

\bibitem{Moore-ergodicity}
C.\,C.\,Moore:
Ergodicity of flows on homogeneous spaces,
\emph{Amer. J. Math.} 88 1966 154--178.
\MR{0193188},
\maynewline
\url{http://www.jstor.org/stable/2373052}

\bibitem{Mozes-MixingAllOrders}
S.\,Mozes:
Mixing of all orders of Lie groups actions,
\emph{Invent. Math.} 107 (1992), no.~2, 235--241. 
(Erratum  119 (1995), no~ 2, 399.)
\MR{1144423}, \MR{1312506},
\maynewline
\url{http://eudml.org/doc/143965}, 
\url{http://eudml.org/doc/251560}
%\url{http://www.digizeitschriften.de/dms/resolveppn/?PPN=GDZPPN002109344},
%\newline
%\url{http://www.digizeitschriften.de/dms/resolveppn/?PPN=GDZPPN002112574}

\bibitem{Rohlin-FundIdeas}
 V.\,A.\,Rohlin:
 On the fundamental ideas of measure theory,
 \emph{Amer. Math. Soc. Transl.} (1) 10 (1962), 1--54.
  (Translated from \emph{Mat. Sbornik N.S.} 25(67), (1949), 107--150.)
 \MR{0047744}

\bibitem{Srivastava-BorelSets}
S.\,M.\,Srivastava:
\emph{A Course on Borel Sets}.
%Graduate Texts in Mathematics, 180. 
Springer, New York, 1998. 
ISBN 0-387-98412-7,
\MR{1619545}

\bibitem{Walters}
 P.\,Walters:
 \emphit{An Introduction to Ergodic Theory.}
% Graduate Texts in Mathematics, 79. 
 Springer, New York, 1982
 ISBN 0-387-90599-5,
 \MR{0648108}
 
% \bibitem{Zimmer-orbitspace}
% R.\,J.\,Zimmer:
% Orbit spaces of unitary representations, ergodic theory, and simple Lie groups,
% \emph{Ann. Math.} (2) 106 (1977), no. 3, 573--588.
% \MR{0466406}

\bibitem{ZimmerBook}
R.\,J.\,Zimmer:
\emph{Ergodic Theory and Semisimple Groups}.
%Monographs in Mathematics, 81. 
Birkh\"auser, Basel, 1984.
ISBN 3-7643-3184-4,
\MR{0776417}

 \end{references}



