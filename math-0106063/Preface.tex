%!TEX root = IntroArithGrps.tex

\refstepcounter{section}% to correct the PDF bookmarks

\chapter*{Preface}
\label{Preface}

This is a book about arithmetic subgroups of semisimple Lie groups, which means that we will discuss the group $\SL(n,\integer)$, and certain of its subgroups. By definition, the subject matter combines algebra (groups of matrices) with number theory (properties of the integers). However, it also has important applications in geometry. In particular, arithmetic groups arise in classical differential geometry as the fundamental groups of locally symmetric spaces. (See \cref{WhatisLocSymmChap,GeomIntroRank} for an elaboration of this line of motivation.) They also provide important examples and test cases in geometric group theory.

My intention in this text is to give a fairly gentle introduction to several of the main methods and theorems in the subject. There is no attempt to be encyclopedic, and proofs are usually only sketched, or only carried out for an illustrative special case. 
Readers with sufficient background will learn much more from \cite{MargulisBook} and \cite{PlatonovRapinchukBook} (written by the masters) than they can find here.

%The goal is to to acquaint graduate students and other non-experts with the background to understand seminars and papers that make use of some of the basic concepts, and perhaps even make use of some of these ideas themselves. Anyone needing a deeper understanding of the topics touched on here should consult the masters, but I hope this work will make those books less forbidding.
 
The book assumes knowledge of algebra, analysis, and topology that might be taught in a first-year graduate course, plus some acquaintance with Lie groups. (\cref{SSGrpsChap} quickly recounts the essential Lie theory, and \cref{BackChap} lists the required facts from graduate courses.) Some individual proofs and examples assume additional background (but may be skipped).

Generally speaking, the chapters are fairly independent of each other (and they all have their own bibliographies), so there is no need to read the book linearly.
To facilitate making a plan of study, the bottom of each chapter's first page states the main prerequisites that are not in \cref{SSGrpsChap,BackChap}.
Individual chapters (or, sometimes, sections) could be assigned for reading in a course or presented in a seminar. 
(The book has been released into the public domain, so feel free to make copies for such purposes.) Notes at the end of each chapter have suggestions for further reading. (Many of the subjects have been given book-length treatments.)
 Several topics (such as amenability and Kazhdan's property~$(T)$) are of interest well beyond the theory of arithmetic groups. 

Although this is a long book, some very important topics have been omitted. In particular, there is almost no discussion of the cohomology of arithmetic groups, even though it is a subject with a long history and continues to be a very active field. (See the lecture notes of Borel \cite{Borel-IntroCoho} for a recent survey.)
Also, there is no mention at all of automorphic forms. (Recent introductions to this subject include \cite{Deitmar-AutoForms} and \cite{PCMI-AutoForms}.)

Among the other books on arithmetic groups, the authoritative monographs of Margulis \cite{MargulisBook} and Platonov-Rapinchuk \cite{PlatonovRapinchukBook} have already been mentioned. They are essential references, but would be difficult reading for my intended audience. 
Some works at a level more comparable to this book include:
	\begin{itemize} \itemsep=\medskipamount
	\medskip

	\item[\cite{Borel-IntroGrpArith}] This classic gives an explanation of reduction theory (discussed here in \cref{ReductionChap}) and some of its important consequences.
	
	\item[\cite{Humphreys-ArithGrps}] This exposition covers reduction theory (at a more elementary level than \cite{Borel-IntroGrpArith}), adeles, ideles, and fundamentals of the Congruence Subgroup Property (mentioned here in \fullcref{MargNormSubgrpRems}{CSP}).
	
	\item[\cite{Ji-ArithGrpsWhatWhyHow}] This extensive survey touches on many more topics than are covered here (or even in \cite{MargulisBook} and \cite{PlatonovRapinchukBook}), with 60 pages of references.

	\item[\cite{MaclachlanReid-ArithHyp3Mflds}] This monograph thoroughly discusses arithmetic subgroups of the groups $\SL(2,\real)$ and $\SL(2,\complex)$.

	\item[\cite{RaghunathanBook}] This is an essential reference (along with \cite{MargulisBook} and \cite{PlatonovRapinchukBook}). It is the standard reference for basic properties of lattices in Lie groups (covered here in \cref{BasicLatticesChap}). It also has proofs of the Godement Criterion (discussed here in \cref{GodementSect}), the existence of both cocompact and noncocompact arithmetic subgroups (discussed here in \cref{CocptSect}), and reduction theory for arithmetic groups of $\rational$-rank one (discussed here in \cref{ReductionChap}). It also includes several topics not covered here, such as cohomology vanishing theorems, and lattices in non-semisimple Lie groups.

	\item[\cite{Sury-CSP}] This textbook provides an elementary introduction to the Congruence Subgroup Property (which has only a brief mention here in \fullcref{MargNormSubgrpRems}{CSP}).

	\item[\cite{ZimmerBook}] After developing the necessary prerequisites in ergodic theory and representation theory, this monograph provides proofs of three major theorems of Margulis: Superrigidity, Arithmeticity, and Normal Subgroups (discussed here in \cref{MargulisSuperChap,NormalSubgroupChap}). It also proves a generalization of the superrigidity theorem that applies to ``Borel cocycles\zz.''
	
	\end{itemize}

\medskip

\hfill \textit{Dave Morris}

 %\hfill \textit{Lethbridge, Alberta, Canada}

 \hfill \textit{April 2015} % !!!




 \newpage

\putintocfalse % do NOT want these references in the Table of Contents
\makemarksfalse % do not want these references to change the funning head

\begin{references}{MR} \itemsep=\bigskipamount
\renewcommand{\\}{\newline}
\global\makemarkstrue

\medskip

%\bibitem[Ben]{Benoist-5Lectures}
%Y.\,Benoist:
%Five lectures on lattices in semisimple Lie groups, 
%in L.\,Bessières, et al., eds.:
%\emph{Géométries à courbure négative ou nulle, groupes discrets et rigidités}.
%%Sémin. Congr., 18, 
%Soc. Math. France, Paris, 2009, pp.~117--176.
%\MR{2655311}


\bibitem[B1]{Borel-IntroGrpArith}
A.\,Borel:
\emph{Introduction aux Groupes Arithm\'etiques}.
\\
%Publications de l'Institut de MathŽmatique de l'UniversitŽ de Strasbourg, XV. ActualitŽs Scientifiques et Industrielles, No. 1341 
 Hermann, Paris, 1969.
\MR{0244260}

\bibitem[B2]{Borel-IntroCoho}
A.\,Borel:
Introduction to the cohomology of arithmetic groups,
\\ in L.\,Ji et al., eds.:
\emph{Lie Groups and Automorphic Forms}. 
%AMS/IP Stud. Adv. Math., 37, 
\\ American Mathematical Society, Providence, RI, 2006,
pp.~51--86.
\\ ISBN 978-0-8218-4198-3,
\MR{2272919}


\bibitem[De]{Deitmar-AutoForms}
A.\,Deitmar:
\emph{Automorphic Forms.}
\\ Springer, London, 2013.
\quad ISBN 978-1-4471-4434-2,
\MR{2977413}

\bibitem[Hu]{Humphreys-ArithGrps}
J.\,E.\,Humphreys:
\emph{Arithmetic Groups}.
%Lecture Notes in Mathematics, 789. 
\\ Springer, Berlin, 1980.
\quad ISBN 3-540-09972-7,
\MR{0584623}

\bibitem[Ji]{Ji-ArithGrpsWhatWhyHow}
L.\,Ji:
\emph{Arithmetic Groups and Their Generalizations}. % What, Why, and How.}
%AMS/IP Studies in Advanced Mathematics, 43. 
\\ American Mathematical Society, Providence, RI,  2008. % International Press, Cambridge, MA, 2008.
\\ ISBN 978-0-8218-4675-9,
\MR{2410298}

\bibitem[MR]{MaclachlanReid-ArithHyp3Mflds}
C.\,Maclachlan \& A.\,Reid:
\emph{The Arithmetic of Hyperbolic 3-Manifolds}.%
%Graduate Texts in Mathematics, 219. 
\\Springer, New York, 2003. 
\quad ISBN 0-387-98386-4,
\MR{1937957}

\bibitem[Ma]{MargulisBook}
 G.\,A.\,Margulis:
 \emph{Discrete Subgroups of Semisimple Lie Groups.}
\\ Springer, {Berlin}, 1991.
\quad ISBN 3-540-12179-X,
\MR{1090825}


\bibitem[PR]{PlatonovRapinchukBook}
 V.\,Platonov \& A.\,Rapinchuk: 
 \emph{Algebraic Groups and Number Theory}.%
 \\ Academic Press, Boston, 1994.
 \quad ISBN 0-12-558180-7,
 \MR{1278263}

\bibitem[Ra]{RaghunathanBook}
 M.\,S.\,Raghunathan: 
 \emph{Discrete Subgroups of Lie Groups.}
 \\ Springer, {New York}, 1972.
\quad ISBN 0-387-05749-8,
\MR{0507234}

\bibitem[SS]{PCMI-AutoForms}
P.\,Sarnak \& F.\,Shahidi, eds.:
\emph{Automorphic Forms and Applications}.%
%Lecture notes from the IAS/Park City Mathematics Institute held in (Park City, UT, July 1--20, 2002. Edited by . IAS/Park City Mathematics Series, 12. 
\\American Mathematical Society, Providence, RI, 
	%; Institute for Advanced Study (IAS), Princeton, NJ, 
2007. 
\\ ISBN 978-0-8218-2873-1,
\MR{2331351}

%\bibitem[Sou]{Soule-IntroArithGrps}
%C.\,Soulé:
%An introduction to arithmetic groups,
%in P.\,Cartier et al., eds.:
%\emph{Frontiers in number theory, physics, and geometry~II
%%On conformal field theories, discrete groups and renormalization. Papers from the meeting held in 
%(Les Houches, March 9--21, 2003)}. 
%%Edited by Pierre Cartier, Bernard Julia, Pierre Moussa and Pierre Vanhove. 
%Springer, Berlin, 2007, pp.~247--276. 
%ISBN 978-3-540-30307-7,
%\MR{2290763}

\bibitem[Su]{Sury-CSP}
B.\,Sury:
\emph{The Congruence Subgroup Problem}. 
	%An Elementary Approach Aimed at Applications.}
% Texts and Readings in Mathematics, 24. 
\\Hindustan Book Agency, New Delhi, 2003. 
\\ ISBN 81-85931-38-0,
\MR{1978430}

\bibitem[Zi]{ZimmerBook}
 R.\,J.\,Zimmer:
 \emph{Ergodic Theory and Semisimple Groups}.
 \\ Birkh\"auser, Boston, 1984.
\quad ISBN 3-7643-3184-4,
 \MR{0776417}

\end{references}

\vfil
 \eject

