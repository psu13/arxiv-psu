%!TEX root = IntroArithGrps.tex


\mychapter{Margulis Superrigidity\texorpdfstring{\\}{ }Theorem}
\label{MargulisSuperChap}

\prereqs{none, except that the proof of the main theorem requires real rank (\cref{RrankChap}), amenability (Furstenberg's Lemma \pref{G/amen->Meas(X)}), and the Moore Ergodicity Theorem (\cref{MooreErgPfSect}).}


Roughly speaking, the Margulis Superrigidity Theorem tells us that homomorphisms defined on~$\Gamma$ can be extended to be defined on all of~$G$ (unless $G$ is either $\SO(1,n)$ or $\SU(1,n)$). 
In cases where it applies, this fundamental theorem is much stronger than the Mostow Rigidity Theorem \pref{MostowRigidity}.
It also implies the Margulis Arithmeticity Theorem (\ref{MargulisArith} or~\ref{MargArithFromSuper}).


\section{Statement of the theorem} \label{MargSuperStatementSect}

It is not difficult to see that every group homomorphism from~$\integer^k$ to~$\real^n$ can be extended to a continuous homomorphism from~$\real^k$ to~$\real^n$ \csee{ZkSuperrigEx}. Noting that $\integer^k$ is a lattice in~$\real^k$, it is natural to hope that, analogously, homomorphisms defined on~$\Gamma$ can be extended to be defined on all of~$G$. The Margulis Superrigidity Theorem shows this is true if $G$ has no simple factors isomorphic to $\SO(1,m)$ or $\SU(1,m)$, except that the conclusion may only be true modulo finite groups and up to a bounded error. Here is an illustrative special case that is easy to state, because the bounded error does not arise. 

\begin{thm}[(Margulis)] \label{MargSuperSL3RNonCpct}
Assume
\noprelistbreak
	\begin{itemize}
	\item $G = \SL(k,\real)$, with $k \ge 3$,
	\item $G/\Gamma$ is \textbf{not} compact,
	and
	\item $\varphi \colon \Gamma \to \GL(n,\real)$ is any homomorphism.
	\end{itemize}
Then there exist:
\noprelistbreak
	\begin{itemize}
	\item a continuous homomorphism $\widehat\varphi \colon G \to \GL(n,\real)$, 
	and
	\item a finite-index subgroup~$\Gamma'$ of~$\Gamma$,
	\end{itemize}
such that  $\widehat\varphi(\gamma) = \varphi(\gamma)$ for all $\gamma \in \Gamma'$.
\end{thm}

\begin{proof}
See \cref{MargSuperNonCpctSCEx}.
\end{proof}

Here is a much more general version of the theorem that has a slightly weaker conclusion. To simplify the statement, we preface it with a definition. 

\begin{defn} \label{AlgSCDefn}
$G$ is \defit[simply connected, algebraically]{algebraically simply connected} if (for every~$\ell$\,) every Lie algebra homomorphism $\Lie G \to \LieSL(\ell,\real)$ is the derivative of a well-defined Lie group homomorphism $G \to \SL(\ell,\real)$.
\end{defn}

\begin{rem}
Every simply connected Lie group is algebraically simply connected, but the converse is not true \csee{SLnASC}. In general, if $G$ is connected, then some finite cover of~$G$ is algebraically simply connected. Therefore, assuming that $G$ is algebraically simply connected is just a minor technical assumption that avoids the need to pass to a finite cover. 
\end{rem}

\begin{thm}[(Margulis Superrigidity Theorem)] \label{MargSuperC}
Assume
\noprelistbreak
	\begin{enumerate} \renewcommand{\theenumi}{\roman{enumi}}
	\item $G$ is connected, and algebraically simply connected,
	\item \label{MargSuperC-notSOSU}
	$G$ is not isogenous to any group that is of the form\/ $\SO(1,m) \times K$ or\/ $\SU(1,m) \times K$, where $K$~is compact,
	\item $\Gamma$ is irreducible,
	and
	\item $\varphi \colon \Gamma \to \GL(n,\real)$ is a homomorphism.
	\end{enumerate}
Then there exist:
\noprelistbreak
	\begin{enumerate}
	\item a continuous homomorphism $\widehat\varphi \colon G \to \GL(n,\real)$,
	\item a compact subgroup~$C$ of\/ $\GL(n,\real)$ that centralizes $\widehat\varphi(G)$,
	and
	\item a finite-index subgroup\/~$\Gamma'$ of\/~$\Gamma$,
	\end{enumerate}
such that $\varphi(\gamma) \in \widehat\varphi(\gamma) \, C$, for all $\gamma \in \Gamma'$.
\end{thm}

\begin{proof}
See \cref{SuperPfSect}.
\end{proof}

\begin{rems} \label{SuperRem} \ 
\noprelistbreak
	\begin{enumerate}
	\item Since $\varphi(\gamma) \in \widehat\varphi(\gamma) \, C$, we have $\widehat\varphi(\gamma)^{-1} \, \varphi(\gamma) \in C$ for all~$\gamma$. Therefore, although $\widehat\varphi(\gamma)$ might not be exactly equal to $\varphi(\gamma)$, the error is an element of~$C$, which is a bounded set (because $C$ is compact). Hence, the size of the error is uniformly bounded on all of~$\Gamma'$.
 
	\item \label{SuperRem-NotSOSU}
	Assumption~\pref{MargSuperC-notSOSU} cannot be removed. For example, if $G = \PSL(2,\real)$, then the lattice $\Gamma$ can be a free group \csee{FreeLattINSL2R}. In this case, there exist many, many homomorphisms from~$\Gamma$ into any group~$G'$, and many of them will not extend to~$G$ \csee{FreeNotSuperrig}.

	\end{enumerate}
 \end{rems}

If we make an appropriate assumption on the range of~$\varphi$, then there is no need for the compact error term~$C$ or the finite-index subgroup~$\Gamma'$:

\begin{cor} \label{MargSuperG'}
Assume
\noprelistbreak
	\begin{enumerate} \renewcommand{\theenumi}{\roman{enumi}}
	\item \label{MargSuperG'-notSOSU}
	$G$ is not isogenous to any group that is of the form\/ $\SO(1,m) \times K$ or\/ $\SU(1,m) \times K$, where $K$~is compact,
	\item  \label{MargSuperG'-irred}
 $\Gamma$ is irreducible, 
 and
	\item  \label{MargSuperG'-G}
 $G$ and $G'$ are connected, with trivial center, and no compact factors.
	\end{enumerate}
If $\varphi \colon \Gamma \to G'$ is any homomorphism, such that $\varphi(\Gamma)$ is Zariski dense in~$G'$, then $\varphi$ extends to a continuous homomorphism $\widehat\varphi \colon G \to G'$.
\end{cor}

\begin{proof}
See \cref{MargSuperG'PfEx}.
\end{proof}
 
Because of our standing assumption \pref{standassump} that $G'$ is semisimple, \cref{MargSuperG'} implicitly assumes that the Zariski closure $\Zar{\varphi(\Gamma)} = G'$ is semisimple. In fact, that is automatically the case:
  
 \begin{cor} \label{MargImgSS}
 Assume
 \noprelistbreak
 	\begin{enumerate} \renewcommand{\theenumi}{\roman{enumi}}
	\item  \label{MargImgSS-notSOSU}
	$G$ is not isogenous to any group that is of the form\/ $\SO(1,m) \times K$ or\/ $\SU(1,m) \times K$, where $K$~is compact,
	\item  \label{MargImgSS-irred}
	 $\Gamma$ is irreducible,
	 and
 	\item \label{MargImgSS-phi}
	$\varphi \colon \Gamma \to \GL(n,\real)$ is a homomorphism.
	\end{enumerate}
Then\/ $\Zar{\varphi(\Gamma)}$ is semisimple.
\end{cor}

\begin{proof}
See \cref{MargImgSSPfEx}.
\end{proof}



 
\begin{exercises}

\item \label{ZkSuperrigEx}
Suppose $\varphi$ is a homomorphism from~$\integer^k$ to~$\real^n$. Show that $\varphi$ extends to a continuous homomorphism from~$\real^k$ to~$\real^n$.
\hint{Let $\widehat\varphi \colon \real^k \to \real^n$ be a linear transformation, such that $\widehat\varphi(\varepsilon_i) = \varphi(\varepsilon_i)$, where $\{\varepsilon_1,\ldots,\varepsilon_k\}$ is the standard basis of~$\real^k$.}

\item \label{SLnASC}
Show that $\SL(n,\real)$ is algebraically simply connected. (On the other hand, $\SL(n,\real)$ is not simply connected, because its fundamental group is nontrivial.)
\hint{By tensoring with~$\complex$, any homomorphism $\LieSL(n,\real) \to \LieSL(\ell,\real)$ extends to a homomorphism $\LieSL(n,\complex) \to \LieSL(\ell,\complex)$, and $\SL(n,\complex)$ is simply connected.}

\item \label{GNoAbelianization}
Assume 
	\begin{itemize}
	\item $G$ is not isogenous to $\SO(1,n)$ or $\SU(1,n)$, for any~$n$,
	\item $\Gamma$ is irreducible,
	and
	\item $G$ has no compact factors.
	\end{itemize}
Use the Margulis Superrigidity Theorem to show that the abelianization $\Gamma / [\Gamma,\Gamma]$ of~$\Gamma$ is finite.
(When $G$ is simple, this was already proved from Kazhdan's property~$(T)$ in \fullcref{KazhdanlatticeCor}{noabel}. We will see yet another proof in \cref{GammaHasFiniteAbelianization}.)

\item Assume $G$, $\Gamma$, $\varphi$, $\widehat\varphi$, $C$, and~$\Gamma'$ are as in \cref{MargSuperC}. Show there is a homomorphism $\epsilon \colon \Gamma' \to C$, such that $\varphi(\gamma) = \widehat\varphi(\gamma) \cdot \epsilon(\gamma)$, for all $\gamma \in \Gamma'$.

\item \label{FreeNotSuperrig}
Suppose $G = \PSL(2,\real)$ and $\Gamma$ is a free group.
Construct a homomorphism $\varphi \colon \Gamma \to \GL(n,\real)$ (for some~$n$), such that, for every continuous homomorphism $\widehat\varphi \colon G \to \GL(n,\real)$, and every finite-index subgroup~$\Gamma'$ of~$\Gamma$, the set $\{\, \widehat\varphi(\gamma)^{-1} \, \varphi(\gamma) \mid \gamma \in \Gamma' \,\}$ is not precompact.
\hint{$\varphi$ may have an infinite kernel.}

\item \label{MargSuperG'PfEx}
Prove \cref{MargSuperG'} from \cref{MargSuperC}.

\item %Assume $G$ has no compact factors. 
Show that the extension~$\widehat\varphi$ in \cref{MargSuperG'} is unique.
\hint{Borel Density Theorem.}

\item \label{MargSuperG'NeedEx}
In each case, find
	\begin{itemize}
	\item a lattice~$\Gamma$ in~$G$
	and
	\item a homomorphism $\varphi \colon \Gamma \to G'$,
	\end{itemize}
such that 
	\begin{itemize}
	\item $\varphi(\Gamma)$ is Zariski dense in~$G'$,
	and
	\item $\varphi$ does not extend to a continuous homomorphism $\widehat\varphi \colon G \to G'$.
	\end{itemize}
Also explain why they are not counterexamples to \cref{MargSuperG'}.
	\begin{enumerate}
	\item $G = G' =  \PSL(2,\real) \times \PSL(2,\real)$.
	\item $G = \PSL(4,\real)$ and $G' = \SL(4,\real)$.
	\item $G = \SO(2,3)$ and $G' = \SO(2,3) \times \SO(5)$.
	\end{enumerate}

\item \label{MargImgSSPfEx}
Prove \cref{MargImgSS} from \cref{MargSuperC}.

\item \label{MargSuperFromG'PfEx}
Derive \cref{MargSuperC} from the combination of \cref{MargSuperG'} and \cref{MargImgSS}. (This is a converse to \cref{MargSuperG'PfEx,MargImgSSPfEx}.)

%\item Suppose the homomorphism $\varphi \colon \Gamma \to \SL(n,\real)$ extends to a continuous homomorphism $\widehat\varphi \colon G \to \SL(n,\real)$. Show that the almost-Zariski closure $\Zar{\varphi(\Gamma)}$ is connected and semisimple.

%\item \label{MargSuperNoncpctEx}
%Show that if 
%\noprelistbreak
%	\begin{itemize}
%	\item $\Rrank G \ge 2$,
%	\item $G/\Gamma$ is not compact,
%	\item $\varphi$ is any homomorphism from~$\Gamma$ to $\SL(n,\real)$,
%	and
%	\item $G'$ is the identity component of $\Zar{\varphi(\Gamma)}$,
%	\end{itemize}
%then there exist:
%\noprelistbreak
%	\begin{enumerate}
%	\item a finite-index subgroup~$\Gamma'$ of~$\Gamma$,
%	\item a finite subgroup~$Z$ of the center of~$G'$,
%	and
%	\item a homomorphism $\widehat \varphi \colon G \to G'/Z$,
%	\end{enumerate}
%such that $\varphi(\gamma) Z = \widehat\varphi(\gamma)$, for all $\gamma \in \Gamma'$. 

\end{exercises}





\section{Applications}

We briefly describe a few important consequences of the Margulis Superrigidity Theorem.

\subsection{Mostow Rigidity Theorem}
\label{StateMostowSubsect}
\thmindex{Mostow Rigidity}

The special case of the Margulis Superrigidity Theorem \pref{MargSuperG'} in which the homomorphism~$\varphi$ is assumed to be an isomorphism onto a lattice~$\Gamma'$ in~$G'$ is very important:

\begin{thm}[(\thmindex{Mostow Rigidity Theorem}Mostow Rigidity Theorem, cf.\  \pref{MostowRigidity})] \label{MostowRigidityIrred}
Assume
\noprelistbreak
	\begin{itemize}
	\item $G_1$ and~$G_2$ are connected, with trivial center and no compact factors,
	\item $G_1 \not\iso \PSL(2,\real)$,
	\item $\Gamma_i$ is an irreducible lattice in~$G_i$, for $i = 1,2$,
	and 
	\item $\varphi \colon \Gamma_1 \to \Gamma_2$ is a group isomorphism.
	\end{itemize}
Then $\varphi$ extends to a continuous isomorphism from~$G_1$ to~$G_2$.
\end{thm}

This theorem has already been discussed in \cref{MostowChap}.
In most cases, it follows easily from the Margulis Superrigidity Theorem \csee{ProveMostMostowEx}. However, since the superrigidity theorem does not apply when $G_1$ is either $\SO(1,m)$ or $\SU(1,m)$, a different argument is needed for those cases; see \cref{MostowPfSect} for a sketch of the proof.


\subsection{Triviality of flat vector bundles over $G/\Gamma$}

\begin{defn} \label{FlatVecBundleDefn}
For any homomorphism $\varphi \colon \Gamma \to \GL(n,\real)$, there is a diagonal action of~$\Gamma$ on $G \times \real^n$, defined by 
	$$ (x,v) \cdot \gamma = \bigl( x \gamma, \varphi(\gamma^{-1}) v \bigr) . $$
Let $\bundle_\varphi = (G \times \real^n) / \Gamma$ 
	\nindex{$\bundle_\varphi$ = flat vector bundle over $G/\Gamma$}
be the space of orbits of this action. Then there is a well-defined map 
	$$ \text{$\pi \colon \bundle_\varphi \to G/ \Gamma$, defined by $\pi \bigl( [x,v] \bigr) = x \Gamma$,} $$
and this makes $\bundle_\varphi$ into a vector bundle over $G/\Gamma$ (with fiber~$\real^n$) \csee{FlatVecBundleDefnEx}.
A vector bundle defined from a homomorphism in this way is said to be a \defit[flat!vector bundle]{flat vector bundle}.
\end{defn}

The Margulis Superrigidity Theorem implies (in some cases) that every flat vector bundle over $G/\Gamma$ is nearly trivial. Here is an example:

\begin{prop} \label{Super->VecBdlTrivial}
Let $G = \SL(n,\real)$ and\/ $\Gamma = \SL(n,\integer)$. If $\bundle_\varphi$ is any flat vector bundle over\/ $G/\Gamma$, then there is a finite-index subgroup\/~$\Gamma'$ of\/~$\Gamma$, such that the lift of $\bundle_\varphi$ to the finite cover\/ $G/\Gamma'$ is trivial.

In other words, if we let $\varphi'$ be the restriction of~$\varphi$ to\/~$\Gamma'$, then the vector bundle $\bundle_{\varphi'}$ is isomorphic to the trivial vector bundle\/ $(G/\Gamma') \times \real^n$.
\end{prop}

\begin{proof}
From \cref{MargSuperSL3RNonCpct}, we may choose $\Gamma'$ so that the restriction~$\varphi'$ extends to a homomorphism $\widehat\varphi \colon G \to \GL(n,\real)$. Define a continuous function $T \colon G \times \real^n \to G \times \real^n$ by
	$$ T(g,v) = \bigl( g, \widehat\varphi(g) v \bigr) .$$
Then, for any $\gamma \in \Gamma'$, a straightforward calculation shows
	\begin{align} \label{FlatBundleTequi}
	 T \bigl(\, (g,v) \cdot \gamma \, \bigr) 
	 &= T(g,v) * \gamma
	 , \text{ where $(g,v) * \gamma = (g \gamma, v)$}
	 \end{align}
%	\begin{align*}
%	 T \bigl(\, (g,v) \cdot \gamma \, \bigr) 
%	 &= T \bigl(\, g \gamma, \varphi(\gamma^{-1}) v \, \bigr) 
%	 =  \bigl(\, g \gamma, \hat\varphi( g \gamma)   \hat\varphi(\gamma^{-1}) v \, \bigr) 
%	 \\&= \bigl(\,g \gamma, \hat\varphi( g \gamma\gamma^{-1}) v \, \bigr) 
%	 = \bigl(\,g \gamma, \hat\varphi( g)  v \, \bigr) 
%	 \\&=  \bigl(\, g, \hat\varphi( g)  v \, \bigr) * \gamma
%	 = T(g,v) * \gamma
%	\end{align*}
\csee{FlatBundleTequiEx}.
Therefore $T$ factors through to a well-defined bundle isomorphism $\bundle_{\varphi'} \stackrel{\cong}{\to} (G/\Gamma') \times \real^n$.
\end{proof}



 
\subsection{Embeddings of locally symmetric spaces from embeddings of lattices} \label{TotGeodSect}

Let $M = \Gamma \backslash G / K$ and $M' = \Gamma ' \backslash G' \! / K'$.
Roughly speaking, the Mostow Rigidity Theorem \pref{MostowRigidity} tells us that if $\Gamma$ is isomorphic to~$\Gamma'$, then $M$ is isometric to~$M'$.  More generally, superrigidity implies that if $\Gamma$ is isomorphic to a subgroup of~$\Gamma'$, then $M$ is isometric to a submanifold of~$M'$ (modulo finite covers).

\begin{prop} \label{TotGeodProp}
Suppose 
	\begin{itemize}
	\item $M = \Gamma \backslash G / K$ and $M' = \Gamma ' \backslash G' \! / K'$ are irreducible locally symmetric spaces with no compact factors,
	\item $\Gamma$ is isomorphic to a subgroup of\/~$\Gamma'$,
	and
	\item the universal cover of~$M$ is neither the real hyperbolic space~$\hyperbolic^n$ nor the complex hyperbolic space~$\complex\hyperbolic^n$.
	\end{itemize}
Then some finite cover of\/ $\Gamma \backslash G / K$ embeds as a totally geodesic submanifold of a finite cover of\/ $\Gamma' \backslash G' \! / K'$.
\end{prop}

\begin{proof}[Idea of proof]
There is no harm in assuming that $G$ and~$G'$ have trivial center and no compact factors.
After passing to a finite-index subgroup of~$\Gamma$ 
	(and ignoring a compact group~$C$),
the Margulis Superrigidity Theorem tells us that the embedding $\Gamma \hookrightarrow \Gamma'$ extends to a continuous embedding $\varphi \colon G \hookrightarrow G'$. Conjugate $\varphi$ by an element of~$G'$, so that $\varphi(K) \subseteq K'$, and $\varphi(G)$ is invariant under the Cartan involution of~$G'$ corresponding to the maximal compact subgroup~$K'$. Then $\varphi$ induces an embedding $\Gamma \backslash G / K \to \Gamma ' \backslash G' \! / K'$ whose image is a totally geodesic submanifold.
\end{proof}


\begin{exercises}

\item \label{ProveMostMostowEx}
Prove the Mostow Rigidity Theorem \pref{MostowRigidityIrred} under the additional assumption that $G_1$ is neither $\PSO(1,n)$ nor $\PSU(1,n)$.

\item The statement of the Mostow Rigidity Theorem in \cref{MostowRigidity} is slightly different from \cref{MostowRigidityIrred}. Show that these two theorems are corollaries of each other.
\hint{\Cref{MostowRigIrredEnough}.}

\item \label{FlatVecBundleDefnEx}
In the notation of \cref{FlatVecBundleDefn}:
\noprelistbreak
	\begin{enumerate}
	\item Show the map~$\pi$ is well defined.
	\item Show $\bundle_\varphi$ is a vector bundle over $G/\Gamma$ with fiber $\real^n$.
	\end{enumerate}

\item \label{FlatBundleTequiEx}
Verify \pref{FlatBundleTequi} for all $\gamma \in \Gamma'$.

\end{exercises}






\section{Why superrigidity implies arithmeticity}
 \label{MargArithPf}

Recall the following major theorem that was stated without proof in \cref{MargulisArith}:

\begin{namedthm}[\thmindex{Margulis!Arithmeticity}{Margulis Arithmeticity Theorem}]
\label{MargArithFromSuper}
Every irreducible lattice in~$G$ is arithmetic, except, perhaps, when $G$ is isogenous to\/ $\SO(1,m) \times K$ or\/ $\SU(1,m) \times K$, for some compact group~$K$.
\end{namedthm}

This important fact is a consequence of the Margulis Superrigidity Theorem, but the implication is not at all obvious.  In this section, we will explain the main ideas that are involved.

In addition to our usual assumption that $G \subseteq \SL(\ell,\real)$, let us also assume, for simplicity:
\noprelistbreak
 \begin{itemize}
% \item $G \subseteq \SL(\ell,\real)$, for some~$\ell$, 
 \item $G \iso \SL(3,\real)$ (or, more generally, $G$ is algebraically simply connected; see \cref{AlgSCDefn}),
 and 
 \item $G / \Gamma$ is not compact. 
 \end{itemize}
 We wish to show that $\Gamma$ is arithmetic. It suffices to
show $\Gamma  \subseteq G_{\integer}$, that is, that every
matrix entry of every element of~$\Gamma$ is an integer, for
then $\Gamma$ is commensurable to $G_{\integer}$
\csee{finext->latt}.

Here is a loose description of the 4 steps of the proof:
	\begin{enumerate}
	\item The Margulis Superrigidity Theorem \pref{MargSuperSL3RNonCpct} implies that every matrix entry of every element of~$\Gamma$ is an algebraic number.
	\item By \term[Restriction of Scalars]{restriction of scalars}, we may assume that these algebraic numbers are rational; that is, $\Gamma \subseteq G_{\rational}$.
	\item For every prime~$p$, a ``$p$-adic'' version of the 
	\thmindex{Margulis!Superrigidity!p-adic@$p$-adic}Margulis Superrigidity Theorem 
	provides a natural number~$N_p$, such that no element of~$\Gamma$ has a matrix entry whose denominator is divisible by~$p^{N_p}$. 
	\item This implies that some finite-index subgroup~$\Gamma'$ of~$\Gamma$ is contained in~$G_{\integer}$.
	\end{enumerate}

\setcounter{step}{0}

\begin{step} \label{ArithThmPf-algic}
 Every matrix entry of every element of\/~$\Gamma$ is an
algebraic number.
 \end{step}
 Suppose some $\gamma_{i,j}$ is transcendental.
 Then, for any transcendental number~$\alpha$, there is a
field automorphism~$\phi$ of~$\complex$ with
$\phi(\gamma_{i,j}) = \alpha$. Applying~$\phi$ to all the
entries of a matrix induces an automorphism~$\widetilde\phi$
of $\SL(\ell,\complex)$. Let
	$$ \text{$\varphi$ be the restriction of~$\widetilde\phi$ to~$\Gamma$,} $$
so $\varphi$ is a homomorphism from~$\Gamma$ to $\SL(\ell,\complex)$.
The Margulis Superrigidity Theorem implies there is a
continuous homomorphism $\widehat\varphi \colon G \to
\SL(\ell,\complex)$, such that $\widehat\varphi = \varphi$ on
a finite-index subgroup of~$\Gamma$ \csee{MargSuperNonCpctSCEx}. By passing to this finite-index subgroup,
we may assume $\widehat\varphi = \varphi$ on all of~$\Gamma$.

Since there are uncountably many transcendental
numbers~$\alpha$, there are uncountably many different
choices of~$\phi$, so there must be uncountably many
different $n$-dimensional representations~$\widehat\varphi$
of~$G$. However, it is well known from the the theory of
``roots and weights'' that $G$ (or, more generally, any
connected, simple Lie group) has
only finitely many non-isomorphic representations
of any given dimension, so this is a contradiction.%
\footnote{Actually, this is not quite a
contradiction, because it is possible that two different
choices of~$\varphi$ yield the same representation of~$\Gamma$,
up to isomorphism; that is, after a change of basis. The
trace of a matrix is independent of the basis, so the
preceding argument really shows that the trace
of~$\varphi(\gamma)$ must be algebraic, for every
$\gamma \in \Gamma$. Then one can use some algebraic methods
to construct some other matrix representation~$\varphi'$
of~$\Gamma$, such that the matrix entries of~$\varphi'(\gamma)$
are algebraic, for every $\gamma \in \Gamma$.}

\begin{step}
 We have\/ $\Gamma \subseteq \SL(\ell,\rational)$.
 \end{step}
 Let $F$ be the subfield of~$\complex$ generated by the 
matrix entries of the elements of~$\Gamma$, so $\Gamma 
\subseteq \SL(\ell,F)$. From \cref{ArithThmPf-algic}, we know
that this is an algebraic extension of~$\rational$.
Furthermore, because $\Gamma$ is finitely generated \csee{GammaFinGen}, 
we see that this field extension is finitely
generated. Therefore, $F$ is finite-degree field extension
of~$\rational$ (in other words, $F$ is an algebraic number
field). This means that $F$ is almost the same
as~$\rational$, so it is only a slight exaggeration to say
that we have proved $\Gamma  \subseteq \SL(\ell,\rational)$.

Indeed, restriction of scalars \pref{ResScal->Latt} provides a way to change $F$
into~$\rational$: there is a representation $\rho \colon G
\to \SL(r,\complex)$, for some~$r$, such that 
$\rho \bigl( G \cap \SL(\ell,F) \bigr) \subseteq \SL(r,\rational)$ 
\csee{ROSPutsGFinGQ}. Therefore, after
replacing~$G$ with $\rho(G)$, we have the
desired conclusion (without any exaggeration).



\begin{step} \label{MargArithPf-BddPowerP}
For every prime~$p$, there is a natural number~$N_p$, such 
that no element of\/~$\Gamma$ has a matrix entry whose 
denominator is divisible by~$p^{N_p}$. 
\end{step}
The fields $\real$ and~$\complex$ are complete (that is,
every Cauchy sequence converges), and they obviously 
contain~$\rational$. For any prime~$p$, the $p$-adic 
numbers~$\rational_p$ are another field that has these
same properties.

As we have stated it, the Margulis Superrigidity Theorem
deals with homomorphisms into $\SL(\ell,\F)$, where $\F = \real$,
but Margulis also proved a version of the theorem
that applies when $\F$ is a $p$-adic field \csee{padicSuper}. 
Now $G$ is connected,
but $p$-adic fields are totally disconnected, so every continuous 
homomorphism from~$G$ to $\SL(\ell,\rational_p)$ is trivial.
Therefore, superrigidity tells us that $\varphi$ is trivial, after we mod
out a compact group \ccf{MargSuperC}.
In other words, the closure of 
$\varphi(\Gamma)$ is compact in $\SL(\ell,\rational_p)$.

This conclusion can be rephrased in more elementary terms,
without any mention of $p$-adic
numbers. Namely, it says that there is a bound on the 
highest power of~$p$ that divides the denominator of any matrix entry of
any element of~$\Gamma$. This is what we wanted.


\begin{step}
Some finite-index subgroup\/~$\Gamma'$ of\/~$\Gamma$ is 
contained in\/ $\SL(\ell,\integer)$.
\end{step}
 Let $D \subseteq \natural$ be the set consisting of the
denominators of the matrix entries of the elements of
$\varphi(\Gamma)$. 

We claim there exists $N \in \natural$, such that every
element of~$D$ is less than~$N$.
Since $\Gamma$ is known to be finitely generated,
some finite set of primes $\{p_1,\ldots,p_r\}$ contains all
the prime factors of every element of~$D$. (If $p$~is in the
denominator of some matrix entry of $\gamma_1
\gamma_2$, then it must
appear in a denominator somewhere in either $\gamma_1$
or~$\gamma_2$.) Therefore, every element of~$D$ is of the
form $p_1^{m_1} \cdots p_r^{m_r}$, for some $m_1,\ldots,m_r
\in \natural$. From \cref{MargArithPf-BddPowerP}, 
we know $m_i < N_{p_i}$,
for every~$i$. Thus, every element of~$D$ is less than 
$p_1^{N_{p_1}} \cdots p_r^{N_{p_r}}$. This establishes
the claim.

 From the preceding paragraph, we see that 
 $\Gamma \subseteq \frac{1}{N!} \Mat_{\ell
\times \ell}(\integer)$. Note that if $N = 1$, then $\Gamma
\subseteq \SL(\ell,\integer)$. In general, $N$ is a finite 
distance from~$1$, so it should not be hard to believe
(and it can indeed be shown) that some 
finite-index subgroup of~$\Gamma$ must be contained
in $\SL(\ell,\integer)$ \csee{GammaBddDenomsEx}. Therefore, a finite-index subgroup
of~$\Gamma$ is contained in~$G_{\integer}$, as desired.
\qed

\medbreak

For ease of reference, we officially record the key fact used in \cref{MargArithPf-BddPowerP}:

\begin{thm}[(\thmindex{Margulis!Superrigidity!p-adic@$p$-adic}%
	Margulis superrigidity over $p$-adic fields)] \label{padicSuper}
Assume
\noprelistbreak
	\begin{enumerate} \renewcommand{\theenumi}{\roman{enumi}}
	\item $G$ is not isogenous to any group that is of the form\/ $\SO(1,m) \times K$ or\/ $\SU(1,m) \times K$, where $K$~is compact,
	\item $\Gamma$ is irreducible,
	\item $\rational_p$ is the field of $p$-adic numbers, for some prime~$p$,
	and
	\item $\varphi \colon \Gamma \to \GL(n,\rational_p)$ is a homomorphism.
	\end{enumerate}
Then $\closure{\varphi(\Gamma)}$ is compact.

In other words, there is some $N \in \integer$, such that every matrix entry of every element of $\varphi(\Gamma)$ is in $p^N \ \integer_p$, where $\integer_p$ is the ring of $p$-adic integers.
\end{thm}

The Margulis Arithmeticity Theorem \pref{MargArithFromSuper} does not apply to lattices in $\SO(1,n)$ or $\SU(1,n)$, but, for those groups, Margulis proved the following characterization of the lattices that are arithmetic:

\begin{namedthm}[\thmindex{Commensurability Criterion for Arithmeticity}{Commensurability Criterion for Arithmeticity} {\normalfont (Margulis)}] \label{CommCriterion}
Assume
	\begin{itemize}
	\item $G$ is connected, with no compact factors,
	and
	\item $\Gamma$ is irreducible.
	\end{itemize}
Then\/ $\Gamma$ is arithmetic if and only if the commensurator\/ $\Comm_G(\Gamma)$ of\/~$\Gamma$ is dense in~$G$.
\end{namedthm}

As was already mentioned in \fullcref{GQComm}{criterion}, the direction ($\Rightarrow$) follows from the simple observation that $\Comm_G(G_\integer)$ contains $G_\rational$. 

The proof of ($\Leftarrow$) is more difficult. It is the same as the proof of the Margulis Arithmeticity Theorem, but replacing the Margulis Superrigidity Theorem \pref{MargSuperG'} with the following superrigidity theorem (and also replacing the $p$-adic superrigidity theorem with a suitable commensurator analogue):

\begin{thm}[(\thmindex{Commensurator Superrigidity}{Commensurator Superrigidity})] \label{CommSuper}
Assume
\noprelistbreak
	\begin{enumerate} \renewcommand{\theenumi}{\roman{enumi}}
	\item $\Gamma$ is irreducible, 
	\item $\Comm_G(\Gamma)$ is dense in~$G$,
 	and
	\item  $G$ and $G'$ are connected, with trivial center, and no compact factors.
	\end{enumerate}
If $\varphi \colon \Comm_G(\Gamma) \to G'$ is any homomorphism whose image is Zariski dense in~$G'$, then $\varphi$ extends to a continuous homomorphism $\widehat\varphi \colon G \to G'$.
\end{thm}

\begin{exercises}

\item \label{ROSPutsGFinGQ}
Suppose 
	\begin{itemize}
	\item $G \subseteq \SL(\ell,\complex)$, 
	\item $\Gamma \subseteq \SL(\ell,F)$, for some algebraic number field~$F$,
	and
	\item $G$ has no compact factors.
	\end{itemize}
Show there is a continuous homomorphism $\rho \colon G
\to \SL(r,\complex)$, for some~$r$, such that $\rho \bigl( G \cap
\SL(\ell,F) \bigr) \subseteq \SL(r,\rational)$.
\hint{Apply restriction of scalars (\S\ref{RestrictScalarsSect}) after noting that the Borel Density Theorem (\S\ref{BDTSect}) implies $G$ is defined over~$F$.}

\item \label{GammaBddDenomsEx}
Show that if $\Lambda$ is a subgroup of $\SL(\ell,\rational)$, and $\Lambda \subseteq \frac{1}{N} \Mat_{\ell\times\ell}(\integer)$, for some $N \in \natural$, then $\SL(\ell,\integer)$ contains a finite-index subgroup of~$\Lambda$.
\hint{The additive group of~$\rational^\ell$ contains a $\Lambda$-invariant subgroup~$V$, such that we have $\integer^\ell \subseteq V \subseteq \frac{1}{N} \integer^\ell$. Choose $g \in \GL(\ell,\rational)$, such that $g(V) = \integer^\ell$. Then $g$ commensurates $\SL(\ell,\integer)$ and we have $g \Lambda g^{-1} \subseteq \SL(\ell,\integer)$.}

%\item \label{BddDenom->finind}
% Suppose $\Lambda$ is a subgroup of $\SL(n,\rational)$, and
%$k$~is a positive integer, such that $k \lambda \in \Mat_{n
%\times n}(\integer)$ for every $\lambda \in \Lambda$. Show
%that $\Lambda \cap \SL(n,\integer)$ has finite index
%in~$\Lambda$.
% \hint{If $k \gamma \equiv k \lambda \pmod{k}$,
%then $\gamma \lambda^{-1} \in \Mat_{n \times
%n}(\integer)$.}

\item \label{EigenValsAreInts}
%(\emph{requires some Algebraic Number Theory})
Assume, as usual, that 
	\begin{itemize}
	\item $G$ is not isogenous to any group that is of the form\/ $\SO(1,m) \times K$ or\/ $\SU(1,m) \times K$, where $K$~is compact,
	and
	\item $\Gamma$ is irreducible.
	\end{itemize}
Use the proof of the Margulis Arithmeticity Theorem to show that if $\varphi \colon \Gamma \to \SL(n,\complex)$ is any homomorphism, then every eigenvalue of every element of $\varphi(\Gamma)$ is an algebraic integer.
%\hint{After a change of basis, and passing to a subgroup of finite index, every entry of every matrix in $\varphi(\Gamma)$ is an algebraic integer.}

\end{exercises}









\section{Homomorphisms into compact groups} \label{SuperCpctSect}

The Margulis Superrigidity Theorem \pref{MargSuperC} does not say anything about homomorphisms whose image is contained in a compact subgroup of $\GL(n,\real)$. (This is because all of $\varphi(\Gamma)$ can be put into the error term~$C$, so the homomorphism $\widehat\alpha$ can be taken to be trivial.) Fortunately, there is a different version that completely eliminates the error term (and applies very generally). Namely, from the Margulis Arithmeticity Theorem \pref{MargulisArith}, we know that the lattice $\Gamma$ must be arithmetic (if no simple factors of~$G$ are $\SO(1,m)$ or $\SU(1,m)$). This means that if we add some compact factors to~$G$, then we can assume that $\Gamma$ is commensurable to~$G_{\integer}$. In this situation, there is no need for the error term~$C$:

\begin{cor} \label{GZSuper}
Assume
\noprelistbreak
	\begin{itemize}
	\item $G$ is connected, algebraically simply connected, and defined over~$\rational$,
	\item $G$ is not isogenous to any group that is of the form\/ $\SO(1,m) \times K$ or\/ $\SU(1,m) \times K$, where $K$~is compact,
	\item $\Gamma$ is irreducible,
	and
	\item $\varphi \colon \Gamma \to \GL(n,\real)$ is a homomorphism.
	\end{itemize}
If\/ $\Gamma$ is commensurable to~$G_{\integer}$, then there exist:
\noprelistbreak
	\begin{enumerate}
	\item a continuous homomorphism $\widehat\varphi \colon G \to \GL(n,\real)$,
	and
	\item a finite-index subgroup\/~$\Gamma'$ of\/~$\Gamma$,
	\end{enumerate}
such that $\varphi(\gamma) = \widehat\varphi(\gamma)$, for all $\gamma \in \Gamma'$.
\end{cor}

Here is a less precise version of \cref{GZSuper} that may be easier to apply in situations where the lattice~$\Gamma$ is not explicitly given as the $\integer$-points of~$G$.  
%It states that homomorphisms of~$\Gamma$ into compact groups are Galois twists of homomorphisms that extend to~$G$ (if we consider each simple factor of $\closure{\alpha(\Gamma)}$ individually).
However, it only applies to the homomorphism into each simple component of $\closure{\alpha(\Gamma)}$, not to the entire homomorphism all at once.

\begin{cor} \label{MargSuperCpct}
Assume
\noprelistbreak
	\begin{itemize}
	\item $G$ is connected, and algebraically simply connected, 
	\item $G$ is not isogenous to any group that is of the form\/ $\SO(1,m) \times K$ or\/ $\SU(1,m) \times K$, where $K$~is compact,
	\item $\Gamma$ is irreducible,
	\item $\varphi \colon \Gamma \to \GL(n,\complex)$ is a homomorphism,
	and
	\item $\closure{\varphi(\Gamma)}$ is simple.
	\end{itemize}
Then there exist:
\noprelistbreak
	\begin{enumerate}
	\item a continuous homomorphism $\widehat\varphi \colon G \to \GL(n,\complex)$,
	\item a finite-index subgroup\/~$\Gamma'$ of\/~$\Gamma$,
	and
	\item a Galois automorphism~$\sigma$ of\/~$\complex$,
	\end{enumerate}
such that $\varphi(\gamma) = \sigma \bigl( \widehat\varphi(\gamma) \bigr)$, for all $\gamma \in \Gamma'$.
\end{cor}

\begin{proof}
We may assume $\closure{\varphi(\Gamma)}$ is compact, for otherwise \cref{MargSuperG'} applies (after modding out the centers of $G$ and $\closure{\alpha(\Gamma)}$). Then every element of $\closure{\varphi(\Gamma)}$ is semisimple. 

Choose some $h \in \varphi(\Gamma)$, such that $h$ has infinite order \csee{ImgNotTorsion}. Then the conclusion of the preceding paragraph implies that some eigenvalue~$\lambda$ of~$h$ is not a root of unity. On the other hand, if $\lambda$~is algebraic, then $p$-adic superrigidity \pref{padicSuper} implies that $\lambda$~is an algebraic integer \csee{EigenValsAreInts}. So there is a Galois automorphism~$\sigma$ of~$\complex$, such that $|\sigma(\lambda)| \neq 1$ \csee{KroneckerNot1}. Then $\{\, \sigma(\lambda)^k \mid k \in \integer \,\}$ is an unbounded subset of~$\complex$, so $\bigl\langle \sigma(h) \bigr\rangle$ is not contained in any compact subgroup of $\GL(n,\complex)$.

Now, let 
\noprelistbreak
	\begin{itemize}
	\item $\varphi'$ be the composition $\sigma \circ \varphi$,
	and
	\item $G'$ be the Zariski closure of $\varphi'(\Gamma)$.
	\end{itemize}
Then $G'$ is simple, and the conclusion of the preceding paragraph implies that $G'$ is not compact (since $\sigma(h) \in G'$). After passing to a finite-index subgroup (so $G'$ is connected), \cref{MargSuperG'} provides a continuous homomorphism $\widetilde\varphi \colon G \to G'$, such that $\varphi'(\gamma) = \widehat\varphi(\gamma)$, for all $\gamma$ in some finite-index subgroup of~$\Gamma$.
\end{proof}

\begin{warn}
Assume $\Gamma$ is irreducible, and $G$ is not isogenous to any group of the form $\SO(1,m) \times K$ or $\SU(1,m) \times K$.
\Cref{MargSuperCpct} implies that if there exists a homomorphism~$\varphi$ from~$\Gamma$ to a compact Lie group (and $\varphi(\Gamma)$ is infinite), then $G/\Gamma$ must be compact \csee{MargNoncpct}. However, \textbf{the converse is not true}.  Namely, \cref{GZSuper} tells us that if $\Gamma$ is commensurable to $G_{\integer}$, where $G$~is defined over~$\rational$, and $G_{\real}$ has no compact factors, then $\Gamma$~does not have any homomorphisms to compact groups (with infinite image). It does not matter whether $G/\Gamma$ is compact or not.
\end{warn}

\begin{exercises}

\item \label{MargNoncpct}
Assume, as usual, that the lattice $\Gamma$~is irreducible, that $G$~is not isogenous to any group of the form $\SO(1,m) \times K$ or $\SU(1,m) \times K$, and that $\varphi \colon \Gamma \to \GL(n,\real)$ is a homomorphism. If $G/\Gamma$ is \textbf{not} compact, show the semisimple group $\Zar{\varphi(\Gamma)}$ has no compact factors.
\hint{Godement's Criterion \pref{GodementCriterion}.}

\item \label{MargSuperNonCpctSCEx}
Assume
	\begin{itemize}
	\item $G$ is algebraically simply connected,
	\item $G$ is not isogenous to any group that is of the form\/ $\SO(1,m) \times K$ or\/ $\SU(1,m) \times K$, where $K$~is compact,
	\item $\Gamma$ is irreducible,
	\item $G/\Gamma$ is not compact,
	and
	\item $\varphi \colon \Gamma \to \SL(n,\real)$ is a homomorphism.
	\end{itemize}
Show there is a continuous homomorphism $\widehat\varphi \colon G \to \SL(n, \real)$, such that $\varphi(\gamma) = \widehat\varphi(\gamma)$ for all $\gamma$ in some finite-index subgroup of~$\Gamma$.
\hint{\cref{MargSuperC,MargImgSS,MargNoncpct}.}

%\item \label{SuperSLnEx}
%Prove \cref{MargSuperSL3RNonCpct}.
%\hint{\cref{MargSuperC,MargImgSS,SLnASC,MargNoncpct}.}

\item 
Assume $\Gamma$ is irreducible, and $G$ has no factors isogenous to $\SO(1,m)$ or $\SU(1,m)$.
Show that if $N$ is an infinite normal subgroup of~$\Gamma$, such that $\Gamma/N$ is linear (i.e., isomorphic to a subgroup of $\GL(\ell,\complex)$, for some~$\ell$), then $\Gamma/N$ is finite.  


\item \label{KroneckerNot1}
(\emph{\thmindex{Kronecker's}{Kronecker's Theorem}})
Assume $\lambda$ is an algebraic integer. Show that if $|\sigma(\lambda)| = 1$ for every Galois automorphism~$\sigma$ of~$\complex$, then $\lambda$ is a root of unity.
\hint{The powers of $\lambda$ form a set that (by restriction of scalars) is discrete in $\bigtimes_{\sigma \in S^\infty} F_\sigma^\times$. Alternate proof: there are only finitely many polynomials of degree $n$ with integer coefficients that are all $\le C$ in absolute value.}

\end{exercises}



\section{Proof of the Margulis Superrigidity Theorem} \label{SuperPfSect}

In order to establish \cref{MargSuperG'}, it suffices to prove the following special case \csee{MargSuperHSimpleEx}:

\begin{thm} \label{MargSuperHSimple}
Suppose
\noprelistbreak
\begin{itemize}
\item $G$ is connected, and it is not isogenous to any group that is of the form\/ $\SO(1,m) \times K$ or\/ $\SU(1,m) \times K$, where $K$~is compact, 
\item the lattice $\Gamma$ is irreducible in~$G$,
\item $H$ is a connected, noncompact, simple subgroup of\/ $\SL(n,\real)$, for some~$n$ {\rm(}and $H$ has trivial center{\rm)},
\item $\varphi \colon \Gamma \to H$ is a homomorphism,
and
\item $\varphi(\Gamma)$ is Zariski dense in~$H$.
\end{itemize}
Then $\varphi$ extends to a continuous homomorphism $\widehat\varphi \colon G \to H$.
\end{thm}

Although it does result in some loss of generality, we assume:

\begin{assump} \label{MargSuperRrankAssump}
$\Rrank G \ge 2$.
\end{assump}

The case where $\Rrank G = 1$ requires quite different methods --- see \cref{SuperRank1Sect} for a very brief discussion.

\subsection{Geometric reformulation}
To set up the proof of \cref{MargSuperHSimple}, let us translate the problem into a geometric setting, by replacing the homomorphism~$\varphi$ with the corresponding flat vector bundle~$\bundle_\varphi$ over~$G/\Gamma$ \csee{FlatVecBundleDefn}.

\begin{rem} \label{pfsuper-sect<>equi}
The sections of the vector bundle~$\bundle_\varphi$ are in natural one-to-one correspondence with the right $\Gamma$-equivariant maps from~$G$ to~$\real^n$ \csee{Equi<>SectionEx}.
\end{rem}

\begin{lem} \label{Extend<>SectionBij}
$\varphi$ extends to a homomorphism $\cover\varphi \colon G \to \GL(n,\real)$ if and only if there exists a $G$-invariant subspace $V \subseteq \Sect(\bundle_\varphi)$, such that the evaluation map $V \to V_{[e]}$ is bijective.
\end{lem}

\begin{proof}
($\Leftarrow$) Since $V$ is $G$-invariant, we have a representation of~$G$ on~$V$; let us say $\pi \colon G \to \GL(V)$. Therefore, the isomorphism $V \to V_{[e]} = \real^n$ yields a representation $\widehat\pi$ of~$G$ on~$\real^n$. It is not difficult to verify that $\widehat\pi$ extends~$\varphi$ \csee{PiHatExtendsPhi}.

($\Rightarrow$) For $v \in \real^n$ and $g \in G$, let 
	$$\xi_v(g) = \cover\varphi(g^{-1}) v .$$
It is easy to verify that $\xi_v \colon G \to \real^n$ is right $\Gamma$-equivariant \csee{XivEqui}, so we may think of~$\xi_v$ as a section of~$\bundle_\varphi$ \csee{pfsuper-sect<>equi}. Let 
	$$V = \{\, \xi_v \mid v \in \real^n\,\} \subseteq \Sect(\bundle_\varphi) .$$
Now the map $v \mapsto \xi_v$ is linear and $G$-equivariant \csee{XiGEqui}, so $V$ is a $G$-invariant subspace of $\Sect(\bundle_\varphi)$. Since 
	$$ \xi_v \bigl( [e] \bigr) = \cover\varphi(e) v = v ,$$
it is obvious that the evaluation map is bijective.
\end{proof}

In fact, if we assume the representation $\varphi$ is irreducible, then it is not necessary to have the evaluation map $V \to V_{[e]}$ be bijective. Namely, in order to show that $\varphi$ extends, it suffices to have $V$ be finite dimensional (and nonzero):

\begin{lem}[\csee{Extend<>FDEx}] \label{Extend<>FD}
Assume that the representation~$\varphi$ is irreducible. If there exists a\/ \textup(nontrivial\/\textup) $G$-invariant subspace $V$ of\/ $\Sect(\bundle_\varphi)$ that is finite dimensional,
then $\varphi$ extends to a continuous homomorphism $\cover\varphi \colon G \to \GL(n,\real)$.
\end{lem}


\subsection{The need for higher real rank}

We now explain how \cref{MargSuperRrankAssump} comes into play. 

\begin{notation}
Let $A$ be a maximal $\real$-split torus of~$G$. For example, if $G = \SL(3,\real)$, we let
	$$ A = \begin{Smallbmatrix} \upast&0&0 \\ 0&\upast&0 \\ 0&0&\upast \end{Smallbmatrix} .$$
By definition, the assumption that $\Rrank G \ge 2$ means $\dim A \ge 2$.
\end{notation}

It is the following result that relies on our assumption $\Rrank G \ge 2$. It is easy to prove if $G$ has more than one noncompact simple factor \csee{G1xG2GenLiEx}, and is not difficult to verify for the case $G = \SL(\ell,\real)$ \ccf{SL3RGenLiEx}.  Readers familiar with the structure of semisimple groups (including the theory of real roots) should have little difficulty in generalizing to any semisimple group of real rank $\ge 2$ \csee{HighRankGenLiEx}.

\begin{lem} \label{HighRankGenLi}
If\/ $\Rrank G \ge 2$, then, for some $r \in \natural$, there exist closed subgroups $L_1,L_2,\ldots,L_r$ of~$G$, such that
	\begin{enumerate}
%	\item $L_i$ is a closed subgroup of~$G$ that is isogenous to $\SL(2,\real)$,
	\item $ G = L_r L_{r-1} \cdots L_1$,
	and
	\item both $H_i$ and $H_i^\perp$ are noncompact, where
		\begin{itemize}
		\item $H_i = L_i \cap A$,
		and
		\item $H_i^\perp = \czer_A(L_i)$ {\rm(}so $L_i$ centralizes~$H_i^\perp${\rm)}.
		\end{itemize}
	\end{enumerate}
\end{lem}


\subsection{Outline of the proof}

The idea for proving \cref{MargSuperHSimple} is quite simple. We begin by finding a (nonzero) $A$-invariant section of~$\bundle_\varphi$; this section spans a ($1$-dimensional) subspace~$V_0$ of~$\bundle_\varphi$ that is invariant under~$A$. Since (by definition) the subgroup~$H_1$ of \cref{HighRankGenLi} is contained in~$A$, we know that $V_0$ is invariant under~$H_1$, so \cref{pfsuper-C(H)V} below provides a subspace of $\Sect(\bundle_\varphi)$ that is invariant under a larger subgroup of~$G$, but is still finite dimensional. Applying the lemma repeatedly yields finite-dimensional subspaces that are invariant under more and more of~$G$. Eventually, the lemma yields a finite-dimensional subspace that is invariant under all of~$G$. Then \cref{Extend<>FD} implies that $\varphi$ extends to a homomorphism that is defined on~$G$, as desired.

\begin{lem} \label{pfsuper-C(H)V}
If
\noprelistbreak
	\begin{itemize}
	\item $H$ is a closed, noncompact subgroup of~$A$,
	and
	\item $V$ is an $H$-invariant subspace of\/ $\Sect(\bundle_\varphi)$ that is finite dimensional,
	\end{itemize}
then $\langle \czer_G(H) \cdot V \rangle$ is finite dimensional.
\end{lem}

\begin{proof}[Idea of proof]
To illustrate the idea of the proof, let us assume that $V = \real \sigma$ is the span of an $H$-invariant section \csee{pfsuper-C(H)VEx}.
Since $H$ is noncompact, the Moore Ergodicity Theorem \pref{MooreErgodicity} tells us that $H$ has a dense orbit on $G/\Gamma$ \csee{AEOrbitDenseInG/Gamma}. (In fact, almost every orbit is dense.) This implies that any continuous $H$-invariant section of~$\bundle_\varphi$ is determined by its value at a single point \csee{InvtSectDetPtEx}, so the space of $H$-invariant sections is finite-dimensional \csee{InvtSectFDEx}. Since this space contains $\langle \czer_G(H) \cdot V \rangle$ \csee{CG(H).HInvtEx}, the desired conclusion is immediate.
\end{proof}

Here is a more detailed outline:

\begin{proof}[Idea of the proof of \cref{MargSuperHSimple}]
Assume there exists a nonzero $A$-invariant section~$\sigma$ of~$\bundle_\varphi$. Let 
	$$ \text{$H_0 = A$ and $V_0 = \langle \sigma \rangle$.} $$
Thus, $V_0$ is a $1$-dimensional subspace of $\Sect(\bundle_\varphi)$ that is $H_0$-invariant.

Now, for $i = 1,\ldots,r$, let
	$$ V_i = \langle L_i \cdot A \cdot L_{i-1} \cdot A \cdots L_1 \cdot A \cdot V_0 \rangle .$$
Since $L_r L_{r-1} \cdots L_1 = G$, it is clear that $V_r$ is $G$-invariant. Therefore, it will suffice to show (by induction on~$i$) that each $V_i$ is finite dimensional.

Since $H_{i-1} \subseteq L_{i-1}$, it is clear that $V_{i-1}$ is $H_{i-1}$-invariant. Therefore, since $A$~centralizes~$H_{i-1}$, \Cref{pfsuper-C(H)V} implies that $\langle A \cdot V_{i-1} \rangle$ is finite dimensional. Now, since $H_i^\perp \subseteq A$, we know that $\langle A \cdot V_{i-1} \rangle$ is $H_i^\perp$-invariant. Then, since $L_i$ centralizes~$H_i^\perp$, \Cref{pfsuper-C(H)V} implies that the subspace $V_i = \langle  L_i \cdot A \cdot V_{i-1} \rangle$ is finite dimensional.
\end{proof}

Therefore, the key to proving \cref{MargSuperHSimple} is finding a nonzero $A$-invariant section~$\sigma$ of~$\bundle_\varphi$. Unfortunately, the situation is a bit more complicated than the above would indicate, because  we will not find a \emph{continuous} $A$-invariant section, but only a \emph{measurable} one \csee{pfsuper-keyfact-meas}. Then the proof appeals to \cref{pfsuper-C(H)V-meas} below, instead of \cref{pfsuper-C(H)V}. We leave the details to the reader \csee{ProveMargSuperHSimpleEx}.

\begin{defn}
Let $\Sectm(\bundle_\varphi)$ be the vector space of measurable sections of~$\bundle_\varphi$, where two sections are identified if they agree almost everywhere.
\end{defn}

\begin{lem}[\csee{pfsuper-C(H)V-measPfEx,pfsuper-C(H)V-measFullPfEx}] \label{pfsuper-C(H)V-meas}
If
\noprelistbreak
	\begin{itemize}
	\item $H$ is a closed, noncompact subgroup of~$A$,
	and
	\item $V$ is a finite-dimensional, $H$-invariant subspace of\/ $\Sectm(\bundle_\varphi)$,
	\end{itemize}
then $\langle \czer_G(H) \cdot V \rangle$ is finite dimensional.
\end{lem}


\begin{exercises}

\item \label{MargSuperHSimpleEx}
Derive \cref{MargSuperG'} as a corollary of \cref{MargSuperHSimple}.

\item \label{Equi<>SectionEx}
Suppose $\xi \colon G \to \real^n$. Show that $\overline\xi \colon G/\Gamma \to \bundle_\varphi$, defined by
	$$ \overline\xi(g \Gamma) = \bigl[ \bigl( g,{\xi}(g) \bigr) \bigr] , $$
is a well-defined section of~$\bundle_\varphi$ if and only if $\xi$ is right $\Gamma$-equivariant; i.e., $\xi(g \gamma) = \varphi(\gamma^{-1}) \,\xi(g)$.

\item \label{PiHatExtendsPhi}
In the notation of the proof of \cref{Extend<>SectionBij}($\Leftarrow$), show $\widehat\pi(\gamma) = \varphi(\gamma)$ for every $\gamma \in \Gamma$.

\item \label{XivEqui}
In the notation of the proof of \cref{Extend<>SectionBij}($\Rightarrow$), show that we have $\xi_v(gh) = \cover\varphi(h^{-1}) \, \xi_v(g)$. Since $\cover\varphi(\gamma^{-1}) = \varphi(\gamma^{-1})$ for all $\gamma \in \Gamma$, this implies that $\xi_v$ is right $\Gamma$-equivariant.

\item \label{XiGEqui}
In the notation of the proof of \cref{Extend<>SectionBij}($\Rightarrow$), show that we have
$\xi_{\cover\varphi(g) v} = g \cdot \xi_v$, where the action of~$G$ on $\Sect(\bundle_\varphi)$ is defined by $(g \cdot \xi_v)(x) = \xi_v(g^{-1} x)$, as usual.

\item \label{Extend<>FDEx}
Prove \cref{Extend<>FD}.
\hint{By choosing $V$ of minimal dimension, we may assume it is an irreducible $G$-module, so the evaluation map is either $0$ or injective. It cannot be~$0$, and then it must also be surjective, since $\varphi$ is irreducible.}

\item \label{G1xG2GenLiEx}
Prove \cref{HighRankGenLi} under that additional assumption that we have $G = G_1 \times G_2$, where $G_1$ and~$G_2$ are noncompact (and semisimple).
\hint{Let $L_i = G_i$ for $i = 1,2$.}

\item \label{SL3RGenLiEx}
Prove the conclusion of \cref{HighRankGenLi} for $G = \SL(3,\real)$.
\hint{A \defit[unipotent!elementary matrix]{unipotent elementary matrix} is a matrix with $1$'s on the diagonal and only one nonzero off-diagonal entry. Every element of $\SL(3,\real)$ is a product of $\le 10$ unipotent elementary matrices, and any such matrix is contained in a subgroup isogenous to $\SL(2,\real)$ that has a $1$-dimensional intersection with~$A$.}

\item \label{HighRankGenLiEx}
Prove \cref{HighRankGenLi}.

\item \label{InvtSectDetPtEx}
Let $H$ be a subgroup of~$G$. Show that if $\sigma_1$ and~$\sigma_2$ are $H$-invariant, continuous sections of~$\bundle_\varphi$, and there is some $x \in G/\Gamma$, such that
	\begin{itemize}
	\item $Hx$ is dense in $G/\Gamma$
	and
	\item $\sigma_1(x) = \sigma_2(x)$,
	\end{itemize}
then $\sigma_1 = \sigma_2$.

\item \label{InvtSectFDEx}
Let $H$ be a subgroup of~$G$, and assume $H$ has a dense orbit in $G/\Gamma$. Show the space of $H$-invariant, continuous sections of $\bundle_\varphi$ has finite dimension.

\item \label{CG(H).HInvtEx}
Let $H$ be a subgroup of~$G$. Show that if $\sigma$ is an $H$-invariant section of~$\bundle_\varphi$, and $c$~is an element of~$G$ that centralizes~$H$, then $\sigma \cdot c$ is also $H$-invariant.

\item \label{pfsuper-C(H)VEx}
Prove \cref{pfsuper-C(H)V} without assuming that $V$ is $1$-dimensional.
\hint{Fix $x \in G$.
For $c \in C_G(H)$ and $\sigma \in V$, define $T \colon V \to \real^n$ by $T(\xi) = \xi (x)$, and note that $(c \sigma)(hx) = T(h^{-1} \cdot \sigma)$ for all $h \in H$. If $H x \Gamma$ is dense, this implies that $c\sigma$ is determined by $\sigma$ and~$T$. So $\dim \bigl( C_G(H) \cdot V) \le ( \dim V) \cdot \bigl( \dim \Hom(V,\real^n) \bigr)$.}
%For $h \in H$, we have $(c\sigma)[h] = \bigl( h^{-1}(c\sigma)\bigr) [e] = \bigl(c (h^{-1}\sigma) \bigr)[e] = T(h^{-1}\sigma)$. 

\item \label{pfsuper-C(H)V-measPfEx}
Prove \cref{pfsuper-C(H)V-meas} in the special case where $V = \real \sigma$ is the span of an $H$-invariant measurable section.
\hint{This is similar to \cref{pfsuper-C(H)V}, but use the fact that $H$ is ergodic on $G/\Gamma$.}

\item \label{pfsuper-C(H)V-measFullPfEx}
Prove \cref{pfsuper-C(H)V-meas} (without assuming $\dim V = 1$).
\hint{This is similar to \cref{pfsuper-C(H)VEx}.}

\item \label{ProveMargSuperHSimpleEx}
Prove \cref{MargSuperHSimple}.

\end{exercises}








\section{An \texorpdfstring{$A$}{A}-invariant section}

This section sketches the proof of the following result, which completes the proof of \cref{MargSuperHSimple} (under the assumption that $\Rrank G \ge 2$).

\begin{keyfact} \label{pfsuper-keyfact-meas}
For some~$n$, there is an embedding of~$H$ in $\SL(n,\real)$, such that%
\noprelistbreak
	\begin{enumerate}
	\item the associated representation $\varphi \colon \Gamma \to H \subseteq \SL(n,\real)$ is irreducible,
	and
	\item there exists a nonzero $A$-invariant $\sigma \in \Sectm(\bundle_\varphi)$.
	\end{enumerate}
\end{keyfact}

\Cref{pfsuper-sect<>equi} allows us to restate this as follows:

\begin{thmref}{pfsuper-keyfact-meas}
\begin{keyfact} \label{pfsuper-keyfact-equi}
For some embedding of~$H$ in $\SL(n,\real)$, 
\begin{enumerate}
\item $H$ acts irreducibly on~$\real^n$, 
and
\item there exists a\/ $\Gamma$-equivariant, measurable function\/ $\xi \colon G/A \to \real^n$ \textup(and $\xi$~is nonzero\/\textup).
\end{enumerate}
\end{keyfact}
\end{thmref}

In this form, the result is closely related to the following consequence of amenability (from \cref{AmenableChap}). For simplicity, it is stated only for the case $G = \SL(3,\real)$.

\begin{thmref}{SL3R/P->Meas(X)}
\begin{prop}[(Furstenberg)] \label{SL3R/P->Meas(X)'}
If
\noprelistbreak
	\begin{itemize}
	\item $G = \SL(3,\real)$,
	\item $ P
	 =  \begin{Smallbmatrix} \upast&\upast&\upast \\ &\upast&\upast \\ &&\upast \end{Smallbmatrix}
	 \subset G $,
	 and
	 \item $\Gamma$ acts continuously on a compact metric space~$X$,
 \end{itemize}
 then there is a Borel measurable map $\psi \colon G/P \to
\Prob(X)$, such that $\psi$ is essentially
$\Gamma$-equivariant.
 \end{prop}
 \end{thmref}

For convenience, let $W = \real^n$. There are 3 steps in the proof of \cref{pfsuper-keyfact-equi}:
	\begin{enumerate}
	\item (amenability) Letting $X$ be the projective space $\projective(W)$, which is compact, \cref{SL3R/P->Meas(X)'} provides a $\Gamma$-equivariant, measurable map $ \widehat \xi \colon G/P \to \Prob \bigl( \projective(W) \bigr)$.
	\item (proximality) The representation of~$\Gamma$ on~$W$ induces a representation of~$\Gamma$ on any exterior power $\bigwedge^k W$. By replacing $W$ with an appropriate subspace of such an exterior power, we may assume there is some $\gamma \in \Gamma$, such that $\gamma$ has a unique eigenvalue of maximal absolute value \csee{UniqMaxEigEx}. Therefore, the action of $\gamma$ on $\projective(W)$ is ``proximal'' \csee{ProxLem}. The theory of proximality (discussed in \cref{QuickProximalitySect}) now tells us that the $\Gamma$-equivariant random map~$\widehat\xi$ must actually be a well-defined map into $\projective(W)$ \csee{pfsuper-xi(x)ptmass}. 

\item (algebra trick) We have a $\Gamma$-equivariant map $\widehat\xi \colon G/P \to \projective(W)$. By the same argument, there is a $\Gamma$-equivariant map $\widehat\xi^* \colon G/P \to \projective(W^*)$, where $W^*$ is the dual of~$W$. Combining these yields a $\Gamma$-equivariant map 
	$$\overline\xi \colon G/P \times G/P \to \projective(W \otimes W^*) \iso \projective \bigl( \End(W) \bigr) .$$
We can lift $\overline\xi$ to a well-defined map 
	$$\xi \colon G/P \times G/P \to \End(W), $$
by specifying that $\trace \bigl( \xi(x) \bigr) = 1$ \csee{trace(xi)=1Ex}.
Since the action of $\Gamma$ on $\End(W)$ is by conjugation \csee{GammaOnEnd(W)ConjEx}
and the trace of conjugate matrices are equal, we see that $\xi$ is $\Gamma$-equivariant \csee{XiGammaEquiEx}.

Finally, note that there is a $G$-orbit in $G/P \times G/P$ whose complement is a set of measure~$0$, and the stabilizer of a point is (conjugate to) the group~$A$ of diagonal matrices \csee{GTransOnFlags,GTransOnP/PxG/P}. Therefore, after discarding a set of measure~$0$, we may identify $G/P \times G/P$ with $G/A$, so $\xi \colon G/A \to \End(W)$.
	\end{enumerate}


\begin{exercises}

\item \label{UniqMaxEigEx}
Let 
	\begin{itemize}
	\item $\gamma$ be a semisimple element of~$\Gamma$, such that some eigenvalue of~$\gamma$ is not of absolute value~$1$. 
	\item $\lambda_1,\ldots,\lambda_k$ be the eigenvalues of~$\gamma$ (with multiplicity) that have maximal absolute value.
	\item $W' = \bigwedge^k W$.
	\end{itemize}
Show that, in the representation of~$\Gamma$ on~$W'$, the element~$\gamma$ has a unique eigenvalue of maximal absolute value.

\item \label{trace(xi)=1Ex}
Let $\cover\xi \colon G/P \to W$ and $\cover\xi^* \colon G/P \to W^*$ be well-defined, measurable lifts of $\widehat\xi$ and $\widehat\xi^*$. 
	\begin{enumerate}
	\item Show, for a.e.\ $x,y \in G/P$, that $\cover\xi(x)$ is not in the kernel of the linear functional $\cover\xi^*(y)$.
	\item Show, for a.e.\ $x,y \in G/P$, that, under the natural identification of $W \otimes W^*$ with $\End(W)$, we have 
		$$\trace \bigl( \cover\xi(x) \otimes \cover\xi^*(y) \bigr) \neq 0 .$$
	\item Show $\overline\xi$ can be lifted to a well-defined measurable function $\xi \colon G/P \times G/P \to \End(W)$, such that $\trace \bigl( \xi(x,y) \bigr) = 1$, for a.e.\ $x,y \in G/P$.
	\end{enumerate}
\hint{$\Gamma$ acts irreducibly on~$W$, and ergodically on $G/P \times G/P$.}

\item \label{GammaOnEnd(W)ConjEx}
Show that the action of $\Gamma$ on $\End(W) \iso W \otimes W^*$ is given by conjugation:
	$ \overline\varphi(\gamma) T = \varphi(\gamma) \, T \, \varphi(\gamma)^{-1} $.

\item \label{XiGammaEquiEx}
Show that $\xi$ is $\Gamma$-equivariant.

\item \label{GTransOnFlags}
Recall that a \defit[flag in $\real^3$]{flag} in~$\real^3$ is a pair $(\ell, \Pi)$, where 
	\begin{itemize}
	\item $\ell$\, is a line through the origin (in other words, a $1$-dimensional linear subspace),
	and
	\item $\Pi$ is a plane through the origin (in other words, a $2$-dimensional linear subspace),
	such that
	\item $\ell \subset \Pi$.
	\end{itemize}
Show:
	\begin{enumerate}
	\item $\SL(3,\real)$ acts transitively on the set of all flags in~$\real^3$,
	and
	\item the stabilizer of any flag is conjugate to the subgroup~$P$ of
 \cref{SL3R/P->Meas(X)'}.
 	\end{enumerate}
Therefore, the set of flags can be identified with $G/P$.

\item \label{GTransOnP/PxG/P}
Two flags $(\ell_1,\Pi_1)$ and $(\ell_2,\Pi_2)$ are in \defit{general position} if 
	$$ \text{
	$\ell_1 \notin \Pi_2$, \  
	and \ 
	$\ell_2 \notin \Pi_1$
	} .$$
Letting $\mathcal{G}$ be the subset of $G/P \times G/P$ corresponding to the pairs of flags that are in general position, show:
	\begin{enumerate}
	\item \label{GTransOnP/PxG/P-trans}
	$\SL(3,\real)$ is transitive on~$\mathcal{G}$,
	\item \label{GTransOnP/PxG/P-stab}
	the stabilizer of any point in~$\mathcal{G}$ is conjugate to the group of diagonal matrices,
	and
	\item the complement of~$\mathcal{G}$ has measure zero in $G/P \times G/P$.
	\end{enumerate}
	\hint{For \pref{GTransOnP/PxG/P-trans} and \pref{GTransOnP/PxG/P-stab}, identify $\mathcal{G}$ with the set of triples $(\ell_1,\ell_2,\ell_3)$ of lines  that are in general position, by letting $\ell_3 = \Pi_1 \cap \Pi_2$.}
\end{exercises}




\section{A quick look at proximality} \label{QuickProximalitySect}

\begin{assump} \label{ProxAssump}
Assume 
	\begin{enumerate}
	\item $\Gamma \subset \SL(\ell,\real)$, 
	\item every finite-index subgroup of~$\Gamma$ is irreducible on~$\real^\ell$,
	and 
	\item there exists a semisimple element $\overline\gamma \in \Gamma$, such that $\overline\gamma$ has a unique eigenvalue~$\overline\lambda$ of maximal absolute value (and the eigenvalue is simple, which means the corresponding eigenspace is $1$-dimensional).
	\end{enumerate}
\end{assump}

\begin{notation} \ 
\begin{enumerate}
\item Let $\overline{v}$ be an eigenvector associated to the eigenvalue~$\overline\lambda$.
\item For convenience, let $W = \real^\ell$.
\end{enumerate}
\end{notation}

\begin{lem}[(Proximality)] \label{ProxLem}
The action of\/~$\Gamma$ on\/ $\projective(W)$ is {\upshape\defit{proximal}}. This means that, for every $[w_1],[w_2] \in \projective(W)$, there exists a sequence $\{\gamma_n\}$ in~$\Gamma$, such that $d\bigl( [\gamma_n(w_1)],[\gamma_n(w_2)] \bigr) \to 0$ as $n \to \infty$.
\end{lem}

\begin{proof}
Assume, to simplify the notation, that all of the eigenspaces of~$\overline\gamma$ are orthogonal to each other. 
Then, for any $w \in W \smallsetminus \overline{v}^\perp$, we have $\overline\gamma^n[w] \to [\overline{v}]$, as $n \to \infty$ \csee{PowerToEigEx}. Since the finite-index subgroups of~$\Gamma$ act irreducibly, there is some $\gamma \in \Gamma$, such that $\gamma(w_1), \gamma(w_2) \notin \overline{v}^\perp$ \csee{MoveBothOutOfOrthCompEx}. Therefore, 
	$$d\bigl( \overline\gamma^n\gamma ([w_1]), \overline\gamma^n\gamma ([w_2]) \bigr) 
	\to d( [\overline{v}],[\overline{v}]) 
	= 0 ,$$
as desired.
\end{proof}

In the above proof, it is easy to see that the convergence $\overline\gamma^n[w] \to [\overline{v}]$ is uniform on compact subsets of $W \smallsetminus \overline{v}^\perp$ \csee{PowerToEigUnifEx}. This leads to the following stronger assertion \csee{MeasureProxEx}:

\begin{prop}[(Measure proximality)] \label{MeasureProx}
Let $\mu$ be any probability measure on\/ $\projective(W)$. Then there is a sequence\/ $\{\gamma_n\}$ in\/~$\Gamma$, such that\/ $(\gamma_n)_*\mu$ converges to a delta-mass supported at a single point of\/ $\projective(W)$.
\end{prop}

It is obvious from \cref{MeasureProx} that there is no $\Gamma$-invariant probability measure on $\projective(W)$. However, it is easy to see that there does exist a probability measure that is invariant ``on average\zz,'' in the following sense \csee{ExistStatMeasEx}:

\begin{defn} \ 
\noprelistbreak
\begin{enumerate}
\item Fix a finite generating set~$S$ of~$\Gamma$, such that $S^{-1} = S$. A probability measure~$\mu$ on $\projective(W)$ is \defit{stationary} for~$S$ if 
	$$ \frac{1}{\#S} \sum_{\gamma \in S} \gamma_* \mu = \mu .$$
\item More generally, let $\nu$ be a probability measure on~$\Gamma$.
 A probability measure~$\mu$ on $\projective(W)$ is \defit[stationary]{$\nu$-stationary} if $\nu * \mu = \mu$. More concretely, this means
	$$  \sum_{\gamma \in \Gamma} \nu(\gamma) \, \gamma_* \mu = \mu .$$
\end{enumerate}
(Some authors call $\mu$ ``\index{harmonic!measure}{harmonic}\zz,'' rather than ``stationary\zz.'')
\end{defn}

\begin{rem}
A random walk on $\projective(W)$ can be defined as follows: Choose a sequence $\gamma_1,\gamma_2, \ldots$ of elements of~$\Gamma$, independently and with distribution~$\nu$. Also choose a random $x_0 \in \projective(W)$, with respect to some probability distribution~$\mu$ on $\projective(W)$. Then $x_n \in \projective(W)$ is defined  by 
	$$x_n = \gamma_1 \gamma_2 \cdots \gamma_n(x_0) ,$$
so $\{x_n\}$ is a random walk on $\projective(W)$.
A stationary measure represents a ``stationary state'' (or equilibrium distribution) for this random walk. Hence the terminology.  
\end{rem}

If the initial distribution~$\mu$ is stationary, then a basic result of probability (the ``\thmindex{Martingale Convergence}Martingale Convergence Theorem'') implies, for almost every sequence $\{\gamma_n\}$, that the resulting random walk $\{x_n\}$ has a limiting distribution; that is, 
	$$ \text{for a.e.\ $\{\gamma_n\}$, \ $(\gamma_1 \gamma_2 \cdots \gamma_n)_*\mu$ converges in 
$\Prob \bigl( \projective(W) \bigr)$}. $$
This theorem applies to stationary measures on any space, with no need for \cref{ProxAssump}. By using measure proximality, we will now show that the limiting distribution is almost always a point mass.

\begin{defn}
A closed, nonempty, $\Gamma$-invariant subset of $\projective(W)$ is \defit[minimal!closed, invariant set]{minimal} if it does not have any nonempty, proper, closed, $\Gamma$-invariant subsets. (Since $\projective(W)$ is compact, the finite-intersection property implies that every nonempty, closed, $\Gamma$-invariant subset of $\projective(W)$ contains a minimal set.)
\end{defn}

\begin{thm}[(\thmindex{mean proximality}Mean proximality)] \label{meanprox}
Assume 
	\begin{itemize}
	\item $\nu$ is a probability measure on~$\Gamma$, such that $\nu(\gamma) > 0$ for all $\gamma \in \Gamma$,
	\item $C$ is a minimal closed, $\Gamma$-invariant subset of $\projective(W)$,
	and
	\item $\mu$ is a $\nu$-stationary probability measure on~$C$. 
	\end{itemize}
Then, for a.e.\ $\{\gamma_n\} \in \Gamma^\infty$, there exists $c \in \projective(W)$, such that 
	$$ \text{$(\gamma_1 \gamma_2 \cdots \gamma_n)_*(\mu) \to \delta_c$ as $n \to \infty$.} $$
\end{thm}

\begin{proof}
It was mentioned above that the Martingale Convergence Theorem implies $(\gamma_1 \gamma_2 \cdots \gamma_n)_*(\mu)$ has a limit (almost surely), so it suffices to show there is (almost surely) a subsequence that converges to a measure of the form~$\delta_c$.

\Cref{MeasureProx} provides a sequence $\{g_k\}$ of elements of~$\Gamma$, such that $(g_k)_*\mu \to \delta_{c_0}$, for some $c_0 \in C$. To extend this conclusion to a.e.\ sequence $\{\gamma_n\}$, we use equicontinuity: we may write $\Gamma$ is the union of finitely many sets $E_1,\ldots,E_r$, such that each $E_i$ is equicontinuous on some nonempty open subset~$U_i$ of~$C$ \csee{EquiOnProj}.

The minimality of~$C$ implies $\Gamma U_i = C$ for every~$i$. Then, by compactness, there is a finite subset $F = \{f_1,\ldots,f_s\}$ of~$\Gamma$, such that $F U_i = C$ for each~$i$. Since $\nu(\gamma) > 0$ for every $\gamma \in \Gamma$, there is (almost surely) a subsequence $\{\gamma_{n_k}\}$ of~$\{\gamma_n\}$, such that, for every~$k$, we have
	$$ \text{$\gamma_{n_k+1} \gamma_{n_k+2}\cdots \gamma_{n_k+j} = f_j^{-1} \, g_k$ \ for $1 \le j \le s$} .$$
By passing to a subsequence, we may assume there is some~$i$, such that 
	$$ \text{$\gamma_1 \gamma_2 \cdots \gamma_{n_k} \in E_i$, for all~$k$} .$$
To simplify the notation, let us assume $i = 1$.

Since $F U_1 = C$, we may write $c_0 = f_j u$, for some $f_j \in F$ and $u \in U_1$.
Then
	$$ (\gamma_{n_k+1} \gamma_{n_k+2}\cdots \gamma_{n_k+j})_* \nu = (f_j^{-1} \, g_k)_* \nu \to  (f_j^{-1})_* \delta_{c_0} = \delta_{f_j^{-1} c_0} = \delta_u.$$
By passing to a subsequence, we may assume $(\gamma_1 \gamma_2 \cdots \gamma_{n_k})u$ converges to some $c \in C$. Then, since $\gamma_1 \gamma_2 \cdots \gamma_{n_k} \in E_1$, 
and $E_1$ is equicontinuous on $U_1$, this implies 
	\begin{align*}
	(\gamma_1 \gamma_2 \cdots \gamma_{n_k+j})_* \nu 
	= (\gamma_1 \gamma_2 \cdots \gamma_{n_k})_*  
	\bigl( (\gamma_{n_k+1} \ \cdots \gamma_{n_k+j})_* \nu \bigr)
	\to \delta_c 
	. & \qedhere \end{align*}
\end{proof}

In order to apply this theorem, we need a technical result, whose proof we omit:

\begin{lem} \label{pfsuper-LebesgueStat}
There exist:
	\begin{itemize}
	\item a probability measure~$\nu$ on\/~$\Gamma$,
	and
	\item a $\nu$-stationary probability measure~$\mu$ on $G/P$,
	\end{itemize}
such that 
	\begin{enumerate}
	\item  the support of~$\nu$ generates\/~$\Gamma$,
	and
	\item $\mu$ is in the class of Lebesgue measure. \textup(That is, $\mu$ has exactly the same sets of measure\/~$0$ as Lebesgue measure does.\textup)
	\end{enumerate}
\end{lem}

Also note that if $C$ is any nonempty, closed, $\Gamma$-invariant subset of $\projective(W)$, then $\Prob(C)$ is a nonempty, compact, convex $\Gamma$-space, so Furstenberg's Lemma \pref{G/amen->Meas(X)} provides a $\Gamma$-equivariant map $\overline{\xi} \colon G/P \to \Prob(C)$. This observation allows us to replace $\projective(W)$ with a minimal subset.

We can now fill in the missing part of the proof of \cref{pfsuper-keyfact-equi}:

\begin{cor} \label{pfsuper-xi(x)ptmass}
Suppose 
\noprelistbreak
	\begin{itemize}
	\item $C$ is a minimal closed, $\Gamma$-invariant subset of\/ $\projective(W)$,
	and
	\item $\overline{\xi} \colon G/P \to \Prob(C)$ is $\Gamma$-equivariant.
	\end{itemize}
Then $\overline{\xi}(x)$ is a point mass, for a.e.\ $x \in G/P$.

Hence, there exists $\hat\xi \colon G/P \to \projective(W)$, such that $\overline{\xi}(x) = \delta_{\hat\xi(x)}$, for a.e.\ $x \in G/P$.
\end{cor}

\begin{proof}
Let 
	\begin{itemize}
	\item $\delta_{\projective(W)} = \{\, \delta_{x} \mid x \in \projective(W)\,\}$
		be the set of all point masses in the space $\Prob \bigl( \projective(W) \bigr)$,
	and
	\item $\mu$ be a $\nu$-stationary probability measure on $G/P$ that is in the class of Lebesgue measure \csee{pfsuper-LebesgueStat}.
	\end{itemize}
We wish to show $\overline{\xi}(x) \in \delta_{\projective(W)}$, for a.e.\ $x \in G/P$. In other words, we wish to show that $\overline{\xi}_*(\mu)$ is supported on $\delta_{\projective(W)}$.

Note that:
	\begin{itemize}
	\item $\delta_{\projective(W)}$ is a closed, $\Gamma$-invariant subset of $\Prob \bigl( \projective(W) \bigr)$,
	and
	\item because $\overline{\xi}$ is $\Gamma$-equivariant, we know that $\overline{\xi}_*(\mu)$ is a $\nu$-stationary probability measure on $\Prob \bigl( \projective(W) \bigr)$. 
	\end{itemize}

Roughly speaking, the idea of the proof is that almost every trajectory of the random walk on $\Prob \bigl( \projective(W) \bigr)$ converges to a point in $\delta_{\projective(W)}$ \see{meanprox}. On the other hand, being stationary, $\overline{\xi}_*(\mu)$ is invariant under the random walk. Therefore, we conclude that $\overline{\xi}_*(\mu)$ is supported on $\delta_{\projective(W)}$, as desired. 

We now make this rigorous. Let
	$$ \mu_{\projective(W)} = \int_{G/P} \overline{\xi}(x) \, d\mu(x) , $$
so $\mu_{\projective(W)}$ is a stationary probability measure on $\projective(W)$. By mean proximality \pref{meanprox}, we know, for a.e.\ $(\gamma_1,\gamma_2,\ldots) \in \Gamma^\infty$, that
	$$ d \bigl( \, (\gamma_1 \gamma_2 \cdots \gamma_n)_*(\mu_{\projective(W)}), \ \delta_{\projective(W)} \, \bigr) \ \stackrel{n \to \infty}{\longrightarrow} \ 0 .$$
For any $\epsilon > 0$, this implies, by using the definition of~$\mu_{\projective(W)}$, that
	$$ \mu \Bigl( \bigset{ x \in G/P }{ d \bigl( \gamma_1 \gamma_2 \cdots \gamma_n \bigl( \overline{\xi}(x) \bigr) 
	, \delta_{\projective(W)} \bigr)  > \epsilon } \Bigr) \ \stackrel{n \to \infty}{\longrightarrow} \ 0 .$$
Since $\overline{\xi}$ is $\Gamma$-equivariant, we may
	$$ \text{replace \ $\gamma_1 \gamma_2 \cdots \gamma_n \bigl( \overline{\xi}(x) \bigr)$
	\ with \ $\overline{\xi}( \gamma_1 \gamma_2 \cdots \gamma_n  x)$} .$$
Then, since the measure $\mu$ on $G/P$ is stationary, we can delete $\gamma_1 \gamma_2 \cdots \gamma_n$, and conclude that
	\begin{align} \label{DeleteGamma}
	 \mu
	\bigset{ x \in G/P }{ 
	d \bigl( \overline{\xi}(x) 
	, \delta_{\projective(W)} \bigr)  > \epsilon } \ \stackrel{n \to \infty}{\longrightarrow} \ 0 
	\end{align}
\csee{DeleteGammaEx}.
Since the left-hand side does not depend on~$n$, but tends to~$0$ as $n \to \infty$, it must be~$0$. Since $\epsilon > 0$ is arbitrary, we conclude that $\overline{\xi}(x)  \in \delta_{\projective(W)}$ for a.e.~$x$, as desired.
\end{proof}


\begin{exercises}

\item \label{PowerToEigEx}
In the notation of \cref{ProxLem}, show, for every $w \in W \smallsetminus \overline{v}^\perp$, that $\overline\gamma^n[w] \to [\overline{v}]$, as $n \to \infty$.

\item \label{MoveBothOutOfOrthCompEx}
Show, for any nonzero $w_1,w_2 \in W$, that there exists $\gamma \in \Gamma$, such that neither $\gamma w_1$ nor $\gamma w_2$ is orthogonal to~$\overline{v}$.
\hint{Let $H$ be the Zariski closure of~$\Gamma$ in $\SL(\ell,\real)$, and assume, by passing to a finite-index subgroup, that $H$ is connected. Then $W_i = \{\, h \in H \mid h w_i \in \overline{v}^\perp \,\}$ is a proper, Zariski-closed subset. Since $\Gamma$ is Zariski dense in~$H$, it must intersection the complement of $W_1 \cup W_2$.}

\item \label{PowerToEigUnifEx}
Show that the convergence in \cref{PowerToEigEx} is uniform on compact subsets of $W \smallsetminus \overline{v}^\perp$.

\item \label{MeasureProxEx}
Prove \cref{MeasureProx}.
\hint{Show $\max_{w \in \projective(W) \\ \nu \in \overline{\Gamma \mu}} \nu(w) = 1$.}

\item \label{ExistStatMeasEx}
Show there exists a stationary probability measure on $\projective(W)$.
\hint{Kakutani-Markov Fixed-Point Theorem \cf{CyclicAmen}.}

\item \label{EquiOnProj}
Let $C$ be a subset of $\projective(\real^n)$, and assume that $C$ is not contained in any $(n-1)$-dimensional hyperplane.
Prove that $\GL(n,\real)$ is the union of finitely many sets $E_1,\ldots,E_r$, such that each $E_i$ is equicontinuous on some nonempty open subset~$U_i$ of~$C$.
\hint{Each matrix $T \in \Mat_{n\times n}(\real)$ induces a well-defined, continuous function $\overline{T} \colon \bigl( \projective(\real^n) \smallsetminus \projective(\ker T) \bigr) \to \projective(\real^n)$. 
If $B_T$ is a small ball around  $\overline{T}$ in $\projective \bigl( \Mat_{n\times n}(\real) \bigr)$, then $B_T$ is equicontinuous on an open set. A compact set can be covered by finitely many balls.}

\item \label{DeleteGammaEx}
Establish \pref{DeleteGamma}.
\hint{Since $\mu$ is stationary, the map
	$$ \Gamma^\infty \times G/P \to G/P \colon \bigl( (\gamma_1,\gamma_2,\ldots), x \bigr)
	\mapsto \gamma_1 \gamma_2 \cdots \gamma_n x $$
is measure preserving.}

\item
Show that if $\projective(W)$ is minimal, then the $\Gamma$-equivariant measurable map $\xi \colon G/P \to \projective(W)$ is unique (a.e.).
\hint{If $\psi$ is another $\Gamma$-equivariant map, then define $\overline\xi \colon G/P \to \Prob \bigl( \projective(W) \bigr)$ by $\overline\xi(x) = \frac{1}{2} ( \delta_{\xi(x)} + \delta_{\psi(x)})$.}

\end{exercises}




\section{Groups of real rank one} \label{SuperRank1Sect}

The Margulis Superrigidity Theorem \pref{MargSuperC} was proved for groups of real rank at least two in \cref{SuperPfSect}. Suppose, now, that $\Rrank G = 1$ (and $G$ has no compact factors). The classification of simple Lie groups tells us that $G$ is isogenous to the isometry group of either:
	\begin{itemize}
	\item real hyperbolic space $\hyperbolic^n$,
	\item complex hyperbolic space $\complex\hyperbolic^n$,
	\item quaternionic hyperbolic space $\quaternion\hyperbolic^n$,
	or
	\item the Cayley hyperbolic plane $\mathbb{O}\hyperbolic^2$ (where $\mathbb{O}$ is the ring of ``Cayley numbers'' or ``octonions'')
	\end{itemize}
 \ccf{rank1simple}.
Assumption~\fullref{MargSuperC}{notSOSU} rules out $\hyperbolic^n$ and $\complex\hyperbolic^n$, so, from the connection of superrigidity with totally geodesic embeddings \ccf{TotGeodSect}, the following result completes the proof:

\begin{thm}
Assume
	\begin{itemize}
	\item $X = \quaternion\hyperbolic^n$ or\/ $\mathbb{O}\hyperbolic^2$,
	\item $\Gamma$ is a torsion-free, discrete group of isometries of~$X$, such that\/ $\Gamma \backslash X$ has finite volume,
	\item $X'$ is an irreducible symmetric space of noncompact type,
	and
	\item $\varphi \colon \Gamma \to \Isom(X')^\circ$ is  a homomorphism whose image is Zariski dense.
	\end{itemize}
Then there is a map $f \colon X \to X'$, such that 
	\begin{enumerate}
	\item $f(X)$ is totally geodesic,
	and
	\item $f$ is $\varphi$-equivariant, which means $f(\gamma x) = \varphi(\gamma) \cdot f(x)$.
	\end{enumerate}
\end{thm}

\begin{proof}[Brief outline of proof]
Choose a (nice) fundamental domain~$\fund$ for the action of~$\Gamma$ on~$X$. For any $\varphi$-equivariant map $f \colon X \to X'$, define the \defit{energy} of~$f$ to be the $L^2$-norm of the derivative of~$f$ over~$\fund$. Since $f$~is $\varphi$-equivariant, and the groups $\Gamma$ and $\varphi(\Gamma)$ act by isometries, this is independent of the choice of the fundamental domain~$\fund$.

It can be shown that this energy functional attains its minimum at some function~$f$. The minimality implies that $f$ is harmonic. Then, by using the geometry of~$X$ and the negative curvature of~$X'$, it can be shown that $f$ must be totally geodesic.
\end{proof}




\begin{notes}

This chapter is largely based on \cite[Chaps.~6 and~7]{MargulisBook}. (However, we usually replace the assumption that $\Rrank G \ge 2$ with the weaker assumption that $G$ is not $\SO(1,m) \times K$ or $\SU(1,m) \times K$. (See \cite[Thm.~5.1.2, p.~86]{ZimmerBook} for a different exposition that proves version \pref{MargSuperG'} for $\Rrank G \ge 2$.)
In particular:
\begin{itemize}
\item For $\Rrank G \ge 2$, our statement of the Margulis Superrigidity Theorem \pref{MargSuperG'} is a special case of \cite[Thm.~7.5.6, p.~228]{MargulisBook}.

\item For $\Rrank G \ge 2$, \cref{MargImgSS} is stated in \cite[Thm.~9.6.15(i)(a), p.~332]{MargulisBook}.

\item For $\Rrank G \ge 2$, \cref{MostowRigidity} is stated in \cite[Thm.~7.7.5, p.~254]{MargulisBook}. (See \cite[Thm.~B]{PrasadMostowRig} for the general case, which does not follow from superrigidity.)

\item \Cref{Extend<>FD} is a version of \cite[Prop.~4.6, p.~222]{MargulisBook}
\item \Cref{HighRankGenLi} is a version of \cite[Lem.~7.5.5, p.~227]{MargulisBook}.
\item \Cref{pfsuper-C(H)V-meas} is \cite[Prop.~7.3.6, p.~219]{MargulisBook}.

\item \Cref{pfsuper-keyfact-equi} is adapted from \cite[Thm.~6.4.3(b)2, p.~209]{MargulisBook}.

\item \Cref{meanprox} is based on \cite[Prop.~6.2.13, pp.~202--203]{MargulisBook}.
\item \Cref{pfsuper-LebesgueStat} is taken from \cite[Prop.~6.4.2, p.~209]{MargulisBook}.
\item \Cref{pfsuper-xi(x)ptmass} is based on \cite[Prop.~6.2.9, p.~200]{MargulisBook}.
\item \Cref{EquiOnProj} is \cite[Lem.~6.3.2, p.~203]{MargulisBook}.
\end{itemize}

Long before the general theorem of Margulis for groups of real rank $\ge 2$, it was proved by Bass, Milnor, and Serre \cite[Thm.~61.2]{BassMilnorSerre-CSP} that the \thmindex{Congruence Subgroup Property}Congruence Subgroup Property implies $\SL(n,\integer)$ is superrigid in $\SL(n,\real)$. 

\defit[Geometric superrigidity]{\normalfont ``Geometric superrigidity''} is the study of differential geometric versions of the Margulis Superrigidity Theorem, such as \cref{TotGeodProp}. (See, for example, \cite{MokSiuYeung-GeomSuper}.)

Details of the derivation of arithmeticity from superrigidity (\cref{MargArithPf}) appear in \cite[Chap.~9]{MargulisBook} and \cite[\S6.1]{ZimmerBook}. 

Proofs of the Commensurability Criterion \pref{CommCriterion} and Commensurator Superrigidity \pref{CommSuper} can be found in \cite{ACampoBurger-reseaux}, \cite[\S9.2.11, pp.~305\emph{ff}, and Thm.~7.5.4, pp.~226--227]{MargulisBook}, and \cite[\S6.2]{ZimmerBook}.

Much of the material in \cref{QuickProximalitySect} is due to Furstenberg \cite{Furstenberg-BdThyStochProc}.

The superrigidity of lattices in the isometry groups of $\quaternion\hyperbolic^n$ and $\mathbb{O}\hyperbolic^2$ \csee{SuperRank1Sect} was proved by Corlette \cite{Corlette-superrig}. The $p$-adic version \pref{padicSuper} for these groups was proved by Gromov and Schoen \cite{GromovSchoen-padicsuper}.

\end{notes}




\begin{references}{9}

\bibitem{ACampoBurger-reseaux}
N.\,A'Campo and M.\,Burger:
R\'eseaux arithm\'etiques et commensurateur d'apr\`es G.\,A.\,Margulis,
\emph{Invent. Math.} 116 (1994) 1--25. 
\MR{1253187},
  \maynewline
  \url{http://eudml.org/doc/144182}
%  \url{http://www.digizeitschriften.de/dms/resolveppn/?PPN=GDZPPN002111705}

\bibitem{BassMilnorSerre-CSP}
H.\,Bass, J.\,Milnor, and J.--P.\,Serre:
Solution of the congruence subgroup problem for $\SL_n$ ($n \ge 3$) and $\Sp_{2n}$ ($n\ge 2$),
\emph{Inst. Hautes \'Etudes Sci. Publ. Math.} 33 (1967) 59--137. 
\MR{0244257},
\url{http://www.numdam.org/item?id=PMIHES_1967__33__59_0}

\bibitem{Corlette-superrig}
K.\,Corlette:
Archimedean superrigidity and hyperbolic geometry,
\emph{Ann. Math.} (2) 135 (1992), no.~1, 165--182. 
\MR{1147961},
\url{http://www.jstor.org/stable/2946567}

\bibitem{Furstenberg-BdThyStochProc}
H.\,Furstenberg:
Boundary theory and stochastic processes on homogeneous spaces, in
C.\,C.\,Moore, ed.:
\emph{Harmonic Analysis on Homogeneous Spaces 
%(Proc. Sympos. Pure Math., Vol. XXVI, Williams Coll., 
(Williamstown, Mass., 1972)}. 
Amer. Math. Soc., Providence, R.I., 1973,
pp.~193--229.
\MR{0352328}

\bibitem{GromovSchoen-padicsuper}
M.\,Gromov and R.\,Schoen:
Harmonic maps into singular spaces and $p$-adic superrigidity for lattices in groups of rank one,
\emph{Inst. Hautes \'Etudes Sci. Publ. Math.} 76 (1992) 165--246. 
\MR{1215595},
\url{http://www.numdam.org/item?id=PMIHES_1992__76__165_0}

\bibitem{MargulisBook}
 G.\,A.\,Margulis:
 \emph{Discrete Subgroups of Semisimple Lie Groups.}
 Springer, {New York}, 1991.
 ISBN 3-540-12179-X,
\MR{1090825}

\bibitem{MokSiuYeung-GeomSuper}
N.\,Mok, Y.\,T.\,Siu, and S.--K.\,Yeung:
 Geometric superrigidity,
 \emph{Invent. Math.} 113 (1993) 57--83.
\MR{1223224},
\maynewline
\url{http://eudml.org/doc/144122}
% \url{http://www.digizeitschriften.de/download/PPN356556735_0113/log10.pdf}

\bibitem{PrasadMostowRig}
 G.\,Prasad:
 Strong rigidity of $\rational$-rank~1 lattices,
 \emph{Invent. Math.} 21 (1973) 255--286.
 \MR{0385005},
 \maynewline
\url{http://eudml.org/doc/142232}
% \url{http://www.digizeitschriften.de/dms/resolveppn/?PPN=GDZPPN002090694}

\bibitem{Siu-GeomSuper}
Y.\,T.\,Siu:
Geometric super-rigidity, in \emph{Geometry and Analysis (Bombay, 1992)},
Oxford U.\ Press, Oxford, 1996, pp.~299--312.
ISBN 0-19-563740-2,
\MR{1351514}

\bibitem{ZimmerBook}
R.\,J.\,Zimmer:
\emph{Ergodic Theory and Semisimple Groups}.
%Monographs in Mathematics, 81. 
Birkh\"auser, Basel, 1984.
ISBN 3-7643-3184-4,
\MR{0776417}



\end{references}