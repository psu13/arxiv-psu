%!TEX root = IntroArithGrps.tex

\mychapter{Real Rank} \label{RrankChap}

\prereqs{none.}

%Although real rank and parabolic parabolic subgroups are a fundamental notion in the theory of semisimple Lie groups, it will not play a major role in most of this book.
%However, it is important in our discussion of lattices that are Gromov hyperbolic \csee{GromovHyperGrpsSect}, and they provide important background for the notion of $\rational$-rank (and parabolic $\rational$-subgroups), which are absolutely crucial for the construction of coarse fundamental domains in \cref{ReductionChap}. 

%\begin{assump}
% Assume that $G$ is a closed subgroup of $\SL(\ell,\real)$,
%for some~$\ell$. (The definitions and results of this
%chapter are independent of the particular embedding chosen.)
% \end{assump}


\section{\texorpdfstring{$\real$}{R}-split tori and \texorpdfstring{$\real$}{R}-rank}

\begin{defn} \label{RsplitDefn}
  A closed, connected subgroup~$T$ of~$G$ is a \defit{torus} if it is diagonalizable
over~$\complex\mk$; that is, if there exists $g \in
\GL(n,\complex)$, such that $g T g^{-1}$ consists entirely
of diagonal matrices. A torus is 
\defit[torus!$\real$-split]{$\real$-split} if it is diagonalizable
over~$\real\mk$; that is, if $g$~may be chosen to be in $\GL(n,\real)$.
 \end{defn}

%Because commuting diagonalizable matrices can be
%simultaneously diagonalized \csee{SimultDiag}, we have the following:
%
%\begin{prop} \label{torus<>diag}
% A subgroup~$T$ of~$G$ is a torus \index{torus} if and
%only if
% \begin{itemize}
% \item $T$ is closed and connected,
% \item $T$ is abelian, and
% \item every element of~$T$ is semisimple.
% \end{itemize}
% \end{prop}

\begin{egs} \ 
\noprelistbreak 
 \begin{enumerate}
 
 \item Let $A$ be the identity component of the group of
diagonal matrices in $\SL(n,\real)$. Then $A$ is obviously an $\real$-split
torus. 
%However, it is homeomorphic to $\real^{n-1}$, so it
%is not a torus in the usual topological sense.

\item $\SO(1,1)^\circ$ is an $\real$-split torus in $\SL(2,\real)$ \csee{SO(11)torus}.

 \item $\SO(2)$ is a torus in $\SL(2,\real)$ that is \emph{not} $\real$-split. It is diagonalizable over~$\complex$ \csee{SO2diag}, but not over~$\real$ \csee{SO2NotOverR}.
%Since every element of~$T$ is a normal linear
%transformation (that is, commutes with its transpose), we
%know from elementary linear algebra that every element of~$T$
%is diagonalizable.
% \item Generalizing the preceding, we see that $T = \SO(2)^n$
%is a torus in $\SL(2n,\real)$. Note that $T$
%is homeomorphic to the $n$-torus~$\torus^n$. (In fact, any
%compact torus subgroup is homeomorphic to a topological
%torus.) This is the motivation for the term
%\defit[torus subgroup]{torus}.
% \item The hypothesis that $T$ is abelian cannot be omitted
%from \cref{torus<>diag}. For example, every element of
%$\SO(n)$ is semisimple, but $\SO(n)$ is not abelian if $n
%\ge 3$.
 \end{enumerate}
 \end{egs}

\begin{warn}
 An $\real$-split torus is \emph{never} homeomorphic
to the topologist's torus~$\torus^n$ (except in the trivial case $n = 0$).
 \end{warn}
 
 \begin{rems} \ 
 \noprelistbreak
 	\begin{enumerate}

	\item If $T$ is an $\real$-split torus, then every element of~$T$ is hyperbolic \csee{hypelluniDefn}. In particular, no nonidentity element of~$T$ is elliptic or unipotent.

	\item  When $G$ is compact, every torus in~$G$ is isomorphic to $\SO(2)^n$, for some~$n$. This is homeomorphic to~$\torus^n$, which is the reason for the terminology ``torus\zz.''
 
	\end{enumerate}
 \end{rems}

%Recall that an element of~$G$ is
%\defit[hyperbolic!element of~$G$]{hyperbolic} if it is
%diagonalizable over~$\real$ \csee{hypelluniDefn}.
% In this
%terminology, we have the following:
%% analogue of \cref{torus<>diag}.
%
%\begin{prop}[\csee{RSplitTorus<>Ex}] \label{RSplitTorus<>}
% A subgroup~$T$ of~$G$ is an $\real$-split torus if and
%only if
%\noprelistbreak
% \begin{itemize}
% \item $T$ is closed and connected,
% \item $T$ is abelian, and
% \item every element of~$T$ is hyperbolic.
% \end{itemize}
% \end{prop}

It is a key fact in the theory of semisimple Lie groups that \emph{maximal} $\real$-split tori are conjugate:

\begin{thm}
If $A_1$ and~$A_2$ are maximal\/ $\real$-split tori in~$G$, then there exists $g \in G$, such that $A_1 = g A_2 g^{-1}$.
\end{thm}

This implies that all maximal $\real$-split tori have the same dimension, which is called the ``{real rank}'' (or ``$\real$-rank'') of~$G$, and is denoted $\Rrank G$:

 \begin{defn} \label{RrankDefn} \nindex{$\Rrank G$ = real rank of~$G$}%
\index{rank!R-@$\real$- or real}\term[R-@$\real$-!rank]{$\Rrank G$} is the dimension of a maximal $\real$-split torus~$A$ in~$G$. 
This is independent of both the choice of~$A$ and the choice of the embedding of~$G$ in $\SL(\ell,\real)$.
 \end{defn}

 
\begin{egs} \ \label{RrankEgs}
 \noprelistbreak
 \begin{enumerate}

 \item $\Rrank \bigl( \SL(n,\real) \bigr) = n-1$. (Let $A$
be the identity component of the group of all diagonal
matrices in $\SL(n,\real)$.) 

 \item We have
 $\Rrank \bigl( \SL(n,\complex) \bigr) = \Rrank \bigl(
\SL(n,\quaternion) \bigr) = n-1$. This is because only the
\emph{real} diagonal matrices remain
diagonal when $\SL(n,\complex)$ or $\SL(n,\quaternion)$ is embedded in
$\SL(2n,\real)$ or $\SL(4n,\real)$, respectively.

\item \label{RrankEgs-cpct}
$\Rrank G = 0$ if and only if $G$~is compact \csee{Rrank0Ex}.
 \end{enumerate}
 \end{egs}

\begin{prop} \label{Rrank(SOmn)}
 $\Rrank \SO(m,n)  = \min\{m,n\}$.
 \end{prop}

\begin{proof} 
Since $\SO(m,n)$ contains a copy of $\SO(1,1)^{\min\{m,n\}}$ \csee{SO11inSOmn}, and the identity component of this subgroup is an $\real$-split torus \ccf{SO(11)torus}, we have
	$$ \Rrank \SO(m,n) \ge \dim \bigl( \SO(1,1)^{\min\{m,n\}} \bigr)^\circ = \min\{m,n\}. $$

We now establish the reverse inequality. Let $A$ be a maximal $\real$-split torus. We may assume $A$ is nontrivial. (Otherwise $\Rrank \SO(m,n) = 0$, so the desired inequality is obvious.) Therefore, there is some nontrivial $a \in A$. Since $a$ is diagonalizable over~$\real$, and nontrivial, there is an eigenvector~$v$ of~$a$, such that $av \neq v$; hence, $av = \lambda v$ for some $\lambda \neq 1$. Now, if we let $\langle \cdot \mid \cdot \rangle_{m,n}$ be an $\SO(m,n)$-invariant bilinear form on~$\real^{m,n}$, we have
	$$ \langle v \mid v \rangle_{m,n}
	= \langle av \mid av \rangle_{m,n}
	= \langle \lambda v \mid \lambda v \rangle_{m,n}
	= \lambda^2 \langle v \mid v \rangle_{m,n}
	. $$
By choosing $a$ to be near~$e$, we may assume $\lambda \approx 1$, so $\lambda \neq -1$. Since, by assumption, we know $\lambda \neq 1$, this implies $\lambda^2 \neq 1$. So we must have $ \langle v \mid v \rangle_{m,n} = 0$;
 that is, $v$~is an \defit[isotropic vector]{isotropic} vector.
 Hence, we have shown that if the real rank is $\ge 1$, then there is an isotropic vector in $\real^{m+n}$. 
 
 By arguing more carefully, it is not difficult to see that if the real rank is at least~$k$, then there is a $k$-dimensional subspace of $\real^{m+n}$ that consists entirely of isotropic vectors \csee{SOmnIsotopic}. Such a subspace is said to be \defit[totally!isotropic]{totally isotropic}. The maximum dimension of a totally isotropic subspace is $\min\{m,n\}$ \csee{SOmnLargestIsotropic}, so we conclude that $\min\{m,n\} \ge \Rrank \SO(m,n)$, as desired.
 \end{proof}

\begin{rems} \label{RrankRems} \ 
\noprelistbreak
	\begin{enumerate}
	\item \label{RrankRems-Rrank(SUmn)}
	Other classical groups, not just $\SO(m,n)$, have the property that their real rank is the maximal dimension of a totally isotropic subspace.
More concretely, we have
 $$ %\Rrank \SO(m,n) = 
 \Rrank \SU(m,n) = \Rrank \Sp(m,n) = \min\{m,n\} .$$
 
	\item 
	The Mostow Rigidity Theorem \pref{MostowIso} will tell us that if $\Gamma$ is (isomorphic to) a lattice in both $G$ and~$G_1$, then $G^\circ$ is isomorphic to~$G_1'$, modulo compact groups. Modding out a compact subgroup does not affect the real rank \ccf{Rrank0Ex}, so this implies that the real rank of~$G$ is uniquely determined by the algebraic structure of~$\Gamma$.

\item \label{RrankRems-Rrank=CoverGamma}
Although it is not usually very useful in practice, we now state an explicit relationship between $\Gamma$ and $\Rrank G$. Let $S_r$ be the set of all elements~$\gamma$ of $\Gamma$, such that the centralizer $C_\Gamma(\gamma)$ is commensurable to a subgroup of the free abelian group $\integer^r$ of rank~$r$. Then it can be shown that
	$$ \Rrank G = \min \bigset{ r \ge 0 }{ \begin{matrix} \text{$\Gamma$ is covered by finitely} \\ \text{many translates of~$S_r$}  \end{matrix}} .$$
We omit the proof, which is based on the very useful (and nontrivial) fact that if $T$ is any maximal torus of~$G$, then there exists $g \in G$, such that $g T g^{-1} / (\Gamma \cap g T g^{-1})$ is compact.
\end{enumerate}
\end{rems}

\begin{exercises}

%\item \label{SimultDiag}
%Suppose $\mathcal{T} \subseteq \GL(n,F)$, such that:
%	\begin{itemize}
%	\item the matrices in $\mathcal{T}$ all commute with each other,
%	and, 
%	\item for all $t \in \mathcal{T}$, there exists $g \in \GL(n,F)$, such that $g t g^{-1}$ is a diagonal matrix. 
%	\end{itemize}
%Show the quantifiers can be reversed: there exists $g \in \GL(n,F)$, such that, for all $t \in \mathcal{T}$, $g t g^{-1}$ is a diagonal matrix. 
%\hint{$t$ is diagonalizable iff there is a basis of~$F^n$ that consists of eigenvectors of~$t$. If $s$ commutes with~$t$, then each eigenspace for~$t$ is $s$-invariant.}

\item \label{SO(11)torus}
Show that the identity component of $\SO(1,1)$ is an $\real$-split torus.
\hint{Let $g = \begin{Smallbmatrix} 1 & 1 \\ 1 & -1 \end{Smallbmatrix}$. Alternatively, note that each element of $\SO(1,1)$ is a symmetric matrix (hence, diagonalizable via an orthogonal matrix), and use the fact that any set of commuting diagonalizable matrices is simultaneously diagonalizable.}
%We have $\SO(1,1)^\circ = 
% \begin{bmatrix}
% \cosh t & \sinh t \\
% \sinh t & \cosh t
% \end{bmatrix}
%$,
%where $\cosh t = (e^t + e^{-t})/2$ and $\sinh t = (e^t
%- e^{-t})/2$, since $\cosh^2 t - \sinh^2 t = 1$).


\item \label{SO2diag}
 For $ g =
 \begin{Smallbmatrix}
 1 & -i \\
 1 & i
 \end{Smallbmatrix}$,
show every element of $g \SO(2) g^{-1}$ is diagonal.

\item \label{SO2NotOverR}
Show that $\SO(2)$ is not diagonalizable over~$\real$.
\hint{If $T$ is diagonalizable over~$\real$, then eigenvalues of the elements of~$T$ are real.}

\item \label{TorusAbel}
Show that every $\real$-split torus is abelian.

%\item \label{RSplitTorus<>Ex}
% Prove \cref{RSplitTorus<>}.
% \hint{\Cref{TorusAbel,SimultDiag}.}

\item Suppose
	\begin{itemize}
	\item $T$ is an $\real$-split torus in~$G$,
	and
	\item $A$ is a maximal $\real$-split torus in~$G$.
	\end{itemize}
Show that $T$ is conjugate to a subgroup of~$A$.
\hint{By considering dimension, it is obvious that $T$ is contained in some maximal $\real$-split torus of~$G$.}

\item \label{MaxRsplitZar}
Show that every maximal $\real$-split torus in~$G$ is almost Zariski closed.

\item \label{SO11inSOmn}
Assume $m \ge n$. Then $m + n \ge 2n$, so there is a natural embedding of $\SO(1,1)^n$ in $\SL(m+n,\real)$.
Show that $\SO(m,n)$ contains a conjugate of this copy of $\SO(1,1)^n$.
\hint{Permute the basis vectors.}
%\hint{Let $G = \SO(B)$, where
% $$ B = \sum_{i = 1}^n (x_{2i-1}^2 - x_{2i}^2 ) + \sum_{j=2n+1}^{m+n} x_j^2 ,$$
%so there is an obvious embedding of the $\real$-split torus $\bigl( \SO(1,1)^n \bigr)^\circ$ in~$G$:
% $$ \begin{bmatrix}
% \SO(1,1) \\
% & \SO(1,1) \\
% & & & \ddots \\
% & & & & \SO(1,1) \\
% & & & & & \Id_{m-n,m-n}
% \end{bmatrix}
% \subset G
% ,$$
% and note that $G$ is conjugate to $\SO(m,n)$, just by permuting the basis vectors.}

\item Prove, directly from \cref{RsplitDefn}, that if
$G_1$ is conjugate to~$G_2$ in $\GL(\ell,\real)$, then
$\Rrank(G_1) = \Rrank(G_2)$.

\item \label{Rrank0Ex}
Show $\Rrank G = 0$ if and only if $G$~is compact.
\hint{\Cref{SL2RinG,SSeltRem}\pref{SSeltRem-cpct}.}

\item \label{SOmnIsotopic}
Show that if $\Rrank \SO(m,n) = r$, then there is an $r$-dimensional subspace~$V$ of~$\real^{m+n}$, such that $\langle v \mid w \rangle_{m,n} = 0$ for all $v,w \in V$.
\hint{Because $A$ is diagonalizable over~$\real$, there is a basis $\{v_1,\ldots,v_{m+n}\}$ of~$\real^{m+n}$ whose elements are eigenvectors for every element of~$A$.
Since $\dim A = r$, we may assume, after renumbering, that for all $\lambda_1,\ldots,\lambda_r \in \real^+$, there exists $a \in A$, such that $av_i = \lambda_i v_i$, for $1 \le i \le r$. This implies $\langle v_1,\ldots,v_r \rangle$ is totally isotropic.}

\item \label{SOmnLargestIsotropic}
Show that if $V$ is a subspace of $\real^{m+n}$ that is totally isotropic for 
\text{$\langle \cdot \mid \cdot \rangle_{m,n}$}, % don't allow bad line break
then $\dim V \le \min\{m,n\}$.
\hint{If $v \neq 0$ and the last $n$ coordinates of~$v$ are~$0$, then $\langle v \mid v \rangle_{m,n} > 0$.}

\item Show $ \Rrank(G_1 \times G_2) =  \Rrank G_1 \  + \ \Rrank G_2$.

\item \label{Rrank0<>SL2R}
Show $\Rrank G \ge 1$ if and only if $G$~contains a subgroup that is isogenous to $\SL(2,\real)$.
\hint{\Cref{SL2RinG}.}

\item Show that $\Gamma$ contains a subgroup that is isomorphic to $\integer^r$, where $r = \Rrank G$.
\hint{You may assume the fact stated in the last sentence % !!!
of \fullcref{RrankRems}{Rrank=CoverGamma}.}

\end{exercises}






\section{Groups of higher real rank} \label{HigherRrankSect}

In some situations, there is a certain subset~$S$
of~$G$, such that the centralizer of each
element of~$S$ is well-behaved, and it would be helpful to know
that these centralizers generate~$G$. The results in this section illustrate that an assumption on the real rank of~$G$ may be exactly
what is needed. (However, we will often only prove the special case where $G = \SL(3,\real)$. A reader familiar with the theory of ``\term{real root}s'' should have no difficulty generalizing the arguments.)


\begin{prop} \label{Rrank2<>C(a)gens}
Let $A$ be a maximal $\real$-split torus in~$G$. Then we have
 $\Rrank G \ge 2$
 if and only if there exist nontrivial elements $a_1$ and~$a_2$ of~$A$,
such that $G = \langle \czer_G(a_1), \czer_G(a_2)\rangle$.
\end{prop}

\begin{proof}
($\Rightarrow$) Assume, for simplicity, that $G = \SL(3,\real)$.
(See \fullcref{CentGensG1xG2Ex}{C(a)} for another special case.) 
 Then we may assume $A$ is the group of diagonal matrices (after replacing it by a conjugate). Let
 $$ a_1 =
 \begin{bmatrix}
 2 & 0 & 0 \\
 0 & 2 & 0 \\
 0 & 0 & 1/4 
 \end{bmatrix}
 \mbox{ \qquad and \qquad }
 a_2 =
 \begin{bmatrix}
 1/4 & 0 & 0 \\
 0 & 2 & 0 \\
 0 & 0 & 2 
 \end{bmatrix}
 .$$
 Then
 $$ \czer_G(a_1) =
 \begin{bmatrix}
 * & * & 0 \\
 * & * & 0 \\
 0 & 0 & * 
 \end{bmatrix}
 \mbox{ \qquad and \qquad }
 \czer_G(a_2) =
 \begin{bmatrix}
 * & 0 & 0 \\
 0 & * & * \\
 0 & * & * 
 \end{bmatrix}
 .$$
 These generate~$G$.
 
 ($\Leftarrow$) Suppose $\Rrank G = 1$, so $\dim A = 1$. Then, since $A$ is almost Zariski closed (and contains $\langle a_1 \rangle$), we have 
 	$\czer_G(a_1) = \czer_G(A) = \czer_G(a_2)$,
so 
	$$\langle \czer_G(a_1) , \czer_G(a_2) \rangle = \czer_G(A) .$$
It is obvious that $\czer_G(A) \neq G$ (because the center of~$G$ is finite, and therefore cannot contain the infinite group~$A$).
\end{proof}

The following explicit description of $\czer_G(A)$ will be used in some of the proofs.

\begin{lem} \label{C(A)=AxC}
If $A$ is any maximal $\real$-split torus in~$G$, then $\czer_G(A) = A \times C$, where $C$~is compact.
\end{lem}

\begin{proof}[Proof]  \optional\ 
A subgroup of $\SL(\ell,\real)$ is said to be \defit{reductive} if it is isogenous to $M \times T$, where $M$~is semisimple and $T$~is a torus. It is known that the centralizer of any torus is reductive \csee{C(T)reductive}, so, if we assume, for simplicity, that $\czer_G(A)$ is connected, then we may write $\czer_G(A) = M \times A$, where $M$ is reductive \csee{C=MA}. The maximality of~$A$ implies that $M$ does not contain any $\real$-split tori, so $M$ is compact \csee{Rrank0Ex}.
\end{proof}

 \begin{prop}[\csee{Rrank2<>au=uaEx}] \label{Rrank2<>au=ua}
 $\Rrank G \ge 2$ if and only if there exist a nontrivial hyperbolic element~$a$ and
a nontrivial unipotent element~$u$, such that $au = ua$.
\end{prop}

For use in the proof of the \lcnamecref{Rrank2<>UnipGens} that follows it, we mention a very useful characterization of a somewhat different flavor:

\begin{lem}[\csee{Rrank1UniqMaxUnipEx}] \label{Rrank1UniqMaxUnip}
$\Rrank G \le 1$ if and only if every nontrivial unipotent subgroup of~$G$ is contained in a \textbf{unique} maximal unipotent subgroup.
\end{lem}

\begin{prop} \label{Rrank2<>UnipGens}
$\Rrank G \ge 2$ if and only if there exist nontrivial unipotent subgroups $U_1,\ldots,U_k$, such that 
	\begin{itemize}
	\item $\langle U_1,\ldots,U_k \rangle = G$,
	and
	\item $U_i$ centralizes~$U_{i+1}$ for each~$i$.
	\end{itemize}
\end{prop}

\begin{proof}
($\Rightarrow$) Assume, for simplicity, that $G = \SL(3,\real)$.
Then we take the sequence
 $$ 
 \begin{Smallbmatrix}
 1 & \upast & 0 \\
 0 & 1 & 0 \\
 0 & 0 & 1 
 \end{Smallbmatrix}
 , \ 
 \begin{Smallbmatrix}
 1 & 0 & \upast \\
 0 & 1 & 0 \\
 0 & 0 & 1 
 \end{Smallbmatrix}
 , \ 
 \begin{Smallbmatrix}
 1 & 0 & 0 \\
 0 & 1 & \upast \\
 0 & 0 & 1 
 \end{Smallbmatrix}
 , \ 
 \begin{Smallbmatrix}
 1 & 0 & 0 \\
 \upast & 1 & 0 \\
 0 & 0 & 1 
 \end{Smallbmatrix}
 , \ 
 \begin{Smallbmatrix}
 1 & 0 & 0 \\
 0 & 1 & 0 \\
 \upast & 0 & 1 
 \end{Smallbmatrix}
 , \ 
 \begin{Smallbmatrix}
 1 & 0 & 0 \\
 0 & 1 & 0 \\
 0 & \upast & 1 
 \end{Smallbmatrix}
 .$$

\medbreak

($\Leftarrow$) Since $U_i$ commutes with $U_{i+1}$, we know that $\langle U_i, U_{i+1} \rangle$ is unipotent, so, if $\Rrank G = 1$, then it is contained in a \emph{unique} maximal unipotent subgroup $\overline{U_i}$ of~$G$. Since $\overline{U_i}$ and $\overline{U_{i+1}}$ both contain~$U_{i+1}$, we conclude that $\overline{U_i} = \overline{U_{i+1}}$ for all~$i$. Hence, $\langle U_1,\ldots,U_k \rangle$ is contained in the unipotent group $\overline{U_1}$, and is therefore not all of~$G$.
 \end{proof}

\begin{rem}
See \cref{HighRankGenLi} for yet another result of the same type, which will be used in the proof of the Margulis Superrigidity Theorem in \cref{SuperPfSect}. A quite different characterization, based on the existence of subgroups of the form $\SL(2,\real) \ltimes \real^n$, appears in \cref{SL2RxRnInG}, and is used in proving Kazhdan's Property~$(T)$ in \cref{KazhdanTChap}.
\end{rem}

We know that $\SL(2,\real)$ is the smallest group of real rank one \csee{Rrank0<>SL2R}. However, the smallest group of real rank two is not unique:

\begin{prop} \label{Rrank2<>SL3orSO23}
Assume $G$ is simple. Then\/ $\Rrank G \ge 2$ if and only if $G$~contains a subgroup that is isogenous to either\/ $\SL(3,\real)$ or\/ $\SO(2,3)$.
\end{prop}


\begin{exercises}

\item \label{CentGensG1xG2Ex}
Prove the following results in the special case where $G = G_1 \times G_2$, and $\Rrank G_i \ge 1$ for each~$i$.
	\begin{enumerate}
	\item \label{CentGensG1xG2Ex-C(a)}
	\cref{Rrank2<>C(a)gens}($\Rightarrow$)
	\item \cref{Rrank2<>au=ua}($\Rightarrow$)
	\item \cref{Rrank1UniqMaxUnip}($\Leftarrow$)
	\item \cref{Rrank2<>UnipGens}($\Rightarrow$)
	\end{enumerate}

\item \label{C(T)reductive} \optional\ 
It is known that if $M$ is a subgroup that is almost Zariski closed, and $M^\transpose = M$, then $M$~is reductive \ccf{SelfAdj->SS}. Assuming this, show that if $T$ is a subgroup of the group of diagonal matrices, and $G^\transpose = G$, then $C_G(T)$ is reductive.

\item \label{C=MA} \optional\ 
Suppose $M$ is reductive, and $A$ is an $\real$-split torus in the center of~$M$. Show there exists a reductive subgroup $L$ of~$M^\circ$, such that $M^\circ = L \times A$.
\hint{Up to isogeny, write $M = M_0 \times T$, with $A \subseteq T$. Then it suffices to show $T = E \times A$ for some~$E$. You may assume, without proof, that, since $T$ is a connected, abelian Lie group, it is isomorphic to $\real^m \times \torus^n$ for some $m$ and~$n$.}

\item \label{Rrank2<>au=uaEx}
	\begin{enumerate}
	\item Prove \cref{Rrank2<>au=ua}($\Rightarrow$) under the additional assumption that $G = \SL(3,\real)$.
	\item \label{Rrank2<>au=uaEx-not1}
	Prove \cref{Rrank2<>au=ua}($\Leftarrow$).
%	\hint{\cref{ZarAandCent}.}
	\end{enumerate}

\item \label{Rrank1UniqMaxUnipEx}
Find a nontrivial unipotent subgroup of $\SL(3,\real)$ that is contained in two different maximal unipotent subgroups.

\item (\emph{Assumes the theory of real roots})
Prove the general case of the following results.
	\begin{enumerate}
	\item \cref{C(A)=AxC}
	\item \cref{Rrank2<>C(a)gens}($\Rightarrow$)
	\item \cref{Rrank2<>au=ua}($\Rightarrow$)
	\item \cref{Rrank1UniqMaxUnip}
	\item \cref{Rrank2<>UnipGens}($\Rightarrow$)
	\end{enumerate}

\item Show (without assuming $G$ is simple): $\Rrank G \ge 2$ if and only if $G$~contains a subgroup that is isogenous to either $\SL(3,\real)$ or $\SL(2,\real) \times \SL(2,\real)$.
\hint{\Cref{Rrank2<>SL3orSO23}. You may assume, without proof, that $\SO(2,2)$ is isogenous to $\SL(2,\real) \times \SL(2,\real)$.}


 \end{exercises}








\section{Groups of real rank one}

As a complement to \cref{HigherRrankSect}, here is an explicit list of the simple groups of real rank one.

\begin{thm} \label{rank1simple}
 If $G$ is simple, and\/ $\Rrank G = 1$, then $G$ is isogenous to
either
 \begin{itemize}
 \item $\SO(1,n)$ for some $n \ge 2$,
 \item $\SU(1,n)$ for some $n \ge 2$,
 \item $\Sp(1,n)$ for some $n \ge 2$, or
 \item $F_4^{-20}$ \textup(also known as~$F_{4,1}$\textup), a certain exceptional group.
 \end{itemize}
 \end{thm}

\begin{rem}
 The special linear groups $\SL(2,\real)$,
$\SL(2,\complex)$ and $\SL(2,\quaternion)$ have real rank
one, but they are already on the list under different names, because
 \begin{enumerate}
 \item $\SL(2,\real)$ is isogenous to $\SO(1,2)$ and
$\SU(1,1)$,
 \item $\SL(2,\complex)$ is isogenous to $\SO(1,3)$ and
$\Sp(1,1)$, and
 \item $\SL(2,\quaternion)$ is isogenous to $\SO(1,4)$.
 \end{enumerate}
 \end{rem}

% \begin{cor}
% $\Rrank(G) = 1$ if and only if $G$ is isogenous to a direct
%product $G_0 \times G_1$, where
% \begin{itemize}
% \item $G_0$ is compact and
% \item $G_1$ is of one of the groups listed in
%\cref{rank1simple}.
% \end{itemize}
% \end{cor}

\begin{rem}
 Each of the simple groups of real rank one has a very important geometric realization. Namely, $\SO(1,n)$, $\SU(1,n)$, $\Sp(1,n)$, and $F_{4,1}$ (respectively) are isogenous to the isometry groups of:
	 \begin{enumerate}

	\item (real) \term[hyperbolic!space!real]{hyperbolic $n$-space} $\hyperbolic^n$,
	 \item \defit[hyperbolic!space!complex]{complex hyperbolic $n$-space} $\complex\hyperbolic^n$,
	 \item \defit[hyperbolic!space!complex]{quaternionic hyperbolic $n$-space} $\quaternion\hyperbolic^n$,
	 and
	 \item the \index{hyperbolic!plane, octonionic}\defit[Cayley!plane]{Cayley plane}, which can be thought of as the hyperbolic plane over the (nonassociative) ring~$\octonion$ of Cayley \term{octonions}.
	 \end{enumerate}
%(It is not possible to define octonionic hyperbolic spaces of higher dimension, only the plane, because the ring $\octonion$ is not associative.) 
\end{rem}









\section{Minimal parabolic subgroups} \label{ParabSubgrpSect}

The group of upper-triangular matrices plays a very important role in the study of $\SL(n,\real)$. In this section, we introduce subgroups that play the same role in other semisimple Lie groups:

\begin{defn} \label{MinParabDefn}
Let $A$ be a maximal $\real$-split torus of~$G$, and let $a$ be a \defit[generic element]{generic} element of~$A$, by which we mean that $\czer_G(a) = \czer_G(A)$. Then the corresponding \defit[parabolic!subgroup!minimal]{minimal parabolic subgroup} of~$G$ is
	\begin{align} \label{P=horo}
	P = \bigset{ g \in G }{ \limsup_{n \to \infty} \| a^{-n} g a^n \| < \infty } 
	. \end{align}
This is a Zariski closed subgroup of~$G$.
 \end{defn}

\begin{thm} \label{MinParabConj}
All minimal parabolic subgroups of~$G$ are conjugate.
\end{thm}

\begin{egs} \ \label{ParabEgs}
 \noprelistbreak
 \begin{enumerate}

\item \label{ParabEgs-SLn}
The group of upper triangular matrices is a minimal parabolic subgroup of $\SL(n,\real)$. To see this, let $A$ be the group of diagonal matrices, and choose $a \in A$ with $a_{1,1} > a_{2,2} > \cdots > a_{n,n} > 0$ \csee{MinParabSLnEx}. 

 \item \label{ParabEgs-SO1n}
 It is easier to describe a minimal parabolic subgroup of $\SO(1,n)$ if
we replace $\Id_{m,n}$ with a different symmetric matrix of
the same signature: let 
 $G = \SO(A;\real)$, for
 $$ A =
 \begin{pmatrix}
 0 & 0 & 1 \\
 0 & \Id_{(n-1) \times (n-1)} & 0 \\
 1 & 0 & 0
 \end{pmatrix}
 .$$
 Then $G$ is conjugate to $\SO(1,n)$ \csee{SOA=SOmn},
 the ($1$-dimensional) group of diagonal matrices in~$G$ form a maximal $\real$-split torus, and a minimal parabolic subgroup in~$G$ is
 $$\left\{\begin{pmatrix}
 t&*&* \\
 0& \SO(n-1) & * \\
 0&0&1/t
 \end{pmatrix}
 \right\}
 $$
 \csee{MinParabSO1nEx}.
 \end{enumerate}
 \end{egs}

The following result explains that a minimal parabolic subgroup of a
classical group is simply the stabilizer of a (certain kind
of) flag. 
%(For the general case, including exceptional groups, the theory of real roots gives a good understanding of the parabolic subgroups.)
Recall that
%, if $\langle \cdot \mid \cdot \rangle$ is a bilinear (or Hermitian) form on a vector space~$V$. A 
a subspace $W$
of a vector space~$V$, equipped with a bilinear (or Hermitian) form $\langle \cdot \mid \cdot \rangle$, is said to be \defit[totally!isotropic]{totally isotropic} if $\langle W \mid W \rangle = 0$.

\begin{thm}[\csee{parab=Stab(flag)Ex}] \ \label{parab=Stab(flag)}
\noprelistbreak
 \begin{enumerate}
 \item \label{parab=Stab(flag)-SLn} 
 A subgroup~$P$ of\/ $\SL(n,\real)$ is a minimal parabolic if and
only if there is a chain $V_0 \subsetneq V_1 \subsetneq \cdots
\subsetneq V_n$ of subspaces of\/ $\real^n$ \textup(with $\dim V_i = i$\textup), such that
 $$ P = \{\, g \in \SL(n,\real) \mid
 \forall i, \ g V_i = V_i \,\} .$$
Similarly for\/ $\SL(n,\complex)$ and\/ $\SL(n,\quaternion)$,
taking chains of subspaces in\/ $\complex^n$
or\/~$\quaternion^n$, respectively.

\item \label{parab=Stab(flag)-SOmn} 
A subgroup~$P$ of\/ $\SO(m,n)$ is a minimal parabolic if and only
if there is a chain $V_0 \subsetneq V_1 \subsetneq \cdots
\subsetneq V_r$ of \textbf{totally isotropic} subspaces of $\real^{m+n}$ \textup(with $\dim V_i = i$ and $r = \min\{m,n\}$\textup), such that
 $$ P = \{\, g \in \SO(m,n) \mid
 \forall i, \ g V_i = V_i \,\} .$$
Similarly for\/ $\SO(n,\complex)$, $\SO(n,\quaternion)$,
$\Sp(2m,\real)$, $\Sp(2m,\complex)$, $\SU(m,n)$ and\/
$\Sp(m,n)$.
 \end{enumerate}
 \end{thm}

%Note that, in each of these cases, we may write $P = MAN$, where
%	\begin{itemize}
%	\item $M$ is compact,
%	\item $A$ is a maximal $\real$-split torus of~$G$,
%	and
%	\item $N$ is a maximal unipotent subgroup of~$G$.
%	\end{itemize}
%
%The group of upper triangular matrices in $\SL(n,\real)$ is of the form $AN$, where
%	\begin{itemize}
%	\item $A$ is a maximal $\real$-split torus,
%	and
%	\item $N$ is a maximal unipotent subgroup that is normalized by~$A$.
%	\end{itemize}
%This description can be extended to the minimal parabolic subgroups of any semisimple group, except that we need to replace $A$ with the slightly larger group $\czer_G(A)$:

%\begin{thm}
%$P$ is a minimal parabolic subgroup of~$G$ if and only there exist%
%	\begin{itemize}
%	\item a maximal $\real$-split torus $A$,
%	and
%	\item a maximal unipotent subgroup~$N$ that is normalized by~$A$,
%	\end{itemize}
%such that $P = \czer_G(A) \, N$.
%\end{thm}

Note that any upper triangular matrix in $\SL(n,\real)$ can be written uniquely in the form $mau$, where
	\begin{itemize}
	\item $a$~belongs to the $\real$-split torus~$A$ of diagonal matrices whose nonzero entries are positive,
	\item $m$ is in the finite group~$M$ consisting of diagonal matrices whose nonzero entries are $\pm1$,
	and
	\item $u$ belongs to the unipotent group~$N$ of upper triangular matrices with $1$'s on the diagonal.
	\end{itemize}
The elements of every minimal parabolic subgroup have a decomposition of this form, except that the subgroup~$M$ may need to be compact, instead of only finite:

\begin{thm}[(\thmindex{Langlands decomposition}Langlands decomposition)] \label{LanglandsDecomp}
 If $P$ is a minimal parabolic subgroup of~$G$, then we may write it in the form
 $P = \czer_G(A)\, N = MAN$, where%
	 \begin{itemize}
	 \item $A$~is a maximal $\real$-split torus, 
	 \item $M$ is a compact subgroup of $\czer_G(A)$,
	 and 
	 \item $N$ is the unique maximal unipotent subgroup of~$P$.
	 \end{itemize}
 Furthermore, $N$ is a maximal unipotent subgroup of~$G$, and, for some generic $a \in A$, we have
 	\begin{align} \label{LanglandsDecomp-N}
	N = \bigset{ u \in G }{ \lim_{n \to \infty} a^{-n} u a^n = e } 
	. \end{align}
 \end{thm}
 
 Before discussing the proof (which is not so important for our purposes), let us consider a few examples:

\begin{eg} \label{MinParabEg} \ 
\noprelistbreak
	\begin{enumerate}
	\item If $G = \SL(n,\complex)$, then, for the Langlands decomposition of the group~$P$ of upper-triangular matrices, we may let:
		\begin{itemize}
		\item $A$ be the group of diagonal matrices in~$G$ whose nonzero entries are positive real numbers (just as for $\SL(n,\real)$),
		\item $M$ be the group of diagonal matrices in~$G$ whose nonzero entries have absolute value~$1$,
		and
		\item $N$ be the group of upper triangular matrices with $1$'s on the diagonal.
		\end{itemize}
	The same description applies to $G = \SL(n,\quaternion)$ (and, actually, also to $\SL(n,\real)$).
	
	\item \label{MinParabEg-SOA}
	Assume $m \le m$, and let $G = \SO(A; \real)$, where 
		$$ A = \begin{bmatrix} 
		0 & 0 & J_m \\
		0 & \Id_{(n-m) \times (n-m)} & 0 \\
		J_m & 0 & 0
		\end{bmatrix}
		\quad  \text{and} \quad
		J_m = \left[\begin{smallmatrix}
		&&&&1 \\
		&\vbox to 0pt{\vss\hbox to 0pt{\Large\hskip-5pt0\hss}\vss}&&1 \\
		&&\nedots \\
		& 1 & \vbox to 0pt{\vss\hbox to 0pt{\Large\hskip10pt 0\hss}\vss\vskip3pt}\\
		 1 
		\end{smallmatrix}\right]
		$$
	(and the size of the matrix $J_m$ is $m \times m$). Then $G$ is conjugate to $\SO(m,n)$ \csee{SOA=SOmn}, and a minimal parabolic~$P$ of~$G$ is:
	$$  \bigset{\begin{pmatrix}
	 b&*& * \\
	 0& k & * \\
	 0&0&b^\dagger
	 \end{pmatrix}
	 }{ 
	 \begin{matrix}
	 \text{$b \in \GL(m,\real)$ is upper triangular} , \\
	 k \in \SO(n-m) \\
	 \end{matrix}}
	 $$
	 where $x^\dagger = J_m (x^{-1})^\transpose J_m$,
	 %denotes the reflection of~$x$ across the aniti-diagonal 
	 so, for example, 
	 	$$\diag(b_1,\ldots,b_m)^\dagger = \diag(1/b_m,\ldots,1/b_1) .$$
	Hence, we may let
		\begin{itemize}
		\item $A = \{\, \diag(a_1,\ldots,a_m, 0, \ldots,0, 1/a_m, \ldots,1/a_1) \mid a_i > 0 \,\}$,
		\item $M \iso \SO(n-m) \times \{\pm1\}^m$,
		and
		\item $N$ be the group of upper triangular matrices with $1$'s on the diagonal that are in~$G$.
		\end{itemize}
	\end{enumerate}
\end{eg}
 
 \begin{proof}[Proof of \cref{LanglandsDecomp} \optional]
 Choose a generic element~$a$ of~$A$ satisfying \pref{P=horo}, and define $N$ as in \pref{LanglandsDecomp-N}. Then, since $a$~is diagonalizable over~$\real$, it is not difficult to see that $P = \czer_G(a)\, N$ \csee{P=CN}.
Since $a$~is a generic element of~$A$, this means $P = \czer_G(A)\, N$.

It is easy to verify that $N$ is normal in~$P$ \csee{NnormP}; then, since $P/N \iso \czer_G(A) = A \times (\text{compact})$ \csee{C(A)=AxC}, and therefore has no nontrivial unipotent elements, it is clear that $N$ contains every unipotent element of~$P$. 
Conversely, the definition of~$N$ implies that it is unipotent \csee{NUnip}.
Therefore, $N$~is the unique maximal unipotent subgroup of~$P$.

Suppose $U$ is a unipotent subgroup of~$G$ that properly contains~$N$.
Since unipotent subgroups are nilpotent \csee{UnipNilp}, then $\nzer_U(N)$ properly contains~$N$ \csee{NilpNorm}.
However, it can be shown that $\nzer_G(N) = P$ \csee{P=normalizer}, so this implies $\nzer_U(N)$ is a unipotent subgroup of~$P$ that properly contains~$N$, 
which contradicts the conclusion of the preceding paragraph. % !!!
 \end{proof}

The subgroups $A$ and~$N$ that appear in the Langlands decomposition of~$P$ are two components of the Iwasawa decomposition of~$G$:

\begin{thm}[(\thmindex{Iwasawa decomposition}Iwasawa decomposition)] \label{IwasawaDecomp}
 Let 
 \noprelistbreak
 \begin{itemize}
 \item $K$ be a maximal compact subgroup of~$G$,
 \item $A$ be a maximal $\real$-split torus, 
 and
 \item $N$ be a maximal unipotent subgroup that is normalized by~$A$.
 \end{itemize}
Then $G = K A N$.

 In fact, every $g \in G$ has a \bemph{unique} representation of the form $g = k a u$ with $k \in K$, $a \in A$, and $u \in N$. 
 \end{thm}

\begin{rem} \label{IwasawaDiffeo}
 A stronger statement is true: if we define a function $\varphi \colon K \times A \times N \to G$ by\/
 $\varphi(k,a,u) = kau$, then $\varphi$ is a (real analytic) diffeomorphism. Indeed, \cref{IwasawaDecomp} tells us that $\varphi$ is a bijection, and it is obviously real analytic. It is not so obvious that the inverse of~$\varphi$ is also real analytic, but this is proved in \cref{IwasawContinuousSLnR} when $G = \SL(n,\real)$, and the general case can be obtained by choosing an embedding of~$G$ in $\SL(n,\real)$ for which the subgroups $K$, $A$, and~$N$ of~$G$ are equal to the intersection of~$G$ with the corresponding subgroups of $\SL(n,\real)$.
 \end{rem}

The Iwasawa decomposition implies $KP = G$ (since $AN \subseteq P$), so it has the following important consequence:

\begin{cor} \label{G/Pcpct}
If $P$ is any minimal parabolic subgroup of~$G$, then $G/P$ is compact.
\end{cor}

\begin{rem} \label{ParabRem}
A subgroup of~$G$ is called \defit[parabolic!subgroup]{parabolic} if it contains a minimal parabolic subgroup. 
	\begin{enumerate}

	\item \label{ParabRem-cocpct}
	\Cref{G/Pcpct} implies that if $Q$ is any parabolic subgroup, then $G/Q$ is compact. The converse
does not hold. (For example, if $P = MAN$ is a minimal
parabolic, then $G/(AN)$ is compact, but $AN$~is not
parabolic unless $M$ is trivial.) However, passing to the
``\term{complexification}'' does yield the converse: $Q$ is parabolic if
and only if $G_{\complex}/Q_{\complex}$ is compact.
Furthermore, $Q$ is parabolic if and only if $Q_{\complex}$
contains a maximal solvable subgroup (``\defit[Borel!subgroup]{Borel subgroup}'') of~$G_{\complex}$.

	\item All parabolic subgroups can be described fairly completely (there are only finitely many that contain any given minimal parabolic), but we do not need the more general theory.
	\end{enumerate}
 \end{rem}

\begin{exercises}

\item \label{MinParabSLnEx}
Let $a$ be a diagonal matrix as described in \fullcref{ParabEgs}{SLn}, and show that the corresponding minimal parabolic subgroup is precisely the group of upper triangular matrices.

\item \label{MinParabSO1nEx}
Show that the subgroup at the end of \fullcref{ParabEgs}{SO1n} is indeed a minimal parabolic subgroup of $\SO(A; \real)$.

\item \label{parab=Stab(flag)Ex}
Show the minimal parabolic subgroups of each of the following groups are as described in \cref{parab=Stab(flag)}:
	\begin{enumerate}
	\item $\SL(n,\real)$.
	\item $\SO(m,n)$.
	\end{enumerate}
\hint{It suffices to find one minimal parabolic subgroup in order to understand all of them \csee{MinParabConj}.}

\item \label{SOA=SOmn}
For $A$ as in \fullcref{MinParabEg}{SOA}, show that $\SO(A; \real)$ is conjugate to $\SO(m,n)$.
\hint{Let $\alpha = 1/\sqrt{2}$, and define $v_i$ to be: $\alpha(e_i + e_{n+1-i})$ for $i \le m$, $e_i$~for $m < i \le n$, and $\alpha(e_i - e_{n+1-i})$ for $i > n$. Then $v_i^\transpose A v_i$ is~$1$ for $i \le n$, and is $-1$~for $i > n$.}

\item \label{P=CN} \optional\ 
For $P$, $a$, and~$N$ as in the proof of \cref{LanglandsDecomp}, show $P = \czer_G(a)\, N$.
\hint{Given $g \in P$, show that $a^{-n} g a^n$ converges to some element~$c$ of $C_G(a)$. Also show $c^{-1} g \in N$. You may assume $a$~is diagonal, with $a_{11} \ge a_{22} \ge \cdots \ge a_{\ell\ell}$ (\emph{why?}).}

\item \label{NnormP}
For $P$, $a$, and~$N$ as in the proof of \cref{LanglandsDecomp}, show $N$ is a normal subgroup of~$P$.

\item \label{NUnip}
 Show that a subgroup~$N$ satisfying \pref{LanglandsDecomp-N} must be unipotent.
 \hint{$u$~has the same characteristic polynomial as~$a^{-n} u a^n$.} 

\item \label{P=normalizer}
For $P$ and~$N$ as in \fullcref{parab=Stab(flag)}{SOmn}, show $P = \nzer_G(N)$.
\hint{$P$~is the stabilizer of a certain flag, and the subgroup~$N$ also uniquely determines this same flag.}

\item \label{UnipNilp}
Show that every unipotent subgroup of $\SL(\ell,\real)$ is nilpotent.
(Recall that a group~$N$ is \defit[nilpotent group]{nilpotent} if there is a series 
	$$\{e\} = N_0 \normal \cdots \normal N_r = N$$
of subgroups of~$N$, such that $[N, N_k] \subseteq N_{k-1}$ for each~$k$.)
\hint{Engel's Theorem \pref{EngelUnip}.}

\item \label{NilpNorm}
Show that if $N$ is a proper subgroup of a nilpotent group~$U$, then $\nzer_U(N) \not\subseteq N$.
\hint{If $[N, U_k] \subseteq N$, then $U_k$ normalizes~~$N$.}

\item  \label{G=KxRn}
Assume $K$ is a maximal compact subgroup of~$G$. Show:
	\noprelistbreak
	\begin{enumerate}
	\item $G$ is diffeomorphic to the cartesian product $K \times \real^n$, for
some~$n$,
	\item \label{G=KxRn-G/K}
	$G/K$ is diffeomorphic to $\real^n$, for some~$n$,
	\item $G$ is connected if and only if $K$ is connected,
	and 
	\item $G$ is simply connected if and only if $K$ is simply connected.
	\end{enumerate}
\hint{\cref{IwasawaDiffeo}.}

%\item \label{G/Ksc}
%Show that if $K$ is any maximal compact subgroup of~$G$, then $G/K$ is connected and simply connected.
%\hint{The Iwasawa decomposition implies that $G/K$ is homeomorphic to $AN$. The subgroups $A$ and~$N$ are each homeomorphic to some Euclidean space. So, in fact, $G/K$ is homeomorphic to some~$\real^k$, and is therefore contractible, but we do not need this fact.}

\end{exercises}



\begin{notes}

The comprehensive treatise of Borel and Tits \cite{BorelTits-GrpRed} is the standard reference on rank, parabolic subgroups, and other fundamental properties of reductive groups over any field.
See \cite[\S7.7, pp.~474--487]{Knapp-BeyondIntro} for a discussion of parabolic subgroups of Lie groups (which is the special case in which the field is~$\real$). 

\fullCref{RrankRems}{Rrank=CoverGamma} is due to Prasad-Raghunathan \cite[Thms.~2.8 and 3.9]{PrasadRaghunathan-Cartan+Latts}.

Proofs of the Iwasawa decomposition for both $\SL(n,\real)$ \pref{IwasawaDecompSLnR} and the general case \pref{IwasawaDecomp} can be found in \cite[Prop.~3.12, p.~129, and Thm.~3.9, p.~131]{PlatonovRapinchukBook}. (Iwasawa's original proof is in \cite[\S3]{Iwasawa-SomeTypes}.) The decomposition also appears in many textbooks on Lie groups. In particular, \cref{IwasawaDiffeo} is proved in \cite[Thm.~6.5.1, pp.~270--271]{HelgasonBook}. 
% Since $A$ and~$N$ are connected, the decomposition implies that every maximal compact subgroup of~$G$ intersects all of the components of~$G$.

Regarding \fullcref{ParabRem}{cocpct}, the obvious cocompact subgroups of~$G$ are parabolic subgroups and (cocompact) lattices. See \cite{Witte-cocpct} for a short proof that every cocompact subgroup is a combination of these two types. (A similar result had  been proved previously in \cite[(5.1a)]{GotoWang-NondiscreteUniform}.)

\end{notes}





\begin{references}{9}

\bibitem{BorelTits-GrpRed}
A.\,Borel and J.\,Tits:
Groupes r\'eductifs,
\emph{Inst. Hautes \'Etudes Sci. Publ. Math.} 27 (1965) 55--150.
\MR{207712},
\maynewline
\url{http://www.numdam.org/item?id=PMIHES_1965__27__55_0}

\bibitem{GotoWang-NondiscreteUniform}
M.\,Goto and H.-C.\,Wang:
Non-discrete uniform subgroups of semisimple Lie groups,
\emph{Math. Ann.} 198 (1972) 259--286.
\MR{0354934},
\maynewline
\url{http://resolver.sub.uni-goettingen.de/purl?GDZPPN002306697}

\bibitem{HelgasonBook} 
S.\,Helgason:
\emph{Differential Geometry, Lie Groups, and Symmetric Spaces.}
% Pure and Applied Mathematics, 80. 
Academic Press, New York, 1978.
ISBN 0-12-338460-5,
\MR{0514561}

\bibitem{Iwasawa-SomeTypes}
K.\,Iwasawa:
On some types of topological groups,
\emph{Ann. of Math.} (2) 50 (1949) 507--558. 
\MR{0029911},
\maynewline
\url{http://dx.doi.org/10.2307/1969548}

\bibitem{Knapp-BeyondIntro}
A.\,W.\,Knapp:
\emph{Lie Groups Beyond an Introduction, 2nd ed.}. 
%Progress in Mathematics, 140. 
Birkhäuser, Boston, 2002. 
ISBN 0-8176-4259-5,
\MR{1920389}

\bibitem{PlatonovRapinchukBook}
 V.\,Platonov and A.\,Rapinchuk: 
 \emph{Algebraic Groups and Number Theory.}
 Academic Press, Boston, 1994.
 ISBN 0-12-558180-7,
 \MR{1278263}
 
 \bibitem{PrasadRaghunathan-Cartan+Latts}
G.\,Prasad and M.\,S.\,Raghunathan:
Cartan subgroups and lattices in semi-simple groups,
\emph{Ann. of Math.} (2) 96 (1972), 296--317. 
\MR{0302822},
\url{http://dx.doi.org/10.2307/1970790}

\bibitem{Witte-cocpct}
D.\,Witte:
Cocompact subgroups of semisimple Lie groups,
in 
G.\,Benkart and J.\,M.\,Osborn, eds.:
\emph{Lie Algebra and Related Topics (Madison, WI, 1988)}.
%Contemp. Math., 110, 
Amer. Math. Soc., Providence, RI, 1990, pp.~309--313. 
ISBN 0-8218-5119-5,
\MR{1079114}


 \end{references}
