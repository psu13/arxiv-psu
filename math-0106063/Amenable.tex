%!TEX root = IntroArithGrps.tex

\mychapter{Amenable Groups}
\label{AmenableChap}

\prereqs{none.}
	%, but unitary representations (\cref{UnitaryRepSect}) and quasi-isometries (\cref{QuasiChap}) are mentioned.}


The classical {Kakutani-Markov Fixed Point Theorem} \pref{AbelAmen} implies that any abelian group of continuous linear operators has a fixed point in any compact, convex, invariant set. This theorem can be extended to some non-abelian groups; the groups that satisfy such a fixed-point property are said to be ``amenable\zz,'' and they have quite a number of interesting features. 
Many important subgroups of~$G$ are amenable, so the theory is directly relevant to the study of arithmetic groups, even though we will see that $G$ and~$\Gamma$ are usually not amenable. In particular, the theory yields an important equivariant map that will be constructed in \cref{AmenEquiMapSect}.



\section{Definition of amenability}

\begin{assump}
Throughout this chapter, $H$ denotes a Lie group.
The ideas here are important even in the special case where $H$ is discrete.
\end{assump}

\begin{defn} \label{CpctCnvxDefn}
Suppose $H$ acts continuously (by linear maps) on a locally convex topological vector space~$\LocConvex$.
Every $H$-invariant, compact, convex subset of~$\LocConvex$ is called a \defit[compact!convex $H$-space]{compact, convex $H$-space}.
\end{defn}

\begin{defn} \label{AmenDefn}
$H$ is \defit[amenable group]{amenable} if and only if $H$ has a fixed point in every nonempty, compact, convex $H$-space.
\end{defn}

This is just one of many different equivalent definitions of amenability. (A few others are discussed in \cref{OtherAmen}.) The equivalence of these diverse definitions is a testament to the fact that this notion is very fundamental. 

\begin{rems} \ 
\noprelistbreak
	\begin{enumerate}

	\item All locally convex topological vector spaces are assumed to be Hausdorff.

	\item In most applications, the locally convex space~$\LocConvex$ is the dual of a separable Banach space, with the weak$^*$ topology \csee{WeakStarDefn}.  In this situation, every compact, convex subset~$C$ is second countable, and is therefore metrizable \csee{Metrizable<>2ndCount}.
	With these thoughts in mind, we feel free to assume metrizability when it eliminates technical difficulties in our proofs.
In fact, we could restrict to these spaces in our definition of amenability, because it turns out that this modified definition results in exactly the same class of groups (if we only consider groups that are second countable) \csee{MetrizableSuffices}.

\item The choice of the term ``\term[amenable group]{amenable}'' seems to have been motivated by two considerations:%
\noprelistbreak
		\begin{enumerate}
		\item The word ``amenable'' can be pronounced ``a-MEAN-able\zz,'' and we will see in \cref{OtherAmen} that a group is amenable if and only if it admits certain types of means.
		\item One definition of ``amenable'' from the \emph{Oxford American Dictionary} is ``capable of being acted on a particular way\zz.'' In other words, in colloquial English, something is ``amenable'' if it is easy to work with. Classical analysis has averaging theorems and other techniques that were developed for the study of functions on the group~$\real^n$. Many of these methods can be generalized to all amenable groups, so amenable groups are easy to work with.
		\end{enumerate}
	\end{enumerate}
\end{rems}

\begin{exercises}

\item Show that every finite group is amenable.
\hint{For some $c_0 \in C$, let 
	$c = \frac{1}{\#H} \sum_{h\in H} hc_0 $.
Then $c \in C$ and $c$ is fixed by~$H$.}

\item \label{QuotAmen}
Show that quotients of amenable groups are amenable. That is, if $H$ is amenable, and $N$ is any closed, normal subgroup of~$H$, then $H/N$ is amenable.

\item Suppose $H_1$ is amenable, and there is a continuous homomorphism $\varphi \colon H_1 \to H$ with dense image. Show $H$ is amenable.

\end{exercises}






\section{Examples of amenable groups}

In this section, we will see that:
\noprelistbreak
\begin{itemize}
\item abelian groups are amenable \see{AbelAmen},
\item compact groups are amenable \see{CpctAmen},
\item solvable groups are amenable \see{AmenExtCor}, because the class of amenable groups is closed under extensions \see{AmenExtension},  
and
\item closed subgroups of amenable groups are amenable \see{SubgrpAmen}.
\end{itemize}
On the other hand, however, it is important to realize that not all groups are amenable. In particular, we will see in \cref{NonamenSect} that:
	\noprelistbreak
	\begin{itemize}
	\item nonabelian free groups are not amenable, 
	and
	\item $\SL(2,\real)$ is not amenable.
	\end{itemize}

\bigbreak

We begin by showing that $\integer$ is amenable:

\begin{prop} \label{CyclicAmen}
Cyclic groups are amenable.
\end{prop}

\begin{proof}
Assume $H = \langle T \rangle$ is cyclic. Given a nonempty, compact, convex $H$-space~$C$, choose some $c_0 \in C$.  For $n \in \natural$, let
	\begin{equation} \label{AmenInvt-c_n}
	c_n  =  \frac{1}{n+1}  \sum_{k = 0}^{n} T^k(c) 
	. \end{equation}
Since $C$ is compact, the sequence $\{c_n\}$ must have an accumulation point $c \in C$. It is not difficult to see that $c$ is fixed by~$T$ \csee{lim(c_n)invt}. Since $T$ generates~$H$, this means that $c$ is a fixed point for~$H$.
\end{proof}

\begin{cor}[(\thmindex{Kakutani-Markov Fixed Point}Kakutani-Markov Fixed Point Theorem)] \label{AbelAmen}
Every abelian group is amenable.
\end{cor}

\begin{proof}
Let us assume $H = \langle g,h \rangle$ is a $2$-generated abelian group. (See \cref{AbelAmenEx} for the general case.) If $C$ is any nonempty, compact, convex $H$-space, then \cref{CyclicAmen} implies that the set $C^g$ of fixed points of~$g$ is nonempty. It is easy to see that $C^g$ is compact and convex \csee{FixedPtsCpctConv}, and, because $H$ is abelian, that $C^g$ is invariant under~$h$ \csee{FixedPtsInvCent}. Hence, $C^g$ is a nonempty, compact, convex $\langle h \rangle$-space. Therefore, \cref{CyclicAmen} implies that $h$ has a fixed point~$c$ in $C^g$. Now $c$ is fixed by~$g$ (because it belongs to~$C^g$), and $c$ is fixed by~$h$ (by definition), so $c$ is fixed by $\langle g,h \rangle = H$.
\end{proof}

Compact groups are also easy to work with:

\begin{prop} \label{CpctAmen}
Compact groups are amenable.
\end{prop}

\begin{proof}
Assume $H$ is compact, and let $\mu$ be a Haar measure on~$H$. 
Given a nonempty, compact, convex $H$-space~$C$, choose some $c_0 \in C$.
Since $\mu$ is a probability measure, we may let 
	\begin{equation} \label{CenterOfHOrbit}
	c = \int_H h(c_0) \, d\mu(h) \in C
	. \end{equation}
(In other words, $c$ is the center of mass of the $H$-orbit of~$c_0$.)
The $H$-invariance of~$\mu$ implies that $c$ is a fixed point for~$H$ \csee{InvtMeas->FP}.
\end{proof}

It is easy to show that amenable extensions of amenable groups are amenable \csee{AmenExtensionEx}:

\begin{prop} \label{AmenExtension}
 If $H$ has a closed, normal subgroup~$N$, such that $N$ and $H/N$ are amenable, then $H$ is amenable.
\end{prop}

Combining the above results has the following consequences:

\begin{cor} \label{AmenExtCor} \ 
\noprelistbreak
\begin{enumerate}
\item \label{AmenExtCor-Solvable} 
 Every solvable group is amenable.

\item \label{AmenExtCor-CocpctSolv}
If $H$ has a solvable, normal subgroup~$N$, such that $H/N$ is compact, then $H$ is amenable.

\end{enumerate}
\end{cor}

\begin{proof}
\Cref{SolvAmen,CocpctSolv->Amen}.
\end{proof}

The converse of \fullcref{AmenExtCor}{CocpctSolv} is true for connected groups \csee{ConnAmen}.

\begin{prop} \label{SubgrpAmen}
Every closed subgroup of any amenable group is amenable.
\end{prop}

\begin{proof}
This proof employs a bit of machinery, so we postpone it to \cref{AmenSubgrpSect}. 
(For discrete groups, the result follows easily from some other characterizations of amenability; see \cref{FinAddRem,FolnerSubgrp} below.)
\end{proof}




\begin{exercises}

\item \label{lim(c_n)invt}
Suppose $T$ is a continuous linear map on a locally convex space~$\LocConvex$. Show that if $c$ is any accumulation point of the sequence $\{c_n\}$ defined by \pref{AmenInvt-c_n}, then $c$ is $T$-invariant.
\hint{If $\| c_n - c \|$ is small, then $\|T(c_n) - T(c)\|$ is small. Show that $\|T(c_n) - c_n\|$ is small whenever $n$~is large. Conclude that $\|T(c) - c \|$ is smaller than every~$\epsilon$.}

\item \label{FixedPtsCpctConv}
Suppose $C$ is a compact, convex $H$-space. Show that the set $C^H$ of fixed points of~$H$ is compact and convex.
\hint{Closed subsets of~$C$ are compact.}

\item \label{FixedPtsInvCent}
Suppose $H$ acts on a space~$C$, $A$~is a subgroup of~$H$, and $h$~is an element of the centralizer of~$A$. Show that the set $C^A$ of fixed points of~$A$ is invariant under~$h$.

\item \label{FixedPtsInvNorm}
Establish \cref{FixedPtsInvCent} under the weaker assumption that $h$ is an element of the \emph{normalizer} of~$A$, not the centralizer.

\item \label{AbelAmenEx}
Prove \cref{AbelAmen}.
\hint{For each $h \in H$, let $C^h$ be the set of fixed points of~$h$. The given argument implies (by induction) that $\{\, C^h \mid h \in H \,\}$ has the finite intersection property, so the intersection of these fixed-point sets is nonempty.}

\item \label{InvtMeas->FP}
Show that if $\mu$ is the Haar measure on~$H$, and $H$ is compact, then the point~$c$ defined in \pref{CenterOfHOrbit} is fixed by~$H$.

\item \label{AmenExtensionEx}
Prove \cref{AmenExtension}.
\hint{\Cref{FixedPtsCpctConv,FixedPtsInvNorm}.}

\item Show that $H_1 \times H_2$ is amenable if and only if $H_1$ and~$H_2$ are both amenable.

\item \label{SolvAmen}
Prove \fullcref{AmenExtCor}{Solvable}.
\hint{\Cref{AmenExtension}.}

\item \label{CocpctSolv->Amen}
Prove \fullcref{AmenExtCor}{CocpctSolv}.
\hint{\Cref{AmenExtension}.}

\item Suppose $H$ is discrete, and $H_1$ is a finite-index subgroup. Show $H$ is amenable if and only if $H_1$ is amenable.

\item Show that if $\Lambda$ is a lattice in~$H$, and $\Lambda$ is amenable, then $H$ is amenable.
\hint{Let $\mu = \int_{H/\Lambda} hv \, d h$, where $v$ is a fixed point for~$\Lambda$.}

\item Assume $H$ is discrete. Show that if every finitely generated subgroup of~$H$ is amenable, then $H$ is amenable.
\hint{For each $h \in H$, let $C^h$ be the set of fixed points of~$h$. Then $\{\, C^h \mid h \in H \,\}$ has the finite intersection property, so $\bigcap_h C^h \neq \emptyset$.}

%\item Is every finite group amenable?

%\item Is $\real$ amenable.

\item \label{BorelSL3Amen}
Let \ 
	$P = \begin{Smallbmatrix} \upast&& \\ \upast&\upast& \\ \upast&\upast&\upast \end{Smallbmatrix} \subset \SL(3,\real) $. \ 
Show that $P$ is amenable.
\hint{$P$ is solvable.}

\item Assume there exists a discrete group that is not amenable. Show the free group~$\free_2$ on $2$ generators is not amenable.
\hint{$\free_n$ is a subgroup of~$\free_2$.}

\item Assume there exists a Lie group that is not amenable. 
\noprelistbreak
	\begin{enumerate}
	\item Show the free group~$\free_2$ on $2$ generators is not amenable.
	\item Show $\SL(2,\real)$ is not amenable.
	\end{enumerate}
%\hint{You may use \cref{SubgrpAmen}.}

%\item \label{FPNonSep}
%Suppose $H$ acts continuously (in norm) on a Banach space~$\Banach$, and $C$ is a weak$^*$ compact, compact, convex, $H$-invariant subset of the dual space~$\Banach^*$. 
%Show that if $H$ is amenable, then $H$ has a fixed point in~$C$, even though $\Banach^*$ may not be Fréchet (because $\Banach$ may not be separable).
%\hint{For each $H$-invariant, separable subspace~$\LocConvex$ of~$\Banach$, let $C_{\LocConvex}^H \subseteq \Banach^*$ consist of the elements of~$C$ whose restriction to~$\LocConvex$ is $H$-invariant. Amenability implies that $\{ C_{\LocConvex}^H \}$ has the finite-intersection property, so some element of~$C$ belongs to every $C_{\LocConvex}^H$.}

\end{exercises}











\section{Other characterizations of amenability}
\label{OtherAmen}

Here are a few of the many conditions that are equivalent to amenability. The necessary definitions are provided in the discussions that follow.

\begin{thm} \label{AmenEquiv}
The following are equivalent:
\noprelistbreak
\begin{enumerate}
\item \label{AmenEquiv-amen}
 $H$ is amenable.
\item \label{AmenEquiv-FP}
 $H$ has a fixed point in every nonempty, compact, convex $H$-space.
\item \label{AmenEquiv-InvMeas}
 If $H$ acts continuously on a compact, metrizable topological space~$X$, then there is an $H$-invariant probability measure on~$X$.
\item \label{AmenEquiv-Mean}
 There is a left-invariant mean on the space $\Cbdd(H)$ of all real-valued, continuous, bounded functions on~$H$.
 \item \label{AmenEquiv-FinAdd}
 There is a left-invariant finitely additive probability measure~$\rho$ defined on the collection of all Lebesgue measurable subsets of~$H$, such that $\rho(E) = 0$ for every set~$E$ of Haar measure~$0$.
\item \label{AmenEquiv-Vectors}
 The left regular representation of~$H$ on $\LL2(H)$ has almost-invariant vectors.
\item \label{AmenEquiv-Folner}
 There exists a F\o lner sequence in~$H$.
\end{enumerate}
\end{thm}

The equivalence ($\ref{AmenEquiv-amen} \Leftrightarrow \ref{AmenEquiv-FP}$) is the definition of amenability \pref{AmenDefn}. Equivalence of the other characterizations will be proved in the remainder of this section.

%\begin{rem}
%  Also equivalent to amenability:
% If $H$ acts continuously\/ {\rm(}by isometries{\rm)} on a Banach space~$\Banach$, and $C$ is a nonempty, convex, $H$-invariant subset of the dual space~$\Banach^*$ that is compact in the weak$^*$ topology, then $H$ has a fixed point in~$C$.
%\end{rem}



\subsection{Invariant probability measures}

\begin{defns} 
Let $X$ be a complete metric space.
	\begin{enumerate}
	\item A measure~$\mu$ on~$X$ is a \defit[measure!probability]{probability measure} if $\mu(X) = 1$.
	\item \nindex{$\Prob(X)$ = $\{$probability measures on~$X$$\}$}%
	$\Prob(X)$ denotes the space of all probability measures on~$X$.
	\end{enumerate}
Any measure on~$X$ is also a measure on the one-point compactification~$X^+$ of~$X$, so, if $X$ is locally compact, then the {Riesz Representation Theorem} \pref{RieszRepThm} tells us that every finite measure on~$X$ can be thought of as a linear functional on the Banach space $C(X^+)$ of continuous functions on~$X^+$. This implies that $\Prob(X)$ is a subset of the closed unit ball in the dual space $C(X^+)^*$, and therefore has a weak$^*$ topology. If $X$ is compact (so there is no need to pass to~$X^+$), then the Banach-Alaoglu Theorem \pref{BanachAlaogluThm} tells us that $\Prob(X)$ is compact \csee{Prob(X)cpctEx}.
\end{defns}

\begin{eg} \label{Prob(X)CpctConvexEg}
If a group~$H$ acts continuously on a compact, metrizable space~$X$,
%The weak$^*$ topology provides $C(X)^*$ with the structure of a locally convex vector space, such that the set $\Prob(X)$ of probability measures on~$X$ is a compact, convex set.
then $\Prob(X)$ is a compact, convex $H$-space \csee{HonProb(X)cont}.
\end{eg}

\begin{rem}[(\thmindex{Urysohn's Metrization}{Urysohn's Metrization Theorem})] \label{Metrizable<>2ndCount}
Recall that a compact, Hausdorff space is metrizable if and only if it is \term{second countable}, so requiring a compact, separable, Hausdorff space to be metrizable is not a strong restriction.
\end{rem}

\begin{prop}[($\ref{AmenEquiv-amen} \Leftrightarrow \ref{AmenEquiv-InvMeas}$)]
$H$ is amenable if and only if for every continuous action of~$H$ on a compact, metrizable space~$X$, there is an $H$-invariant probability measure~$\mu$ on~$X$.
\end{prop}

\begin{proof}
($\Rightarrow$) If $H$ acts on~$X$, and $X$ is compact, then $\Prob(X)$ is a nonempty, compact, convex $H$-space \csee{Prob(X)CpctConvexEg}. So $H$ has a fixed point in $\Prob(X)$; this fixed point is the desired $H$-invariant measure.

\medbreak

($\Leftarrow$)
Suppose $C$ is a nonempty, compact, convex $H$-space.
By replacing $C$ with the closure of the convex hull of a single $H$-orbit, we may assume $C$ is separable; then $C$ is metrizable  \csee{CpctFrechet->Metrizable}. Since $H$ is amenable, this implies there is an $H$-invariant probability measure~$\mu$ on~$C$.
Since $C$ is convex and compact, the center of mass
	$$ p = \int_C c \, d\mu(c) $$
belongs to~$C$ \csee{CenterInC}. Since $\mu$ is $H$-invariant (and the $H$-action is by linear maps), a simple calculation shows that $p$ is $H$-invariant \csee{CenterInvt}.
\end{proof}



\subsection{Invariant means}

\begin{defn}
Suppose $\LocConvex$ is some linear subspace of $\LL\infty(H)$, and assume $\LocConvex$ contains the constant function~$1_H$ that takes the value~$1$ at every point of~$H$.
A \defit{mean} on~$\LocConvex$ is
a linear functional
$\lambda$ on $\LocConvex$, such that
\noprelistbreak
 \begin{itemize}
 \item $\lambda(1_H) = 1$,
 and
 \item $\lambda$ is positive, i.e., $\lambda(f) \ge 0$
whenever $f \ge 0$.
 \end{itemize}
 \end{defn}

\begin{rem} \label{MeanNorm1}
Any mean is a continuous linear functional; indeed, $\|\lambda\| = 1$ \csee{MeanNorm1Ex}.
\end{rem}
 
 It is easy to construct means:
 
 \begin{eg} \label{EasyMeans}
 If $\phi$ is any unit vector in $\LL1(H)$, and $\mu$ is the left Haar measure on~$H$, then defining
 	$$ \lambda(f) = \int_H f\, |\phi| \, d\mu$$
produces a mean (on any subspace of $\LL\infty(H)$ that contains~$1_H$). Means constructed in this way are (weakly) dense in the set of all means \csee{EasyMeansDense}.
 \end{eg}
 
Compact groups are the only ones with invariant probability measures, but invariant means exist more generally:

\begin{prop}[($\ref{AmenEquiv-amen} \Rightarrow \ref{AmenEquiv-Mean}$)] \label{Amen->Mean}
If $H$ is amenable, then there exists a left-invariant mean on the space $\Cbdd(H)$ of bounded, continuous functions on~$H$.
\end{prop}

\begin{proof}
%To avoid technical problems, let us assume $H$ is discrete.
The set of means on $\Cbdd(H)$ is obviously nonempty, convex and invariant under left translation \csee{SetOfMeans}. Furthermore, it is a weak$^*$ closed subset of the unit ball in $\Cbdd(H)^*$ \csee{MeansCpct}, so it is compact by the Banach-Alaoglu Theorem (\cref{BanachAlaogluThm}).
Therefore, the amenability of~$H$ implies that some mean is left-invariant.
(Actually, there is a slight technical problem here if $H$ is not discrete: the action of~$H$ on $\Cbdd(H)$ may not be continuous in the sup-norm topology, because continuous functions do not need to be uniformly continuous.)
 \end{proof}

\begin{rem} \label{Amen<>MeanLinfty}
With a bit more work, it can be shown that if $H$ is amenable, then there is a left-invariant mean on $\LL\infty(H)$, not just on $\Cbdd(H)$ \csee{MeanLinfty}.
Therefore, $\Cbdd(H)$ can be replaced with $\LL\infty(H)$ in \fullcref{AmenEquiv}{Mean}.
Furthermore, there exists a mean on $\LL\infty(H)$ that is \defit[mean!bi-invariant]{bi-invariant} (both left-invariant \emph{and} right-invariant) \ccf{DiscreteBiInv}.
\end{rem}

\begin{prop}[($\ref{AmenEquiv-Mean} \Rightarrow \ref{AmenEquiv-InvMeas}$)]
Suppose $H$ acts continuously on a compact, metrizable space~$X$. If there is a left-invariant mean on $\Cbdd(H)$, then there is an $H$-invariant probability measure on~$X$.
\end{prop}

\begin{proof}
 Fix some $x \in X$. Then we have a continuous, $H$-equivariant linear map from $C(X)$ to $\Cbdd(H)$, defined by
 	$$ \overline{f}(h) = f(hx) .$$
Therefore, any left-invariant mean on $\Cbdd(H)$ induces an $H$-invariant mean~$\lambda$ on $C(X)$ \csee{MapMeans}. Since $X$ is compact, the {Riesz Representation Theorem} \pref{RieszRepThm} tells us that any continuous, positive linear functional on $C(X)$ is a measure; thus, this $H$-invariant mean~$\lambda$ can be represented by an $H$-invariant measure~$\mu$ on~$X$. Since $\lambda$ is a mean, we have $\lambda(1) = 1$, so $\mu(X) = 1$, which means that $\mu$ is a probability measure.
 \end{proof}




\subsection{Invariant finitely additive probability measures}

The following proposition is based on the observation that, just as probability measures on~$X$ correspond to elements of the dual of $C(X)$, finitely additive probability measures correspond to elements of the dual of $\LL\infty(X)$.

\begin{prop}[($\ref{AmenEquiv-Mean} \Leftrightarrow \ref{AmenEquiv-FinAdd}$)] \label{Amen<>FinAdd}
There is a left-invariant mean on $\LL\infty(X)$ if and only if there is a left-invariant finitely additive probability measure~$\rho$ defined on the collection of all Lebesgue measurable subsets of~$H$, such that $\rho(E) = 0$ for every set~$E$ of Haar measure~$0$.
\end{prop}

\begin{proof}
($\Rightarrow$) Because $H$ is amenable, there exists a left-invariant mean~$\lambda$ on $\LL\infty(H)$ \csee{Amen<>MeanLinfty}. For a measurable subset~$E$ of~$H$, let
$\rho(E) = \lambda(\chi_E)$, where $\chi_E$ is the characteristic function of~$E$.
It is easy to verify that $\rho$ has the desired properties \csee{RhoIsFinAdd}.

($\Leftarrow$) We define a mean~$\lambda$ via an approximation by step functions: for $f \in \LL\infty(H)$, let
	$$ \lambda(f) = \inf \bigset{ \sum_{i=1}^n a_i \rho(E_i) }{ 
	f \le  \sum_{i=1}^n a_i \chi_{E_i} \text{ a.e.}
	} .$$
Since $\rho$ is finitely additive, it is straightforward to verify that $\lambda$ is a mean on $\LL\infty(H)$ \csee{LambdaIsMean}. Since $\rho$ is bi-invariant, we know that $\lambda$ is also bi-invariant.
\end{proof}

\begin{rem} \label{FinAddRem} \ 
\noprelistbreak
\begin{enumerate}
\item \label{FinAddRem-subgrp}
 \cref{Amen<>FinAdd} easily implies that every subgroup of a discrete amenable group is amenable \csee{SubgrpDiscAmenFinAdd}, establishing \cref{SubgrpAmen} for the case of discrete groups. In fact, it is not very difficult to prove the general case of \cref{SubgrpAmen} similarly \csee{SubgrpAmenFinAdd}.
\item Because any amenable group~$H$ has a bi-invariant mean on $\LL\infty(H)$ \csee{Amen<>MeanLinfty}, the proof of \cref{Amen<>FinAdd}($\Rightarrow$) shows that the finitely additive probability measure~$\rho$ can be taken to be bi-invariant.
\end{enumerate}
\end{rem}





\subsection{Almost-invariant vectors}

\begin{defn} \label{AlmInvtVecDefn}
 An action of~$H$ on a normed vector space~$\Banach$ has \emph{almost-invariant} vectors if, for every
compact subset~$C$ of~$H$ and every $\epsilon > 0$, there
is a unit vector
 $v \in \Banach$, such that 
 \begin{equation} \label{epsCInvt}
 \|c v - v \| < \epsilon \text{\quad for all $c \in C$}
.\end{equation}
(A unit vector satisfying \pref{epsCInvt} is said to be $(\epsilon,C)$-invariant.)
 \end{defn}

\begin{eg}
Consider the regular
representation of~$H$ on $\LL2(H)$.
\begin{enumerate}
\item  If $H$ is a compact Lie group, then the constant
function~$1_H$ belongs to $\LL2(H)$, so $\LL2(H)$
has an $H$-invariant unit vector.

\item If $H = \real$, then $\LL2(H)$ does not have any (nonzero)
$H$-invariant vectors \csee{NonCpctNoInvtVect}, but it does have almost-invariant vectors: Given $C$ and~$\epsilon$, choose $n \in
\natural$ so large that $C \subseteq [-n,n]$ and $2/\sqrt{n}
< \epsilon$. Let $\phi = \frac{1}{n} \chi_{n^2}$, where
$\chi_{n^2}$ is the characteristic function of $[0,n^2]$.
 Then $\phi$ is a unit vector and, for $c \in C$, we
have 
  $$ ||c \phi - \phi ||^2 \le \int_{-n}^n \frac{1}{n^2}
\, dx
 + \int_{n^2-n}^{n^2 + n} \frac{1}{n^2} \, dx
 = \frac{4}{n} 
 < \epsilon^2.$$
 \end{enumerate}
 \end{eg}
 
 \begin{rem} \label{L2invtIffL1inv}
 $\LL2(H)$ has almost-invariant vectors if and only if $\LL1(H)$ has almost-invariant vectors \csee{L2invtIffL1invEx}. Therefore,  $\LL2(H)$ may be replaced with $\LL1(H)$ in \fullcref{AmenEquiv}{Vectors}. (In fact, $\LL2(H)$ may be replaced with $\LL{p}(H)$, for any $p \in [1,\infty)$ \csee{L1invtIffLpInvEx}.)
 \end{rem}

 \begin{prop}[($\ref{AmenEquiv-Mean} \Leftrightarrow \ref{AmenEquiv-Vectors}$)]
There is a left-invariant mean on $\LL\infty(H)$ if and only if $\LL2(H)$ has almost-invariant vectors.
 \end{prop}
 
\begin{proof}
Because of \cref{L2invtIffL1inv}, we may replace $\LL2(H)$ with $\LL1(H)$.

($\Leftarrow$) By applying the construction of means in \cref{EasyMeans} to almost-invariant vectors in $\LL1(H)$, we obtain almost-invariant means on $\LL\infty(H)$. A limit of almost-invariant means is invariant \csee{LimAlmInvMean}.

($\Rightarrow$) Because the means constructed in \cref{EasyMeans} are dense in the space of all means, we can approximate a left-invariant mean by an $\LL1$~function. Vectors close to an invariant vector are almost-invariant, so $\LL1(H)$ has almost-invariant vectors. However, there are technical issues here; one problem is that the approximation is in the weak$^*$ topology, but we are looking for vectors that are almost-invariant in the norm topology. See \cref{Amen->VectorsEx} for a correct proof in the case of discrete groups (using the fact that a convex set has the same closure in both the norm topology and the weak$^*$ topology).
\end{proof}

 
 
 
 
 \subsection{F\o lner sequences}
 
 \begin{defn}
 Let  $\{F_n\}$ be a sequence of measurable sets in~$H$, such that $0 < \mu(F_n) < \infty$ for every~$n$. We say $\{F_n\}$ is a \defit[Folner@F\o lner!sequence]{F\o lner sequence} if, for every compact subset~$C$ of~$H$, we have 
 	\begin{equation} \label{FolnerLim}
	 \lim_{n \to \infty} \max_{c \in C} \frac{ \mu( F_n \symmdiff c F_n) }{ \mu(F_n) } = 0 ,
	 \end{equation}
where $\mu$ is the Haar measure on~$H$.
\end{defn}

\begin{eg} \label{FolnerEg} \ 
	\begin{enumerate}
	\item  \label{FolnerEg-Rl}
	If $F_n = B_n(0)$ is the ball of radius~$n$ in~$\real^\ell$, then $\{F_n\}$ is a F\o lner sequence in~$\real^\ell$ \csee{FolnerRlEx}.
	\item The free group $\free_2$ on 2 generators does not have F\o lner sequences \csee{FreeNoFolnerEx}.
	\end{enumerate}
\end{eg}

The reason that $\real^\ell$ has a F\o lner sequence, but the free group~$\free_2$ does not, is that $\real^\ell$ is amenable, but $\free_2$ is not:

\begin{prop}[($\ref{AmenEquiv-Vectors} \Leftrightarrow \ref{AmenEquiv-Folner}$)] \label{Amen<>Folner}
There is an invariant mean on $\LL2(H)$ if and only if $H$ has a F\o lner sequence.
\end{prop}

\begin{proof}
($\Leftarrow$) Normalized characteristic functions of F\o lner sets are almost invariant vectors in $\LL1(H)$ \csee{Amen<-Folner}.

($\Rightarrow$) Let us assume $H$ is discrete.
Given $\epsilon > 0$, and a finite subset~$C$ of~$H$, we wish to find a finite subset~$F$ of~$H$, such that
	$$ \frac{ \# \bigl( F \symmdiff c(F) \bigr) }{ \#(F_n) } < \epsilon 
	\quad \text{for all $c \in C$} .$$
Since $H$ is amenable, we know $\LL1(H)$ has almost-invariant vectors \csee{L2invtIffL1inv}; hence, there exists $f \in \LL1(H)$, such that 
\noprelistbreak
	\begin{enumerate}
	\item $f \ge 0$,
	\item $\| f \|_1 = 1$,
	and
	\item $\|c f - f \|_1 < \epsilon/\#C$, for every $c \in C$.
	\end{enumerate}
Note that if $f$ were the normalized characteristic function of a set~$F$, then this set~$F$ would be what we want; for the general case, we will approximate $f$ by a sum of such characteristic functions.

Approximating $f$ by a step function, we may assume $f$ takes only finitely many values. Hence, there exist:
\noprelistbreak
	\begin{itemize}
	\item finite subsets $A_1 \subseteq A_2 \subseteq \cdots \subseteq A_n$ of~$H$,
	and
	\item real numbers $\alpha_1, \ldots,\alpha_n > 0$,
	\end{itemize}
such that
\noprelistbreak
	\begin{enumerate}
	\item $\alpha_1 + \alpha_2 + \cdots  + \alpha_n = 1$
	and
	\item $f = \alpha_1 f_1 + \alpha_2 f_2 + \cdots \alpha_n f_n$,
	\end{enumerate}
where $f_i$ is the normalized characteristic function of~$A_i$ \csee{StepFuncIsSumEx}. For all $i$ and~$j$, and any $c \in H$, we have
	\begin{equation} \label{cAiDisjAj}
	\text{$A_i \smallsetminus cA_i$ is disjoint from $cA_j \smallsetminus A_j$} 
	\end{equation}
\csee{cAiDisjAjEx}, so, for any $x \in H$, we have
	$$f_i(x) > (c f_i)(x) \implies f_j(x) \ge (cf_j)(x)$$
and
	$$f_i(x) < (cf_i)(x) \implies f_j(x) \le (cf_j)(x) .$$
Therefore
	$$ \text{$\displaystyle | (cf - f)(x)| = \sum_i  \alpha_i | (cf_i - f_i)(x)|$ \quad for all $x \in H$} .$$
Summing over~$H$ yields
	$$ \sum_i \alpha_i \|cf_i - f_i \|_1
	= \|cf - f \|_1
	< \frac{\epsilon}{\#C} .$$
Summing over~$C$, we conclude that
	$$ \sum_i \alpha_i \sum_{c \in C} \|cf_i - f_i \|_1 < \epsilon .$$
Since $\sum_i \alpha_i = 1$ (and all terms are positive), this implies there is some~$i$, such that
	$$ \sum_{c \in C} \|cf_i - f_i \|_1 < \epsilon .$$
Hence, $\|cf_i - f_i \|_1 < \epsilon$, for every~$c \in C$, so we may let $F = A_i$.
\end{proof}
 
 
\begin{rem} \label{FolnerSubgrp}
F\o lner sets provide an easy proof that subgroups of discrete amenable groups are amenable.
\end{rem}

\begin{proof}
Let
\noprelistbreak
	\begin{itemize}
	\item $A$ be a closed subgroup of a discrete, amenable group~$H$,
	\item $C$ be a finite subset of~$A$,
	and
	\item $\epsilon > 0$.
	\end{itemize}
Since $H$ is amenable, there is a corresponding F\o lner set~$F$ in~$H$. 

It suffices to show there is some $h \in H$, such that $Fh \cap A$ is a F\o lner set in~$A$.
We have
	\begin{align*}
	\#F
	= \sum_{Ah \in A \backslash H} \# (F \cap Ah)
	\end{align*}
and, letting $\epsilon' = \epsilon \#C$, we have
	\begin{align*}
	(1 + \epsilon') \#F
	\ge \#(C F) 
	= \sum_{Ah \in A \backslash H} \# \bigl( C (F \cap Ah) \bigr)
	, \end{align*}
so there must be some $Ah  \in A \backslash H$, such that 
	$$ \# \bigl( C (F \cap Ah) \bigr) \le (1 + \epsilon') \# (F \cap Ah)  $$
(and $F \cap Ah \neq \emptyset$). Then, letting $F' = Fh^{-1} \cap A$, we have
	$$ \# (CF') = \# \bigl( C (F \cap Ah) \bigr) \le (1 + \epsilon') \# (F \cap Ah) = (1 + \epsilon') \#F' ,$$
so $F'$ is a F\o lner set in~$A$.
\end{proof}
 
 
 



\begin{exercises}

\item[]
\exersubsection{Invariant probability measures}

\item \label{Prob(X)cpctEx}
In the setting of \cref{Prob(X)CpctConvexEg}, show that $\Prob(X)$ is a compact, convex subset of $C(X)^*$.
\hint{You may assume the Banach-Alaoglu Theorem (\cref{BanachAlaogluThm}).}

\item \label{HonProb(X)cont}
Suppose $H$ acts continuously on a compact, metrizable space~$X$. There is an induced action of~$H$ on $\Prob(X)$ defined by
	$$ \text{$(h_*\mu)(A) = \mu (h^{-1} A)$ \quad for $h \in H$, $\mu \in \Prob(X)$, and $A \subseteq X$} .$$
Show that this induced action of~$H$ on $\Prob(X)$ is continuous (with respect to the weak$^*$ topology on $\Prob(X)$).

\item \label{CpctFrechet->Metrizable}
Let $A$ be a separable subset of a Fréchet space~$\LocConvex$. Show
	\begin{enumerate}
	\item $A$ is second countable.
	\item \label{CpctFrechet->Metrizable-metrizable}
 If $A$ is compact, then $A$ is metrizable.
	\end{enumerate}
\hint{\pref{CpctFrechet->Metrizable-metrizable}~\cref{Metrizable<>2ndCount}.}

\item \label{CenterInC}
Let $\mu$ be a probability measure on a compact, convex subset~$C$ of a Fréchet space~$\LocConvex$. The \defit[center!of mass]{center of mass} of~$C$ is a point $c \in \LocConvex$, such that, for every continuous linear functional~$\lambda$ on~$\LocConvex$, we have
	$$ \lambda (c) = \int_C \lambda(x) \, d\mu(x) .$$
Show the center of mass of~$\mu$ exists and is unique, and is an element of~$C$.

\item
Give an example of a probability measure~$\mu$ on a Fréchet space, such that the center of mass of~$\mu$ does not exist.
\hint{There are probability measures on~$\real^+$, such that the center of mass is infinite.}

\item \label{CenterInvt}
Show that if $p$ is the center of mass of a probability measure~$\mu$ on a Fréchet space~$\LocConvex$, then $p$~is invariant under every continuous, linear transformation of~$\LocConvex$ that preserves~$\mu$.

\item Suppose $H$ acts continuously on a compact, metrizable space~$X$. Show that the map
	$$ H \times \Prob(X) \colon (h,\mu) \mapsto h_* \mu $$
defines a continuous action of~$H$ on $\Prob(X)$.


\exersubsection{Left-invariant means}

\item \label{MeanNorm1Ex}
Verify \cref{MeanNorm1}.
\hint{$\lambda(1_H) = 1$ implies $\|\lambda\| \ge 1$. For the other direction, note that if $\|f\|_\infty \le 1$, then $1_H-f \ge 0$ a.e., so $\lambda(1_H - f) \ge 0$; similarly, $\lambda(f + 1_H) \ge 0$.}

\item \label{MeanLtoC}
Show that the restriction of a mean is a mean. More precisely, let $\LocConvex_1$ and $\LocConvex_2$ be linear subspaces of $\LL\infty(H)$, with $1_H \in \LocConvex_1 \subseteq \LocConvex_2$. Show that if $\lambda$ is a mean on~$\LocConvex_2$, then the restriction of~$\lambda$ to~$\LocConvex_1$ is a mean on~$\LocConvex_1$.

\item Suppose $\lambda$ is a mean on $\Cbdd(H)$, the space of bounded, continuous functions on~$H$. For $f \in \Cbdd(H)$, show 
	$$ \min f \le \lambda(f) \le \max  f .$$

\item For $h \in H$, define $\delta_h \colon \Cbdd(H) \to \real$ by $\delta_h(f) = f(h)$. Show $\delta_h$ is a mean on $\Cbdd(H)$.

\item \label{EasyMeansDense}
Let $\Banach$ be any linear subspace of $\LL\infty(H)$, such that $\Banach$ contains $1_H$ and is closed in the $\LL\infty$-norm. Show that the means constructed in \cref{EasyMeans} are weak$^*$ dense in the set of all means on~$\Banach$.
\hint{If not, then the Hahn-Banach Theorem implies there exist $\epsilon > 0$, a mean~$\lambda$, and some $f \in (\Banach^*)^* = \Banach$, such that 
	$$  \lambda(f) > \epsilon + \int_H f \, |\phi| \, d\mu ,$$
for every unit vector~$\phi$ in $\LL1(H)$. This contradicts the fact that $\lambda(f) \le \mathop{\text{ess.\ sup}} f$.}

\item \label{SetOfMeans}
Let $\mathcal{M}$ be the set of means on $\Cbdd(H)$. Show:
	\begin{enumerate}
	\item \label{SetOfMeans-notempty}
	$\mathcal{M} \neq \emptyset$.
	\item $\mathcal{M}$ is convex.
	\item $\mathcal{M}$ is $H$-invariant.
	\end{enumerate}
\hint{\pref{SetOfMeans-notempty}~Evaluation at any point is a mean.}

\item \label{MeansCpct}
Let $\mathcal{M}$ be the set of means on $\Cbdd(H)$. Show:
	\begin{enumerate}
	\item $\mathcal{M}$ is contained in the closed unit ball of $\Cbdd(H)^*$. (That is, we have $|\lambda(f) \le \|f\|_\infty$ for every $f \in \Cbdd(H)$.)
	\item $\mathcal(M)$ is weak$^*$ closed.
	\item $\mathcal{M}$ is compact in the weak$^*$ topology.
	\end{enumerate}
\hint{You may assume the Banach-Alaoglu Theorem (\cref{BanachAlaogluThm}).}

\label{MeanLinfty}
Show that if $H$ is amenable, then there is a left-invariant mean on $\LL\infty(H)$.
\hint{Define $\lambda(f) = \mu_0(f * \eta)$, where $\lambda_0$ is a left-invariant mean on $\Cbdd(H)$, and $\eta$ is a nonnegative function of integral~$1$.}

\item \label{MapMeans}
Suppose $\psi \colon Y \to X$ is continuous, and $\lambda$ is a mean on $\Cbdd(Y)$. Show that $\psi_*\lambda$ (defined by $(\psi_*\lambda)(f) = \lambda ( f \circ \psi )$) is a mean on $\Cbdd(X)$.

%\item \label{MeanOnCpctIsMeas}
%Suppose $\lambda$ is a mean on $\Cbdd(X)$, and $X$ is compact. Show that $\lambda$ is a probability measure.
%\hint{You may assume the Riesz Representation Theorem, which states that any continuous, positive linear functional on $C_c(X)$ is a measure. (Here, $C_c(X)$ is the space of continuous functions with compact support, under the supremum norm.}

\item \label{DiscreteBiInv}
Assume $H$ is amenable and discrete. Show there is a bi-invariant mean on $\LL\infty(H)$.
\hint{Since $\LL\infty(H) = \Cbdd(H)$, amenability implies there is a left-invariant mean on $\LL\infty(H)$ \fullcsee{AmenEquiv}{Mean}. Now $H$ acts by right translations on the set of all such means, so amenability implies that some left-invariant mean is right-invariant.}

\item \label{MetrizableSuffices}
\harder
Assume $H$ has a fixed point in every \emph{metrizable}, nonempty, compact, convex $H$-space (and $H$ is second countable). Show $H$ is amenable.
\hint{To find a fixed point in~$C$, choose some $c_0 \in C$. For each mean~$\lambda$ on $\Cbdd(H)$ and each $\rho \in \LocConvex^*$, define $\phi_\lambda(\rho) = \lambda \bigl(h \mapsto \rho(hc_0) \bigr)$, so $\phi_\lambda \in (\LocConvex^*)'$, the algebraic dual of~$\LocConvex^*$. 
If $\lambda$ is a convex combination of evaluations at points of~$H$, 
 it is obvious there exists $c_\lambda \in C$, such that $\phi_\lambda(\rho) = \rho(c_\lambda)$. Since the map $\lambda \mapsto \phi_\lambda$ is continuous (with respect to appropriate weak topologies), this implies $c_\lambda$ exists for every~$\lambda$. The proof of \cref{Amen->Mean} shows that $\lambda$ may be chosen to be left-invariant, and then $c_\lambda$ is $H$-invariant.}
%\hint{The proof of \cref{Amen->Mean} shows there is a left-invariant mean~$\lambda$ on $\Cbdd(H)$.
%Now, to find a fixed point in~$C$, fix some $c_0 \in C$. For each $\rho \in \LocConvex^*$, define %$f_\rho \in \Cbdd(H)$ by $f_\rho(h) = \rho(hc_0)$, and then define $\phi(\rho) = \lambda(f_\rho)$, 
%$\phi(\rho) = \lambda \bigl(h \mapsto \rho(hc_0) \bigr)$, 
%so $\phi \in \LocConvex^{*{}*}$. Since $\phi$ is continuous for the weak$^*$ topology, we have $\phi \in \LocConvex$. In fact, $\phi$ is an $H$-invariant element of~$C$.}


\exersubsection{Invariant finitely additive probability measures}

\item \label{RhoIsFinAdd}
Verify that~$\rho$, as defined in the proof of \cref{Amen<>FinAdd}($\Rightarrow$), has the properties specified in the statement of the \lcnamecref{Amen<>FinAdd}.

\item \label{LambdaIsMean}
Let $\rho$ and $\lambda$ be as in the proof of \cref{Amen<>FinAdd}($\Leftarrow$). 
	\begin{enumerate}
	\item \label{LambdaIsMean-well}
 If $\displaystyle \sum_{i=1}^m a_i \, \chi_{E_i} = \sum_{j=1}^n b_j \, \chi_{F_j}$ a.e., show $\displaystyle \sum_{i=1}^m a_i \, \rho(E_i) = \sum_{j=1}^n \, b_j \rho(F_j)$.
	\item \label{LambdaIsMean-le}
 If $\displaystyle \sum_{i=1}^m a_i \, \chi_{E_i} \le \sum_{j=1}^n b_j \, \chi_{F_j}$ a.e., show $\displaystyle \sum_{i=1}^m a_i \, \rho(E_i) \le \sum_{j=1}^n b_j \, \rho(F_j)$.
 	\item Show that $\lambda(1_H) = 1$.
	 \item Show that if $f \ge 0$, then $\lambda(f) \ge 0$.
	 \item Show that 
	 $$ \lambda(f) = \sup \bigset{ \sum_{i=1}^n a_i \rho(E_i) }{ 
		f \ge  \sum_{i=1}^n a_i \chi_{E_i} \text{ a.e.}
		} .$$
	\item Show that $\lambda$ is a mean on $\LL\infty(H)$.
	\end{enumerate}
\hint{(\ref{LambdaIsMean-well},\ref{LambdaIsMean-le})~By passing to a refinement, arrange that $\{E_i\}$ are pairwise disjoint, $\{F_j\}$ are pairwise disjoint, and each $E_i$ is contained in some~$F_j$.
}

\item \label{SubgrpDiscAmenFinAdd}
Use \cref{Amen<>FinAdd} to prove that every subgroup~$A$ of a discrete amenable group~$H$ is amenable.
\hint{Let $X$ be a set of representatives of the right cosets of~$A$ in~$H$,
and let $\lambda$ be a left-invariant finitely additive probability measure on~$H$. For $E \subseteq A$, define
	$ \overline\lambda(E) = \lambda( EX )$.}

\item \label{SubgrpAmenFinAdd}
Use \cref{Amen<>FinAdd} to prove that every closed subgroup~$A$ of an amenable group~$H$ is amenable.
\hint{Let $X$ be a Borel set of representatives of the right cosets of~$A$ in~$H$, and define $\overline\lambda$ as in the solution of \cref{SubgrpDiscAmenFinAdd}. Fubini's Theorem implies that if $E$ has measure~$0$ in~$A$, then $XA$ has measure~$0$ in~$H$. You may assume (without proof) the fact that if $f \colon M \to N$ is a continuous function between manifolds $M$ and~$N$, and $E$ is a Borel subset of~$M$, such that the restriction of~$f$ to~$E$ is one-to-one, then $f(E)$ is a Borel set in~$N$.
}

\exersubsection{Almost-invariant vectors}

\item \label{NonCpctNoInvtVect} 
\begin{enumerate}
\item For $v \in \LL2(H)$, show that $v$ is invariant under translations if and only if $v$~is constant (a.e.).
\item Show that $H$ is compact if and only if  $\LL2(H)$ has a nonzero vector that is invariant under translation.
\end{enumerate}

\item \label{L2invtIffL1invEx}
Show that $\LL2(H)$ has almost-invariant vectors if and only if $\LL1(H)$ has almost-invariant vectors.
\hint{Note that $f^2 - g^2 = (f - g)(f + g)$, so $\|f^2 - g^2 \|_1 \le \| f - g\|_2 \, \| f + g \|_2$. 
Conversely, for $f,g \ge 0$, we have $(f-g)^2 \le |f^2-g^2|$, so $\| f-g \|_2^2 \le \|f^2 - g^2\|_1$.}

\item \label{L1invtIffLpInvEx}
For $p \in [1,\infty)$, show that $\LL1(H)$ has almost-invariant vectors if and only if $\LL{p}(H)$ has almost-invariant vectors. 
\hint{If $p < q$, then almost-invariant vectors in $\LL{p}(H)$ yield almost-invariant vectors in $\LL{q}(H)$, because $|(f - g)|^{q/p} \le |f^{q/p} - g^{q/p}|$. And almost-invariant vectors in $\LL{p}(H)$ yield almost-invariant vectors in $\LL{p/2}(H)$, by the argument of the first hint in \cref{L2invtIffL1invEx}.}

\item \label{LimAlmInvMean}
Let
\noprelistbreak
	\begin{itemize}
	\item $\{C_n\}$ be an increasing sequence of compact subsets of~$H$, such that $\bigcup_n C_n = H$,
	\item $\epsilon_n = 1/n$,
	\item $\phi_n$ be an $(\epsilon_n,C_n)$-invariant unit vector in $\LL1(H)$, 
	\item $\lambda_n$ be the mean on $\LL\infty$ obtained from $\phi_n$ by the construction in \cref{EasyMeans},
	and
	\item $\lambda$ be an accumulation point of $\{\lambda_n\}$.
	\end{itemize}
Show that $\lambda$ is invariant.

\item \label{Amen->VectorsEx}
Assume $H$ is discrete. Let 
	$$\mathcal{P} = \{\, \phi \in \LL1(H) \mid \phi \ge 0, \|\phi\|_1 = 1 \,\} .$$
Suppose $\{\phi_i\}$ is a net in $\mathcal{P}$, such that the corresponding means $\lambda_i$ converge weak$^*$ to an invariant mean~$\lambda$ on $\LL\infty(H)$.
	\begin{enumerate}
	\item For each $h \in H$, show that the net $\{h^*\phi_i - \phi_i\}$ converges weakly to~$0$.
	\item Take a copy $\LL1(H)_h$ of $\LL1(H)$ for each $h \in H$, and let 		$$\LocConvex = \bigtimes_{h \in H} \LL1(H)_h $$
	with the product of the norm topologies. Show that $\LocConvex$ is a Fréchet space.
	\item Show that the weak topology on~$\LocConvex$ is the product of the weak topologies on the factors.
	\item Define a linear map $T \colon \LL1(H) \to \LocConvex$ by $T(f)_h = h^* f - f$.
	\item Show that the net $\{T(\phi_i\})$ converges to $0$ weakly.
	\item Show that $0$ is in the strong closure of $T(\mathcal{P})$.
	\item Show that $\LL1(H)$ has almost-invariant vectors.
	\end{enumerate}

\item Show that if $H$ is amenable, then $H$ has the \defit{Haagerup property}. By definition, this means there is a unitary representation of~$H$ on a Hilbert space~$\Hilbert$, such that there are almost-invariant vectors, and all matrix coefficients decay to~$0$ at~$\infty$ as in the conclusion of \cref{DecayMatCoeffSimple}. (A group with the Haagerup property is also said to be \defit[a-T-menable|indsee{Haagerup property}]{a-T-menable}.)


\exersubsection{F\o lner sequences}

\item Show that $\{F_n\}$ is a F\o lner sequence if and only if, for every compact subset~$C$ of~$H$, we have
	$$ \lim_{n \to \infty} \frac{ \mu( F_n \cup c F_n) }{ \mu(F_n) } = 1 .$$

\item \label{FolnerRlEx}
Justify \fullcref{FolnerEg}{Rl}.
\hint{$C \subseteq B_r(0)$, for some~$r$. We have $\mu\bigl(B_{r+\ell}(0) \bigr)/\mu\bigl(B_{\ell}(0) \bigr) \to 1$.}

\item \label{Amen<-Folner}
Prove \Cref{Amen<>Folner}($\Leftarrow$).
\hint{Normalizing the characteristic function of $F_n$ yields an almost-invariant unit vector.}

\item \label{Folner<>ratio}
Show \pref{FolnerLim} is equivalent to
	 $$ \lim_{n \to \infty} \max_{c \in C} \frac{ \mu( F_n \cup c F_n) }{ \mu(F_n) } = 1 .$$

\item \label{Folner<>CF}
Assume $H$ is discrete. Show that a sequence $\{F_n\}$ of finite subsets of~$H$ is a F\o lner sequence if and only if, for every finite subset~$C$ of~$H$, we have
	 $$ \lim_{n \to \infty} \frac{ \#( C F_n ) }{ \#(F_n) } = 1 .$$

\item \label{StepFuncIsSumEx}
Given a step function~$f$, as in the proof of \cref{Amen<>Folner}($\Rightarrow$), let 
	\begin{itemize}
	\item $a_1 > a_2 > \cdots > a_n$ be the finitely many positive values taken by~$f$,
	\item $A_i = \{\, h \in H \mid f(h) \ge a_i \,\}$,
	and
	\item $f_i$ be the normalized characteristic function of~$A_i$.
	\end{itemize}
Show 
	\begin{enumerate}
	\item $A_1 \subseteq A_2 \subseteq \cdots \subseteq A_n$,
	\item there exist real numbers $\alpha_1,\ldots,\alpha_n > 0$, such that 
	$$ f = \alpha_1 f_1 + \cdots + \alpha_n f_n ,$$
	and
	\item $\alpha_1 + \cdots + \alpha_n = 1$.
	\end{enumerate}

\item \label{cAiDisjAjEx}
Prove \pref{cAiDisjAjEx}.
\hint{Note that either $A_i \subseteq A_j$ or $A_j \subseteq A_i$.}

\item 
\harder
Use F\o lner sets to prove \cref{FolnerSubgrp} (without assuming $H$ is discrete).
\hint{Adapt the proof of the discrete case. There are technical difficulties, but begin by replacing the sum over $A \backslash H$ with an integral over $A \backslash H$.}

\item A finitely generated (discrete) group~$\Lambda$ is said to have \defit{subexponential growth} if there exists a generating set~$S$ for $\Lambda$, such that, for every $\epsilon > 0$, 
	$$ \text{$\# (S \cup S^{-1})^n \le e^{\epsilon n}$ for all large~$n$.} $$
Show that every group of subexponential growth is amenable.

\item Give an example of an finitely generated, amenable group that does \emph{not} have subexponential growth.

\end{exercises}





\section{Some nonamenable groups} \label{NonamenSect}

Other proofs of the following proposition appear in \cref{FreeNotAmenMeas,FreeNoFolnerEx}.

\begin{prop} \label{FreeNotAmen}
Nonabelian free groups are not amenable.
\end{prop}

\begin{proof}
For convenience, we consider only the free group~$\free_2$ on two generators $a$ and~$b$. Suppose $\free_2$ has a left-invariant finitely additive probability measure~$\rho$. (This will lead to a contradiction.)

We may write $\free_2 = A \cup A^- \cup B \cup B^- \cup \{e\}$, where $A$, $A^-$, $B$, and~$B^-$ consist of the reduced words whose first letter is~$a$, $a^{-1}$, $b$, or~$b^{-1}$, respectively. Assume, without loss of generality, that $\rho(A \cup A^-) \le \rho(B \cup B^-)$ and $\rho(A) \le \rho(A^-)$. Then
	$$ \text{$\displaystyle \rho \bigl( B \cup B^- \cup \{e\} \bigr) \ge \frac{1}{2}$
	\quad
	and
	\quad
	$\displaystyle \rho(A) \le \frac{1}{4}$}
	.$$
Then, by left-invariance, we have
	$$ \rho \Bigl( a \bigl( B \cup B^- \cup \{e\} \bigr) \Bigr)
	= \rho \bigl( B \cup B^- \cup \{e\} \bigr)
	\ge \frac{1}{2}
	>  \rho(A) .$$
This contradicts the fact that $a \bigl( B \cup B^- \cup \{e\} \bigr) \subseteq A$.
\end{proof}

Combining this with the fact that subgroups of discrete amenable groups are amenable \csee{SubgrpAmen}, we have the following consequence:

\begin{cor} \label{FreeSubgrp->Nonamen}
Suppose $H$ is a discrete group. 
If $H$ contains a nonabelian, free subgroup, then $H$ is not amenable.
\end{cor}

\begin{rems} \label{FreeSubgrpRem} \ 
\noprelistbreak
\begin{enumerate}
\item  \label{FreeSubgrpRem-Olshanskii}
The converse of \cref{FreeSubgrp->Nonamen} is known as ``\thmindex{von\,Neumann's Conjecture}von\,Neumann's Conjecture\zz,'' but it is false: a nonamenable group with no nonabelian free subgroups was found by  Ol'shanskii in 1980. 
(The name is misleading: apparently, the conjecture is due to M.\,Day, and was never stated by Von Neumann.)
\item The assumption that $H$ is discrete cannot be deleted from the statement of \cref{FreeSubgrp->Nonamen}.
For example, the orthogonal group $\SO(3)$ is amenable (because it is compact), but the Tits Alternative \pref{TitsAlternative} implies that it contains nonabelian free subgroups. 
\item \label{FreeSubgrpRem-BanachTarski}
 The nonamenability of nonabelian free subgroups of $\SO(3)$ is the basis of the famous \thmindex{Banach-Tarski Paradox}Banach-Tarski Paradox: A $3$-dimensional ball~$B$ can be decomposed into finitely many subsets $X_1,\ldots,X_n$, such that these subsets can be reassembled to form the union of two disjoint balls of the same radius as~$B$. (More precisely, the union $B_1 \cup B_2$ of two disjoint balls of the same radius as~$B$ can be decomposed into subsets $Y_1,\ldots,Y_n$, such that $Y_i$ is congruent to~$X_i$, for each~$i$.)
\item If $H$ contains a \emph{closed}, nonabelian, free subgroup, then $H$ is not amenable.
\end{enumerate}
\end{rems}

Here is an example of a nonamenable connected group:

\begin{prop} \label{SL2RNotAmen}
$\SL(2,\real)$ is not amenable.
\end{prop}

\begin{proof}
Let $G = \SL(2,\real)$. The action of $G$ on $\real \cup \{\infty\} \iso \circle$ by linear-fractional transformations is transitive, and the stabilizer of the point~$0$ is the subgroup
	$ P = \begin{Smallbmatrix} *&* \\ &* \end{Smallbmatrix} $,
so $G/P$ is compact. However, the Borel Density Theorem implies there is no $G$-invariant probability measure on $G/P$ \csee{BDT-G/H}. (See \cref{NoProbMeasOnRP1} for a direct proof that there is no $G$-invariant probability measure.) So $G$ is not amenable.
\end{proof}

More generally:

\begin{prop} \label{GNotAmen}
If a connected, semisimple Lie group $G$ is not compact, then $G$ is not amenable.
\end{prop}

\begin{proof}
The {Jacobson-Morosov Lemma} \pref{JacobsonMorosov} tells us that $G$ contains a closed subgroup isogenous to $\SL(2,\real)$. Alternatively, recall that any lattice $\Gamma$ in~$G$ must contain a nonabelian free subgroup \csee{FreeInGamma}, and, being discrete, this is a closed subgroup of~$G$.
\end{proof}

\begin{rem}
Readers familiar with the structure of semisimple Lie groups will see that the proof of \cref{SL2RNotAmen} generalizes to the situation of \cref{GNotAmen}: Since $G$ is not compact, it has a proper parabolic subgroup~$P$. Then $G/P$ is compact, but the Borel Density Theorem implies that $G/P$ has no $G$-invariant probability measure.
\end{rem}

Combining this result with the structure theory of connected Lie groups yields the following classification of connected, amenable Lie groups:

\begin{prop} \label{ConnAmen}
A connected Lie group~$H$ is amenable if and only if $H$ contains a connected, closed, solvable normal subgroup~$N$, such that $H/N$ is compact.
\end{prop}

\begin{proof}
($\Leftarrow$) \fullcref{AmenExtCor}{CocpctSolv}.

($\Rightarrow$) The structure theory of Lie groups tells us that there is a connected, closed, solvable, normal subgroup~$R$ of~$H$, such that $H/R$ is semisimple. (The subgroup~$R$ is called the \defit[radical!of a Lie group]{radical} of~$H$.) Since quotients of amenable groups are amenable \csee{QuotAmen}, we know that $H/R$ is amenable. So $H/R$ is compact \csee{GNotAmen}.
\end{proof}

\begin{exercises}

\item \label{FreeNotAmenMeas}
\noprelistbreak
	\begin{enumerate}
	\item \label{FreeNotAmenMeas-ptmass}
Find a homeomorphism $\phi$ of the circle~$\circle$, such that the only $\phi$-invariant probability measure is the delta mass at a single point~$p$.
	\item \label{FreeNotAmenMeas-diff}
 Find two homeomorphisms $\phi_1$ and~$\phi_2$ of~$\circle$, such that the subgroup $\langle \phi_1,\phi_2 \rangle$ they generate has no invariant probability measure.
	\item Deduce that the free group~$\free_2$ on 2 generators is not amenable.
	\end{enumerate}
\hint{\pref{FreeNotAmenMeas-ptmass}~Identifying $S^1$ with $[0,1]$, let $\phi(x) = x^2$. For any $x \in (0,1)$, we have $\phi \bigl( (0,x) \bigr) = (0,x^2)$, so $\mu \bigl( (x^2,x) \bigr) = 0$. Since $(0,1)$ is the union of countably many such intervals, this implies that $\mu \bigl( (0,1) \bigr) = 0$.}

\item \label{FreeNoFolnerEx}
Show explicitly that free groups do not have F\o lner sequences. More precisely, let $\free_2$ be the free group on two generators~$a$ and~$b$, and show that if $F$ is any nonempty, finite subset of~$\free_2$, then there exists $c \in \{a,b,a^{-1},b^{-1}\}$, such that $\#(F \smallsetminus cF) \ge (1/4) \#F$.
This shows that $\free_2$ free groups is not amenable.
\hint{Suppose $F = A \cup B \cup A^- \cup B^-$, where words in $A,B,A^-,B^-$ start with $a,b,a^{-1},b^{-1}$, respectively. If $\#A \le \# A^-$ and $\# (A \cup A^-) \le \#(B \cup B^-)$, then $\#(aF \smallsetminus F) \ge \#(B \cup B^-) - \#A$.}

\item Assume that $H$ is discrete, and that $H$ is isomorphic to a (not necessarily discrete) subgroup of $\SL(\ell,\real)$. Show: 
\noprelistbreak
	\begin{enumerate}
	\item $H$ is amenable if and only if $H$ has no nonabelian, free subgroups.
	\item $H$ is amenable if and only if $H$ has a solvable subgroup of finite index.
	\end{enumerate}
\hint{Tits Alternative \pref{TitsAlternative}.}

\item \label{NoProbMeasOnRP1}
Let $G = \SL(2,\real)$ act on $\real \cup \{\infty\}$ by linear-fractional transformations, as usual.
\noprelistbreak
	\begin{enumerate}
	\item \label{NoProbMeasOnRP1-u}
 For 
		$ u = \left[\begin{smallmatrix} 1 & 1 \\ 0 & 1 \end{smallmatrix} \right] \in G $,
	show that the only $u$-invariant probability measure on $\real \cup \{\infty\}$ is concentrated on the fixed point of~$u$.
	\item Since the fixed point of~$u$ is not fixed by all of~$G$, conclude that there is no $G$-invariant probability measure on $\real \cup \{\infty\}$.
	\end{enumerate}
\hint{\pref{NoProbMeasOnRP1-u}~The action of~$u$ is conjugate to the homeomorphism~$\phi$ in the hint to \fullcref{FreeNotAmenMeas}{ptmass}, % !!!
so a similar argument applies.}

\item \label{GammaNotAmen}
Show that if a semisimple Lie group~$G$ is not compact, then every lattice~$\Gamma$ in~$G$ is not amenable.

%\item Show that if $G$ is amenable, then $G$ is compact.
%\hint{If $G$ is not compact, then $G$ has a proper parabolic subgroup~$P$. Since $G/P$ is compact, amenability implies there is a $G$-invariant probability measure on $G/P$. The Borel Density Theorem \pref{BDT-G/H} provides a contradiction.}

\item Give an example of a nonamenable Lie group that has a closed, cocompact, amenable subgroup. (By \cref{ConnAmen}, the subgroup cannot be normal.)


\end{exercises}










\section{Closed subgroups of amenable groups}
\label{AmenSubgrpSect}

Before proving that closed subgroups  of amenable groups  are amenable (\cref{SubgrpAmen}), we introduce some notation and establish a \lcnamecref{Linfty(H;C)good}. (Proofs for the case of discrete groups have already been given in \cref{FinAddRem,FolnerSubgrp}.)

\begin{notation} \ 
\noprelistbreak
	\begin{enumerate}
	\item We use \nindex{$\LL\infty(H;C) = \{\text{bounded functions from~$H$ to~$C$} \}$}$\LL\infty(H;C)$ to denote the space of all measurable functions from the Lie group~$H$ to the compact, convex set~$C$, where two functions are identified if they are equal a.e.\ (with respect to the Haar measure on~$H$). 
	\item If $\Lambda$ is a closed subgroup of~$H$, and $C$ is a $\Lambda$-space, then%
	\nindex{$\LLequi\Lambda(H;C) = \{\text{essentially $\Lambda$-equivariant maps in $\LL\infty(H;C)$}\}$} % no page break here !!!
		$$ \LLequi\Lambda(H;C) 
		= \bigset{ \psi \in \LL\infty(H;C) }{ 
		\begin{matrix} \text{$\psi$ is essentially} \\ \text{$\Lambda$-equivariant} \end{matrix}
		} .$$
	(To say $\psi$ is \defit[essentially!$\Lambda$-equivariant]{essentially $\Lambda$-equivariant} means, for each $\lambda \in \Lambda$, that $\psi(\lambda h) = \lambda \cdot \psi(h)$ for a.e.\ $h \in H$.)
	\end{enumerate}
\end{notation}

\begin{egs} \ 
\noprelistbreak
\begin{enumerate}
\item Suppose $H$ is discrete. Then every function on~$H$ is measurable, so $\LL\infty(H;C) = C^H$ is the cartesian product of countably many copies of~$C$. Therefore, in this case, {Tychonoff's Theorem}~\pref{TychonoffThm} implies that $\LL\infty(H;C)$ is compact.
\item If $C$ is the closed unit disk in the complex plane (and $H$ is arbitrary), then $\LL\infty(H;C)$ is the closed unit ball in the Banach space $\LL\infty(H)$, so the Banach-Alaoglu Theorem (\cref{BanachAlaogluThm}) states that it is compact in the weak$^*$ topology. 
\end{enumerate}
\end{egs}

More generally, if we put a technical restriction on~$C$, then there is a weak topology on $\LL\infty(H;C)$ that makes it into a compact, convex $H$-space:

\begin{lem} \label{Linfty(H;C)good}
Assume 
\noprelistbreak
	\begin{itemize}
	\item $\Lambda$ is a closed subgroup of~$H$,
	\item $C$ is a nonempty, compact, convex $H$-space,
	and
	\item $C$ is contained in the dual of some separable Banach space~$\Banach$.
	\end{itemize}
Then $\LL\infty(H; C)$ and $\LLequi\Lambda(H; C)$ are nonempty, compact, convex $H$-spaces.
\end{lem}

\begin{proof}
Let $\LL\infty(H;\Banach^*)$ be the space of all bounded measurable functions from~$H$ to~$\Banach^*$ (where two functions are identified if they are equal a.e.). This is the dual of the (separable) Banach space $\LL1(H;\Banach)$, so it has a natural weak$^*$ topology. Since $\LL\infty(H; C)$ is a closed, bounded, convex subset of $\LL\infty(H;\Banach^*)$, the Banach-Alaoglu Theorem \pref{BanachAlaogluThm} tells us that it is weak$^*$ compact.
In addition, the action of~$H$ by right-translation on $\LL\infty(H; C)$ is continuous \csee{HContOnL(H;C)}.

It is not difficult to see that $\LLequi\Lambda\!(H;C)$ is a nonempty, closed, convex, $H$-invariant subset \csee{Equi(H;C)ClosedConvex}.
\end{proof}

\begin{proof}[\bf Proof of \cref{SubgrpAmen}]
Let $\Lambda$ be a closed subgroup of an amenable Lie group~$H$.
Given any continuous action of~$\Lambda$ on a compact, metrizable space~$X$, it suffices to show there is a $\Lambda$-invariant probability measure on~$X$ \fullcsee{AmenEquiv}{InvMeas}.
From \cref{Linfty(H;C)good}, we know that $\LLequi\Lambda \bigl( H; \Prob(X) \bigr)$ is a nonempty, compact, convex $H$-space. Therefore, the amenability of~$H$ implies that $H$ has a fixed point~$\psi$ in~$\LLequi\Lambda (H;C)$. So $\psi$ is essentially $H$-invariant. If we fix any $\lambda \in \Lambda$, then, for a.e.\ $h \in H$, we have
	\begin{align*}
	\lambda \cdot \psi(h) 
	&= \psi(\lambda h) 
	&& \text{($\psi$ is essentially $\Lambda$-equivariant)} 
	\\&= \psi(h) 
	&& \text{($\psi$ is essentially $H$-invariant)} 
	. \end{align*}
If we assume, for simplicity, that $\Lambda$ is countable \csee{SubgrpIsAmen(NotDiscrete)}, then the quantifiers can be reversed (because the union of countably many null sets is a null set), so we conclude that the probability measure $\psi(h)$ is $\Lambda$-invariant.
\end{proof}

\begin{exercises}

\item \label{HContOnL(H;C)}
Show that the action of~$H$ on $\LL\infty(H;C)$ by right translations is continuous in the weak$^*$-topology.

\item \label{GammaEquiEx}
  Suppose $\Lambda$ is a closed subgroup of~$H$, and that $\Lambda$ acts measurably on a measure
space~$\Omega$. Show there is a $\Lambda$-equivariant,
measurable map $\psi \colon H \to \Omega$.
 \hint{$\psi$ can be defined arbitrarily on a strict
fundamental domain for~$\Lambda$ in~$H$.}

\item \label{Equi(H;C)ClosedConvex}
Show that $\LLequi\Lambda(H;C)$ is a nonempty, closed, convex, $H$-invariant subset of $\LL\infty(H;C)$.

\item \label{SubgrpIsAmen(NotDiscrete)}
Prove \cref{SubgrpAmen} without assuming $\Lambda$ is countable.
\hint{Consider $\lambda$ in a countable dense subset of~$\Lambda$.}

\end{exercises}




 
 
 \section{Equivariant maps from \texorpdfstring{$G/P$}{G/P} to \texorpdfstring{$\Prob(X)$}{Prob(X)}}\label{AmenEquiMapSect}

We now use amenability to prove a basic result that has important consequences for the theory of arithmetic groups. In particular, it is an ingredient in two fundamental results of G.\,A.\,Margulis: his Superrigidity Theorem \pref{MargSuperG'} and his Normal Subgroups Theorem \pref{MargNormalSubgrpsThm}.

\begin{prop}[(\thmindex{Furstenberg's Lemma}Furstenberg's Lemma)] \label{G/amen->Meas(X)}
 If 
\noprelistbreak
 \begin{itemize}
% \item $\Gamma$ is a lattice in~$G$,
 \item $P$ is a closed, amenable subgroup of~$G$,
 and
 \item $\Gamma$ acts continuously on a compact metric space~$X$,
 \end{itemize}
 then there is a Borel measurable map $\psi \colon G/P \to
\Prob(X)$, such that $\psi$ is essentially $\Gamma$-equivariant.
 \end{prop}

\begin{proof}
%The argument is similar to the proof of \cref{SubgrpAmen}.
%
\Cref{Linfty(H;C)good} tells us that $\LLequi\Gamma \bigl( G;\Prob(X) \bigr)$
is a nonempty, compact, convex $G$-space. By restriction, it is also a nonempty, compact, convex $P$-space, so $P$ has a fixed point~$\psi_0$ (under the action by right-translation). 
Then $\psi_0$ factors through
to an (essentially) well-defined map $\psi \colon G/P \to
\Prob(X)$. Because $\psi_0$ is $\Gamma$-equivariant, it is
immediate that $\psi$~is $\Gamma$-equivariant.
 \end{proof}
 
In applications of \cref{G/amen->Meas(X)}, the subgroup~$P$ is usually taken to be a \index{parabolic!subgroup!minimal}minimal parabolic subgroup. % \csee{MinParabDefn,MinParabIsAmen}. 
Here is an example of this:

\begin{cor} \label{SL3R/P->Meas(X)}
If
\noprelistbreak
	\begin{itemize}
	\item $G = \SL(3,\real)$,
	\item $ P
	 = \begin{Smallbmatrix} \upast&& \\ \upast&\upast& \\ \upast&\upast&\upast \end{Smallbmatrix}
	 \subset G $,
	 and
	 \item $\Gamma$ acts continuously on a compact metric
space~$X$,
 \end{itemize}
 then there is a Borel measurable map $\psi \colon G/P \to
\Prob(X)$, such that $\psi$ is essentially
$\Gamma$-equivariant.
 \end{cor}

\begin{proof}
$P$ is amenable \csee{BorelSL3Amen}.
\end{proof}

\begin{rem} The function~$\psi$ that is provided by Furstenberg's Lemma \pref{G/amen->Meas(X)} (or \cref{SL3R/P->Meas(X)}) can be thought of as being a ``random'' map from~$G/P$ to~$X$; for each $z \in G/P$, the value of~$\psi(z)$ is a probability distribution that defines a random value for the function at the point~$z$. However, we will see in \cref{QuickProximalitySect} that the theory of proximality makes it possible to show, in certain cases, that $\psi(z)$ is actually a single well-defined point of~$X$, not a random value that varies over some range.
\end{rem}

\begin{exercises}

\item \label{MinParabIsAmen}
Show that every minimal parabolic subgroup of~$G$ is amenable.
\hint{Langlands decomposition \pref{LanglandsDecomp}.}

\end{exercises}











\section{More properties of amenable groups \texorpdfstring{\optional}{(optional)}}

In this section, we mention (without proof, and without even defining all of the terminology) a variety of very interesting properties of amenable groups. For simplicity, 
	$$ \text{we assume $\Lambda$ is a discrete group.} $$





\subsection{Bounded harmonic functions}

\begin{defn} 
Fix a probability measure~$\mu$ on~$\Lambda$.	
	\begin{enumerate}
	\item A function $f \colon \Lambda \to \real$ is \defit[harmonic!function]{$\mu$-harmonic} if $f = \mu *f$. This means, for every $\lambda \in \Lambda$, 
	$$ f(\lambda) = \sum_{x \in \Lambda} \mu(x) \, f( x \lambda )  .$$
	\item $\mu$ is \defit[symmetric!measure]{symmetric} if $\mu(A^{-1}) = \mu(A)$ for every $A \subseteq \Lambda$.
	\end{enumerate}
\end{defn}

\begin{thm} \label{AmenNoBddHarm}
$\Lambda$ is amenable if and only if there exists a symmetric probability measure~$\mu$ on~$\Lambda$, such that 
\noprelistbreak
	\begin{enumerate}
	\item the support of~$\mu$ generates~$\Lambda$,
	and
	\item every bounded, $\mu$-harmonic function on~$\Lambda$ is constant.
	\end{enumerate}
\end{thm}

Because any harmonic function is the Poisson integral of a function on the Poisson boundary (and vice-versa), this result can be restated in the following equivalent form:

\begin{cor}
$\Lambda$ is amenable if and only if there exists a symmetric probability measure~$\mu$ on~$\Lambda$, such that
\noprelistbreak
	\begin{enumerate}
	\item the support of~$\mu$ generates~$\Lambda$,
	and
	\item the Poisson boundary of~$\Lambda$ {\rm(}with respect to~$\mu${\rm)} consists of a single point.
	\end{enumerate}
\end{cor}




\subsection{Norm of a convolution operator}

\begin{defn}
For any probability measure~$\mu$ on~$\Lambda$, there is a corresponding convolution operator $C_\mu$ on $\LL2(\Lambda)$, defined by
	$$ (C_\mu f )(\lambda) = \sum_{x \in \Lambda} \mu(x) \, f(x^{-1} \lambda) .$$
\end{defn}

\begin{thm} \label{Amen<>ConvOp} 
Let $\mu$ be any probability measure on~$\Lambda$, such that the support of~$\mu$ generates~$\Lambda$. Then $\| C_\mu \| = 1$ if and only if $\Lambda$ is amenable.
\end{thm}



\subsection{Spectral radius}

In geometric terms, the following famous result characterizes amenability in terms of the spectral radius of random walks on Cayley graphs.

\begin{thm}[(Kesten)]
Let $\mu$ be a finitely supported, symmetric probability measure on~$\Lambda$, such that the support of~$\mu$ generates~$\Lambda$. Then $\mu$ is amenable if and only if
	$$ \lim_{n \to \infty} \left( 
	\raise13pt\hbox{$\displaystyle % @@@ raise to reduce blank space in the formula
	 \sum_{\begin{matrix}
	g_1,\ldots,g_{2n} \in \mathop{\mathrm{supp}} \mu
	\\
	g_1g_2 \cdots g_{2n} = e
	\end{matrix}}
	\mu(g_1) \, \mu(g_2) \cdots \mu(g_{2n}) 
	$}
	\right)^{1/2n} = 1
	. $$
\end{thm}




\subsection{Positive-definite functions}

\begin{defn}[\ccf{PosDefTerm}]
A $\complex$-valued function~$\varphi$ on~$\Lambda$ is \defit[positive!definite]{positive-definite} if, for all $a_1,\ldots,a_n \in \Lambda$, the matrix
	$$ [a_{i,j}]_{i,j=1}^n = \bigl[ \varphi( a_i^{-1} a_j ) \bigr] $$
is Hermitian and has no negative eigenvalues.
\end{defn}


\begin{thm}
$\Lambda$ is amenable if and only if\/ $\sum_{g \in \Lambda} \varphi(g) \ge 0$ for every\/ \textup(finitely supported\/\textup) positive-definite function~$\varphi$ on~$\Lambda$.
\end{thm}

%Wikipedia: Godement condition. Every bounded positive-definite measure ? on G satisfies ?(1) ³ 0. Valette (1998) improved this criterion by showing that it is sufficient to ask that, for every continuous positive-definite compactly supported function f on G, the function ?Ð?f has non-negative integral with respect to Haar measure, where ? denotes the modular function.



\subsection{Growth}

\begin{defn}
Assume $\Lambda$ is finitely generated, and fix a symmetric generating set~$S$ for~$\Lambda$. 
%Assume, for simplicity, that $e \in S$.
\noprelistbreak
	\begin{enumerate}
	\item For each $r \in \integer^+$, let $B_r(\Lambda)$ be the ball of radius~$r$ centered at~$e$, More precisely,
		$$ B_r(\Lambda; S) = \{\, \lambda \in \Lambda \mid \exists s_1,s_2,\ldots,s_r \in S \cup \{e\}, \lambda = s_1 s_2 \cdots s_r \,\} .$$
%	\item We say $\Lambda$ has \emph{polynomial growth} if there is some polynomial function $f \colon \real \to \real$, such that $\# B_r < f(r)$, for all $r \in \integer^+$.
	\item We say $\Lambda$ has \emph{subexponential growth} if for every $\epsilon > 0$, we have $\# B_r(\Lambda; S) < e^{\epsilon r}$, for all sufficiently large $r \in \integer^+$.
	\end{enumerate}
\end{defn}

\begin{prop}[\csee{SubexpIsAmenEx}] \label{SubexpIsAmen}
If $\Lambda$ has subexponential growth, then $\Lambda$ is amenable.
\end{prop}

\begin{warn}
The implication in \cref{SubexpIsAmen} goes only one direction: there are many groups (including many solvable groups) that are amenable, but do not have subexponential growth \csee{AmenNotSubExpEx}.
\end{warn}


\subsection{Cogrowth}

\begin{defn} \label{CoGrowthDefn}
Assume $\Lambda$ is finitely generated. Let:
	\begin{enumerate}
	\item  $S = \{s_1,s_2,\ldots,s_k\}$ be a finite generating set of~$\Lambda$.  
	\item  $F_k$ be the free group on~$k$ generators $x_1,\ldots,x_k$.
	\item  $\phi_S \colon F_k \to \Lambda$ be the homomorphism defined by $\phi(x_i) = s_i$.
	\end{enumerate}
The \defit[cogrowth of a discrete group]{cogrowth} of~$\Lambda$ (with respect to~$S$) is
	$$ \lim_{r \to \infty} \frac{1}{r} \log_{2k-1}  \# \bigl( (\ker \phi_S) \cap B_r(F_k; x_1^{\pm1},\ldots,x_k^{\pm1} \bigr) .$$
\end{defn}

Note that $\#B_r (F_k; x_1^{\pm1},\ldots,x_k^{\pm1})$ is equal to the number of reduced words of length~$r$ in the symbols $x_1^{\pm1},\ldots,x_k^{\pm1}$, which is approximately $(2k-1)^r$. Therefore, it is easy to see that that the cogrowth of~$\Lambda$ is between $0$ and $1$ \csee{CoGrowthBtwn01Ex}. The maximum value is obtained if and only if $\Lambda$~is amenable:

\begin{thm} \label{AmenCoGrowth}
$\Lambda$ is amenable if and only if the cogrowth of~$\Lambda$ is~$1$, with respect to some\/ \textup(or, equivalently, every\/\textup) finite generating set~$S$.
\end{thm}


\subsection{Unitarizable representations}

\begin{defn}
Let $\rho \colon \Lambda \to \bddop(\Hilbert)$ be a (not necessarily unitary) representation of~$\Lambda$ on a Hilbert space~$\Hilbert$.
	\begin{enumerate}
	\item $\rho$ is \defit[representation!uniformly bounded]{uniformly bounded} if there exists $C > 0$, such that $\| \rho(\lambda) \| < C$, for all $\lambda \in \Lambda$.
	\item $\rho$ is \defit[representation!unitarizable]{unitarizable} if it is conjugate to a unitary representation. This means there is an invertible operator~$T$ on~$\Hilbert$, such that the representation $\lambda \mapsto T^{-1} \, \rho(\lambda) \, T$ is unitary.
	\end{enumerate}
\end{defn}

It is fairly obvious that every unitarizable representation is uniformly bounded \csee{UnitIsUnifBddEx}. The converse is not true, although it holds for amenable groups:

\begin{thm} \label{AmenUnifBddIsUnit}
If $\Lambda$ is amenable, then every uniformly bounded representation of~$\Lambda$ is unitarizable.
\end{thm}

\begin{rem}
The converse of \cref{AmenUnifBddIsUnit} is an open question.
\end{rem}




\subsection{Almost representations are near representations}

\begin{defn}
Fix $\epsilon > 0$, and let $\varphi$ be a function from~$\Lambda$ to the group $\unitop(\Hilbert)$ of unitary operators on a Hilbert space~$\Hilbert$.
	\begin{enumerate}
	\item $\varphi$ is \defit[representation!almost]{$\epsilon$-almost} a unitary representation if 
		$$ \text{$\| \varphi(\lambda_1 \lambda_2) -  \varphi(\lambda_1) \, \varphi( \lambda_2) \| < \epsilon$ for all $\lambda_1,\lambda_2 \in \Lambda$.} $$
	\item $\varphi$ is \defit[representation!near]{$\epsilon$-near} a unitary representation if there exists a unitary representation $\rho \colon \Lambda \colon \unitop(\Hilbert)$, such that 
		$$ \text{$\| \varphi(\lambda) - \rho(\lambda) \| < \epsilon$ for every $\lambda \in \Lambda$.} $$
	\end{enumerate}
\end{defn}

For amenable groups, every almost representation is near a representation:

\begin{thm} \label{AlmRepsAreNear}
Assume $\Lambda$ is amenable. Given $\epsilon > 0$, there exists $\delta > 0$, such that if $\varphi$ is $\delta$-almost a unitary representation, then $\varphi$~is $\epsilon$-near a unitary representation.
\end{thm}



\subsection{Bounded cohomology}

The bounded cohomology groups of~$\Lambda$ are defined just like the ordinary cohomology groups, except that all cochains are assumed to be bounded functions.

\begin{defn}
Assume $\Banach$ is a Banach space.
\noprelistbreak
	\begin{enumerate}
	\item $\Banach $ is a \defit[Banach!$\Lambda$-module]{Banach $\Lambda$-module} if $\Lambda$ acts continuously on~$\Banach$, by linear isometries.
	\item $\cocyc1\bdd(\Lambda;\Banach) = \cocyc1(\Lambda;\Banach) \cap \LL\infty(\Lambda;\Banach)$.
	\item $\coho1\bdd(\Lambda;\Banach) = \cocyc1\bdd(\Lambda;\Banach) / \cobdry1(\Lambda;\Banach)$.
	\end{enumerate}
\end{defn} 

\begin{thm} \label{AmenBddCoho0}
$\Lambda$ is amenable if and only if\/ $\coho1\bdd(\Lambda;\Banach) = 0$ for every Banach $\Lambda$-module~$\Banach$.
\end{thm}

\begin{rem} \label{AmenBddCoho0Alln}
In fact, if $\Lambda$ is amenable, then $\coho {\raise 1pt \hbox{$\scriptstyle n$}}\bdd(\Lambda;\Banach) = 0$ for all~$n$.
\end{rem}





\subsection{Invariance under quasi-isometry}

\begin{prop}[\csee{AmenQIinvtEx}] \label{AmenQIinvt}
Assume $\Lambda_1$ and~$\Lambda_2$ are finitely generated groups, such that $\Lambda_1$~is quasi-isometric to~$\Lambda_2$ \csee{QuasiIsomDefn}.
Then $\Lambda_1$~is amenable if and only if $\Lambda_2$~is amenable.
\end{prop}


\subsection{Ponzi schemes}

Assume $\Lambda$ is finitely generated, and let $d$ be the word metric on~$\Lambda$, with respect to some finite, symmetric generating set~$S$ \csee{WordMetricDefn}.

\begin{defn}
A function $f \colon \Lambda \to \Lambda$ is a \defit{Ponzi scheme on~$\Lambda$} if there is some $C > 0$, such that, for all $\lambda \in \Lambda$, we have:
		\begin{enumerate}
		\item $\# \, f^{-1}(\lambda) \ge 2$,
		and
		\item $d \bigl( f(\lambda), \lambda \bigr) < C$.
		\end{enumerate}
\end{defn}


\begin{thm} \label{AmenNoPonzi}
%Assume $S$ is a finite, symmetric generating set of~$\Lambda$.
 $\Lambda$ is amenable if and only if there does not exist a Ponzi scheme on~$\Lambda$.
\end{thm}





\begin{exercises}

\item \label{SubexpIsAmenEx}
Prove \cref{SubexpIsAmen}.
\hint{If no balls are F\o lner sets, then the group has exponential growth.}

\item \label{AmenNotSubExpEx}
Choose a prime number~$p$, and let
	$$ \Lambda = \bigset{
	\begin{bmatrix} p^k & mp^n \\ 0 & p^{-k} \end{bmatrix} 
	}{ k,m,n \in \integer}
	\subset \SL(2,\rational) ,$$
with the discrete topology. Show $\Lambda$ is an amenable group that does not have subexponential growth.

\item \label{CoGrowthBtwn01Ex}
In the notation of \cref{CoGrowthDefn}, show:
	\begin{enumerate}
	\item $\# B_r(F_k; x_1^{\pm1},\ldots,x_k^{\pm1}) = 2k(2k-1)^{r-1}$.
	\item If $\mathop{\text{\rm cog}} \Lambda$ is the cogrowth of~$\Lambda$, then $0 \le \mathop{\text{\rm cog}} \Lambda \le 1$.
	\end{enumerate}

\item \label{UnitIsUnifBddEx}
Show that every unitarizable representation is uniformly bounded.

\item For every $\epsilon > 0$, show there exists $\delta > 0$, such that if $\varphi$ is $\delta$-near a unitary representation, then $\varphi$ is $\epsilon$-almost a unitary representation.

\item \label{AmenQIinvtEx}
Prove \cref{AmenQIinvt}.
\hint{Show that if $\Lambda$ is not amenable, then, for every $k$, it has a finite subset~$S$, such that $\#(S F) \ge k \cdot \# F$ for every finite subset~$F$ of~$\Lambda$.}

\item Explicitly construct a Ponzi scheme on the free group with two generators.

\item Show (without using \cref{AmenNoPonzi}) that if $\Lambda$ is amenable, then there does not exist a Ponzi scheme on~$\Lambda$.
\hint{F\o lner sets.}

\end{exercises}



\begin{notes}

The notion of amenability is attributed to J.\,von\,Neumann \cite{vonNeumann-AllgemeineMasses}, but he used the German word ``messbar'' (which can be translated as ``measurable''). The term ``amenable'' was apparently introduced into the literature by M.\,Day \cite[\#507, p.~1054]{Day-MeansAbstract} in the announcement of a talk.

The monographs \cite{PatersonBook,PierBook} are standard references on amenability.
Briefer treatments are in \cite[App.~G]{BekkaHarpeValette-T}, \cite{GreenleafBook}, and \cite[\S4.1]{ZimmerBook}.
Quite a different approach to amenability appears in \cite[Chaps.~10--12]{WagonBook} (for discrete groups only).

The fact that closed subgroups of amenable groups are amenable (\cref{SubgrpAmen}) is proved in \cite[Thm.~2.3.2, pp.~30--32]{GreenleafBook}, \cite[Prop.~13.3, p.~118]{PierBook}, and \cite[Prop.~4.2.20, p.~74]{ZimmerBook}.

See \cite[p.~67]{GreenleafBook} for a proof of \cref{Amen<>Folner}($\Rightarrow$) that does not require $H$ to be discrete.

\Cref{L2invtIffL1inv} is proved in \cite[pp.~46--47]{GreenleafBook}.

The solution of \cref{MetrizableSuffices} can be found in \cite[Thm.~5.4, p.~45]{PierBook}.

For a proof of the fact (mentioned in the hint to \cref{SubgrpAmenFinAdd}) that the one-to-one continuous image of a Borel set is Borel, see \cite[Thm.~3.3.2, p.~70]{Arveson-Cstar}.

Our proof of \cref{Amen<>Folner}($\Rightarrow$) is taken from \cite[pp.~66-67]{GreenleafBook}.

\fullCref{FreeSubgrpRem}{Olshanskii}, the existence of a nonamenable group with no nonabelian free subgroup, is due to Olshanskii \cite{Olshanskii-ExInvMean}. (In this example, called an ``Olshanskii Monster'' or ``Tarski Monster\zz,'' every proper subgroup of the group is a cyclic group of prime order, so there is obviously no free subgroup.) A much more elementary example has recently been constructed by N.\,Monod \cite{Monod-PiecewiseProj}.

The book of S.\,Wagon \cite{WagonBook} is one of the many places to read about the \thmindex{Banach-Tarski Paradox}Banach-Tarski Paradox (\fullcref{FreeSubgrpRem}{BanachTarski}).

Furstenberg's Lemma \pref{G/amen->Meas(X)} appears in \cite[Thm.~15.1]{Furstenberg-BdThyStochProc}. Another proof can be found in \cite[Prop.~4.3.9, p.~81]{ZimmerBook}. 

\Cref{AmenNoBddHarm} is due to Kaimanovich and Vershik \cite[Thms.~4.2 and~4.4]{KaimanovichVershik-RandWalksBdryEntropy} and (independently) Rosenblatt \cite[Props.~1.2 and 1.9 and Thm.~1.10]{Rosenblatt-ErgMixRandWalks}.

\Cref{Amen<>ConvOp} is due to H.\,Kesten (if $\mu$ is symmetric). See \cite[G.4.4]{BekkaHarpeValette-T} for a proof.

A proof of \cref{SubexpIsAmen} can be found in \cite[Props.~12.5 and~12.5]{PierBook}.

\Cref{AmenCoGrowth} was proved by R.\,I.\,Grigorchuk \cite{Grigorchuk-SymmRandWalks} and J.\,M.\,Cohen \cite{Cohen-cogrowth} (independently). 

\Cref{AmenUnifBddIsUnit} was proved by J.\,Dixmier and M.\,Day in 1950 (independently). 
See \cite{Pisier-UnitAmen} for historical remarks and progress on the converse.
(Another result on the converse is proved in \cite{MonodOzawa-Unitarizable}.)

\Cref{AlmRepsAreNear} is due to D.\,Kazhdan \cite{Kazhdan-EpsilonRep}.

\Cref{AmenBddCoho0,AmenBddCoho0Alln} are due to B.\,E.\,Johnson \cite{Johnson-CohoBanachAlg}. See \cite{Monod-InvBddCoho} (and its many references) for an introduction to bounded cohomology.

\Cref{AmenNoPonzi} appears in \cite[6.17 and 6.17\raise1pt\hbox{$\scriptstyle\frac{1}{2}$}, p.~328]{Gromov-MetricStructures}.

\end{notes}


\begin{references}{[99]}

\bibitem{Arveson-Cstar}
W.\,Arveson:
\emph{An Invitation to $C^*$-algebras.} 
%Graduate Texts in Mathematics, No. 39. 
Springer, New York, 1976.
ISBN 0-387-90176-0,
\MR{0512360}

\bibitem{BekkaHarpeValette-T}
B.\,Bekka, P.\,de\,la\,Harpe, and A.\,Valette:
\emph{Kazhdan's Property (T),}
Cambridge U.\ Press, Cambridge, 2008.
ISBN 978-0-521-88720-5,
\MR{2415834},
\maynewline
\url{http://perso.univ-rennes1.fr/bachir.bekka/KazhdanTotal.pdf}

\bibitem{Cohen-cogrowth}
J.\,M.\,Cohen:
Cogrowth and amenability of discrete groups,
\emph{J. Funct. Anal.} 48 (1982), no.~3, 301--309. 
\MR{0678175},
\url{http://dx.doi.org/10.1016/0022-1236(82)90090-8}

\bibitem{Day-MeansAbstract}
J.\,W.\,Green:
 The summer meeting in Boulder,
 \emph{Bulletin of the American Mathematical Society} 55 (1949), no.~11, 1035--1081. 
 \maynewline
 \url{http://projecteuclid.org/euclid.bams/1183514222}

\bibitem{Furstenberg-BdThyStochProc}
H.\,Furstenberg:
Boundary theory and stochastic processes on homogeneous spaces, in
C.\,C.\,Moore, ed.:
\emph{Harmonic Analysis on Homogeneous Spaces 
%(Proc. Sympos. Pure Math., Vol. XXVI, Williams Coll., 
(Williamstown, Mass., 1972)}. 
Amer. Math. Soc., Providence, R.I., 1973,
pp.~193--229.
ISBN 0-8218-1426-5,
\MR{0352328}

\bibitem{GreenleafBook}
F.\,P.\,Greenleaf:
\emph{Invariant Means on Topological Groups and Their Applications}.
Van Nostrand, New York, 1969.
ISBN 0-442-02857-1,
\MR{0251549}

\bibitem{Gromov-MetricStructures}
M.\,Gromov:
\emph{Metric Structures for Riemannian and Non-Riemannian Spaces.}
Birkh\"auser, Boston, 2007. 
ISBN 978-0-8176-4582-3,
\MR{2307192}

\bibitem{Grigorchuk-SymmRandWalks}
R.\,I.\,Grigorchuk:
Symmetrical random walks on discrete groups,
in: 
\emph{Multicomponent Random Systems.}
%Adv. Probab. Related Topics, 6, 
Dekker, New York, 1980, pp.~285--325.
ISBN 0-8247-6831-0,
\MR{0599539}

\bibitem{Johnson-CohoBanachAlg}
B.\,E.\,Johnson:
\emph{Cohomology in Banach algebras.}
%Memoirs of the American Mathematical Society, No. 127. 
American Mathematical Society, Providence, R.I., 1972.
\MR{0374934}

\bibitem{KaimanovichVershik-RandWalksBdryEntropy}
 V.\,A.\,Kaimanovich and A.\,M.\,Vershik
 Random walks on discrete groups: boundary and entropy,
 \emph{Ann. Probab.} 11 (1983), no.~3, 457--490.
 \MR{0704539},
 \maynewline
 \url{http://www.jstor.org/stable/2243645}

\bibitem{Kazhdan-EpsilonRep}
D.\,Kazhdan:
On $\varepsilon$-representations, 
\emph{Israel J. Math.} 43 (1982), no.~4, 315--323.
\MR{0693352},
\url{http://dx.doi.org/10.1007/BF02761236}

\bibitem{Monod-InvBddCoho}
N.\,Monod:
An invitation to bounded cohomology,
in: \emph{International Congress of Mathematicians, Vol.~II,}
Eur. Math. Soc., Z\"urich, 2006, pp.~1183--1211.
\MR{2275641},
\maynewline
\url{http://www.mathunion.org/ICM/ICM2006.2/}

\bibitem{Monod-PiecewiseProj}
N.\,Monod:
Groups of piecewise projective homeomorphisms
\emph{Proc. Natl. Acad. Sci. USA} 110 (2013), no.~12, 4524--4527. 
\MR{3047655}, 
\maynewline
\url{http://dx.doi.org/10.1073/pnas.1218426110}

\bibitem{MonodOzawa-Unitarizable}
N.\,Monod and N.\,Ozawa:
The Dixmier problem, lamplighters and Burnside groups,
\emph{J. Funct. Anal.} 258 (2010), no.~1, 255--259. 
\MR{2557962},
\maynewline
\url{http://dx.doi.org/10.1016/j.jfa.2009.06.029}

\bibitem{Olshanskii-ExInvMean}
A.\,Yu.\,Ol'shanskii:
On the problem of the existence of an invariant mean on a group.
\emph{Russian Mathematical Surveys} 35 (1980), No.~4, 180--181.
\MR{0586204},
\maynewline
\url{http://dx.doi.org/10.1070/RM1980v035n04ABEH001876}

\bibitem{PatersonBook}
A.\,L.\,T.\,Paterson:
\emph{Amenability.}
%Mathematical Surveys and Monographs, 29. 
American Mathematical Society, Providence, RI, 1988. 
ISBN 0-8218-1529-6,
\MR{0961261}

\bibitem{PierBook}
J.-P.\,Pier:
\emph{Amenable Locally Compact Groups}.
%Pure and Applied Mathematics (New York). Wiley-Interscience Publication. 
Wiley, New York, 1984.
ISBN 0-471-89390-0,
\MR{0767264}

\bibitem{Pisier-UnitAmen}
G.\,Pisier:
Are unitarizable groups amenable?,
in  
L.\,Bartholdi et al., eds.:
\emph{Infinite Groups: Geometric, Combinatorial and Dynamical Aspects}.
%Progr. Math., 248, 
Birkh\"auser, Basel, 2005, pp.~323--362.
ISBN 3-7643-7446-2,
\MR{2195457}

\bibitem{Rosenblatt-ErgMixRandWalks}
J.\,Rosenblatt:
Ergodic and mixing random walks on locally compact groups,
\emph{Math. Ann.} 257 (1981), no.~1, 31--42. 
\MR{0630645},
\maynewline
\url{http://eudml.org/doc/163540}
%\url{http://resolver.sub.uni-goettingen.de/purl?GDZPPN002320959}

\bibitem{vonNeumann-AllgemeineMasses}
J.\,von\,Neumann:
Zur allgemeinen Theorie des Masses,
\emph{Fund. Math.} 13, no.~1 (1929) 73--116.
\Zbl{55.0151.02},
\maynewline 
\url{http://eudml.org/doc/211916}
%\url{http://pldml.icm.edu.pl/mathbwn/element/bwmeta1.element.bwnjournal-article-fmv13i1p6bwm}

\bibitem{WagonBook}
S.\,Wagon:
\emph{The Banach-Tarski Paradox}. 
%With a foreword by Jan Mycielski. Corrected reprint of the 1985 original. 
Cambridge U.\ Press, Cambridge, 1993. 
ISBN 0-521-45704-1,
\MR{1251963}

\bibitem{ZimmerBook}
R.\,J.\,Zimmer:
\emph{Ergodic Theory and Semisimple Groups}.
Monographs in Mathematics, 81. 
Birkh\"auser, Basel, 1984.
ISBN 3-7643-3184-4,
\MR{0776417}

 \end{references}

