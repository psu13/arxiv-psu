%!TEX root = IntroArithGrps.tex

\mychapter{Unitary Representations} \label{UnitaryRepChap}

\prereqs{none.}

Unitary representations are of the utmost importance in the study of Lie groups. For our purposes, one of the main applications is the proof of the Moore Ergodicity Theorem \pref{MooreErgBasicThm} in \cref{MooreErgPfSect}, but they are also the foundation of the definition (and study) of Kazhdan's Property~$(T)$ in \cref{KazhdanTChap}.



\section{Definitions} \label{UnitaryRepSect}

\begin{defn}
Assume $\Hilbert$ is a Hilbert space, with inner product $\langle~\mid~\rangle$,
\nindex{$\Hilbert$ = Hilbert space with inner product $\langle~\mid~\rangle$}
and $H$~is a Lie group.
	\begin{enumerate}
	
	\item $\unitary(\Hilbert)$ is the group of {unitary operators} on~$\Hilbert$. 
	\nindex{$\unitary(\Hilbert)$ = group of unitary operators on~$\Hilbert$}
%	Therefore, 
%	$$ \unitary(\Hilbert) = \bigset{
%		T \colon \Hilbert \to \Hilbert
%		}{
%		\begin{matrix}
%		\text{$T$ is linear, and} \\
%		\langle Tv \mid Tw \rangle = \langle v \mid w \rangle, \ 
%		\forall v,w \in \Hilbert
%		\end{matrix}
%		} .$$
	
	\item A \defit[unitary!representation]{unitary representation} of the Lie group~$H$ on the Hilbert space~$\Hilbert$ is a homomorphism 
	$\pi \colon H \to \unitary(\Hilbert)$, 
	\nindex{$\pi$ = unitary representation}
	such that the map $h \mapsto \pi(h) \varphi$ is continuous, for each $\varphi \in \Hilbert$. (If we wish to spell out that a unitary representation is on a particular Hilbert space~$\Hilbert$, we may refer it as $(\pi, \Hilbert)$, rather than merely~$\pi$.) 
	
	\item The \defit[dimension of a representation]{dimension} of a unitary representation $(\pi,\Hilbert)$ is the dimension of the Hilbert space~$\Hilbert$.

	\item Suppose $(\pi_1, \Hilbert_1)$ and $(\pi_2, \Hilbert_2)$ are unitary representations  of~$H$.
\noprelistbreak
		\begin{enumerate}
		
		\item The \defit[direct sum!of representations]{direct sum} of the representations $\pi_1$ and~$\pi_2$ is the unitary representation $\pi_1 \oplus \pi_2$ of~$H$ on $\Hilbert_1 \oplus \Hilbert_2$ that is defined by 
			$$(\pi_1 \oplus \pi_2)(h)(\varphi_1,\varphi_2) = \bigl( \pi_1(h) \varphi_1, \pi_2(h) \varphi_2 \bigr), $$
		for $h \in H$ and $\varphi_i \in \Hilbert_i$.
		
		\item $\pi_1$ and~$\pi_2$ are \defit[isomorphic unitary representations]{isomorphic} if there is a Hilbert-space isomorphism $T \colon \Hilbert_1 \stackrel{\iso}{\longrightarrow} \Hilbert_2$ that \defit[intertwining operator]{intertwines} the two representations. This means $T \bigl( \pi_1(h) \varphi \bigr) = \pi_2(h) \, T(\varphi)$, for all $h \in H$ and $\varphi \in \Hilbert_1$.
		\end{enumerate}

	\end{enumerate}
\end{defn}

\begin{eg}
Every group~$H$ has a \defit[representation!trivial]{trivial representation}, denoted by~$\trivrep$
	\nindex{$\trivrep$ = trivial representation}%
(or $\trivrep_H$, if it will avoid confusion). It is a unitary representation on the $1$-dimensional Hilbert space~$\complex$, and is defined by $\trivrep(h) \varphi = \varphi$ for all $h \in H$ and $\varphi \in \complex$.
\end{eg}

Here is a more interesting example:

\begin{eg} \label{RepL2}
Suppose 
	\begin{itemize}
	\item $H$ is a Lie group,
	\item $H$ acts continuously on a locally compact, metrizable space~$X$, 
	and 
	\item $\mu$ is an $H$-invariant Radon measure on~$X$.
	\end{itemize}
Then there is a unitary representation of~$H$ on $\LL2(X,\mu)$,
	% be translations yields a representation $\pi \colon G \to \unitary \bigl( \LL2(X) \bigr)$, 
defined by
	$$ \bigl( \pi(h) \varphi \bigr)(x) = \varphi(h^{-1} x) $$
\ccf{GContOnLp}.
For the action of~$H$ on itself by translations (on the left), the resulting representation~$\regrep$\nindex{$\regrep$ = regular representation}
of~$H$ on $\LL2(H)$ is called the (left) \defit[regular!representation]{regular representation}\index{representation!regular|indsee{regular representation}} of~$H$.
\end{eg}

\begin{defn} \label{OrthoCompDefn}
Suppose $\pi$ is a unitary representation of~$H$ on~$\Hilbert$, $H'$~is a subgroup of~$H$, and $\mathcal{K}$ is a closed subspace of~$\Hilbert$.
	\begin{enumerate}
	\item $\mathcal{K}$ is \defit[invariant!subspace]{$H'$-invariant} if $\pi(h') \mathcal{K} = \mathcal{K}$, for all $h' \in H'$. (If the representation~$\pi$ is not clear from the context, we may also say that $\mathcal{K}$ is $\pi(H')$-invariant.)
	
	\item For the special case where $H' = H$, an $H$-invariant subspace is simply said to be \defit[invariant!subspace]{invariant}, and the representation of~$H$ on any such subspace is called a \defit{subrepresentation} of~$\pi$. More precisely, if $\mathcal{K}$ is $H$-invariant (and closed), then the corresponding subrepresentation is the unitary representation $\pi_{\mathcal{K}}$ of~$H$ on~$\mathcal{K}$, defined by $\pi_{\mathcal{K}}(h)\varphi = \pi(h) \varphi$, for all $h \in H$ and $\varphi \in \mathcal{K}$.
	\end{enumerate}
\end{defn}

\begin{lem}[\csee{subspace->sumEx}] \label{subspace->sum}
If $(\pi,\Hilbert)$ is a unitary representation of~$H$, and $\mathcal{K}$ is a closed, $H$-invariant subspace of~$\Hilbert$, then $\pi \iso \pi_{\mathcal{K}} \oplus \pi_{\mathcal{K}^\perp}$.
\end{lem}

The above \lcnamecref{subspace->sum} shows that any invariant subspace leads to a decomposition of the representation into a direct sum of subrepresentations. This suggests that the fundamental building blocks are the representations that do not have any (interesting) subrepresentations. Such representations are called ``irreducible:''

\begin{defns}
Let $H$ be a Lie group.
\noprelistbreak
	\begin{enumerate}
	\item A unitary representation $(\pi,\Hilbert)$ of~$H$ is \defit[irreducible!representation]{irreducible} if it has no nontrivial, proper, closed, invariant subspaces. That is, the only closed, $H$-invariant subspaces of~$\Hilbert$ are $\{0\}$ and~$\Hilbert$.
	\item The set of all irreducible representations of~$H$ (up to isomorphism) is called the \defit[unitary!dual]{unitary dual} of~$H$, and is denoted $\widehat H$.
	\end{enumerate}
\end{defns}

\begin{warns} \label{UnitaryWarns} \ 
\noprelistbreak
	\begin{enumerate}
	\item Unfortunately, it is usually not the case that every unitary representation of~$H$ is a direct sum of irreducible representations. (This is a generalization of the fact that if $U$ is a unitary operator on~$\Hilbert$, then $\Hilbert$~may not be a direct sum of eigenspaces of~$U$.) However, it will be explained in \cref{DirectIntegralSect} that every unitary representation is a ``direct integral'' of irreducible representations. 
In the special case where $H = \integer$, this is a restatement of the Spectral Theorem for unitary operators \ccf{SpectralThm}.

	\item \label{UnitaryWarns-tame}
	Although the unitary dual $\widehat H$ has a fairly natural topology, it can be quite bad. In particular, the topology may not be Hausdorff. Indeed, in some cases, the topology is so bad that there does not exist an injective, Borel measurable function from~$\widehat H$ to~$\real$. Fortunately, though, the worst problems do not arise for semisimple Lie groups: the unitary dual is always ``tame'' (measurably, at least) in this case.
	
	\end{enumerate}
\end{warns}

\begin{exercises}

%\item \label{RepL2Ex}
%Show that the map~$\pi$ of \cref{RepL2} is indeed a unitary representation of~$H$.
%\hint{It is obvious that $\pi(h)$ is linear, and the invariance of~$\mu$ easily implies that it preserves the inner product on $\LL2(X)$. Also, it is easy to check that $\pi$ is a homomorphism. Continuity follows from the fact that continuous functions are dense in~$\LL2(X)$ (by {Lusin's Theorem} \pref{LusinsThm}).}

\item Suppose $\pi$ is a unitary representation of~$H$ on~$\Hilbert$, and define a map $\xi \colon H \times \Hilbert \to \Hilbert$ by $\xi(h,v) = \pi(h) v$. Show that $\xi$ is continuous.
\hint{Use the fact that $\pi(H)$ consists of unitary operators.}

\item \label{subspace->sumEx}
Prove \cref{subspace->sum}.
\hint{If $\mathcal{K}$ is invariant, then $\mathcal{K}^\perp$ is also invariant. %(Since $(\mathcal{K}^\perp)^\perp = \mathcal{K}$, this also implies the converse.)
Define $T \colon \mathcal{K} \oplus \mathcal{K}^\perp \to \Hilbert$ by $T(\varphi,\psi) = \varphi + \psi$.}

\item \label{SchursLemmaUnitary}
(\thmindex{Schur's Lemma}Schur's Lemma)
Suppose $(\pi,\Hilbert)$ is an irreducible unitary representation of~$H$, and $T$~is a bounded operator on~$\Hilbert$ that commutes with every element of $\pi(\Hilbert)$.
Show there exists $\lambda \in \complex$, such that $T \varphi = \lambda \, \varphi$, for every $\varphi \in \Hilbert$.
\hint{Assume $T$ is normal, by considering $T+ T^*$ and $T - T^*$, and apply the Spectral Theorem \pref{SpectralThm}.}

\item Suppose $(\pi,\Hilbert)$ is an irreducible unitary representation of~$H$, and $\langle~\mid~\rangle'$ is another $H$-invariant inner product on~$\Hilbert$ that defines the same topology on~$\Hilbert$ as the original inner product $\langle~\mid~\rangle$. Show there exists $c \in \real^+$, such that $\langle~\mid~\rangle' = c \langle~\mid~\rangle$.
\hint{Each inner product provides an isomorphism of~$\Hilbert$ with~$\Hilbert^*$. \Cref{SchursLemmaUnitary} implies they are the same, up to a scalar multiple.}
	

\item \label{RegRepNoInvtMeas}
\optional\ 
In the situation of \cref{RepL2}, a weaker assumption on~$\mu$ suffices to define a unitary representation on $\LL2(X,\mu)$. Namely, instead of assuming that $\mu$ is invariant, it suffices to assume only that the \emph{class} of~$\mu$ is invariant. This means, for every measurable subset~$A$, and all $h \in H$, we have $\mu(A) = 0 \Leftrightarrow \mu(hA) = 0$. Then, for each $h \in H$, the Radon-Nikodym Theorem \pref{RadonNikodym} provides a function $D_h \colon X \to \real^+$, such that $h_* \mu = D_h \mu$. Show that the formula
	$$\bigl( \pi(h) \varphi \bigr) \  = \  \sqrt{D_h(x)} \  \varphi( h^{-1} x ) $$
defines a unitary representation of~$H$ on $\LL2(X,\mu)$.

\end{exercises}



\section{Proof of the Moore Ergodicity Theorem} \label{MooreErgPfSect}

Recall the following result that was proved only in a special case:

\begin{thm}[{(\thmindex{Moore Ergodicity}{Moore Ergodicity Theorem} \pref{MooreErgBasicThm})}] \label{MooreErgBasicThmReprise}
Suppose
	\begin{itemize}
	\item $G$ is connected and simple,
	\item $H$ is a closed, noncompact subgroup of~$G$,
	\item $\Lambda$ is a discrete subgroup of~$G$,
	and
	\item $\phi$ is an $H$-invariant $\LL{p}$-function on~$G/\Lambda$ \textup(with $1 \le p < \infty$\textup).
	\end{itemize}
Then $\phi$ is constant\/ \textup(a.e.\textup).
\end{thm}

This is an easy consequence of the following result in representation theory \csee{MooreErgodicitySimplePf}. 

\begin{thm}[(\thmindex{Decay of matrix coefficients}{Decay of matrix coefficients})] \label{DecayMatCoeffSimple}
 If
 \begin{itemize}
 \item $G$ is simple,
 \item $\pi$ is a unitary representation of~$G$ on a Hilbert
space~$\Hilbert$, such that no nonzero vector is fixed by $\pi(G)$, and
 \item $\{g_j\}$ is a sequence of elements of~$G$, such that $\lVert g_j
\rVert \to \infty$,
 \end{itemize}
 then $\langle \pi(g_j) \phi \mid \psi \rangle \to 0$, for every $\phi,\psi \in \Hilbert$.
 \end{thm}

\begin{proof}
Assume, for simplicity, that 
	$$G = \SL(2,\real) .$$
(A reader familiar with the theory of real roots and Weyl chambers should have little difficulty in extending this proof to the general case; cf.~\cref{DecayMatCoeffSL3}.) Let
	$$ A = \begin{Smallbmatrix} \upast&\\ &\upast \end{Smallbmatrix} \subset G .$$
Further assume, for simplicity, that 
	$$ \{g_j\} \subseteq A .$$
(It is not difficult to eliminate this hypothesis; see \cref{DecayMatCoeffSimpleNotAEx}.)
 By passing to a subsequence, we may assume $\pi(g_j)$ converges weakly, to
some operator~$E$; that is,
 $$ \langle \pi(g_j) \phi \mid \psi \rangle
 \to \langle E \phi \mid \psi \rangle
 \mbox{ \ for every $\phi,\psi \in \Hilbert$} $$
 \csee{SubseqConvWeakly}.
 Let 
 \begin{align} \label{horodefn}
 U &= \{\, v \in G \mid g_j^{-1} v g_j \to e \,\} \\
 \intertext{and} 
 U^- &= \{\, u \in G \mid g_j u g_j^{-1} \to e \,\} 
. \end{align}
 For $u \in U^-$ and $\phi,\psi \in \Hilbert$, we have
 \begin{align*}
 \langle E\pi(u) \phi \mid \psi \rangle
 &= \lim \langle \pi(g_j u) \phi \mid \psi \rangle
 \\&= \lim \langle \pi(g_j u g_j^{-1}) \, \pi(g_j) \phi \mid \psi \rangle
 \\&= \lim \langle \pi(g_j) \phi \mid \psi \rangle
 && \text{\csee{MatCoeffConv}}
 \\&= \langle E \phi \mid \psi \rangle 
 , \end{align*}
 so $E \pi(u) = E$. Therefore, letting 
 	$$\Hilbert^{U^-} = \{\, \phi \in \Hilbert \mid \text{$\pi(u) \phi = \phi$ for all $u \in U^-$} \,\} $$
be the space of $U^-$-invariant vectors in~$\Hilbert$, we have
 	\begin{align} \label{UperpInKer}
	(\Hilbert^{U^-})^\perp \subseteq \ker E
	\end{align}
\csee{UperpInKerEx}.
Similarly, since
 $$ \langle E^* \phi \mid \psi \rangle
 =  \langle \phi \mid E \psi \rangle
 = \lim \langle \phi \mid \pi(g_j) \psi \rangle
 = \lim \langle \pi(g_j^{-1}) \phi \mid  \psi \rangle ,$$
the same argument, with $E^*$ in the place of~$E$ and $g_j^{-1}$ in the
place of~$g_j$, shows that 
	$$(\Hilbert^U)^\perp \subseteq \ker E^* .$$

We also have
\begin{align*}
 \langle \pi(g_j) \phi \mid \pi(g_k) \phi \rangle
 &= \langle \pi(g_k^{-1} g_j) \phi \mid \phi \rangle
 	&& (\text{$\pi(g_k^{-1})$ is unitary})
 \\&= \langle \pi(g_j g_k^{-1}) \phi \mid \phi \rangle
 	&& (\text{$A$ is abelian}) 
 \\&= \langle \pi(g_k^{-1}) \phi \mid \pi(g_j^{-1}) \phi \rangle
.\end{align*}
Applying $\lim_{j \to \infty} \lim_{k \to \infty}$ to both sides yields
$ \lVert E\phi\rVert^2 = \lVert E^* \phi\rVert^2 $,
 and this implies $\ker E = \ker E^* $. 
Hence, 
 \begin{align*}
 \ker E
 &= \ker E + \ker E^*
 \supset (\Hilbert^{U^-})^\perp + (\Hilbert^U)^\perp
 \\&= (\Hilbert^{U^-} \cap \Hilbert^U)^\perp
 = (\Hilbert^{\langle U, U^- \rangle})^\perp
 . \end{align*}
 Now, by passing to a subsequence of $\{g_j\}$, we may assume $\langle U, U^-
\rangle = G$ \csee{horosubseq}. Then $\Hilbert^{\langle U, U^- \rangle} =
\Hilbert^G = 0$,
 so $\ker E
 \supset 0^\perp
 = \Hilbert$.
 This implies that, for all $\phi,\psi \in \Hilbert$, we have
 \begin{align*}
  \lim \langle \pi(g_j) \phi \mid \psi \rangle
 = \langle E \phi \mid \psi \rangle
 = \langle 0 \mid \psi \rangle
 = 0 . & \qedhere \end{align*}
 \end{proof}

\begin{rem}
If $A$ is a bounded operator on a Hilbert space~$\Hilbert$, and $\phi,\psi \in \Hilbert$, then the inner product $\langle A \phi \mid \psi \rangle$ is called a \defit{matrix coefficient} of~$A$. The motivation for this terminology is that if $A \in \Mat_{n \times n}(\real)$, and $\varepsilon_1,\ldots,\varepsilon_n$ is the standard basis of $\Hilbert = \real^n$, then $\langle A \varepsilon_j \mid \varepsilon_i \rangle$ is the $(i,j)$ matrix entry of~$A$. 
\end{rem}

%The following example illustrates \cref{horosubseq}.

%\begin{eg} 
% Let $G = \SL(3,\real)$, let
% $$ H_1 = 
% \bigset{
% \begin{bmatrix}
% e^{t} & 0 & 0 \\
% 0 & e^{t} & 0 \\
% 0 & 0 & e^{-2t}
% \end{bmatrix}
% }{
% t \in \real
% } ,$$
% and suppose $\{g_j\}$ is some sequence of
%elements of~$H_1$, such that $\lVert g_j \rVert \to \infty$. We may write
% $$ g_j = 
% \begin{bmatrix}
% e^{t_j} & 0 & 0 \\
% 0 & e^{t_j} & 0 \\
% 0 & 0 & e^{-2t_j}
% \end{bmatrix}
% , $$
% where $t_j \in \real$.
% By passing to a subsequence, we may assume that either $t_j \to \infty$ or
%$t_j \to -\infty$. If $t_j \to \infty$, then, in the notation
%of~\pref{horodefn}, we have
% $$
% U = 
% \left\{
% \begin{bmatrix}
% 1 & 0 & * \\
% 0 & 1 & * \\
% 0 & 0 & 1
% \end{bmatrix}
% \right\}
% \mbox{ \qquad and \qquad}
% U^- = 
% \left\{
% \begin{bmatrix}
% 1 & 0 & 0 \\
% 0 & 1 & 0 \\
% * & * & 1
% \end{bmatrix}
% \right\}
% ;$$
% if $t_j \to -\infty$, then $U$ and~$U^-$ are interchanged.
% Hence, in either case, $\Lie U$ is the sum of two root spaces of~$\Lie G$, and
%$\Lie U^-$ is the sum of the two opposite root spaces. It is not difficult to
%see that $[\Lie U, \Lie U^-]$ is the sum of~$\Lie A$ and the remaining two
%root spaces. Therefore, we have $\langle \Lie U, \Lie U^- \rangle = \Lie G$,
%so $\langle U, U^- \rangle = G$.
% \end{eg}

%\begin{lem} \label{horosubseq}
% If $G$ and~$\{g_j\}$ are as in \cref{matcoeffs->0}, and $\{g_j\}
%\subseteq A$, then, after replacing $\{g_j\}$ by a subsequence, we have $\langle
%U,U^-\rangle = G$, where $U$ and~$U^-$ are defined in \pref{horodefn}.
% \end{lem}

%\begin{proof}
% By passing to a subsequence, we may assume $\{g_j\}$ is contained in a single
%Weyl chamber, which we may take to be $A^+$. Then, by passing to a
%subsequence yet again, we may assume, for every positive real root~$\alpha$,
%that either $\alpha(g_j) \to \infty$ or $\alpha(g_j)$ is bounded. Let
% \begin{itemize}
% \item $\Phi^+$ be the set of positive real roots;
% \item $\Delta$ be the set of positive simple real roots;
% \item $ \Psi = \{\, \alpha \in \Phi^+ \mid 
% \mbox{$\alpha(g_j)$ is bounded} \,\}$;
% and
% \item $T = \cap_{\psi\in \Psi} \ker \psi = \cap_{\psi\in \Psi \cap \Delta}
%\ker \psi$. 
% \end{itemize}
% There is a compact subset~$C$ of~$A$, such that $\{g_j\} \subseteq C T$, so,
%because $\lVert g_j \rVert \to \infty$, we know that $T$ is not trivial.

%For each real root~$\alpha$, let $\rsp_\alpha$ be the corresponding root
%subspace of~$\Lie G$. Then
% $$ \Lie U = \bigoplus_{\alpha \in \Phi^+ \smallsetminus \Psi} \rsp_\alpha
% \mbox{ \qquad and \qquad}
% \Lie U^- = \bigoplus_{\alpha \in \Phi^+ \smallsetminus \Psi} \rsp_{-\alpha}
% .$$
% Now, for $\alpha \in \Phi^+$, we have $\alpha \in \Psi$ if and only if
%$\alpha$ is in the linear span of $\Psi \cap \Delta$. Therefore, we see that $\Lie
%U$ is precisely the unipotent radical of the standard parabolic subalgebra
%$\Lie P = \Lie C_{\Lie G}(T) + \Lie A + \Lie N$ corresponding to the set
%$\Psi \cap \Delta$ of simple roots \cite[4.2,
%pp.~85--86]{BorelTits-Reductive}. 

%Similarly, $\Lie U^-$ is the unipotent radical of the opposite parabolic
%algebra $\Lie P^- = \Lie C_{\Lie G}(T) + \Lie A + \Lie N^-$. Because $G$ is
%simple, the unipotent radicals of opposite parabolics generate~$\Lie G$
%\cite[Prop.~4.11, p.~89]{BorelTits-Reductive}, so $\langle U, U^- \rangle =
%G$, as desired.
% \end{proof}

The above argument yields the following more general result.

\begin{cor}[(of proof)] \label{DecayMatCoeffNoCpct}
 Assume
 \begin{itemize}
 \item $G$ has no compact factors,
 \item $\pi$ is a unitary representation of~$G$
on a Hilbert space~$\Hilbert$, and
 \item $\{g_n\} \to \infty$ in $G/N$, for every proper,
normal subgroup~$N$ of~$G$.
 \end{itemize}
 Then $\langle g_n \phi \mid \psi \rangle \to 0$, for
every $\phi,\psi \in (\Hilbert^G)^\perp$.
 \end{cor}

This has the following consequence, which implies the {Moore Ergodicity Theorem} \csee{MautnerPhenom->MooreErgodicityEx}.

\begin{cor}[({\thmindex{Mautner phenomenon}Mautner phenomenon \csee{MautnerPhenomPfEx}})] \label{MautnerPhenom}
 Assume
 \begin{itemize}
 \item $G$ has no compact factors,
 \item $\pi$ is a unitary representation of~$G$ on a Hilbert space~$\Hilbert$, and
 \item $H$ is a closed subgroup of~$G$.
 \end{itemize}
 Then there is a closed, normal subgroup~$N$ of~$G$, containing a cocompact
subgroup of~$H$, such that every $\pi(H)$-invariant vector in~$\Hilbert$ is
$\pi(N)$-invariant.
 \end{cor}
 
 \begin{rem} \label{QuantitativeDecay}
 \Cref{DecayMatCoeffSimple} does not give any information about the \emph{rate} at which the function $\langle \phi(g) \phi \mid \psi \rangle$ tends to~$0$ as $\|g\| \to \infty$. For some applications, it is helpful to know that, for many choices of the vectors $\phi$ and~$\psi$, the inner product decays exponentially fast:

 	If $G$, $\pi$, and~$\Hilbert$ are as in \cref{DecayMatCoeffSimple}, then there is a dense, linear subspace~$\Hilbert_\infty$ of~$\Hilbert$, such that, for all $\phi , \psi \in \Hilbert_\infty$, there exist $a,b > 1$, such that 
		$$ \text{$\displaystyle |\langle \phi(g) \phi \mid \psi \rangle| < \frac{b}{a^{\|g\|}}$ \quad for all $g \in G$} . $$
Specifically, if $K$ is a maximal compact subgroup of~$G$, then we may take
	$$\Hilbert_\infty = \bigset{ \phi \in \Hilbert }{  \text{the linear span of $K \phi$ is finite-dimensional}} \! .$$
(So $\Hilbert_\infty$ is the space of ``\defit[K-finite vector@$K$-finite vector]{\mathversion{bold}$K$-finite}'' vectors.)
 \end{rem}
 
 
 \begin{exercises}
 
 \item \label{MooreErgodicitySimplePf}
Show that \cref{MooreErgBasicThmReprise} is a corollary of \cref{DecayMatCoeffSimple}.
 \hint{If $\phi$ is an $H$-invariant function in $\LL{p}(G/\Gamma)$, let $\phi' = |\phi|^{p/2} \in \LL2(G/\Gamma)$. Then $\langle \phi', g_j \phi' \rangle = \langle \phi', \phi' \rangle$ for every $g_j \in H$.}
  
\item \label{SubseqConvWeakly}
 Let $\{T_j\}$ be a sequence of unitary operators on a Hilbert space~$\Hilbert$.
 Show there is a subsequence $\{T_{j_i}\}$ of $\{T_j\}$ and a bounded operator~$E$ on~$\Hilbert$, such that $\langle T_{j_i} v \mid w \rangle \stackrel{i \to \infty}{\longrightarrow} \langle E v \mid w \rangle$ for all $v,w \in \Hilbert$.
\hint{Choose an orthonormal basis $\{e_p\}$. For each $p,q$, the sequence $\{\langle T_j e_p \mid e_q \rangle$ is bounded, and therefore has a subsequence that converges to some~$\alpha_{p,q}$. Cantor diagonalization implies that we may assume, after passing to a subsequence, that $\langle T_j e_p \mid e_q \rangle \to \alpha_{p,q}$ for all $p$ and~$q$.}

\item \label{MatCoeffConv}
Suppose 
	\begin{itemize}
	\item $\pi$ is a unitary representation of~$G$ on~$\Hilbert$,
	\item $\{\phi_j\}$ is a sequence of unit vectors in~$\Hilbert$, 
	and 
	\item $u_j \to e$ in~$G$.
	\end{itemize}
Show $\lim \langle \pi(u_j) \phi_j \mid \psi \rangle = \lim \langle \phi_j \mid \psi \rangle$, for all $\psi \in \Hilbert$.
\hint{Move $\pi(u_j)$ to the other side of the inner product. Then use the continuity of~$\pi$ and the boundedness of $\{\phi_j\}$.}

\item \label{UperpInKerEx}
Prove \pref{UperpInKer}.
\hint{Let $\mathcal{K}$ be the closure of $\{\, \pi(u) \phi - \phi \mid u \in U^-, \phi \in (\Hilbert^{U^-})^\perp \,\}$, and note that $\mathcal{K} \subseteq \ker E$. If $\psi \in \mathcal{K}^\perp$, then $\pi(u) \psi - \psi = 0$ for all $u \in U^-$ (why?), so $\psi \in  \Hilbert^{U^-}$.}

 \item \label{DecayMatCoeffSimpleNotAEx}
 Eliminate the assumption that $\{g_j\} \subseteq A$ from the proof of \cref{DecayMatCoeffSimple}.
 \hint{You may assume the \index{Cartan!decomposition}{Cartan decomposition}, which states that $G = KAK$, where $K$ is compact. Hence, $g_j = c_j a_j c'_j$,
with $c_j,c'_j \in K$ and $a_j \in A$. Assume,
by passing to a subsequence, that $\{c_j\}$ and $\{c'_j\}$ converge. Then
	$$ \lim \langle \pi(g_j) \phi \mid \psi \rangle
	= \lim \bigl\langle \pi(a_j) \bigl( \pi(c') \phi \bigr) 
	 \mathrel{\big|} \pi(c)^{-1}
	\psi \bigr\rangle= 0$$
if $c_j \to c$ and $c'_j \to c'$.}

\item \label{DecayMatCoeffSL3}
Prove \cref{DecayMatCoeffSimple} for the special case where $G = \SL(n,\real)$.

\item \label{horosubseq}
For $G$, $A$, $\{g_j\}$, $U$, and~$U^-$ as in the proof of \cref{DecayMatCoeffSimple} (with $\{g_j\} \subseteq A)$, show that if $\{g_j\}$ is replaced by an appropriate subsequence, then $\langle U , U^- \rangle = G$.
\hint{Arrange that $U$ is $\begin{Smallbmatrix} 1&\upast \\ &1 \end{Smallbmatrix}$ and $U^-$ is $\begin{Smallbmatrix} 1&\\ \upast&1 \end{Smallbmatrix}$, or vice versa.}

\item \label{MautnerPhenomPfEx}
Derive \cref{MautnerPhenom} from \cref{DecayMatCoeffNoCpct}.

\item \label{MautnerPhenom->MooreErgodicityEx}
Derive \cref{MooreErgodicity} from \cref{MautnerPhenom}.
\hint{If $f$~is $H$-invariant, then $\langle \pi(h)f \mid f \rangle = \langle f \mid f \rangle$ for all $h \in H$.}

\item \label{MooreErgNonsimple}
Suppose
	\begin{itemize}
	\item $G$ is connected, with no compact factors,
	\item $\Lambda$ is a discrete subgroup of~$G$,
	\item $H$ is a subgroup of $G$ whose projection to every simple factor of~$G$ is not precompact, 
	and
	\item $\phi$ is an $H$-invariant $\LL{p}$ function on $G/\Lambda$, with $1 \le p < \infty$.
	\end{itemize}
Show that $\phi$ is constant (a.e.).

\item \label{MooreErgLatticeEx}
Suppose
	\begin{itemize}
	\item $H$ is a noncompact, closed subgroup of $G$,
	\item $\Gamma$ is irreducible,
	and
	\item $\phi \colon G/\Gamma \to \real$ is any $H$-invariant, measurable function.
	\end{itemize}
Show that $\phi$ is constant (a.e.).
\hint{There is no harm in assuming that $\phi$ is bounded (why?), so it is in $\LL2(G/\Gamma)$ (why?). Apply \cref{MautnerPhenom}.}

 \end{exercises}




\section{Induced representations} \label{InducedRepSect}

It is obvious that if $H$ is a subgroup of~$G$, then any unitary representation of~$G$ can be restricted to a unitary representation of~$H$. (That is, if we define $\pi|_H$ by $\pi|_H(h) = \pi(h)$ for $h \in H$, then $\pi|_H$ is a unitary representation of~$H$.) What is not so obvious is that, conversely, every unitary representation of~$H$ can be ``induced'' to a unitary representation of~$G$. We will need only the special case where $H = \Gamma$ is a lattice in~$G$ (but see \cref{GenInducedRepEx} for the definition in general). This construction will be a key ingredient of the proof in \cref{LattInTSect} that $\Gamma$ often has Kazhdan's property~$(T)$.

\begin{defn}[{(\term[representation!induced]{Induced representation})}] \label{InducedDefn}
Suppose $\pi$ is a unitary representation of~$\Gamma$ on~$\Hilbert$. 
\begin{enumerate}

\item A measurable function $\varphi \colon G \to \Hilbert$ is said to be (essentially) right \defit[equivariant]{$\Gamma$-equivariant} if, for each $\gamma \in \Gamma$, we have 
	$$ \text{$\varphi(g\gamma^{-1}) = \pi(\gamma) \, \varphi(g)$ for a.e.\ $g \in G$.} $$

\item We use 
	\nindex{$\LGamma{}(G; \Hilbert)$ = $\{$~$\Gamma$-equivariant functions from~$G$ to~$\Hilbert$~$\}$}%
	$\LGamma{}(G; \Hilbert)$ 
	to denote the space of right $\Gamma$-equivariant measurable functions from~$G$ to~$\Hilbert$, where two functions are identified if they agree almost everywhere.

\item For $\varphi \in \LGamma{}(G; \Hilbert)$, we have $\| \varphi(g \gamma) \|_\Hilbert = \| \varphi(g) \|_\Hilbert$ for every $\gamma \in \Gamma$ and a.e.\ $g \in G$ \csee{Norm(gGamma)WD}. Hence, $\|\varphi(g)\|_\Hilbert$ is a well-defined function on $G/\Gamma$ (a.e.), so we may define the $\LL2\,$-norm of $\varphi$ by
	$$ \| \varphi \|_2 = \left( \int_{G/\Gamma} \| \varphi(g) \|_\Hilbert^2 \, d g \right) ^{1/2}.$$

\item We use 
	\nindex{$\LGamma2(G; \Hilbert)$ = $\{$~square-integrable functions in $\LGamma{}(G; \Hilbert)$~$\}$}%
	$\LGamma2(G; \Hilbert)$ to denote the subspace of $\LGamma{}(G; \Hilbert)$ consisting of the functions with finite $\LL2\,$-norm. It is a Hilbert space \csee{L2(G;V)Hilbert}.

\item \label{InducedDefn-formula}
Note that $G$ acts by unitary operators on $\LGamma2(G; \Hilbert)$, via
	\begin{align*}
	\text{$(g \cdot \varphi)(x) = \varphi( g^{-1} x)$ \quad for $g \in G$, $\varphi \in \LGamma2(G; \Hilbert)$, and $x \in G$} 
	\end{align*}
 \csee{GActsL2(G;V)}. This unitary representation of~$G$ is called the representation \defit[representation!induced]{induced} from~$\pi$, and it is denoted 
 	\nindex{$\Ind_\Gamma^G(\pi)$ = induced representation}%
 	$\Ind_\Gamma^G(\pi)$.

 \end{enumerate}
\end{defn}

\begin{exercises}

\item \label{GenInducedRepEx}
\optional\ 
Suppose $(\pi,\Hilbert)$ is a unitary representation of a closed subgroup~$H$ of~$G$. 
Define $\Ind_H^G(\pi)$, without assuming that there is a $G$-invariant measure on $G/H$.
\hint{Since $G/H$ is a $C^\infty$ manifold \csee{HomosAreSmooth}, we may use a nowhere-vanishing differential form to choose a measure~$\mu$ on~$G/H$, such that $f_*\mu$ is in the class of~$\mu$, for every diffeomorphism~$f$ of $G/H$. A unitary representation of~$G$ on $\LL2(G/H, \mu)$ can be defined by using Radon-Nikodym derivatives, as in \cref{RegRepNoInvtMeas}, and the same idea yields a unitary representation on a space of $H$-equivariant functions.}

\item \label{Norm(gGamma)WD}
Let $\varphi \in \LGamma{}(G; \Hilbert)$ and $\gamma \in \Gamma$, where $\pi$ is a unitary representation of~$\Gamma$ on~$\Hilbert$. Show $\| \varphi(g \gamma) \|_\Hilbert = \| \varphi(g) \|_\Hilbert$, for a.e.\ $g \in G$.

\item \label{L2(G;V)Hilbert}
Show $\LGamma2(G;\Hilbert)$ is a Hilbert space (with the given norm, and assuming that two functions represent the same element of the space if and only if they are equal a.e.).

\item \label{GActsL2(G;V)}
Show that the formula in \fullcref{InducedDefn}{formula} defines a unitary representation of~$G$ on $\LGamma2(G;\Hilbert)$.

\item \label{Ind1=L2(G/Gamma)}
Show that $\Ind_\Gamma ^G(\trivrep)$ is (isomorphic to) the usual representation of~$G$ on $\LL2(G/\Gamma)$ (by left translation).

\item \label{IndIrred->Irred}
Show that if $\Ind_\Gamma^G(\pi)$ is irreducible, then $\pi$ is irreducible.

\item Show that the converse of \cref{IndIrred->Irred} is false.
\hint{Is the representation of~$G$ on $\LL2(G/\Gamma)$ irreducible?}

\end{exercises}





\section{Representations of compact groups}

\begin{eg} \label{FourierSeriesEg}
Consider the circle $\real/\integer$. For each $n \in \integer$, define
	$$ \text{$e_n \colon \real/\integer \to \complex$ by $e_n(t) = e^{2\pi i n t}$} .$$
The theory of Fourier Series tells us that $\{e_n\}$ is an orthonormal basis of $\LL2(\real/\integer)$, which means we have the direct-sum decomposition
	$$ \LL2(\real/\integer) = \bigoplus_{n \in \integer} \complex e_n . $$
Furthermore, it is easy to verify that each subspace $\complex e_n$ is an invariant subspace for the regular representation \csee{TorusEigenvector}, and, being $1$-dimensional, is obviously irreducible. Hence, we have a decomposition of the regular representation into a direct sum of irreducible representations. In addition, it is not difficult to see that every irreducible representation of~$\torus$ occurs exactly once in this representation.

More generally, it is not difficult to show that every unitary representation of~$\torus$ is a direct sum of $1$-dimensional representations.
\end{eg}

The following theorem generalizes this to any compact group. However, for nonabelian groups, the irreducible representations cannot all be $1$-dimensional \csee{1D->Abel}.

\begin{thm}[(\thmindex{Peter-Weyl}Peter-Weyl Theorem)] \label{PeterWeyl}
Assume $H$ is \textbf{compact}. Then:
	\begin{enumerate}
	\item \label{PeterWeyl-sum}
	Every unitary representation of~$H$ is\/ \textup(isomorphic to\/\textup) a direct sum of irreducible representations.
	\item \label{PeterWeyl-fd}
	Every irreducible representation of~$H$ is finite-dimensional. 
	\item \label{PeterWeyl-countable}
	$\widehat H$ is countable.
	\item \label{PeterWeyl-regrep}
	For the particular case of the regular representation $\bigl(\regrep, \LL2(H) \bigr)$, we have
		$$\regrep \iso \bigoplus_{(\pi,\Hilbert) \in \widehat H} (\dim \Hilbert) \cdot \pi , $$
	where $k \cdot \pi$ denotes the direct sum $\pi \oplus \cdots \oplus \pi$ of $k$~copies of\/~$\pi$. That is, the ``multiplicity'' of each irreducible representation is equal to its dimension.
	\end{enumerate}
\end{thm}

\begin{proof}
In order to establish both \pref{PeterWeyl-sum} and~\pref{PeterWeyl-fd} simultaneously, it suffices to show that if $(\pi,\Hilbert)$ is any unitary representation of~$H$, then $\Hilbert$ is a direct sum of finite-dimensional, invariant subspaces.
Zorn's Lemma \pref{ZornsLemma} provides a subspace~$\mathcal{M}$ of~$\Hilbert$ that is maximal among those that are a direct sum of finite-dimensional, invariant subspaces. By passing to the orthogonal complement of~$\mathcal{M}$, we may assume that $\Hilbert$ has no nonzero, finite-dimensional, invariant subspaces.

Let 
\noprelistbreak
	\begin{itemize}
	\item $P$ be the orthogonal projection onto some nonzero subspace of~$\Hilbert$ that is finite-dimensional, 
	\item $\mu$ be the Haar measure on~$H$,
	and
	\item $ \overline{P} = \int_H \pi(h) \, P \, \pi(h^{-1}) \, d\mu(h)$.
	\end{itemize}
Note that, since $\overline{P}$ commutes with $\pi(H)$ \csee{AvgOpCommutes}, every eigenspace of~$\overline{P}$ is $H$-invariant \csee{Commutes->EigInvt}. 

Since $P$ is self-adjoint and each $\pi(h)$ is unitary (so $\pi(h^{-1}) = \pi(h)^*$), it is not difficult to see that $\overline{P}$ is self-adjoint. It is also compact \csee{PbarCpct} and nonzero \csee{PbarNonzero}. Therefore, the Spectral Theorem \pref{SpectralThmCpct} implies that $\overline{P}$ has at least one eigenspace~$E$ that is finite-dimensional. By contradicting the fact that $\Hilbert$ has no nonzero, finite-dimensional, invariant subspaces, this completes the proof of \pref{PeterWeyl-sum} and~\pref{PeterWeyl-fd}.

\medbreak

Note that \pref{PeterWeyl-countable} is an immediate consequence of \pref{PeterWeyl-regrep}, since Hilbert spaces are assumed to be separable \csee{HilbertSpaceSeparable}, and therefore cannot have uncountably many terms in a direct sum.

\medbreak

We now give the main idea in the proof of \pref{PeterWeyl-regrep}.
Given an irreducible representation $(\pi, \complex^k)$, we will not calculate the exact multiplicity of~$\pi$, but only indicate how to obtain the correct lower bound by using properties of matrix coefficients. 
Write $\pi(x) = \bigl[ f_{i,j}(x) \bigr]$. Then
	\begin{align}  \label{PeterWeylMatCoeff}
	\begin{split}
	\bigl[ \bigl( \regrep(h)f_{i,j} \bigr)(x) \bigr]
	&= \bigl[ f_{i,j}(h^{-1} x) \bigr]
	= \pi(h^{-1} x)
	\\&= \pi(h^{-1}) \, \pi(x)
	= \pi(h^{-1}) \bigl[ f_{i,j}(x) \bigr] 
	. \end{split} \end{align}
Now, for $1 \le j \le k$, define $ T_j \colon \complex^k \to \LL2(H)$, by
	$$ T_j(a_1,\ldots,a_k) = a_1  f_{1,j} + a_2 f_{2,j} + \cdots + a_k f_{k,j} .$$
Equating the $j$th columns of the two ends of \pref{PeterWeylMatCoeff} tells us that
	$$ T_j \bigl( \pi(h) v \bigr) = \regrep(h) \, T_j(v) ,$$
so $T_j(\complex^k)$ is an invariant subspace, and the corresponding subrepresentation is isomorphic to~$\pi$. Therefore, there are (at least) $k$~different copies of~$\pi$ in~$\regrep$ (one for each value of~$j$). Since, by definition, $k = \dim \pi$, this establishes the correct lower bound for the multiplicity of~$\pi$.
\end{proof}



As an illustrative, simple case of the main results in \cref{RepRnSect,DirectIntegralSect}, we present two different reformulations of the Peter-Weyl Theorem for the special case of abelian groups, after some preliminaries.

\begin{defn}
Let $A$ be an abelian Lie group.
	\begin{enumerate}
	\item A \defit[character (of an abelian group)]{character} of~$A$ is a continuous homomorphism $\chi \colon A \to \torus$, where $\torus = \{\, z \in \complex \mid |z| = 1 \,\}$.
	\item The set of all characters of~$A$ is called the \defit{Pontryagin dual} of~$A$, and is denoted~%
	\nindex{$A^*$ = $\{$characters of abelian group~$A$$\}$}%
	$A^*$.
	 It is an abelian group under the operation of pointwise multiplication. (That is, the product $\chi_1 \chi_2$ is defined by $(\chi_1 \chi_2)(a) = \chi_1(a) \, \chi_2(a)$, for $\chi_1, \chi_2 \in A^*$ and $a \in A$.) Furthermore, if $A/A^\circ$ is finitely generated, then $A^*$ is a Lie group (with the topology of uniform convergence on compact sets).
	\end{enumerate}
\end{defn}

\begin{obs}
If $A$ is any abelian Lie group\/ \textup(compact or not\/\textup), then every irreducible representation $(\pi, \Hilbert)$ of~$A$ is $1$-dimensional \csee{AbelIrred1D}. Therefore, the unitary dual~$\widehat A$ can be identified with the Pontryagin dual~$A^*$ \csee{1Drep<>Char}.
\end{obs}

Hence, for the special case where $H = A$ is abelian, we have the following reformulation of the Peter-Weyl Theorem:

\begin{cor}[\csee{PeterWeylAbelEx}] \label{PeterWeylAbel}
Assume $(\pi,\Hilbert)$ is a unitary representation of a compact, abelian Lie group~$A$.  For each $\chi \in A^*$, let
	\begin{itemize}
	\item $ \Hilbert_\chi = \{\, \varphi \in \Hilbert \mid 
		\text{$\phi(a) \varphi = \chi(a) \, \varphi$, for all $a \in A$} \,\} $,
	and
	\item $P_\chi \colon \Hilbert \to \Hilbert_\chi$ be the orthogonal projection.
	\end{itemize}
Then $\Hilbert = \bigoplus_{\chi \in A^*} \Hilbert_\chi$, so, for all $a \in A$, we have
	$$ \pi(a) = \sum_{\chi \in A^*} \chi(a) \, P_\chi. $$
\end{cor}

Here is another way of saying the same thing:

\begin{cor}[\csee{PeterWeylAbelL2Ex}] \label{PeterWeylAbelL2}
Assume $(\pi,\Hilbert)$ is a unitary representation of a compact, abelian Lie group~$A$.  Then there exist
	\begin{itemize}
	\item a Radon measure $\mu$ on a locally compact metric space~$Y$,
	and
	\item a Borel measurable function $\chi \colon Y \to A^* \colon y \mapsto \chi_y$ \textup(where the countable set~$A^*$ is given the discrete topology\/\textup)
	\end{itemize}
such that $\pi$ is isomorphic to the the unitary representation $\rho_\chi$ of~$A$ on $\LL2(Y, \mu)$ that is defined by
	$$ \bigl( \rho_\chi(a) \varphi  \bigr)(y) = \chi_y(a) \, \varphi(y)
	 \quad \text{for $a \in A$, $\varphi \in \LL2(Y, \mu )$, and $y \in Y$} .$$
%where, for convenience, we write $\chi_y$ for $\chi(y) \in A^*$.
\end{cor}

An analogue of this result for semisimple groups will be stated in \cref{DirectIntegralSect}, after we define the ``direct integral'' of a family of representations.

\begin{exercises}

\item \label{TorusEigenvector} 
In the notation of \cref{FourierSeriesEg}, show $\regrep(h) e_n = e^{-2\pi i h} e_n$, for all $h \in \real/\integer$.

\item \label{MatCoeffOfCpct}
Suppose $(\pi,\Hilbert)$ is a unitary representation of a compact group~$H$, let $\varphi,\psi \in \Hilbert$, and define $f \colon H \to \complex$ by $f(h) = \langle \pi(h) \varphi \mid \psi \rangle$. Show $f \in \LL2(H)$.
\hint{It is a bounded function on a compact set.}

\item \label{AvgOpCommutes}
Suppose $(\pi,\Hilbert)$ is a unitary representation of a compact group~$H$. Show that if $T$ is any bounded operator on~$\Hilbert$, then 
	$$\overline{T} = \int_H \pi(h) \, T \, \pi(h^{-1}) \, d\mu(h)$$
is an operator that commutes with every element of $\pi(H)$.
\hint{The invariance of Haar measure implies $\pi(g) \overline{T} \pi(g^{-1}) = \overline{T}$.}

\item \label{PbarCpct}
Show that the operator $\overline{P}$ in the proof of \cref{PeterWeyl} is compact.
\hint{Apply \cref{CpctOpBasics}, by noting that any integral can be approximated by a finite sum, and the finite sum is an operator whose range is finite-dimensional.}

\item \label{PbarNonzero}
Show that the operator $\overline{P}$ in the proof of \cref{PeterWeyl} is nonzero.
\hint{Choose some nonzero $\varphi \in \Hilbert$, such that $P \varphi = \varphi$.
Then $\langle \overline{P} \varphi \mid \varphi \rangle > 0$, since $\langle P \psi \mid \psi \rangle \ge 0$ for all $\psi \in \Hilbert$.}

\item \label{Commutes->EigInvt}
Suppose $(\pi,\Hilbert)$ is a unitary representation of a Lie group~$H$, $T$~is a bounded operator on~$\Hilbert$, $\lambda \in \complex$, and $\varphi \in \Hilbert$. Show that if $T$ commutes with every element of $\pi(H)$, and $T(\varphi) = \lambda \varphi$, then $T \bigl( \pi(h) \varphi \bigr) = \lambda \, \pi(h) \varphi$, for every $h \in H$.

\item Assume $H$ is compact. Show that $H$ is finite if and only if it has only finitely many different irreducible unitary representations (up to isomorphism).
\hint{You may assume \cref{PeterWeyl}.}

\item \label{AbelIrred1D}
Show that every irreducible representation of any abelian Lie group is $1$-dimensional.
\hint{If $\pi(a)$ is not a scalar, for some $a$, then the Spectral Theorem \pref{SpectralThm} yields an invariant subspace.}

\item \label{1Drep<>Char}
Let $H$ be a Lie group. Show there is a natural bijection between the set of $1$-dimensional unitary representations (modulo isomorphism) and the set of continuous homomorphisms from~$H$ to~$\torus$.
\hint{Any $1$-dimensional unitary representation is isomorphic to a representation on~$\complex$.}

\item \label{PeterWeylAbelEx}
Derive \cref{PeterWeylAbel} from \cref{PeterWeyl}.

\item \label{PeterWeylAbelL2Ex}
Prove \cref{PeterWeylAbelL2}.
\hint{If $\Hilbert_\chi \neq \{0\}$, then $\Hilbert_\chi$ is isomorphic to some $\LL2(Y_\chi, \mu_\chi)$. Let $Y = \bigcup_\chi Y_\chi$.}

\end{exercises}







\section{Unitary representations of \texorpdfstring{$\real^{\lowercase{n}}$}{Rn}} \label{RepRnSect}

Any character~$\chi$ of $\real^n$ is of the form 
	$$ \chi(a) = e^{2 \pi i \, (a \cdot y)} \qquad
	\text{for some (unique) $y \in \real^n$} $$
\csee{RnChars}.  Therefore, the Pontryagin dual~$(\real^n)^*$ (or, equivalently, the unitary dual~$\widehat{\real^n}$) can be identified with~$\real^n$ (by matching $\chi$ with the corresponding vector~$y$). In particular, unlike in \cref{PeterWeyl}, the unitary dual is uncountable.

Unfortunately, however, not every representation of~$\real^n$ is a direct sum of irreducibles.
For example, the regular representation~$\regrep$ of $\real^n$ on $\LL2(\real^n)$ has no $1$-dimensional, invariant subspaces \csee{L2RnNo1D}, so it does not even contain a single irreducible representation and is therefore not a sum of them. Indeed, \term{Fourier Analysis} tells us that a function in $\LL2(\real^n)$ is not a \emph{sum} of exponentials, but an \emph{integral}:
	$$ \varphi(a) = \int_{\real^n} \widehat\varphi(y) \, e^{2\pi i \, (a \cdot y)} \, dy ,$$
where $\widehat\varphi$ is the Fourier transform of~$\varphi$.
Now, for each Borel subset~$E$ of~$\real^n$, let 
	\begin{align} \label{HEFourier}
	\Hilbert_E = \{\, f \in \LL2(\real^n) \mid \text{$\widehat f(y) = 0$ for a.e.\ $y \notin E$} \,\} 
	. \end{align}
Then it is not difficult to show that $\Hilbert_E$ is a closed, $\regrep$-invariant subspace \csee{HEinvt}.

Now, let $P(E)\colon \Hilbert \to \Hilbert_E$ be the orthogonal projection. Then we can think of $P$ as a projection-valued measure on $\real^n$ (or on $(\real^n)^*$), and, for all $a \in \real^n$,  we have
	$$\regrep(a) = \int_{\real^n} e^{i (a \cdot y)} \, dP(y)
		= \int_{(\real^n)^*} \chi(a) \, dE(\chi) .$$
If we let $\pi = \regrep$, this is a perfect analogue of the conclusion of \cref{PeterWeylAbel}, but with the sum replaced by an integral.

A version of the \thmindex{Spectral}{Spectral Theorem} tells us that this generalizes in a natural way to all unitary representations of~$\real^n$, or, in fact, of any abelian Lie group:

\begin{prop} \label{ProjValMeas}
Suppose $\pi$ is a unitary representation of an abelian Lie group~$A$ on~$\Hilbert$. Then there is a \textup(unique\/\textup) projection-valued measure~$P$ on\/~$A^*$, such that
	$$ \text{$\pi(a) = \int_{A^*} \chi(a) \, dP(\chi)$ \ for all $a \in A$} . $$
\end{prop}

This can be reformulated as a generalization of \cref{PeterWeylAbelL2}:

\begin{cor} \label{IntIrredAbel}
Let $(\pi,\Hilbert)$ be a unitary representation of an abelian Lie group~$A$.  Then there exist
	\begin{itemize}
	\item a probability measure $\mu$ on a locally compact metric space~$Y$,
	and
	\item a Borel measurable function $\chi \colon Y \to A^* \colon y \mapsto \chi_y$,
	\end{itemize}
such that $\pi$ is isomorphic to the unitary representation $\rho_\chi$ of~$A$ on $\LL2(Y, \mu)$ that is defined by
	$$ \bigl( \rho_\chi(a) \varphi  \bigr)(y) = \chi_y(a) \, \varphi(y)
	 \quad \text{for $a \in A$, $\varphi \in \LL2(Y, \mu )$, and $y \in Y$} .$$
\end{cor}

\begin{proof}
Let $P$ be the projection-valued measure given by \cref{ProjValMeas}.
A closed subspace~$\Hilbert'$ of~$\Hilbert$ is said to be \defit[cyclic, for a projection-valued measure]{cyclic} for~$P$ if there exists $\psi \in \Hilbert'$, such that the span of $\{\, P(E)\,\psi \mid E \subset A^* \,\}$ is a dense subspace of~$\Hilbert'$. It is not difficult to see that $\Hilbert$ is an orthogonal direct sum of countably many cyclic subspaces \csee{IntIrredCyclicEx}. Therefore, we may assume $\Hilbert$ is cyclic \csee{AssumeHCyclicEx} (and nonzero). So we may fix some unit vector $\psi$ that generates a dense subspace of~$\Hilbert$.

Define a probability measure $\mu$ on $A^*$ by 
	$$ \mu(E) = \langle P(E)\psi \mid \psi \rangle = \langle P(E)\psi \mid P(E)\psi \rangle ,$$
and let $\Id$ be the identity map on~$A^*$.

For the characteristic function~$f_E$ of each Borel subset~$E$ of~$A^*$, define $\Phi(f_E) = P(E) \psi$. Then $\langle \Phi(f_{E_1}) \mid  \Phi(f_{E_2}) \rangle =  \langle f_{E_1} \mid  f_{E_2} \rangle$, by the definition of~$\mu$, so $\Phi$ extends to a norm-preserving linear map~$\Phi'$ from $\LL2(A^*,\mu)$ to~$\Hilbert$. Since $\psi$ is a cyclic vector for~$\Hilbert$, we see that $\Phi'$ is surjective, so it is an isomorphism of Hilbert spaces. Indeed, $\Phi'$ is an isomorphism from $\rho_{\Id}$ to~$\pi$ \csee{IsoRhoPi}.
\end{proof}

\begin{exercises}

\item \label{RnChars}
Show that every character~$\chi$ of $\real^n$ is of the form 
	$ \chi(a) = e^{2 \pi i \, (a \cdot t)}$ for some $t \in \real^n$.
\hint{Since $\real^n$ is simply connected, any continuous homomorphism from~$\real^n$ to~$\torus$ can be lifted to a homomorphism into the universal cover, which is~$\real$. Apply \cref{RnHomIsLinear}.}

\item \label{L2RnNo1D}
Let $H$ be a noncompact Lie group. Show that the regular representation of~$H$ has no $1$-dimensional, invariant subspaces.
\hint{If $\varphi$ is in a $1$-dimensional, invariant subspace of $\LL2(H)$, then $|\varphi|$~is constant (a.e.).}

\item \label{HEinvt}
For every measurable subset~$E$ of~$\real^n$, show that the subspace $\Hilbert_E$ defined in \pref{HEFourier} is closed, and is invariant under $\regrep(\real^n)$.
\hint{It is clear that $\{\, \widehat f \mid f \in \Hilbert_E \,\}$ is closed. For invariance, note that the Fourier transform of $\regrep(a)f$ is $e^{-2\pi i (a \cdot y)} \widehat f(y)$.}

\item \label{IntIrredCyclicEx}
Given a projection-valued measure~$P$ on a Hilbert space~$\Hilbert$, show that $\Hilbert$ is the orthogonal direct sum of countably many cyclic subspaces.
\hint{Every vector in~$\Hilbert$ is contained in a cyclic subspace, the orthogonal complement of a $P(E)$-invariant subspace is $P(E)$-invariant, and all Hilbert spaces are assumed to be separable.} 

\item \label{AssumeHCyclicEx}
In the notation of \cref{IntIrredAbel}, suppose $\rho_{\chi_i}$ is the representation on $\LL2(Y_i, \mu_i)$ corresponding to some $\chi_i \colon Y_i \to A^*$. Show $\bigoplus_{i=1}^\infty \rho_{\chi_i} \iso \rho_\chi$, for some $Y$, $\mu$, and~$\chi$.
\hint{Let $(Y,\mu)$ be the disjoint union of (copies of) $(Y_i,\mu_i)$.}

\item \label{IsoRhoPi}
In the notation of the proof of \cref{IntIrredAbel}, show that $\Phi'$ is an isomorphism from $\rho_{\Id}$ to~$\pi$.
\hint{Given $a \in A$ and $E \subset A^*$, write $E$ as the disjoint union of small sets $E_1,\ldots,E_n$ (so $\chi \mapsto \chi(a)$ is almost constant on each~$E_i$). 
Then
	$$ \textstyle \Phi' \bigl( \rho_{\Id}(a) f_E \bigr)
	\approx \Phi' \bigl( \sum_i \chi_i(a) \, f_{E_i} \bigr)
	= \sum_i \chi_i(a) \, P(E_i) \, \psi
	\approx \pi \bigl( P(E) \psi \bigr)
	= \pi \bigl( \Phi'(f_E) \bigr)
	, $$
for any $\chi_i \in E_i$.}

\end{exercises}






\section{Direct integrals of representations} \label{DirectIntegralSect}

Before we define the direct integral of a collection of unitary representations, we first discuss the simpler case of a direct sum of a sequence $\{(\pi_n, \Hilbert_n)\}_{n= 1}^\infty$ of unitary representations.

\begin{defn} \label{HilbertDirSumInfty}
If $\{\Hilbert_n \}_{n=1}^\infty$ is a sequence of Hilbert spaces, then the \defit[direct sum!of Hilbert spaces]{direct sum} $\bigoplus_{n= 1}^\infty \Hilbert_n$ consists of all sequences $\{\varphi_n\}_{n= 1}^\infty$, such that 
	\begin{itemize}
	\item $\varphi_n \in \Hilbert_n$ for each~$n$, 
	and
	\item $\sum_{n= 1}^\infty \| \varphi_n\|^2 < \infty$.
	\end{itemize}
This is a Hilbert space, under the inner product 
	$$ \bigl\langle \{\varphi_n\} \mid \{\psi_n\} \bigr\rangle = \sum_{n= 1}^\infty \langle \varphi_n \mid \psi_n \rangle .$$
It contains a copy of $\Hilbert_n$, for each~$n$, such that $\Hilbert_m \perp \Hilbert_n$, for $m \neq n$.
\end{defn}

Suppose, now, that all of the Hilbert spaces in the sequence are the same; say, $\Hilbert_n = \Hilbert$, for all~$n$. Then $\bigoplus_{n=1}^\infty \Hilbert_n$ is equal to the set of square-integrable functions from $\integer^+$ to $\Hilbert$, which can be denoted $\LL2(\integer^+ ; \Hilbert)$. In this notation, the direct sum of unitary representations is easy to describe:

\begin{defn}
If $\{\, \pi_n\}_{n= 1}^\infty$ is a sequence of unitary representations of~$H$ on a fixed Hilbert space~$\Hilbert$, then $\bigoplus_{n= 1}^\infty \pi_n$ is the unitary representation $\pi$ on $\LL2(\integer^+ ; \Hilbert)$ that is defined by
	$$ \bigl( \pi(h) \varphi  \bigr)(n) = \pi_n(h) \, \varphi(n)
	 \quad \text{for $h \in H$, $\varphi \in \LL2(\integer^+ ; \Hilbert)$, and $n \in \integer^+$} .$$
\end{defn}

This description of the direct sum naturally generalizes to a definition of the direct integral of representations:

\begin{defn} \label{DirectIntegDefn}
Suppose 
	\begin{itemize}
	\item $\Hilbert$ is a Hilbert space,
	\item $\{\pi_x\}_{x \in X}$ is a measurable family of unitary representations of~$H$ on~$\Hilbert$, which means:
		\begin{itemize}
		\item $X$~is a locally compact metric space,
		\item $\pi_x$ is a unitary representation of~$H$ on~$\Hilbert$, for each $x \in X$,
		and
		\item for each fixed $\varphi,\pi \in \Hilbert$, the expression $\langle \pi_x(h) \varphi \mid \psi \rangle$ is a Borel measurable function on $X \times H$,
		\end{itemize}
	and
	\item $\mu$ is a Radon measure on~$X$.
	\end{itemize}
Then $\int_X \pi_x \, d\mu(x)$ is the unitary representation~$\pi$ of~$H$ on $\LL2(X, \mu ; \Hilbert)$ that is defined by
	$$ \bigl( \pi(h) \varphi  \bigr)(x) = \pi_x(h) \, \varphi(x)
	 \quad \text{for $h \in H$, $\varphi \in \LL2(X, \mu ; \Hilbert)$, and $x \in X$} .$$
This is called the \defit{direct integral} of the family of representations $\{\pi_x\}$.
\end{defn}

The above definition is limited by requiring all of the representations to be on the same Hilbert space. The construction can be generalized to eliminate this assumption \csee{GeneralDirectIntegDefn}, but there is often no need: 

\begin{thm} \label{UnitaryGDirectInt}
Assume 	
	\begin{itemize}
	\item $\pi$ is a unitary representation of~$G$, 
	\item $G$ is connected, and has no compact factors,
	and
	\item no nonzero vector is fixed by every element of $\pi(G)$.
	\end{itemize}
Then there exist $\Hilbert$, $\{\pi_x\}_{x \in X}$, and~$\mu$, as in \cref{DirectIntegDefn}, such that 
	\begin{enumerate}
	\item $\pi \iso \int_X \pi_x \, d\mu(x)$, 
	and
	\item $\pi_x$ is irreducible for every $x \in X$.
	\end{enumerate}
\end{thm}


\begin{rem} \label{GeneralDirectIntegDefn}
Up to isomorphism, there are only countably many different Hilbert spaces (since any two Hilbert spaces of the same dimension are isomorphic). It is therefore not difficult to generalize \cref{DirectIntegDefn} to deal with a family of representations in which the Hilbert space varies with~$x$. Such a generalization allows every unitary representation of any Lie group to be written as a direct integral of representations that are irreducible.

Here is one way.
Let us say that
$\{(\pi_x,\Hilbert_x\}_{x \in X}$ is a measurable family of unitary representations of~$H$ if:
	\begin{itemize}
	\item $X = \bigcup_{n=1}^\infty$ is the union of countably many locally compact metric spaces~$X_n$,
	\item for each~$n$, there is Hilbert space~$\Hilbert_n$, such that $\Hilbert_x = \Hilbert_n$ for all $x \in X_n$
	\item $\pi_x$ is a unitary representation of~$H$ on~$\Hilbert_x$, for each $x \in X$,
	\item for each~$n$ and each $\varphi,\pi \in \Hilbert_n$, the expression $\langle \pi_x(h) \varphi \mid \psi \rangle$ is a Borel measurable function on $X_n \times H$,
	and
	\item $\mu$ is a Radon measure on~$X$.
	\end{itemize}
Given such a family of representations, we define
	$$ \int_X \pi_x \, d\mu(x) = \bigoplus_{n=1}^\infty \int_{X_n} \pi_x \, d\mu(x) .$$
With this, more general, notion of direct integral, it can be proved that every unitary representation of any Lie group is isomorphic to a direct integral of a measurable family of irreducible unitary representations.
\end{rem}


\begin{exercises}

\item \label{1D->Abel}
Let $H$ be a Lie group. Show that if the regular representation of~$H$ is a direct integral of $1$-dimensional representations, then $H$ is abelian.

\end{exercises}


\begin{notes}

There are many books on the theory of unitary representations, including the classics of Mackey \cite{Mackey-ThyUnitaryGrpReps,Mackey-UnitaryRepsPhysProbNT}. Several books, such as \cite{Knapp-RepThySSGrps}, specifically focus on the representations of semisimple Lie groups.

The Moore Ergodicity Theorem \pref{MooreErgBasicThmReprise} is due to C.\,C.\,Moore \cite{Moore-ergodicity}. 

\Cref{DecayMatCoeffNoCpct} is due to R.\,Howe and C.\,C.\,Moore
\cite[Thm.~5.1]{HoweMoore} and (independently) R.\,J.\,Zimmer
\cite[Thm.~5.2]{Zimmer-orbitspace}. 
%It appears in \cite[Thm.~2.2.20, p.~23]{ZimmerBook}.
The elementary proof we give here was found by
R.\,Ellis and M.\,Nerurkar \cite{EllisNerurkar}.
%(See \cite{Pittet} for a more complete exposition.) -- preprint hasn't been published? @@@
Other proofs are in \cite[\S2.3, pp.~85--92]{MargulisBook} and \cite[\S2.4, pp.~28--31]{ZimmerBook}.

A more precise form of the quantitative estimate in \cref{QuantitativeDecay} can be found in \cite[Cor.~7.2]{Howe-rank}.
(As stated there, the result requires the matrix coefficient $\langle \pi(g)\phi \mid \psi \rangle$ to be an $\LL{p}$~function of~$g$, for $\phi,\psi$ in a dense subspace of~$\Hilbert$, and for some $p < \infty$, but it was proved in \cite[Thm.~2.4.2]{Cowling-coeffs} that this integrability hypothesis always holds.)

\Cref{PeterWeyl} is proved in \cite[Chap.~3]{Sepanski-CpctLieGrps}

See \cite[Chap.~2]{Helson-SpectralThm} for a nice proof of \cref{ProjValMeas}. (Although most of the proof is written for $n = 1$, it is mentioned on p.~31 that the argument works in general.) 

See \cite[Thm.~2.9, p.~108]{Mackey-ThyUnitaryGrpReps} for a proof of \cref{GeneralDirectIntegDefn}'s statement that every unitary representation is a direct integral of irreducibles. (This is a generalization of \cref{UnitaryGDirectInt}.)

Regarding \fullcref{UnitaryWarns}{tame}, groups for which the set of irreducible unitary representations admits an injective Borel map to $[0,1]$ are called ``Type~I'' (and the others are ``Type~II''). See \cite[\S2.3, pp.~77--85]{Mackey-ThyUnitaryGrpReps} for some discussion of this.

\end{notes}


\begin{references}{99}

\bibitem{Cowling-coeffs}
M.\,Cowling:
Sur les coefficients des repr\'esentations unitaires des groupes de Lie simples,
in: Pierre Eymard et al., eds.,
\emph{Analyse Harmonique sur les Groupes de Lie~II}.
%(S\'eminaire Nancy-Strasbourg 1976--1978).
%Lecture Notes in Math., 739, 
Springer, Berlin, 1979, pp.~132--178.
ISBN 3-540-09536-5,
\MR{0560837}

 \bibitem{EllisNerurkar}
 R.\,Ellis and M.\,Nerurkar:
 Enveloping semigroup in ergodic theory and a proof of
Moore's ergodicity theorem,
 in: J.\,C.\,Alexander., ed.,
 \emph{Dynamical Systems}.
 % \textup(College Park, MD, 1986--87\textup)},  
%Lecture Notes in Math. \#1342,  
Springer, New York, 1988, pp.~172--179.
ISBN 3-540-50174-6,
 \MR{0970554}

\bibitem{Helson-SpectralThm}
H.\,Helson:
\emph{The Spectral Theorem}.
%Lecture Notes in Mathematics, 1227. 
Springer, Berlin, 1986.
ISBN 3-540-17197-5,
\MR{0873504}

\bibitem{Howe-rank}
R.\,Howe:
On a notion of rank for unitary representations of the classical groups,
in A.\,Figa-Talamanca, ed.:
\emph{Harmonic Analysis and Group Representations}, 
Liguori, Naples, 1982, 
pp.~223--331.
ISBN 978-3-642-11115-0,
\MR{0777342}

\bibitem{HoweMoore}
R.\,E.\,Howe and C.\,C.\,Moore:
Asymptotic properties of unitary representations,
\emph{J. Funct. Anal.} 32 (1979), no. 1, 72--96. 
 \MR{0533220},
 \maynewline
 \url{http://dx.doi.org/10.1016/0022-1236(79)90078-8}

\bibitem{Knapp-RepThySSGrps}
A.\,W.\,Knapp:
\emph{Representation Theory of Semisimple Groups}. 
%An overview based on examples. Reprint of the 1986 original. 
%Princeton Landmarks in Mathematics. 
Princeton University Press, Princeton, 2001.
ISBN 0-691-09089-0,
\MR{1880691}

\bibitem{Mackey-ThyUnitaryGrpReps}
G.\,Mackey:
\emph{The Theory of Unitary Group Representations}. 
University of Chicago Press, Chicago, 1976.
ISBN 0-226-50052-7,
\MR{0396826}

\bibitem{Mackey-UnitaryRepsPhysProbNT}
G.\,Mackey:
\emph{Unitary Group Representations in Physics, Probability, and Number Theory}.
%Mathematics Lecture Note Series, 55. 
Benjamin, Reading, Mass., 1978. 
ISBN 0-8053-6702-0,
\MR{0515581}

\bibitem{MargulisBook}
 G.\,A.\,Margulis:
 \emph{Discrete Subgroups of Semisimple Lie Groups.}
 Springer, New York, 1991.
ISBN 3-540-12179-X,
\MR{1090825}

\bibitem{Moore-ergodicity}
C.\,C.\,Moore:
Ergodicity of flows on homogeneous spaces,
\emph{Amer. J. Math.} 88 1966 154--178.
\MR{0193188},
\maynewline
\url{http://www.jstor.org/stable/2373052}
 
\bibitem{Sepanski-CpctLieGrps}
M.\,Sepanski:
\emph{Compact Lie Groups}. 
%Graduate Texts in Mathematics, 235. 
Springer, New York, 2007.
ISBN 978-0-387-30263-8,
  \MR{2279709}
 
 \bibitem{Zimmer-orbitspace}
 R.\,J.\,Zimmer:
 Orbit spaces of unitary representations, ergodic theory, and simple Lie groups,
 \emph{Ann. Math.} (2) 106 (1977), no. 3, 573--588.
 \MR{0466406},
 \maynewline
 \url{http://dx.doi.org/10.2307/1971068}

\bibitem{ZimmerBook}
R.\,J.\,Zimmer:
\emph{Ergodic Theory and Semisimple Groups}.
%Monographs in Mathematics, 81. 
Birkh\"auser, Basel, 1984.
ISBN 3-7643-3184-4,
\MR{0776417}

 \end{references}



