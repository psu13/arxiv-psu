%!TEX root = IntroArithGrps.tex

\mychapter{Brief Summary}
\label{SummaryChap}

\makemarksfalse % don't let the sections in this chapter change the running heads

\prereqs{none for most of the non-proof material.}

This book is about arithmetic subgroups, and other lattices, in semisimple Lie groups.
Given a lattice~$\Gamma$ in a semisimple Lie group~$G$, we will investigate both the algebraic structure of~$\Gamma$, and properties of the corresponding homogeneous space $G/\Gamma$. We will also study the close relationship between~$G$ and~$\Gamma$. For example, we will see that~$G$ is essentially the only semisimple group in which $\Gamma$ can be embedded as a lattice (``\thmindex{Mostow Rigidity}{Mostow Rigidity Theorem}''), and, conversely, we will usually be able to make a list of all the lattices in~$G$ (``\thmindex{Margulis!Arithmeticity}{Margulis Arithmeticity Theorem}''). 

%\begin{rem}
%Geometers interested in the locally symmetric space $\Gamma
%\backslash G/K$ usually place additional
%restrictions on $G$ and~$\Gamma$:
% \begin{itemize}
% \item Geometers assume that the center of $G$ is trivial,
%for otherwise $G$ does not act \emph{faithfully} as a group
%of isometries of the symmetric space $G/K$.
% \item Geometers assume that $\Gamma$ is torsion-free (that
%is, that $\Gamma$ has no nontrivial elements of finite
%order), for otherwise $\Gamma$ does not act freely on $G/K$.
% \end{itemize}
%\end{rem}


This chapter provides a very compressed outline of the material in this book. To help keep it brief, let us assume, for the remainder of the chapter, that
	$$ \text{\mathversion{bold}\bf$G$ is a noncompact, simple Lie group, and $\Gamma$ is a lattice in~$G$} .$$
This means \csee{LatticeDefn}:
	\begin{itemize}
	\item $\Gamma$ is a discrete subgroup of~$G$, 
	and 
	\item the homogeneous space $G/\Gamma$ has finite volume (with respect to the Haar measure on~$G$).
	\end{itemize} 
(If $G/\Gamma$ is compact, which is a very important special case, we say $\Gamma$ is \defit[cocompact subgroup]{cocompact}.)



\notocsection*{\cref{IntroductionPart}. Introduction}
All three chapters in this part of the book are entirely optional; none of the material will be needed later (although some examples and remarks do refer back to it). \Cref{WhatisLocSymmChap,GeomIntroRank} provide geometric motivation for the study of arithmetic groups, by explaining the connection with locally symmetric spaces. The present chapter (\cref{SummaryChap}) is a highly condensed version of the entire book.




\notocsection*{\cref{FundamentalsPart}. Fundamentals}
This part of the book presents definitions and other foundational material for the study of arithmetic groups.

\subsection*{\cref{BasicLatticesChap}. Basic Properties of Lattices} 
This chapter presents a few important definitions, including the notions of \emph{lattice subgroups}, \emph{commensurable subgroups}, and \emph{irreducible lattices}. It also proves a number of fundamental algebraic and geometric consequences of the assumption that $\Gamma$ is a lattice, including the following.

\smallbreak

\pref{GammaUnip->notcpct} Recall that an element~$u$ of $\SL(n,\real)$ is \emph{unipotent} if its characteristic polynomial is $(x-1)^n$ (or, in other words, its only eigenvalue is~$1$).
If $G/\Gamma$ is compact, then $\Gamma$~does not have any nontrivial unipotent elements.
This is proved by combining the Jacobson-Morosov Lemma \pref{JacobsonMorosov} with the observation that if a sequence $c_i \Gamma$ leaves all compact sets, and $U$~is a precompact set in~$G$, then, after passing to a subsequence, the sets $Uc_1 \Gamma, Uc_2 \Gamma, \ldots$ are all disjoint. However, $G/\Gamma$ has finite volume, so it cannot have infinitely many disjoint open sets that all have the same volume.

\smallbreak

\pref{GammaNotinConn} (\emph{Borel Density Theorem}) $\Gamma$ is not contained in any connected, proper, closed subgroup of~$G$. 
Assuming that $G/\Gamma$ is compact, the key to proving this is to note that if $\rho \colon G \to \GL(m,\real)$ is any continuous homomorphism, $u$~is any unipotent element of~$G$, and $v \in \real^m$, then the coordinates of the vector $\rho(u^k) v$ are polynomial functions of~$k$. However, if $G/\Gamma$ is compact, and $v$~happens to be $\rho(\Gamma)$-invariant, then the coordinates are all bounded. Since every bounded polynomial is constant, we conclude that every $\rho(\Gamma)$-invariant vector is $\rho(G)$-invariant. From this, the desired conclusion follows by looking at the action of~$G$ on exterior powers of its Lie algebra.

\smallbreak

\pref{GammaFinPres}  $\Gamma$ is finitely presented. When $G/\Gamma$ is compact, this follows from the fact that the fundamental group of any compact manifold is finitely presented. For the noncompact case, it follows from the existence of a nice fundamental domain for the action of~$\Gamma$ on~$G$ (which will be explained in \cref{ReductionChap}).

\smallbreak

\pref{torsionfree} (\emph{Selberg's Lemma}) $\Gamma$ has a torsion-free subgroup of finite index. For example, if $\Gamma = \SL(3,\integer)$, then the desired torsion-free subgroup can be obtained by choosing any prime $p \ge 3$, and taking the matrices in~$\Gamma$ that are congruent to the identity matrix, modulo~$p$.

\smallbreak

\pref{GammaResidFinite} $\Gamma$ is residually finite. For example, if $\Gamma = \SL(3,\integer)$, then no nontrivial element of~$\Gamma$ is in the intersection of the finite-index subgroups used in the preceding paragraph's % still preceding paragraph @@@
proof of Selberg's Lemma.

\smallbreak

\pref{FreeInGamma} (\emph{Tits Alternative}) $\Gamma$ contains a nonabelian free subgroup. This is proved by using the \emph{Ping-Pong Lemma} \pref{PingPong}, which, roughly speaking, states that if homeomorphism~$a$ contracts all of the space toward one point, and homeomorphism~$b$ contracts all of the space toward a different point, then the group generated by $a$ and~$b$ is free.
	

\smallbreak

\pref{MooreErgBasicThm} (\emph{Moore Ergodicity Theorem}) If $H$ is any noncompact, closed subgroup of~$G$, then every real-valued, $H$-invariant, measurable function on $G/\Gamma$ is constant (a.e.). 
The general case will be proved in \cref{MooreErgPfSect},
%The proof of this (and other theorems of this type) will be discussed in \cref{ErgodicChap} (Ergodic Theory).
but suppose, for example, that $G = \SL(2,\real)$, $H = \{a^s\}$ is the group of diagonal matrices, and $f$ is an $H$-invariant function that, for simplicity, we assume is uniformly continuous. If we let $\{u^t\}$ be the group of upper-triangular matrices with $1$'s on the diagonal, then we have
	$$ u^t \cdot f = u^t a^s \cdot f = a^s u^{e^{-s} t} \cdot f \stackrel{s \to \infty}{\ \longrightarrow\ } a^s u^0 \cdot f = f ,$$
so $f$~is invariant under~$\{u^t\}$. Similarly, it is also invariant under the group of lower-triangular matrices. So $f$ is $G$-invariant, and therefore constant. 
%A similar argument applies when $f$~is merely measurable, not uniformly continuous.





\subsection*{\cref{ArithGrpsChap}. What is an Arithmetic Group?} 
Roughly speaking, an \emph{arithmetic subgroup}~$G_\integer$ of~$G$ is obtained by embedding $G$ in some $\SL(\ell,\real)$, and taking the resulting set of integer points of~$G$. That is, $G_\integer$ is the intersection of~$G$ with $\SL(\ell,\integer)$. However, in order for~$G_\integer$ to be called an arithmetic subgroup, the embedding $G \hookrightarrow \SL(\ell,\integer)$ is required to satisfy a certain technical condition (``defined over~$\rational$'') \csee{DefdQDefn}.

\smallbreak

\pref{arith->latt} Every arithmetic subgroup of~$G$ is a lattice in~$G$. This fundamental fact will be proved in \cref{SLnZLattChap,ReductionChap}.

\smallbreak

\pref{MargulisArith} (\emph{Margulis Arithmeticity Theorem})
Conversely, if $G$ is neither $\SO(1,n)$ nor $\SU(1,n)$, then every lattice in~$G$ is an arithmetic subgroup. Therefore, in most cases, ``arithmetic subgroup'' is synonymous with ``lattice\zz.''
This amazing theorem will be proved in \cref{MargArithPf}.

It is a hugely important result. The definition of ``lattice'' is quite abstract, but a fairly explicit list of all the lattices in~$G$ can be obtained by combining this theorem with the classification of arithmetic subgroups that will be given in \cref{ArithClassicalChap}.


\smallbreak

\pref{GodementCriterion} (\emph{Godement Compactness Criterion}) 
$G/G_\integer$ is compact if and only if the identity element is the only unipotent element of~$G_\integer$. The direction ($\Rightarrow$) is very elementary and was proved in the previous chapter \see{GammaUnip->notcpct}. The converse uses the same main idea, combined with the simple observation that if a polynomial has integer coefficients, and all of its roots are close to~$1$, then all of its roots are exactly equal to~$1$.

\smallbreak

\pref{RestrictScalarsSect} The embedding of~$G$ in $\SL(\ell,\real)$ is not at all unique, and different embeddings can yield quite different arithmetic subgroups~$G_\integer$. One very important method of constructing non-obvious embeddings is called \emph{Restriction of Scalars}. It starts by choosing a field~$F$ that is a finite extension of~$\rational$. If we think of~$F$ as a vector space over~$\rational$, then it can be identified with some~$\rational^n$, in such a way that the ring~$\ints$ of algebraic integers of~$F$ is identified with~$\integer^n$. This implies that the group $G_{\ints}$ is isomorphic to $G'_\integer$, where $G'$ is a semisimple group that has~$G$ as one of its factors. Therefore, this method allows arithmetic subgroups to be constructed not only from ordinary integers, but also from algebraic integers.





\subsection*{\cref{EgArithGrpsChap}. Examples of Arithmetic Groups} 
This chapter explains how to construct many arithmetic subgroups of $\SL(2,\real)$, %\csee{ArithLattSL2}, 
$\SO(1,n)$, %\csee{ArithSO1nSect}, 
and $\SL(n,\real)$,
% \csee{NonLattinSL3Sect,CocpctLattSL3R,LattSlnRSect}, 
by using unitary groups and quaternion algebras (and other division algebras). (Restriction of scalars is also used for some of the cocompact ones.) It will be proved in \cref{ArithClassicalChap} that these fairly simple constructions actually produce all of the arithmetic subgroups of these groups.

\pref{NonArithSO1nSect} There exist non-arithmetic lattices in $\SO(1,n)$ for every~$n$. This was proved by M.\,Gromov and I.\,Piatetski-Shapiro. They ``glued together'' two arithmetic lattices to create a ``hybrid'' lattice that is not arithmetic.








\subsection*{\cref{SLnZLattChap}. $\SL(n,\integer)$ is a lattice in $\SL(n,\real)$} 
This chapter explains two different proofs of the fundamental fact (already mentioned in \cref{arith->latt}) that $G_\integer$ is a lattice in~$G$, in the illustrative special case where $G = \SL(n,\real)$ and $G_\integer = \SL(n,\integer)$.

\smallbreak

The first proof is quite short and elementary, and is presented fairly completely. It constructs a nice set that is (approximately) a fundamental domain for the action of~$\Gamma$ on~$G$.
%, by using the Iwasawa decomposition $G = KAN$. This proof 
The key notion is that of a \emph{Siegel set}.  We begin with the Iwasawa decomposition $G = KAN$.%
\noprelistbreak
	\begin{itemize}
	\item $K = \SO(n)$ is a maximal compact subgroup of~$G$.
	\item The group~$A$ of diagonal matrices in~$G$ is isomorphic to~$\real^{n-1}$, so we can think of it as a real vector space. Under this identification, the ``simple roots'' are linear functionals $\alpha_1, \ldots,\alpha_{n-1}$ on~$A$. Choose any $t \in \real$, and let
		$$ A_t = \{\, a \in A \mid \text{$\alpha_i(a) \ge t$ for all~$i$} \,\} ,$$
	so $A_t$ is a polyhedral cone in~$A$.
	
	\item $N$ is the group of upper-triangular matrices with $1$'s on the diagonal.
	\item Choose any compact subset~$N_0$ of~$N$.
	\end{itemize}
Then the product $\Siegel = N_0 \, A_t \,K$ is a Siegel set \csee{SiegelSLnZSect}. It depends on the choice of~$t$ and~$N_0$.

A straightforward calculation shows that every Siegel set has finite volume \csee{SiegelSLnRFinMeas}. It is also not terribly difficult to find a Siegel set~$\Siegel$ with the property that $G_\integer  \cdot \Siegel = G$ \csee{SiegelFundDomSLnZ}. This implies that $G/G_\integer$ has finite volume, so $G_\integer$ is a lattice in~$G$, or, in other words, $\SL(n,\integer)$ is a lattice in $\SL(n,\real)$.

Unfortunately, some difficulties arise when generalizing this method to other groups, because it is more difficult to use Siegel sets to construct an appropriate fundamental domain in the general case. The main ideas will be explained in \cref{ReductionChap}.

\smallbreak

So we also present a different proof that is much easier to generalize \csee{SLNZISLATTSlickSect}. Namely, the general case is quite easy to prove if one accepts the following key fact that was proved by Margulis: If 
	\begin{itemize}
	\item $u^t$ is any unipotent $1$-parameter subgroup of\/ $\SL(n,\real)$,
	and
	\item $x \in \SL(n,\real)/\SL(n,\integer)$,
	\end{itemize}
then there is a compact subset~$C$ of\/ $\SL(n,\real)/\SL(n,\integer)$, and some $\epsilon > 0$, such that at least $\epsilon \%$ of the orbit $\{u^t x\}_{t \in \real}$ is in the set~$C$ \csee{DaniMargulisUnipReturns}.





\notocsection*{\cref{ConceptsPart}. Important Concepts}
This part of the book explores several fundamental ideas that are important not only for their applications to arithmetic groups, but much more generally.


\smallbreak

\subsection*{\cref{RrankChap}.  Real rank}
This chapter defines the real rank of~$G$, which is an important invariant in the study of semisimple Lie groups. It also describes some consequences of assuming that the real rank is at least two, and presents the definition and basic structure of the minimal parabolic subgroups of~$G$.




\smallbreak

\subsection*{\cref{QrankChap}.  $\rational$-rank}
This chapter, unlike the others in this part of the book, discusses a topic that is primarily of interest in the theory of arithmetic groups (and related algebraic groups). Largely parallel to \cref{RrankChap}, it defines the $\rational$-rank of~$\Gamma$, describes some consequences of assuming that the $\rational$-rank is at least two, and presents the definition and basic structure of the minimal parabolic $\rational$-subgroups of~$G$.




\smallbreak

\subsection*{\cref{QuasiChap}. Quasi-isometries}
Any finite generating set~$S$ for~$\Gamma$ yields a metric~$d_S$ on~$\Gamma$: the distance from~$x$ to~$y$ is the minimal number of elements of~$S$ that need to be multiplied together to obtain $x^{-1} y$. Unfortunately, this ``word metric'' is not canonical, because it depends on the choice of the generating set~$S$. However, it is  well-defined up to a bounded factor, so, to get a geometric object that is uniquely determined by~$\Gamma$, we consider two metric spaces to be equivalent (or \emph{quasi-isometric}) if there is a map between them that only distorts distances by a bounded factor \csee{QuasiIsomDefn}. 

\smallbreak

\pref{CocpctActIsQI} Some quasi-isometries arise from cocompact actions: it is not difficult to see that if $\Gamma$ acts cocompactly, by isometries on a (nice) space~$X$, then there is a quasi-isometry from~$\Gamma$ to~$X$. Thus, for example, any cocompact lattice in $\SO(1,n)$ is quasi-isometric to the hyperbolic space~$\hyperbolic^n$.

\smallbreak

\pref{GromovHyperGrpsSect}
$\Gamma$ is \emph{Gromov hyperbolic} if and only if $\Rrank G = 1$ and $\Gamma$~is compact, except that all lattices in $\SL(2,\real)$ are hyperbolic, not only the cocompact ones. One direction is a consequence of the well-known fact that $\integer \times \integer$ is not contained in any hyperbolic group. The other direction (for the cocompact case) is a special case of the fact that the fundamental group of any closed manifold of strictly negative sectional curvature is hyperbolic.






\smallbreak

\subsection*{\cref{UnitaryRepChap}. Unitary representations}
This chapter presents some basic concepts in the theory of unitary representations, the study of group actions on Hilbert spaces. The Moore Ergodicity Theorem \pref{MooreErgBasicThm} is proved in \cref{MooreErgPfSect}, and the ``induced representations'' defined in \cref{InducedRepSect} will be used in \cref{LattInTSect} to prove that $\Gamma$ has Kazhdan's Property~$(T)$ if $\Rrank G \ge 2$.

\smallbreak

\pref{DecayMatCoeffSimple} (\emph{Decay of matrix coefficients}) If $\pi$ is a continuous homomorphism from~$G$ to the unitary group of a Hilbert space~$\Hilbert$, then 
	$$ \text{$\lim_{\|g\|\to \infty} \langle \pi(g) \phi \mid \psi \rangle = 0$, \ for all $\phi,\psi \in \Hilbert$} .$$
This yields the Moore Ergodicity Theorem \pref{MooreErgBasicThm} as an easy corollary, and the proof is based on the existence of $a \in G$ and (unipotent) subgroups $U^+$ and~$U^-$ of~$G$, such that $\langle U^+, U^-\rangle = G$ and $a^n u a^{-n} \to e$ as $n \to \infty$ (or $-\infty$), for all $u \in U^+$ (or $U^-$, respectively). 

\smallbreak

\pref{PeterWeyl}
Every unitary representation of any compact Lie group is a direct sum of finite-dimensional, irreducible unitary representations.

 \smallbreak

\pref{IntIrredAbel}
Every unitary representation of any abelian Lie group is a direct integral of one-dimensional unitary representations.

 \smallbreak

\pref{UnitaryGDirectInt}
Every unitary representation of~$G$ is a direct integral of irreducible unitary representations.




\subsection*{\cref{AmenableChap}. Amenable Groups}
Amenability is such a fundamental notion that it has very many quite different definitions, all of which determine exactly the same class of groups \csee{AmenDefn,AmenEquiv}. One useful choice is that a group~$\Lambda$ is \emph{amenable} if every continuous action of~$\Lambda$ on a compact, metric space has a finite, invariant measure. 

\smallbreak

\pref{FreeSubgrp->Nonamen} The fact that the lattice~$\Gamma$ contains a nonabelian free subgroup \csee{FreeInGamma} implies that it is not amenable. This is because subgroups of amenable groups are amenable \csee{SubgrpAmen}, and free groups do have actions (such as the actions described in the Ping-Pong Lemma \pref{PingPong}) that do not have a finite, invariant measure.

\smallbreak

Even so, amenability plays an important role in the study of~$\Gamma$, through the following observation:

\smallskip

\pref{G/amen->Meas(X)} (\emph{Furstenberg Lemma})
If $P$ is an amenable subgroup of~$G$, and we have a continuous action of~$\Gamma$ on some compact, metric space~$X$, then there exists a measurable, $\Gamma$-equivariant map from $G/P$ to the space $\Prob(X)$ of measures~$\mu$ on~$X$, such that $\mu(X) = 1$. To prove this, let $\mathcal{F}$ be the set of measurable, $\Gamma$-equivariant maps from $G$ to $\Prob(X)$. With an appropriate weak topology, this is a compact, metrizable space, and $P$~acts on it by translation on the right. Since $P$ is amenable, there is a $P$-invariant, finite measure~$\mu$ on~$\mathcal{F}$. The barycenter of this measure is a fixed point of~$P$ in~$\mathcal{F}$, and this fixed point is a function on~$G$ that factors through to a well-defined $\Gamma$-equivariant map from $G/P$ to $\Prob(X)$.

\smallbreak



\subsection*{\cref{KazhdanTChap}. Kazhdan's Property $(T)$}
To say $\Gamma$ has Kazhdan's property~$(T)$ means that if a unitary representation of~$\Gamma$ does not have any (nonzero) vectors that are fixed by~$\Gamma$, then it does not \emph{almost-invariant vectors}, that is, vectors that are moved only a small distance by the elements of any given finite subset of~$\Gamma$ \csee{KazhdanTDefn}.

\smallbreak

\pref{T+amen->finite} Kazhdan's property~$(T)$ is, in a certain sense, the antithesis of amenability: a discrete group cannot have both properties unless it is finite.
This is because the regular representation of any amenable group has almost-invariant vectors.

\smallbreak

\pref{KazhdanEasy} Every discrete group with Kazhdan's property~$(T)$ is finitely generated. To see this, let $\Hilbert = \bigoplus_F \LL2(\Lambda/F)$, where $F$ ranges over all the finitely generated subgroups of~$\Lambda$. Then, by construction, every finite subset of~$\Lambda$ fixes some nonzero vector in~$\Hilbert$.

\smallbreak

\pref{WhichGKazhdan} $G$ has Kazhdan's property~$(T)$, unless $G$ is either $\SO(1,n)$ or $\SU(1,n)$.
To prove this for $G = \SL(3,\real)$, first note that the semidirect product $\SL(2,\real) \ltimes \real^2$ can be embedded in~$G$. Also note that there are elements $a$ and~$b$ of $\SL(2,\real)$, such that, if $Q$ is any of the $4$~quadrants of~$\real^2$, then either $aQ$ or~$bQ$ is disjoint from~$Q$ (except for the $0$ vector). Applying this to the Pontryagin dual of~$\real^2$ implies that if a representation of the semidirect product $\SL(2,\real) \ltimes \real^2$ has almost-invariant vectors, then it must have a nonzero vector that is invariant under~$\real^2$. This vector must be invariant under all of $\SL(3,\real)$, by a generalization of the Moore Ergodicity Theorem that is called the \emph{Mautner phenomenon} \pref{MautnerPhenom}.

\smallbreak

\pref{Kazhdan:G->Gamma} $\Gamma$ has Kazhdan's property~$(T)$, unless $G$ is either $\SO(1,n)$ or $\SU(1,n)$. Any unitary representation~$\pi$ of~$\Gamma$ can be ``induced'' to a representation~$\pi_\Gamma^G$ of~$G$. If $\pi$ has almost-invariant vectors, then the induced representation has almost-invariant vectors, and, since $G$ has Kazhdan's property~$(T)$, this implies that $\pi_\Gamma^G$ has $G$-invariant vectors. Any such vector must come from a $\Gamma$-invariant vector in~$\pi$.

\smallbreak

\pref{T<>FH} A group has Kazhdan's property~$(T)$ if and only if every action of the group by (affine) isometries on any Hilbert space has a fixed point. 
This is not at all obvious, but here is the proof of one direction. 

Suppose $\Gamma$ does not have Kazhdan's property~$(T)$, so there exists a unitary representation of~$\Gamma$ on some Hilbert space~$\Hilbert$ that has almost-invariant vectors, but does not have invariant vectors. 
Choose an increasing chain $F_1 \subseteq F_2 \subseteq \cdots$ of finite subsets whose union is all of~$\Gamma$. Since $\Hilbert$ has almost-invariant vectors, there exists a unit vector $v_n \in \Hilbert$, such that $\| f v_n - v_n\| < 1/2^n$ for all $f \in F_n$.
Now, define $\alpha \colon \Gamma \to \Hilbert^\infty$ by 
	$$ \alpha(g)_n = n\bigl( gv_n - v_n \bigr) .$$
Then $\alpha$ is a $1$-cocycle, so defining
	$ g * v = gv + \alpha(g)$
yields an action of~$\Gamma$ on the Hilbert space~$\Hilbert^\infty$. Since $\Hilbert$ has no nonzero invariant vectors, it is not difficult to see that $\alpha$ is an unbounded function on~$\Gamma$, so $\alpha$ is not a coboundary. This implies that the corresponding action on~$\Hilbert^\infty$ has no fixed points.




\smallbreak

\subsection*{\cref{ErgodicChap}. Ergodic Theory}
\emph{Ergodic Theory} can be defined as the measure-theoretic study of group actions. In this category, the analogue of the transitive actions are the so-called \emph{ergodic} actions, for which every measurable, invariant function is constant (a.e.) \csee{ErgodicDefn}. 

%\smallbreak
%
%\pref{MooreErgodicity} (\emph{Moore Ergodicity Theorem}) 
%If $H$ is any noncompact, closed subgroup of~$G$, then the action of~$H$ on  $G/\Gamma$ is ergodic. 
%This was already stated in \cref{MooreErgBasicThm}, but this section provides the proof \csee{MooreErgPfSect}.

\smallbreak

\pref{PointwiseErgThm} (\emph{Pointwise Ergodic Theorem})
If $\integer$ acts ergodically on~$X$, with finite invariant measure, and $f$~is any $L^1$-function on~$X$, then the average of~$f$ on almost every $\integer$-orbit is equal to the average of~$f$ on the entire space~$X$.

\smallbreak

\pref{ErgodicDecomp}
Every measure-preserving action of~$G$ can be measurably decomposed into a union of ergodic actions.
%If $G$ acts continuously on a complete metric space~$X$, with a $\sigma$-finite, invariant measure~$\mu$ measure-preserving action of~$G$ can be decomposed (measurably) as a union of ergodic actions.

\smallbreak

\pref{GMixing}
If the action of~$G$ on a space~$X$ is ergodic, with a finite, invariant measure, then the action of~$G$ on $X \times X$ is also ergodic.












\notocsection*{\cref{MajorPart}. Major Results}

Here are some of the major theorems in the theory of arithmetic groups.

\subsection*{\cref{MostowChap}. Mostow Rigidity Theorem} \ 
% check to make sure there's not a page break here !!!

\smallskip

\pref{MostowRigidity} (\emph{Mostow Rigidity Theorem}) 
Suppose $\Gamma_i$ is a lattice in~$G_i$, for $i = 1,2$, and $\varphi \colon \Gamma_1 \to \Gamma_2$. If $G_i$ has trivial center and no compact factors, and is not $\PSL(2,\real)$, then $\varphi$ extends to an isomorphism $\overline\varphi \colon G_1 \to G_2$. 

In most cases, the desired conclusion is a consequence of the Margulis Superrigidity Theorem, which will be discussed in \cref{MargulisSuperChap}. 
However, a different proof is needed when $G_1 = G_2 = \SO(1,n)$ (and some other cases). Assuming that the lattices are cocompact, the proof uses the fact (mentioned in \cref{CocpctActIsQI}) that $\Gamma_1$ and~$\Gamma_2$ are quasi-isometric to~$\hyperbolic^n$. Comparing the two embeddings yields a quasi-isometry $\varphi$ from $\hyperbolic^n$ to itself. By proving that this quasi-isometry induces a map on the boundary that is conformal (i.e., preserves angles), it is shown that the two embeddings are conjugate by an isometry of~$\hyperbolic^n$.

\smallbreak

\pref{LotsOfLattsSL2} Mostow's theorem does not apply to $\PSL(2,\real)$: in this group, there are uncountably many lattices that are isomorphic to each other, but are not conjugate. This follows from the fact that there are uncountably many different right-angled hexagons in the hyperbolic plane~$\hyperbolic^2$. A compact surface of genus~$g$ can be constructed by gluing $4g-4$ of these hexagons together, in such a way that the fundamental group is a cocompact lattice in $\PSL(2,\real)$. The uncountably many different hexagons yield uncountably many non-conjugate lattices.

\smallbreak

\pref{QI->GIso} From Mostow's Theorem, we know that lattices in two different groups $G_1$ and~$G_2$ cannot be isomorphic. In fact, the lattices cannot even be quasi-isometric.
Some ideas in the proof of this fact are similar to the argument of Mostow's theorem, but we omit the details.





\subsection*{\cref{MargulisSuperChap}. Margulis Superrigidity Theorem} \ 

\smallskip

\pref{MargSuperStatementSect} (\emph{Margulis Superrigidity Theorem}) Suppose $\rho \colon \Gamma \to \GL(n,\real)$ is a homomorphism. If $G$ is neither $\SO(1,n)$ nor $\SU(1,n)$, and  mild hypotheses are satisfied, then $\rho$ extends to a homomorphism $\overline\rho \colon G \to \GL(n,\real)$. 

Assuming $\Rrank G \ge 2$, a proof is presented in \cref{SuperPfSect}. Start by letting $H$ be the Zariski closure of $\rho(\Gamma)$, and let $Q$ be a parabolic subgroup of~$H$. Furstenberg's Lemma \pref{G/amen->Meas(X)} provides a $\Gamma$-equivariant map $\psi \colon G/P \to \Prob(\real \projective^n)$. By using ``proximality\zz,'' $\psi$~can be promoted to a map $\widehat\psi \colon G/A \to \real^n$ (where $A$ is a maximal $\real$-split torus of~$G$). Thus, we have an $A$-invariant (measurable) section of the flat vector bundle over $G/\Gamma$ that is associated to~$\varphi$. Since $G$ is generated by the centralizers of nontrivial, connected subgroups of~$A$, this implies there is a finite-dimensional, $G$-invariant space of sections of the bundle, from which it follows that $\varphi$ has the desired extension to a homomorphism defined on all of~$G$.

\smallbreak

\pref{MostowRigidityIrred} This theorem of Margulis is a strengthening of the Mostow Rigidity Theorem \pref{MostowRigidity}, because the homomorphism $\rho$ is not required to be an isomorphism.  (On the other hand, Mostow's theorem applies to the groups $\SO(1,n)$ and $\SU(1,n)$, which are not allowed in the superrigidity theorem.)

\smallbreak

\pref{Super->VecBdlTrivial} In geometric terms, the superrigidity theorem implies (under mild hypotheses) that flat vector bundles over $G/\Gamma$ become trivial on a finite cover.

\smallbreak

\pref{MargArithPf} (\emph{Margulis Arithmeticity Theorem}) If $G$ is neither $\SO(1,n)$ nor $\SU(1,n)$, then the superrigidity theorem implies that every lattice in~$G$ is an arithmetic subgroup (as was stated without proof in \cref{MargulisArith}). 

The basic idea of the proof is that if there is some $\rho(\gamma)$ with a matrix entry that is transcendental, then composing $\rho$ with arbitrary elements of the Galois group $\Gal(\complex/\rational)$ would result in uncountably many different $n$-dimensional representations of~$\Gamma$. Since $G$ has only finitely many representations of each dimension, this would contradict superrigidity. Thus, we conclude that $\rho(\Gamma) \subseteq \GL(n,\overline\rational)$. By using a $p$-adic version of the superrigidity theorem, $\overline\rational$ can be replaced with~$\integer$.

\smallbreak

\pref{SuperRank1Sect} For groups of real rank one, the proof of superrigidity described in \cref{SuperPfSect} does not apply, because $A$ does not have any nontrivial, proper subgroups. Instead, a more geometric approach is used (but only a brief sketch will be provided). Let $X$ and~$Y$ be the symmetric spaces associated to~$G$ and~$H$, respectively, where $H$ is the Zariski closure of $\rho(\Gamma)$. By minimizing a certain energy functional, one can show there is a harmonic $\Gamma$-equivariant map $\psi \colon X \to Y$. Then, by using the geometry of $X$ and~$Y$, it can be shown that this harmonic map must be a totally geodesic embedding. This provides an embedding of the isometry group of~$X$ in the isometry group of~$Y$. In other words, an embedding of~$G$ in~$H$.


\subsection*{\cref{NormalSubgroupChap}. Normal Subgroups of~$\Gamma$} \ 

\smallskip

\pref{MargNormalSubgrpsThm} If $\Rrank G \ge 2$, then $\Gamma$ is almost simple. More precisely, every normal subgroup of~$\Gamma$ either is finite, or has finite index.
This is proved by showing that if $N$ is any infinite, normal subgroup of~$\Gamma$, then the quotient $\Gamma/N$ is amenable. Since $\Gamma/N$ has Kazhdan's property~$(T)$ (because we saw in \cref{Kazhdan:G->Gamma} that $\Gamma$ has this property), this implies $\Gamma / N$ is finite.

\smallbreak

\pref{Rrank1->GammaNotAlmSimple} On the other hand, if $\Rrank G = 1$, then $\Gamma$ is very far from being simple --- there are many, many infinite normal subgroups of~$\Gamma$. In fact, $\Gamma$~is ``SQ-universal\zz,'' which means that if $\Lambda$ is any finitely generated group, then there is a normal subgroup~$N$ of~$\Gamma$, such that $\Lambda$ is isomorphic to a subgroup of~$\Gamma/N$ \csee{Rank1SQUniv}.

\goodbreak


\subsection*{\cref{ArithClassicalChap}. Arithmetic Subgroups of Classical Groups} \ 
The main result of this chapter is the table on \cpageref{IrredInG} that provides a list of all of the arithmetic subgroups of~$G$ (unless $G$ is either an exceptional group or a group whose complexification~$G_\complex$ is isogenous to $\SO(8,\complex)$). Inspection of the list establishes several results that were stated without proof in previous chapters.

\smallskip

\pref{QFormsOfSLnSect} It was stated without proof in \cref{LattSlnRSect} that every arithmetic subgroup of $\SL(n,\real)$ is either a special linear group or a unitary group (if we allow division algebras in the construction). The proof of this fact is based on a calculation of the group cohomology of Galois groups (or \emph{Galois cohomology}, for short). To introduce this method in a simpler setting, it is first proved that the only $\real$-forms of the complex Lie group $\SL(n,\complex)$ are $\SL(n,\real)$, $\SL(n/2, \quaternion)$, and $\SU(k,\ell)$ \csee{GaloisCohoRealFormsSect}.

\smallbreak

\pref{QFormClassicalSect} The same methods show that all the $\rational$-forms of any classical group~$G$ are classical groups (except that there is a problem when $G$ is a real form of $\SO(8,\complex)$ \csee{D4weird}). However, we do not provide the calculations.

\smallbreak

\pref{Isotypic->irred} We say that a semisimple group $H = G_1 \times \cdots \times G_r$ is \emph{isotypic} if all the simple factors of~$H_\complex$ are isogenous to each other.
A theorem of Borel and Harder \pref{BorelHarderLocGlob} on Galois cohomology implies that if $H$ is isotypic, then it has an arithmetic subgroup that is \emph{irreducible}: it is not commensurable to a nontrivial direct product $\Gamma_1 \times \Gamma_2$. (The converse follows from the Margulis Arithmeticity Theorem unless $H$ is either $\SO(1,n) \times K$ or $\SU(1,n) \times K$.)





\subsection*{\cref{ReductionChap}. Construction of a Coarse Fundamental Domain}
This chapter presents some of the main ideas involved in the construction of a nice subset of~$G$ that approximates a fundamental domain for $G/\Gamma$ (when $\Gamma$ is an arithmetic subgroup). This generalizes the construction for $\Gamma = \SL(n,\integer)$ that was explained in \cref{SLnZLattChap}.

As in \cref{SLnZLattChap}, the key notion is that of a \emph{Siegel set}. 
The main difference is that, instead of the maximal $\real$-split torus~$A$, we must work with a subtorus~$T$ of~$A$ that is $\rational$-split, not merely $\real$-split:
	\begin{itemize}
	\item $K$ is a maximal compact subgroup of~$G$ (same as before),
	\item $S$ is a maximal $\rational$-split torus in~$G$,
	\item $S_t$ is a sector in~$S$,
	\item $P$ is a minimal parabolic $\rational$-subgroup of~$G$ that contains~$S$,
	and
	\item $P_0$ is a compact subset of~$P$.
	\end{itemize}
Then $K S_t P_0$ is a \emph{Siegel set} for~$\Gamma$ in~$G$ \csee{ArithSiegelDefn}.

It may not be possible to find a Siegel set~$\Siegel$, such that $\Siegel \cdot \Gamma = G$ \csee{SiegelNotFundEg}. (When $\dim S = 1$, this is because each Siegel set can only cover one cusp, and $G/\Gamma$ may have several cusps.) However, there is always a finite union of (translates of) Siegel sets that will suffice \csee{ReductThyArithGrps}.

\smallbreak

The existence of a nice set~$\fund$, such that $\fund \cdot \Gamma = G$, has  important consequences, such as the fact that $\Gamma$ is finitely presented \csee{FinPresSect}. This fact was stated in \cref{GammaFinPres}, but could only be proved for the cocompact case there.






\subsection*{\cref{RatnerChap}. Ratner's Theorems on Unipotent Flows}
If $\{a^t\}$ is any $1$-parameter group of diagonal matrices in~$G$, then there are $\{a^t\}$-orbits in $G/\Gamma$ that have bad closures: the closure is a fractal. M.\,Ratner proved that if a subgroup~$V$ is generated by $1$-parameter unipotent subgroups, then it is much better behaved: the closure of every $V$-orbit is a $C^\infty$~submanifold of $G/\Gamma$ \csee{Ratner-OrbitClosure}. 

\smallbreak

This theorem has important consequences in geometry and number theory. As a sample application in the theory of arithmetic groups, we mention that if $\Gamma_1$ and~$\Gamma_2$ are any two lattices in~$G$, then the subset $\Gamma_1 \, \Gamma_2$ of~$G$ is either discrete or dense \csee{ProdLattsDense}. This is proved by letting $\Gamma = \Gamma_1 \times \Gamma_2$ in $G \times G$, and letting $V$ be the diagonal embedding of~$G$ in the same group.

\smallbreak

Ratner proved that the actions of $1$-parameter unipotent subgroups on $G/\Gamma$ also have nice measurable properties: every finite, invariant probability measure is the Haar measure on a closed orbit of some subgroup of~$G\mk$ \csee{Ratner-MeasClass}, and every dense orbit is uniformly distributed \csee{Ratner-Equidistribution}.

\smallbreak

We will not prove Ratner's theorems, but some of the ideas in the proof will be described. One of the main ingredients is called ``shearing'' \csee{RatnerShearingSect}. For example, suppose $G = \SL(2,\real)$ and $V = \{u^t\}$ is a $1$-parameter unipotent subgroup. Then the key point is that if $x$ and~$y$ are two nearby points in $G/\Gamma$ (and are not on the same $\{u^t\}$-orbit), then the fastest relative motion between the two points is along the $V$-orbits. More precisely, there is some $t$, such that $u^t x$ is close to either $u^{t+1} y$ or $u^{t-1} y$.




\notocsection*{Appendices}
The main text is followed by three appendices. The first two (\cref{SSGrpsChap,BackChap}) recall some facts that are used in the main text. The third (\cref{SarithChap}) defines the notion of \emph{$S$-arithmetic group}, and quickly summarizes how the results on arithmetic groups extend to this more general setting. 

\makemarkstrue % resume updates of the running heads

