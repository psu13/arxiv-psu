%!TEX root = IntroArithGrps.tex


\mychapter{Normal Subgroups of\texorpdfstring{~$\Gamma$}{ Γ}}
\label{NormalSubgroupChap}

\prereqs{amenability (Furstenberg's Lemma \pref{G/amen->Meas(X)}) and Kazhdan's Property~$(T)$ (\cref{KazhdanTChap}). 
\emph{Also used:} the $\sigma$-algebra of Borel sets modulo sets of measure~$0$ (\cref{ErgDecompSect}) and manifolds of negative curvature.}


This chapter presents a contrast between the lattices in groups of real rank~$1$ and those of higher real rank:
\begin{itemize}
\item If $\Rrank G = 1$, then $\Gamma$ has many, many normal subgroups, so $\Gamma$ is very far from being simple.
\item If $\Rrank G > 1$ (and $\Gamma$ is irreducible), then $\Gamma$ is simple modulo finite groups. More precisely, if $N$ is any normal subgroup of~$\Gamma$, then either $N$ is finite, or $\Gamma/N$ is finite.
\end{itemize}









\section{\texorpdfstring{Normal subgroups in lattices of real rank $\ge 2$}{Normal subgroups in lattices of real rank at least two}}

\begin{thm}[(\thmindex{Margulis!Normal Subgroups}Margulis Normal Subgroups Theorem)] \label{MargNormalSubgrpsThm}
	\thmindex{Margulis!Normal Subgroups}
Assume 
\noprelistbreak
	\begin{itemize}
	\item $\Rrank G \ge 2$,
	\item $\Gamma$ is an irreducible lattice in~$G$,
	and
	\item $N$ is a normal subgroup of\/~$\Gamma$.
	\end{itemize}
Then either $N$ is finite, or\/ $\Gamma/N$ is finite.
\end{thm}

\begin{eg}
Every lattice in $\SL(3,\real)$ is simple, modulo finite groups.
In particular, this is true of $\SL(3,\integer)$.
\end{eg}

\begin{rems} \label{MargNormSubgrpRems} \ 
\noprelistbreak
\begin{enumerate}
\item The hypotheses on $G$ and~$\Gamma$ are essential:
	\noprelistbreak
	\begin{enumerate}
	\item If $\Rrank G = 1$, then every lattice in~$G$ has an infinite normal subgroup of infinite index \csee{Rrank1->GammaNotAlmSimple}. 
	\item \label{MargNormSubgrpRems-hyp-red}
	If $\Gamma$ is reducible (and $G$ has no compact factors), then $\Gamma$ has an infinite normal subgroup of infinite index \csee{MargNormSubgrpRems-hyp-redEx}.
	\end{enumerate}
\item The finite normal subgroups of~$\Gamma$ are easy to understand (if $\Gamma$ is irreducible): the Borel Density Theorem implies that they are the subgroups of the finite abelian group $\Gamma \cap Z(G)$ \csee{NinZ(G)}.
\item \label{MargNormSubgrpRems-finind}
If $\Gamma$ is infinite, then $\Gamma$ has infinitely many normal subgroups of finite index \csee{MargNormSubgrpRems-finindEx}, so $\Gamma$ is \emph{not} simple.

\item \label{MargNormSubgrpRems-CSP}
In most cases, the subgroups of finite index are described by the ``\thmindex{Congruence Subgroup Property}Congruence Subgroup Property\zz.'' For example, if $\Gamma = \SL(3,\integer)$, then the principal congruence subgroups are obvious subgroups of finite index \csee{CSGfinite}. More generally, any subgroup of~$\Gamma$ that contains a principal congruence subgroup obviously has finite index. The Congruence Subgroup Property is the assertion that every finite-index subgroup is one of these obvious ones. It is true for $\SL(n,\integer)$, whenever $n \ge 3$, and a similar (but slightly weaker) statement is conjectured to be true whenever $\Rrank G \ge 2$ and $\Gamma$~is irreducible.
\end{enumerate}
\end{rems}




The remainder of this section presents the main ideas in the proof of \cref{MargNormalSubgrpsThm}. In a nutshell, we will show that if $N$ is an infinite, normal subgroup of~$\Gamma$, then
	\begin{enumerate}
	\item $\Gamma/N$ has Kazhdan's property $(T)$,
	and
	\item $\Gamma/N$ is amenable.
	\end{enumerate}
This implies that $\Gamma/N$ is finite \csee{T+amen->finite}.

In most cases, it is easy to see that $\Gamma/N$ has Kazhdan's property (because $\Gamma$ has the property), so the main problem is to show that $\Gamma/N$ is amenable. This amenability follows easily from an ergodic-theoretic result that we will now describe. 

\begin{assump}
To minimize the amount of Lie theory needed, let us assume 
	$$G = \SL(3,\real) .$$
\end{assump}

\begin{notation} \label{MargNormalP=upper}
Let 
	$$P = \begin{bmatrix}  *&& \\ *&*& \\ *&*&* \end{bmatrix} \subset \SL(3,\real) = G .$$
Hence, $P$ is a (minimal) parabolic subgroup of~$G$.
\end{notation}

Note that if $Q$ is any closed subgroup of~$G$ that contains~$P$, then the natural map $G/P \to G/Q$ is $G$-equivariant, so we may say that $G/Q$ is a $G$-equivariant quotient of $G/P$. Conversely, it is easy to see that spaces of the form $G/Q$ are the only $G$-equivariant quotients of $G/P$. In fact, these are the only quotients even if we only assume that quotient map is equivariant \emph{almost} everywhere \csee{GQuot(G/H)}.

Furthermore, since $\Gamma$ is a subgroup of~$G$, it is obvious that every $G$-equivariant map is $\Gamma$-equivariant. Conversely, the following surprising result shows that every $\Gamma$-equivariant quotient of $G/P$ is $G$-equivariant (up to a set of measure~$0$):

\begin{thm}[(Margulis)] \label{blackbox} Suppose
\noprelistbreak
\begin{itemize}
\item $\Rrank G \ge 2$,
\item $P$ is a minimal parabolic subgroup of~$G$,
\item $\Gamma$ is irreducible,
\item $\Gamma$ acts by homeomorphisms on a compact, metrizable space~$Z$,
and
\item  $\psi \colon G/P \to Z$ is essentially\/ $\Gamma$-equivariant\/ \textup(and measurable\/\textup).
\end{itemize}
Then the action of\/~$\Gamma$ on~$Z$ is measurably isomorphic to the natural action of\/~$\Gamma$ on~$G/Q$\/ \textup(a.e.\textup), for some closed subgroup~$Q$ of~$G$ that contains~$P$.
\end{thm}

\begin{rem} \label{BlackboxRem} \ 
\noprelistbreak
\begin{enumerate}
\item \label{BlackboxRem-meas}
Perhaps we should clarify the choice of measures in the statement of \cref{blackbox}.
(A measure class on $G/P$ is implicit in the assumption that $\psi$ is \emph{essentially} $\Gamma$-equivariant. Measure classes on~$Z$ and~$G/Q$ are implicit in the ``(a.e.)'' in the conclusion of the theorem.)
	\begin{enumerate}
	\item Because $G/P$ and $G/Q$ are $C^\infty$ manifolds, Lebesgue measure supplies a measure class on each of these spaces. The Lebesgue class is invariant under all diffeomorphisms, so, in particular, it  is $G$-invariant.
	\item There is a unique measure class on~$Z$ for which $\psi$ is measure-class preserving \csee{UniqMeasClassPresEx}.
	\end{enumerate}
\item The proof of \cref{blackbox} will be presented in \cref{QuotG/PPfSect}. It may be skipped on a first reading.
%\item Although \cref{blackbox} is stated only for $G = \SL(3,\real)$, it is valid (and has essentially the same proof) whenever $\Rrank G \ge 2$ and $\Gamma$~is irreducible.
\end{enumerate}
\end{rem}


\begin{proof}[\textbf{\upshape Proof of \cref{MargNormalSubgrpsThm}}]
Let $N$ be a normal subgroup of~$\Gamma$, and assume $N$ is infinite. We wish to show $\Gamma/N$ is finite.  
Let us assume, for simplicity, that $\Gamma$ has Kazhdan's Property $(T)$. (For example, this is true if $G = \SL(3,\real)$, or, more generally, if $G$ is simple \csee{GammaHasT}.)
Then $\Gamma/N$ also has Kazhdan's Property $(T)$ \csee{KazhdanEasy}, so it suffices to show that $\Gamma/N$ is amenable \csee{T+amen->finite}.

Suppose $\Gamma/N$ acts by homeomorphisms on a compact, metrizable space~$X$.
In order to show that $\Gamma/N$ is amenable, it suffices to find an invariant probability measure on~$X$ \fullcsee{AmenEquiv}{InvMeas}. In other words, we wish to show that $\Gamma$ has a fixed point in $\Prob(X)$.
\begin{itemize}
\item Because $P$ is amenable, there is an (essentially) $\Gamma$-equivariant measurable map $\psi \colon G/P \to \Prob(X)$ \csee{SL3R/P->Meas(X)}.
\item From \cref{blackbox}, we know there is a closed subgroup~$Q$ of~$G$, such that the action of~$\Gamma$ on $\Prob(X)$ is measurably isomorphic (a.e.)\  to the natural action of~$\Gamma$ on~$G/Q$.
 \end{itemize}
 Since $N$ acts trivially on~$X$, we know it acts trivially on $\Prob(X) \iso G/Q$.
Hence, the kernel of the $G$-action on $G/Q$ is infinite \csee{NormalPf-NTrivEx}.
However, $G$ is simple (modulo its finite center), 
 so this implies that the action of~$G$ on $G/Q$ is trivial \csee{NormalPf-GTrivEx}. 
 (It follows that $G/Q$ is a single point, so $Q = G$, but we do not need quite such a strong conclusion.) Since $\Gamma \subseteq G$, then the action of $\Gamma$ on~$G/Q$ is trivial. In other words, every point in $G/Q$ is fixed by~$\Gamma$.
 Since $G/Q \iso \Prob(X)$ (a.e.), we conclude that almost every point in $\Prob(X)$ is fixed by~$\Gamma$; therefore, $\Gamma$ has a fixed point in $\Prob(X)$, as desired.
 \end{proof}

\begin{rem} \label{AE1Pt}
The proof of \cref{MargNormalSubgrpsThm} concludes that ``almost every point in $\Prob(X)$ is fixed by~$\Gamma$\zz,'' so it may seem that the proof provides not just a single $\Gamma$-invariant measure, but many of them. This is not the case: The proof implies that $\psi$ is essentially constant \csee{AE1PtEx}. This means that the $\Gamma$-invariant measure class $[\psi_* \mu]$ is supported on a single point of $\Prob(X)$, so ``a.e\zz.'' means only one point.
\end{rem}

\begin{exercises}

\item \label{GammaHasFiniteAbelianization}
Assume 
	\begin{itemize}
	\item $G$ is not isogenous to $\SO(1,n)$ or $\SU(1,n)$, for any~$n$,
	\item $\Gamma$ is irreducible,
	and
	\item $G$ has no compact factors.
	\end{itemize}
In many cases, Kazhdan's property $(T)$ implies that the abelianization $\Gamma / [\Gamma,\Gamma]$ of~$\Gamma$ is finite \fullcsee{KazhdanlatticeCor}{noabel}. Use \cref{MargNormalSubgrpsThm} to prove this in the remaining cases. (We saw a different proof of this in \cref{GNoAbelianization}.)

\item \label{MargNormSubgrpRems-hyp-redEx}
Verify \fullcref{MargNormSubgrpRems}{hyp-red}.
\hint{\Cref{prodirredlatt}.}

\item Suppose $\Gamma$ is a lattice in $\SL(3,\real)$. Show that $\Gamma$ has no nontrivial, finite, normal subgroups.

\item Suppose $\Gamma$ is an irreducible lattice in~$G$. Show that $\Gamma$ has only finitely many finite, normal subgroups.

\item \label{MargNormSubgrpRems-finindEx}
Show that if $\Gamma$ is infinite, then it has infinitely many normal subgroups of finite index.
\hint{\Cref{GammaResidFinite}.}

\item \label{GQuot(G/H)}
Suppose 
\noprelistbreak
	\begin{itemize}
	\item $H$ is a closed subgroup of~$G$, 
	\item $G$ acts continuously on a metrizable space~$Z$,
	and
	\item $\psi \colon G/H \to Z$ is essentially $G$-equivariant (and measurable).
	\end{itemize}
Show the action of~$G$ on~$Z$ is measurably isomorphic to the action of~$G$ on $G/Q$ (a.e.), for some closed subgroup~$Q$ of~$G$ that contains~$H$. More precisely, show there is a measurable $\phi \colon Z \to G/Q$, such that:%
\noprelistbreak
	\begin{enumerate}
	\item $\phi$ is measure-class preserving (i.e., a subset $A$ of~$G/Q$ has measure~$0$ if and only if its inverse image $\phi^{-1}(A)$ has measure~$0$),
	\item $\phi$ is one-to-one (a.e.) (i.e., $\phi$ is one-to-one on a conull subset of~$Z$),
	and
	\item $\phi$ is essentially $G$-equivariant.
	\end{enumerate}
\hint{See \fullcref{BlackboxRem}{meas} for an explanation of the measure classes to be used on $G/H$, $G/Q$, and~$Z$. For each $g \in G$, the set $\{\, x \in G/H \mid \psi(gx) = g \cdot \psi(x) \,\}$ is conull. By Fubini's Theorem, there is some $x_0 \in G/H$, such that $\psi(gx_0) = g \cdot \psi(x_0)$ for a.e.~$g$. Show the $G$-orbit of $\psi(x_0)$ is conull in~$Z$, and let $Q = \Stab_G \bigl( \psi(x_0) \bigr)$.}

\item \label{UniqMeasClassPresEx}
Suppose 
\noprelistbreak
	\begin{itemize}
	\item $\psi \colon Y \to Z$ is measurable,
	and
	\item $\mu_1$ and~$\mu_2$ are measures on~$Y$ that are in the same measure class.
	\end{itemize}
Show:
\noprelistbreak
	\begin{enumerate}
	\item The measures $\psi_*(\mu_1)$ and $\psi_*(\mu_2)$ on~$Z$ are in the same measure class.
	\item For any measure class on~$Y$, there is a unique measure class on~$Z$ for which $\psi$ is measure-class preserving.
	\end{enumerate}

\item In the setting of \cref{blackbox}, show that $\psi$ is essentially onto. That is, the image $\psi(G/P)$ is a conull subset of~$Z$.
\hint{By choice of the measure class on~$Z$, we know that $\psi$ is measure-class preserving.}

\item Let $G = \SL(3,\real)$ and $\Gamma = \SL(3,\integer)$.
Show that the natural action of~$\Gamma$ on $\real^3 / \integer^3 = \torus^3$ is a $\Gamma$-equivariant quotient of the action on~$\real^3$, but is not a $G$-equivariant quotient.

\item \label{NormalPf-NTrivEx}
In the proof of \cref{MargNormalSubgrpsThm}, we know that $\Prob(X) \iso G/Q$ (a.e.), so each element of~$N$ fixes a.e.\ point in $G/Q$. Show that $N$ acts trivially on $G/Q$ (everywhere, not only a.e.).
\hint{The action of $N$ is continuous.}

\item \label{NormalPf-GTrivEx}
In the notation of the proof of \cref{MargNormalSubgrpsThm}, show that the action of~$G$ on $G/Q$ is trivial.
\hint{Show that the kernel of the action of~$G$ on $G/Q$ is closed. You may assume, without proof, that $G$ is an almost simple Lie group. This means that every proper, closed, normal subgroup of~$G$ is finite.}

\item  \label{AE1PtEx}
In the setting of the proof of \cref{MargNormalSubgrpsThm}, show that $\psi$ is constant (a.e.).
\hint{The proof shows that a.e.\ point in the image of~$\psi$ is fixed by~$G$. Because $\psi$ is $G$-equivariant, and $G$ is transitive on $G/P$, this implies that $\psi$ is constant (a.e.).}
\end{exercises}






\section{Normal subgroups in lattices of rank one}

\Cref{MargNormalSubgrpsThm} assumes $\Rrank G \ge 2$. The following result shows that this condition is necessary:

\begin{thm} \label{Rrank1->GammaNotAlmSimple}
If\/ $\Rrank G = 1$, then\/ $\Gamma$ has a normal subgroup~$N$, such that neither $N$ nor\/ $\Gamma/N$ is finite.
\end{thm}

\begin{proof}[Proof \normalfont (assumes familiarity with manifolds of negative curvature)]
For simplicity, assume:
\noprelistbreak
	\begin{itemize}
	\item $\Gamma$ is torsion free, so it is the fundamental group of the locally symmetric space $M = \Gamma \backslash G / K$ (where $K$ is a maximal compact subgroup of~$G$).
	\item $M$ is compact.
	\item The locally symmetric metric on~$M$ has been normalized to have sectional curvature $\le -1$.
	\item The injectivity radius of~$M$ is $\ge 2$.
	\item There are closed geodesics $\gamma$ and $\lambda$ in~$M$, such that
		 $\length(\lambda) > 2 \pi$
		and
		 $\dist(\gamma,\lambda) > 2$.
	\end{itemize}
The geodesics $\gamma$ and~$\lambda$ represent (conjugacy classes of) nontrivial elements $\widehat\gamma$ and $\widehat\lambda$ of the fundamental group~$\Gamma$ of~$M$. Let $N$ be the smallest normal subgroup of~$\Gamma$ that contains~$\widehat\lambda$.

It suffices to show that $\widehat\gamma^n$ is nontrivial in $\Gamma/N$, for every $n \in \integer^+$ \csee{EnoughGammaNontrivEx}.
Construct a CW complex $\overline{M}$ by gluing the boundary of a $2$-disk~$D_\lambda$ to~$M$ along the curve~$\lambda$, so the fundamental group of~$\overline{M}$ is $\Gamma/N$. 

We wish to show that $\gamma^n$ is not null-homotopic in~$\overline{M}$. 
Suppose there is a continuous map $f \colon D^2 \to \overline{M}$, such that the restriction of~$f$ to the boundary of~$D^2$ is~$\gamma^n$. Let 
	$$D^2_0 = f^{-1}(M) ,$$
so $D^2_0$ is a surface of genus~$0$ with some number~$k$ of boundary curves. 
We may assume $f$ is \index{minimal!surface}{minimal} (i.e., the area of~$D^2$ under the pull-back metric is minimal). 
Then $D^2_0$ is a surface of curvature $\kappa(x) \le -1$ whose boundary curves are geodesics. 
Note that $f$ maps
	\begin{itemize}
	\item one boundary geodesic onto~$\gamma^n$,
	and
	\item the other $k-1$ boundary geodesics onto multiples of~$\lambda$.
	\end{itemize}
This yields a contradiction:
\begin{align*}
2 \pi (k-2)
&= -2 \pi \, \chi(D^2_0)
&& \text{\csee{Chi(PuncturedDiskEx)}}
\\&= - \int_{D^2_0} \kappa(x) \, dx 
&& \text{(Gauss-Bonnet Theorem)}
\\&\ge  \int_{D^2_0} 1 \, dx 
\\&\ge (k-1) \length(\lambda) 
&& \text{\csee{IntBdryCollarsEx}}
\\&> 2 \pi (k-1)
.  &&\qedhere \end{align*}
\end{proof}

%\begin{thm}
%If\/ $\Rrank G = 1$, then\/ $\Gamma$ has a normal subgroup~$N$, such that\/ $\Gamma/N$ is infinite, but every element of\/ $\Gamma/N$ has finite order. 
% Or maybe this needs $\Gamma$ to be cocompact?
%\end{thm}

\begin{rem}
Perhaps the simplest example of \cref{Rrank1->GammaNotAlmSimple} is when $G = \SL(2,\real)$ and $\Gamma$~is a free group \csee{FreeLattINSL2R}. In this case, it is easy to find a normal subgroup~$N$, such that $N$ and~$\Gamma/N$ are both infinite. (For example, we could take $N = [\Gamma,\Gamma]$.)
\end{rem}

There are numerous strengthenings of \cref{Rrank1->GammaNotAlmSimple} that provide infinite quotients of~$\Gamma$ with various interesting properties (if $\Rrank G = 1$). 
We will conclude this section by briefly describing just one such example.

A classical theorem of 
	\label{HNNThm}\thmindex{Higman-Neumann-Neumann}%
	Higman, Neumann, and Neumann states that every countable group can be embedded in a $2$-generated group. Since $2$-generated groups are precisely the quotients of the free group~$\free_2$ on $2$~generators, this means that $\free_2$ is ``SQ-universal'' in the following sense:

\begin{defn}
$\Gamma$ is \emph{SQ-universal} if every countable group is isomorphic to a subgroup of a quotient of~$\Gamma$. 
 (The letters ``SQ'' stand for ``subgroup-quotient\zz.'')

More precisely, the SQ-universality of~$\Gamma$ means that if $\Lambda$ is any countable group, then there exists a normal subgroup~$N$ of~$\Gamma$, such that $\Lambda$ is isomorphic to a subgroup of $\Gamma/N$.
\end{defn}

\begin{eg} \label{FnSQUnivEg}
$\free_n$ is SQ-universal, for any $n \ge 2$ \csee{FnSQUnivEx}.
\end{eg}

SQ-universality holds not only for free groups, which are lattices in $\SL(2,\real)$ \csee{FreeLattINSL2R}, but for any other lattice of real rank one:

\begin{thm} \label{Rank1SQUniv}
If\/ $\Rrank G = 1$, then\/ $\Gamma$ is SQ-universal.
\end{thm}

\begin{rem}
Although the results in this section have been stated only for~$\Gamma$, which is a lattice, the theorems are valid for a much more general class of groups. 
This is because normal subgroups can be obtained from an assumption of negative curvature (as is illustrated by the proof of \cref{Rrank1->GammaNotAlmSimple}). Indeed, \cref{Rrank1->GammaNotAlmSimple,Rank1SQUniv} remain valid when $\Gamma$ is replaced with 
%the fundamental group of any compact manifold of strictly negative curvature. More generally, the theorems are valid for 
any group that is \term[hyperbolic!Gromov]{Gromov hyperbolic} \csee{GromovHyperDefn}, or even \index{hyperbolic!group, relatively}``relatively'' hyperbolic (and not commensurable to a cyclic group).
\end{rem}


\begin{exercises}

\item \label{EnoughGammaNontrivEx}
Suppose 
	\begin{itemize}
	\item $\gamma$ and~$\lambda$ are nontrivial elements of~$\Gamma$,
	\item $\Gamma$ is torsion free,
	\item $N$ is a normal subgroup of~$\Gamma$,
	\item $\lambda \in N$,
	and
	\item $\gamma^n \notin N$, for every positive integer~$n$.
	\end{itemize}
Show that neither $N$ nor $\Gamma/N$ is infinite.

\item \label{Chi(PuncturedDiskEx)}
Show that the {Euler characteristic} of a $2$-disk with $k-1$ punctures is $2 - k$.

\item \label{IntBdryCollarsEx}
In the notation of the proof of \cref{Rrank1->GammaNotAlmSimple}, show
	$$  \int_{D^2_0} 1 \, dx \ge  (k-1) \length(\lambda) .$$
\hint{All but one of the boundary components are at least as long as~$\lambda$, and a boundary collar of width~$1$ is disjoint from the collar around any other boundary component.}

\item \label{FnSQUnivEx}
Justify \cref{FnSQUnivEg}.
\hint{You may assume the theorem of Higman, Neumann, and Neumann on embedding countable groups in $2$-generated groups.}


\end{exercises}






\section{\texorpdfstring{$\Gamma$}{Γ}-equivariant quotients of \texorpdfstring{$G/P$}{G/P}
	\texorpdfstring{\optional}{(optional)}} \label{QuotG/PPfSect}

In this section, we explain how to prove \cref{blackbox}. However, we will assume $G = \SL(2,\real) \times \SL(2,\real)$, for simplicity.

The space~$Z$ is not known explicitly, so it is difficult to study directly. Instead, as in the proof of the ergodic decomposition in \cref{ErgDecompSect}, we will look at the $\sigma$-algebra $\Bool(Z)$ of Borel sets, modulo the sets of measure~$0$. (We will think of this as the set of $\{0,1\}$-valued functions in $\LL\infty(Z)$, by identifying each set with its characteristic function.)
Note that $\psi$ induces a $\Gamma$-equivariant inclusion
	 $$\psi^* \colon \Bool(Z) \hookrightarrow \Bool(G/P) $$
\csee{B(Z)InjectsEx}. 
Via the inclusion $\psi^*$, we can identify $\Bool(Z)$ with a sub-$\sigma$-algebra of $\Bool(G/P)$:
	 $$\Bool(Z) \subseteq \Bool(G/P) .$$
%Now $\Bool(Z)$ is a $\Gamma$-invariant sub-$\sigma$-algebra of $\Bool(G/P)$ \cref{B(Z)ClosedBoolean}. 
In order to establish that $Z$ is a $G$-equivariant quotient of $G/P$, we wish to show that $\Bool(Z)$ is $G$-invariant \csee{ZGInvt->GQuot}. Therefore, \cref{blackbox}  can be reformulated as follows:

\begin{thmrefer}{blackbox}
\begin{thm} \label{blackbox'}
If $\Bool$ is any\/ $\Gamma$-invariant sub-$\sigma$-algebra of $\Bool(G/P)$, then $\Bool$ is $G$-invariant.
\end{thm}
\end{thmrefer}

To make things easier, let us settle for a lesser goal temporarily:

\begin{defn} 
The \emph{trivial} Boolean sub-$\sigma$-algebra of $\Bool(G/P)$ is $\{0,1\}$ (the set of constant functions).
\end{defn}

\begin{prop} \label{lessergoal}
If $\Bool$ is any nontrivial, $\Gamma$-invariant sub-$\sigma$-algebra of $\Bool(G/P)$, then $\Bool$ contains a nontrivial $G$-invariant Boolean algebra.
\end{prop}

\begin{rem} \ 
\noprelistbreak
\begin{enumerate}
\item To establish \cref{lessergoal}, we will find a characteristic function $\overline f \in \Bool(G/P) \setminus \{0,1\}$, such that $G \, \overline f \subseteq \Bool$.
\item The proof of \cref{blackbox'} is similar: let $\Bool_G$ be the (unique) maximal $G$-invariant Boolean subalgebra of~$\Bool$. If $\Bool_G \neq \Bool$, we will find some $\overline f \in \Bool(G/P) \setminus \Bool_G$, such that $G \, \overline f \subseteq \Bool$. (This is a contradiction.)
\end{enumerate}
\end{rem}

\begin{assump} 
To  simplify the algebra in the proof of \cref{lessergoal}, let us assume $G = \SL(2,\real) \times \SL(2,\real)$.
\end{assump}

\begin{notation} \label{BlackBoxPfNotn} \ 
\noprelistbreak
\begin{itemize}
\item $G = G_1 \times G_2$, where $G_1 = G_2 = \SL(2,\real)$,
\item $P = P_1 \times P_2$, where $P_i = \begin{Smallbmatrix} \upast& \\ \upast&\upast \end{Smallbmatrix}
\subset G_i$,
\item $U = U_1 \times U_2$, where $U_i = \begin{Smallbmatrix} 1& \\ \upast&1 \end{Smallbmatrix}
\subset P_i$,
\item $V = V_1 \times V_2$, where $V_i = \begin{Smallbmatrix} 1&\upast \\ &1 \end{Smallbmatrix}
\subset G_i$,
\item $\Gamma = $ some irreducible lattice in~$G$,
and
\item $\Bool$ = some $\Gamma$-invariant sub-$\sigma$-algebra of $\Bool(G/P)$ .
\end{itemize}
\end{notation}

\begin{rem} \label{G/PisR2}
We have $G/P = (G_1/P_1) \times (G_2/ P_2)$. Here are two useful, concrete descriptions of this space:
\begin{itemize}
\item $G/P = \RP1 \times \RP1 \iso \real^2$ (a.e.),
and
\item $G/P \iso V_1 \times V_2$ (a.e.) \csee{Gi/Pi=ViEx}.
\end{itemize}
Note that, if we identify $G/P$ with $\real^2$ (a.e.), then, for the action of~$G_1$ on~$G/P$, we have
\begin{itemize}
\item $\begin{bmatrix} k& \\ &k^{-1} \end{bmatrix} (x,y) = (k^2 x, y)$,
and
\item $\begin{bmatrix} 1& t \\ &1 \end{bmatrix} (x,y) = (x + t, y)$
\end{itemize}
\csee{GiActsByLinFracEx}.
\end{rem}

The proof of \cref{lessergoal} employs two preliminary results.
The first is based on a standard fact from first-year analysis:

\begin{lem}[(\thmindex{Lebesgue Differentiation}Lebesgue Differentiation Theorem)] \label{LebDiffThm}
Let 
	\begin{itemize}
	\item $f \in \LL1(\real^n)$, 
	\item $\lambda$ be the Lebesgue measure on~$\real^n$,
	and
	\item $B_r(p)$ be the ball of radius~$r$ centered at~$p$.
	\end{itemize}
For a.e.\ $p \in \real^n$, we have
	\begin{equation} \label{LebDiffThmEq}
	\lim_{r \to 0} \frac{1}{\lambda \bigl( B_r(p) \bigr)}\int_{B_r(p)} f \, d\lambda = f(p) 
	. \end{equation}
%where $\lambda$ is Lebesgue measure, and $B_r(x)$ is the ball of radius~$r$ centered at~$x$.
\end{lem}

Letting $n = 1$ and applying Fubini's Theorem yields:

\begin{cor} \label{LebConvMeasToProj}
Let
\noprelistbreak
\begin{itemize}
\item $f \in \LL\infty(\real^2)$,
\item $a = \begin{bmatrix} k & \\ &k^{-1} \end{bmatrix} \in G_1$, for some $k > 1$,
and
\item $\pi_2 \colon \real^2 \to \{0\}\times \real$ be the projection onto the $y$-axis.
\end{itemize}
Then, for a.e.\ $v \in V_1$,
	$$ \text{$a^n v f$ converges in measure to $(vf) \circ \pi_2$ as $n \to \infty$} . $$
%\textup(Here, $(a^n v f)(p) = f \bigl( a^{-n}(p-v) \bigr)$.\textup)
\end{cor}

\begin{proof}
\Cref{LebConvMeasToProjPfEx}.
\end{proof}

The other result to be used in the proof of \cref{lessergoal} is a consequence of the Moore Ergodicity Theorem:

\begin{prop} \label{GammaVADense}
For a.e.\ $v \in V_1$, \ $\Gamma v^{-1} a^{-\natural}$ is dense in~$G$.
\end{prop}

\begin{proof}
Taking inverses, we wish to show $\closure{a^{\natural} v \Gamma} = G$; 
i.e., the (forward) $a$-orbit of $v \Gamma$ is dense in $G/\Gamma$, for a.e.\ $v \in V_1$.
We will show that 
	$$ \text{$\closure{a^{\natural} g \Gamma} = G$, for a.e.\ $g \in G$,} $$
and leave the remainder of the proof to the reader \csee{a^nvGammaDenseEx}.

Given a nonempty open subset~$\open$\, of~$G/\Gamma$,
let 
	$$E = \bigcup_{n>0} a^{-n} \open .$$
Clearly, $a^{-1} E \subseteq E$. Since $\mu(a^{-1} E) = \mu(E)$ (because the measure on $G/\Gamma$ is $G$-invariant), this implies
	 $E$ is $a$-invariant (a.e.).
Since the Moore Ergodicity Theorem \pref{MooreErgodicity} tells us that $a$ is ergodic on $G/\Gamma$, 
we conclude that $E = G/\Gamma$ (a.e.). This means that, for a.e.\ $g \in G$, the forward $a$-orbit of~$g$ intersects~$\open$\,. 

Since $\open$\, is an arbitrary open subset, and $G/\Gamma$ is second countable, we conclude that the forward $a$-orbit of a.e.~$g$ is dense.
\end{proof}

\begin{proof}[\upshape \textbf{\mathversion{bold}Proof of \cref{lessergoal} for $G = \SL(2,\real) \times \SL(2,\real)$}] 
Identify $G/P$ with~$\real^2$, as in \cref{G/PisR2}.
Since $\Bool$ is nontrivial, it contains some nonconstant~$f$.
Now $f$ cannot be essentially constant both on almost every vertical line and on almost every horizontal line \csee{ConstVert+Horiz->Const}, so we may assume there is a non-null set of vertical lines on which it is not constant. This means that
	 $$ \text{$\bigset{v \in V_1 }{ \begin{matrix} \text{$(vf) \circ \pi_2$ is not} \\ \text{essentially constant} \end{matrix} }$ 
	has positive measure.} $$
\Cref{LebConvMeasToProj,GammaVADense} tell us we may choose~$v$ in this set, with the additional properties that 
\begin{itemize}
\item $a^n v f \to (vf) \circ \pi_2$,
and
\item $\Gamma v^{-1} a^{-\natural}$ is dense in~$G$.
\end{itemize}
Let $\overline{f} = (vf) \circ \pi_2$, so
	$$ a^n v f \to \overline{f} .$$
Now, for any $g \in G$, there exist $\gamma_i \in \Gamma$ and $n_i \to \infty$, such that 
	 $$g_i \defd \gamma_i v^{-1} a^{-n_i} \to g .$$
Then we have 
	$$g_i a^{n_i} v = \gamma_i \in \Gamma ,$$
so the $\Gamma$-invariance of~$\Bool$ implies
	$$\Bool \ni \gamma_i f = g_i \, a^{n_i} v f\to g \, \overline {f} $$
\csee{ContOnB(G/P)}.
Since $\Bool$ is closed \csee{WeakClosed->ConvMeasEx}, we conclude that $g \, \overline {f} \in \Bool$. Since $g$ is an arbitrary element of~$G$, this means $G \, \overline {f} \subseteq \Bool$.
Also, from the choice of~$v$, we know that $\overline {f} = (vf) \circ \pi_2$ is not essentially constant.
\end{proof}

Combining the above argument with a list of the $G$-invariant Boolean subalgebras of $\Bool(G/P)$ yields \cref{blackbox'}:

\begin{proof}[\upshape \textbf{\mathversion{bold}Proof of \cref{blackbox'} for $G = \SL(2,\real) \times \SL(2,\real)$}] \ 
Let $\Bool_G$ be the largest $G$-invariant subalgebra of~$\Bool$, and suppose $\Bool \neq \Bool_G$. (This will lead to a contradiction.)

It is shown in \cref{EquivQuotsSL2xSL2/PEx} that the only $G$-invariant subalgebras of $\Bool(G/P) = \Bool(\real^2)$ are  
\begin{itemize}
\item $\Bool(\real^2)$,
\item $\{\, \text{functions constant on horizontal lines (a.e.)} \,\}$,
\item $\{\, \text{functions constant on vertical lines (a.e.)} \,\}$,
and
\item $\{\, 0,1 \,\}$.
\end{itemize}
So $\Bool_G$ must be one of these $4$ subalgebras.

We know $\Bool_G \neq \Bool(\real^2)$ (otherwise $\Bool = \Bool_G$). 
Also, we know $\Bool$ is nontrivial (otherwise $\Bool = \{0,1\} = \Bool_G$), so \cref{lessergoal} tells us that $\Bool_G \neq \{0,1\}$.
Hence, we may assume, by symmetry, that 
	 \begin{equation} \label{BoolG=ConstVertEq}
	 \Bool_G = \{\, \text{functions constant on vertical lines (a.e.)} \,\}  
	 . \end{equation}
Since $\Bool \neq \Bool_G$, there is some $f \in \Bool$, such that $f$ is \emph{not} essentially constant on vertical lines.
Applying the proof of \cref{lessergoal} yields $\overline{f}$, such that
	\begin{itemize}
	\item $G \, \overline{f} \subseteq \Bool$, so $\overline{f} \in \Bool_G$,
	and
	\item $\overline{f}$ is \emph{not} essentially constant on vertical lines.
	\end{itemize}
This contradicts \pref{BoolG=ConstVertEq}.
\end{proof}



Very similar ideas yield the general case of \cref{blackbox}, if one is familiar with real roots and parabolic subgroups. To illustrate this, without using extensive Lie-theoretic language, let us explicitly describe the setup for $G = \SL(3,\real)$.

\subsection*{\hskip-\parindent Modifications for $\SL(3,\real)$.}
\noprelistbreak
\begin{itemize}
\smallskip
\item $P = \begin{Smallbmatrix} \upast&& \\ \upast&\upast& \\ \upast&\upast&\upast \end{Smallbmatrix} $,
	\quad
	$V = \begin{Smallbmatrix} 1&\upast&\upast \\ &1&\upast \\ &&1 \end{Smallbmatrix}$,
	\quad
	$V_1 = \begin{Smallbmatrix} 1&\upast& \\ &1& \\ &&1 \end{Smallbmatrix}$,
	\quad
	$V_2 = \begin{Smallbmatrix} 1&& \\ &1&\upast \\ &&1 \end{Smallbmatrix}$.
\\[3pt] Note that $V = \langle V_1, V_2 \rangle$.

\item There are exactly four subgroups containing $P$, namely,
$$ \text{$P$, \quad $G$, \quad $P_1 =
 \begin{Smallbmatrix} \vphantom{1}\upast&\upast& \\ \vphantom{1}\upast&\upast& \\ \vphantom{1}\upast&\upast&\upast \end{Smallbmatrix} = \langle V_1,  P \rangle$, \quad 
%\quad and \quad
 $P_2 = \begin{Smallbmatrix} \vphantom{1}\upast&& \\ \vphantom{1}\upast&\upast&\upast \\ \vphantom{1}\upast&\upast&\upast \end{Smallbmatrix} = \langle V_2, P \rangle$.} $$
Hence, there are precisely four $G$-invariant subalgebras of $\Bool(G/P)$. Namely, if we identify $\Bool(G/P)$ with $\Bool(V)$, then the $G$-invariant subalgebras of $\Bool(V)$ are 
\begin{itemize}
\item $\Bool(V)$, 
\item $\{0,1\}$, 
\item right $V_1$-invariant functions,
\item right $V_2$-invariant functions.
\end{itemize}

\begin{rem} \label{GeomInterpG/Pi}
The homogeneous spaces $G/P_1$ and $G/P_2$ are $\RP2$ and the Grassmannian $\Grass23$ of $2$-planes in~$\real^3$ \csee{GeomInterpG/PiEx}. Hence, in geometric terms, the $G$-invariant Boolean subalgebras of $\Bool(G/P)$ are 
	 $\Bool(G/P)$, $\{0,1\}$, $\Bool(\RP2)$, and $\Bool(\Grass23)$.
\end{rem}

\item Let $\pi_2$ be the projection onto~$V_2$ in the natural semidirect product 
 $V = V_2 \ltimes V_2^\perp$, where $V_2^\perp = \begin{Smallbmatrix} 1&&\upast \\ &1&\upast \\ &&1 \end{Smallbmatrix}$. 

\item For $a = \begin{Smallbmatrix} k && \\ & k & \\ &&1/k^2 \end{Smallbmatrix} \in G$, \cref{ADilateInSL3Ex} tells us
	 \begin{equation} \label{ADilateInSL3}
	 a \begin{bmatrix} 1&x&z \\  &1&y \\ &&1 \end{bmatrix} P
	= \begin{bmatrix} 1&x&k^3 z \\  &1&k^3 y \\ &&1 \end{bmatrix} P 
	. \end{equation}
	\item A generalization of the Lebesgue Differentiation Theorem tells us, for $f \in \Bool(G/P) = \Bool(V)$ and a.e.\ $v \in V_2^\perp$, that
	 $$ \text{$a^n v f$ converges in measure to $(vf) \circ \pi_2$.} $$

\end{itemize}
With these facts in hand, it is not difficult to prove \cref{blackbox'} under the assumption that $G = \SL(3,\real)$ \csee{BlackBoxSL3PfEx}. 





\begin{exercises}

\item \label{B(Z)InjectsEx}
In the setting of \cref{blackbox}, define $\psi^* \colon \Bool(Z) \to \Bool(G/P)$ by $\psi^*(f) = f \circ \psi$.
Show that $\psi^*$ is injective and $\Gamma$-equivariant.
\hint{Injectivity relies on the fact that $\psi$ is measure-class preserving.}

%\item \label{B(Z)ClosedBoolean}
%In the setting of \cref{blackbox}, show that $\psi^* \bigl( \Bool(Z) \bigr)$ is a $\Gamma$-invariant sub-$\sigma$-algebra of $\Bool(G/P)$.

\item \label{ZGInvt->GQuot}
In the setting of \cref{blackbox}, show that if the sub-$\sigma$-algebra $\psi^* \bigl( \Bool(Z) \bigr)$ of $\Bool(G/P)$ is $G$-invariant, then $Z$ is a $G$-equivariant quotient of $G/P$ (a.e.).
\hint{To reduce problems of measurability, you may pretend that $G$ is countable. More precisely, use \cref{BoolPtwise} to show that if $H$ is any countable subgroup of~$G$ that contains~$\Gamma$, then the $\Gamma$-action can be extended to an action of~$H$ on~$Z$ by Borel maps, such that, for each $h \in H$, we have $\psi(hx) = h \, \psi(x)$ for a.e.\ $x \in G/P$.}

\item \label{GiActsByLinFracEx}
Let $G_i$ and~$P_i$ be as in \cref{BlackBoxPfNotn}.  Show that choosing appropriate coordinates on $\RP1 = \real \cup \{\infty\}$ identifies the action of $G_i$ on $G_i/P_i$ with the action of $G_i = \SL(2,\real)$ on $\real \cup \{\infty\}$ by linear-fractional transformations:
	$$ \begin{bmatrix} a & b \\ c & d \end{bmatrix} (x) = \frac{ax + b}{cx + d} .$$
In particular, 
	$$\begin{bmatrix} k & \\ & k^{-1} \end{bmatrix} (x) = k^2 x 
	\text{\qquad and\qquad}
	 \begin{bmatrix} 1 & t\\ & 1 \end{bmatrix} (x) = x + t . $$
\hint{Map a nonzero vector $(x_1,x_2) \in \real^2$ to its reciprocal slope $x_1/x_2 \in \real \cup \{\infty\}$.}

\item \label{Gi/Pi=ViEx}
Let $G_i$, $P_i$, and~$V_i$ be as in \cref{BlackBoxPfNotn}. 
Show that the map $V_i \to G_i/P_i \colon v \mapsto v P_i$ injective and measure-class preserving.
\hint{\Cref{GiActsByLinFracEx}.} 

\item \label{LebDiffThmEquivEx}
Show that \cref{LebDiffThmEq} is equivalent to
	$$ \lim_{k \to \infty} \frac{1}{\lambda \bigl( B_1(0) \bigr)} \int_{B_1(0)} f\left(p + \frac{x}{k}\right) \, d\lambda(x)= f(p) .$$
\hint{A change of variables maps $B_1(0)$ onto $B_r(p)$ with $r = 1/k$.}

\item \label{LebConvMeasToProjPfEx}
Prove \cref{LebConvMeasToProj}.
\hint{\Cref{LebDiffThmEquivEx}.}
 
 \item \label{a^nvGammaDenseEx}
 Complete the proof of \cref{GammaVADense}: assume, for a.e.\ $g \in G$, that $a^{\natural} g \Gamma$ is dense in~$G$, and show, for a.e.\ $v \in V_1$, that $a^{\natural} v \Gamma$ is dense in~$G$.
 \hint{If $a^{\natural} g \Gamma$ is dense, then the same is true when $g$ is replaced by any element of $\czer_G(a) \, U_1 \, g$.}

\item \label{ConstVert+Horiz->Const}
Let $f \in \Bool(\real^2)$. Show that if $f$ is essentially constant on a.e.\ vertical line and on a.e.\ horizontal line, then $f$~is constant (a.e.).
 
 \item \label{ParabolicsInSL2xSL2Ex}
Assume \cref{BlackBoxPfNotn}. Show that the only subgroups of~$G$ containing~$P$ are  $P$, $G_1\times P_2$, $P_1\times G_2$, and~$G$. 
\hint{$P$ is the stabilizer of a point in $\RP1 \times \RP1$, and has only $4$ orbits.} 

\item \label{EquivQuotsSL2xSL2/PEx}
Assume \cref{BlackBoxPfNotn}. 
Show that the only $G$-equivariant quotients of $G/P$ are
	$G/P$, $G_2/P_2$, $G_1/P_1$, and $G/G$.
\hint{\cref{ParabolicsInSL2xSL2Ex}.}

\item \label{WeakClosed->ConvMeasEx}
Suppose $\Bool$ is a sub-$\sigma$-algebra of $\Bool(G/P)$. Show that $\Bool$ is closed under \term{convergence in measure}. 
\par
More precisely, fix a probability measure $\mu$ in the Lebesgue measure class on $G/P$, and show that $\Bool$ is a closed in the topology corresponding to the metric on $\Bool(G/P)$ that is defined by $d(A_1,A_2) = \mu( A_1 \symmdiff A_2)$. 

\item \label{ContOnB(G/P)}
Show that the action of~$G$ on $\Bool(G/P)$ is continuous.
\hint{Suppose $g_n \to e$ and $\mu(A_n \symmdiff A) \to 0$. The Radon-Nikodym derivative $d (g_n)_*\mu/d\mu$ tends uniformly to~$1$, so $\mu(g_n A_n \symmdiff g_nA) \to 0$. To bound $\mu(g_n A \symmdiff A)$, note that $\int_{g_n A} \varphi \, d\mu \to \int_A \varphi d\mu$, for every $\varphi \in C_c(G/P)$.}

\item \label{GeomInterpG/PiEx}
In the notation of \cref{GeomInterpG/Pi}, show that $G/P_1$ and $G/P_2$ are $G$-equivariantly diffeomorphic to  $\RP2$ and~$\Grass23$, respectively.
\hint{Verify that the stabilizer of a point in $\RP2$ is~$P_1$, and the stabilizer of a point in $\Grass23$ is~$P_2$.}

\item \label{ADilateInSL3Ex}
Verify \cref{ADilateInSL3}.
\hint{Since $a \in P$, we have $agP = (a g a^{-1}) P$, for any $g \in G$.}

\item \label{BlackBoxSL3PfEx}
Prove \cref{blackbox'} under the assumption that $G = \SL(3,\real)$.
\hint{You may assume (without proof) the facts stated in the ``Modifications for $\SL(3,\real)$\zz.''}
 
 \end{exercises}







\begin{notes}



The Normal Subgroups Theorem \pref{MargNormalSubgrpsThm} is due to G.\,A.\,Margulis \cite{MargulisFactorGroupsDoklady,MargulisFactorGroups,MargulisFactorGroupsRank1}.
Expositions of the proof appear in \cite[Chap.~4]{MargulisBook} and \cite[Chap.~8]{ZimmerBook}.
(However, the proof in \cite{ZimmerBook} assumes that $G$ has Kazhdan's property~$(T)$.) 

When $\Gamma$ is not cocompact, the Normal Subgroups Theorem can be proved by algebraic methods derived from the proof of the Congruence Subgroup Problem (see \cite[Thms.~A and~B, p.~109]{RaghunathanCSP} and \cite[Cor.~1, p.~75]{RaghunathanCSP2}). 
On the other hand, it seems that the ergodic-theoretic approach of Margulis provides the only known proof in the cocompact case.

Regarding \fullcref{MargNormSubgrpRems}{CSP}, see 
	%\cite[Chap.~6]{Humphreys-ArithmeticGroups} or 
\cite{Sury-CSPBook} for an introduction to the Congruence Subgroup Property. 

\Cref{blackbox} is stated for general $G$ of real rank $\ge 2$ in \cite[Cor.~2.13]{MargulisBook} and \cite[Thm.~8.1.4]{ZimmerBook}.
\Cref{blackbox'} is in \cite[Thm.~4.2.11]{MargulisBook} and \cite[Thm.~8.1.3]{ZimmerBook}. See \cite[\S8.2 and \S8.3]{ZimmerBook} and \cite[\S4.2]{MargulisBook} for expositions of the proof.

The proof of \cref{Rrank1->GammaNotAlmSimple} is adapted from \cite[5.5.F, pp.~150--152]{Gromov-HypGrp}.

The Higman-Neumann-Neumann Theorem on SQ-universality of~$\free_2$ (see p.~\pageref{HNNThm}) was proved in \cite{HNN-F2IsSQUniv}. A very general version of \cref{Rank1SQUniv} that applies to all relatively hyperbolic groups was proved in \cite{HyperIsSQUniv}.
(The notion of a \index{hyperbolic!group, relatively}relatively hyperbolic group was introduced in~\cite{Gromov-HypGrp}, and generalized in \cite{Farb-RelHypGrp}.)

\end{notes}


\begin{references}{99}

\bibitem{HyperIsSQUniv}
G.\,Arzhantseva, A.\,Minasyan, and D.\,Osin:
 The SQ-universality and residual properties of relatively hyperbolic groups,
\emph{J. Algebra} 315 (2007), no.~1, pp.~165--177.
\MR{2344339},
\maynewline
\url{http://dx.doi.org/10.1016/j.jalgebra.2007.04.029}

\bibitem{Farb-RelHypGrp}
B.\,Farb:
Relatively hyperbolic groups,
\emph{Geom. Funct. Anal.} 8 (1998), no.~5, 810--840. 
\MR{1650094},
\maynewline
\url{http://dx.doi.org/10.1007/s000390050075}

\bibitem{Gromov-HypGrp}
M.\,Gromov:
 Hyperbolic groups, in:
S.\,M.\,Gersten, ed., \emph{Essays in Group Theory.}
%Math. Sci. Res. Inst. Publ., 8, 
 Springer, New York, 1987,
pp.~75--263.
 ISBN 0-387-96618-8,
\MR{0919829}

\bibitem{HNN-F2IsSQUniv}
G.\,Higman, B.\,H.\,Neumann and H.\,Neumann:
 Embedding theorems for groups, 
\emph{J. London Math. Soc.} 24 (1949), 247--254.
\MR{0032641},
\maynewline
\url{http://dx.doi.org/10.1112/jlms/s1-24.4.247}

\bibitem{MargulisFactorGroupsDoklady}
 G.\,A.\,Margulis:
Factor groups of discrete subgroups,
 \emph{Soviet Math. Doklady} 19 (1978), no.~5, 1145--1149 (1979).
\MR{0507138} % no URL available @@@

\bibitem{MargulisFactorGroups}
 G.\,A.\,Margulis:
Quotient groups of discrete subgroups and measure theory,
 \emph{Func. Anal. Appl.} 12 (1978), no.~4, 295--305 (1979).
\MR{0515630},
\maynewline
\url{http://dx.doi.org/10.1007/BF01076383}

\bibitem{MargulisFactorGroupsRank1}
 G.\,A.\,Margulis:
Finiteness of quotient groups of discrete subgroups,
 \emph{Func. Anal. Appl.}  13  (1979), no. 3, 178--187 (1979).
\MR{0545365},
\maynewline
\url{http://dx.doi.org/10.1007/BF01077485}

\bibitem{MargulisBook}
 G.\,A.\,Margulis:
 \emph{Discrete Subgroups of Semisimple Lie Groups.}
 Springer, {Berlin Heidelberg New York}, 1991.
ISBN 3-540-12179-X,
\MR{1090825}

\bibitem{RaghunathanCSP}
M.\,S.\,Raghunathan:
 On the congruence subgroup problem,
 \emph{Publ. Math. IHES} 46 (1976) 107--161.
\MR{0507030}
\url{http://www.numdam.org/item?id=PMIHES_1976__46__107_0}

\bibitem{RaghunathanCSP2}
M.\,S.\,Raghunathan:
 On the congruence subgroup problem, II,
 \emph{Invent. Math.} 85 (1986), no.~1, 73--117.
\MR{0842049},
\maynewline
\url{http://eudml.org/doc/143360}
%\url{http://www.digizeitschriften.de/dms/resolveppn/?PPN=GDZPPN002102978}

\bibitem{Sury-CSPBook}
B.\,Sury:
\emph{The Congruence Subgroup Problem}. 
 %An elementary approach aimed at applications. Texts and Readings in Mathematics, 24. 
 Hindustan Book Agency, New Delhi, 2003.
 ISBN 81-85931-38-0,
\MR{1978430}

\bibitem{ZimmerBook}
Robert J.\ Zimmer:
 \emph{Ergodic Theory and Semisimple Groups}.
%Monographs in Mathematics, 81. 
Birkh\"auser, Basel, 1984.
ISBN 3-7643-3184-4,
\MR{0776417}


\end{references}
