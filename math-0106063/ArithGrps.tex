%!TEX root = IntroArithGrps.tex

\mychapter{\texorpdfstring{What is an\\Arithmetic Group?}%
	{What is an Arithmetic Group?}}
\label{ArithGrpsChap}

\prereqs{\Cref{LatticeDefn,CommensDefn} (lattice subgroups and commensurability).}

$\SL(n,\integer)$ is the most basic example of an ``arithmetic group\zz.'' We will see that, by definition, the other arithmetic groups are obtained by intersecting $\SL(n,\integer)$ with some semisimple subgroup~$G$ of $\SL(n,\real)$. More precisely, if $G$ is a subgroup of $\SL(n,\real)$ that satisfies certain technical conditions (to be explained in \cref{ArithLattDefnSect}), then $G \cap \SL(n,\integer)$ (the group of ``integer points'' of~$G$) is said to be an \defit[arithmetic!subgroup]{arithmetic subgroup} of~$G$. However, the official definition \pref{ArithDefn} also allows certain modifications of this subgroup to be called \defit[arithmetic!subgroup]{arithmetic}.

Different embeddings of~$G$ into $\SL(n,\real)$ can yield different intersections with $\SL(n,\integer)$, so $G$ has many different arithmetic subgroups. 
(Examples can be found in \cref{EgArithGrpsChap}.) 
\Cref{arith->latt} tells us that all of them are lattices in~$G$. In particular, $\SL(n,\integer)$ is a lattice in $\SL(n,\real)$.




\section{Definition of arithmetic subgroups} \label{ArithLattDefnSect}

We are assuming that $G$ is a subgroup of $\SL(\ell,\real)$ \csee{standassump}, and we are interested in $\Gamma = G \cap \SL(\ell,\integer)$, the set of 
``integer points'' of~$G$. However, in order for the integer
points to form a lattice, $G$~needs to be well-placed with
respect to $\SL(\ell,\integer)$. (If we replace $G$ by a
conjugate under some terrible irrational matrix, perhaps $G
\cap \SL(\ell,\integer)$ would become trivial
\csee{conjnoZ}.) The following \lcnamecref{RnDefdQ<>Latt} is an elementary
illustration of this idea.

\begin{prop} \label{RnDefdQ<>Latt}
 The following are
equivalent, for every subspace~$W$ of\/~$\real^\ell$:
 \begin{enumerate}
 \item \label{RnDefdQ<>Latt-latt}
 $W \cap \integer^\ell$ is a cocompact lattice in~$W$.
 \item \label{RnDefdQ<>Latt-span}
 $W$ is spanned by $W \cap \integer^\ell$.
 \item \label{RnDefdQ<>Latt-dense}
 $W \cap \rational^\ell$ is dense in~$W$.
 \item \label{RnDefdQ<>Latt-eqs}
 $W$ can be defined by a set of linear equations with coefficients in~$\rational$.
\end{enumerate}
 \end{prop}

\begin{proof}
Let $k = \dim W$.

($\ref{RnDefdQ<>Latt-latt} \Rightarrow
\ref{RnDefdQ<>Latt-span}$) Let $V$ be the $\real$-span of $W
\cap \integer^\ell$. Then $W/V$, being a vector space
over~$\real$, is homeomorphic to~$\real^d$, for some~$d$. On
the other hand, we know that $W \cap \integer^\ell \subset
V$, and that $W/(W \cap \integer^\ell)$ is compact, so $W/V$
is compact. Hence $d = 0$, so $V = W$.

($\ref{RnDefdQ<>Latt-span} \Rightarrow \ref{RnDefdQ<>Latt-latt}$) 
Let $\{\varepsilon_1,\ldots,\varepsilon_k\}$ be the standard
basis of~$\real^k$. 
Because $W \cap \integer^\ell$ contains a basis of~$W$, there is a
linear isomorphism $T \colon \real^k \to W$, such that
$T\bigl( \{\varepsilon_1,\ldots,\varepsilon_k\} \bigr) \subseteq W \cap \integer^\ell$. This implies that $T(\integer^k) \subseteq W \cap
\integer^\ell$. Since $\real^k/\integer^k$ is compact, and $T$~is continuous, we
conclude that $W/(W \cap \integer^\ell)$ is compact.

 ($\ref{RnDefdQ<>Latt-span} \Rightarrow \ref{RnDefdQ<>Latt-dense}$) 
 As in the proof of ($\ref{RnDefdQ<>Latt-span} \Rightarrow \ref{RnDefdQ<>Latt-latt}$), 
there is a linear isomorphism $T \colon \real^k \to W$, such that
$T(\integer^k) \subseteq W \cap \integer^\ell$. 
Then $T(\rational^k) \subseteq W \cap
\rational^\ell$. Since $\rational^k$ is dense in~$\real^k$,
and $T$~is continuous, we conclude that $T(\rational^k)$ is
dense in~$W$.

 ($\ref{RnDefdQ<>Latt-eqs} \Rightarrow
\ref{RnDefdQ<>Latt-span}$) By assumption, $W$~is the
solution space of a system of linear equations whose
coefficients belong to~$\rational$. (Since $\real^\ell$ is 
finite dimensional, only finitely many of the equations are 
necessary.) Therefore, by elementary linear algebra (row 
reductions), we may find a basis for~$W$ that
consists entirely of vectors in~$\rational^\ell$.
Multiplying by a scalar to clear the denominators, we may
assume that the basis consists entirely of vectors
in~$\integer^\ell$.

 ($\ref{RnDefdQ<>Latt-dense} \Rightarrow
\ref{RnDefdQ<>Latt-eqs}$) 
Since $W \cap \rational^\ell$ is
dense in~$W$, we know that the orthogonal complement $W^\perp$ is defined by a set
of linear equations with rational coefficients. (For each 
$w \in W \cap \rational^\ell$, we write the equation 
$w \cdot x = 0$.) Thus, from 
($\ref{RnDefdQ<>Latt-eqs} \Rightarrow
\ref{RnDefdQ<>Latt-span}$), we conclude that there is a
basis $v_1,\ldots,v_m$ of~$W^\perp$, such that each $v_j \in
\rational^\ell$. Then $W = (W^\perp)^\perp$ is defined by the
system of equations $v_1 \cdot x= 0$, \dots, $v_m \cdot x =
0$.
 \end{proof}

With the above \lcnamecref{RnDefdQ<>Latt} in mind, we make the following
definition.

% cref won't combine these references, because one is defn and the other is defns !!!
\begin{defn}[(cf.\ Definitions \ref{AlgicGrpDefn} and~\ref{AlmZarDefn})] \label{DefdQDefn}
 Let $H$ be a closed subgroup of $\SL(\ell,\real)$. We say
 $H$~is \index{algebraic!group!over Q@over~$\rational$}\defit[defined!over Q@over~$\rational$]{defined over\/~$\rational$} (or that $H$ is a
\defit[Q-@$\rational$-!subgroup]{$\rational$-subgroup\/}) if there is a
subset~$\mathcal{Q}$ of $\rational[x_{1,1}, \ldots,
x_{\ell,\ell}]$, such that 
 \begin{itemize}
 \item \nindex{$\Var(\mathcal{Q})$ = $\{\, g \in \SL(\ell,\real) \mid
Q(g) = 0, \ \forall Q \in \mathcal{Q} \,\}$}
 $\Var(\mathcal{Q}) = \{\, g \in \SL(\ell,\real) \mid
Q(g) = 0, \ \forall Q \in \mathcal{Q} \,\}$ is a subgroup of
$\SL(\ell,\real)$,
 \item $H^\circ = \Var(\mathcal{Q})^\circ$, and
 \item $H$ has only finitely many components.
 \end{itemize}
 In other words, $H$~is commensurable to the variety
$\Var(\mathcal{Q})$, for some set~$\mathcal{Q}$ of
$\rational$-polynomials.
 \end{defn}

\begin{egs} \ 
\noprelistbreak
 \begin{enumerate}
 \item $\SL(\ell,\real)$ is defined over~$\rational$: let
$\mathcal{Q} = \emptyset$.
 \item If $n < \ell$, we may embed $\SL(n,\real)$ in the top
left corner of $\SL(\ell,\real)$. This copy of $\SL(n,\real)$ is defined over~$\rational$: let
 $$ \mathcal{Q} = 
 \{\, x_{i,j} - \delta_i^j \mid \max\{i,j\} > n \,\} .$$
 \item For $A \in \SL(\ell,\rational)$, 
 the group 
 $\SO_\ell(A; \real) = \{ g \in \SL(\ell,\real) \mid g A g^T = A \}$ % @@@ should have \, 
 is defined over~$\rational$: let 
 $$ \mathcal{Q} =
 \bigset{
 \sum_{1\le p,q \le m+n}
 x_{i,p} A_{p,q} x_{j,q}
 - A_{i,j}
 }{1 \le i,j \le m+n}
 .$$
 In particular, $\SO(m,n)$, under its usual embedding in
$\SL(m+n,\real)$, is defined over~$\rational$.
 \item $\SL(n,\complex)$, under its usual embedding in
$\SL(2n,\real)$, is defined over~$\rational$
\fullccf{EgZarClosed}{SLnC}.
 \end{enumerate}
 \end{egs}

\begin{rems} \label{DefQRems}\ 
\noprelistbreak
 \begin{enumerate}
 \item
 There is always a subset~$\mathcal{Q}$ of $\real[x_{1,1},
\ldots, x_{\ell,\ell}]$, such that $G$ is commensurable to
$\Var(\mathcal{Q})$ \csee{GisAlgic}; that is, $G$~is
\defit[defined!over R@over~$\real$]{defined
over\/~$\real$}. However, it may not be possible to find a
set~$\mathcal{Q}$ that consists entirely of polynomials
whose coefficients are rational, so $G$ may not be defined
over~$\rational$.
 \item If $G$ is defined over~$\rational$, then the
set~$\mathcal{Q}$ of \cref{DefdQDefn} can be chosen to
be finite (because the ring $\rational[x_{1,1}, \ldots,
x_{\ell,\ell}]$ is \term[Noetherian ring]{Noetherian}). 
 \end{enumerate}
 \end{rems}

\begin{prop} \label{hasQform}
 $G$ is isogenous to a group that is defined over\/~$\rational$.
 \end{prop}

\begin{proof}
 It is easy to handle direct products, so the crucial case
is when $G$ is simple. This is easy if $G$ is classical.
Indeed, the groups in \cref{classical-fulllinear,classical-orthogonal} are defined over~$\rational$
(after identifying $\SL(\ell,\complex)$ and
$\SL(\ell,\quaternion)$ with appropriate subgroups of
$\SL(2\ell,\real)$ and $\SL(4\ell,\real)$, in a natural
way). 

The general case is not difficult for someone familiar with
exceptional groups. Namely, since $\Ad G$ is a
finite-index subgroup of $\Aut(\Lie G)$,
it suffices to find a basis of~$\Lie G$, for which the
structure constants of the Lie algebra are rational. We omit the details.
 \end{proof}

\begin{notation} 
For each subring~$\ints$
of~$\real$ (containing~$1$), we construct
\nindex{$G_A$ = $G \cap \SL(n,A)$}$G_{\ints} = G \cap
\SL(n,\ints)$. That is, $G_\ints$ is the subgroup consisting of
the elements of~$G$ whose matrix entries all belong to~$\ints$. 
%(The same notation can be applied when $G \hookrightarrow \SL(\ell,\complex)$, and $A$~is any subring of~$\complex$.)
 \end{notation}

%\begin{notation} \nindex{$G_A$ = $G \cap \SL(n,A)$}
%If we have an embedding of~$G$ in some $\SL(n,\complex)$, and
% $A$~is any subring
%of~$\complex$ (containing~$1$), we let $G_{A} = G \cap
%\SL(n,A)$. That is, $G_A$ is the subgroup consisting of
%the elements of~$G$ whose matrix entries all belong to~$A$. 
% \end{notation}

\begin{eg} \label{SLnCQ}
 Let $\phi \colon \SL(n,\complex) \to \SL(2n,\real)$ be the
natural embedding. Then
 $$ \phi \bigl( \SL(n,\complex) \bigr)_{\rational} 
 = \phi \bigl( \SL(n,\rational[i]) \bigr) .$$
 Therefore, if we think of $\SL(n,\complex)$ as a Lie group
over~$\real$, then $\SL(n,\rational[i])$ represents the
``$\rational$-points'' of $\SL(n,\complex)$.
 \end{eg}

The following result provides an alternate point of view on
being defined over~$\rational$.
It is the nonabelian version of  ($\ref{RnDefdQ<>Latt-dense} \Leftrightarrow \ref{RnDefdQ<>Latt-eqs}$) of \cref{RnDefdQ<>Latt}.

\begin{prop} \label{QptsDense}
 Let $H$ be a connected subgroup of\/ $\SL(\ell,\real)$ that
is almost Zariski closed. The group $H$ is defined
over\/~$\rational$ if and only if $H_{\rational}$ is dense
in~$H$.
 \end{prop}

\begin{proof}
($\Leftarrow$) Let $\mathcal{Q}_\complex = 
	\{\, Q \in \complex[x_{1,1},\ldots,x_{\ell,\ell}] \mid Q(h) = 0, \ \forall h \in H \,\}$. Also, for $d \in \natural$, let $\mathcal{Q}_\complex^d = \{\, Q \in \mathcal{Q}_\complex \mid \deg Q \le d \,\}$. Since $H_\rational$ is dense in~$H$ (and polynomials are continuous), it is clear that $\mathcal{Q}_\complex^d$ is invariant under the Galois group $\Gal(\complex/\rational)$, so it is not difficult to see that $\mathcal{Q}_\complex^d$ is spanned (as a vector space over~$\complex$) by a collection $\mathcal{Q}^d$ of polynomials with rational coefficients \csee{InvtIffDefdQ}. Since $H$ is almost Zariski closed, and polynomial rings are Noetherian, we have $H^\circ = \Var(\mathcal{Q}^d)\!^\circ$ for $d$ sufficiently large. The polynomials in $\mathcal{Q}^d$ all have rational coefficients, so this implies that $H$ is defined over~$\rational$.

($\Rightarrow$)  See \cref{QDenseGSimple} for a proof when $G$ is simple and $G/G_\integer$ is not compact. The general case utilizes a fact from the theory of algebraic groups that will not be proved in this book \csee{GuniratSoQDense}. 
\end{proof}

\begin{warn} \label{HQnotdense}
 \Cref{QptsDense} requires the assumption
that $H$ is connected; there are subgroups~$H$ of
$\SL(\ell,\real)$, such that $H$ is defined
over~$\rational$, but $H_{\rational}$ is not dense in~$H$.
For example, let 
 $$ H = \{\, h \in \SO(2) \mid h^8 = \Id \,\} .$$
 \end{warn}
 
 \begin{rem} \label{JacobsonMorosovOverF}
The \thmindex{Jacobson-Morosov}{Jacobson-Morosov Lemma}~\pref{JacobsonMorosov}
has a relative version: if $G$~is defined over~$\rational$, and $u$ is a nontrivial, unipotent element of $G_\rational$, then there is a (polynomial)
homomorphism $\phi \colon \SL(2,\real) \to G$, such that
 $\phi 
\left( \left[ \begin{smallmatrix}
 1& 1 \\
 0 & 1
 \end{smallmatrix}
 \right] \right)
 = u$
 and
%$\phi$ is defined over~$F$
$\phi \bigl( \SL(2,\rational) \bigr) \subseteq G_\rational$.
\end{rem}


We now state a theorem of fundamental importance in the theory of lattices and
arithmetic groups. It is a nonabelian analogue of the obvious fact that $\integer^\ell$ is a lattice in~$\real^\ell$, and of ($\ref{RnDefdQ<>Latt-eqs} \Rightarrow \ref{RnDefdQ<>Latt-latt}$) of \cref{RnDefdQ<>Latt}.

\begin{major} \label{arith->latt}
	\thmindex{arithmetic subgroups are lattices}%
 If $G$ is defined over~$\rational$, then $G_{\integer}$ is
a lattice in~$G$.
 \end{major}

\begin{proof}
The statement of this theorem is more important than its proof, so, for
most purposes, the reader could accept this fact as an axiom, without 
learning the proof.\label{arith->lattNotPf}
For those who do not want to take this on faith, a discussion of two different proofs 
can be found in \cref{SLnZLattChap} (with some additional details in \cref{ReductionChap}).
\end{proof}


\begin{eg} \label{ArithLattEg}
 Here are some standard cases of
\cref{arith->latt}.%
\noprelistbreak
 \begin{enumerate}
 \item $\SL(2,\integer)$ is a lattice in $\SL(2,\real)$. 
 (We proved this in \cref{SL2Zlatt}.)
 \item \label{ArithLattEg-SLnZ}
$\SL(n,\integer)$ is a lattice in $\SL(n,\real)$.
  (We will prove this in \cref{SLnZLattChap}.) 

 \item $\SO(m,n)_{\integer}$ is a lattice in $\SO(m,n)$.
 \item $\SL(n, \integer[i])$ is a lattice in
$\SL(n,\complex)$ \ccf{SLnCQ}.
 \end{enumerate}
 \end{eg}

\begin{eg}
 As an additional example, let 
 	$$G = \SO(7 x_1^2 - x_2^2 - x_3^2; \real) \iso \SO(1,2). $$
Then \cref{arith->latt}
implies that $G_{\integer}$ is a lattice in~$G$. This
illustrates that the \lcnamecref{arith->latt} is a highly nontrivial result.
For example, in this case, it may not even be obvious to the
reader that $G_{\integer}$ is infinite. 
 \end{eg}

\begin{warn}
 \Cref{arith->latt} requires our standing assumption
that $G$ is semisimple; there are subgroups~$H$ of
$\SL(\ell,\real)$, such that $H$ is defined
over~$\rational$, but $H_{\integer}$ is not a lattice in~$H$.
For example, if $H$ is the group of diagonal matrices in
$\SL(2,\real)$, then $H_{\integer}$ is finite, not a lattice in~$H$.
 \end{warn}

 \begin{rem} The converse of \cref{arith->latt} holds
when $G$ has no compact factors \csee{arithlatt->defdQ}.
 \end{rem}

Combining \cref{hasQform} with \cref{arith->latt}
yields the following important conclusion:

\begin{cor} \label{GHasLatt}
 $G$ has a lattice.
 \end{cor}

 In fact, a more careful look at the proof shows that if $G$ is not compact, then the lattice we constructed is not cocompact:

\begin{cor} \label{GHasNoncpctLatt}
If $G$ is not compact, then $G$ has a noncocompact lattice.
\end{cor}

\begin{proof}
Assume that $G$ is classical, which means it is one of the groups listed in \cref{classical-fulllinear,classical-orthogonal}. As was mentioned in the proof of \cref{hasQform}, each of these groups has an obvious $\rational$-form $G_\rational$, obtained by replacing $\real$ with~$\rational$ (or replacing $\complex$ with $\rational[i]$), in a natural way.
Whenever $G$ is noncompact, it is not difficult to see that $G_\rational$ has a nontrivial unipotent element \csee{UnipInClassicalQ}, so \cref{GammaUnip->notcpct} tells us that $G/G_\integer$ is not compact. 
\end{proof}

\begin{rem}
We will show in \cref{GHasCpctLatt} that $G$ also has a cocompact lattice, and a special case that illustrates the main idea of the proof will be seen much earlier, in \cref{SO(12;Z[sqrt2])}.
\end{rem}

A lattice of the form~$G_{\integer}$ is said to be
\emph{arithmetic}.  However, for the following reasons, a
somewhat more general class of lattices is also said to be
arithmetic. The idea is that there are some obvious
modifications of~$G_{\integer}$ that are also lattices, and
any subgroup that is obviously a lattice should be called
arithmetic.
 \begin{itemize}
 \item If $\phi \colon G_1 \to G_2$ is an isomorphism, and
$\Gamma_1$ is an arithmetic subgroup of~$G_1$, then we wish
to be able to say that $\phi(\Gamma_1)$ is an arithmetic
subgroup in~$G_2$.
 \item We wish to ignore compact
groups; that is, modding out a compact subgroup should not
affect arithmeticity. So we wish to be able to say that if
$K$ is a compact normal subgroup of~$G$, and $\Gamma$ is a
lattice in~$G$, then $\Gamma$ is arithmetic if and only if
$\Gamma K/K$ is an arithmetic subgroup of~$G/K$
 \item Arithmeticity should be independent of
commensurability. 
 \end{itemize}
 The following formal definition implements these
considerations.

\begin{defn} \label{ArithDefn}
 $\Gamma$ is an \defit[arithmetic!subgroup]{arithmetic} subgroup of~$G$ if and
only if there exist
 \begin{itemize}
 \item a closed, connected, semisimple subgroup~$G'$ of
some $\SL(n,\real)$, such that $G'$ is defined
over~$\rational$,
 \item compact normal subgroups $K$ and~$K'$ of $G^\circ$
and~$G'$, respectively, and
 \item an isomorphism $\phi \colon G^\circ/K \to G'/K'$, 
 \end{itemize}
 such that $\phi(\overline{\Gamma})$ is commensurable to
$\overline{G'_{\integer}}$, where $\overline{\Gamma}$
and~$\overline{G'_{\integer}}$ are the images of $\Gamma \cap G^\circ$
and~$G'_{\integer}$ in $G^\circ/K$ and $G'/K'$, respectively.
 \end{defn}

\begin{rems} \label{ArithDefnRem} \ 
\noprelistbreak
	\begin{enumerate}
	\item \label{ArithDefnRem-noK}
	If $G$ has no compact factors, then it is obvious that the subgroup~$K$ in \cref{ArithDefn} must be finite. 

	\item \Cref{IrredNoncpct->NoROS} will show that if $G/\Gamma$ is not compact (and $\Gamma$ is irreducible), then the annoying compact subgroups are not needed in \cref{ArithDefn}.

	\item On the other hand, if $\Gamma$ is cocompact, then a nontrivial
(connected) compact group~$K'$ may be required (even if $G$
has no compact factors). We will see many examples of this
phenomenon, starting with \cref{SO(12;Z[sqrt2])}.

	\item  Up to conjugacy, there are only countably many arithmetic
lattices in~$G$, because there are only countably many
finite subsets of the polynomial ring $\rational[x_{1,1},\ldots,x_{\ell,\ell}]$.

	\end{enumerate}
 \end{rems}

\begin{terminology}
 Our definition of \defit[arithmetic!subgroup]{arithmetic subgroup} assumes the perspective of Lie theory, where $\Gamma$ is assumed
to be embedded in some Lie group~$G$. The theory of algebraic groups
has a more strict definition, which requires
$\Gamma$ to be commensurable to~$G_{\integer}$: arbitrary
isomorphisms are not allowed, and compact subgroups cannot
be ignored. At the other extreme, abstract group theory 
has a much looser definition, which completely ignores~$G$:
if an abstract group~$\Lambda$ is abstractly commensurable to a
group that is arithmetic in our sense, then $\Lambda$~is considered to be arithmetic.
 \end{terminology}

\begin{exercises}

\item \label{conjnoZ}
 Show that if
 $G$ is connected, 
 $G \subseteq \SL(\ell,\real)$,
 and
 $-\Id \notin G$,
 then there exists $h \in \SL(\ell,\real)$, such that
$(h^{-1} G h) \cap \SL(\ell,\integer)$ is trivial.
 \hint{For each nontrivial $\gamma \in \SL(\ell,\integer)$,
let 
 \ $ X_\gamma = \{\, h \in \SL(\ell,\real) \mid h \gamma
h^{-1} \in G \,\} $. \ 
 Then each $X_\gamma$ is nowhere dense in $\SL(\ell,\real)$
\fullcsee{VarClosed}{nodense}.}

\item \label{InvtIffDefdQ}
Let $W$ be a vector subspace of~$\complex^n$, for some~$n$. Show that $W$ is invariant under $\Gal(\complex/\rational)$ if and only if $W$~is spanned by a set of vectors with rational coordinates. 
\hint{($\Rightarrow$) Choose $w \in W \smallsetminus \{0\}$ with a minimal number of nonzero coordinates, and multiply by a scalar to assume at least one coordinate is a nonzero rational. Since $\sigma(w) - w \in W$ for all $\sigma \in \Gal(\complex/\rational)$, the minimality implies $w \in \rational^n$. Mod out $w$ and induct on the dimension.}

\item \label{GuniratSoQDense} 
It can be shown that $G^\circ$ is \defit{unirational}. This means there exists an open subset~$U$ of some~$\real^n$, and a function $f \colon U \to G^\circ$, such that 
	\begin{itemize}
	\item $f(U)$ contains an open subset of~$G$, 
	and
	\item each matrix entry of $f(x)$ is a rational function of~$x$ (that is, a quotient of two polynomials).
	\end{itemize}
Furthermore, if $G$ is defined over~$\rational$, then $f$ can be chosen to be defined over~$\rational$ (that is, all of the coefficients of~$f$ are in~$\rational$).

Assuming the above, show that $G_\rational$ is dense in~$G$ if $G$~is connected and $G$~is defined over~$\rational$.
\hint{Unirationality implies that $\closure{G_\rational}$ contains an open subset of~$G$.}

\item \label{HQnotdenseEx}
 For $H$ as in \cref{HQnotdense}, show that
$H_{\rational}$ is not dense in~$H$.
\hint{$H$ is finite, and $H_\rational \neq H$.}

\item \label{arithlatt->defdQ}
 Show that if $G \subseteq \SL(\ell,\real)$, $G$ has no
compact factors, and $G_{\integer}$ is a lattice in~$G$,
then $G$ is defined over~$\rational$. 
\hint{See the proof of \cref{QptsDense}($\Leftarrow$). % !!!
Since $G_{\integer}$ is a lattice in~$G$, the Borel Density Theorem \pref{BDT(Zardense)} implies that $\mathcal{Q}_\complex^d$ is invariant under the Galois group.} 

\item \label{GZzardense->latt}
 Show that if 
 \begin{itemize}
 \item $G \subseteq \SL(\ell,\real)$, and
 \item $G_{\integer}$ is Zariski dense in~$G$,
 \end{itemize}
 then $G_{\integer}$ is a lattice in~$G$.
 \hint{It suffices to show that $G$~is defined
over~$\rational$.}

\item \label{IntZarDense->EqualEx}
 Show that if 
 \begin{itemize}
 \item $G$ has no compact factors,
 \item $\Gamma_1$ and~$\Gamma_2$ are arithmetic subgroups
of~$G$, and
 \item $\Gamma_1 \cap \Gamma_2$ is Zariski dense in~$G$,
 \end{itemize}
 then $\Gamma_1$ is commensurable to~$\Gamma_2$.
 \hint{Suppose $\phi_j \colon G \to H_j$ is an isomorphism, such
that $\phi_j(\Gamma_j) = (H_j)_{\integer}$. Define $\phi
\colon G \to H_1 \times H_2$ by $\phi(g) = \bigl( \phi_1(g),
\phi_2(g) \bigr)$. 
 Then $\phi(G)_{\integer}
 = \phi(\Gamma_1 \cap \Gamma_2)$
 is Zariski dense in $\phi(G)$, so $\Gamma_1 \cap \Gamma_2$
is a lattice in~$G$ \csee{GZzardense->latt}.
A similar (but slightly more complicated) argument applies if $\phi_j \colon G \to H_j/K_j$, where $K_j$ is compact.}

\item \label{UnipInClassicalQ}
For each classical simple group~$G$ in \cref{classical-fulllinear,classical-orthogonal}, let $G_\rational$ be the subgroup obtained by replacing $\real$ with~$\rational$, $\complex$ with $\rational[i]$, or $\quaternion$ with $\quaternion_\rational = \rational + \rational i+ \rational j+ \rational k$, as appropriate. Show that if $G$ is not compact, then $G_\rational$ contains a nontrivial unipotent element.
\hint{Show that $G_\rational$ contains a copy of either $\SL(2,\rational)$, $\SO(1,2)_\rational$, or $\SU(1,1)_\rational$ \ccf{SL2RinG}.}

\end{exercises}


\section{Margulis Arithmeticity Theorem}

The following astonishing theorem shows that taking integer
points is usually the only way to make a lattice. (See \cref{MargArithPf}
for a sketch of the proof.)

\begin{thm}[(\thmindex{Margulis!Arithmeticity}{Margulis Arithmeticity Theorem})]
\label{MargulisArith}
 If
 \begin{itemize}
 \item $G$ is not isogenous to\/ $\SO(1,n) \times K$ or\/
$\SU(1,n) \times K$, for any compact group~$K$, and
 \item $\Gamma$ is irreducible,
 \end{itemize}
 then $\Gamma$ is arithmetic.
 \end{thm}

\begin{warn}
 Unfortunately,
 \begin{itemize}
 \item $\SL(2,\real)$ is isogenous to $\SO(1,2)$, and 
 \item $\SL(2,\complex)$ is isogenous to $\SO(1,3)$,
 \end{itemize}
 so the arithmeticity theorem says nothing about the lattices in these two
important groups.
 \end{warn}

\begin{rem} \label{MargArithThmCpctFacts}
 The conclusion of \cref{MargulisArith} can be
strengthened: the subgroup~$K$ of \cref{ArithDefn} can
be taken to be finite. More precisely, if $G$ and~$\Gamma$
are as in \cref{MargulisArith}, and $G$~is noncompact
and has trivial center, then there exist
 \begin{itemize}
 \item a closed, connected, semisimple subgroup~$G'$ of
some $\SL(\ell,\real)$, such that $G'$ is defined
over~$\rational$, and
 \item a surjective (continuous) homomorphism $\phi \colon
G' \to G$, 
 \end{itemize}
 such that
 \begin{enumerate}
 \item $\phi( G'_{\integer} )$ is commensurable
to~$\Gamma$; and
 \item the kernel of~$\phi$ is compact.
 \end{enumerate}
 \end{rem}

\begin{rems} \ 
\noprelistbreak
	\begin{enumerate}
	\item For any~$G$, it is possible to give a reasonably complete
description of the arithmetic subgroups of~$G$ (up to
conjugacy and commensurability).
Some examples are worked out in fair detail in \cref{EgArithGrpsChap}.
More generally, \cref{ArithLattsAreClassical} (or the table on \cpageref{IrredInG})
essentially provides a list of all the
irreducible arithmetic subgroups of almost all of the
classical groups. Thus, for most
groups, the Margulis Arithmeticity Theorem provides a
list of all the lattices in~$G$.

	\item Furthermore, knowing that $\Gamma$ is arithmetic provides a
foothold to use algebraic and number-theoretic techniques to
explore the detailed structure of~$\Gamma$. For example, we saw that it is easy to show $\Gamma$ is torsion free if $\Gamma$ is
arithmetic \csee{torsionfree}. A more important example is that (apparently) the
only known proof that every lattice is finitely presented
\csee{GammaFinPres} relies on the Margulis Arithmeticity Theorem.

	\item It is known that there are nonarithmetic lattices in
$\SO(1,n)$ for every~$n$ \csee{NonarithInSO1n}, but we do
not yet have a theory that describes them all when $n \ge 3$.
 Also, nonarithmetic lattices have been constructed in
$\SU(1,n)$ for $n \in \{1,2,3\}$, but (apparently) it is
still not known whether they exist when $n \ge 4$.
	\end{enumerate}
 \end{rems}
 
 \begin{rem} \label{GQComm}
The subgroup \nindex{$\Comm_G(\Gamma)$ = commensurator of~$\Gamma$ in~$G$}
	$$\Comm_G(\Gamma) = \{\, g \in G \mid \text{$g \Gamma g^{-1}$ is commensurable to~$\Gamma$} \,\} $$
is called the \defit{commensurator} of~$\Gamma$ in~$G$. It is easy to see that if $G$ is defined over~$\rational$, 
then $G_{\rational} \subseteq \Comm_G(G_\integer)$
\ccf{SLQCommSLZ}. 
\begin{enumerate}

\item \label{GQComm-criterion}
This implies that if $\Gamma$ is arithmetic (and $G$ is connected, with no compact factors), then $\Comm_G(G_{\integer})$ is dense in~$G$ \csee{QptsDense}.
Margulis proved a converse. Namely, if $G$ is connected and has no compact factors, then
{\bf\mathversion{bold} $\Gamma$ is arithmetic iff\/ $\Comm_G(\Gamma)$ is dense in~$G$}
\csee{CommCriterion}.
This is known as the \thmindex{Commensurability Criterion for Arithmeticity}{Commensurability Criterion for Arithmeticity}.

\item \label{GQComm-bigger}
In some cases, the commensurator of $G_{\integer}$ is much larger than $G_{\rational}$ \csee{Comm(SL2Z)}. However, it was observed by Borel that this never happens when the ``\term{complexification}'' of~$G^\circ$ has trivial center (and other minor conditions are satisfied) \csee{Comm=GC}.
%(Intuitively, the \defit{complexification}~$G_\complex$ of~$G$ is the
%complex Lie group that is obtained from~$G$ by replacing
%real numbers with complex numbers. 
% For example,
%the complexification of $\SL(n,\real)$ is $\SL(n,\complex)$, and the complexification of $\SO(2,3)$ is (isomorphic to) $\SO(5)$.
(See \cref{RFormsOfCGrps} for an explanation of the complexification.)


\end{enumerate}
 \end{rem}


\begin{exercises}
\noprelistbreak
\item \label{Comm(SL2Z)}
Let $G = \SL(2,\real)$ and $G_\integer = \SL(2,\integer)$.
Show  $\Comm_G(G_{\integer})$ is \emph{not}
commensurable to~$G_{\rational}$.
\hint{The diagonal matrix $\diag \bigl( \sqrt{p} , \, 1/ \! \sqrt{p} \bigr)$
 commensurates $G_\integer$, for all $p \in \integer^+$.}

\item \label{Comm(SL3Z)}
Show $\Comm_{\SL(3,\real)} \bigl( \SL(3,\integer) \bigr)$
is \emph{not} commensurable to $\SL(3,\rational)$. 
% \hint{See \cref{Comm(SL2Z)}.}

\item \label{Comm(Gamma)discrete}
 Show that if $G$ is simple and $\Gamma$ is not arithmetic,
then $\Gamma$, $\nzer_G(\Gamma)$, and $\Comm_G(\Gamma)$ are
commensurable to each other.

\item \label{Comm=GC}
(\emph{requires some knowledge of algebraic groups})
Assume $G$ is connected and $G_{\integer}$ is Zariski dense in~$G$ \ccf{BDT(Zardense)}. 
The complexification $G \otimes \complex$ is defined in \cref{GxCNot}.

Show that if $Z(G \otimes \complex) = \{e\}$, then $\Comm_G(G_{\integer}) = G_{\rational}$.
\hint{For $g \in \Comm_G(G_{\integer})$, we know that $\Ad g$ is an automorphism of the Lie algebra~$\Lie G$ that is defined over~$\rational$, so $\Ad g \in (\Ad G)_{\rational}$. However, the assumptions imply that the adjoint representation is an isomorphism (and it is defined over~$\rational$).}

\item
Show that the assumption $Z(G \otimes \complex) = \{e\}$ cannot be replaced with the weaker assumption $Z(G) = \{e\}$ in \cref{Comm=GC}.
\hint{%Let $G = \SL(3,\real)$. 
Any matrix in $\GL(3,\rational)$ has a scalar multiple that is in $\SL(3,\real)$, but $\SL(3,\rational)$ has infinite index in $\GL(3,\rational)$.}

\end{exercises}





\section{Unipotent elements of noncocompact lattices} \label{GodementSect}

The following result answers one of the most basic topological questions about the manifold $G/G_\integer\mk$: is it compact?

\begin{prop}[(\thmindex{Godement Compactness Criterion}{Godement Compactness Criterion})]
\label{GodementCriterion}
 Assume that $G$ is defined over\/~$\rational$. The homogeneous
space $G/G_{\integer}$ is compact if and only if\/
$G_{\integer}$~has no nontrivial unipotent elements.
 \end{prop}

\begin{proof}
 ($\Rightarrow$) This is the easy direction
\csee{GammaUnip->notcpct}.

($\Leftarrow$) We prove the contrapositive: suppose
$G/G_{\integer}$ is not compact. (We wish to show
that $G_{\integer}$ has a nontrivial unipotent element.)
 From \cref{diverge<>contract} (and the fact that
$G_{\integer}$ is a lattice in~$G$ \csee{arith->latt}), we
know that there exist nontrivial $\gamma \in G_{\integer}$
and $g \in G$, such that ${}^g \gamma \approx \Id$. Because the
characteristic polynomial of a matrix is a continuous
function of the matrix entries of the matrix, we conclude
that the characteristic polynomial of ${}^g \gamma$ is
approximately $(x-1)^\ell$ (the characteristic polynomial
of~$\Id$). On the other hand, similar matrices have the same
characteristic polynomial, so this means that the
characteristic polynomial of~$\gamma$ is approximately
$(x-1)^\ell$. Now all the coefficients of the characteristic
polynomial of~$\gamma$ are integers (because $\gamma$ is an
integer matrix), so the only way this polynomial can be
close to $(x-1)^\ell$ is by being exactly equal to
$(x-1)^\ell$. Therefore, the characteristic polynomial
of~$\gamma$ is $(x-1)^\ell$, so $\gamma$~is unipotent.
 \end{proof}

The following important consequence of the Godement Criterion tells us that there is often no need for compact subgroups in \cref{ArithDefn}, the definition of an arithmetic group:

\begin{cor} \label{IrredNoncpct->NoROS}
 Assume 
 	\begin{itemize}
	\item $\Gamma$ is an irreducible, arithmetic subgroup of~$G$, 
	\item $G/\Gamma$ is not compact,
	and
	\item $G$ is connected and has no compact factors.
	\end{itemize}
Then, perhaps after replacing $G$ by an isogenous group, there is an
embedding of~$G$ in some\/ $\SL(\ell,\real)$, such that 
 	\begin{enumerate}
 	\item $G$ is defined over\/~$\rational$, and
	 \item $\Gamma$ is commensurable to $G_{\integer}$.
 	\end{enumerate}
\end{cor}

\begin{proof}
 From \cref{ArithDefn} (and \fullcref{ArithDefnRem}{noK}) we know that (up to isogeny and
commensurability) there is a compact group~$K'$, such that we
may embed $G' = G \times K'$ in some $\SL(\ell,\real)$, such
that $G'$~is defined over~$\rational$, and $\Gamma K' =
G'_{\integer} K'$.

Let $N$ be the almost-Zariski closure of the subgroup
of~$G'$ generated by all of the unipotent elements of
$G'_{\integer}$. 
Since $G/\Gamma$ is not compact, the \lcnamecref{GodementCriterion} % @@@
implies $N$ is infinite.
However, $K'$ has no unipotent elements \fullcsee{SSeltRem}{cpct}, so $N \subseteq G$. 
Also, the definition of $N$ implies that it is normalized by the Zariski closure of~$G'_{\integer}$.
Therefore, the Borel Density Theorem \pref{BDTNotProj} implies that $N$
is a normal subgroup of~$G$. 

Assume, for simplicity, that $G$ is simple \csee{NoROSPf-NotSimpleEx}. Then the conclusion of the preceding paragraph tells us that $N = G$.  
Therefore, $G$ is the almost-Zariski closure of a subset
of~$G'_{\integer}$, which
implies that $G$ is defined over~$\rational$
\ccf{arithlatt->defdQ}. Hence, $G_{\integer}$ is a lattice
in~$G$, and it is easy to see that it is commensurable to~$\Gamma$ \csee{NoROSPf-GZ=GammaEx}.
 \end{proof}

In the special case where $\Gamma$ is arithmetic, the following
result is an easy consequence of \cref{GodementCriterion}, but we will not prove the general case (which is more difficult). The
assumption that $G$~has no compact factors cannot be
eliminated \csee{NonCocpctNoUnip}.

\begin{thm} \label{GodementNoCpctFactor}
 Assume $G$~has no
compact factors. The homogeneous space $G/\Gamma$
is compact if and only if\/ $\Gamma$~has no nontrivial
unipotent elements.
 \end{thm}

The above proof of \cref{GodementCriterion} relies on
the fact that $G_{\integer}$ is a lattice in~$G$, which
will not be proved until \cref{SLnZLattChap}. The following result illustrates that
the cocompactness of~$G_\integer$ can sometimes be proved quite
easily from the Mahler Compactness Criterion
\pref{MahlerCpct}, without assuming that it is a lattice.

\begin{prop} \label{anis->cocpct}
 If 
 \noprelistbreak
 	\begin{itemize}
	\item $B(x,y)$ is a symmetric, bilinear form on\/~$\rational^\ell$, 
	such that
	\item  $B(x,x) \neq 0$ for all nonzero $x \in \rational^\ell$,
	\end{itemize}
then\/ $\SO(B)_{\integer}$
is cocompact in $\SO(B)_{\real}$.
 \end{prop}

\begin{proof}
 Let $G = \SO(B)$ and $\Gamma = \SO(B)_{\integer} = G_\integer$.
(Our proof will not use the fact that $\Gamma$~is a
lattice in~$G$.)
Replacing $B$ by an integer multiple to clear the
denominators, we may assume $B(\integer^\ell,\integer^\ell)
\subseteq \integer$.

\goodbreak % @@@

\setcounter{step}{0}

\begin{step} \label{anis->cocpct-precpct}
 The image of~$G$ in\/ $\SL(\ell,\real)
/ \SL(\ell,\integer)$ is precompact.
 \end{step}
 Let 
 \begin{itemize}
 \item $\{g_n\}$ be a sequence of elements of~$G$ and
 \item $\{v_n\}$ be a sequence of elements of $\integer^\ell
\smallsetminus \{0\}$. 
 \end{itemize}
 Suppose that $g_n v_n \to 0$. (This will lead to a
contradiction, so the desired conclusion follows from the
Mahler Compactness Criterion \pref{MahlerCpct}.)

Since $B(v,v) \neq 0$ for all
nonzero $v \in \integer^\ell$, and $B(\integer^\ell,\integer^\ell)
\subseteq \integer$, we have $|B(v_n,v_n)| \ge 1$
for all~$n$. Therefore
 $$ 1 \le |B(v_n,v_n)| = |B(g_n v_n, g_n v_n)| \to |B(0,0)|
= 0 .$$
 This is a contradiction.

\begin{step} \label{anis->cocpct-closed}
 The image of~$G$ in\/ $\SL(\ell,\real) /
\SL(\ell,\integer)$ is closed.
 \end{step}
 Suppose 
 	$$ \text{$g_n \gamma_n \to h \in \SL(\ell,\real)$, \ with $g_n
\in G$ and $\gamma_n \in \SL(\ell,\integer)$} . $$
We wish to show $h \in G \, \SL(\ell,\integer)$.

Let $\{\varepsilon_1,\cdots,\varepsilon_\ell\}$ be the standard basis
of~$\real^\ell$ (so each $\varepsilon_j \in \integer^\ell$).  Then
 $$ B(\gamma_n \varepsilon_j, \gamma_n \varepsilon_k)
 \in B(\integer^\ell,\integer^\ell)
 \subseteq \integer .$$
 We also have
 $$ B(\gamma_n \varepsilon_j, \gamma_n \varepsilon_k)
 =  B(g_n \gamma_n \varepsilon_j, g_n \gamma_n \varepsilon_k)
 \to B(h \varepsilon_j, h \varepsilon_k) .$$
 Since $\integer$ is discrete, we conclude that
 $ B(\gamma_n \varepsilon_j, \gamma_n \varepsilon_k)
 = B(h \varepsilon_j, h \varepsilon_k) $
 for any sufficiently large~$n$.
 Therefore $h \gamma_n^{-1} \in \SO(B)$ \csee{hgammainG},
 so we have $h \in G \gamma_n \subseteq G \, \SL(\ell,\integer)$.

\begin{step}
 Completion of the proof.
 \end{step}
 Define $\phi \colon G/\Gamma \to \SL(\ell,\real) /
\SL(\ell,\integer)$ by $\phi(g\Gamma) = g
\SL(\ell,\integer)$. By combining
\cref{anis->cocpct-precpct,anis->cocpct-closed}, we see that the image
of~$\phi$ is compact.  Therefore, it suffices to show that $\phi$
is a homeomorphism onto its image.

Given a sequence $\{g_n\}$ in~$G$, such that $\{\phi(g_n
\Gamma)\}$ converges, we wish to show that $\{g_n \Gamma\}$
converges. There is a sequence $\{\gamma_n\}$ in
$\SL(\ell,\integer)$, and some $h \in G$, such that $g_n
\gamma_n \to h$. The proof of \cref{anis->cocpct-closed}
shows, for all large~$n$, that $h \in G \gamma_n$. Then
$\gamma_n \in Gh = G$ (and we know $\gamma_n \in \SL(\ell,\integer)$), so $\gamma_n \in G_{\integer} =
\Gamma$. Therefore, $\{g_n \Gamma\}$ converges (to
$h\Gamma$), as desired.
 \end{proof}


\begin{exercises}

\item \label{NoROSPf-NotSimpleEx}
 The proof of \cref{IrredNoncpct->NoROS} assumes that $G$ is simple. Eliminate this hypothesis.
 \hint{The proof shows that $N \cap \Gamma$ is a lattice in~$N$. Since $\Gamma$ is irreducible, this implies $N = G$.}
 
\item \label{NoROSPf-GZ=GammaEx}
 At the end of the proof of \cref{IrredNoncpct->NoROS}, show that $G_{\integer}$ is commensurable to~$\Gamma$.
 \hint{We know $G'_{\integer} K' = \Gamma K'$, and $G_{\integer}$ has finite index in $G'_{\integer}$ \csee{finext->latt}. Mod out~$K'$.}

\item \label{NonCocpctNoUnip}
 Show there is a noncocompact lattice~$\Gamma$ in
$\SL(2,\real) \times \SO(3)$, such that no nontrivial
element of~$\Gamma$ is unipotent.
\noprelistbreak % @@@
\hint{$\SL(2,\real)$ has a lattice~$\Gamma'$ that is free. Let $\Gamma$ be the graph of a  homomorphism from $\Gamma'$ to $\SO(3)$.}

\item Suppose $G \subseteq \SL(\ell,\real)$ is defined
over~$\rational$.
 \begin{enumerate}
 \item Show that if $N$ is a closed, normal subgroup of~$G$,
and $N$~is defined over~$\rational$, then $G_{\integer} N$
is closed in~$G$.
 \item Show that $G_{\integer}$ is irreducible if and only
if no proper, closed, connected, normal subgroup of~$G$ is
defined over~$\rational$. (That is, if and only if $G$~is
\defit[Q-@$\rational$-!simple group]{$\rational$-simple}.)
 \item Let $H$ be the Zariski closure of the subgroup
generated by the unipotent elements of~$G_{\integer}$. Show
that $H$~is defined over~$\rational$.
% \item Provide a direct proof of \cref{noncocpct->nocpct}, without using
%\cref{GZirred->scalars}, or any other results on
%Restriction of Scalars.
 \end{enumerate}

\item \label{Cocpct->AllSS}
Show that if every element of~$\Gamma$ is semisimple, then $G/\Gamma$ is compact. 
\hint{There is no harm in assuming that $G$ has no compact factors (why?), so \cref{GodementNoCpctFactor} applies.}

\item \label{CocpctAllSS}
(\emph{assumes some familiarity with reductive groups})
Prove the converse of \cref{Cocpct->AllSS}.
\hint{Let $kau$ be the real Jordan decomposition of an element~$g$ of~$\Gamma$. Since $C_G(ka)$ is reductive \csee{C(T)reductive}, the Jacobson-Morosov Lemma provides a subgroup~$L$ of $C_G(ka)$ that contains~$u$ and is isogenous to $\SL(2,\real)$. So $ka$ is in the closure of ${}^G \!g$. However, ${}^G \! g$ is closed, since $\Gamma$ is discrete and cocompact. Therefore $g = ka$ is semisimple.}

 \item \label{GZCocpctIff}
Assuming $\Gamma = G_\integer$ is arithmetic (and $G$ is defined over~$\rational$), prove the following are equivalent:
 	\begin{enumerate}

	\item \label{GZCocpctIff-cpct}
	$G/G_\integer$ is compact.

	 \item \label{GZCocpctIff-unip}
	 $G_{\rational}$ has no nontrivial unipotent elements.

	 \item \label{GZCocpctIff-GQss}
	 Every element of~$G_\rational$ is semisimple.

	 \item \label{GZCocpctIff-ss}
	 Every element of\/~$\Gamma$ is semisimple.

	 \item \label{GZCocpctIff-SL2Q}
	 $G_\rational$ does not contain a subgroup isogenous to $\SL(2,\rational)$. 
	 (More precisely, there does not exist a continuous homomorphism $\rho \colon \SL(2,\real) \to G$, such that $\rho \bigl( \SL(2,\rational) \bigr) \subseteq G_\rational$.)
	
	 \end{enumerate}
 \hint{($\ref{GZCocpctIff-unip} \Rightarrow \ref{GZCocpctIff-GQss}$)~Jordan decomposition.
 ($\ref{GZCocpctIff-SL2Q} \Rightarrow \ref{GZCocpctIff-unip}$)~Jacobson-Morosov Lemma \pref{JacobsonMorosovOverF}.}


\item \label{QDenseGSimple}
Show that $G_\rational$~is dense in~$G$ if $G$ is defined over~$\rational$, $G$~is simple, and $G/G_\integer$ is not compact.
\hint{The Godement Criterion implies that $G_\rational$ has a nontrivial unipotent element~$u$. Write $u = \exp T = \sum_{k=0}^\ell T^k/k!$ (where $T \in \Mat_{\ell\times\ell}(\rational)$ and $T^{\ell+1} = 0$). Then $\exp(rT) \in G_\rational$ for all $r \in \rational$, so the identity component of $\closure{G_\rational}$ is nontrivial. Combining \cref{arith->latt} with the Borel Density Theorem \pref{BDT-normalize} implies that $\closure{G_\rational} = G$.}
 
 \item \label{hgammainG}
 Let $B(x,y)$ be a symmetric, bilinear form on~$\real^\ell$, 
 let $\{v_1,\cdots,v_\ell\}$ be a  basis of~$\real^\ell$,
and let $\gamma,h \in \SL(\ell,\real)$.
 If $B(\gamma v_j, \gamma v_k) = B(h v_j, h v_k)$ for all $j$ and~$k$, 
 show that $h \gamma^{-1} \in \SO(B)$.
\hint{$\{\gamma v_1,\ldots,\gamma v_\ell\}$ is a basis of~$\real^\ell$.}

\end{exercises}






\section{How to make an arithmetic subgroup}
\label{MakeArithLattSect}

The definition that (modulo commensurability, isogenies, and
compact factors) an arithmetic subgroup must be the
$\integer$-points of~$G$ has the virtue of being concrete.
However, this concreteness imposes a certain lack of
flexibility. (Essentially, we have limited ourselves to the
standard basis of the vector space~$\real^n$, ignoring the
possibility that some other basis might be more convenient
in some situations.) We now describe a more abstract
viewpoint that makes the construction of general arithmetic
lattices more transparent. (In particular, this approach
will be used in \S\ref{RestrictScalarsSect}.) The key point is that there are analogues of $\integer^\ell$ and~$\rational^\ell$ in any real
vector space, not just~$\real^\ell$ \fullcsee{VQ=Qd}{VQ}.

\begin{defns} \label{DefdQAbstract}
 Let $V$ be a real vector space.
 \begin{enumerate}
 \item
 A $\rational$-subspace $V_{\rational}$ of~$V$ is a
\defit[Q-@$\rational$-!form]{$\rational$-form} of~$V$ if the natural $\real$-linear
map $V_{\rational} \otimes_{\rational} \real \to V$ is an isomorphism
\csee{Qform<>basis}. (The map is defined
by $v \otimes t \mapsto tv$.)
 \item 
 A polynomial~$f$ on~$V$ is \defit[defined!over Q@over~$\rational$]{defined
over $\rational$} (with respect to the $\rational$-form
$V_{\rational}$) if $f(V_{\rational}) \subseteq \rational$
\csee{Qdefd<>Qcoeffs}.
 \item
 A subgroup~\nindex{$\Zlatt$ = $\integer$-lattice in a vector space}$\Zlatt$ of the additive
group of~$V_{\rational}$ is a \defit[Z-lattice@$\integer$-lattice]{$\integer$-lattice}
in~$V_{\rational}$ if it is finitely generated and the natural $\rational$-linear map $\Zlatt \otimes_{\integer} \rational  \to V_{\rational}$ is an isomorphism
\csee{vsLatt<>basis}.  (The map is defined
by $v \otimes t \mapsto tv$.)

 \item Each $\rational$-form~$V_{\rational}$ of~$V$
yields a corresponding $\rational$-form of the real
vector space $\End(V)$ by
 $ \End(V)_{\rational} = \{\, A \in \End(V) \mid
A(V_{\rational}) \subseteq V_{\rational} \,\} $
 \csee{VQ->End(V)Q}.
 \item A function~$Q$ on a real vector space~$W$ is a
\defit[polynomial!on a vector space]{polynomial} if for
some (hence, every) $\real$-linear isomorphism $\phi \colon
\real^\ell \iso W$, the composition $f \circ \phi$ is a
polynomial function on~$\real^\ell$.
 \item A subgroup~$H$ of $\SL(V)$ is
\defit[defined!over Q@over~$\rational$]{defined over~$\rational$} (with
respect to the $\rational$-form $V_{\rational}$) if there
exists a set~$\mathcal{Q}$ of polynomials on $\End(V)$, such
that 
 \begin{itemize}
 \item every $Q \in \mathcal{Q}$ is defined
over~$\rational$ (with respect to the $\rational$-form
$V_{\rational}$),
 \item 
 $\Var(\mathcal{Q}) = \{\, g \in \SL(V) \mid
\mbox{$Q(g) = 0$ for all $Q \in \mathcal{Q}$} \,\}$
 is a subgroup of $\SL(V)$, and
 \item $\Var(\mathcal{Q})^\circ$ is a finite-index subgroup of~$H$.
 \end{itemize}
 \end{enumerate}
 \end{defns}

\begin{rems} \ 
\noprelistbreak
 \begin{enumerate}
 \item Suppose $G \subseteq \SL(\ell,\real)$, as usual. For the standard
$\rational$-form $\rational^\ell$ of~$\real^\ell$, it is easy to see
that $G$ is defined over~$\rational$ in terms of
\cref{DefdQAbstract} if and only if it is defined
over~$\rational$ in terms of \cref{DefdQDefn}. 
 \item Some authors simply call $\Zlatt$ a 
 \defit[lattice!in~$V_{\rational}$|indsee{$\integer$-lattice}]{lattice in~$V_{\rational}$}, 
 but this could cause confusion, because
$\Zlatt$ is \emph{not} a lattice in~$V_{\rational}$, in the
sense of \cref{LatticeDefn} (although it \emph{is} a
lattice in~$V$).
 \end{enumerate}
 \end{rems}

A $\rational$-form~$V_{\rational}$
and $\integer$-lattice~$\Zlatt$ simply represent
$\rational^\ell$ and~$\integer^\ell$, under some
identification of~$V$ with~$\real^\ell$:

\begin{lem} \label{VQ=Qd}
 Let\/ $V$ be an $\ell$-dimensional real vector space.
 \begin{enumerate}
 \item \label{VQ=Qd-VQ}
 If\/ $V_{\rational}$ is a\/ $\rational$-form of\/~$V$,
then there exists an\/ $\real$-linear isomorphism $\phi \colon V
\to \real^\ell$, such that $\phi(V_{\rational}) =
\rational^\ell$. Furthermore, if $\Zlatt$ is any
$\integer$-lattice in\/~$V_{\rational}$, then $\phi$ may be
chosen so that $\phi(\Zlatt) = \integer^\ell$.
 \item A polynomial~$f$ on\/~$\real^\ell$ is defined
over\/~$\rational$ \textup(with respect
to the standard\/ $\rational$-form~$\rational^\ell$\textup) if and
only if every coefficient of~$f$ is in\/~$\rational$
\csee{Qdefd<>Qcoeffs}.
 \end{enumerate}
 \end{lem}

Also note that any two $\integer$-lattices
in~$V_{\rational}$ are commensurable:

\begin{lem}[\csee{VQ->vslatt}] \label{pLambda1inLambda2}
 If $\Zlatt_1$ and~$\Zlatt_2$ are two $\integer$-lattices
in $V_{\rational}$, then there is some nonzero $p \in
\integer$, such that $p \Zlatt_1 \subseteq \Zlatt_2$ and $p
\Zlatt_2 \subseteq \Zlatt_1$.
 \end{lem}

It is now easy to prove the following more abstract
characterization of arithmetic subgroups \csee{VQ->LattInG,GLambdaUnique}.

\begin{prop} \label{AbstractArith}
 Suppose $G \subseteq \GL(V)$, and $G$ is defined
over\/~$\rational$, with respect to the\/
$\rational$-form~$V_{\rational}$.
 \begin{enumerate}
 \item \label{AbstractArith-arith}
 If $\Zlatt$ is any $\integer$-lattice in~$V_{\rational}$,
then
 $$ G\!_{\Zlatt} = \{\, g \in G \mid g \Zlatt = \Zlatt \,\}
$$
 is an arithmetic subgroup of~$G$.
 \item \label{AbstractArith-comm}
 If $\Zlatt_1$ and~$\Zlatt_2$ are
$\integer$-lattices in~$V_{\rational}$, then
$G\!_{\Zlatt_1}$ is commensurable to~$G\!_{\Zlatt_1}$.
 \end{enumerate}
 \end{prop}

From \fullcref{AbstractArith}{comm}, we see that the
arithmetic subgroup~$G\!_{\Zlatt}$ is almost entirely determined
by the $\rational$-form~$V_{\rational}$; choosing a different $\integer$-lattice 
in $V_\rational$ will yield a commensurable arithmetic subgroup.


\begin{exercises}

\item \label{Qform<>basis}
 Show that a $\rational$-subspace $V_{\rational}$ of~$V$ is
a $\rational$-form if an only if there is a subset~$\basis$
of~$V_{\rational}$, such that $\basis$ is both a
$\rational$-basis of~$V_{\rational}$ and an $\real$-basis
of~$V$.

\item \label{Qdefd<>Qcoeffs}
 For the standard $\rational$-form $\rational^\ell$
of~$\real^\ell$, show that a polynomial is defined
over~$\rational$ if and only if all of its coefficients are
rational.

\item \label{vsLatt<>basis}
 Show that a subgroup~$\Zlatt$ of~$V_{\rational}$ is a
$\integer$-lattice in~$V_{\rational}$ if and only if there
is a $\rational$-basis~$\basis$ of~$V_{\rational}$, such
that $\Zlatt$ is the additive abelian subgroup
of~$V_{\rational}$ generated by~$\basis$.

\item \label{VSLatt<>Rank}
 Let $V$ be a real vector space of dimension~$\ell$,
and let $\Zlatt$~be a discrete subgroup of the additive
group of~$V$. Recall that the \defit[rank!of an abelian group]{rank} of an abelian group is the largest~$r$, such that the group contains a copy of~$\integer^r$.
 \begin{enumerate}
 \item Show that $\Zlatt$ is a finitely generated, abelian
group of rank $\le \ell$, with equality if and only if the
$\real$-span of~$\Zlatt$ is~$V$.
 \item Show that if the rank of~$\Zlatt$ is~$\ell$, then
the $\rational$-span of~$\Zlatt$ is a $\rational$-form
of~$V$, and $\Zlatt$~is a $\integer$-lattice
in~$V_{\rational}$.
 \end{enumerate}
 \hint{Induction on~$\ell$. For $\lambda \in \Zlatt$, show
that the image of~$\Zlatt$ in $V/\real \lambda$ is
discrete.}

\item \label{VQ->End(V)Q}
 Verify: if $V_{\rational}$ is a $\rational$-form
of~$V$, then $\End(V)_{\rational}$ is a
$\rational$-form of $\End(V)$.

\item \label{VQ->vslatt}
 Prove \cref{pLambda1inLambda2}. Conclude that
$\Lambda_1$ and~$\Lambda_2$ are commensurable.

\item \label{VQ->LattInG}
 Prove \fullcref{AbstractArith}{arith}. [{\it Hint:}
Use \cref{VQ=Qd}.]

\item \label{GLambdaUnique}
 Prove \fullcref{AbstractArith}{comm}. [{\it Hint:}
Use \cref{pLambda1inLambda2}.]


%\item \label{ChevalleyStabOverQ} % cf. \ref{ChevalleyStabDiscreteZ}
%Assume $G$ is defined over~$\rational$ (and connected).
%Show there exist
%	\begin{itemize}
%	\item a $\rational$-form $V_{\rational}$ of some finite-dimensional real vector space~$V$, 
%	\item a $\integer$-lattice~$\Zlatt$ in~$V_{\rational}$,
%	\item a vector~$v$ in~$\Zlatt$,
%	and
%	\item a homomorphism $\rho \colon \SL(\ell,\real) \to \SL(V)$,
%	\end{itemize}
%such that 
%	\begin{enumerate}
%	\item $G = \Stab_{\SL(\ell,\real)}(v)^\circ$,
%	\item $\rho \bigl( \SL(\ell,\rational) \bigr) v \subseteq V_{\rational}$,
%	and 
%	\item $\rho \bigl( \SL(\ell,\integer) \bigr) v \subseteq \Zlatt$.
%	\end{enumerate}
%\hint{See the hint to \cref{ChevalleyStabEx}.}

\end{exercises}






\section{Restriction of scalars} \label{RestrictScalarsSect}

We know that $\SL(2,\integer)$ is an arithmetic subgroup of
$\SL(2,\real)$. In this section, we explain that $\SL \bigl(
2, \integer[\sqrt{2}] \bigr)$ is an arithmetic subgroup of the group
$\SL(2,\real) \times \SL(2,\real)$ (see
\cref{SL(2Z[sqrt2])}). More generally, recall that any finite extension of~$\rational$
is called an \defit[algebraic!number field]{algebraic number field}. We will see that if
$\ints$ is the ring of algebraic integers in any
algebraic number field~$F$, and $G$~is defined over~$F$,
then $G_{\ints}$ is an arithmetic subgroup of a certain
group~$G'$ that is related to~$G$. 

\begin{rem}
 In practice, we do not require $\ints$ to be the
entire ring of algebraic integers in~$F$: it suffices for the ring~$\ints$ to have finite index in the ring of integers
(as an additive group); equivalently, the $\rational$-span
of~$\ints$ should be all of~$F$, or, in other words,
the ring $\ints$ should be a $\integer$-lattice
in~$F$. (A $\integer$-lattice in~$F$ that is also a
subring is called an \defit[order (in an algebraic number
field)]{order} in~$F$.)
 \end{rem}

Any complex vector space can be thought of as a real vector
space (of twice the dimension). Similarly, any complex Lie
group can be thought of as a real group (of twice the
dimension). Restriction of scalars is the generalization of
this idea to any field extension $F/L$, not just
$\complex/\real$. This yields a general method to construct
arithmetic subgroups. 

\begin{eg}
 Let 
 \begin{itemize}
 \item $F = \rational[\!\sqrt{2}]$,
 \item $\ints = \integer[\!\sqrt{2}]$, and
 \item $\sigma$ be the nontrivial Galois automorphism of~$F$,
 \end{itemize}
 and define a ring homomorphism $\Delta \colon F \to
\real^2$ by 
 $\Delta(x) = \bigl( x, \sigma(x) \bigr) $.

 It is easy to show that $\Delta(\ints)$ is discrete
in~$\real^2$. Namely, for $x \in \ints$, the product
of the coordinates of~$\Delta(x)$ is the product $x \cdot
\sigma(x)$ of all the Galois conjugates of~$x$. This is the
\defit[norm!of an algebraic number]{norm} of the algebraic
number~$x$. Because $x$ is an algebraic integer, its norm is
an ordinary integer; hence, its norm is bounded away
from~$0$. So it is impossible for both coordinates of
$\Delta(x)$ to be small simultaneously.

More generally, if $\ints$ is the ring of integers of
any algebraic number field~$F$, this same argument shows that if we let
$\{ \sigma_1, \ldots, \sigma_r \}$ be the set of all
embeddings of~$\ints$ in~$\complex$, and define
 $\Delta \colon \ints \to \complex^r$
 by 
 $$ \Delta(x) = \bigl( \sigma_1(x), \ldots, \sigma_r(x) \bigr) , $$
 then $\Delta( \ints )$ is a \emph{discrete} subring
of~$\complex^r$. 

Now $\Delta$ induces a homomorphism $\Delta_* \colon \SL(\ell, \ints) \to \SL(\ell, \complex^r)$ (because $\SL(\ell, \,{\cdot}\,)$ is a functor from the category of commutative rings to the category of groups). Furthermore, the group $\SL(\ell, \complex^r)$ is naturally isomorphic to $\SL(\ell, \complex)^r$. Therefore, we have a homomorphism (again called~$\Delta$) from $\SL(\ell, \ints)$ to  $\SL(\ell, \complex)^r$. Namely, for $\gamma \in \SL(\ell, \ints)$, we let $\sigma_i(\gamma) \in \SL(\ell,\complex)$ be obtained by applying $\sigma_i$ to each entry of~$\gamma$, and then
 $$ \Delta(\gamma) = \bigl( \sigma_1(\gamma), \ldots, \sigma_r(\gamma) \bigr) .$$
Since $\Delta(\ints)$ is discrete in~$\complex^r$, it is obvious that the image of~$\Delta_*$ is discrete in $\SL(\ell, \complex^r)$, so $\Delta(\Gamma)$ is a discrete subgroup
of $\SL(\ell,\complex)^r$, for any subgroup~$\Gamma$ of
$\SL(\ell,\ints)$. 
 \end{eg}

The main goal of this \lcnamecref{RestrictScalarsSect} is to show that if $\Gamma = G_{\ints}$, and $G$~is defined
over~$F$, then the discrete group~$\Delta(\Gamma)$ is an arithmetic subgroup of a certain subgroup of
$\SL(\ell,\complex)^r$.

To illustrate, let us show that
$\SL \bigl( 2,\integer[\!\sqrt{2}] \bigr)$ is isomorphic to an
arithmetic subgroup of $\SL(2,\real) \times
\SL(2,\real)$.

\begin{eg} \label{SL(2Z[sqrt2])}
 Let
 \noprelistbreak
 \begin{itemize}
 \item $\Gamma = \SL \bigl( 2,\integer[\!\sqrt{2}] \bigr)$,
 \item $G = \SL(2, \real) \times \SL(2, \real)$, and
 \item $\sigma$ be the conjugation on $\rational[\!\sqrt{2}]$
\textup(so $\sigma \bigl( a + b \sqrt{2} \bigr) = a - b
\sqrt{2}$, for $a,b \in \rational$\textup),
 \end{itemize}
 and define $\Delta \colon \Gamma \to G$ by
$\Delta(\gamma) = \bigl(\gamma, \sigma(\gamma) \bigr)$.

Then $\Delta(\Gamma)$ is an irreducible, arithmetic subgroup
of~$G$.
 \end{eg}

\begin{proof}
 Let $F = \rational[\!\sqrt{2}]$ and $\ints =
\integer[\!\sqrt{2}]$. Then $F$ is a 2-dimensional vector
space over~$\rational$, and $\ints$ is a
$\integer$-lattice in~$F$.

Since $\bigl\{ (1,1) , (\sqrt{2},-\sqrt{2}) \bigr\}$ is both
a $\rational$-basis of~$\Delta(F)$ and an $\real$-basis
of~$\real^2$, we see that $\Delta(F)$ is a $\rational$-form
of~$\real^2$. Therefore,
 $$\Delta(F^2) = \bigl\{\, \bigl( u, \sigma(u) \bigr) \in F^4
\mid u \in F^2 \,\bigr\} $$
 is a $\rational$-form of~$\real^4$, and
$\Delta(\ints^2)$ is a $\integer$-lattice in
$\Delta(F^2)$.

Now $G$ is defined over~$\rational$ \csee{SL2xSL2Defd/Q}, so
$G_{\Delta(\ints^2)}$ is an arithmetic subgroup of~$G$.
It is not difficult to see that $G_{\Delta(\ints^2)} =
\Delta(\Gamma)$ \csee{SL2xSL2-Delta(Gamma)=}. Furthermore,
because $\Delta(\Gamma) \cap \bigl( \SL(2,\real) \times e
\bigr)$ is trivial, we see that the lattice $\Delta(\Gamma)$
must be irreducible in~$G$ \csee{prodirredlatt}.
 \end{proof}

More generally, the proof of \cref{SL(2Z[sqrt2])}
shows that if $G$ is defined over~$\rational$, then
$G_{\integer[\!\sqrt{2}]}$ is isomorphic to an (irreducible)
arithmetic subgroup of $G \times G$. 

Here is another sample application of the method.

\begin{eg} \label{SO(12;Z[sqrt2])}
 Let $G = \SO(x^2 + y^2 - \sqrt{2} z^2 ; \real) \iso
\SO(1,2)$. Then $G_{\integer[\!\sqrt{2}]}$ is a
cocompact, arithmetic subgroup of~$G$.
 \end{eg}

\begin{proof}
 As above, let $\sigma$ be the conjugation on
$\rational[\!\sqrt{2}]$. Let $\Gamma = G_{\integer[\!\sqrt{2}]}$.

Let $K' = \SO(x^2 + y^2 + \sqrt{2} z^2) \iso \SO(3)$, so
$\sigma(\Gamma) \subseteq K'$. (However, $\sigma(\Gamma) \not\subseteq G$.) Then, we may 
	$$ \text{define \ $\Delta
\colon \Gamma \to G \times K'$ \ by \ $\Delta(\gamma) = \bigl( \mkern2mu
 \gamma, \sigma(\gamma) \bigr)$} .$$
Arguing as in the proof of \cref{SL(2Z[sqrt2])} establishes that
$\Delta(\Gamma)$ is an arithmetic subgroup of $G \times K'$.
(See \cref{GxG-Defd/Q} for the technical point of
verifying that $G \times K'$ is defined over~$\rational$.)
Since $K'$ is compact, we see, by modding out~$K'$, that
$\Gamma$ is an arithmetic subgroup of~$G$. (This type of
example is the reason for including the compact normal
subgroup~$K'$ in \cref{ArithDefn}.)

 Let $\gamma$ be any nontrivial element of~$\Gamma$. Since
$\sigma(\gamma) \in K'$, and compact groups have no
nontrivial unipotent elements \fullcsee{SSeltRem}{cpct}, we know that
$\sigma(\gamma)$ is not unipotent. Therefore, $\sigma(\gamma)$
has some eigenvalue $\lambda \neq 1$. Hence, $\gamma$ has
the eigenvalue $\sigma^{-1}(\lambda) \neq 1$, so $\gamma$ is
not unipotent. Therefore, Godement's Criterion
\pref{GodementCriterion} implies that $\Gamma$ is cocompact.
Alternatively, this conclusion can easily be obtained
directly from the Mahler Compactness Criterion
\pref{MahlerCpct} \csee{SO(B)Z2cocpt(Mahler)}.
 \end{proof}

Let us consider one more example before stating the general
result.

\begin{eg}
 Let
 \begin{itemize}
 \item $F = \rational[\!\! \root 4\!\! \of 2]$,
 \item $\ints = \integer[\!\!\root 4\!\!\of 2]$,
 \item $\Gamma = \SL(2,\ints)$, and
 \item $G = \SL(2,\real) \times \SL(2,\real) \times
\SL(2,\complex)$.
 \end{itemize}
 Then $\Gamma$ is isomorphic to an irreducible, arithmetic
subgroup of~$G$.
 \end{eg}

\begin{proof}
 For convenience, let $\alpha = \!\root 4\!\! \of 2$. There are
exactly 4 distinct embeddings $\sigma_0$, $\sigma_1$,
$\sigma_2$, $\sigma_3$ of~$F$ in~$\complex$ (corresponding
to the 4~roots of $x^4 - 2 = 0$); they are determined by:
 $$ \text{$\sigma_0(\alpha) = \alpha$ \  (so $\sigma_0 = \Id$),
 \  $\sigma_1(\alpha) = -\alpha$,
 \  $\sigma_2(\alpha) = i\alpha$, 
 \ and
 \  $\sigma_3(\alpha) = -i\alpha$}
 . $$
 Define $\Delta \colon F \to \real \oplus \real \oplus
\complex$ by $\Delta(x) = \bigl( x, \sigma_1(x), \sigma_2(x)
\bigr)$. Then, arguing much as before, we see that
$\Delta(F^2)$ is a $\rational$-form of~$\real^2 \oplus
\real^2 \oplus \complex^2$, $G$~is defined over~$\rational$,
and $G_{\Delta(\ints^2)} = \Delta(\Gamma)$.
 \end{proof}

These examples illustrate all the ingredients of the general result that will be stated in \cref{ResScal->Latt} after the necessary definitions.

\begin{defn} \label{PlaceDefn}
 Let $F$ be an algebraic number field (or, in other words, let $F$ be a finite extension of~$\rational$).
 \begin{enumerate}
 \item Two distinct embeddings $\sigma_1, \sigma_2 \colon F
\to \complex$ are said to be \defit[equivalent embeddings in
$\complex$]{equivalent} if $\sigma_1(x) =
\overline{\sigma_2(x)}$, for all $x \in F$ (where
$\overline{z}$ denotes the usual complex conjugate of the
complex number~$z$).
 \item A \defit[place!of~$F$]{place} of~$F$ is an equivalence
class of embeddings in~$\complex$. Therefore, each place
consists of either one or two embeddings of~$F$: 
	\begin{itemize}
	\item a \defit[place!real]{real place} consists of only one embedding (with
$\sigma(F) \subset \real$), but
	\item a \defit[place!complex]{complex place} consists
of two embeddings (with $\sigma(F) \not\subset \real$).
	\end{itemize}
 \item We let $S^\infty = \{\, \mbox{places of~$F$} \,\}$,
or, abusing notation, we assume that $S^\infty$ is a set of
embeddings, consisting of exactly one embedding from each
place.
 \item For $\sigma \in S^\infty$, we let
 $$ F_\sigma = 
 \begin{cases}
 \real & \mbox{if $\sigma$ is real}, \\
 \complex & \mbox{if $\sigma$ is complex}.
 \end{cases}
 $$
 Note that $\sigma(F)$ is dense in~$F_\sigma$, so $F_\sigma$
is often called the \defit[completion of~$F$]{completion}
of~$F$ at the place~$\sigma$.
 \item For $\mathcal{Q} \subset
F_\sigma[x_{1,1},\ldots,x_{\ell,\ell}]$, let
 $$ \Var_{F_\sigma}(\mathcal{Q})
 = \{\, g \in \SL(\ell,F_\sigma) \mid Q(g) = 0, \ \forall Q
\in \mathcal{Q} \,\} .$$
 Thus, for $F_\sigma = \real$, we have
$\Var_{\real}(\mathcal{Q}) = \Var(\mathcal{Q})$, and
$\Var_{\complex}(\mathcal{Q})$ is analogous, using the
field~$\complex$ in place of~$\real$.
 \item Suppose $G \subseteq \SL(\ell,\real)$, and $G$ is
defined over~$F$, so there is some subset~$\mathcal{Q}$ of
$F[x_{1,1},\ldots,x_{\ell,\ell}]$, such that $G^\circ =
\Var(\mathcal{Q})^\circ$. For each place~$\sigma$ of~$F$, let
 $$ G^\sigma = \Var_{F_\sigma}\bigl( \sigma(\mathcal{Q})
\bigr)^\circ .$$
 Then $G^\sigma$, the \defit[Galois!conjugate of~$G$]{Galois
conjugate} of~$G$ by~$\sigma$, is defined over~$\sigma(F)$.
 \end{enumerate}
 \end{defn}

\begin{other} \label{onlyinfinite}
Our definition requires places to be \defit[place!infinite]{infinite} (or
\defit[place!archimedean]{archimedean}); that is the
reason for the superscript~$\infty$ on~$S^\infty$. Other authors also
allow places that are \defit[place!finite]{finite} (or
\defit[place!nonarchimedean]{nonarchimedean}, or
\defit[place!p-adic@$p$-adic]{$p$-adic}). These additional places are of
fundamental importance in number theory, and, therefore, in
deeper aspects of the theory of arithmetic groups. For
example, superrigidity at the finite places will play a crucial
role in the proof of the Margulis Arithmeticity Theorem
in \cref{MargArithPf}. 
Finite places are also essential for the definition of the ``$S$-arithmetic'' groups discussed in \cref{SarithChap}.
 \end{other}

\begin{prop} \label{ResScal->Latt}
 If $G$ is defined over an algebraic
number field~$F \subset \real$, and $\ints$ is the
ring of integers of~$F$, then
there is a finite-index subgroup $\dot G_\ints$ of~$G_\ints$, such that
	$$ \text{$\dot G_{\ints}$ embeds as an arithmetic subgroup of
 \ $ \displaystyle \prod_{\sigma \in S^\infty} G^\sigma $} , $$
 via the natural embedding $\Delta \colon \gamma \mapsto
\bigl( \sigma(\gamma) \bigr)_{\sigma \in S^\infty}$ 

Furthermore, if $G$ is simple, then the lattice~$\Delta(G_{\ints})$ is irreducible.
 \end{prop}

\begin{warn}
By our definition, $G^\sigma$ is always connected, since it is the identity component of $\Var_{F_\sigma}\bigl( \sigma(\mathcal{Q})
\bigr)$. If $G$ is assumed to be Zariski closed (so it is equal to $\Var(\mathcal{Q})$, rather than merely being isogenous to it), then it is sometimes more convenient to define $G^\sigma$ to be the entire variety $\Var_{F_\sigma}\bigl( \sigma(\mathcal{Q})
\bigr)$, rather than merely the identity component. In particular, that would eliminate the need to pass to a finite-index subgroup $\dot G_\ints$ in the statement of \cref{ResScal->Latt}. Taking the best of both worlds, we will usually ignore the difference between $G_\ints$ and $\dot G_\ints$, and pretend that the map $\Delta$ of \cref{ResScal->Latt} is defined on all of $G_\ints$. For example, the statements of \cref{scalars->cpct,GZirred->scalars} below omit the dots that should be in $\Delta(\dot G_\ints)$ and $\phi \bigl( \Delta(\dot H_\ints) \bigr)$.
\end{warn}

The argument in the last paragraph of the proof of
\cref{SO(12;Z[sqrt2])} shows the following:

\begin{cor} \label{scalars->cpct}
 If $G^\sigma$ is compact, for some $\sigma \in S^\infty$,
then $\Delta(G_{\ints})$ is cocompact.
 \end{cor}

\begin{rem} \label{ResScal(FnotinR)}
 \Cref{ResScal->Latt} is stated only for real
groups, but the same conclusions hold if
\noprelistbreak
 \begin{itemize}
 \item $G \subseteq \SL(\ell,\complex)$, 
 \item $F$~is an algebraic number field, such that $F \not
\subset \real$, and
 \item $G$ is defined over~$F$, as an algebraic
group over~$\complex$; that is, there is  a
subset~$\mathcal{Q}$ of $F[x_{1,1},\ldots,x_{\ell,\ell}]$,
such that $G^\circ = \Var_{\complex}(\mathcal{Q})^\circ$  \csee{GxCNot}.
 \end{itemize}
 For example, we have the following irreducible arithmetic lattices:
 \begin{enumerate}
 \item $\SO \bigl( n, \integer [ i, \!\sqrt{2} ]
\bigr)$ 
in
$\SO(n,\complex) \times \SO(n,\complex)$, and
 \item
  $\SO \! \left( n, \integer \left[ \! \sqrt{1 - \sqrt{2}} \right]
\right)$ in
$\SO(n,\complex) \times \SO(n,\real) \times \SO(n,\real)$.
 \end{enumerate}
 \end{rem}

The following converse shows that restriction of scalars is
the only way to make a group of $\integer$-points that is irreducible.

\begin{prop} \label{GZirred->scalars}
 If\/ $\Gamma = G_{\integer}$ is an irreducible lattice
in~$G$ \textup(and $G$ is connected\/\textup), then there exist
\noprelistbreak
 \begin{enumerate}
 \item an algebraic number field~$F$, with
completion~$F_\infty$ \textup($= \real$
or\/~$\complex$\textup),
 \item a connected, simple subgroup~$H$ of\/
$\SL(\ell,F_{\infty})$, for some~$\ell$,  such that $H$~is
defined over~$F$ \textup(as an algebraic
group over~$F_{\infty}$\textup), and
 \item an isogeny
 $$\phi \colon \prod_{\sigma \in S^\infty} H^\sigma \to G ,$$
 \end{enumerate}
 such that 
 $\phi \bigl( \Delta(H_{\ints}) \bigr)$
 is commensurable to~$\Gamma$.
 \end{prop}

\begin{proof} 
 It is easier to work with the algebraically closed
field~$\complex$, instead of~$\real$, so, to avoid minor
complications, let us assume that $G \subseteq \SL(\ell,
\complex)$ is defined over~$\rational[i]$ (as an algebraic
group over~$\complex$), and that $\Gamma = G_{\integer[i]}$.
This assumption results in a loss of generality, but similar
ideas apply in general.

Write $G = G_1 \times \cdots \times G_r$, where each
$G_i$~is simple. Let $H = G_1$. We remark that if $r = 1$,
then the desired conclusion is obvious: let $F =
\rational[i]$, and let $\phi$~be the identity map.

Let $\Sigma$ be the Galois group of~$\complex$
over~$\rational[i]$. Because $G$ is defined
over~$\rational[i]$, we have $\sigma(G) = G$ for every
$\sigma \in \Sigma$. Hence, $\sigma$~must permute the simple
factors $\{G_1,\ldots,G_r\}$. 

We claim that $\Sigma$ acts transitively on
$\{G_1,\ldots,G_r\}$. To see this, suppose, for example,
that $r = 5$, and that $\{G_1,G_2\}$ is invariant
under~$\Sigma$. Then $A = G_1 \times G_2$ is invariant
under~$\Sigma$, so $A$ is defined
over~$\rational[i]$. Similarly, $A' = G_3 \times G_4 \times
G_5$ is also defined over~$\rational[i]$. Then
$A_{\integer[i]}$ and $A'_{\integer[i]}$ are lattices in $A$
and~$A'$, respectively, so $\Gamma = G_{\integer[i]} \approx
A_{\integer[i]} \times A'_{\integer[i]}$ is reducible. This
is a contradiction.

Let
 $$ \Sigma_1 = \{\, \sigma \in \Sigma \mid \sigma(G_1) = G_1
\,\}$$
 be the stabilizer of~$G_1$, and let
 $$ F = \{\, z \in \complex \mid \sigma(z) = z, \ \forall
\sigma \in \Sigma_1 \,\} $$
 be the fixed field of~$\Sigma_1$. Because $\Sigma$ is
transitive on a set of $r$~elements, we know that $\Sigma_1$
is a subgroup of index~$r$ in~$\Sigma$, so Galois Theory
tells us that $F$~is an extension of $\rational[i]$ of
degree~$r$.

Since $\Sigma_1$ is the Galois group of~$\complex$ over~$F$,
and $\sigma(G_1) = G_1$ for all $\sigma \in \Sigma_1$, we see
that $G_1$ is defined over~$F$. 

Let $\sigma_1,\ldots,\sigma_r$ be coset representatives
of~$\Sigma_1$ in~$\Sigma$. Then $\sigma_1|_F, \ldots,
\sigma_r|_F$ are the $r$~places of~$F$ and, after
renumbering, we have $G_j = \sigma_j(G_1)$.
So (with $H = G_1)$, we have
 $$\prod_{\sigma \in S^\infty} H^\sigma
 = H^{\sigma_1|_F} \times \cdots \times H^{\sigma_r|_F}
 = \sigma_1(G_1) \times \cdots \times \sigma_r(G_1)
 = G_1 \times \cdots \times G_r
 = G .$$
 Let $\phi$ be the identity map.

For $h \in H_F$, let $\Delta'(h) = \prod_{j=1}^r
\sigma_j(h)$. Then $\sigma \bigl( \Delta'(h) \bigr) =
\Delta'(h)$ for all $\sigma \in \Sigma$, so $\Delta'(h) \in
G_{\rational[i]}$. In fact, it is not difficult
to see that $\Delta'(H_F) = G_{\rational[i]}$, and then one
can verify that
 $\Delta'(H_{\ints}) \approx G_{\integer[i]} =
\Gamma$, so $\phi \bigl( \Delta(H_{\ints})$ is
commensurable to~$\Gamma$.
 \end{proof}

\begin{rem} \label{ROSAbsSimple}
Although it may not be clear from our proof, the group $G'$ in \cref{irred->scalars} can be chosen to be ``\term[simple!absolutely]{absolutely simple}\zz.'' This means that if $F \subset \real$, then the following three equivalent conditions must be true: $G'$ remains simple over~$\complex$, $\Lie G' \otimes_\real \complex$ is simple, and $G'$ is not isogenous to any ``complexification'' $(G'')_\complex$.
\end{rem}

Combining \cref{GZirred->scalars} with
\cref{scalars->cpct} yields the following result.

\begin{cor} \label{noncocpct->nocpct}
 If $G_{\integer}$ is an irreducible lattice in~$G$, and
$G/G_{\integer}$ is not cocompact, then $G$ has no compact
factors.
 \end{cor}

By combining \cref{GZirred->scalars} with \cref{ArithDefn}, we see that every irreducible arithmetic subgroup can be constructed by using restriction of scalars, and then modding out a compact subgroup:

\begin{cor} \label{irred->scalars}
 If\/ $\Gamma$ is an irreducible, arithmetic lattice in~$G$ \textup(and $G$ is connected\/\textup), then there exist
 \begin{enumerate}
 \item an algebraic number field~$F$, with
completion~$F_\infty$ \textup($= \real$
or~$\complex$\textup),
 \item a connected, simple subgroup~$G'$ of\/
$\SL(\ell,F_{\infty})$, for some~$\ell$,  such that $G'$~is
defined over~$F$ \textup(as an algebraic group
over~$F_{\infty}$\textup), and
 \item a continuous surjection 
 $$\phi \colon \prod_{\sigma \in S^\infty} (G')^\sigma \to G ,$$
 with compact kernel,
 \end{enumerate}
 such that 
 $\phi \bigl( \Delta(G'_{\ints}) \bigr)$
 is commensurable to~$\Gamma$.
 \end{cor}
 
 When $G$ is simple, the restriction of~$\phi$ to some simple factor of $\prod_{\sigma \in S^\infty} (G')^\sigma$ must be an isogeny, so the conclusion can be stated in the following much simpler form:

\begin{cor} \label{simple->Arith=Ints}
 If\/ $\Gamma$ is an arithmetic subgroup of~$G$, and $G$ is simple, then there exist
 \begin{enumerate}
 \item an algebraic number field~$F$, with
completion~$F_\infty$ \textup($= \real$
or~$\complex$\textup),
 \item a connected, simple subgroup~$G'$ of\/
$\SL(\ell,F_{\infty})$, for some~$\ell$,  such that $G'$~is
defined over~$F$ \textup(as an algebraic group
over~$F_{\infty}$\textup), and
 \item an isogeny
 $\phi \colon G' \to G$,
 \end{enumerate}
 such that 
 $\phi( G'_{\ints})$
 is commensurable to~$\Gamma$.
 \end{cor}

However, we should point out that this result is of interest only when $\Gamma$ is cocompact (or is reducible with at least one cocompact factor). This is because there is no need for restriction of scalars when the irreducible lattice~$\Gamma$ is not cocompact \csee{IrredNoncpct->NoROS}.

\begin{exercises}

\item \label{End(R4)Q}
 In the notation of the proof of \cref{SL(2Z[sqrt2])},
show, for the $\rational$-form $\Delta(F^2)$ of~$\real^4$,
that
 $$ \End(\real^4)_{\rational}
 = \bigset{
 \begin{bmatrix}
 A & B \\
 \sigma(B) & \sigma(A)
 \end{bmatrix}
 }{
 A,B \in \Mat_{2 \times 2}(F)
 } .$$
\hint{Since the $F$-span of $\Delta(F^2)$ is~$F^4$,
we have $\End(\real^4)_{\rational} \subseteq \Mat_{4 \times
4}(F)$. Thus, for any $T \in \End(\real^4)_{\rational}$, we
may write
 $ T = \begin{Smallbmatrix}
 A & B \\
 C & D
 \end{Smallbmatrix}$,
 with $A,B,C,D \in \Mat_{2 \times 2}(F)$. Now use the fact
that, for all $u \in F^2$, we have $T(u) = \bigl( v,
\sigma(v) \bigr)$, for some $v \in F^2$.}

\item \label{SL2xSL2Defd/Q}
 In the notation of the proof of \cref{SL(2Z[sqrt2])},
let 
 \begin{align*}
  \mathcal{Q} = &\bigset{ x_{i,j+2}  + x_{i+2,j}, ~ x_{i,j+2} x_{i+2,j} }{  1 \le i, j \le 2 }
	\\& \quad \cup \left\{ \frac{1}{\sqrt{2}} \Bigl( (x_{1,1} x_{2,2} - x_{1,2} x_{2,1}) - (x_{3,3} x_{4,4} - x_{3,4} x_{4,3}) \Bigr) \right\} 
	. \end{align*}
 \begin{enumerate}
 \item Use the conclusion of \cref{End(R4)Q} to show
that each $Q \in \mathcal{Q}$ is defined over~$\rational$.
 \item Show that $\Var(\mathcal{Q})^\circ = \SL(2,\real)
\times \SL(2,\real)$.
 \end{enumerate}

\item \label{SL2xSL2-Delta(Gamma)=}
 In the notation of the proof of \cref{SL(2Z[sqrt2])},
use \cref{End(R4)Q} to show that
$G_{\Delta(\ints^2)} = \Delta(\Gamma)$.

\item \label{GxG-Defd/Q}
 Let $F$, $\ints$, $\sigma$, $\Delta$ be as in the
proof of \cref{SL(2Z[sqrt2])}.
If $G \subseteq \SL(\ell,\real)$, and $G$ is defined over~$F$,
show $G \times G$ is defined over~$\rational$ (with
respect to the $\rational$-form on $\End(\real^{2\ell})$
induced by the $\rational$-form $\Delta(F^\ell)$
on~$\real^{2\ell}$).
\hint{For each $Q \in
\rational[x_{1,1},\ldots,x_{\ell,\ell}]$, let us define a
corresponding polynomial $Q^+ \in
\rational[x_{\ell+1,\ell+1},\ldots,x_{2\ell,2\ell}]$ by
replacing every occurrence of each variable $x_{i,j}$ with
$x_{\ell+i,\ell+j}$. For example, if $\ell = 2$, then
 $$ (x_{1,1}^2 + x_{1,2} x_{2,1} - 3 x_{1,1} x_{2,2})^+
 = x_{3,3}^2 + x_{3,4} x_{4,3} - 3 x_{3,3} x_{4,4} .$$
 Choose $\mathcal{Q}_0 \subset
\rational[x_{1,1},\ldots,x_{\ell,\ell}]$ that defines $G$ as
a subgroup of $\SL(\ell,\real)$, and let
 $$ \mathcal{Q}_1 = 
 \{\, Q + \sigma(Q^+), ~ Q \, \sigma(Q^+) \mid Q \in
\mathcal{Q}_0 \,\}.$$
 A natural generalization of \cref{SL2xSL2Defd/Q} shows
that $\SL(\ell,\real) \times \SL(\ell,\real)$ is defined
over~$\rational$: let $\mathcal{Q}_2$ be the corresponding
set of $\rational$-polynomials. Now define $\mathcal{Q} =
\mathcal{Q}_1 \cup \mathcal{Q}_2$.}

\item \label{Delta(O)=VSLatt}
 Suppose $\ints$ is the ring of integers of an
algebraic number field~$F$.
 \begin{enumerate}
 \item \label{Delta(O)=VSLatt-discrete}
 Show $\Delta(\ints)$ is discrete in
$\bigoplus_{\sigma \in S^\infty} F_{\sigma}$.
 \item \label{Delta(O)=VSLatt-F}
 Show $\Delta(F)$ is a $\rational$-form of
$\bigoplus_{\sigma \in S^\infty} F_{\sigma}$.
 \item \label{Delta(O)=VSLatt-O}
 Show $\Delta(\ints)$ is a $\integer$-lattice
in $\Delta(F)$.
 \end{enumerate}

\item \label{SO(B)Z2cocpt(Mahler)}
 Let 
 \begin{itemize}
 \item $B(v,w) = v_1 w_1 + v_2 w_2 - \sqrt{2} v_3 w_3$, for
$v,w \in \real^3$,
 \item $G = \SO(B)^\circ$,
 \item $G^* = G \times G^\sigma$,
 \item $\Gamma = G_{\integer[\sqrt{2}]}$,
 and 
 \item $\Gamma^* = \Delta(\Gamma)$.
 \end{itemize}
 Show:
 \begin{enumerate}
 \item The image of~$G^*$ in $\SL(6,\real) /
\SL(6,\real)_{\Delta(\ints^3)}$ is precompact (by using
the Mahler Compactness Criterion).
 \item The image of~$G^*$ in $\SL(6,\real) /
\SL(6,\real)_{\Delta(\ints^3)}$ is closed.
 \item $G^*/\Gamma^*$~is compact.
 \item $G/\Gamma$~is compact (without using the fact that 
$\Gamma$ is a lattice in~$G$).
 \end{enumerate}
 \hint{This is similar to \cref{anis->cocpct}.}

\item \label{SLnFsigma/Q}
 For any algebraic number field~$F$,
the $\rational$-form $\Delta(F^\ell)$ on $\bigoplus_{\sigma \in
S^\infty} (F_{\sigma})^\ell$ induces a natural $\rational$-form on $\End_{\real} \bigl( \bigoplus_{\sigma
\in S^\infty} (F_{\sigma})^\ell \bigr)$.
 Show the group
 $\prod_{\sigma \in S^\infty} \SL(\ell,F_\sigma)$ is
defined over~$\rational$, with respect to this $\rational$-form.
 \hint{This is a generalization of
\cref{SL2xSL2Defd/Q}. That proof is based on the
elementary symmetric functions of two variables:
$P_1(a_1,a_2) = a_1 + a_2$ and $P_2(a_1,a_2) = a_1 a_2$. For
the general case, use symmetric functions of $d$~variables,
where $d$~is the degree of~$F$ over~$\rational$.}

\item Suppose $G \subseteq \SL(\ell,\real)$, and $G$ is
defined over an algebraic number field $F \subset \real$. Show
$\prod_{\sigma \in S^\infty} G^\sigma$ is defined
over~$\rational$, with respect to the
$\rational$-form on $\End_{\real} \bigl( \bigoplus_{\sigma
\in S^\infty} (F_{\sigma})^\ell \bigr)$ induced by the
$\rational$-form $\Delta(F^\ell)$ on~$\bigoplus_{\sigma \in
S^\infty} (F_{\sigma})^\ell$.
 \hint{This is a generalization of \cref{GxG-Defd/Q}.
See the hint to \cref{SLnFsigma/Q}.}

\item \label{DenseProjSO(n)}
 Show, for all $m,n \ge 1$, with $m + n \ge 3$, that there exist 
a lattice~$\Gamma$ in $\SO(m,n)$,
 and
 a homomorphism $\rho \colon \Gamma \to \SO(m+n)$,
 such that $\rho(\Gamma)$ is dense in $\SO(m+n)$.

\end{exercises}


\section{Only isotypic groups have irreducible lattices}

Intuitively, the \defit{complexification}~$G_\complex$ of~$G$ is the
complex Lie group that is obtained from~$G$ by replacing
real numbers with complex numbers. 
For example, $\SL(n,\real)_\complex = \SL(n,\complex)$, and $\SO(n)_\complex = \SO(n,\complex)$.
(See \cref{RFormsOfCGrps} for more discussion of this.)

\begin{defn} \label{IsotypicDefn}
 $G$ is \defit[isotypic semisimple Lie group]{isotypic} if
all of the simple factors of~$G_\complex$ are isogenous to each other.
 \end{defn}

For example, $\SL(2,\real) \times \SL(3,\real)$ is not
isotypic, because $\SL(2,\complex)$ is not isogenous to $\SL(3,\complex)$.
Similarly, $\SL(5,\real)
\times \SO(2,3)$ is not isotypic, because the complexification of 
 $\SL(5,\real)$ is $\SL(5,\complex)$, but the complexification of $\SO(2,3)$ is (isomorphic to) $\SO(5, \complex)$. Therefore, the
following consequence of the arithmeticity theorem implies
that neither $\SL(2,\real) \times \SL(3,\real)$ nor
$\SL(5,\real) \times \SO(2,3)$ has an irreducible lattice.

\begin{thm}[(Margulis)] \label{irred->isotypic}
 Assume that $G$ has no compact factors. 
If $G$ has an \index{irreducible!lattice}{irreducible lattice}, then $G$ is \term[isotypic semisimple Lie group]{isotypic}.
 \end{thm}

\begin{proof} 
Suppose $\Gamma$ is an irreducible lattice
in~$G$. We may assume that $G$ is not simple (otherwise,
the desired conclusion is trivially true), so $G$~is
neither $\SO(1,n)$ nor $\SU(1,n)$. Therefore, from the Margulis
Arithmeticity Theorem \pref{MargulisArith}, we know that
$\Gamma$ is arithmetic. Then, since $\Gamma$ is irreducible,
\cref{irred->scalars} implies there is a simple subgroup~$G'$ of some $\SL(\ell,\real)$, and a compact group~$K$,
such that
 \begin{itemize}
 \item $G'$ is defined over a number field~$F$, 
 and
 \item $G  \times K$ is isogenous to $\prod_{\sigma \in S^\infty}
(G')^\sigma$.
 \end{itemize}
 So the simple factors of $G \times K$ are all in $\{\, (G')^\sigma \mid \sigma \in S^\infty\,\}$ (up to isogeny). 
 It then follows from \cref{Gsigma=G} below that $G$ is isotypic.
 \end{proof}

\begin{rems} \label{Irred->IsotypicRem} \ 
\noprelistbreak
	\begin{enumerate}
	\item We will prove the converse of \cref{irred->isotypic} in \cref{Irred->Isotypic} (without the assumption that $G$ has no compact factors).
	
	\item By arguing just a bit more carefully, it can be shown
that \cref{irred->isotypic} remains valid when the assumption that $G$ has no compact factors is replaced with the weaker hypothesis that $G$ is not isogenous to $\SO(1,n)
\times K$ or $\SU(1,n) \times K$, for any nontrivial,
connected compact group~$K$ \csee{irred<>isotypic(cpct)}. 

	\end{enumerate}
 \end{rems}

The following example shows that a nonisotypic group can have irreducible lattices, so some restriction on~$G$ is
necessary in \cref{irred->isotypic}.

\begin{eg}
 $\SL(2,\real) \times K$ has an irreducible lattice, for any
connected, compact Lie group~$K$ \ccf{irredinSL2xSO3}.
 \end{eg}

We now complete the proof of \cref{irred->isotypic}:

\begin{lem} \label{Gsigma=G}
 Assume $G$~is defined over an algebraic number field~$F$.
 If $\sigma$ is a place of~$F$,
 and
  $G$ is simple,
  then the complexification of~$G$ is isogenous to the complexification of~$G^\sigma$.
 \end{lem}

%\begin{lem} \label{Gsigma=G(real)}
% If
% \begin{itemize}
% \item $G$~is defined over an algebraic number field $F
%\subset \real$, and
% \item $\sigma$ is an embedding of~$F$ in~$\real$,
% \end{itemize}
% then $G \otimes \complex$ is isogenous to $G^\sigma \otimes
%\complex$.
% \end{lem}

\begin{proof}
 Extend~$\sigma$ to an automorphism~$\widehat\sigma$
of~$\complex$. Then $\widehat\sigma(G_\complex) =
(G^\sigma)_\complex$, so it is clear that $G_\complex$ is isomorphic to $(G^\sigma)_\complex$.
Unfortunately, however, the automorphism~$\widehat\sigma$ is not
continuous (not even measurable) unless it happens to be the
usual complex conjugation, so we have only an isomorphism of
abstract groups, not an isomorphism of Lie groups. Hence, this observation is not
a proof,
although it is suggestive.
To give a rigorous proof, it is easier to work at the Lie algebra
level.

First, let us make an observation that will also be pointed out in \cref{GxC=GxGiff}. If $G = \SL(n,\complex)$, or, more generally, if $G$ is isogenous to a complex group $G'_\complex$, then $G_\complex = G \times G$ (because $\complex \otimes_\real \complex \iso \complex \oplus \complex$). So $G_\complex$ is not simple. However, it can be shown that this is the only situation in which the complexification of a simple group fails to be simple: if $G$ is simple, but $G_\complex$ is not simple, then $G$ is isogenous to a complex simple group $G'_\complex$. 
Therefore, although the complexification of a simple group is not always simple, it is always isotypic.


Now assume, for definiteness, that $F \subset \real$ \csee{CPfOfGsigma=G}.
Since $G$ is defined over~$F$, its Lie algebra~$\Lie G$ is also defined over~$F$. This means there is a basis $\{v_1,\ldots,v_n\}$ of~$\Lie G$, such that the corresponding structure constants $\{c_{j,k}^\ell\}_{j,k,\ell=1}^n$ all belong to~$F$; recall that the structure constants are defined by the formula
 	$$ \textstyle [v_j,v_k] = \sum_{\ell=1}^n c_{j,k}^\ell v_\ell .$$
 
 Because $G$ is isogenous to a group that is defined
over~$\rational$ \csee{hasQform}, there is also a basis
$\{u_1,\ldots,u_n\}$ of~$\Lie G$ whose structure
constants are in~$\rational$. Write
 $ v_k = \sum_{\ell=1}^n \alpha_k^\ell u_\ell $
 with each $\alpha_k^\ell \in \real$, and define
 $$ \textstyle v_k^\sigma = \sum_{\ell=1}^n \widehat\sigma(\alpha_k^\ell)
u_\ell . $$
 Then $v_1^\sigma, \ldots, v_n^\sigma$ is a basis of $\Lie G
\otimes_{\real} \complex$ whose structure constants are
$\bigl\{ \sigma(c_{j,k}^\ell) \bigr\}_{j,k,\ell=1}^n$. These
are obviously the structure constants of the Lie algebra $\Lie G^\sigma$ of~$G^\sigma$.

If $\sigma(F) \subset \real$, then the $\real$-span of $\{v_1^\sigma, \ldots, v_n^\sigma\}$ is (isomorphic to)~$\Lie G^\sigma$, so its $\complex$-span is $\Lie G^\sigma
\otimes_{\real} \complex$. Since $v_1^\sigma, \ldots, v_n^\sigma$ is also a basis of $\Lie G
\otimes_{\real} \complex$, we conclude that $(G^\sigma)_\complex$ is isogenous to~$G_\complex$.

Finally, if $\sigma(F) \not\subset \real$, then the $\complex$-span of $\{v_1^\sigma, \ldots, v_n^\sigma\}$ is (isomorphic to)~$\Lie G^\sigma$, so $\Lie G
\otimes_{\real} \complex = \Lie G^\sigma$. This implies that $(G^\sigma)_\complex$ is isogenous to~$G_\complex$.
 \end{proof}

\begin{rem} \label{Gsigma=GRems} 
The proof of \cref{Gsigma=G} used our standing assumption that $G$~is
semisimple only to show that $G$~is isogenous to a group
that is defined over~$\rational$. 
See \cref{H<>Hsigma} for an example of a Lie
group~$H$, defined over an algebraic number field $F \subset
\real$, and an embedding~$\sigma$ of~$F$ in~$\real$, such
that $H \times H^\sigma$ is not isotypic.
 \end{rem}


\begin{exercises}

\item Show, for $m,n \ge 2$, that $\SL(m,\real) \times
\SL(n,\real)$ has an irreducible lattice if and only if $m =
n$.

\item \label{irred<>isotypic(cpct)}
 Suppose $G$ is not isogenous to $\SO(1,n)
\times K$ or $\SU(1,n) \times K$, for any nontrivial,
connected compact group~$K$. Show that if $G$~has an
irreducible lattice, then $G$~is isotypic.
 \hint{Use \cref{MargArithThmCpctFacts} to modify the proof of \cref{irred->isotypic}.}

\item \label{H<>Hsigma} \optional\ 
 For $\alpha \in \complex \smallsetminus \{0,-1\}$, let
$\Lie H_\alpha$ be the 7-dimensional, nilpotent Lie algebra
over~$\complex$, generated by $\{x_1,x_2,x_3\}$, such that 
 \begin{itemize}
 \item $[\Lie H_\alpha, x_1,x_1] = [\Lie H_\alpha, x_2,x_2]
= [\Lie H_\alpha,x_3,x_3] = 0$, and
 \item $[x_2,x_3,x_1] = \alpha[x_1,x_2,x_3]$.
 \end{itemize}
 \begin{enumerate}
 \item Show that $[x_3,x_1,x_2] = -(1+\alpha) [x_1,x_2,x_3]$.
 \item For $h \in \Lie H_\alpha$, show that $[\Lie
H_\alpha,h,h] = 0$ if and only if there exists $x \in
\{x_1,x_2,x_3\}$ and $t \in \complex$, such that $h \in t x
+ [\Lie H_\alpha,\Lie H_\alpha]$.
 \item Show $\Lie H_\alpha \iso \Lie H_\beta$ iff 
 $\beta \in \left\{\alpha, \frac{1}{\alpha}, -(1+\alpha),
-\frac{1}{1+\alpha},
-\frac{\alpha}{1+\alpha}, -\frac{1+\alpha}{\alpha} \right\} $.
 \item Show that if the degree of $\rational(\alpha)$
over~$\rational$ is at least~$7$, then there is a
place~$\sigma$ of $\rational(\alpha)$,
such that $\Lie H_\alpha$ is not
isomorphic to $(\Lie H_\alpha)^\sigma$.  
 \end{enumerate}

\item \optional\ 
In the notation of \cref{H<>Hsigma}, show that if
the degree of $\rational(\alpha)$ over~$\rational$ is at
least~$7$, then  $\Lie H_\alpha$ is not isomorphic to any
Lie algebra that is defined over~$\rational$.

\item \optional\ 
In the notation of \cref{H<>Hsigma}, show, for
$\alpha = \sqrt{2} - (1/2)$, that $\Lie H_\alpha$ is
isomorphic to a Lie algebra that is defined over~$\rational$.
 \hint{Let $y_1 = x_1 + x_2$ and $y_2 = (x_1 -
x_2)/\sqrt{2}$. Show that the $\rational$-subalgebra
of~$\Lie H_\alpha$ generated by $\{y_1,y_2,x_3\}$ is a
$\rational$-form of~$\Lie H_\alpha$.}

\item \label{CPfOfGsigma=G}
Carry out the proof of \cref{Gsigma=G} for the case where $F \not\subset \real$. 
\hint{Write $\Lie G = \Lie G' \otimes_\real \complex$ and let $\{u_1,\ldots,u_n\}$ be a basis of~$\Lie G'$ with rational structure constants. Show that $G$ is isogenous to either $G^\sigma$ or $(G^\sigma)_\complex$.}

\end{exercises}




\begin{notes}

The fact that $G$ is \term{unirational} (used in \cref{GuniratSoQDense}) is proved in \cite[Thm.~18.2, %(ii), 
p.~218]{Borel-LinAlgGrps}.

The Margulis Arithmeticity Theorem \pref{MargulisArith} was
proved by Margulis \cite{MargulisArithProp, MargulisArith}
under the assumption that $\Rrank G \ge 2$. (Proofs also
appear in \cite[Thm.~A, p.~298]{MargulisBook} and
\cite{ZimmerBook}.) Much later, the superrigidity theorems
of Corlette \cite{Corlette} and Gromov-Schoen
\cite{GromovSchoen} extended this to all groups except
$\SO(1,n)$ and $\SU(1,n)$.
% We remark that these proofs rely on the fact that $\Gamma$
%is finitely generated; Venkataramana \cite{Venky-fg} showed
%how to avoid using this assumption.

\Cref{hasQform} is a weak version of a theorem of Borel \cite{Borel-CK}. (A proof also appears in \cite[Chap.~14]{RaghunathanBook}.)

The Commensurability Criterion (\fullref{GQComm}{criterion}) is due to Margulis \cite{Margulis-DiscGrpMot}. We will see it again in \cref{CommCriterion}, and it is proved in
\cite{A'CampoBurger}, \cite{MargulisBook}, and
\cite{ZimmerBook}.

The fact that all noncocompact lattices have unipotent elements (that is, the generalization of \cref{GodementNoCpctFactor} to the nonarithmetic case) is due to D.\,Kazhdan and G.\,A.\,Margulis \cite{KazhdanMargulis} (or see \cite{Borel-KazhdanMargulisBourbaki} or \cite[Cor.~11.13, p.~180]{RaghunathanBook}).

The standard reference on restriction of scalars is \cite[\S1.3, pp.~4--9]{Weil-AdelesAlgGrps}.
(A discussion can also be found in \cite[\S2.1.2, pp.~49--50]{PlatonovRapinchukBook}.)

\Cref{GZirred->scalars} (and \cref{ROSAbsSimple}) is due to A.\,Borel and J.\,Tits \cite[6.21(ii), p.~113]{BorelTits-GrpRed}.

See \cite[Cor.~IX.4.5, p.~315]{MargulisBook} for a proof of \cref{irred->isotypic}.

\end{notes}



\begin{references}{99}

\bibitem{A'CampoBurger}
 N.\,A'Campo and M.\,Burger:
 R\'eseaux arithm\'etiques et commensurateur d'apr\`es
G.\,A.\,Margulis,
 \emph{Invent. Math.} 116 (1994) 1--25.
 \MR{1253187},
  \maynewline 
 \url{http://eudml.org/doc/144182}
%\url{http://www.digizeitschriften.de/dms/resolveppn/?PPN=GDZPPN002111705}

\bibitem{Borel-CK}
A.\,Borel:
Compact Clifford-Klein forms of symmetric spaces,
\emph{Topology} 2 (1963) 111--122.
\MR{0146301},
\maynewline
\url{http://dx.doi.org/10.1016/0040-9383(63)90026-0}

\bibitem{Borel-KazhdanMargulisBourbaki}
A.\,Borel:
Sous-groupes discrets de groups semi-simples (d'apr\`es D.\,A.\,Kajdan et G.\,A.\,Margoulis),
\emph{S\'eminaire Bourbaki} 1968/1969, % (Juin 1969), 
no.~358.
\emph{Springer Lecture Notes in Math.} 175 (1971) 199--215.
\MR{3077127},
\maynewline
\url{http://eudml.org/doc/109759}
%\url{http://www.numdam.org/item?id=SB_1968-1969__11__199_0}
%\Zbl{0225.22017}

\bibitem{Borel-LinAlgGrps}
A.\,Borel:
\emph{Linear Algebraic Groups, 2nd ed.}.
Springer, New York, 1991.
ISBN 0-387-97370-2,
\MR{1102012}

\bibitem{BorelTits-GrpRed}
A.\,Borel and J.\,Tits:
Groupes r\'eductifs,
\emph{Inst. Hautes \'Etudes Sci. Publ. Math.} 27 (1965) 55--150.
\MR{207712},
\maynewline
\url{http://www.numdam.org/item?id=PMIHES_1965__27__55_0}

\bibitem{Corlette}
 K.\,Corlette:
 Archimedean superrigidity and hyperbolic geometry,
 \emph{Ann. Math.} 135 (1992) 165--182.
 \MR{1147961},
 \maynewline
 \url{http://dx.doi.org/10.2307/2946567}

\bibitem{GromovSchoen}
 M.\,Gromov and R.\,Schoen:
 Harmonic maps into singular spaces and $p$-adic
superrigidity for lattices in groups of rank one,
 \emph{Publ. Math. Inst. Hautes \'Etudes Sci.} 76 (1992) 165--246.
\MR{1215595},
\maynewline
\url{http://www.numdam.org/item?id=PMIHES_1992__76__165_0}

\bibitem{KazhdanMargulis}
D.\,A.\,Ka\v zdan and G.\,A.\,Margulis:
A proof of Selberg's Conjecture,
\emph{Math. USSR--Sbornik} 4 (1968), no.~1, 147--152.
\MR{0223487},
\maynewline
\url{http://dx.doi.org/10.1070/SM1968v004n01ABEH002782}

\bibitem{MargulisArithProp}
 G.\,A.\,Margulis: 
 Arithmetic properties of discrete subgroups, 
 \emph{Russian Math. Surveys} 29:1 (1974) 107--156.
Translated from Uspekhi Mat. Nauk 29:1 (1974) 49--98.
\MR{0463353},
\maynewline
\url{http://dx.doi.org/10.1070/RM1974v029n01ABEH001281}

\bibitem{Margulis-DiscGrpMot}
 G.\,A.\,Margulis: 
 Discrete groups of motions of manifolds of non-positive curvature,
 \emph{Amer. Math. Soc. Translations} 109 (1977) 33--45.
 \MR{0492072} % no URL available @@@

\bibitem{MargulisArith}
 G.\,A.\,Margulis: 
 Arithmeticity of the irreducible lattices in the
semi-simple groups of rank greater than~$1$.
 (Appendix to Russian translation of \cite{RaghunathanBook},
1977.)
 English translation in:
 \emph{Invent. Math.} 76 (1984) 93--120.
 \MR{0739627},
 \maynewline
 \url{http://eudml.org/doc/143118}
%\url{http://www.digizeitschriften.de/dms/resolveppn/?PPN=GDZPPN002100444}

\bibitem{MargulisBook}
 G.\,A.\,Margulis:
 \emph{Discrete Subgroups of Semisimple Lie Groups.}
 Springer, {New York}, 1991.
ISBN 3-540-12179-X,
\MR{1090825}

\bibitem{PlatonovRapinchukBook}
 V.\,Platonov and A.\,Rapinchuk: 
 \emph{Algebraic Groups and Number Theory.}
 Academic Press, Boston, 1994.
 ISBN 0-12-558180-7,
 \MR{1278263}

\bibitem{RaghunathanBook}
 M.\,S.\,Raghunathan: 
 \emph{Discrete Subgroups of Lie Groups.}
 Springer, {New York}, 1972.
 ISBN 0-387-05749-8,
\MR{0507234}

% \bibitem{Tits-Classification}
%J.\,Tits:
%Classification of algebraic semisimple groups,
%in A.\,Borel and G.\,D.\,Mostow, eds.:
%\emph{Algebraic Groups and Discontinuous Subgroups (Boulder, Colo., 1965)},
%Amer. Math. Soc., Providence, R.I., 1966,  pp.~33--62.
%\MR{0224710}

%\bibitem{Venky-fg}
% T.\,N.\,Venkataramana:
% On the arithmeticity of certain rigid subgroups,
% C.~R.~Acad.~Sci.~Paris 316, S\'er.~I (1993) 321--326.
% \MR{1204297},
% \maynewline
% \url{http://gallica.bnf.fr/ark:/12148/bpt6k5471009x/f325.image}

\bibitem{Weil-AdelesAlgGrps}
A.\,Weil:
\emph{Adeles and Algebraic Groups.}
Birkh\"auser, Boston, 1982. 
ISBN 3-7643-3092-9,
\MR{0670072}

\bibitem{ZimmerBook}
 R.\,J.\,Zimmer:
 \emph{Ergodic Theory and Semisimple Groups}.
 Birkh\"auser, Boston, 1984.
 ISBN 3-7643-3184-4,
 \MR{0776417}

 \end{references}