%!TEX root = IntroArithGrps.tex

\mychapter{Basic Facts about Semisimple Lie Groups}
\label{SSGrpsChap}


\section{Definitions} \label{SSLieGrpDefnSect}

We are interested in groups of matrices that are (topologically) closed:

\begin{defns} \label{LieDefn} \ 
\noprelistbreak
	\begin{enumerate}
	\item Let $\Mat_{\ell \times \ell}(\real)$ 
	\nindex{$\Mat_{\ell \times \ell}(\real)$ = $\{ \text{$\ell \times \ell$ matrices with real entries} \}$}%
	be the set of all $\ell \times \ell$ matrices with real entries. This has a natural topology, obtained by identifying it with the Euclidean space $\real^{\ell^2}$. 
	%This is a vector space over~$\real$.
	
	\item Let 
	\nindex{$\SL(\ell,\real)$ = $\{ \text{$\ell \times \ell$ matrices of determinant~$1$} \}$}%
	$\SL(\ell,\real) = \{\, g \in \Mat_{\ell \times \ell}(\real) \mid \det g = 1 \,\}$. This is a group under matrix multiplication \csee{SLisGrp}, and it is a closed subset of $\Mat_{\ell \times \ell}(\real)$ \csee{SLisClosed}.

	\item \label{LieDefn-LieGrp}
	A \defit[Lie!group]{Lie group} is any (topologically) closed subgroup of some $\SL(\ell,\real)$. 
	\end{enumerate}
\end{defns}


Recall that an abstract group is \defit[simple!abstract
group]{simple} if it has no nontrivial, proper, normal
subgroups. For Lie groups, we relax this to allow normal
subgroups that are discrete (except that the
one-dimensional abelian groups $\real$ and~$\torus$ are not
considered to be simple).

\begin{defn} \label{simpleDefn}
A Lie group~$G$ is \defit[simple!Lie group]{simple} if it has no
nontrivial, connected, closed, proper, normal subgroups,
and $G$~is not abelian.
 \end{defn}

\begin{eg}
It can be shown that $G = \SL(\ell,\real)$ is a simple Lie group (when $\ell > 1$). If $\ell$ is even, then $\{\pm \Id\}$ is a subgroup of~$G$, and it is normal,
but, because this subgroup is not connected, it does not
disqualify~$G$ from being simple as a Lie group.
 \end{eg}

\begin{rem}
Although \cref{simpleDefn} only refers to
\emph{closed} normal subgroups, it turns out that, except
for the center, there are no normal subgroups at all:
if $G$ is simple, then every proper, normal subgroup of~$G$ is
contained in the center of~$G$.
\end{rem}

\begin{terminology}
 Some authors say that $\SL(n,\real)$ is
\defit[simple!almost]{almost simple}, and reserve
the term ``simple'' for groups that have no (closed) normal
subgroups at all, not even finite ones.
 \end{terminology}

A Lie group is said to be \emph{semisimple} if it is a direct product of simple groups, modulo passing to a finite-index subgroup and/or modding out a finite group: 

\begin{defns} \label{SSDefn} \ 
\noprelistbreak
	\begin{enumerate}
	\item $G_1$ is \defit{isogenous} to~$G_2$ if there is a finite, normal subgroup~$N_i$ of a finite-index subgroup $G_i'$ of~$G_i$, for $i = 1,2$, such that $G_1'/N_1$ is isomorphic to $G_2'/N_2$.
	\item \label{SSDefn-SS}
	$G$ is \defit[semisimple!Lie group]{semisimple} if it is
isogenous to a direct product of simple Lie groups. That
is, $G$ is isogenous to $G_1 \times \cdots \times G_r$,
where each $G_i$ is simple.
	\end{enumerate}
 \end{defns}
 
 \begin{eg}
 $\SL(2,\real) \times \SL(3,\real)$ is a semisimple Lie group that is not simple (because $\SL(2,\real)$ and $\SL(3,\real)$ are normal subgroups).
 \end{eg}
 
 \begin{rem}[\csee{SSHasNoCenter}]
 If $G$ is semisimple, then the center of~$G$ is finite.
 \end{rem}

\begin{assump}[\ccf{standassump}] \label{FinCompsAssump}
Now that we have the definition of a semisimple group, we will henceforth assume in this chapter that the symbol~$G$ always denotes a semisimple Lie group with only finitely many connected components (but the symbol~$\Gamma$ will never appear).
\end{assump}

\begin{warn}
A \emph{Lie group} is usually defined to be any group that is also a smooth manifold, such that the group operations are~$C^\infty$ functions. \fullCref{HomosAreSmooth}{subgrp} below shows that every closed subgroup of $\SL(\ell,\real)$ is a Lie group in the usual sense. However, the converse is false: not every Lie group (in the usual sense) can be realized as a subgroup of some $\SL(\ell,\real)$. (In other words, not every Lie group is \defit[Lie!group!linear]{linear}.) Therefore, our \fullcref{LieDefn}{LieGrp} is more restrictive than the usual definition.
% (That is, the maps $H \times H \to H \colon (g,h) \mapsto gh$ and $H \to H \colon h \mapsto h^{-1}$ are $C^\infty$.) 
%(Manifolds are assumed to have only countably many connected components, so they are second countable.)
(However, every connected Lie group is ``locally isomorphic'' to a linear Lie group.) 
\end{warn}



\begin{exercises}

\item \label{SLisGrp}
Show that $\SL(\ell,\real)$ is a group under matrix multiplication.
\hint{You may assume (without proof) basic facts of linear algebra, such as the fact that a square matrix is invertible if and only if its determinant is not~$0$.}

\item \label{SLisClosed}
Show that $\SL(\ell,\real)$ is a closed subset of $\Mat_{\ell\times\ell}(\real)$.
\hint{For a continuous function, the inverse image of a closed set is closed.}

\item Recall that
	\nindex{$\GL(\ell,\real)$ = $\{ \text{invertible $\ell \times \ell$ matrices} \}$}%
	$\GL(\ell,\real) = \{\, g \in \Mat_{\ell\times\ell}(\real) \mid \det g \neq 0 \,\}$, and that this is a group under matrix multiplication. Show that it is (isomorphic to) a Lie group, by showing it is isomorphic to a closed subgroup of $\SL(\ell+1,\real)$.

\item \label{Normal=ProdGi}
Suppose $G = G_1 \times \cdots \times G_r$, where each $G_i$~is
simple, and $N$~is a connected, closed, normal subgroup
of~$G$. Show there is a subset~$S$ of
$\{1,\ldots,r\}$, such that $N = \prod_{i \in S} G_i$.
\hint{If the projection of~$N$ to~$G_i$ is all of~$G_i$, then $G_i = [G_i,G_i] = [N, G_i] \subseteq N$.}

\item Show that if $G$ is semisimple, and $N$ is any closed,
normal subgroup of~$G$, then $G/N$ is semisimple.
\hint{\Cref{Normal=ProdGi}.}

\item \label{G=NxH}
 Suppose $N$ is a connected, closed, normal subgroup
of $G$. Show that there is a connected, closed, normal
subgroup~$H$ of $G$, such that $G$~is isogenous to $N \times
H$.
\hint{\Cref{Normal=ProdGi}.}


\end{exercises}





\section{The simple Lie groups}

It is clear from \fullcref{SSDefn}{SS} that the study of semisimple groups requires a good understanding of the simple groups.
Probably the most elementary examples of simple Lie groups are special linear groups and orthogonal groups, but symplectic groups and unitary groups are also fundamental. A group of any of these types is called ``classical\zz.'' (The other simple groups are ``\index{exceptional Lie group}exceptional\zz,'' and are less easy to construct.)

\begin{defn} \label{ClassicalDefn}
 $G$ is a \defit[classical!group]{classical group} if it is isogenous to the
direct product of any collection of the groups constructed in
\cref{classical-fulllinear,classical-orthogonal} below. That is, each simple
factor of~$G$ is either a special linear group or the
isometry group of a bilinear, Hermitian, or skew-Hermitian form, over
$\real$, $\complex$, or~$\quaternion$ (where $\quaternion$
is the algebra of quaternions). 
%See \cref{SOmnIsom,SUmnIsom,SpIsom}. 
 \end{defn}
 

%\begin{warn}
% Contrary to the usage of some authors, we do not
%require a form to be positive-definite in order to be
%called ``Hermitian\zz.''
% \end{warn}

\begin{notation}
Let
\noprelistbreak
% For natural numbers $m$ and~$n$, let  
 \begin{itemize}
 \item \nindex{$g^\transpose$ = transpose of the matrix~$g$}$g^\transpose$ denote the transpose of the
matrix~$g$, 
 \item $g^*$ denote the adjoint (that
is, the conjugate-transpose) of~$g$,
	\item $G^\circ$ denote the identity component of the Lie group~$G$,
	and

%	 \item $I_{m,n}$ be the $(m
%+ n) \times (m + n)$ diagonal matrix whose diagonal entries
%are $m$~$1$'s followed by $n$~$-1$'s,
	 \item \nindex{$\diag(a_1,a_2\ldots,a_n)$ = diagonal matrix with entries $a_1,\ldots,a_n$}
	 $I_{m,n} = \diag(1,1,\ldots,1,-1,-1,\ldots,-1) \in \Mat_{(m+n)\times(m+n)}(\real)$, 
	 \\ % @@@
	 where the number of~$1$'s is~$m$, and the number of~$-1$'s is~$n$.


 \end{itemize}
 \end{notation}

\begin{eg} \label{classical-fulllinear} \ 
\noprelistbreak \nindex{$\SL(n,\real)$, $\SO(m,n)$, $\SU(m,n)$, $\Sp(2m,\real)$: classical Lie groups}
 \begin{enumerate}
 \item The \defit{special linear group}\/ $\SL(n,\real)$ is a
simple Lie group (if $n \ge 2$). 
%It is connected.

 \item \defit[orthogonal!group, special]{Special orthogonal
group}. Let
 $$ \SO(m,n) = \{\, g \in \SL(m+n,\real) \mid g^\transpose
I_{m,n} \, g = I_{m,n} \,\} .$$
This is always semisimple (if $m + n \ge 3$).
It may not be connected, but the identity component $\SO(m,n)^\circ$ is simple if either $m+n = 3$ or $m+n \ge 5$. 
(Furthermore, the index of $\SO(m,n)^\circ$ in $\SO(m,n)$ is $\le 2$.)
%However, $\SO(m+n)$ is
%not connected unless either $m = 0$ or $n = 0$
%\ccf{SO1nNotConn,SOmnNotConn}; otherwise, it has
%exactly two components. 
%See \cref{SO(m+n=4)} for a discussion of the special case where $m + n = 4$.

We use $\SO(n)$ to denote
$\SO(n,0)$ (or $\SO(0,n)$, which is the same group).


 \item \defit[unitary!group, special]{Special unitary
group}. Let
 $$ \SU(m,n) = \{\, g \in \SL(m+n,\complex) \mid g^*
I_{m,n} \, g = I_{m,n} \,\} .$$
 Then $\SU(m,n)$ is simple if $m+n \ge 2$. 
 	%It is connected. 

We use $\SU(n)$ to denote $\SU(n,0)$ (or $\SU(0,n)$).

 \item \defit[symplectic group]{Symplectic group}. Let 
 $$J_{2m} = \begin{pmatrix}
 0 & \Id_{m\times m} \\
 -\Id_{m\times m} & 0
 \end{pmatrix}
 \in \GL(2m,\real) $$
 (where $\Id_{m\times m}$ denotes the $m \times m$ identity
matrix),
 and let 
 $$ \Sp(2m,\real) = \{\, g \in \SL(2m,\real) \mid
g^\transpose J_{2m} \, g = J_{2m} \,\} .$$
 Then $\Sp(2m,\real)$ is simple if $m \ge 1$. 
 %It is connected.
 \end{enumerate}
 \end{eg}

\begin{eg} \label{classical-orthogonal}
 Additional simple groups can be constructed by replacing
the field~$\real$ with either the field~$\complex$ of complex
numbers or the division ring~$\quaternion$ of quaternions:
 \begin{enumerate}
 	\nindex{$\SL(n,\complex)$, $\SL(n,\quaternion)$, $\SO(n,\complex)$, $\SO(n,\quaternion)$, $\Sp(2m,\complex)$, $\Sp(m,n)$: more classical groups}

 \item \defit[special linear group]{Complex and
quaternionic special linear groups}: $\SL(n,\complex)$ and
$\SL(n,\quaternion)$ are simple Lie groups (if $n \ge 2$).
%Each is connected.

 {\it Note:} The noncommutativity of~$\quaternion$ causes
some difficulty in defining the determinant of a
quaternionic matrix. To avoid this problem, we define the
\defit{reduced norm} of a quaternionic $n \times n$
matrix~$g$ to be the determinant of the $2n \times 2n$
complex matrix obtained by identifying $\quaternion^n$
with~$\complex^{2n}$. Then, by definition, $g$ belongs to
$\SL(n,\quaternion)$ if and only if its reduced norm
is~$1$. It is not difficult to see that the reduced norm of a
quaternionic matrix is always a (nonnegative) real number \csee{RedNormPosEx}.

 \item \label{classical-orthogonal-SOCH}
 \defit[orthogonal!group, special]{Complex and
quaternionic special orthogonal groups}: 
 $$\SO(n,\complex) 
 = \{\, g \in \SL(n,\complex) \mid g^\transpose \Id\, g = \Id
\,\} $$
 and
 $$ \SO(n,\quaternion)
 = \{\, g \in \SL(n,\quaternion) \mid \tau_r(g^\transpose)
\Id\, g = \Id \,\} ,$$
 where $\tau_r$ is the \defit[reversion anti-involution]{reversion} on~$\quaternion$
defined by
 $$ \tau_r(a_0 + a_1 i + a_2 j + a_3 k)
 =  a_0 + a_1 i - a_2 j + a_3 k .$$
 (Note that $\tau_r(ab) = \tau_r(b) \, \tau_r(a)$
\csee{tau=Antiaut}; $\tau_r$ is included in the
definition of $\SO(n,\quaternion)$ in order to compensate
for the noncommutativity of~$\quaternion$
\csee{tau(transpose)}.)

 \item \defit[symplectic group!complex]{Complex symplectic group}:
Let
 $$\Sp(2m,\complex) 
 = \{\, g \in \SL(2m,\complex)
 \mid g^\transpose J_{2m} \, g = J_{2m} \,\} .$$

 \item \defit[symplectic group!unitary]{Symplectic unitary
groups}: Let
 $$\Sp(m,n)
 = \{\, g \in \SL(m+n,\quaternion)
 \mid g^* I_{m,n} \, g = I_{m,n} \,\}
 .$$
 Here, as usual, $g^*$ denotes the conjugate-transpose of~$g$; recall that the \defit[conjugate (of a quaternion)]{conjugate} of a quaternion is defined by%
 \nindex{$\overline{x}$ = conjugate of the quaternion~$x$}% no page break here !!!
 	$$ \overline{a + bi + cj + dk} = a - bi - cj - dk $$
(and that $\overline{xy} = \overline{y} \, \overline{x}$).
  We use $\Sp(n)$ to denote $\Sp(n,0)$ (or $\Sp(0,n)$).
 \end{enumerate}
 \end{eg}

\begin{terminology} \label{SOstarTerminology}
 Some authors use
 \begin{itemize}
 \item  $\SU^*(2n)$ to denote $\SL(n,\quaternion)$,
 \item $\SO^*(2n)$ to denote $\SO(n,\quaternion)$,
 or
 \item $\Sp(n,\real)$ to denote $\Sp(2n,\real)$.
 \end{itemize}
 \end{terminology}

\begin{rem} \label{SL2RinG}
 $\SL(2,\real)$ is the smallest connected, noncompact, simple
Lie group; it is contained (up to isogeny) in any other. For example:
 \begin{enumerate}
 
 \item If $\SL(n,\real)$, $\SL(n,\complex)$, or
$\SL(n,\quaternion)$ is not compact, then $n \ge 2$, so the group contains $\SL(2,\real)$.

 \item If $\SO(m,n)$ is semisimple and  not compact, then
$\min\{m,n\} \ge 1$ and $\max\{m,n\} \ge 2$, so it contains $\SO(1,2)$, which is isogenous to $\SL(2,\real)$.

 \item If $\SU(m,n)$ or $\Sp(m,n)$ is not compact, then $\min\{m,n\} \ge
1$, so the group contains $\SU(1,1)$, which is isogenous to $\SL(2,\real)$.
 
\item $\Sp(2m,\real)$ and $\Sp(2m,\complex)$ both contain $\Sp(2,\real)$, which is equal to $\SL(2,\real)$.

\item If $\SO(n,\complex)$ is semisimple and not compact, then $n \ge 3$, so the group contains $\SO(1,2)$, which is isogenous to $\SL(2,\real)$.

\item If $\SO(n,\quaternion)$ is not compact, then $n \ge 2$, so it contains a subgroup conjugate to $\SU(1,1)$, which is isogenous to $\SL(2,\real)$.

 \end{enumerate}
% Note that $\SO(1,2)$ and $\SU(1,1)$ are isogenous to
%$\SL(2,\real)$ \fullsee{isogTypes}{A1SO12}; and $\Sp(1,1)$
%is isogenous to $\SO(1,4)$ \fullsee{isogTypes}{C2Sp11},
%which contains $\SO(1,2)$.
 \end{rem}

%\begin{rem}
% There is some redundancy in the above lists. (For example,
% $\SL(2,\real)$, $\SU(1,1)$, $\SO(1,2)$, and
%$\Sp(2,\real)$ are isogenous to each other
%\fullsee{isogTypes}{A1SO12}.) A complete list of these
%redundancies is given in \cref{ClassicalIsogChap} below.
% \end{rem}

The classical groups are just examples, so one would expect
there to be many other (more exotic) simple Lie groups.
Amazingly, that is not the case --- there are only finitely many others:

\begin{thm}[(\'E.\,Cartan)] \label{RealSimpleGrps}
 Every simple Lie group is isogenous to either 
 \begin{enumerate}
 \item a classical group, or
 \item one of the finitely many exceptional groups.
 \end{enumerate}
 \end{thm}

 See \cref{RFormsOfCGrps,GaloisCohoRealFormsSect} for an indication of the proof of \cref{RealSimpleGrps}.


\begin{exercises}

\item \label{RedNormPosEx} For all nonzero $g \in \Mat_{n \times n}(\quaternion)$, show that the reduced norm of~$g$ is a nonnegative real number.
\hint{Use row and column operations in $\Mat_{n \times n}(\quaternion)$ to reduce to the case where $g$ is upper triangular. For $n = 1$, the reduced norm of~$g$ is $g \, \overline{g}$.}

%\item \label{SO(m+n=4)}
%Show that if $m + n = 4$, then $\SO(m,n)$ is semisimple. More precisely:
% 	\begin{enumerate}
%	\item $\SO(4)$ is isogenous to $\SO(3) \times \SO(3)$, so it is not simple.
%	\item $\SO(1,3)$ is isogenous to $\SL(2,\complex)$, so it is simple.
%	\item $\SO(2,2)$ is isogenous to $\SL(2,\real) \times \SL(2,\real)$, so it is not simple.
%	\end{enumerate}
%\hint{@@@}

%\item \label{SU*=SU}
% For $A \in \GL(\ell,\complex)$, define
% $$ \SU(A) = \{\, g \in \SL(\ell,\complex) \mid
% g^* A g = A \,\} .$$
% Show that if $A^2 = \pm \Id$, then $\SU(A)^* = \SU(A)$.
% \hint{For $g \in \SU(A)$, we have $A =
%(g^*)^{-1} A g^{-1}$. Taking inverses of both sides, and
%noting that $A^{-1} = \pm A$, conclude that $g^* \in
%\SU(A)$.}
%
%\item \label{SO*=SO}
% For $A \in \GL(\ell,\real)$, define
% $$ \SO(A) = \{\, g \in \SL(\ell,\complex) \mid
% g^\transpose A g = A \,\} .$$
% Show that if $A^2 = \pm \Id$, then $\SO(A)^\transpose
%= \SO(A)^* = \SO(A)$.
% \hint{The argument of \cref{SU*=SU} shows that
%$\SO(A)^\transpose = \SO(A)$. Since $A \in \GL(\ell,\real)$,
%it is easy to see that $\SO(A)$ is also closed under complex
%conjugation.}
%
%\item \label{SO(B)=SOmn}
% Let $B$ be a symmetric, invertible $\ell \times \ell$
%real matrix. Define 
% $$\SO(B;\real) = \{\, g \in \SL(\ell,\real) \mid
%g^\transpose B g = B \,\} .$$
% Show $\SO(B;\real) \iso \SO(m,n)$, for some~$m,n$.
%
%\item Let $B$ be a Hermitian, invertible $\ell \times \ell$
%matrix. Define 
% $$\SU(B) = \{\, g \in \SL(\ell,\complex) \mid
%g^* B g = B \,\} .$$
% Show $\SU(B) \iso \SU(m,n)$, for some~$m,n$.
%
%\item Let $B$ be a skew-symmetric, invertible $\ell \times
%\ell$ real matrix. Define 
% $$\Sp(B;\real) = \{\, g \in \SL(\ell,\real) \mid
%g^\transpose B g = B \,\} .$$
% Show $\ell$~is even, and $\Sp(B;\real) \iso
%\Sp(\ell,\real)$.
%
%\item Let $B \in \GL(\ell,\quaternion)$, such that
%$\tau_r(B^\transpose) = B$. Define 
% $$\SU(B;\quaternion,\tau_r) = \{\, g \in
%\SL(\ell,\quaternion) \mid \tau_r(g^\transpose) B g = B
%\,\} .$$
% Show $\SU(B;\quaternion,\tau_r) \iso
%\SO(\ell,\quaternion)$.
%
%\item Let $B \in \GL(\ell,\quaternion)$, such that
%$\tau_c(B^\transpose) = B$. Define 
% $$\SU(B;\quaternion, \tau_c) = \{\, g \in
%\SL(\ell,\quaternion) \mid \tau_c(g^\transpose) B g = B
%\,\} .$$
% Show $\SU(B;\quaternion, \tau_c) \iso
%\Sp(m,n)$, for some~$m,n$.
%
%\item \label{SOmnIsom}
% Define a symmetric bilinear form $\langle \cdot \mid
%\cdot \rangle_{\real^{m,n}}$ on~$\real^{m+n}$ by
% $$\langle x \mid y \rangle_{\real^{m,n}} =
% \sum_{i=1}^m x_i y_i - \sum_{i = 1}^n x_{m+i} y_{m+i} .$$
%Show that
% $$ \SO(m,n) = \{\, g \in \SL(m+n,\real) \mid \forall x,y \in
%\real^{m+n}, \
% \langle gx \mid gy \rangle_{\real^{m,n}}
% = \langle x \mid y \rangle_{\real^{m,n}} \,\} .$$
% \hint{$\langle x \mid y \rangle_{\real^{m,n}} 
% = x^\transpose I_{m,n} y$.}
%
%\item \label{SO1nNotConn}
% Show that $\SO(1,n)$ is not connected.
% \hint{The subset~$X_{1,n}^+$ of~$\real^{n+1}$ (as in
%\cref{HyperModel}) is invariant under
%$\SO(1,n)^\circ$, but there is some $g \in \SO(1,n)$, such
%that $g X_{1,n}^+ = -X_{1,n}^+ \neq X_{1,n}^+$.}
%
%\item \label{SOmnNotConn}
% Show that $\SO(m,n)$ is not connected if $m,n > 0$.
% \hint{Assume $m \le n$, and let $\pi \colon \real^{m+n}
%\to \real^m$ be projection onto the first~$m$ coordinates.
%For any $m$-dimensional, totally
%isotropic subspace~$L$ of $\real^{m+n}$, the linear map
%$\pi|_L$ is a bijection onto~$\real^m$. Show $f(g) = \det
%\bigl( \pi \circ g \circ (\pi|_L)^{-1} \bigr)$ is a
%continuous, surjective function from $\SO(m,n)$ to $\{\pm
%1\}$.}
%
%\item \label{SUmnIsom}
% Define a Hermitian form $\langle \cdot \mid
%\cdot \rangle_{\complex^{m,n}}$ on~$\complex^{m+n}$ by
% $$\langle x \mid y \rangle_{\complex^{m,n}} =
% \sum_{i=1}^m x_i \overline{y_i} - \sum_{i = 1}^n x_{m+i}
%\overline{y_{m+i}} .$$
% (When $n = 0$, this is the usual Hermitian inner product
%on~$\complex^m$.) Show that
% $$ \SU(m,n) = \{\, g \in \SL(m+n,\complex) \mid \forall x,y
%\in \complex^{m+n}, \ \langle gx \mid gy
%\rangle_{\complex^{m,n}}
% = \langle x \mid y \rangle_{\complex^{m,n}} \,\} .$$
%
%\item \label{SpIsom}
% Define a skew-symmetric bilinear form $\langle \cdot \mid
%\cdot \rangle_{\Sp}$ on~$\real^{2m}$ by
% $$\langle x \mid y \rangle_{\Sp} =
% \sum_{i=1}^m ( x_i y_{m+i} - x_{m+i} y_i ) .$$
% Show that
% $$ \Sp(2m,\real) = \{\, g \in \SL(2m,\real) \mid \forall
%x,y \in \real^{2m}, \ \langle gx \mid gy \rangle_{\Sp}
% = \langle x \mid y \rangle_{\Sp} \,\} .$$

\item \label{tau=Antiaut}
 In the notation of \cref{classical-orthogonal},
show that $\tau_r(ab) = \tau_r(b) \, \tau_r(a)$ for all $a,b \in
\quaternion$.
 \hint{Calculate explicitly, or note that $\tau_r(x) =
j \, \overline{x}  j^{-1}$ (and $\overline{xy} = \overline{y} \, \overline{x}$).}

%\item Give an example of two matrices $g,h \in
%\SL(n,\quaternion)$, such that $(gh)^\transpose \neq
%h^\transpose g^{\transpose}$ and $(gh)^\transpose \neq
%g^\transpose h^{\transpose}$.

\item \label{tau(transpose)}
 For $g,h \in \Mat_{n \times n}(\quaternion)$, show that
$\tau_r \bigl( (gh)^\transpose \bigr) = \tau_r(
h^\transpose) \, \tau_r( g^{\transpose})$.

\item Show that $\SO(n,\quaternion)$ is a subgroup of
$\SL(n,\quaternion)$.

%\item \label{conjInvSO}
% For any $g \in \Ortho(\ell)$, the map $\phi_g \colon
%\SO(\ell) \to \SO(\ell)$, defined by $\phi_g(x) = g x
%g^{-1}$, is an automorphism of $\SO(n)$. 
% Show that $\ell$ is odd if and only if, for every $g \in
%\Ortho(\ell)$, there exists $h \in \SO(\ell)$, such that
%$\phi_h = \phi_g$.

\end{exercises}






\section{Haar measure}

Standard texts on real analysis construct a translation-invariant measure on~$\real^n$. this is called \defit[Lebesgue!measure]{Lebesgue measure}, but the analogue for other Lie groups is called ``Haar measure:''

\begin{prop}[(\thmindex{Existence of Haar Measure}{Existence and Uniqueness of Haar Measure})]\index{Haar measure} \label{HaarMeasure}
 If $H$ is any Lie group, then there
is a unique\/ \textup(up to a scalar multiple\textup)
$\sigma$-finite Borel measure~$\mu$ on~$H$, such that
 \begin{enumerate}
 \item \label{HaarMeasure-mu(C)}
 $\mu(C)$ is finite, for every compact subset~$C$
of~$H$, 
%\item \sigma(\open) > 0$, for every nonempty open set~$\open$,
and
 \item $\mu(hA) = \mu(A)$, for every Borel subset~$A$ of~$H$,
and every $h \in H$.
 \end{enumerate}
 \end{prop}
 
 \begin{defns} \ 
 \noprelistbreak
 \begin{enumerate}
 \item The measure~$\mu$ of
\cref{HaarMeasure} is called the \defit[Haar
measure]{left Haar measure} on~$H$.
Analogously, there is a unique \defit[Haar measure!right]{right Haar measure}
with $\mu(Ah) = \mu(A)$ \csee{RightHaarMeasure}.
 \item $H$ is \defit[unimodular group]{unimodular} if the left Haar measure is also a right Haar measure. (This means $\mu(hA) = \mu(Ah) = \mu(A)$.)
 \end{enumerate}
 \end{defns}

\begin{rem}
Haar measure is always \defit[regular!measure, inner]{inner regular}: $\mu(A)$ is the supremum of the measures of the compact subsets of~$A$.
\end{rem}

\begin{prop} \label{modularfuncofG}
 There is a continuous homomorphism $\Delta \colon H \to \real^+$, such that, if $\mu$ is any \textup(left or right\textup) Haar measure on~$H$, then
 	$$ \text{$\mu(h A h^{-1}) = \Delta(h)\, \mu(A)$,
	 for all $h \in H$ and any Borel set $A \subseteq H$}
	 . $$
 \end{prop}

\begin{proof}
Let $\mu$ be a left Haar measure.
 For each $h \in H$, define $\phi_h \colon H \to H$ by $\phi_h(x) = h x h^{-1}$.
Then $\phi_h$ is an automorphism of~$H$, so $(\phi_h)_*\mu$
is a left Haar measure. By uniqueness, we
conclude that there exists $\Delta(h) \in \real^+$, such that
$(\phi_h)_*\mu = \Delta(h) \, \mu$. It is easy to see that
$\Delta$ is a continuous homomorphism. By using the construction of right Haar measure in \cref{RightHaarMeasure}, it is easy to verify that the same formula also applies to it.
 \end{proof} 

\begin{defn}
 The function~$\Delta$ 
 	\nindex{$\Delta$ = modular function of~$H$}
 defined in \cref{modularfuncofG}
is called the \defit{modular function} of~$H$.
 \end{defn}

\begin{terminology}
 Some authors call $1/\Delta$ the modular function, because
they use the conjugation $h^{-1} A h$, instead of $h A
h^{-1}$.
 \end{terminology}

\begin{cor}
 Let $\Delta$ be the modular function of~$H$, and let $A$ be a Borel subset of~$H$. 
 \begin{enumerate}
 \item If $\mu$ is a right Haar measure on~$H$, then
$\mu(hA) = \Delta(h)\, \mu(A)$, for all $h \in H$.
 \item If $\mu$ is a left Haar measure on~$G$, then
$\mu(Ah) = \Delta(h^{-1})\, \mu(A)$, for all $h \in H$.
 \item $H$ is unimodular if and only if $\Delta(h) = 1$, for
all $h \in H$. 
 \item $\Delta(h) = | \det (\Ad_H h)|$ for all $h \in H$ \csee{AdDefn}.
 \end{enumerate}
 \end{cor}

\begin{rem} \label{SS->unimod}
 $G$ is unimodular, because semisimple groups have no nontrivial (continuous)
homomorphisms to $\real^+$ \csee{SS->perfect}.
 \end{rem}

\begin{prop}
 Let $\mu$ be a left Haar measure on a Lie group~$H$. Then $\mu(H) < \infty$ if and only if $H$ is compact.
 \end{prop}

\begin{proof}
 ($\Leftarrow$) See \fullcref{HaarMeasure}{mu(C)}.
 %Haar measure is finite on compact sets \fullcsee{HaarMeasure}{mu(C)}.

($\Rightarrow$) Since $\mu(H) < \infty$ (and the measure $\mu$ is inner regular),
there is a compact subset~$C$ of~$H$, such that $\mu(C) > \mu(H)/2$. 
Then, for any $h \in H$, we have
	$$ \mu(hC) + \mu(C) = \mu(C) + \mu(C) = 2 \mu(C) > \mu(H) ,$$
so $hC$ cannot be disjoint from~$C$. This implies that $h$~belongs to the set $C \cdot C^{-1}$, which is compact. Since $h$~is an arbitrary element of~$H$, we conclude that $H = C \cdot C^{-1}$ is compact.
%
%We prove the contrapositive. Let $C$ be a
%compact subset of nonzero measure. Because $G \times G \to G
%\colon (g,h) \mapsto gh^{-1}$ is continuous, and the
%continuous image of a compact set is compact, we know $C
%C^{-1}$ is compact. Since $G$ is not compact, then there
%exists $g_1 \notin C C^{-1}$; therefore $g_1 C$ is disjoint
%from~$C$. Continuing, we construct, by induction on~$n$, a
%sequence $\{g_n\}$ of elements of~$G$, such that $\{g_n C\}$
%is a collection of pairwise disjoint sets. They all have the
%same measure (since $\mu$ is $G$-invariant), so we conclude
%that 
% $$ \mu(G) \ge \mu \left( \bigcup_{n=1}^\infty g_n C \right) =
%\infty .$$
 \end{proof}

\begin{exercises}

\item Prove the existence (but not uniqueness) of Haar measure on~$H$, without using \cref{HaarMeasure}, under the additional assumption that the Lie group~$H$ is a $C^\infty$ submanifold of $\SL(\ell,\real)$ \fullccf{HomosAreSmooth}{subgrp}.
\hint{For $k = \dim H$, there is a differential $k$-form on~$H$ that is invariant under left translations.}

\item \label{RightHaarMeasure}
Suppose $\mu$ is a left Haar measure on~$H$, and define $\widetilde\mu(A) = \mu(A^{-1})$. Show $\widetilde\mu$ is a right Haar measure.

\item \label{SS->perfect}
% \begin{enumerate}
% \item 
Assume $G$ is connected. Show that if $\phi \colon G \to A$ is a continuous homomorphism, and $A$~is abelian, then $\phi$~is trivial.
% \item Show that $[G,G] = G$.
% \end{enumerate}
\hint{The kernel of a continuous homomorphism is a closed, normal subgroup.}

\end{exercises}




\section{\texorpdfstring{$G$}{G} is almost Zariski closed} \label{ZariskiSect}

%Every classical group has only
%finitely many connected components. This is a special case
%of the following much more general result.

\begin{defns} \  \label{AlgicGrpDefn}
 \noprelistbreak 
 \begin{enumerate}
 \item We use $\real[x_{1,1}, \ldots, x_{\ell,\ell}]$ to
denote the set of real polynomials in the $\ell^2$ variables
$\{\, x_{i,j} \mid 1 \le i,j \le \ell\, \}$. 
 \item For any $Q \in \real[x_{1,1}, \ldots, x_{\ell,\ell}]$,
and any $g \in \Mat_{\ell\times \ell}(\complex)$, we use
$Q(g)$ to denote the value obtained by substituting the
matrix entries~$g_{i,j}$ into the variables~$x_{i,j}$. For
example, if $Q= x_{1,1} x_{2,2} - x_{1,2} x_{2,1}$, then
$Q(g)$ is the determinant of the first principal $2 \times
2$ minor of~$g$.
 \item For any subset~$\mathcal{Q}$ of
$\real[x_{1,1}, \ldots, x_{\ell,\ell}]$, let
  $$\Var(\mathcal{Q}) = \{\, g \in \SL(\ell,\real) \mid
Q(g) = 0, \ \forall Q \in \mathcal{Q} \,\} .$$
 This is the \defit{variety} associated to~$\mathcal{Q}$.
 \item A subset~$H$ of $\SL(\ell,\real)$ is \defit[Zariski!closed]{Zariski
closed} if there exists a subset~$\mathcal{Q}$ of $\real[x_{1,1}, \ldots,
x_{\ell,\ell}]$, such that $H = \Var(\mathcal{Q})$. (In the
special case where $H$ is a sub\emph{group} of
$\SL(\ell,\real)$, we may also say that $H$ is a \defit[algebraic!group!real]{real
algebraic group} or an 
\defit[algebraic!group!over R@over~$\real$]{algebraic group that is defined over~$\real$}.)
 \item The \defit[Zariski!closure]{Zariski closure} of a subset~$H$ of
$\SL(\ell,\real)$ is the (unique) smallest Zariski closed
subset of $\SL(\ell,\real)$ that contains~$H$. This is sometimes denoted~%
	\nindex{$\Zar{H}$ = Zariski closure of~$H$}$\Zar{H}$.
(It can also be denoted $\overline{H}$, if this will not lead to confusion with the closure of~$H$ in the ordinary topology.)
 \end{enumerate}
 \end{defns}

\begin{eg} \label{EgZarClosed}
 \ 
 \begin{enumerate}
 \item $\SL(\ell,\real)$ is Zariski closed. Let $\mathcal{Q}
= \emptyset$.
 \item \label{EgZarClosed-diag}
 The group of diagonal matrices in $\SL(\ell,\real)$
is Zariski closed. Let 
 $\mathcal{Q} = \{\, x_{i,j} \mid i \neq j \,\}$.
 \item \label{EgZarClosed-central}
 For any $A \in \GL(\ell,\real)$, the centralizer
of~$A$ is Zariski closed. Let
 $$ \mathcal{Q} = \bigset{
 \sum_{k=1}^\ell ( x_{i,k} A_{k,j}
-  A_{i,k} x_{k,j}) 
 }{
 1 \le i,j\le \ell 
 }.$$
 \item \label{EgZarClosed-SLnC}
 If we identify $\SL(n,\complex)$ with a subgroup of
$\SL(2n,\real)$, by identifying $\complex$ with~$\real^2$,
then $\SL(n,\complex)$ is Zariski closed, because it is the
centralizer of~$T_i$, the linear transformation in
$\GL(2n,\real)$ that corresponds to scalar multiplication
by~$i$.
 \item The classical groups of \cref{classical-fulllinear,classical-orthogonal} are Zariski closed (if we
identify $\complex$ with~$\real^2$ and $\quaternion$
with~$\real^4$ where necessary).
 \end{enumerate}
 \end{eg}

\begin{terminology} \ 
\noprelistbreak
 \begin{itemize}
 \item Other authors use $\GL(\ell,\real)$ in the definition
of $\Var(\mathcal{Q})$, instead of $\SL(\ell,\real)$. Our
choice leads to no loss of generality, and simplifies the
theory slightly. (In the $\GL$ theory, one should, for
technical reasons, stipulate that the function $1/\det(g)$ is
considered to be a polynomial. In our setting, $\det g$ is the constant
function~$1$, so this is not an issue.)
 \item What we call $\Var(\mathcal{Q})$ is actually only the
\emph{real} points of the variety. Algebraic geometers
usually consider the solutions in~$\complex$, rather
than~$\real$, but our preoccupation with real Lie groups leads
to our emphasis on real points.
 \end{itemize}
 \end{terminology}

\begin{eg} \label{EgNotZarClosed}
 Let 
 $$ H =
 \bigset{
 \begin{pmatrix}
 e^{t} & 0 & 0 & 0\\
 0 & e^{-t} & 0 & 0\\
 0 & 0 & 1 & t \\
 0 & 0 & 0 & 1 \\
 \end{pmatrix}
 }{
 t \in \real
 }
 \subset \SL(4,\real)
 .$$
 Then $H$ is a 1-dimensional subgroup that is not Zariski
closed. Its Zariski closure is
 $$ \Zar{H} =
 \bigset{
 \begin{pmatrix}
 a & 0 & 0 & 0\\
 0 & 1/a & 0 & 0\\
 0 & 0 & 1 & t \\
 0 & 0 & 0 & 1 \\
 \end{pmatrix}
 }{
 \begin{matrix}
 a \in \real \smallsetminus \{0\}, \\
 t \in \real
 \end{matrix}
 }
 \subset \SL(4,\real)
 .$$
 The point here is that the exponential function is
transcendental, not polynomial, so no polynomial can capture
the relation that ties the diagonal entries to the
off-diagonal entry in~$H$. Therefore, as far as polynomials are
concerned, the diagonal entries in the upper left are independent of the
off-diagonal entry, as we see in the Zariski closure.
 \end{eg}

\begin{rem}
 If $H$ is Zariski closed, then the set~$\mathcal{Q}$ of
\cref{AlgicGrpDefn} can be chosen to be finite
(because the ring $\real[x_{1,1}, \ldots, x_{\ell,\ell}]$ is
Noetherian). 
 \end{rem}

Everyone knows that a (nonzero) polynomial in one variable
has only finitely many roots. The following important fact
generalizes this observation to any collection of
polynomials in any number of variables.

\begin{thm} \label{Zar->AlmConn}
 Every Zariski closed subset of\/ $\SL(\ell,\real)$ has only
finitely many connected components.
 \end{thm}

\begin{defn} \label{AlmZarDefn}
 A closed subgroup~$H$ of $\SL(\ell,\real)$ is \emph{almost
Zariski closed} if it has only finitely many components, and
there is a Zariski closed subgroup~$H_1$ of $\SL(\ell,\real)$,
%(which also has only finitely many components, by \cref{Zar->AlmConn}), 
such that $H^\circ = H_1^\circ$.
In other words, in the terminology of \cref{CommensDefn},
$H$~is \defit{commensurable} to a Zariski closed subgroup.
 \end{defn}

\begin{egs} \ 
\noprelistbreak
 \begin{enumerate}
 \item Let $H$ be the group of diagonal matrices in
$\SL(2,\real)$. Then $H$ is Zariski closed
\fullcsee{EgZarClosed}{diag}, but $H^\circ$ is not: any
polynomial that vanishes on the diagonal matrices with
positive entries will also vanish on the diagonal matrices
with negative entries. So $H^\circ$ is almost Zariski closed,
but it is not Zariski closed.

\item Let $G = \SO(1,2)^\circ$. Then $G$ is almost Zariski
closed (because $\SO(1,2)$ is Zariski closed), but $G$ is
not Zariski closed \csee{SO12notZar}.
 \end{enumerate}
 These examples are typical: a connected Lie group is almost Zariski closed if
and only if it is the identity component of a group that is Zariski closed.
 \end{egs}

The following fact gives the Zariski closure a central role
in the study of semisimple Lie groups.

\begin{thm} \label{GisAlgic}
 If $G \subseteq \SL(\ell,\real)$, then $G$ is almost Zariski closed.
 \end{thm}

\begin{proof}
 Let $\Zar{G}$ be the Zariski closure of~$G$. Then
$\Zar{G}$ is semisimple. (For example, if $G$ is
irreducible in $\SL(\ell,\complex)$, then $\Zar{G}$
is also irreducible, so \cref{irred->SS} below implies that
$\Zar{G}^\circ$ is semisimple.)

Since $G$ has only finitely many connected components \csee{FinCompsAssump}, we may assume, by passing to a subgroup of finite index, that it is connected. This implies that the normalizer $\nzer_{\SL(\ell,\real)}(G)$
is Zariski closed \csee{N(G)ZarClosed}. Therefore
$\Zar{G}$ is contained in the normalizer, which means that $G$ is a normal subgroup
of~$\Zar{G}$. 

Hence (up to isogeny), we have
$\Zar{G} = G \times H$, for some closed, normal
subgroup~$H$ of~$\Zar{G}$ \csee{G=NxH}. 
So $G = \czer_{\Zar{G}}(H)^\circ$ is almost Zariski closed
\fullcsee{EgZarClosed}{central}.
 \end{proof}

\begin{warn}
 \Cref{GisAlgic} relies on our standing
assumption that $G$ is semisimple \csee{EgNotZarClosed}.
(Actually, it suffices to know that, besides being
connected, $G$ is perfect; that is, $G = [G,G]$ is equal to
its commutator subgroup.)
 \end{warn}

\begin{exercises}

\item \label{SO12notZar}
 Show that $\SO(1,2)^\circ$ is not Zariski closed.
 \hint{We have
 $$ \frac{1}{2}
 \begin{pmatrix}
 s + \frac{1}{s} & s - \frac{1}{s} & 0 \\
 s - \frac{1}{s} & s + \frac{1}{s} & 0 \\
 0 & 0 & 2
 \end{pmatrix}
 \in \SO(1,2)^\circ
 \qquad \Leftrightarrow \qquad
 s > 0 .$$
 If a rational function $f \colon \real \smallsetminus \{0\}
\to \real$ vanishes on~$\real^+$, then it also vanishes
on~$\real^-$.}

\item \label{N(G)ZarClosed}
 Show that if $H$~is a connected Lie subgroup of
$\SL(\ell,\real)$, then the normalizer $\nzer_{\SL(\ell,\real)}(H)$ is Zariski
closed.
 \hint{$g \in \nzer(H)$ if and only if $g \Lie H g^{-1} = \Lie
H$, where $\Lie H \subseteq \Mat_{\ell\times\ell}(\real)$ is
the Lie algebra of~$H$.}

\item Show that if $\Zar{H}$ is the Zariski closure of a
subgroup~$H$ of~$G$, then $g \Zar{H} g^{-1}$ is the
Zariski closure of $g H g^{-1}$, for any $g \in G$.

\item \label{VarClosed}
 Suppose $G$ is a connected subgroup of
$\SL(\ell,\real)$ that is almost Zariski closed, and that $\mathcal{Q} \subset
\real[x_{1,1},\ldots,x_{\ell,\ell}]$.
 \begin{enumerate}
 \item \label{VarClosed-closed}
 Show that $G \cap \Var(\mathcal{Q})$ is a closed subset of $G$.
 \item \label{VarClosed-nodense}
 Show that if $G \not\subseteq \Var(\mathcal{Q})$, then $G
\cap \Var(\mathcal{Q})$ does not contain any nonempty open
subset of~$G$.
 \item \label{VarClosed-meas0}
 Show that if $G \not\subseteq \Var(\mathcal{Q})$, then $G
\cap \Var(\mathcal{Q})$ has measure zero, with respect to the
Haar measure on~$G$.
 \end{enumerate}
 \hint{For \pref{VarClosed-nodense} and \pref{VarClosed-meas0}, you may assume, 
 without proof, % @@@
 that, for some $d$, there exist 
 	$$ \emptyset = \Var(\mathcal{Q}_{-1}) \subseteq \Var(\mathcal{Q}_0) \subseteq \Var(\mathcal{Q}_1) \subseteq \cdots \subseteq \Var(\mathcal{Q}_d) = \Var(\mathcal{Q}) ,$$
such that $G \cap \bigl( \Var(\mathcal{Q}_k) \smallsetminus \Var(\mathcal{Q}_{k-1}) \bigr)$ is a (possibly empty) $k$-dimensional $C^\infty$ submanifold of~$G$, for $0 \le k \le d$. ($G \cap \Var(\mathcal{Q}_{k-1})$ is called the \defit{singular set} of the variety $G \cap \Var(\mathcal{Q}_k)$.)} 

\item Show, for any subspace~$V$ of~$\real^\ell$, that 
 $$ \Stab_{\SL(\ell,\real)}(V) = \{\, g \in \SL(\ell,\real)
\mid gV = V \,\}$$
 is Zariski closed.

\item \label{IrredCompsOfVariety}
 A Zariski-closed subset of $\SL(\ell,\real)$ is
\defit[irreducible!Zariski-closed subset]{irreducible} if it
cannot be written as the union of two Zariski-closed, proper
subsets. Show that every Zariski-closed subset~$A$ of
$\SL(\ell,\real)$ has a unique decomposition as an
irredundant, finite union of irreducible, Zariski-closed
subsets. (By irredundant, we mean that no one of the sets is
contained in the union of the others.)
 \hint{The ascending chain condition on ideals of
$\real[x_{1,1},\ldots,x_{\ell,\ell}]$ implies the descending
chain condition on Zariski-closed subsets, so $A$~can be
written as a finite union of irreducibles. To make the union
irredundant, the irreducible subsets must be maximal.}

\item \label{Conn->ZarConn}
 Let $H$ be a connected subgroup of $\SL(\ell,\real)$. Show
that if $H \subseteq A_1 \cup A_2$, where $A_1$ and~$A_2$ are
Zariski-closed subsets of $\SL(\ell,\real)$, then either $H
\subseteq A_1$ or $H \subseteq A_2$.
 \hint{The Zariski closure $\Zar{H} = B_1 \cup \cdots
\cup B_r$ is an irredundant union of irreducible,
Zariski-closed subsets \csee{IrredCompsOfVariety}. For $h
\in H$, we have $\Zar{H} = hB_1 \cup \cdots \cup hB_r$,
so uniqueness implies that $h$~acts as a permutation of
$\{B_j\}$. Because $H$~is connected, conclude that
$\Zar{H} = B_1$ is irreducible.}

\item \label{ChevalleyStabEx}
Assume $G$ is connected, and $G \subseteq \SL(\ell,\real)$.
Show there exist
	\begin{itemize}
	\item a finite-dimensional real vector space~$V$,
	\item a vector~$v$ in~$V$,
	and
	\item a continuous homomorphism $\rho \colon \SL(\ell,\real) \to \SL(V)$,
	\end{itemize}
such that $G = \Stab_{\SL(\ell,\real)}(v)^\circ$.
\hint{Let $V_n$ be the vector space of polynomial functions on $\SL(\ell,\real)$, and let $W_n$ be the subspace consisting of polynomials that vanish on~$G$. Then $\SL(\ell,\real)$ acts on~$V_n$ by translation, and $W_n$ is $G$-invariant. For $n$ sufficiently large, $W_n$ contains generators of the ideal of all polynomials vanishing on~$G$, so $G = \Stab_{\SL(\ell,\real)}(W_n)^\circ$. Now let $V$ be the exterior power $\bigwedge^d V_n$, where $d = \dim W_n$, and let $v$ be a nonzero vector in $\bigwedge^d W_n$.}

\item \label{SSHasNoCenter}
Show that the center of~$G$ is finite.
\hint{The identity component of the Zariski closure of~$Z(G)$ is a connected, normal subgroup of~$G$.}

\end{exercises}




\section{Three useful theorems}

\subsection{Real Jordan decomposition}

\begin{defn} \label{hypelluniDefn}
 Let $g \in\GL(n,\real)$. We say that $g$ is
 \begin{enumerate}
 \item \defit[semisimple!element of $G$]{semisimple} if $g$ is
diagonalizable (over~$\complex$),
 \item  \defit[hyperbolic!element of~$G$]{hyperbolic} if
 \begin{itemize}
 \item $g$ is semisimple, and
 \item every eigenvalue of~$g$ is real and positive,
 \end{itemize}
 \item \defit[elliptic element of~$G$]{elliptic} if 
 \begin{itemize}
 \item $g$ is semisimple, and
 \item every eigenvalue of~$g$ is on the unit circle
in~$\complex$,
 \end{itemize}
 \item \defit[unipotent!element]{unipotent} (or
\defit[parabolic!element of~$G$]{parabolic}) if $1$~is the
only eigenvalue of~$g$ over~$\complex$.
 \end{enumerate}
 \end{defn}

\begin{rem} \label{SSeltRem}
 A matrix $g$~is semisimple if and only
if the minimal polynomial of~$g$ has no repeated factors.
 \begin{enumerate}
 \item Because its eigenvalues are real, any hyperbolic~$g$
element is diagonalizable over~$\real$. That is, there is
some $h \in \GL(\ell,\real)$, such that $h^{-1} g h$ is a
diagonal matrix.
 \item \label{SSeltRem-cpct}
 An element is elliptic if and only if it is
contained in some compact subgroup of $\GL(\ell,\real)$. In
particular, if $g$ has finite order (that is, if $g^n = \Id$
for some $n > 0$), then $g$ is elliptic.
 \item A matrix $g \in \GL(\ell,\real)$ is unipotent if and
only if the characteristic polynomial of~$g$ is $(x-1)^\ell$.
(That is, $1$~is the only root of the characteristic
polynomial, with multiplicity~$\ell$.) Another way of saying
this is that $g$~is unipotent if and only if $g - \Id$ is
nilpotent (that is, if and only if $(g - \Id)^n = 0$ for some
$n \in \natural$).
 \end{enumerate}
 \end{rem}
 
 \begin{rem}
 \Cref{SL2RinG} implies that if $G$ is not compact, then it contains nontrivial hyperbolic elements, nontrivial elliptic elements, and nontrivial unipotent elements.
\end{rem}

\begin{prop}[{(\thmindex{Jordan Decomposition, real}Real Jordan Decomposition)}] \label{RealJordanDecomp} 
 Any element~$g$ of~$G$ can be written uniquely
as the product $g = aku$ of three \textbf{commuting} elements
$a,k,u$ of~$G$, such that $a$~is hyperbolic, $k$~is elliptic,
and $u$~is unipotent.
 \end{prop}


\subsection{Engel's Theorem on unipotent subgroups}

\begin{defn} \label{unipDefn}
A subgroup~$U$ of $\SL(\ell,\real)$ is said to be \defit[unipotent!subgroup]{unipotent} if all of its elements are unipotent.
\end{defn}

\begin{eg} \label{Nunip}
Let~$N$ be the group of upper-triangular matrices with $1$'s on the diagonal; that is,
	$$N = \left\{
 	 \begin{bmatrix}
 	1 \\
 	 & 1 &  \vbox to 0pt{\vss \hbox to 0pt{\ \Huge $*$\hss}} \\
 	 &  \vbox to 0pt{\vss\hbox to 0pt{\hss\Huge $0$\ }\vss} & \ddots \\
  	& & & 1
  	\end{bmatrix}
  	\right\}
	\subseteq \SL(\ell,\real) 
	. $$
It is obvious that $N$ is unipotent.
\end{eg}

Therefore, it is obvious that every subgroup of~$N$ is unipotent. Conversely:

\begin{thm}[(\thmindex{Engel's}Engel's Theorem)] \label{EngelUnip}
Every unipotent subgroup of\/ $\SL(\ell,\real)$ is conjugate to a subgroup of the group~$N$ of \cref{Nunip}.
\end{thm}


\subsection{Jacobson-Morosov Lemma}

%We now mention (without proof) a fundamental result that is often useful in the study of Lie groups.

%\begin{defn}
%A matrix $g \in \GL(\ell,\real)$ is unipotent if and
%only if the characteristic polynomial of~$g$ is $(x-1)^\ell$.
%(That is, $1$~is the only root of the characteristic
%polynomial, with multiplicity~$\ell$.) Another way of saying
%this is that $g$~is unipotent if and only if $g - \Id$ is
%nilpotent (that is, if and only if $(g - \Id)^n = 0$ for some
%$n \in \natural$).
%\end{defn}

\begin{thm}[(\thmindex{Jacobson-Morosov}{Jacobson-Morosov Lemma})]
\label{JacobsonMorosov}
 For every unipotent element~$u$ of~$G$, there is a
subgroup~$H$ of~$G$ isogenous to $\SL(2,\real)$, such that
$u \in H$.
\end{thm}


\begin{exercises}

\item \label{NunipEx}
Show that an element of $\SL(\ell,\real)$ is unipotent if and only if it is conjugate to an element of the subgroup~$N$ of \cref{Nunip}.
	\hint{If $g$ is unipotent, then all of its eigenvalues are real, so it can be triangularized over~$\real$.}

\item Show that the Zariski closure of every unipotent subgroup is unipotent. 
%(See \cref{UnipExpLog} for other useful facts about unipotent groups.)
%
%\item \label{UnipExpLog}
%Let 
%	\begin{itemize}
%	\item $N$ be as in \cref{Nunip},
%	\item $\Lie N$ be the space of strictly upper-triangular matrices in $\SL(\ell,\real)$ (with $0$'s on the diagonal),
%	\item $\exp \colon \Lie N \to N$ be defined by $\exp x = \Id + x + x^2 + \cdots + x^\ell$,
%	and
%	\item $\log \colon N \to \Lie N$ be defined by $\Log u = \hat u - \frac{1}{2} \hat u^2 + \frac{1}{3} \hat u^3 - \cdots \pm \hat u^\ell$, where $\hat u %= u - \Id$.
%	\end{itemize}
%Show:
%	\begin{enumerate}
%	
%	\item $\log$ is the inverse of $\exp$. (Since these functions are polynomials, this implies that $\Lie N$ and~$N$ are isomorphic as varieties.)
%	
%	\item For every nontrivial $u \in N$, there is a unique $C^\infty$ homomorphism $f \colon \real \to N$, such that $f(1) = u$. (In other words, $u$~is %contained in a unique one-parameter subgroup of~$N$.)
%
%	\item \label{UnipExpLog-conn}
%	Every connected, unipotent subgroup of $\SL(\ell,\real)$ is Zariski closed.
%	
%	\item \label{UnipExpLog-closed}
%	Every Zariski-closed, unipotent subgroup of $\SL(\ell,\real)$ is connected.
%	
%	\end{enumerate}
%	\hint{@@@}

\end{exercises}



\section{The Lie algebra of a Lie group}

\begin{defn}
 A map~$\rho$ from one Lie group to another is a \defit[homomorphism!of Lie groups]{homomorphism} if%
		\begin{itemize}
		\item it is a homomorphism of abstract groups (i.e., $\rho(ab) = \rho(a) \, \rho(b)$), 
		and
		\item it is continuous.
		\end{itemize}
	(Hence, an \defit[isomorphism!of Lie groups]{isomorphism} of Lie groups is a continuous isomorphism of abstract groups, whose inverse is also continuous.)
\end{defn}

Although the definition only requires homomorphisms to be continuous, it turns out that they are always infinitely differentiable:

\begin{prop} \label{HomosAreSmooth}
Suppose $H_1$ and~$H_2$ are closed subgroups of\/ $\GL(\ell_i,\real)$, for $i = 1,2$. Then
	\begin{enumerate}
	\item \label{HomosAreSmooth-subgrp}
	$H_i$ is a $C^\infty$ submanifold of\/ $\GL(\ell_i,\real)$,
	\item every\/ \textup(continuous\textup) homomorphism from~$H_1$ to~$H_2$ is~$C^\infty$
	and
	\item if $H_1 \subseteq H_2$, then the coset space $H_2/H_1$ is a $C^\infty$ manifold.
	\end{enumerate}
\end{prop}

\begin{rem}
In fact, the submanifolds and homomorphisms are real analytic, not just~$C^\infty$, but we will have no need for this stronger statement.
\end{rem}

\begin{rem} \label{OutGFinite}
If $H$ is any Lie group, then conjugation by any element~$h$ of~$H$ is an automorphism. That is, if we define a map $\varphi_h \colon H \to H$ by $\varphi_h(x) = h^{-1} x h$, then $\varphi_h$ is a continuous automorphism of~$H$. Any such automorphism is said to be ``\defit[inner!automorphism]{inner}\zz.'' The group of all inner automorphisms is isomorphic to $H/Z(H)$, where $Z(H)$ is the center of~$H$. For some groups, there are many other automorphisms. For example, every inner automorphism of an abelian group is trivial, but the automorphism group of~$\real^n$ is $\GL(n,\real)$, which is quite large. In contrast, it can be shown that the group of inner automorphisms of~$G$ has finite index in $\Aut(G)$ (since $G$ is semisimple).
\end{rem}

\begin{defns} \ 
\noprelistbreak
	\begin{enumerate}

	\item For $A,B \in \Mat_{\ell \times \ell}(\real)$, the \defit{commutator} (or \defit[Lie!bracket]{Lie bracket}) of $A$ and~$B$ is the matrix
		$ [A,B] = AB - BA $.
	\item A vector subspace $\Lie H$ of $\Mat_{\ell \times \ell}(\real)$ is a \defit[Lie!algebra]{Lie algebra} if it is closed under the Lie bracket. That is, for all $A,B \in \Lie H$, we have $[A,B] \in \Lie H$.
	\item A map~$\rho$ from one Lie algebra to another is a 
	\defit[homomorphism!of Lie algebras]{homomorphism}
	if
		\begin{itemize}
		\item it is a linear transformation,
		and
		\item it preserves brackets (that is, $[\rho(A), \rho(B)] = \rho \bigl( [A,B] \bigr)$).
		\end{itemize}
	\item Suppose $H$ is a closed subgroup of $\GL(\ell,\real)$. Then $H$ is a $C^\infty$ manifold, so it has a tangent space at every point; the tangent space at the identity element~$e$ is called the \defit[Lie!algebra]{Lie algebra} of~$H$. Note that, since $H$ is contained in the vector space $\Mat_{\ell \times \ell}(\real)$, its Lie algebra can be identified with a vector subspace of $\Mat_{\ell \times \ell}(\real)$.
	\end{enumerate}
\end{defns}

\begin{notation}
Lie algebras are usually denoted by lowercase German letters: the Lie algebras of $G$ and~$H$ are 
	\nindex{$\Lie G$, $\Lie H$ = Lie algebras of $G$ and~$H$}%
$\Lie G$ and~$\Lie H$, respectively.
\end{notation}

\begin{egs} \label{LieAlgEgs} \ 
\noprelistbreak
	\begin{enumerate}
	
%	\item Since $\GL(\ell,\real)$ is open in $\Mat_{\ell \times \ell}(\real)$, the Lie algebra of $\GL(\ell,\real)$ is all of $\Mat_{\ell \times \ell}(\real)$.
	
	\item \label{LieAlgEgs-SL}
	The Lie algebra $\LieSL(\ell,\real)$ of $\SL(\ell,\real)$ is the set of matrices whose trace is~$0$ \csee{LieAlgSLEx}.
	
	\item \label{LieAlgEgs-SO}
	The Lie algebra $\Lie{SO}(n)$ of $\SO(n)$ is the set of $n \times n$ skew-symmetric matrices of trace~$0$ \csee{LieAlgSOEx}.

%	Let $\Ortho(n) = \{\, A \in \Mat_{n \times n}(\real) \mid A^\transpose A = \Id \,\}$, where $A^\transpose$ is the transpose of~$A$, and $\Id$~is the $n \times n$ identity matrix.
	
	\end{enumerate}
\end{egs}

It is an important fact that the Lie algebra of~$H$ is indeed a Lie algebra:

\begin{prop}
If $H$ is a closed subgroup of\/ $\SL(\ell,\real)$, then the Lie algebra of~$H$ is closed under the Lie bracket.
\end{prop}

Here is a very useful reformulation:

\begin{cor}
Suppose $H_1$ and~$H_2$ are Lie groups, with Lie algebras\/ $\Lie H_1$ and\/~$\Lie H_2$. If $H_1$ is a subgroup of~$H_2$, then\/ $\Lie H_1$ is a Lie subalgebra of\/~$\Lie H_2$.
\end{cor}

Hence, for every closed subgroup of~$H$, there is a corresponding Lie subalgebra of~$\Lie H$. Unfortunately, the converse may not be true: although every Lie subalgebra corresponds to a subgroup, the subgroup might not be closed. 

\begin{eg}
The $2$-torus $\torus^2 = \real^2/\integer^2$ can be identified with the Lie group $\SO(2) \times \SO(2)$. For any line through the origin in~$\real^2$, there is a corresponding $1$-dimensional subgroup of~$\torus^2$. However, if the slope of the line is irrational, then the corresponding subgroup of~$\torus^2$ is dense, not closed.
\end{eg}

Therefore, in order to obtain a subgroup corresponding to each Lie subalgebra, we need to allow subgroups that are not closed:

\begin{defn}
Suppose $H_1$ and~$H_2$ are Lie groups, and $\rho \colon H_1 \to H_2$ is a homomorphism. Then $\rho(H_1)$ is a \defit[Lie!subgroup]{Lie subgroup} of~$H_2$.
\end{defn}

\begin{prop}
If $H$ is a Lie group with Lie algebra\/~$\Lie H$, then there is a one-to-one correspondence between the connected Lie subgroups of~$H$ and the Lie subalgebras of\/~$\Lie H$.
\end{prop}

%\begin{notation}
%\nindex{$H^\circ$ = identity component of~$H$}
%$H^\circ$ denotes the \defit{identity component} of the Lie group~$H$ (that is, the connected component of~$H$ that contains the identity element~$e$).
%\end{notation}

%\begin{prop}
%Suppose\/ $\Lie H_1$ and\/~$\Lie H_2$ are the Lie algebras of closed subgroups $H_1$ and~$H_2$ of\/ $\SL(\ell,\real)$. If\/ $\Lie H_1 = \Lie H_2$, then $H_1^\circ = H_2^\circ$.
%\end{prop}

\begin{defns}
Let $H$ be a Lie group in $\SL(\ell,\real)$.
	\begin{enumerate}
	
	\item If $h \colon \real \to H$ is any (continuous) homomorphism, we call $h$ a \defit{one-parameter subgroup} of~$H$, and we usually write $h^t$, instead of $h(t)$.
	
	\item We define $\exp \colon \Mat_{\ell \times \ell}(\real) \to \GL(\ell,\real)$ by
		$$ \exp X = \sum_{k=0}^\infty \frac{1}{k!} X^k .$$
	This is called the \defit{exponential map}.
	\end{enumerate}
\end{defns}

\begin{prop}
Let\/ $\Lie H$ be the Lie algebra of a Lie group $H \subseteq \SL(\ell,\real)$.
	\begin{enumerate}
	\item For any $X \in \Lie H$, the function $x^t = \exp(tX)$ is a one-parameter subgroup of~$H$.
	
	\item Conversely, every one-parameter subgroup of~$H$ is of this form, for some unique $X \in \Lie H$.
	\end{enumerate}
Furthermore, for $X \in \Lie \Mat_{\ell \times \ell}(\real)$, we have \ 
			$$ X \in \Lie H \ \Leftrightarrow \ \forall t \in \real, \ \exp(tX) \in H .$$
\end{prop}

\begin{defn}
Lie groups $H_1$ and~$H_2$ are \defit[locally!isomorphic]{locally isomorphic} if there is a connected Lie group~$H$ and homomorphisms $\rho_i \colon H \to H_i^\circ$, for $i = 1,2$, such that each $\rho_i$ is a covering map.
\end{defn}

\begin{prop} \label{LocIsoIff}
Two Lie groups are locally isomorphic if and only if their Lie algebras are isomorphic. 
\end{prop}

\begin{notation}[(adjoint representation)] \label{AdDefn}
Suppose $\Lie H$ is the Lie algebra of a closed subgroup $H$ of $\SL(\ell,\real)$.
For $h \in H$ and $x \in \Lie H \subseteq \Mat_{\ell \times \ell}(\real)$, we define%
	\nindex{$\Ad_H$ = adjoint representation of~$H$}
	$$ (\Ad_H h)(x) = h x h^{-1} \in \Lie H.$$
Then $\Ad_H \colon H \to \GL(\Lie H)$ is a (continuous) homomorphism. It is called the \defit[adjoint!representation]{adjoint representation} of~$H$.
\end{notation}

%\begin{rem} % \label{Gislinear}
%In most of this book, the assumption that $G$ is linear could
%be replaced with the weaker condition that the center $Z(G)$ is 
%finite. This is because  the kernel of the adjoint representation is
%$Z(G)$, so $G/Z(G)$ is isomorphic to a group of matrices;
%that is, $G/Z(G)$ is always linear. If $Z(G)$ is finite, this implies that
%$G$ is isogenous to $G/Z(G)$, so $G$ is isogenous to a
%linear group.
% \end{rem}



\begin{exercises}

\item \label{RnHomIsLinear}
Suppose $\rho \colon \real^m \to \real^n$ is a continuous map that preserves addition. (That is, we have $\rho(x + y) = \rho(x) + \rho(y)$.) Show (without using \cref{HomosAreSmooth}) that $\rho$ is a linear transformation (and is therefore~$C^\infty$). This is a very special case of \cref{HomosAreSmooth}.
\hint{By assumption, we have $\rho(k x) = k \rho(x)$ for all $k \in \integer$, so $\rho(tx) = t \rho(x)$ for all $t \in \rational$ (why?). Then continuity implies this is true for all $t \in \real$.}

\item \label{LieAlgSLEx}
Verify \fullcref{LieAlgEgs}{SL}.
\hint{$A \in \LieSL(\ell,\real)$ iff $\frac{d}{dt} \det(\Id + tA)\bigr|_{t = 0} = 0$,
and, letting $\lambda = 1/t$, we have 
$\det(\Id + tA) = t^\ell \det(\lambda I + A) = t^\ell \bigl( \lambda^\ell + (\trace A) \lambda^{\ell-1} + \cdots = 1 + (\trace A) t + \cdots$.}

\item \label{LieAlgSOEx}
Verify \fullcref{LieAlgEgs}{SO}.
\hint{A matrix $A$ of trace~$0$ is in $\Lie{SO}(n)$ iff $\frac{d}{dt} (\Id + tA)^\transpose (\Id + tA)\bigr|_{t = 0} = 0$. Calculate the derivative by using the Product Rule.}

\item In the notation of \cref{AdDefn}, show, for all $h \in H$, that $\Ad_H h$ is an automorphism of the Lie algebra~$\Lie H$. (In particular, it is an invertible linear transformation, so it is in $\GL(\Lie H)$.
\hint{The map $a \mapsto h a h^{-1}$ is a diffeomorphism of~$H$ that fixes~$e$, so its derivative is a linear transformation of the tangent space at~$e$.}

\end{exercises}





\section{How to show a group is semisimple}

A semisimple group $G = G_1 \times \cdots G_r$ will often have
connected, normal subgroups (such as the simple
factors~$G_i$). However, these normal subgroups cannot be
abelian \csee{SS->Nnotabel}. The converse is a major
theorem in the structure theory of Lie groups:

\begin{thm} \label{SS<>noAbelNorm}
 A connected Lie group~$H$ is semisimple if and only if it
has no nontrivial, connected, abelian, normal subgroups.
 \end{thm}

\begin{rem}
 A connected Lie group~$R$ is \defit[solvable Lie
group]{solvable} if every nontrivial quotient of~$R$ has a
nontrivial, connected, abelian, normal subgroup. (For
example, abelian groups are solvable.) It can be shown that
every connected Lie group~$H$ has a unique maximal connected,
closed, solvable, normal subgroup. This subgroup is called the
\defit[radical!of a Lie group]{radical} of~$H$, and is
denoted $\Rad H$. Our statement of
\cref{SS<>noAbelNorm} is equivalent to the more usual
statement that $H$~is semisimple if and only if $\Rad H$ is
trivial \csee{SS<>Nnotsolv}. 
 \end{rem}

%\begin{defn}
%To say that $G$ is \defit[Lie!group!linear]{linear} means $G$ is
%isomorphic to a group of matrices; more precisely, $G$ is
%isomorphic to a closed subgroup of $\SL(\ell,\real)$, for
%some~$\ell$. 
% \end{defn}

The following result makes it easy to see that the
classical groups, such as $\SL(n,\real)$, $\SO(m,n)$, and
$\SU(m,n)$, are semisimple (except a few abelian groups in
small dimensions).

\begin{defn} \label{irredrepDefn}
 A subgroup~$H$ of $\GL(\ell,\real)$ (or
$\GL(\ell,\complex)$) is \defit[irreducible!linear group]{irreducible} if there are no nontrivial,
proper, $H$-invariant subspaces of~$\real^\ell$
(or~$\complex^\ell$, respectively).
 \end{defn}

\begin{eg}
 $\SL(\ell,\real)$ is an irreducible subgroup of
$\SL(\ell,\complex)$ \csee{SLnRIrred}.
 \end{eg}

\begin{warn}
 In a different context, the adjective ``irreducible'' can
have a completely different meaning when it is applied to a
group. For example, saying that a lattice is irreducible
(as in \cref{irreducibleLattice}) has nothing to do
with \cref{irredrepDefn}.
 \end{warn}

\begin{rem}
 If $H$ is a subgroup of $\GL(\ell,\complex)$ that is
\emph{not} irreducible (that is, if $H$ is
\defit[reducible!linear group]{reducible}), then, after a
change of basis, we have
 $$ H \subseteq 
 \begin{pmatrix}
 \GL(k,\complex) & * \\
 0 & \GL(n-k,\complex)
 \end{pmatrix}
 ,$$
 for some~$k$ with $1 \le k \le n-1$.

Similarly for $\GL(\ell,\real)$.
 \end{rem}

\begin{cor} \label{irred->SS}
 If $H$ is a nonabelian, closed, connected, irreducible
subgroup of\/ $\SL(\ell,\complex)$, then $H$ is semisimple.
 \end{cor}

\begin{proof}
 Suppose $A$ is a connected, abelian, normal subgroup
of~$H$. For each function $w \colon A \to \complex^\times$,
let 
 $$ V_w = \{\, v \in \complex^\ell \mid \forall a \in A, \
a(v) = w(a) \, v \,\} .$$
 That is, a nonzero vector~$v$ belongs to~$V_w$ if 
 \begin{itemize}
 \item $v$ is an eigenvector for every element of~$A$, and 
 \item the corresponding eigenvalue for each element of~$a$
is the number that is specified by the function~$w$.
 \end{itemize}
 Of course, $0 \in V_w$ for every function~$w$; let $W =
\{\, w \mid V_w \neq 0\,\}$. (This is called the set of
\defit[weights of a representation]{weights} of~$A$
on~$\complex^\ell$.)

Each element of~$a$ has an eigenvector (because $\complex$
is algebraically closed), and the elements of~$A$ all
commute with each other, so there is a common eigenvector
for the elements of~$A$. Therefore, $W \neq \emptyset$. From
the usual argument that the eigenspaces of any linear
transformation are linearly independent, one can show that
the subspaces $\{\,V_w \mid w \in W \,\}$ are linearly
independent. Hence, $W$ is finite.

For $w \in W$ and $h \in H$, a straightforward calculation
shows that 
 $h V_w = V_{h(w)}$, where  $\bigl( h(w) \bigr) (a) =
w(h^{-1} a h)$. That is, $H$ permutes the subspaces
$\{V_w\}_{w \in W}$. Because $H$ is connected and $W$ is
finite, this implies $h V_w = V_w$ for each~$w$; that is,
$V_w$ is an $H$-invariant subspace of~$\complex^\ell$. Since
$H$ is irreducible, we conclude that $V_w = \complex^\ell$.

Now, for any $a \in A$, the conclusion of the preceding
paragraph implies that $a(v) = w(a)\, v$, for all $v \in
\complex^\ell$. Therefore, $a$ is a scalar matrix. 

Since $\det a = 1$, this scalar is an
$\ell^{\text{th}}$~root of unity. So $A$ is a subgroup of
the group of $\ell^{\text{th}}$~roots of unity, which is
finite. Since $A$ is connected, we conclude that $A =
\{e\}$, as desired.
 \end{proof}

Here is another useful characterization of semisimple
groups.

\begin{cor} \label{SelfAdj->SS}
 Let $H$ be a closed, connected subgroup of\/
$\SL(\ell,\complex)$. If
 \begin{itemize} 
 \item the center $Z(H)$ is finite, and
 \item $H^* = H$ \textup(where $*$ denotes the ``adjoint\zz,''
or conjugate-transpose\textup),
 \end{itemize}
 then $H$ is semisimple.
 \end{cor}

\begin{proof}
 Because $H^* = H$, it is not difficult to show that $H$ is
\defit{completely reducible}: there is a
direct sum decomposition 
 $ \complex^\ell = \bigoplus_{j=1}^r V_j $,
 such that the restriction $H|_{V_j}$ is irreducible, for each~$j$
\csee{H*=H->CompRed}.
 
  Let $A$ be a connected, normal subgroup of~$H$. 
The proof of \cref{irred->SS} (omitting the
final paragraph) shows that $A|_{V_j}$ consists of scalar
multiples of the identity, for each~$j$. Hence $A \subset
Z(H)$. Since $A$ is connected, but (by assumption) $Z(H)$
is finite, we conclude that $A$~is trivial.
 \end{proof}

\begin{rem} \label{SS->SelfAdj}
 There is a converse: if $G$ is semisimple (and connected),
then $G$ is conjugate to a subgroup~$H$, such that $H^* =
H$. However, this is more difficult to prove.
 \end{rem}


\begin{exercises}

\item \label{SS->Nnotabel}
 Prove ($\Rightarrow$) of \cref{SS<>noAbelNorm}.

\item \label{SS<>Nnotsolv}
 Show that a connected Lie group~$H$ is semisimple if and
only if $H$ has no nontrivial, connected, solvable, normal
subgroups.
 \hint{If $R$ is a solvable, normal subgroup of~$H$, then
$[R,R]$ is also normal in~$H$. Repeating this eventually yields an abelian, normal subgroup.}

\item \label{SLnRIrred}
 Show that no nontrivial, proper $\complex$-subspace of
$\complex^\ell$ is invariant under $\SL(\ell,\real)$.
 \hint{Suppose $v,w \in \real^\ell$, not both~$0$.
 If they are linearly independent, then there exists $g \in \SL(\ell,\real)$ with
$g(v + iw) = v - iw$. Otherwise, there exists nonzero $\lambda \in
\complex$ with $\lambda(v + iw) \in \real^\ell$.}

\item Give an example of a nonabelian, closed, connected,
irreducible subgroup~$H$ of $\SL(\ell,\real)$, such that $H$ is
not semisimple. 
 \hint{$\mathrm{U}(2)$ is an irreducible subgroup of $\SO(4)$.}

\item Suppose $H \subseteq \SL(\ell,\complex)$. Show that $H$
is completely reducible if and only if, for every
$H$-invariant subspace~$W$ of $\complex^\ell$, there is an
$H$-invariant subspace~$W'$ of~$\complex^\ell$, such that
$\complex^\ell = W \oplus W'$.
 \hint{($\Rightarrow$) If $W' = V_1 \oplus \cdots \oplus
V_s$, and $W' \cap W = \{0\}$, but $(W' \oplus V_j) \cap W
\neq \{0\}$ for every $j > s$, then $W' + W =
\complex^\ell$. ($\Leftarrow$) Let $W$ be maximal among the 
subspaces that are direct sums of irreducibles, and let $V$
be a minimal $H$-invariant subspace of~$W'$. Then $W \oplus
V$ contradicts the maximality of~$W$.}

\item \label{H*=H->CompRed}
 Suppose $H = H^* \subseteq \SL(\ell,\complex)$. 
 \begin{enumerate}
 \item Show that if $W$ is an $H$-invariant subspace of
$\complex^\ell$, then the orthogonal complement $W^\perp$
is also $H$-invariant.
 \item Show that $H$ is completely reducible.
 \end{enumerate}

\end{exercises}







\begin{notes}

See \cite{Howe-VeryBasic} for a very brief introduction to Lie groups, compatible with \fullcref{LieDefn}{LieGrp}. Similar elementary approaches are taken in the books \cite{Baker-MatGrps} and \cite{Hall-LieGrps}.

Almost all of the material in this \lcnamecref{SSGrpsChap} (other than \S\ref{ZariskiSect}) % @@@
 can be found in
Helgason's book \cite{HelgasonBook}. However, we do not
follow Helgason's notation for some of the classical groups
\csee{SOstarTerminology}.
%The lists of isogenies
%in \cref{isogTypes} and \cref{IsogInType} are
%largely copied from there \cite[\S10.6.4,
%pp.~519--520]{HelgasonBook}.

\Cref{GisAlgic} is proved in \cite[Thm.~8.3.2, p.~112]{Hochschild-AlgicGrps}.

%Helgason's book \cite[\S10.2, pp.~447--451]{HelgasonBook}
%proves that all of the (simply connected) classical simple groups except
%$\SO(m,n)$ are connected. For a geometric proof of
%\fullref{isogTypes}{D2-Sp4SO23}, see \cite[\S10.A.2,
%p.~521]{HelgasonBook}.


\Cref{RealJordanDecomp} can be found in
\cite[Lem.~IX.7.1, p.~430]{HelgasonBook}.

See \cite[Prop.~2 in \S11.2 of Chapter~8, p.~166]{BourbakiLie7-9} or \cite[Thm.~3.17, p.~100]{Jacobson-LieAlgs} for a proof of the Jacobson-Morosov Lemma \pref{JacobsonMorosov}.

See \cite[Thm.~2.7.5, p.~71]{VaradarajanBook} for a proof of \cref{LocIsoIff}.

\Cref{SS->SelfAdj} is due to Mostow
\cite{Mostow-SelfAdjoint}.


%The fact that measurable homomorphisms are
%continuous \see{MeasHomoIsCont} was proved by G.~Mackey.
% %% need reference
% A proof  can be found in \cite[Thm.~B.3,
%p.~198]{ZimmerBook}. 

%A proof that continuous homomorphisms are
%real analytic (hence~$C^\infty$) can be found in
%\cite[Prop.~1 of \S4.8, pp.~128--129]{Chevalley-LieGroups}.


\end{notes}




\begin{references}{9}

\bibitem{Baker-MatGrps}
A.\,Baker:
\emph{Matrix Groups}.
%An introduction to Lie group theory. Springer Undergraduate Mathematics Series. 
Springer, London, 2002.
ISBN 1-85233-470-3,
\MR{1869885}


%\bibitem{Borel-AlgicGrps}
% A.\,Borel:
% \emph{Linear Algebraic Groups}, 2nd ed.,
% Springer, New York, 1991.
% \MR{1102012}
 
 \bibitem{BourbakiLie7-9}
N.\,Bourbaki:
Lie Groups and Lie Algebras, Chapters 7--9.
%Translated from the 1975 and 1982 French originals by Andrew Pressley. Elements of Mathematics (Berlin). 
Springer, Berlin, 2005. 
ISBN 3-540-43405-4,
\MR{2109105}

%\bibitem{Chevalley-LieGroups}
% C.\,Chevalley:
% \emph{Theory of Lie Groups,}
% Princeton Univ. Press, Princeton, 1946.
% \MR{0015396}

%\bibitem{Curtis-MatGrps}
%M.\,L.\,Curtis:
%\emph{Matrix Groups}.
%Springer, New York, 1979. 
%ISBN 0-387-90462-X 
%\MR{0550439}

\bibitem{Hall-LieGrps}
B.\,Hall:
Lie Groups, Lie Algebras, and Representations.
%An elementary introduction. Graduate Texts in Mathematics, 222. 
Springer, New York, 2003. 
ISBN 0-387-40122-9,
\MR{1997306}

\bibitem{HelgasonBook} 
S.\,Helgason:
\emph{Differential Geometry, Lie Groups, and Symmetric Spaces.}
% Pure and Applied Mathematics, 80. 
Academic Press, New York, 1978.
ISBN 0-12-338460-5,
\MR{0514561}

%\bibitem{Hochschild-Lie}
% G.\,Hochschild:
% \emph{The Structure of Lie Groups.}
% Holden-Day, San Francisco, 1965.
% \MR{0207883}

\bibitem{Hochschild-AlgicGrps}
 G.\,P.\,Hochschild:
 \emph{Basic Theory of Algebraic Groups and Lie Algebras.}
 %Graduate Texts in Mathematics, 75. 
 Springer, New York, 1981.
 ISBN 0-387-90541-3,
  \MR{0620024}
 

\bibitem{Howe-VeryBasic}
R.\,Howe:
Very basic Lie theory,
\emph{Amer. Math. Monthly} 90 (1983), no.~9, 600--623. 
Correction 91 (1984), no.~4, 247.
\MR{0719752},
\url{http://dx.doi.org/10.2307/2323277}

%\bibitem{Iwasawa-SomeTypes}
% K.\,Iwasawa:
% On some types of topological groups, 
% \emph{Ann. Math.} 50 (1949) 507--558.
% \MR{0029911}
 
\bibitem{Jacobson-LieAlgs}
N.\,Jacobson:
\emph{Lie Algebras}.
%Republication of the 1962 original. 
Dover, New York, 1979.
ISBN 0-486-63832-4,
\MR{0559927}

\bibitem{Mostow-SelfAdjoint}
  G.\,D.\,Mostow: 
 Self adjoint groups,
 \emph{Ann. Math.} 62 (1955) 44--55.
 \MR{0069830},
 \maynewline
 \url{http://www.jstor.org/stable/2007099}

%\bibitem{PlatonovRapinchukBook}
% V.\,Platonov and A.\,Rapinchuk: 
% \emph{Algebraic Groups and Number Theory.}
% Academic Press, Boston, 1994.
%ISBN 0-12-558180-7,
% \MR{1278263}

\bibitem{VaradarajanBook}
 V.\,S.\,Varadarajan: 
 \emph{Lie Groups, Lie Algebras, and their Representations.}
 Springer, {Berlin Heidelberg New York}, 1984. 
 ISBN 0-387-90969-9,
 \MR{0746308}

%\bibitem{Weil-Classical}
% A.\,Weil:
% Algebras with involution and the classical groups,
% \emph{J. Indian Math. Soc.} 24 (1960) 589--623.
% \MR{0136682}

%\bibitem{ZimmerBook}
% R.\,J.\,Zimmer:
% \emph{Ergodic Theory and Semisimple Groups}.
% Birkh\"auser, Boston, 1984.
%ISBN 3-7643-3184-4,
% \MR{0776417}

 \end{references}




