%!TEX root = IntroArithGrps.tex

\mychapter{Kazhdan's Property \texorpdfstring{$(T)$}{(T)}}
\label{KazhdanTChap}

\prereqs{Unitary representations (\cref{UnitaryRepSect,InducedRepSect,RepRnSect}).}
	%Quasi-isometries (\cref{QuasiChap}) are mentioned.

Recall that if a Lie group~$H$ is not amenable, then $\LL2(H)$ does not
have almost-invariant vectors \fullcsee{AmenEquiv}{Vectors}. Kazhdan's property~$(T)$ is the much stronger condition that \textbf{no} unitary representation
of~$H$ has almost-invariant vectors (unless it has a vector that is fixed by~$H$).
Thus, in a sense, Kazhdan's property is the antithesis of amenability.

We already know that $\Gamma$ is not amenable (unless it is finite) \csee{GammaNotAmen}. In this chapter, we will see that $\Gamma$ usually has Kazhdan's Property~$(T)$, and we will look at some of the consequences of this.


\section{Definition and basic properties}

Part~\pref{KazhdanTDefn-alminvt} of the following definition is repeated from \cref{AlmInvtVecDefn}, but the second half is new.

\begin{defn} \label{KazhdanTDefn}
Let $H$ be a Lie group. 
\noprelistbreak
\begin{enumerate}
\item \label{KazhdanTDefn-alminvt}
An action of~$H$ on a normed vector space~$\Banach$ has \defit[almost-invariant vector]{almost-invariant} vectors if, for every
compact subset~$C$ of~$H$ and every $\epsilon > 0$, there
is a unit vector
 $v \in \Banach$, such that 
 \begin{equation} \label{KazhdanTDefn-epsCInvt}
 \|c v - v \| < \epsilon \text{\quad for all $c \in C$}
.\end{equation}
(A unit vector satisfying \pref{KazhdanTDefn-epsCInvt} is said to be $(\epsilon,C)$-invariant.)

\item $H$ has \defit[Kazhdan!property T@property~$(T)$]{Kazhdan's property~$(T)$} if every unitary representation of~$H$ that has almost-invariant vectors also has (nonzero) invariant vectors.
\end{enumerate}
We often abbreviate ``Kazhdan's property~$(T)$'' to ``Kazhdan's property\zz.'' Also, a group that has Kazhdan's property is often said to be a \defit[Kazhdan!group]{Kazhdan group}.
 \end{defn}
 
\begin{warn} \label{TnotBanach}
By definition, unitary representations are actions on Hilbert spaces, so Kazhdan's property says nothing at all about actions on other types of topological vector spaces. In particular, there are actions of Kazhdan groups by norm-preserving linear transformations on some Banach spaces that have almost-invariant vectors, without having invariant vectors \csee{BanachNoInvt}. On the other hand, it can be shown that there are no such examples on $\LL{p}$~spaces (with $1 \le p < \infty$).
\end{warn}

\begin{prop} \label{Kazhdan+amenable}
A Lie group is compact if and only if it is amenable and has Kazhdan's
property.
 \end{prop}

\begin{proof} 
\Cref{Cpct->Kazhdan+amenable,Cpct<-Kazhdan+amenable}.
 \end{proof}
 
 \begin{cor} \label{T+amen->finite}
 A discrete group~$\Lambda$ is finite if and only if it is amenable and has Kazhdan's
property.
\end{cor}

\begin{eg}
$\integer^n$ does not have Kazhdan's property, because it is a discrete, amenable group that is not finite.
\end{eg}

\begin{prop} \label{KazhdanEasy} 
If $\Lambda$ is a discrete group with Kazhdan's property, then:
\begin{enumerate}
\item \label{KazhdanEasy-quotient}
every quotient $\Lambda/N$ of~$\Lambda$ has Kazhdan's property,
\item \label{KazhdanEasy-Abel}
 the abelianization $\Lambda/[\Lambda,\Lambda]$ of~$\Lambda$ is finite,
 and
 \item \label{KazhdanEasy-fg}
 $\Lambda$ is finitely generated.
\end{enumerate}
 \end{prop}
 
 
\begin{proof}
 For \pref{KazhdanEasy-quotient} and \pref{KazhdanEasy-Abel}, see \Cref{KazhdanEasy-quotientPf,KazhdanEasy-AbelPf}.
 
 \pref{KazhdanEasy-fg}
  Let $\{ \Lambda_n \}$ be the collection of all finitely
generated subgroups of~$\Lambda$.
  We have a unitary representation of~$\Lambda$ on each $\LL2(\Lambda/\Lambda_n)$, 
given by $(\gamma f)(x \Lambda_n) = f(\gamma^{-1} x \Lambda_n)$.
The direct sum of these is a unitary representation on
 $$\Hilbert = \LL2(\Lambda/\Lambda_1) \oplus \LL2(\Lambda/\Lambda_2) \oplus
\cdots. $$

Any compact set $C \subseteq \Lambda$ is finite, so we have
$C \subseteq \Lambda_n$, for some~$n$.
 Then $C$ fixes the base point $p = \Lambda_n/\Lambda_n$ in $\Lambda
/\Lambda_n$,
 so, letting $f = \delta_p$ be a nonzero function in
$\LL2(\Lambda/\Lambda_n)$ that is supported on $\{p\}$,
 we have $\gamma f = f$ for all $\gamma \in C$. Therefore,
$\Hilbert$ has almost-invariant vectors, so there must be
an $H$-invariant vector in~$\Hilbert$.

 So some $\LL2(\Lambda/\Lambda_n)$  has an invariant vector.
 Since $\Lambda$ is transitive on $\Lambda/\Lambda_n$, 
  an invariant function must be constant.
 So a (nonzero) constant function is in $\LL2(\Lambda/\Lambda_n)$,
 which means $\Lambda /\Lambda_n$ is finite.
 Because $\Lambda_n$ is finitely generated, this implies that 
$\Lambda $ is finitely generated.
 \end{proof}

Since the abelianization of any (nontrivial) free group is infinite, we have the following example:

\begin{cor} \label{FreeNotT}
 Free groups do not have Kazhdan's property.
 \end{cor}
 
\begin{rem}
\Cref{KazhdanEasy} can be generalized to groups that are not required to be discrete, if we replace the word ``finite'' with ``compact'' (see \cref{KazhdanQuotient,KazhdanAbelianization,Kazhdan->CpctGen}). This leads to the following definition:
\end{rem}

\begin{defn} \label{CpctGenDefn}
A Lie group $H$ is \defit{compactly generated} if there exists a compact subset that generates~$H$. 
\end{defn}

 \begin{warn} \label{TIsQuotOfFP}
 Although discrete Kazhdan groups are always finitely generated \fullcsee{KazhdanEasy}{fg}, they need not be finitely presented. (In fact, there are uncountably many non-isomorphic discrete groups with Kazhdan's property~$(T)$, and only countably many of them can be finitely presented.) However, it can be shown that every discrete Kazhdan group is a quotient of a finitely presented Kazhdan group.
 \end{warn}


\begin{exercises}

\item \label{BanachNoInvt}
Let $C_0(H)$ be the Banach space of continuous functions on~$H$ that tend to~$0$ at infinity (with the supremum norm). Show:
	\begin{enumerate}
	\item $C_0(H)$ has almost-invariant vectors of norm~$1$, 
	but
	\item $C_0(H)$ does not have $H$-invariant vectors other than~$0$, unless $H$ is compact.
	\end{enumerate}
\hint{Choose a uniformly continuous function $f(h)$ that tends to~$+\infty$ as $h$~leaves compact sets. For large~$n$, the function $h \mapsto n/ \bigl( n + f(h) \bigr)$ is almost invariant.}

\item \label{Cpct->Kazhdan+amenable}
Prove \Cref{Kazhdan+amenable}($\Rightarrow$).
\hint{If $H$ is compact, then almost-invariant vectors are invariant.}
 
\item \label{Cpct<-Kazhdan+amenable}
Prove \Cref{Kazhdan+amenable}($\Leftarrow$).
\hint{Amenability plus Kazhdan's property implies $\LL2(H)$ has an invariant vector.}

\item \label{KazhdanEasy-quotientPf}
Prove \fullcref{KazhdanEasy}{quotient}
\hint{Any representation of $\Lambda/N$ is also a representation of $\Lambda$.}

\item \label{KazhdanQuotient}
Show that if $H$ has Kazhdan's property, and $N$ is a closed, normal subgroup of~$H$, then $H/N$ has Kazhdan's property.

\item \label{KazhdanEasy-AbelPf}
Prove \fullcref{KazhdanEasy}{Abel}.
\hint{$\Lambda/[\Lambda,\Lambda]$ is amenable and has Kazhdan's property.}

\item \label{KazhdanAbelianization}
Show that if $H$ has Kazhdan's property, then $H/[H,H]$ is compact.

\item Show that if $N$ is a closed, normal subgroup of~$H$, such that $N$ and $H/N$ have Kazhdan's property, then $H$ has Kazhdan's property.
\hint{The space of $N$-invariant vectors is $H$-invariant (why?).}

\begin{warn*}
The converse is not true: there are examples in which a normal subgroup of a Kazhdan group is not Kazhdan \csee{SL3xR3HasT}.
\end{warn*}

\item Show that $H_1 \times H_2$ has Kazhdan's property if and only if $H_1$ and~$H_2$ both have Kazhdan's property.

\item Let $(\pi,V)$ be a unitary representation of a Kazhdan group~$H$. Show that almost-invariant vectors in~$V$ are near invariant vectors. More precisely, given $\epsilon > 0$, find a compact subset~$C$ of~$H$ and $\delta > 0$, such that if $v$ is any $(\delta,C)$-invariant vector in~$V$, then there is an invariant vector~$v_0$ in~$V$, such that $\| v - v_0 \| < \epsilon$.
\hint{There are no almost-invariant unit vectors in~$(V^H)^\perp$, the orthogonal complement of the space of invariant vectors.}

\item \label{KazhdanConstEx}
Suppose $S$ is a generating set of a discrete group~$\Lambda$, and $\Lambda$~has Kazhdan's property. Show there exists $\epsilon > 0$, such that if $\pi$ is any unitary representation of~$\Lambda$ that has an $(\epsilon,S)$-invariant vector, then $\pi$ has an invariant vector. 
(The point here is to reverse the quantifiers: the same $\epsilon$ works for every~$\pi$.)
Such an~$\epsilon$ is called a \defit[Kazhdan!constant]{Kazhdan constant} for~$\Lambda$.

\item \label{T->NotHaagerup}
Recall that we say $H$ has the \defit{Haagerup property} if it has a unitary representation, such that there are almost-invariant vectors, and all matrix coefficients decay to~$0$ at~$\infty$.
Show that if $H$ is a noncompact group with Kazhdan's property, then $H$ does not have the Haagerup property.

\item Assume:
	\begin{itemize}
	\item $\varphi \colon H_1 \to H_2$ is a homomorphism with dense image,
	and
	\item $H_1$ has Kazhdan's property.
	\end{itemize}
Show $H_2$ has Kazhdan's property.

\item \label{CpctGen<>fg}
Show that a Lie group~$H$ is compactly generated if and only if $H/H^\circ$ is finitely generated.
\hint{($\Leftarrow$) Since $H^\circ$ is connected, it is generated by any subset with nonempty interior.}

\item \label{Kazhdan->CpctGen}
Show that every Lie group with Kazhdan's property is compactly generated.
\hint{Either adapt the proof of \fullcref{KazhdanEasy}{fg}, or use \fullcref{KazhdanEasy}{fg} together with \cref{KazhdanQuotient,CpctGen<>fg}.}

\item \label{Kazhdan->Expander}
Assume $\Gamma$ has Kazhdan's property~$(T)$, and $S$~is a finite generating set for~$\Gamma$. Show there exists $\epsilon > 0$, such that if $N$~is any finite-index normal subgroup of~$\Gamma$, and $A$~is any subset of $\Gamma/N$, then
	$$ \#(SA \cup A) \ge \min \left\{\, (1+\epsilon) \cdot \#A, \ {\textstyle\frac{1}{2}} |\Gamma / N| \,\right\} .$$
{\smaller (In graph-theoretic terminology, this means the Cayley graphs $\mathrm{Cay}(\Gamma/N_k; S)$ form a family of \defit{expander graphs} if $N_1,N_2,\ldots$ are finite-index normal subgroups, such that $|\Gamma/N_k| \to \infty$.)\par}

\end{exercises}

 
 
 
 
 \section{Semisimple groups with Kazhdan's property} \label{SSKazhdanSect}
 
 \begin{thm}[(Kazhdan)] \label{SL3RHasT}
 $\SL(3,\real)$ has Kazhdan's property.
 \end{thm}
 
 This theorem is an easy consequence of the following \lcnamecref{RelTForSL2RxR2}, which will be proved in \cref{RelTForSL2RxR2Sect}.
 
 \begin{lem} \label{RelTForSL2RxR2}
 Assume
 \noprelistbreak
 	\begin{itemize}
	\item $\pi$ is a unitary representation of the natural semidirect product 
		$$\SL(2,\real) \ltimes \real^2
		 =	\begin{Smallbmatrix}
			 \upast & \upast & \upast \\
			 \upast & \upast & \upast \\
			0 & 0 & 1 
			\end{Smallbmatrix}
			\subset \SL(3,\real) ,$$
	and 
	\item $\pi$ has almost-invariant vectors.
	\end{itemize}
Then $\pi$ has a nonzero vector that is invariant under the subgroup\/~$\real^2$.
\end{lem}

\begin{terminology*}
Suppose $R$ is a subgroup of a topological group~$H$. The pair $(H,R)$ is said to have \defit[Kazhdan!property T@property~$(T)$!relative]{relative property~$(T)$} if every unitary representation of~$H$ that has almost-invariant vectors must also have an $R$-invariant vector. In this terminology, \cref{RelTForSL2RxR2} states that the pair $\bigl( \SL(2,\real) \ltimes \real^2, \real^2 \bigr)$ has relative property~$(T)$.
\end{terminology*}

\begin{proof}[Proof of \cref{SL3RHasT}]
Let 
	$$ \text{$G = \SL(3,\real) $,
	\ 
	$R = 
	\begin{Smallbmatrix}
	1 & 0 & \upast \\
	0 & 1 & \upast \\
	0 & 0 & 1 
	\end{Smallbmatrix}
	\iso \real^2$,
	\ 
	and
	\ 
	$H = 
	\SL(2,\real) \ltimes R$}
	, $$
and suppose $\pi$ is a unitary representation of~$G$ that has almost-invariant vectors. Then it is obvious that the restriction of~$\pi$ to~$H$ also has almost-invariant vectors \csee{RestrictAlmInvt}, so \cref{RelTForSL2RxR2} implies there is a nonzero vector~$v$ that is fixed by~$R$.
Then the {Moore Ergodicity Theorem} \pref{MautnerPhenom} implies that $v$ is fixed by all of~$G$.
So $\pi$ has a fixed vector (namely,~$v$).
\end{proof}

If $G$ is simple, and $\Rrank G \ge 2$, then $G$ contains a subgroup isogenous to $\SL(2,\real) \ltimes \real^n$, for some~$n$ \ccf{SL2RxRnInG}, so a modification of the above argument shows that $G$ has Kazhdan's property. On the other hand, it is important to know that not all simple Lie groups have the property:

\begin{eg}
 $\SL(2,\real)$ does \emph{not} have Kazhdan's property.
 \end{eg}
 
 \begin{proof}
 Choose a torsion-free lattice~$\Gamma$ in
$\SL(2,\real)$. Then $\Gamma$ is either a surface group or
a nonabelian free group. In either case,
$\Gamma/[\Gamma,\Gamma]$ is infinite, so $\Gamma$ does not
have Kazhdan's property. Therefore, we conclude from
\cref{Kazhdan:G->Gamma} below that $\SL(2,\real)$ does not
have Kazhdan's property.
\end{proof}

\begin{proof}[Alternate proof]
A reader familiar with the unitary representation theory of $\SL(2,\real)$ can easily construct a sequence of representations in the principal series  whose limit is the trivial representation. The 
direct sum of this sequence of representations has almost-invariant vectors.
 \end{proof}

We omit the proof of the following precise characterization of the semisimple groups that have Kazhdan's property:

\begin{thm} \label{WhichGKazhdan}
 $G$ has Kazhdan's property if and only if no simple
factor of~$G$ is isogenous to\/ $\SO(1,n)$ or\/ $\SU(1,n)$.
\end{thm}

\begin{exercises}

\item \label{RestrictAlmInvt}
Assume $\pi$ is a unitary representation of~$H$ that has almost-invariant vectors, and $L$~is a subgroup of~$H$. Show that the restriction of~$\pi$ to~$L$ has almost-invariant vectors. 

\item \label{SL2RxRnInG}
(\emph{Assumes familiarity with real roots})
Assume $G$ is simple. Show $\Rrank G \ge 2$ if and only if some connected subgroup~$L$ of~$G$ is isogenous to $\SL(2,\real)$ and normalizes (but does not centralize) a nontrivial, unipotent subgroup~$U$ of~$G$.
\hint{($\Rightarrow$)~An entire maximal parabolic subgroup of~$G$ normalizes a nontrivial unipotent subgroup.
($\Leftarrow$)~Construct two unipotent subgroups of~$G$ that both contain~$U$, but generate a subgroup that is not unipotent.}

\item
Suppose $H$ is a closed, noncompact subgroup of~$G$, and $G$~is simple.
Show that the pair $(G,H)$ has relative property $(T)$ if and only if $G$ has Kazhdan's property.

\item Suppose $G$ has Kazhdan's property. Show there is a compact subset~$C$ of~$G$ and some $\epsilon > 0$, such that every unitary representation of~$G$ with $(\epsilon,C)$-invariant vectors has invariant vectors.

\end{exercises}




\section{Proof of relative property \texorpdfstring{$(T)$}{(T)}} \label{RelTForSL2RxR2Sect}

In this \lcnamecref{RelTForSL2RxR2Sect}, we prove \cref{RelTForSL2RxR2}, thereby completing the proof that $\SL(3,\real)$ has Kazhdan's property~$(T)$. The argument relies on a decomposition theorem for representations of~$\real^n$.
%, or as a consequence of elementary multivariate Fourier analysis that we  (or, in our terminology,) that will not be needed elsewhere in the book. 

%\begin{notation}
%For convenience, in this \lcnamecref{RelTForSL2RxR2Sect}, we let $R = \real^2$.
%\end{notation}

\begin{proof}[Proof of \cref{RelTForSL2RxR2}]
For convenience, let $H = \SL(2,\real) \ltimes \real^2$. % and $R = \real^2 \normal H$. 
Given a unitary representation $(\pi,\Hilbert)$ of~$H$ that has almost-invariant vectors, we wish to show that some nonzero vector in~$\Hilbert$ is fixed by the subgroup~$\real^2$ of~$H$. In other words, if we let $E$ be the projection-valued measure provided by \cref{ProjValMeas} (for the restriction of~$\pi$ to~$\real^2$), then we wish to show $E \bigl( \{0\} \bigr)$ is nontrivial.

Letting 
	\nindex{$\bddop(\Hilbert)$ = $\{\text{bounded operators on~$\Hilbert$} \}$}%
	$\bddop(\Hilbert)$ 
	be the algebra of bounded linear operators on~$\Hilbert$, and using the fact that $\pi$ has almost-invariant vectors, \cref{ConstructPhiEx} provides a continuous, linear functional $\lambda \colon \bddop(\Hilbert) \to \complex$, such that 
	\begin{itemize}
	\item $\lambda(\Id) = 1$,
	\item $\lambda( E ) \ge 0$ for every orthogonal projection~$E$,
	and
	\item $\lambda$ is bi-invariant under $H$. (More precisely, for all $h_1,h_2 \in H$ and $T \in \bddop(\Hilbert)$ we have $\lambda \bigl( \pi(h_1) \, T  \, \pi(h_2) \bigr) = \lambda(T)$.)
	\end{itemize}

Now, let $\mu$ be the composition of $\lambda$ with~$E$ (that is, let $\mu(A) = \lambda \bigl(E(A) \bigr)$ for $A \subseteq \real^n$), so $\mu$ is a finitely additive probability measure on~$\real^n$ \csee{MuIsFinAdd}. Since $\real^2 \normal H$, there is an action of~$H$ on~$\real^2$ by conjugation. One can show that the probability measure~$\mu$ is invariant under this action \csee{muIsHInvtEx}.

On the other hand, the only $\SL(2,\real)$-invariant, finitely additive probability measure on~$\real^2$ is the point-mass supported at the origin \csee{InvtMeanOnR2}. 
Therefore, we must have $\mu \bigl( \{0\} \bigr) = 1 \neq 0$. Hence, $E \bigl( \{0\} \bigr)$ is nonzero, as desired. 
\end{proof}

\begin{exercises}

\item \label{ConstructPhiEx}
Prove the existence of the linear functional $\lambda \colon \bddop(\Hilbert) \to \complex$ in the proof of \cref{RelTForSL2RxR2}.
\hint{For $T \in \bddop(\Hilbert)$, define $\lambda_n(T) = \langle Tv_n \mid v_n \rangle$, where $\{v_n\}$ is a sequence of unit vectors in~$\Hilbert$, such that $\| \pi(h) v_n - v_n\| \to 0$ for every $h \in H$. Let $\lambda$ be an accumulation point of $\{\lambda_n\}$ in an appropriate weak topology.}

\item \label{MuIsFinAdd}
Let $\mu$ be as defined near the end of the proof of \cref{RelTForSL2RxR2}. Show:
	\begin{enumerate}
	\item $\mu(\real^2) = 1$.
	\item If $A_1$ and~$A_2$ are disjoint Borel subsets of~$\real^2$, then we have $\mu(A_1 \cup A_2) = \mu(A_1) + \mu(A_2)$.
	\item $\mu(A) \ge 0$ for every Borel set $A \subseteq \real^2$.
	\end{enumerate}

\item \label{muIsHInvtEx}
Show the finitely additive measure~$\mu$ in the proof of \cref{RelTForSL2RxR2} is invariant under the action of~$H$ on~$\dual R$.
\hint{Since 
	\begin{align*}
	\int_{\dual R} \tau(r) \, dE(\tau^h)
	&= \int_{\dual R} \tau^{h^{-1}}(r) \, dE(\tau)
	= \int_{\dual R} \tau(h^{-1} r h) \, dE(\tau)
	= \pi(h^{-1} r h)
	\\&= \pi(h^{-1}) \, \pi(r) \, \pi(h)
	= \int_{\dual R} \tau(r) \, \bigl( \pi(h^{-1}) \,  dE(\tau) \, \pi(h) \bigr) ,
	\end{align*}
we have $E(A^h) = \pi(h^{-1}) \,  E(A)  \, \pi(h)$ for $A \subseteq \dual R$.}

\item \label{InvtMeanOnR2}
Show that any $\SL(2,\real)$-invariant, finitely additive probability measure~$\mu$ on~$\real^2$ is supported on $\{(0,0)\}$.
\hint{Let $V = \{\, (x,y) \mid y > |x| \,\}$ and $h = \text{\smaller[2]$\begin{bmatrix} 1 & 2 \\ 0 & 1 \end{bmatrix}$}$. Then $h^iV$ is disjoint from $h^j V$ for $i \neq j \in \integer^+$, so $\mu(V) = 0$. All of $\real^2 \smallsetminus \{(0,0
)\}$ is covered by finitely many sets of the form $hV$ with $h \in \SL(2,\real)$.}

\item \label{SL3xR3HasT}
Show that the natural semidirect product $\SL(3,\real) \ltimes \real^3$ has Kazhdan's property.
\hint{We know $\SL(3,\real)$ has Kazhdan's property, and the proof of \cref{RelTForSL2RxR2} shows that the pair $\bigl( \SL(3,\real) \ltimes \real^3 , \real^3 \bigr)$ has relative property~$(T)$.}

\item Show that the direct product $\SL(3,\real) \times \real^3$ does \emph{not} have Kazhdan's property. (Comparing this with \cref{SL3xR3HasT} shows that, for group extensions, Kazhdan's property may depend not only the groups involved, but also on the details of the particular extension.)

\end{exercises}



\section{Lattices in groups with Kazhdan's property} \label{LattInTSect}

In this \lcnamecref{LattInTSect}, we will use basic properties of induced representations to prove the following important result:

\begin{prop} \label{Kazhdan:G->Gamma}
 If $G$ has Kazhdan's property, then  $\Gamma$ has
Kazhdan's property.
 \end{prop}
  
 Combining this with \cref{WhichGKazhdan}, we obtain:
 
\begin{cor} \label{GammaHasT}
\label{Kazhdanlattice}
 If no simple factor of~$G$ is isogenous to\/ $\SO(1,n)$ or\/
$\SU(1,n)$, then\/  $\Gamma$ has Kazhdan's property.
\end{cor}

By \cref{KazhdanEasy}, this has two important consequences:

\begin{cor} \label{KazhdanlatticeCor}
 If no simple factor of~$G$ is isogenous to\/ $\SO(1,n)$ or\/
$\SU(1,n)$, then 
 \begin{enumerate}
\item \label{KazhdanlatticeCor-fg}
 $\Gamma $ is finitely generated,
 and
 \item \label{KazhdanlatticeCor-noabel}
 $\Gamma/[\Gamma ,\Gamma ]$ is finite.
   \end{enumerate}
\end{cor}

\begin{rems} \label{KazhdanLattRem} \ 
\noprelistbreak
\begin{enumerate}
\item It was pointed out in \cref{GammaFinGen} that \pref{KazhdanlatticeCor-fg} remains true without any assumption on the simple factors of~$G$. In fact, $\Gamma$ is always finitely presented, not merely finitely generated.
\item \label{KazhdanLattRem-abel}
On the other hand, \pref{KazhdanlatticeCor-noabel} is not always true, because lattices in $\SO(1,n)$ and $\SU(1,n)$ can have infinite abelian quotients. (In fact, it is conjectured that every lattice in $\SO(1,n)$ has a finite-index subgroup with an infinite abelian quotient, and this is known to be true when $n = 3$.) The good news is that the Margulis Normal Subgroup Theorem implies these are the only examples (modulo multiplying $G$ by a compact factor) if we make the additional assumption that $\Gamma$ is irreducible (see \cref{GNoAbelianization} or \cref{GammaHasFiniteAbelianization}).
\end{enumerate}
\end{rems}

The proof of \cref{Kazhdan:G->Gamma} uses some machinery from the theory of unitary representations.

\begin{notation}
 Let $(\pi, \Hilbert)$ and $(\sigma,\mathcal{K})$ be unitary representations of a Lie group~$H$. (In our applications, $H$~will be either $G$ or~$\Gamma$.)
 \begin{enumerate}
 
 \item We write 
 	\nindex{$\sigma \le \pi$ means $\sigma$ is a subrepresentation of~$\pi$}%
	$\sigma \le \pi$ if $\sigma$ is (isomorphic to) a \defit{subrepresentation} of~$\pi$. 
This means there exist
	\begin{itemize}
	\item a closed, $H$-invariant subspace $\Hilbert'$ of~$\Hilbert$,
	and
	\item a bijective, linear isometry $T \colon \mathcal{K} \stackrel{\iso}{\longrightarrow} \Hilbert'$,
	\end{itemize}
such that $T \bigl( \sigma(h) \phi \bigr) = \pi(h) \, T(\phi)$, for all $h \in H$ and $\phi \in \mathcal{K}$.

 \item We write 
  	\nindex{$\sigma \wl \pi$ means $\sigma$ is weakly contained in~$\pi$}%
	$\sigma \wl \pi$ if $\sigma$ is \defit{weakly contained} in~$\pi$. This means that, for every compact set $C$ in~$H$, every $\epsilon > 0$,
 and all unit vectors $\phi _1, \ldots,\phi_n  \in \mathcal{K}$,
 there exist unit vectors
$\psi_1,\ldots,\psi_n \in \Hilbert$, such that, for all $h \in C$ and
all $1 \le i,j \le n$, we have
 $$  \bigl| \langle \sigma(h) \phi_i \mid \phi_j \rangle - \langle \pi(h) \psi_i 
\mid  \psi_j  \rangle \bigr| < \epsilon . $$
 \end{enumerate}
\end{notation}

\begin{rems} \label{WeakContainRem}\ 
\noprelistbreak
	\begin{enumerate}
	\item \label{WeakContainRem-weaker}
	It is obvious that if $\sigma \le \pi$, then $\sigma \wl \pi$. 
	
	\item We have:
		\begin{itemize}
		\item $\pi$ has invariant vectors if and only if $\trivrep\le \pi$,
		and
		\item $\pi$ has almost-invariant vectors if and only if $\trivrep \wl \pi$. 
		\end{itemize}
	Therefore, Kazhdan's property asserts the converse to \pref{WeakContainRem-weaker} in the special case where $\sigma = \trivrep$: for all~$\pi$, if $\trivrep \wl \pi$, then $\trivrep \le \pi$.
	\end{enumerate}
\end{rems}

It is not difficult to show that induction preserves weak containment \csee{IndWeakContEx}:

\begin{lem} \label{IndWeakCont}
If $\sigma \wl \pi$, then $\Ind_\Gamma ^G(\sigma ) \wl  \Ind_\Gamma ^G(\pi )$.
\end{lem}

This (easily) implies the main result of this \lcnamecref{LattInTSect}:

\begin{proof}[Proof of \cref{Kazhdan:G->Gamma}]
Suppose a representation $\pi$ of~$\Gamma $ has
almost-invariant vectors.
 Then $\pi \wg \trivrep$, so 
  $$ \Ind_\Gamma ^G(\pi ) \wg \Ind_\Gamma ^G(\trivrep) =
\LL2(G/\Gamma ) \ge \trivrep $$
\csee{Ind1=L2(G/Gamma),1<L2(G/Gamma)}.
 Because $G$ has Kazhdan's property, we conclude that
$\Ind_\Gamma ^G(\pi ) \ge \trivrep$.
 This implies $\pi \ge \trivrep$ \csee{1<Ind->1<rho}, as desired.
 \end{proof}
 
\begin{rem} \label{ExplicitKazhdanConstants}
If $\Gamma$ has Kazhdan's property, and $S$ is any generating set of~$\Gamma$, then there is some $\epsilon > 0$, such that every unitary representation of~$\Gamma$ with an $(\epsilon,S)$-invariant unit vector must have invariant vectors \csee{KazhdanConstEx}. Our proof does not provide any estimate on~$\epsilon$, but, in many cases, including $\Gamma = \SL(n,\integer)$, an explicit value of~$\epsilon$ can be obtained by working directly with the algebraic structure of~$\Gamma$ (rather than using the fact that $\Gamma$ is a lattice).
\end{rem}

\begin{rem} \label{OtherT}
For many years, lattices (and some minor modifications of them) were the only discrete groups known to have Kazhdan's property~$(T)$, but other constructions are now known. In particular:
\noprelistbreak
	\begin{enumerate}
	\item \label{OtherT-random}
	Groups can be defined by generators and relations. It can be shown that if the relations are selected at random (with respect to a certain probability distribution), then the resulting group has Kazhdan's property~$(T)$ with high probability.
	\item \label{OtherT-Alg}
	An algebraic approach that directly proves Kazhdan's property for $\SL(n,\integer)$, without using the fact that it is a lattice, has been generalized to allow some other rings, such as polynomial rings, in the place of~$\integer$. In particular, $\SL \bigl( n, \integer[X_1,\ldots,X_k] \bigr)$ has Kazhdan's property~$(T)$ if $n \ge k + 3$.
	\end{enumerate}
\end{rem}



\begin{rem} \label{TnotQI}
We saw in \cref{AmenQIinvt} that amenability is invariant under quasi-isometry \csee{QuasiIsomDefn}. In contrast, this is not true for Kazhdan's property~$(T)$.
To see this, let 
\noprelistbreak
	\begin{itemize}
	\item $G$ be a simple group with Kazhdan's property~$(T)$, 
	\item $\cover G$ be the universal cover of~$G$,
	\item $\Gamma$ be a cocompact lattice in~$G$,
	and
	\item $\cover\Gamma$ be the inverse image of~$\Gamma$ in~$\cover G$, so $\cover\Gamma$ is a lattice in~$\cover G$.
	\end{itemize}
Then $\cover\Gamma$ has Kazhdan's property~$(T)$ (because $\cover G$ has the property). However, if $G = \Sp(4,\real)$ (or, more generally, if the fundamental group of~$G$ is an infinite cyclic group), then $\cover\Gamma$ is quasi-isometric to $\Gamma \times \integer$, which obviously does not have Kazhdan's property (because its abelianization is infinite).

Here is a brief explanation of why $\cover\Gamma$ is quasi-isometric to $\Gamma \times \integer$. Note that $\cover\Gamma/\integer \iso \Gamma$ yields a $2$-cocycle $\alpha \colon \Gamma \times \Gamma \to \integer$ of group cohomology. 
Since $G/\Gamma$ is compact, it turns out that $\alpha$ can be chosen to be uniformly bounded, as a function on $\Gamma \times \Gamma$. This implies that the extension~$\cover\Gamma$ is quasi-isometric to the extension corresponding to the trivial cocycle. This extension is $\Gamma \times \integer$.
\end{rem}



\begin{exercises}

\item \label{IndWeakContEx}
Prove \cref{IndWeakCont}.

\item \label{1<L2(G/Gamma)}
Show $\trivrep \le \LL2(G/\Gamma)$.

\item \label{1<Ind->1<rho}
Show that if $\pi$ is a unitary representation of~$\Gamma$, and $\trivrep \le \Ind_\Gamma^G(\pi)$, then $\trivrep \le \pi$.

\item \label{GammaT->GT}
Prove the converse of \cref{Kazhdan:G->Gamma}: Show that if $\Gamma$ has Kazhdan's property, then $G$ has Kazhdan's property.
\hint{Any $\Gamma$-invariant vector~$v$ can be averaged over $G/\Gamma$ to obtain a $G$-invariant vector. If $v$ is $\epsilon$-invariant for a compact set whose projection to $G/\Gamma$ has measure $> 1 - \epsilon$, then the average is nonzero.}

\end{exercises}
 
 
 
 
 \section{Fixed points in Hilbert spaces}

We now describe an important geometric interpretation of Kazhdan's
property. 
%For simplicity, we assume all Hilbert spaces are \emph{real} (that is, the scalars come from~$\real$, rather than~$\complex$).

 \begin{defn} \label{AffIsomDefn}
 Let $\Hilbert$ be a Hilbert space. A bijection $T
\colon \Hilbert \to \Hilbert$ is an \defit{affine isometry}
of~$\Hilbert$ if there exist a unitary operator~$U$ on~$\Hilbert$, and $b \in \Hilbert$,
such that 
	$$ \text{$T(v) = Uv + b$ \  for all $v \in \Hilbert$} .$$
 \end{defn}

\begin{eg} \label{NoFPEg}
Let $w_0$ be a nonzero vector in a Hilbert space~$\Hilbert$. For $t \in \real$, define an affine isometry $\phi^t$ of~$\Hilbert$ by $\phi^t (v) = v + t w_0\mk$; this yields an action of~$\real$ on~$\Hilbert$ by affine isometries. Since $\phi^1(v) = v + w_0 \neq v$, we know that the action has no fixed point. 
\end{eg}

The main theorem of this section shows that  the groups that do not have Kazhdan's property are characterized by the existence of a fixed-point-free action as in \cref{NoFPEg}. However, before stating the result, let us introduce some notation, so that we can also state it in cohomological terms. 

 \begin{defn} \label{HilbertCohoDefn}
 Suppose $(\pi, \Hilbert)$ is a unitary representation of a Lie group~$H$. Define
 \begin{enumerate}
 \item $C(H;\Hilbert) = \{\, \text{continuous functions $f \colon H \to \Hilbert$} \,\}$,
 
 \item $\cocyc1(H; \pi) \,{=}\, 
 \{ f \in C(H;\Hilbert) \mid  \forall g,h \in H, \, f(gh) = f(g) + \pi(g)  f(h) \}$,

 \item $\cobdry1(H; \pi) = 
 \{ f \in C(H;\Hilbert) \mid  \exists v \in \Hilbert, \, \forall h \in H, \, f(h) = v - \pi(h) v  \}$,
 
 \item $\coho1(H; \pi) = \cocyc1(H; \pi) / \cobdry1(H; \pi)$
\csee{B1inZ1}.
 \end{enumerate}
If the representation~$\pi$ on~$\Hilbert$ is clear from the context, we may write $\cocyc1(H; \Hilbert)$, $\cobdry1(H; \Hilbert)$, and $\coho1(H; \Hilbert)$,
instead of $\cocyc1(H; \pi)$, $\cobdry1(H; \pi)$, and $\coho1(H; \pi)$.
 \end{defn}

\begin{thm} \label{T<>FH}
 For a Lie group~$H$, the following are equivalent:
 \noprelistbreak
 \begin{enumerate}
 \item  \label{T<>FH-T}
 $H$ has Kazhdan's property.
 \item \label{T<>FH-FP}
  For every Hilbert space~$\Hilbert$, every continuous action of~$H$ by affine isometries
on~$\Hilbert$  has a fixed point.
\item   \label{T<>FH-H1=0}
$\coho1(H; \pi) = 0$, for every unitary representation~$\pi$ of~$H$.
 \end{enumerate}
 \end{thm}

\begin{proof}[\bf \mathversion{bold} Proof of \pref{T<>FH-FP}$\implies$\pref{T<>FH-H1=0}]
Given $f \in \cocyc1(H;\pi)$, define an action of~$f$ on~$\Hilbert$ via affine isometries by defining 
	$$ \text{$hv = \pi(h) v + f(h)$ for $h \in H$ and $v \in \Hilbert$} $$
\csee{Z1<>act}. By assumption, this action must have a fixed point~$v_0$. For all $h \in H$, we have
	$ v_0 = h v_0 = \pi(h) v_0 + f(h) $, 
	so $ f(h) = v_0 - \pi(h) v_0$.
Therefore $f \in \cobdry1(H;\pi)$. Since $f$ is an arbitrary element of $\cocyc1(H;\pi)$, this implies $\coho1(H; \pi) = 0$.
\end{proof}

\begin{proof}[\bf \mathversion{bold} Proof of \pref{T<>FH-H1=0}$\implies$\pref{T<>FH-T}]
We prove the contrapositive: assume $H$ does not have Kazhdan's property.
This means a unitary representation of~$H$ on some Hilbert space~$\Hilbert$ has almost-invariant vectors, but does not have invariant vectors. We claim $\coho1(H; \pi^\infty) \neq 0$, where $\pi^\infty$ is the obvious diagonal action of~$H$ on the Hilbert space $\Hilbert^\infty = \Hilbert \oplus \Hilbert \oplus \cdots$.

Choose an increasing chain $C_1 \subseteq C_2 \subseteq \cdots$ of compact subsets of~$G$, such that $G = \bigcup_n C_n$. For each~$n$, since $\Hilbert$ has almost-invariant vectors, there exists a unit vector $v_n \in \Hilbert$, such that
	$$ \text{$\displaystyle \| c v_n - v_n \| < \frac{1}{2^n}$ for all $c \in C_n$} . $$
Now, define $f \colon H \to \Hilbert^\infty$ by 
	$$ f(h)_n = n\bigl( hv_n - v_n \bigr) $$
\csee{HinftyFunc}, so $f \in \cocyc1(H; \pi^\infty)$ \csee{HinftyCocyc}.
However, it is easy to see that $f$ is an unbounded function on~$H$ \csee{HinftyUnbdd}, so $f \notin \cobdry1(H; \Hilbert^\infty)$ \csee{CobdryIsBdd}. Therefore $f$ represents a nonzero cohomology class in $\coho1(H; \pi^\infty)$. 
\end{proof}

\begin{proof}[\bf \mathversion{bold} Alternate proof of \pref{T<>FH-H1=0}$\implies$\pref{T<>FH-T}]
Assume the unitary representation~$\pi$ has no invariant vectors. (We wish to show this implies there are no almost-invariant vectors.)
Define a linear map 
	$$ \text{$F \colon \Hilbert \to \cocyc1(H;\pi)$ by $ F_v(h) = \pi(h)v - v$} .$$
Assume, for simplicity, that $H$ is compactly generated \csee{AltFH->T-NotCpctGen}, so some compact, symmetric set~$C$ generates~$H$. By enlarging~$C$, we may assume $C$~has nonempty interior. 
Then the supremum norm on~$C$ turns $\cocyc1(H;\pi)$ into a Banach space \csee{CocycIsBanach}, and the map~$F$ is continuous in this topology \fullcsee{AltFH->T-FEx}{cont}. 

Since there are no invariant vectors in~$\Hilbert$, we know that $F$ is injective \fullcsee{AltFH->T-FEx}{inj}.
Also, the image of~$F$ is obviously $\cobdry1(H;\pi)$. Since $\coho1(H;\pi) = 0$, this means that $F$ is surjective. Therefore, $F$~is a bijection.
So the Open Mapping Theorem (\fullref{OpenMappingThm}{bij}) provides a constant $\epsilon > 0$, such that $\|F_v\| > \epsilon$ for every unit vector~$v$. This means there is some $h \in C$, such that $\| \pi(h)v - v \| > \epsilon$, so $v$ is not $(C,\epsilon)$-invariant. Therefore, there are no almost-invariant vectors.
\end{proof}

\begin{proof}[\bf \mathversion{bold} Sketch of proof of \pref{T<>FH-T}$\implies$\pref{T<>FH-FP}]
We postpone this proof to \cref{PosDefFuncSect}, where functions of positive type are introduced. They yield an embedding of~$\Hilbert$ in the unit sphere of a (larger) Hilbert space~$\widehat\Hilbert$. This embedding is nonlinear and non-isometric, but there is a unitary representation~$\widehat\pi$ on~$\widehat\Hilbert$ for which the embedding is equivariant. Kazhdan's property provides an invariant vector in~$\widehat\Hilbert$, and this pulls back to a fixed point in~$\Hilbert$.
See \cref{PosDefFuncSect} for more details.
\end{proof}

\begin{rem}
If $H$ satisfies \pref{T<>FH-FP} of \cref{T<>FH}, it is said to have ``\term[property FH@property $(FH)$]{property $(FH)$}'' (because it has \underline{\textbf{F}}ixed points on \underline{\textbf{H}}ilbert spaces).
\end{rem}

In \cref{HilbertCohoDefn}, the subspace $\cobdry1(H; \pi)$ may fail to be closed \csee{AlmInvt->B1NotClosed}. In this case, the quotient space $\coho1(H; \pi)$ does not have a good topology. Fortunately, it can be shown that \cref{T<>FH} remains valid even if we replace $\cobdry1(H;\pi)$ with its closure:

\begin{defn}
In the notation of \cref{HilbertCohoDefn}, let:
	\begin{enumerate}
	\item $\overline{\cobdry1(H; \pi)}$ be the closure of $\cobdry1(H; \pi)$ in $\cocyc1(H;\pi)$,
	and
	\item $\redcoho1(H; \pi) = \cocyc1(H; \pi) / \overline{\cobdry1(H; \pi)}$. This is called the \emph{reduced} $1$st~cohomology.
	\end{enumerate}
\end{defn}

The following result requires the technical condition that $H$ is compactly generated \csee{CpctGenDefn,RedCohoCpctGenEx}.

\begin{thm} \label{T<>H1bar=0}
A compactly generated Lie group~$H$ has Kazhdan's property if and only if\/
 $\redcoho1(H; \pi) = 0$, for every unitary
representation~$\pi$ of~$H$.
\end{thm}

Because reduced cohomology behaves well with respect to the direct integral decomposition of a unitary representation
(although the unreduced cohomology does not), this theorem implies that it suffices to consider only the irreducible representations of~$H$:

\begin{cor} \label{T<>H1irred=0}
A compactly generated Lie group~$H$ has Kazhdan's property if and only if\/
 $\redcoho1(H; \pi) = 0$, for every \textbf{irreducible} unitary representation~$\pi$ of~$H$.
\end{cor}

\begin{rem} \label{T->FPonOther}
We have seen that a group with Kazhdan's property has bounded orbits whenever it acts isometrically on a Hilbert space. The same conclusion has been proved for isometric actions on some other spaces, including real hyperbolic $n$-space~$\hyperbolic^n$, complex hyperbolic $n$-space~$\hyperbolic_{\complex}^n$, and all ``median spaces'' (including all $\real$-trees). (In many cases, the existence of a bounded orbit implies the existence of a fixed point.)
See \cref{WatataniThm} for an example.
\end{rem}

\begin{exercises}

\item \label{AffineIsomAxioms}
Let $T \colon \Hilbert \to \Hilbert$. Show that if $T$ is an affine isometry, then
 \begin{enumerate}
 \item $T(v - w) = T(v) - T(w) + T(0)$,
 and
 \item $\|T(v)-T(w)\| = \|v-w\|$,
 \end{enumerate}
 for all $v,w \in \Hilbert$. 

\item \label{AffIsom<>}
Prove the converse of \cref{AffineIsomAxioms}.

\item \label{B1inZ1}
 In the notation of \cref{HilbertCohoDefn}, show that
$\cobdry1(H;\pi) \subseteq \cocyc1(H;\pi)$ (so the quotient $\cocyc1(H; \pi) / \cobdry1(H; \pi)$ is defined).

\item \label{CobdryIsBdd}
Suppose $f \in \cobdry1(H;\pi)$ so $f \colon H \to \Hilbert$. Show $f$ is bounded.

\item \label{Z1<>act}
Suppose 
	\begin{itemize}
	\item $(\pi,\Hilbert)$ is a unitary representation of~$H$,
	and
	\item $\tau \colon H \to \Hilbert$.
	\end{itemize}
For $h \in H$ and $v \in \Hilbert$, let $\alpha(h)v = \pi(h)\, v + \tau(h)$, so $\alpha(h)$ is an affine isometry of~$\Hilbert$. Show that $\alpha$ defines a continuous action of~$H$ on~$\Hilbert$ if and only if $\tau \in \cocyc1(H;\pi)$ and $\tau$ is continuous.

\item \label{AffineMustFromRep}
Suppose $H$ acts continuously by affine isometries on the Hilbert space~$\Hilbert$. Show there is a unitary representation~$\pi$ of~$H$ on~$\Hilbert$, and some $\tau \in \cocyc1(H;\pi)$, such that $hv =  \pi(h)\, v + \tau(h)$ for every $h \in H$ and $v \in \Hilbert$. 

 \item \label{FP<>SomeOrbitBdd<>EveryOrbitBddEx}
  Suppose $H$ acts continuously by affine isometries
on the Hilbert space~$\Hilbert$. Show the following are equivalent:
	\begin{enumerate}
	\item $H$ has a fixed point in~$\Hilbert$.
	\item The orbit $Hv$ of each vector~$v$ in~$\Hilbert$ is a bounded subset of~$\Hilbert$.
	\item The orbit $Hv$ of some vector~$v$ in~$\Hilbert$ is a bounded subset of~$\Hilbert$.
	\end{enumerate}
\hint{You may use (without proof) the fact that every nonempty, bounded subset~$X$ of a Hilbert space has a unique circumcenter. By definition, the \defit{circumcenter} is a point~$c$, such that, for some $r > 0$, the set $X$ is contained in the closed ball of radius~$r$ centered at~$c$, but $X$~is not contained in any ball of radius $< r$ (centered at any point).}

\item \label{FP->H1=0Ex}
Prove directly that \fullref{T<>FH}{H1=0}$\implies$\fullref{T<>FH}{FP}, without using Kazhdan's property.
\hint{For each $h \in H$, there is a unique unitary operator $\pi(h)$, such that we have $hv = \pi(h)v + h(0)$ for all $v \in \Hilbert$.
Fix $v \in \Hilbert$ and define $f \in \cocyc1(H;\pi)$ by $f(h) = hv - v$. If $f \in \cobdry1(H;\pi)$, then $H$ has a fixed point.}

\item \label{HinftyFunc}
In the notation of the proof of \ref{T<>FH}($\ref{T<>FH-H1=0}\Rightarrow\ref{T<>FH-T}$), show $f \colon H \to \Hilbert^\infty$.
\hint{For each $h \in H$, show the sequence $\bigl\{ \|f(h)_n\| \bigr\}$ is square-summable.}

\item \label{HinftyCocyc}
In the notation of the proof of  \ref{T<>FH}($\ref{T<>FH-H1=0}\Rightarrow\ref{T<>FH-T}$), show $f \in \coho1(H; \pi^\infty)$.

\item \label{HinftyUnbdd}
In the notation of the proof of \ref{T<>FH}($\ref{T<>FH-H1=0}\Rightarrow\ref{T<>FH-T}$), show $f$ is unbounded.
\hint{You may use (without proof) the fact that every nonempty, bounded subset of a Hilbert space has a unique circumcenter, as in \cref{FP<>SomeOrbitBdd<>EveryOrbitBddEx}.}

\item \label{CocycIsBanach}
Assume $C$ is a compact, symmetric set that generates~$H$, and has nonempty interior. For each $f \in \cocyc1(H;\Hilbert)$, let $\xi(f)$ be the restriction of~$f$ to~$C$. Show that $\xi$ is a bijection from $\cocyc1(H;\Hilbert)$ onto a closed subspace of the Banach space of continuous functions from~$C$ to~$\Hilbert$.

\item \label{AltFH->T-FEx}
In the notation of the alternate proof of \ref{T<>FH}($\ref{T<>FH-H1=0}\Rightarrow\ref{T<>FH-T}$), show:
	\begin{enumerate}
	\item  \label{AltFH->T-FEx-cont}
	$F$ is continuous.
	\item  \label{AltFH->T-FEx-inj}
	$F$ is injective.
	\end{enumerate}
\hint{\pref{AltFH->T-FEx-inj}~If $F_v = F_w$, then what is $\pi(h)(v - w)$?}

\item \label{AltFH->T-NotCpctGen}
Remove the assumption that $H$ is compactly generated from the alternate proof of \pref{T<>FH-H1=0}$\implies$\pref{T<>FH-T}.
\hint{The topology of uniform convergence on compact sets makes $\cocyc1(H;\pi)$ into a Fréchet space.}

\item Assume
 \begin{itemize}
 \item $\Gamma$ has Kazhdan's property~$T$,
 \item $V$ is a vector space,
 \item $\Hilbert$ is a Hilbert space that is contained in~$V$,
 \item $v \in V$,
 and
 \item $\sigma \colon \Gamma \to \GL(V)$ is any homomorphism, such
that 
 \begin{itemize}
 \item the restriction $\sigma(\gamma)|_{\Hilbert}$ is unitary, for
every $\gamma \in \Gamma$, 
 and
 \item $\Hilbert + v$ is $\sigma(\Gamma)$-invariant.
 \end{itemize}
 \end{itemize}
 Show $\sigma(\Gamma)$ has a fixed point in $\Hilbert + v$.
 \hint{\fullCref{T<>FH}{FP}.}
 
 \item \label{AlmInvt->B1NotClosed}
 Show that if $\pi$ has almost-invariant vectors, then $\cobdry1(H;\pi)$ is not closed in $\cocyc1(H;\pi)$.
 \hint{See the alternate proof of \cref{T<>FH}(\ref{T<>FH-H1=0}$\implies$\ref{T<>FH-T}).}
 
 \item \label{RedCohoCpctGenEx}
 Show the assumption that $H$ is compactly generated cannot be removed from the statement of \cref{T<>H1bar=0}.
 \hint{Let $H$ be an infinite, discrete group, such that every finitely generated subgroup of~$H$ is finite.}
 
 \item \label{TiffH1(irred)}
 Show $H$ has Kazhdan's property if and only if 
 $\coho1(H; \pi) = 0$, for every \textbf{irreducible} unitary
representation~$\pi$ of~$H$.
\hint{You may assume \cref{T<>H1bar=0}.}

\begin{defn*}
A \defit{tree} is a contractible, $1$-dimensional simplicial complex.
\end{defn*}

\item  \label{WatataniThm}
(Watatani)
Suppose 
	\begin{itemize}
	\item $\Lambda$ is a discrete group that has Kazhdan's property, 
	and 
	\item acts by isometries on a tree~$T$. 
	\end{itemize}
Show $\Lambda$ has a fixed point in~$T$ (without assuming \cref{T->FPonOther}). 
\hint{Fix an orientation of~$T$, and fix a vertex $v$ in~$T$. For each $\lambda \in \Lambda$, the geodesic path in~$T$ from $v$ to $\lambda(v)$ can be represented by a $\{0, \pm1\}$-valued function~$P_\lambda$ on the set~$E$ of edges of~$T$. Verify that $\lambda \mapsto P_\lambda$ is in $\cocyc1\bigl(\Lambda; \LL2(E) \bigr)$, and conclude that the orbit of~$v$ is bounded.}

\item \label{SO1nNotKazhdan}
Show $\SO(1,n)$ and $\SU(1,n)$ do not have Kazhdan's property.
\hint{You may assume the facts stated in \cref{T->FPonOther}.}

\item It is straightforward to verify that all of the results in this \lcnamecref{KazhdanTChap} remain valid if we require $\Hilbert$ to be a \textbf{real} Hilbert space (instead of a Hilbert space over~$\complex$), as in \cref{RealHilbertAssump} below. In this setting, there is no need to restrict attention to \emph{affine} isometries in the statement of \fullcref{T<>FH}{FP}, because \emph{all} isometries are affine:

Let $\Hilbert$ be a real Hilbert space, and let $\varphi \colon \Hilbert \to \Hilbert$ be any distance-preserving bijection (so $\| \varphi(v) - \varphi(w) \| = \| v - w \|$ for all $v,w \in \Hilbert$). Show that $\varphi$ is an affine isometry.
\hint{The main problem is to show that if $\varphi(0) = 0$, then $\varphi$ is $\real$-linear. This is well known (and easy to prove) when $\Hilbert = \real^2$. The general case follows from this.}

\end{exercises}



\section{Functions on\texorpdfstring{~$H$}{ H} that are of positive type} \label{PosDefFuncSect}

This section completes the proof of \cref{T<>FH}, by showing that affine isometric actions of Kazhdan groups on Hilbert spaces always have fixed points. For this purpose, we develop some of the basic theory of functions of positive type. 

\begin{assump} \label{RealHilbertAssump}
To simplify some details, Hilbert spaces in this section are assumed to be real, rather than complex. (That is, the field of scalars is~$\real$, rather than~$\complex$.)
\end{assump}

\begin{defn} \label{PosTypeDefn} \  
\noprelistbreak
	\begin{enumerate}
	\item Let $A$ be an $n \times n$ real symmetric matrix.
		\noprelistbreak
		\begin{enumerate}
		\item \label{PosTypeDefn-Pos}
		$A$ is of \defit[positive!type]{positive type} if $\langle Av \mid v \rangle \ge 0$ for all $v \in \real^n$. Equivalently, this means all of the eigenvalues of~$A$ are $\ge 0$ \csee{PosTypeIffInnerProdEx}.
		\item  \label{PosTypeDefn-Cond}
		$A$ is \defit[positive!type!conditionally]{conditionally of positive type} if 
			\begin{enumerate}
			\item $\langle Av \mid v \rangle \ge 0$ for all $v = (v_1,\ldots,v_n) \in \real^n$, such that we have $v_1 + \cdots + v_n = 0$,
			and
			\item  \label{PosTypeDefn-Cond-0}
			all the diagonal entries of~$A$ are~$0$.
			\end{enumerate}
		(The word ``conditionally'' refers to the fact that the inequality on $\langle Av \mid v \rangle$ is only required to be satisfied when a particular condition is satisfied, namely, when the sum of the coordinates of~$v$ is~$0$.)
		\end{enumerate}
	\item A continuous, real-valued function~$\varphi$ on a topological group~$H$ is said to be of \defit[positive!type]{positive type} (or \defit[positive!type!conditionally]{conditionally of positive type}, respectively) if, for all~$n$ and all $h_1,\ldots,h_n \in H$, the matrix $\bigl( \varphi(h_i^{-1} h_j ) \bigr)$ is a symmetric matrix of the said type.
	\end{enumerate}
\end{defn}

\begin{warn}
A function that is of positive type is almost never conditionally of positive type. This is because a  matrix satisfying \pref{PosTypeDefn-Pos} of \cref{PosTypeDefn}  will almost never satisfy \pref{PosTypeDefn-Cond-0} \csee{PosAndCond->0}.
\end{warn}

\begin{terminology} \label{PosDefTerm}
Functions of positive type are often called \defit[positive!definite]{positive definite} or \defit[positive!semi-definite]{positive semi-definite}.
\end{terminology}

Such functions arise naturally from actions of~$H$ on Hilbert spaces:

\begin{lem} \label{PosTypeFromAction}
Suppose
\noprelistbreak
	\begin{itemize}
	\item $H$ is a topological group,
	\item $H$ acts continuously by affine isometries on a Hilbert space~$\Hilbert$,
	\item $v \in \Hilbert$,
	\item $\varphi \colon H \to \real$ is defined by $\varphi(h) = -\| hv - v \|^2$ for $h \in H$,
	and
	\item $\psi \colon H \to \real$ is defined by $\psi(h) = \langle hv \mid v \rangle$ for $h \in H$.
	\end{itemize}
Then:
	\begin{enumerate}
	\item  \label{PosTypeFromAction-Cond}
	$\varphi$ is conditionally of positive type,
	and
	\item  \label{PosTypeFromAction-Pos}
	$\psi$ is of positive type if $h(0) = 0$ for all $h \in H$.
	\end{enumerate}
\end{lem} 

\begin{proof}
\Cref{CondPosTypeFromActionEx,PosTypeFromActionEx}.
\end{proof}

Conversely, the following result shows that all functions of positive type arise from this construction. 
(The ``GNS'' in its name stands for Gelfand, Naimark, and Segal.) 

\begin{prop}[(``GNS construction'')] \label{GNS}
If $f \colon H \to \real$ is of positive type, then there exist
\noprelistbreak
	\begin{itemize}
	\item a continuous action of~$H$ by linear isometries on a Hilbert space~$\Hilbert$ {\rm(}so $h(0) = 0$ for all~$h${\rm)},
	and
	\item $v \in \Hilbert$,
	\end{itemize}
such that $f(h) = \langle hv \mid v \rangle$ for all $h \in H$.
\end{prop}

\begin{proof}
Let $\real[H]$ be the vector space of functions on~$H$ with finite support. Since the set of delta functions $\{\,\delta_h \mid h \in H \,\}$ is a basis, there is a unique bilinear form on $\real[H]$, such that 
	$$ \text{$\langle \delta_{h_1} \mid  \delta_{h_2} \rangle = f(h_1^{-1} h_2)$ \ for all $h_1,h_2$} .$$
Since $f$ is of positive type, this form is symmetric and satisfies the inequality $\langle w \mid w \rangle \ge 0$ for all~$w$. Let $Z$ be the \defit[radical!of a bilinear form]{radical} of the form, which means
	$$ Z = \{\, z \in \real[H] \mid \langle z \mid z \rangle = 0 \,\} ,$$
so $\langle \, \mid \, \rangle$ factors through to a well-defined positive-definite, symmetric bilinear form on the quotient $\real[H]/Z$. This makes the quotient into a pre-Hilbert space; let $\Hilbert$ be its completion, which is a Hilbert space, and let $v$ be the image of~$\delta_e$ in~$\Hilbert$. 

The group~$H$ acts by translation on $\real[H]$, and it is easy to verify that the action is continuous, and preserves the bilinear form \csee{GNSPf-ActCont&IsomEx}. Therefore, the action extends to a unitary representation of~$H$ on~$\Hilbert$. Furthermore, for any $h \in H$, we have
	$$ 
	f(h)
	= f(e \cdot h)
	= \langle  \delta_e \mid \delta_h \rangle
	= \langle  \delta_e \mid h\delta_e \rangle
	=  \langle v \mid hv \rangle
	= \langle hv \mid v \rangle
	, $$
as desired.
\end{proof}

We will also use the following important relationship between the two concepts:

\begin{lem}[(Schoenberg's Lemma)] \label{Schoenberg}
If $\varphi$ is conditionally of positive type, then $e^\varphi$ is of positive type.
\end{lem}

\begin{proof}
A function $\kappa \colon H \times H \to \real$ is said to be a \defit{kernel of positive type} if the matrix $\bigl( \kappa(h_i,h_j) \bigr)$ is a symmetric matrix of positive type, for all~$n$ and all $h_1,\ldots,h_n \in H$.

Define $\kappa \colon H \times H \to \real$ by
	$$ \kappa(g,h) = \varphi(g^{-1} h) - \varphi(g) - \varphi(h) .$$
Then:
	\begin{itemize}
	\item $\kappa$ is a kernel of positive type \csee{SchoenbergEx-K},
	\item so $e^\kappa$ is a kernel of positive type \csee{SchoenbergEx-exp},
	\item and $e^{\varphi(g)} e^{\varphi(h)}$ is a kernel of positive type \csee{SchoenbergEx-func},
	\item so the product $e^{\kappa(g,h)} \bigl( e^{\varphi(g)} e^{\varphi(h)} \bigr)$ is a kernel of positive type \csee{SchoenbergEx-prod}.
	\end{itemize}
This product is $e^{\varphi(g^{-1} h)}$, so $e^\varphi$ is a function of positive type.
\end{proof}

With these tools, it is not difficult to show that affine isometric actions of Kazhdan groups on Hilbert spaces always have fixed points:

\begin{proof}[\bf \mathversion{bold} Proof of \cref{T<>FH} ($\ref{T<>FH-T} \Rightarrow\ref{T<>FH-FP}$)]
Let $\alpha$ be the given action of~$H$ on~$\Hilbert$ by affine isometries, and let $\pi$ be the corresponding unitary representation \csee{AffineMustFromRep}. Therefore, we have
	$$ \text{$\alpha(h) v = \pi(h)v + \tau(h)$ for $h \in H$ and $v \in \Hilbert$,} $$
where $\tau \in \cocyc1(H;\pi)$.

Let $\widehat H = \Hilbert \rtimes H$ be the semidirect product of (the additive group of)~$\Hilbert$ with~$H$, where $H$ acts on~$\Hilbert$ via~$\pi$. This means the elements of~$\widehat H$ are the ordered pairs $(v,h)$, and, for $v_1,v_2 \in \Hilbert$ and $h_1,h_2 \in H$, we have
	$$ (v_1,h_1) \cdot  (v_2,h_2) = (v_1 + \pi(h_1) v_2, h_1 h_2 ) .$$
This semidirect product is a topological group, so we can apply the above theory of functions of positive type to it.
Define a continuous action~$\widehat\alpha$ of $\widehat H$ on~$\Hilbert$ by
	\begin{align} \label{T<>FH-alphahatdefn}
	\widehat\alpha(v,h)w = \alpha(h)w + v
	\end{align}
\csee{alphahatIsAction}, and define
	$$ \text{$\widehat\varphi \colon \widehat H \to \real$ by $\widehat\varphi(v,h) = -\| \widehat\alpha(v,h)(0) \|^2$.} $$
Since $\widehat\alpha(v,h)$ is an affine isometry for every $v$ and~$h$, we know $\widehat\varphi$ is conditionally of positive type \fullcsee{PosTypeFromAction}{Cond}. Therefore $e^{\widehat\varphi}$ is of positive type \csee{Schoenberg}. Hence, the GNS construction \pref{GNS} provides a unitary representation~$\widehat\pi$ of~$\widehat H$ on a Hilbert space~$\widehat\Hilbert$ and some $\hat v \in \widehat \Hilbert$, such that 
	\begin{align} \label{T->FH-GNSEquality}
	\text{$\langle \widehat\pi(v,h) \hat v \mid \hat v \rangle = e^{\widehat\varphi(v,h)}$ for all $v \in \Hilbert$ and $h \in H$.} 
	\end{align}
We now define
	$$ \text{$\Phi \colon \Hilbert \to \widehat\Hilbert$ by
	$ \Phi(v) = \widehat\pi(v,e) \hat v $} .$$
We have
	\begin{align} \label{T<>FH-PhiIsEqui}
	\text{$\Phi \bigl( \alpha(h)v \bigr) = \widehat\pi( 0,h ) \, \Phi(v)$ for $h \in H$ and $v \in \Hilbert$} 
	\end{align}
\csee{T<>FH-PhiIsEquiEx}, so $\Phi$ converts the affine action of~$H$ on~$\Hilbert$ to a linear action on~$\widehat\Hilbert$.
Since the linear span of $\Phi(\Hilbert)$ contains~$\hat v$ and is invariant under $\widehat\pi \bigl( \widehat H \bigr)$  \csee{SpanIsHInvt}, there is no harm in assuming that its closure is all of~$\widehat\Hilbert$.


It is clear from the definition of $\widehat\varphi$ that $\widehat\varphi(0,e) = 0$, so we know that $\hat v$ is a unit vector. Therefore
	\begin{align}
	\| \widehat\pi(v,h) \hat v - \hat v\|^2 
	&= \bigl\langle \, \widehat\pi(v,h) \hat v - \hat v, \ \widehat\pi(v,h) \hat v - \hat v \,\bigr\rangle 
	\notag
	\\&= 2 \bigl( 1 - \langle \widehat\pi(v,h) \hat v \mid \hat v \rangle \bigr)
	\label{T<>FHPf-AlmInvt}
	\\& = 2 \bigl( 1 - e^{\widehat\varphi(v,h)} \bigr) 
	\notag
	 .\end{align}

Since $H$ has Kazhdan's property, there is a compact subset~$C$ of~$H$ and some $\epsilon > 0$, such that every unitary representation of~$H$ that has a $(C,\epsilon)$-invariant vector must have an invariant vector. There is no harm in multiplying the norm on~$\Hilbert$ by a small positive scalar, so we may assume $\widehat\varphi(0,h)$ is as close to~$0$ as we like, for all $h \in C$. Then \pref{T<>FHPf-AlmInvt} tells us that $\hat v$ is $(C,\epsilon)$-invariant, so $\widehat\Hilbert$ must have a nonzero $H$-invariant vector~$\hat v_0$.

Suppose the affine action~$\alpha$ does not have any fixed points. (This will lead to a contradiction.) This implies that every $H$-orbit on~$\Hilbert$ is unbounded \csee{FP<>SomeOrbitBdd<>EveryOrbitBddEx}. Hence, for any fixed $v \in \Hilbert$, and all $h \in H$, we have
	\begin{align*}
	\langle \Phi(v) \mid \hat v_0 \rangle
	&= \langle \Phi(v) \mid \widehat\pi(0,h^{-1}) \hat v_0 \rangle
	&& \text{($\hat v_0$ is $H$-invariant)}
	\\&= \langle  \widehat\pi(0,h) \, \Phi(v)  \mid \hat v_0 \rangle
	&& \text{($\pi$ is unitary)}
%	\\&= \langle  \widehat\pi \bigl( (0,h) \cdot(v,e) \bigr) \hat v \mid \hat v_0 \rangle
%	\\&= \langle  \widehat\pi \bigl( \pi(h) v,h \bigr) \hat v \mid \hat v_0 \rangle
	\\&= \langle  \Phi( \alpha(h)v \bigr) \mid \hat v_0 \rangle
	&& \text{(\ref{T<>FH-PhiIsEqui})}
	\\&\to 0 \text{\quad as $\|\alpha(h) v\| \to \infty$}
	&& \text{(\cref{T->FH-WeaklyToZeroEx})}
	.\end{align*}
So $\hat v_0$ is orthogonal to the linear span of $\Phi(\Hilbert)$, which is dense in~$\widehat\Hilbert$. Therefore $\hat v_0 = 0$. This is a contradiction.
\end{proof}




\begin{exercises}

\item \label{PosTypeIffInnerProdEx}
Let $A$ be a real symmetric matrix.
Show $A$ is of positive type if and only if all of the eigenvalues of~$A$ are $\ge 0$.
\hint{$A$ is diagonalizable by an orthogonal matrix.}

\item \label{PosAndCond->0}
Suppose $A$ is an $n \times n$ real symmetric matrix that is of positive type and is also conditionally of positive type. Show $A = 0$.
\hint{What does \fullref{PosTypeDefn}{Cond-0} say about the trace of~$A$?}

\item \label{alphahatIsAction}
In the notation of the proof of \cref{T<>FH}, show
	$$ \widehat\alpha(v_1,h_1) \cdot \widehat\alpha(v_2,h_2) 
	= \widehat\alpha \bigl( (v_1,h_1) \cdot (v_2,h_2) \bigr) $$
for all $v_1,v_2 \in \Hilbert$ and $h_1,h_2 \in H$.

%\item 
%In the notation of the proof of \cref{T<>FH}, show
%	$$ \widehat\pi(e,h)\hat v  = \widehat\pi \bigl( \tau(h), e \bigr) \hat v .$$

\item \label{SpanIsHInvt}
In the notation of the proof of \cref{T<>FH}, show that the linear span of $\Phi(\Hilbert)$ is invariant under $\widehat\pi \bigl( \widehat H \bigr)$.
\hint{See~\pref{T<>FH-PhiIsEqui}.}

\item \label{T->FH-WeaklyToZeroEx}
In the notation of the proof of \cref{T<>FH}, show that if $v \in \Hilbert$, and $\{w_n\}$ is a sequence in~$\Hilbert$, such that $\| w_n \| \to \infty$, then
	$\Phi( w_n ) \to 0$ weakly.
\hint{If $\widehat w \in \Phi(\Hilbert)$, then \pref{T->FH-GNSEquality} implies $\bigl\langle \Phi( w_n ) \mid \widehat w \bigr\rangle \to 0$.% We may assume the linear span of $\Phi(\Hilbert)$ is dense in~$\widetilde\Hilbert$.
}

\item \label{CondPosTypeFromActionEx}
Prove \fullcref{PosTypeFromAction}{Cond}.
\hint{If $\sum_i t_i = 0$, then
$\sum_{i,j} t_i \, t_j \, \varphi( h_i^{-1} h_j v ) = 2 \, \bigl\| \sum_i t_i h_i v \bigr\|^2$.}

\item \label{PosTypeFromActionEx}
Prove \fullcref{PosTypeFromAction}{Pos}.
\hint{$\sum_{i,j} t_i \, t_j \, \psi( h_i^{-1} h_j v ) = \bigl\| \sum_i t_i h_i v \bigr\|^2$.}

\item \label{GNSPf-ActCont&IsomEx}
Prove that the action of~$H$ acts on $\real[H]$ by translation is continuous, and preserves the bilinear form defined in the proof of \cref{GNS}. 

\item \label{SchoenbergEx-K} 
Show that the kernel~$\kappa$ in the proof of \cref{Schoenberg} is of positive type.
\hint{For $v_1,\ldots,v_n \in \real$ and $h_1,\ldots,h_n \in H$, let $v_0 = -\sum v_i$ and $h_0 = e$. Then $\sum_{i,j} v_i v_j \kappa(h_i,h_j) \ge 0$ since $\varphi$ is conditionally of positive type and $\sum v_i = 0$. However, the terms with either $i = 0$ or $j = 0$ have no net contribution, since $\varphi(e) = 0$.}

\item \label{SchoenbergEx-prod} 
Show that if $\kappa$ and~$\lambda$ are kernels of positive type, then $\kappa \lambda$ is a kernel of positive type.
\hint{Given $h_1,\ldots,h_n \in H$, show there is a matrix~$L$, such that $L^2 = \bigl( \lambda(h_i,h_j) \bigr)$. Note that
	$\sum\nolimits_{i,j} v_i \, v_j \, \kappa(h_i,h_j) \, \lambda(h_i,h_j) = \sum\nolimits_k \sum\nolimits_{i,j} ( L_{i,k} v_i)  \,( L_{k,j} v_j) \, \kappa(h_i, h_j) \ge 0$.}

\item \label{SchoenbergEx-exp} 
If $\kappa$ is a kernel of positive type, show $e^\kappa$ is a kernel of positive type.
\hint{Since $\kappa^n$ is a kernel of positive type for all~$n$, the same is true of $\sum \kappa^n/n!$.}

\item \label{SchoenbergEx-func}
For every $\varphi \colon H \to \real$, show $\varphi(g)\, \varphi(h)$ is a kernel of positive type.

\item Suppose $\varphi \colon H \to \real$.
Prove the following converse of \cref{Schoenberg}:
If $e^{t\varphi}$ is of positive type for all $t > 0$, then $\varphi$ is conditionally of positive type.
\hint{Verify that $e^{t\varphi} - 1$ is conditionally of positive type.
Then $\lim_{t \to 0^+} (e^{t\varphi} - 1)/t$ has the same type.}

\item \label{T<>FH-PhiIsEquiEx}
Verify \pref{T<>FH-PhiIsEqui}.
\hint{Since $\widehat\alpha\bigl( -\tau(h), h \bigr)(0) = 0$, we have $\bigl\langle \widehat\pi(e,h) \hat v \mid \widehat\pi \bigl( \tau(h), e \bigr) \hat v \bigr\rangle = 1$. Since they are of norm~$1$, the two vectors must be equal.}

\item \label{Haagerup<>proper}
Recall that a Lie group~$H$ has the \defit{Haagerup property} if it has a unitary representation, such that there are almost-invariant vectors, and all matrix coefficients decay to~$0$ at~$\infty$.
It is known that $H$ has the Haagerup property if and only if it has a continuous, \emph{proper} action by affine isometries on some Hilbert space. Prove the implication ($\Leftarrow$) of this equivalence.

\end{exercises}




\begin{notes}

The monograph \cite{BekkaHarpeValette-T} is the standard reference on Kazhdan's property~$(T)$. 
%It includes much material that is not covered here.
%Brief treatments can also be found in \cite[Chap.~3]{MargulisBook} and \cite[Chap.~7]{ZimmerBook}.
The property was defined by D.\,Kazhdan in \cite{KazhdanT}, where \cref{KazhdanEasy,SL3RHasT,Kazhdan:G->Gamma} were proved.

See \cite{BaderFurmanGelanderMonod-BanachT} for a discussion of the generalization of Kazhdan's property to actions on Banach spaces, including \cref{TnotBanach,BanachNoInvt}.
See \cite{CherixEtAl-Haagerup} for a discussion of the \term{Haagerup property} that is mentioned in \cref{T->NotHaagerup,Haagerup<>proper}.
See \cite{Lubotzky-ExpandingGrps} for much more information about expander graphs and their connection with Kazhdan's property, mentioned in \cref{Kazhdan->Expander}.

Regarding \cref{TIsQuotOfFP}:
\noprelistbreak
	\begin{itemize} %\itemsep=\smallskipamount
	\item By showing that $\SL\bigl( 3, \mathbf{F}_q[t] \bigr)$ is not finitely presentable, H.\,Behr \cite{Behr-SL3NotFP} provided the first example of a group with Kazhdan's property that is not finitely presentable.
	\item The existence of uncountably many Kazhdan groups was proved by M.\,Gromov \cite[Cor.~5.5.E, p.~150]{Gromov-HypGrp}. More precisely, any cocompact lattice in $\Sp(1,n)$ has uncountably many different quotients (because it is a ``hyperbolic'' group), and all of these quotient groups have Kazhdan's property.
	\item Y.\,Shalom \cite[p.~5]{Shalom-RigidCommens} proved that every discrete Kazhdan group is a quotient of a finitely presented Kazhdan group. The proof can also be found in 
\cite[Thm.~3.4.5, p.~187]{BekkaHarpeValette-T}.
	\end{itemize}

Our proof of \cref{SL3RHasT} is taken from \cite[\S1.4]{BekkaHarpeValette-T}. 

\Cref{WhichGKazhdan} appears in \cite[Thm.~3.5.4, p.~177]{BekkaHarpeValette-T}. It combines work of several people, including Kazhdan \cite{KazhdanT} %Wang \cite{Wang-DualSpace}, 
and Kostant \cite{Kostant-Tannounce,Kostant-Tpaper}. See \cite[pp.~5--7]{BekkaHarpeValette-T} for an overview of the various contributions to this theorem.

A detailed solution of \cref{ConstructPhiEx} can be found in \cite[Lem.~1.4.1]{BekkaHarpeValette-T}.

See \cite[Thm.~1.7.1, p.~60]{BekkaHarpeValette-T} for a proof of \cref{Kazhdan:G->Gamma} and its converse (\cref{GammaT->GT}).

Regarding \fullcref{KazhdanLattRem}{abel}, see \cite{Lubotzky-BettiNum} (and its references) for a discussion of W.\,Thurston's conjecture that lattices in $\SO(1,n)$ have finite-index subgroups with infinite abelian quotients. (For $n = 3$, the conjecture was proved by Agol \cite{Agol-VirtHakenConj}.)
Lattices in $\SU(1,n)$ with an infinite abelian quotient were found by D.\,Kazhdan \cite{Kazhdan-WeilRep}.

Explicit Kazhdan constants for $\SL(n,\integer)$ \ccf{ExplicitKazhdanConstants} were first found by M.\,Burger \cite[Appendix]{delaHarpeValette-T} (or see \cite[\S4.2]{BekkaHarpeValette-T}). 
An approach developed by Y.\,Shalom (see \cite{Shalom-AlgT}) applies to more general groups (such as $\SL \bigl( n, \integer[x] \bigr)$) that are not assumed to be lattices.

\fullCref{OtherT}{random} is a theorem of Zuk \cite[Thm.~4]{Zuk-TandConsts}.
(Or see \cite{KotowskiKotowski-RandGrps} for a more detailed proof.)
\fullCref{OtherT}{Alg} is explained in \cite{Shalom-AlgT}.

\Cref{TnotQI} is due to S.\,Gersten (unpublished). A proof (based on the same example, but rather different from our sketch) is in \cite[Thm.~3.6.5, p.~182]{BekkaHarpeValette-T}.

\Cref{T<>FH} is due to Delorme \cite[Thm.~V.1]{Delorme-T<>FH} (for ($\ref{T<>FH-T}\Rightarrow\ref{T<>FH-FP}$)) and Guichardet \cite[Thm.~1]{Guichardet-T<>FH} (for ($\ref{T<>FH-H1=0} \Rightarrow \ref{T<>FH-T}$)). 

\Cref{T<>H1bar=0} was proved for discrete groups by Korevaar and Schoen \cite[Cor.~4.1.3]{KorevaarSchoen-Global}. The general case is due to Shalom  \cite[Thm.~6.1]{Shalom-RigidCommens}. 
\Cref{T<>H1irred=0,TiffH1(irred)} are also due to Shalom \cite[proof of Thm.~0.2]{Shalom-RigidCommens}. 

The part of \cref{T->FPonOther} dealing with real or complex hyperbolic spaces is in \cite[Cor.~2.7.3]{BekkaHarpeValette-T}. See \cite{ChatterjiEtAl-Median} for median spaces.

The existence and uniqueness of the circumcenter (mentioned in the hints to \cref{FP<>SomeOrbitBdd<>EveryOrbitBddEx,HinftyUnbdd}) is proved in \cite[Lem.~2.2.7]{BekkaHarpeValette-T}.

\Cref{WatataniThm} is due to Watatani \cite{Watatani-T->FA}, and can also be found in \cite[\S2.3]{BekkaHarpeValette-T}.
Serre's book \cite{Serre-Trees} is a very nice exposition of the theory of group actions on trees, but, unfortunately, does not include this theorem.

See \cite[\S2.10--\S2.12 and \S C.4]{BekkaHarpeValette-T} for the material of \cref{PosDefFuncSect}.
\end{notes}


\begin{references}{99}

\bibitem{Agol-VirtHakenConj}
I.\,Agol: 
The virtual Haken conjecture,
\emph{Doc. Math.} 18 (2013), 1045--1087. 
\MR{3104553},
\url{http://www.math.uni-bielefeld.de/documenta/vol-18/33.html}

\bibitem{BaderFurmanGelanderMonod-BanachT}
U.\,Bader, A.\,Furman, T.\,Gelander, and N.\,Monod:
Property $(T)$ and rigidity for actions on Banach spaces,
\emph{Acta Math.} 198 (2007), no.~1, 57--105. 
\MR{2316269},
\url{http://dx.doi.org/10.1007/s11511-007-0013-0}

\bibitem{Behr-SL3NotFP}
H.\,Behr:
$\SL_3 \bigl( \mathbf{F}_q[t] \bigr)$ is not finitely presentable,
in: C.\,T.\,C.\,Wall, ed.,
\emph{Homological Group Theory (Proc. Sympos., Durham, 1977)}.
%London Math. Soc. Lecture Note Ser., 36, 
Cambridge Univ. Press, Cambridge, 1979, pp.~213--224.
ISBN 0-521-22729-1,
\MR{0564424}

\bibitem{BekkaHarpeValette-T}
B.\,Bekka, P.\,de\,la\,Harpe, and A.\,Valette:
\emph{Kazhdan's Property (T).}
Cambridge U.\ Press, Cambridge, 2008.
ISBN 978-0-521-88720-5,
\MR{2415834},
\maynewline
\url{http://perso.univ-rennes1.fr/bachir.bekka/KazhdanTotal.pdf}

\bibitem{ChatterjiEtAl-Median}
I.\,Chatterji, C.\,Dru\c tu, and F.\,Haglund:
Kazhdan and Haagerup properties from the median viewpoint. 
\emph{Adv. Math.} 225 (2010), no.~2, 882--921.
\MR{2671183},
\maynewline
\url{http://dx.doi.org/10.1016/j.aim.2010.03.012}

\bibitem{CherixEtAl-Haagerup}
P.--A.\,Cherix, M.\,Cowling, P.\,Jolissaint, P.\,Julg, and A.\,Valette:
\emph{Groups with the Haagerup property.
Gromov's a-T-menability.}
% Progress in Mathematics, 197. 
Birkh\"auser, Basel, 2001. 
ISBN 3-7643-6598-6,
\MR{1852148}

\bibitem{delaHarpeValette-T}
 P.\,de\,la\,Harpe and A.\,Valette:
 \emph{La propri\'et\'e (T) de Kazhdan pour les groupes
localement compacts,}
 \emph{Ast\'erisque \#175,}
 Soc. Math. France, 1989.
 \MR{1023471} % no URL available @@@

\bibitem{Delorme-T<>FH}
P.\,Delorme:
$1$-cohomologie des repr\'esentations unitaires des groupes de Lie semisimples
et r\'esolubles. Produits tensoriels continus de repr\'esentations,
\emph{Bull. Soc. Math. France} 105 (1977), 281--336.
\MR{0578893},
\maynewline
\url{http://www.numdam.org/item?id=BSMF_1977__105__281_0}

%\bibitem{FarautHarzallah}
%J.\,Faraut and K.\,Harzallah:
%Distances hilbertiennes invariantes sur un espace homog\`ene,
%\emph{Ann. Inst. Fourier (Grenoble)}
%24 (1974), no.~3, 171--217.
%\MR{0365042}

\bibitem{Gromov-HypGrp}
M.\,Gromov:
 Hyperbolic groups, in:
S.\,M.\,Gersten, ed., \emph{Essays in Group Theory.}
%Math. Sci. Res. Inst. Publ., 8, 
 Springer, New York, 1987,
pp.~75--263.
 ISBN 0-387-96618-8,
\MR{0919829}

\bibitem{Guichardet-T<>FH}
A.\,Guichardet:
Sur la cohomologie des groupes topologiques~II,
\emph{Bull. Sci. Math.} 96 (1972), 305--332.
\MR{0340464} % no URL available @@@

\bibitem{KazhdanT}
 D.\,A.\,Kazhdan:
Connection of the dual space of a group with the
structure of its closed subgroups, 
 \emph{Func. Anal. Appl.} 1 (1967) 63--65.
 \MR{0209390},
 \maynewline
 \url{http://dx.doi.org/10.1007/BF01075866}

\bibitem{Kazhdan-WeilRep}
 D.\,A.\,Kazhdan:
Some applications of the Weil representation,
\emph{J. Analyse Mat.} 32 (1977), 235--248. 
\MR{0492089},
\maynewline
\url{http://dx.doi.org/10.1007/BF02803582}

\bibitem{KorevaarSchoen-Global}
N.\,J.\,Korevaar and R.\,M.\,Schoen:
Global existence theorems for harmonic maps to non-locally compact spaces. 
\emph{Comm. Anal. Geom.} 5 (1997), no.~2, 333--387.
\MR{1483983} % no URL available @@@

\bibitem{Kostant-Tannounce}
B.\,Kostant:
On the existence and irreducibility of certain series of representations,
\emph{Bull. Amer. Math. Soc.} 75 (1969) 627--642.
\MR{0245725},
\maynewline
\url{http://dx.doi.org/10.1090/S0002-9904-1969-12235-4}

\bibitem{Kostant-Tpaper}
B.\,Kostant:
On the existence and irreducibility of certain series of representations, 
in I.\,M.\,Gelfand, ed.:
\emph{Lie groups and their representations.}
 Halsted Press (John Wiley \& Sons), New York, 1975, pp.~231--329.
\MR{0399361}

\bibitem{KotowskiKotowski-RandGrps}
M.\,Kotowski and M.\,Kotowski:
Random groups and property~$(T)$: Zuk's theorem revisited,
\emph{J. Lond. Math. Soc.} (2) 88 (2013), no.~2, 396--416.
\MR{3106728},
\maynewline
\url{http://dx.doi.org/10.1112/jlms/jdt024}

%\bibitem{Lubotzky}
% A.\,Lubotzky:
% \emph{Discrete Groups, Expanding Graphs and Invariant
%Measures}.
% Birkh\"auser, Boston, 1994.

\bibitem{Lubotzky-BettiNum}
A.\,Lubotzky:
Free quotients and the first Betti number of some hyperbolic manifolds,
\emph{Transform. Groups} 1 (1996), no.~1--2, 71--82. 
\MR{1390750},
\maynewline
\url{http://dx.doi.org/10.1007/BF02587736}

\bibitem{Lubotzky-ExpandingGrps}
A.\,Lubotzky:
\emph{Discrete Groups, Expanding Graphs and Invariant Measures.}
Birkhäuser, Basel, 2010. 
ISBN 978-3-0346-0331-7,
 \MR{2569682}

%\bibitem{MargulisBook}
% G.\,A.\,Margulis:
% \emph{Discrete Subgroups of Semisimple Lie Groups.}
% Springer, {Berlin Heidelberg New York}, 1991.
%ISBN 3-540-12179-X,
%\MR{1090825}

\bibitem{Serre-Trees}
J.--P.\,Serre:
\emph{Trees.}
%Translated from the French original by John Stillwell. 
%Corrected 2nd printing of the 1980 English translation. Springer Monographs in Mathematics. 
Springer, Berlin, 1980 and 2003. 
ISBN 3-540-44237-5,
\MR{1954121}

\bibitem{Shalom-RigidCommens}
Y.\,Shalom:
Rigidity of commensurators and irreducible lattices,
\emph{Inventiones Math.} 141 (2000) 1--54.
\MR{1767270},
\maynewline
\url{http://dx.doi.org/10.1007/s002220000064}

\bibitem{Shalom-AlgT}
Y.\,Shalom:
The algebraization of Kazhdan's property~(T).
\emph{International Congress of Mathematicians, Vol.~II,}
Eur. Math. Soc., Z\"urich, 2006, pp.~1283--1310, 
\MR{2275645},
\maynewline
\url{http://www.mathunion.org/ICM/ICM2006.2/}

%\bibitem{Wang-DualSpace}
% S.\,P.\,Wang:
% The dual space of semi-simple Lie groups, 
% \emph{Amer. J. Math.} 91 (1969) 921--937.
% \MR{0259023}
 
\bibitem{Watatani-T->FA}
Y.\,Watatani:
Property (T) of Kazhdan implies property (FA) of Serre,
\emph{Math. Japonica} 27 (1982) 97--103.
\MR{0649023} % no URL available @@@

%\bibitem{ZimmerBook}
%R.\,J.\,Zimmer:
%\emph{Ergodic Theory and Semisimple Groups}.
%%Monographs in Mathematics, 81. 
%Birkh\"auser, Basel, 1984.
%ISBN 3-7643-3184-4,
%\MR{0776417}

\bibitem{Zuk-TandConsts}
A.\,Zuk:
Property $(T)$ and Kazhdan constants for discrete groups,
\emph{Geom. Funct. Anal.} 13 (2003), no.~3, 643--670.
\MR{1995802},
\maynewline
\url{http://dx.doi.org/10.1007/s00039-003-0425-8}

\end{references}




