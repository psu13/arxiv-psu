%!TEX root = IntroArithGrps.tex

\mychapter{Assumed Background} \label{BackChap}

Since the target audience of this book includes mathematicians from a variety of backgrounds (and because very different theorems sometimes have names that are similar, or even identical), this chapter lists (without proof or discussion) specific notations, definitions, and theorems of graduate-level mathematics that are assumed in the main text. (Undergraduate-level concepts, such as the definitions of groups, metric spaces, and continuous functions, are generally not included.)
All of this material is standard, so proofs can be found in graduate textbooks (and on the internet).



\section{Groups and group actions}

\begin{notation} 
Let $H$ be a group, and let $K$ be a subgroup.
	\begin{enumerate}
	\item \nindex{$e$ = identity element of a group}%
	We usually use $e$ to denote the identity element.
	
	\item \nindex{$Z(H)$ = center of the group~$H$}%
	$Z(H) = \{\, z \in H \mid \text{$hz = zh$ for all $h \in H$} \,\}$ is the \defit[center!of a group]{center} of~$H$.
	
	\item \nindex{$\czer_H(K)$ = centralizer of~$K$ in~$H$}%
	$\czer_H(K) = \{\,h \in H \mid \text{$h k = kh$ for all $k \in K$}\,\}$ is the \defit{centralizer} of~$K$ in~$H$.
	
	\item \nindex{$\nzer_H(K)$ = normalizer of~$K$ in~$H$}%
	$\nzer_H(K) = \{\, h \in H \mid h K h^{-1} = K \,\}$ is the \defit{normalizer} of~$K$ in~$H$.
	\end{enumerate}
\end{notation}


\begin{defn}
An \defit[action of a group]{action} of a Lie group~$H$ on a topological space~$X$ is a
%homomorphism $\phi \colon G \to \Homeo(X)$, where
%$\Homeo(X)$ is the group of all permutations
%of~$X$. Equivalently, an \defit[action of~$G$]{action} is a
continuous function $\alpha \colon H \times X \to X$, such that
 \begin{itemize}
 \item $\alpha(e,x) = x$ for all $x \in X$, and
 \item $\alpha \bigl( g, \alpha(h,x) \bigr) = \alpha(gh,x)$
for $g,h \in H$ and $x \in X$.
\end{itemize}
% \item The action is \emph{proper}\index{proper
%action} if the map~$\alpha$ is proper; that is, if the
%inverse image of every compact set is compact.
% \end{enumerate}
 \end{defn}

%\begin{rem}
%If $\alpha$ is an action of~$G$ on~$X$, then we obtain a homomorphism $\varphi \colon G \to \Homeo(X)$, by $\varphi(g)(x) = \alpha(g,x)$.
%\end{rem}


\begin{defns}
 Let a (discrete) group $\Lambda$ act on a topological space~$M$. 
 \begin{enumerate}
 \item The action is \defit[free action]{free} if no
nonidentity element of~$\Lambda$ has a fixed point.
 \item It is \defit{properly discontinuous} if, for every
compact subset~$C$ of~$M$,
 $$ \text{the set $\{\, \lambda \in \Lambda \mid C \cap (\lambda C) \neq \emptyset
\,\}$ is finite} .$$
 \item For any $p \in M$, we define 
 	\nindex{$\Stab_\Lambda(p)$ = stabilizer of~$p$}
	$\Stab_\Lambda(p)
 = \{\, \lambda \in \Lambda \mid \lambda p = p \,\}$. This is a
subgroup of~$\Lambda$ called the \defit{stabilizer} of~$p$ in~$\Lambda$.

\item $M$ is \defit{connected} if it is \textbf{not} the union of two nonempty, disjoint, proper, open subsets.
\item $M$ is \defit[locally!connected]{locally connected} if every neighborhood of every $p \in M$ contains a connected neighborhood of~$p$.
 \end{enumerate}
 \end{defns}

\begin{prop} \label{PropdiscFree->cover}
 If $\Lambda$ acts freely and properly discontinuously on a
topological space~$M$, then the natural map $\pi \colon M \to \Lambda
\backslash M$ is a {\normalfont\defit{covering map}}.

Under the simplifying assumption that $M$ is locally connected, this means that every $p \in \Lambda \backslash M$ has a connected neighborhood~$U$, such that the restriction of~$\pi$ to each connected component of $\pi^{-1}(U)$ is a homeomorphism onto~$U$.
 \end{prop}

%\begin{prop} \label{Isom->proper}
% If $M$ is any locally compact metric space, then $\Isom(M)$ is a
%locally compact topological group, under the compact-open topology
%\textup(that is, under the topology of uniform convergence
%on compact sets\textup). The action of $\Isom(M)$ on~$M$
%is proper; that is, for every compact subset~$C$ of~$M$, the
%set
% $$ \{\, \phi \in \Isom(M) \mid \phi(C) \cap C \neq \emptyset \,\}$$
% is compact.
%
%If $M$ is a smooth manifold, then the topological group $\Isom(M)$
%can be given the structure of a Lie group, so that the action of
%$\Isom(M)$ on~$M$ is smooth.
% \end{prop}



\section{Galois theory and field extensions}

\begin{thm}[{(\thmindex{Fundamental Theorem of Algebra}{Fundamental Theorem of Algebra})}]
 The field~$\complex$ of complex numbers is algebraically
closed; that is, every nonconstant polynomial $f(x) \in
\complex[x]$ has a root in~$\complex$.
 \end{thm}

%\begin{proof}
% This can be proved algebraically, by combining Galois
%Theory with the elementary fact that every real polynomial of
%odd degree has a real zero \csee{AlgicPfFundThmAlg}, but we
%use a bit of complex analysis.
%
%Suppose $f(x)$ has no root. Then $1/f$ is holomorphic
%on~$\complex$. Furthermore, because $f(z) \to \infty$ as $z
%\to \infty$, it is easy to see that $1/f$ is bounded
%on~$\complex$. Hence, Liouville's Theorem asserts that $1/f$
%is constant. This contradicts the fact that $f$~is not
%constant.
% \end{proof}

%\begin{prop}
% Suppose $F$ is a field, and $\alpha$ is a root of some
%irreducible polynomial $f(x) \in F[x]$. Then the extension
%field $F[\alpha]$ is isomorphic to $F[x]/I$, where $I = f(x)
%\, F[x]$ is the principal ideal of $F[x]$ generated by $f(x)$.
% \end{prop}
%
%\begin{proof}
% The map $\phi \colon F[x] \to F[\alpha]$ defined by
%$\phi \bigl( g(x) \bigr) = g(\alpha)$ is a surjective ring
%homomorphism whose kernel is~$I$.
% \end{proof}
%
%\begin{cor}
%  Suppose $F$ is a field. If $\alpha$ and~$\beta$ are two
%roots of an irreducible polynomial $f(x) \in F[x]$, then
%there is an isomorphism $\sigma \colon F[\alpha] \to
%F[\beta]$ with $\sigma(\alpha) = \beta$.
% \end{cor}

\begin{prop}
 Let $F$ be a subfield of~$\complex$, and let $\sigma \colon
F \to \complex$ be any embedding. Then $\sigma$~extends to an
automorphism~$\widehat\sigma$ of~$\complex$.
 \end{prop}

\begin{notation}
 If $F$ is a subfield of a field~$L$, then 
 	\nindex{$|L:F|$ = degree of field extension}$|L:F|$
denotes
$\dim_F L$, the dimension of~$L$ as a vector space over~$F$. This is called the \defit[degree! of a field extension]{degree} of~$L$ over~$F$.
 \end{notation}

\begin{prop}
  If $F$ and~$L$ are subfields of~$\complex$, such that $F
\subseteq L$, then $|L:F|$ is equal to the number of
embeddings~$\sigma$ of~$L$ in~$\complex$, such that
$\sigma|_F = \Id$.
 \end{prop}

\begin{defn}
 An extension~$L$ of a field~$F$ (of characteristic zero) is \defit[Galois!extension]{Galois} if, for every irreducible polynomial $f(x)
\in F[x]$, such that $f(x)$ has a root in~$L$, there exist
$\alpha_1,\ldots,\alpha_n \in L$, such that 
 $$ f(x) = (x - \alpha_1) \cdots (x - \alpha_n) .$$
 That is, if an irreducible polynomial in $F[x]$ has a root
in~$L$, then all of its roots are in~$L$.
 \end{defn}

\begin{defn}
 Let $L$ be a Galois extension of a field~$F$. Then
 	\nindex{$\Gal(L/F)$ = Galois group}
 $$ \Gal(L/F) = \{\, \sigma \in \Aut(L) \mid \sigma|_F = \Id
\,\} .$$
 This is the \defit[Galois!group]{Galois group} of~$L$ over~$F$.
 \end{defn}

\begin{prop}
  If $L$ is a Galois extension of a field~$F$ of
characteristic~$0$, then $|{\Gal(L/F)}| = |L:F|$.
 \end{prop}

\begin{cor}
 If $L$ is a Galois extension of a field~$F$ of
characteristic\/~$0$, then there is a one-to-one correspondence
between
 \begin{itemize}
 \item the subfields~$K$ of~$L$, such that $F \subseteq K$, and
 \item the subgroups~$H$ of\/ $\Gal(L/F)$.
 \end{itemize}
 Specifically, the subgroup of\/ $\Gal(L/F)$ corresponding to the subfield~$K$ is\/ $\Gal(L/K)$.
 \end{cor}

%\begin{proof}
% Given $K$, with $F \subseteq K \subseteq L$, let $H_K =
%\Gal(L/K)$.
% Conversely, given a subgroup~$H$ of $\Gal(L/F)$, let
% $$ K_H = \{\, x \in L \mid \sigma(x) = x, \ \forall \sigma
%\in H \,\} $$
% be the fixed field of~$H$.
% \end{proof}

%\begin{exercises}
%
%\item \label{AlgicPfFundThmAlg}
% Let $F$ be a field of characteristic zero, such that
% \begin{enumerate} 
% \item if $f \in F[x]$ has odd degree, then $f$ has a root
%in~$F$,
% \item if $a \in F$, then either $a$ or~$-a$ has a square
%root in~$F$, and
% \item $-1$ does not have a square root in~$F$.
% \end{enumerate}
% Show that $F[i]$ is algebraically closed (where $i =
%\sqrt{-1}$).
% \hint{Let $L$ be a finite, Galois extension of $F[i]$. If
%$P$ is a Sylow $2$-subgroup of $\Gal(L/F)$, then the fixed
%field of~$P$ has odd degree over~$F$, so this fixed field must
%be trivial. Therefore, $|L:F|$ is a power of~$2$. Hence, $L$ can be
%obtained by a series of quadratic extensions. Since every
%element of~$F$ has a square root in $F[i]$, the half-angle
%formulas show that every element of $F[i]$ has a square root
%in $F[i]$. Therefore $L \subseteq F[i]$.}
%
%\end{exercises}




\section{Algebraic numbers and transcendental numbers}

\begin{defns} \ 
 \noprelistbreak
 \begin{enumerate}
 \item A complex number~$z$ is \defit[algebraic!number]{algebraic} if there is a nonzero polynomial $f(x) \in
\integer[x]$, such that $f(z) = 0$. 
 \item A complex number is \defit[transcendental number]{transcendental} if it is not algebraic.
 \item A (nonzero) polynomial is \defit[polynomial!monic]{monic} if its leading coefficient is~$1$; that is,
we may write $f(x) = \sum_{k=0}^n a_k x^k$ with $a_n = 1$.
 \item A complex number~$z$ is an \defit[algebraic!integer]{algebraic integer} if
there is a \emph{monic} polynomial $f(x) \in \integer[x]$,
such that $f(z) = 0$. 
 \end{enumerate}
 \end{defns}

%\begin{prop}[{}{(\thmindex{Z@$\integer$ is integrally closed}$\integer$ is
%integrally closed}}] \label{Zintegclosed}
% A rational number $t \in \rational$ is an algebraic integer if and only if
%$t \in \integer$.
% \end{prop}

%\begin{proof}
% ($\Leftarrow$) $t$~is a root of the monic polynomial $x - t$.
%
%($\Rightarrow$) Suppose $f(t) = 0$, where $f(x) = x^n +
%\sum_{k=0}^{n-1} a_k x^k$ with each $a_k \in \integer$.
%Writing $t = p/q$ (in lowest terms) with $p,q \in \integer$,
%we have
% $$ 0 = q^n \cdot 0 = q^n f(t)
% = q^n \left( \frac{p^n}{q^n} + \sum_{k=0}^{n-1} a_k
%\frac{p^k}{q^k} \right)
% = p^n + \sum_{k=0}^{n-1} a_k p^k q^{n-k}
% \equiv p^n \pmod{q} .$$
% Since $p^n$~is relatively prime to~$q$ (recall that $t = p/q$
%is in lowest terms), we conclude that $q = 1$, so $t = p/1
%\in \integer$.
% \end{proof}

\begin{prop}
 If $\alpha$ is an algebraic number, then there is some
nonzero $m \in \integer$, such that $m \alpha$ is an
algebraic integer.
 \end{prop}

%\begin{proof}
% Suppose $g(\alpha) = 0$, where $g(x) = \sum_{k=0}^n b_k
%x^k$, with each $b_k \in \integer$, and $b_n \neq 0$. Let 
% \begin{itemize}
% \item $m = a_n$,
% \item $a_k = m^{n-k-1} b_k$, and
% \item $f(x) = \sum_{k=0}^n a_k x^k$.
% \end{itemize}
% Then $f(x)$ is a monic, integral polynomial, and 
% $$ f(m \alpha) = \sum_{k=0}^n (m^{n-k-1} b_k) (m\alpha)^k
% = m^{n-1}  \sum_{k=0}^n b_k \alpha^k
% = m^{n-1} g(\alpha)
% = m^{n-1} \cdot 0
% = 0 .$$
% \end{proof}

%\begin{lem}
% For $t \in \complex$, the following are equivalent:
% \begin{enumerate}
% \item $t$ is an algebraic integer.
% \item $\integer[t]$ is a finitely-generated
%$\integer$-module.
%  \item $\integer[t]$ is a Noetherian
%$\integer$-module.
% \end{enumerate}
% \end{lem}

\begin{prop}
 The set of algebraic integers is a subring of\/~$\complex$.
 \end{prop}

\begin{prop} \label{Gal(cyclotomic)}
Fix some $n \in \natural^+$.
 Let 
 \begin{itemize}
 \item $\omega$ be a primitive $n^\text{th}$~root of unity,
and
 \item $\integer_n^\times$ be the multiplicative group
of units modulo~$n$.
 \end{itemize}
Then there is an isomorphism 
 $$f \colon \integer_n^\times \to \Gal \bigl(
\rational[\omega]/\rational \bigr) 
\colon k \mapsto f_k,$$
 such that $f_k(\omega) = \omega^k$, for all $k \in
\integer_n^\times$.
 \end{prop}





\section{Polynomial rings}

\begin{defn}
 A commutative ring~$R$ is \defit[Noetherian ring]{Noetherian}
if the following equivalent conditions hold:
 \begin{enumerate}
 \item Every ideal of~$R$ is finitely generated.
% \item Every nonempty collection of ideals of~$R$ has a
%maximal element.
 \item If $I_1 \subseteq I_2 \subseteq \cdots$ is any increasing
chain of ideals of~$R$, then there is some~$m$, such that $I_m = I_{m+1} = I_{m+2} = \cdots$.
 \end{enumerate}
 \end{defn}

%\begin{thm} \label{R[x]Noetherian}
% If $R$ is Noetherian, then the polynomial ring $R[x]$ is
%Noetherian.
% \end{thm}
%
%\begin{proof}
% Suppose $J$ is an idea of $R[x]$. (We wish to show that
%$J$~is finitely generated.) For $d \in \natural$, let
% $$ J_d = \{0\} \cup \{\, \operatorname{lead}(f) \mid f \in
%J, \ \deg f = d \,\} .$$
% Then $J_d$ is an ideal of~$R$, and we have
% $J_1 \subseteq J_2 \subseteq \cdots$, so there is some $d_0$,
%such that $J_d = J_{d_0}$, for all $d \ge d_0$.
%
%For each $d$, let $F_d$ be a finite set of polynomials of
%degree~$d$, such that $\{\, \operatorname{lead}(f) \mid f \in
%F_d \,\}$ generates~$J_d$.
%
%Then $F_0 \cup F_1 \cup \ldots \cup F_{d_0}$ generates~$J$.
%(For any $f \in J$, there exists $f' \in \langle F_0 \cup F_1
%\cup \ldots \cup F_{d_0} \rangle$, such that $\deg f' = \deg
%f$, and $\operatorname{lead}(f') = \operatorname{lead}(f)$.
%Then $\deg(f - f') < \deg f$, so we may assume, by induction,
%that $f - f' \in \langle F_0 \cup F_1 \cup \ldots \cup
%F_{d_0} \rangle$.
% \end{proof}

\begin{prop}[(\thmindex{Hilbert Basis}Hilbert Basis Theorem)]
 For any field~$F$, the polynomial ring $F[x_1,\ldots,x_s]$
\textup(in any number of variables\textup) is
Noetherian.
 \end{prop}

%\begin{proof}
% Note that $F$ has only one proper ideal, namely~$\{0\}$, so
%it is obviously Noetherian. Now use \cref{R[x]Noetherian}
%to induct on~$s$.
% \end{proof}

%There are many equivalent formulations of the following
%important theorem.

\begin{thm} \label{Nullstellensatz-fg}
 Let $F$ be a subfield of a field~$L$. If $L$ is finitely
generated as an $F$-algebra \textup(that is, if there
exist $c_1,\ldots,c_r \in L$, such that $L =
F[c_1,\ldots,c_r]$\textup), then $L$~is algebraic over~$F$.
 \end{thm}

%\begin{proof}
% Suppose $L$ is transcendental over~$F$. (This will lead to a
%contradiction.) Let $\{x_1,\ldots,x_n\}$ be a transcendence
%basis for~$L$ over~$F$. (That is, $L$ is algebraic over
%$F[x_1,\ldots,x_n]$, and $x_k$ is transcendental over
%$F[x_1,\ldots,x_{k-1}]$, for each~$k$.) By replacing $F$ with
%$F[x_1,\ldots,x_{n-1}]$, we may assume $n = 1$.\
% Therefore, $L =
%F[x,a_1,\ldots,a_m]$, where each $a_j$ is algebraic over
%$F[x]$.
%
%From the proof of \cref{Zintegclosed}, we see that there
%exists $g \in F[x]$, such that $g a_1,\ldots,g a_m$ are
%integral over $F[x]$. Choose some irreducible $f \in F[x]$,
%such that 
% $$f \nmid g .$$
% We have
% $$ 1/f \in L
% = F[x,a_1,\ldots,a_m]
% = F[x,ga_1,\ldots,ga_m,1/g] ,$$
% so $g^k/f \in  F[x,ga_1,\ldots,ga_m]$, for some $k
%\ge 0$. Therefore, $g^k/f$ is integral over~$F[x]$.
%
%From the proof of \cref{Zintegclosed}, we conclude that
%$g^k/f \in F[x]$. (The ring $F[x]$ is integrally closed.) This
%contradicts the choice of~$f$.
% \end{proof}
%
%\begin{cor} \label{Nullstellensatz-maxideal}
% Let 
% \begin{itemize}
% \item $F$ be an algebraically closed field, 
% \item $F[x_1,\ldots,x_r]$ be a polynomial ring over~$F$, and
% \item $\mathfrak{m}$ be any maximal ideal of
%$F[x_1,\ldots,x_r]$.
% \end{itemize}
% Then 
% \begin{enumerate}
% \item \label{Nullstellensatz-maxideal-iso}
% the natural inclusion $F \hookrightarrow
%F[x_1,\ldots,x_r]/\mathfrak{m}$ is an isomorphism, and
% \item \label{Nullstellensatz-maxideal-roots}
% there exist $a_1,\ldots,a_r \in F$, such that
% $ \mathfrak{m} = \langle (x_1 - a_1), \ldots, (x_r -
%a_r) \rangle $.
% \end{enumerate}
% \end{cor}
%
%\begin{proof}
% \pref{Nullstellensatz-maxideal-iso} Let $L =
%F[x_1,\ldots,x_r]/\mathfrak{m}$. Then $L$~is a field (because
%$\mathfrak{m}$ is maximal), so \cref{Nullstellensatz-fg}
%implies $L$~is algebraic over~$F$. Since $F$~is algebraically
%closed, we conclude that $L = F$, as desired.
%
%\pref{Nullstellensatz-maxideal-roots} From
%\pref{Nullstellensatz-maxideal-iso}, we know there exists
%$a_j \in F$, such that $a_j \equiv x_j \pmod{\mathfrak{m}}$;
%let
% $$ I = \bigl\langle (x_1 - a_1), \ldots, (x_r - a_r)
%\bigr\rangle .$$
% From the choice of $a_1,\ldots,a_r$, we have $I \subset
%\mathfrak{m}$. On the other hand, it is easy to see that 
% $$F[x_1,\ldots,x_r] /I \iso F $$
% is a field, so $I$ must be a maximal ideal. Hence, it is
%equal to~$\mathfrak{m}$.
% \end{proof}

\begin{prop}[(\thmindex{Nullstellensatz}Nullstellensatz)] \label{Nullstellensatz}
 Let 
 \begin{itemize}
 \item $F$ be an algebraically closed field, 
 \item $F[x_1,\ldots,x_r]$ be a polynomial ring over~$F$, and
 \item $I$ be any proper ideal of $F[x_1,\ldots,x_r]$.
 \end{itemize}
 Then there exist $a_1,\ldots,a_r \in F$, such that
 $f(a_1,\ldots,a_r) = 0$ for all $f(x_1,\ldots,x_r) \in I$.
 \end{prop}

%\begin{proof}
% Let $\mathfrak{m}$ be a maximal ideal that contains~$I$, and
%choose $a_1,\ldots,a_r \in F$ as in
%\fullref{Nullstellensatz-maxideal}{roots}. Then
%$f(a_1,\ldots,a_r) = 0$ for all $f(x_1,\ldots,x_r) \in
%\mathfrak{m}$, so, since $I \subseteq \mathfrak{m}$, the desired
%conclusion follows.
% \end{proof}

\begin{cor} \label{Nullstellensatz-ringhomo}
 If $B$ is any finitely generated subring of\/~$\complex$, then
there is a nontrivial homomorphism from~$B$ to the
algebraic closure\/~$\overline{\rational}$ of\/~$\rational$.
 \end{cor}

%\begin{proof}
% We have $B = \integer[b_1,\ldots,b_r]$, for some
%$b_1,\ldots,b_r \in B$. There is a homomorphism 
% $$\phi \colon \overline{\rational}[x_1,\ldots,x_r] \to
%\complex ,$$
% defined by $\phi \bigl( f(x_1,\ldots,x_r) \bigr) =
%f(b_1,\ldots,b_r)$. Let $I$~be the kernel of~$\phi$, and
%choose a maximal ideal~$\mathfrak{m}$ that contains~$I$. Then
%$I \subseteq \mathfrak{m}$, so there is a natural homomorphism 
% \begin{align*}
%  B &= \phi \bigl( \integer[x_1,\ldots,x_r] \bigr)
% \subseteq \phi \bigl( \overline{\rational}[x_1,\ldots,x_r]
%\bigr) 
% \\&\iso \frac{ \overline{\rational}[x_1,\ldots,x_r] }{I}
% \to \frac{ \overline{\rational}[x_1,\ldots,x_r]
%}{\mathfrak{m}}
% \iso \overline{\rational} 
% \end{align*}
% \fullsee{Nullstellensatz-maxideal}{iso}.
% \end{proof}

%\begin{exercises}
%
%\item 
% Derive \cref{Nullstellensatz-fg} as a corollary of
%\cref{Nullstellensatz-maxideal}.
% \hint{Let $\overline{F}$ be the algebraic closure of~$F$. The
%proof of \cref{Nullstellensatz-ringhomo} shows that there
%is a nontrivial homomorphism $\phi \colon L \to
%\overline{F}$, such that $\phi|_F = \Id$. Show $\phi$~is
%injective, so $L$~is isomorphic to a subfield
%of~$\overline{F}$.}
%
% \end{exercises}
 
 
 


%\section{Eisenstein Criterion}
%
%\begin{notation}
%For $f(x) \in \integer[x]$, we use $\content f(x)
%$ to denote the \defit[content of an integral
%polynomial]{content} of $f(x)$, that is, the greatest common
%divisor of the coefficients of $f(x)$.
%\end{notation}
%
%\begin{lem} \label{content(product)}
%If $f(x),g(x) \in \integer[x]$, then $\content \bigl(f(x) g(x)
%\bigr)  = \bigl(\content f(x)
%\bigr) \cdot  \bigl( \content g(x)\bigr)$.
%\end{lem}
%
%\begin{proof}
%Dividing each polynomial by a scalar, we may assume $\content f(x)
% =  \content g(x) = 1$. If there is a prime divisor~$p$ of $\content \bigl(f(x) g(x)
%\bigr)$, then the product $f(x) \, g(x)$ is~$0$ modulo~$p$, contradicting the fact that $\integer_p[x]$ is an integral domain.
%\end{proof}
%
%\begin{lem} \label{irredZ->irredQ}
% If $f(x) \in \integer[x]$ is irreducible over~$\integer$,
%then it is irreducible over~$\rational$.
% \end{lem}
%
%\begin{proof}
% We prove the contrapositive: suppose $f(x)$ is reducible
%over~$\rational$. Clearing denominators, we may write $n
%f(x) = g_1(x) g_2(x)$, for some nonzero $n \in \integer$,
%with $g_j(x) \in \integer[x]$ and $\deg g_j(x) \ge 1$.
%
% Dividing $f(x)$ by an integer constant, we may assume
%$\content \bigl(f(x) \bigr) = 1$. Then, letting $d_j =
%\content \bigl(g_j(x) \bigr)$, we have
% $$ n = \content \bigl(n f(x) \bigr)
% = \content \bigl(g_1(x) g_2(x) \bigr)
% = d_1 d_2 . $$ 
% Therefore, letting $\hat g_j(x) =
%\frac{1}{d_j} g_j(x) \in \integer[x]$, we have
% $$ f(x) = \frac{g_1(x) g_2(x)}{d_1 d_2} = \hat  g_1(x)
%\hat g_2(x) ,$$
% so $f(x)$ is reducible over~$\integer$, as desired.
% \end{proof}

\begin{lem}[(\thmindex{Eisenstein Criterion}{Eisenstein Criterion})] \label{Eisenstein}
 Let $f(x) \in \integer[x]$. If there
is a prime number~$p$, and some $a \in \integer_p
\smallsetminus \{0\}$, such that 
 \begin{itemize}
 \item $f(x) \equiv a x^n \pmod{p}$, where $n = \deg f(x)$,
and
 \item $f(0) \not\equiv 0 \pmod{p^2}$,
 \end{itemize}
 then $f(x)$ is irreducible over~$\rational$.
 \end{lem}

%\begin{proof}
% Suppose $f(x)$ is reducible over~$\rational$. (This will
%lead to a contradiction.) Then $f(x)$ is also reducible
%over~$\integer$ \see{irredZ->irredQ}, so we may write $f(x)
%= g_1(x) g_2(x)$, with $g_j(x) \in \integer[x]$ and $\deg
%g_j(x) \ge 1$.  Then 
% $$ g_1(x) g_2(x) = f(x) \equiv a x^n \pmod{p} .$$
% From the unique factorization of polynomials in
%$\integer_p[x]$ (recall that $\integer_p[x]$ is a Euclidean
%domain, because $\integer_p$~is a field), we conclude that
%there exist $b_1,b_2 \in \integer_p \smallsetminus \{0\}$
%and $m_1 , m_2 \in \natural$, such that $g_j(x) \equiv b_j
%x^{m_j} \pmod{p}$. Since
% $$ m_1 + m_2 = n = \deg f(x) = \deg \bigl( g_1(x) g_2(x)
%\bigr) = \deg g_1(x) + \deg g_2(x) ,$$
% and $m_j \le \deg g_j(x)$, we conclude that $m_j = \deg
%g_j(x) > 1$. Therefore $g_j(0) \equiv 0 \pmod{p}$, so $f(0)
%= g_1(0) g_2(0) \equiv 0 \pmod{p^2}$. This is a
%contradiction.
% \end{proof}










\section{General topology} \label{TopologySect}

\begin{defns}
Let $X$ be a topological space.
	\begin{enumerate}
	\item A subset~$C$ of~$X$ is \defit{precompact} (or \defit{relatively compact}) if the closure of~$C$ is compact.
	\item $X$ is \defit[locally!compact]{locally compact} if every point of~$X$ is contained in a precompact, open subset.
	\item $X$ is \defit[separable topological space]{separable} if it has a countable, dense subset.
%	\item $X$ is \defit[metrizable topological space]{metrizable} if there is a metric on~$X$, such that the open sets in~$X$ are precisely the sets that are unions of open balls with respect to the metric.
	\item If $I$ is an index set (of any cardinality), and $X_i$ is a topological space, for each $i \in I$, then the Cartesian product $\bigtimes_{i \in I} X_i$ has a natural ``\defit[product!topology]{product topology}\zz,'' in which a set is open if and only if it is a union (possibly infinite) of sets of the form $\bigtimes_{i \in I} U_i$, where each $U_i$ is an open subset of~$X_i$, and we have $U_i = X_i$ for all but finitely many~$i$.
	
	\end{enumerate}
\end{defns}

\begin{thm}[(\thmindex{Tychonoff's}Tychonoff's Theorem)] \label{TychonoffThm}
If $X_i$ is a compact topological space, for each $i \in I$, then the Cartesian product $\bigtimes_{i \in I} X_i$ is also compact\/ \textup(with respect to the product topology\/\textup).
\end{thm}

\begin{prop}[(\thmindex{Zorn's Lemma}Zorn's Lemma)] \label{ZornsLemma}
Suppose $\le$ is a binary relation on a set~$\mathcal{P}$, such that:
	\begin{itemize}
	\item If $a \le b$ and $b \le c$, then $a \le c$.
	\item If $a \le b$ and $b \le a$, then $a = b$.
	\item $a \le a$ for all~$a$.
	\item If $\mathcal{C} \subseteq \mathcal{P}$, such that, for all $c_1,c_2 \in \mathcal{C}$, either $c_1 \le c_2$ or $c_2 \le c_1$, then there exists $b \in \mathcal{P}$, such that $c \le b$, for all $c \in \mathcal{C}$.
	\end{itemize}
Then there exists $a \in \mathcal{P}$, such that $a \not\le b$, for all $b \in \mathcal{P}$.
\end{prop}





\section{Measure theory} \label{MeasThySect}

\begin{assump}
Throughout this section, % @@@ \lcnamecref{MeasThySect}, 
$X$ and~$Y$ are complete, separable metric spaces. (Recall that \defit[complete metric space]{complete} means all Cauchy sequences converge.)
\end{assump}

\begin{defns} \ 
%Suppose $X$ and~$Y$ are topological spaces.
\noprelistbreak
	\begin{enumerate}
	\item The \defit[Borel!sigma-algebra@$\sigma$-algebra]{Borel $\sigma$-algebra} 
	\nindex{$\Borel(X)$ = $\{$Borel subsets of~$X$$\}$}
	$\Borel(X)$ of~$X$ is the smallest collection of subsets of~$X$ that:
		\begin{itemize}
		\item contains every open set,
		\item is closed under countable unions (that is, if $A_1,A_2,\ldots \in \Borel$, then $\bigcup_{i=1}^\infty A_i \in \Borel$),
		and
		\item is closed under complements (that is, if $A \in \Borel$, then $X \smallsetminus A \in \Borel$).
		\end{itemize}
	\item Each element of $\Borel(X)$ is called a \index{Borel!set}\defit[Borel!set]{Borel set}.
	\item A function $f \colon X \to Y$ is \defit[Borel!function]{Borel measurable} if $f^{-1}(A)$ is a Borel set in~$X$, for every Borel set~$A$ in~$Y$.
	\item A function $\mu \colon \Borel(X) \to [0,\infty]$ is called a \defit{measure} if it is \defit{countably additive}. This means that if $A_1,A_2,\ldots$ are pairwise disjoint, then 
		$$ \mu \left( \bigcup_{i=1}^\infty A_i \right) = \sum_{i = 1}^\infty \mu(A_i) .$$
	
	\item A measure~$\mu$ on~$X$ is \defit[measure!Radon]{Radon} if $\mu(C) < \infty$, for every compact subset~$C$ of~$X$.

	\item A measure~$\mu$ on~$X$ is \defit[sigma-finite measure@$\sigma$-finite measure]{$\sigma$-finite} if $X$ is the union of countably many sets of finite measure. This means $X = \bigcup_{i = 1}^\infty A_i$, with $\mu(A_i) < \infty$ for each~$i$.
	
	\end{enumerate}
\end{defns}

\begin{prop}
If $\mu$ is a measure on~$X$, and $f$~is a measurable function on~$X$, such that $f \ge 0$, then the integral $\int_X f \, d\mu$ is a well-defined element of $[0,\infty]$, such that:
	\begin{enumerate}
	\item $\int_X \chi_A \, d\mu = \mu(A)$ if $\chi_A$ is the characteristic function of~$A$.
	\item $\int_X (a_1 f_1 + a_2 f_2) \, d\mu = a_1 \int_X f_1 \, d\mu + a_2 \int_X f_2 \, d\mu$ for $a_1,a_2 \in [0,\infty)$.
	\item if $\{f_n\}$ is a sequence of measurable functions on~$X$, such that we have\/ $0 \le f_1 \le f_2 \le \cdots$, then
		$$ \int_X \ \lim_{n\to \infty} f_n \  d\mu = \lim_{n\to \infty} \int_X f_n \, d\mu . $$
	\end{enumerate}
\end{prop}

\begin{cor}[(Fatou's Lemma)] \label{FatousLemma}
If $\{f_n\}_{n=1}^\infty$ is a sequence of measurable functions on~$X$, with $f_n \ge 0$ for all~$n$, and $\mu$~is a measure on~$X$, then
	$$ \int_X \liminf_{n \to \infty} f_n \, d\mu \le \liminf_{n \to \infty} \int_X f_n \, d\mu .$$
\end{cor}

\begin{prop}
If $X$~is locally compact and separable, then every Radon measure~$\mu$ on~$X$ is \defit[regular!measure, inner]{inner regular}. This means 
	$$ \text{$\mu(E) = \sup\{\, \mu(C) \mid \text{$C$ is a compact subset of~$E$} \,\}$,
	\ for every Borel set~$E$}. $$
\end{prop}

\begin{prop}[(\thmindex{Lusin's}Lusin's Theorem)] \label{LusinsThm}
Assume $\mu$ is a Radon measure on~$X$, and $X$~is locally compact. Then, for every measurable function $f \colon X \to \real$, and every $\epsilon > 0$, there is a continuous function $g \colon X \to \real$, such that 
	$$ \mu \bigl( \{\, x \in X \mid f(x) \neq g(x) \,\} < \epsilon .$$
\end{prop}

\begin{defn} \label{PushForwardDefn}
If $\mu$ is a measure on~$X$, and $f \colon X \to Y$ is measurable, then the \defit[push-forward of measure~$\mu$]{push-forward} of~$\mu$ is the measure 
	\nindex{$f_*\mu$ = push-forward of measure~$\mu$}%
	$f_*\mu$ on~$Y$ that is defined by
		$$ (f_* \mu)(A) = \mu \bigl( f^{-1}(A) \bigr) 
		\quad \text{for $A \subseteq Y$} . $$
\end{defn}

\begin{prop}[{(\thmindex{Fubini's}Fubini's Theorem)}]
Assume
	\begin{itemize}
	\item $X_1$ and $X_2$ are complete, separable metric spaces, 
	and
	\item $\mu_i$ is a $\sigma$-finite measure on~$X_i$, for $i = 1,2$.
	\end{itemize}
Then there is a measure $\nu = \mu_1 \times \mu_2$ on $X_1 \times X_2$, such that:
	\begin{enumerate}
	\item $\nu(E_1 \times E_2) = \nu(E_1) \cdot \nu(E_2)$ when $E_i$ is a Borel subset of~$X_i$ for $i = 1,2$,
	and
	\item $\int_{X_1 \times X_2} f \, d\nu = \int_{X_1} \int_{X_2} f(x_1,x_2) \, d\mu_2(x_2) \, d \mu_1(x_1)$ when the function
	$f \colon X_1 \times X_2 \to [0,\infty]$ is Borel measurable.
	\end{enumerate}
\textup(In particular, $\int_{X_2} f(x_1,x_2) \, d\mu_2(x_2)$ is a measurable function of~$x_1$.\textup)
\end{prop}

\begin{defns} \ 
\noprelistbreak
\begin{enumerate} 
	\item The \defit{support} of a function $f \colon X \to \complex$ is defined to be the closure of $\{\, x \in X \mid f(x) \neq 0 \,\}$.
	\item \nindex{$C_c(X)$ = $\{$continuous functions with compact support$\}$}%
	$C_c(X) = \{\, \text{continuous functions $f \colon X \to \complex$ with compact support} \,\}$.

	\item $\lambda \colon C_c(X) \to \complex$ is a \defit[linear functional! positive]{positive linear functional} on $C_c(X)$ if:
		\begin{itemize}
		\item it is linear (that is, $\lambda( a_1 f_1 + a_2 f_2) = a_1 \lambda(f_1) + a_2 \lambda(f_2)$ for $a_1,a_2 \in \complex$ and $f_1,f_2 \in C(X)$),
		and
		\item it is positive (that is, if $f(x) \ge 0$ for all~$x$, then $\lambda(f) \ge 0$).
		\end{itemize}
\end{enumerate}
\end{defns}

\begin{thm}[(\thmindex{Riesz Representation}Riesz Representation Theorem)] \label{RieszRepThm}
Assume $X$~is locally compact and separable.
If\/ $\lambda$ is any positive linear functional on $C_c(X)$, then there is a Radon measure~$\mu$ on~$X$, such that
	$$ \lambda(f) = \int_X f \, d\mu 
	\quad \text{for all $f \in C_c(X)$} .$$
%Furthermore, $\mu(C) < \infty$ for every compact subset~$C$ of~$X$.
\end{thm}

\begin{defns}
Assume  $\mu$ is a measure on~$X$, and the function $\varphi \colon X \to \complex$ is measurable.

	\begin{enumerate}
	\item For $1 \le p < \infty$, the \defit[Lp-@$\LL{p}$-!norm]{$\pmbLL{p}$-norm} of~$\varphi$ is 
	$$ \|\varphi\|_p = \left( \int_{X} |\varphi(x)|^p \, d\mu(x) \right) ^{1/p} .$$
	
	\item An assertion $P(x)$ is said to be true for \defit[almost!all]{almost all} $x \in X$ (or to be true \defit[almost!everywhere]{almost everywhere}, which is usually abbreviated to \defit{a.e.}),  if $\mu \bigl( \{\, x \mid \text{$P(x)$ is false}\,\} \bigr) = 0$.
	
	\item In particular, two functions $\varphi_1$ and $\varphi_2$ are equal (a.e.) if 
		$$\mu\bigl( \{\, x \mid \varphi_1(x) \neq \varphi_2(x) \,\} \bigr) = 0 .$$ 
	This defines an equivalence relation on the set of (measurable) functions on~$X$.

	\item The \defit[Lq-norm@$\LL{\infty}$-norm]{$\pmbLL{\infty}$-norm} (or \defit{essential supremum}) of~$\varphi$ is 
	$$ \|\varphi\|_\infty = \min \bigl\{\, a \in (-\infty, \infty] \mid \text{$\varphi(x) \le a$ for a.e.~$x$} \,\bigr\} .$$

	\item \nindex{$ \LL{p}(X,\mu)$ = $\{$$\LL{p}$-functions on~$X$$\}$}
	$ \LL{p}(X,\mu) = \{\, \varphi \colon X \to \complex \mid \| \varphi \|_p < \infty \,\} $, for $1 \le p \le \infty$. An element of $\LL{p}(X,\mu)$ is called an \defit[Lp-@$\LL{p}$-!function]{$\pmbLL{p}$-function} on~$X$. Actually, two functions in $\LL{p}(X,\mu)$ are identified if they are equal almost everywhere, so, technically, $\LL{p}(X,\mu)$ should be defined to be a set of equivalence classes, instead of a set of functions.
	
	\end{enumerate}
%Note that $\|\varphi\|_p = 0$ if and only if $\varphi = 0$~a.e.
\end{defns}


\begin{defn}
Two measures $\mu$ and~$\nu$ on~$X$ are in the same \defit[measure!class]{measure class}
		%(or $\nu$ is \defit[equivalent!measure]{equivalent} to~$\mu$) 
if they have exactly the same sets of measure~$0$:
		$$ \mu(A) = 0 \iff \nu(A) = 0 .$$
	(This defines an equivalence relation.) 
%	Note that if $\nu = f \mu$, for some real-valued, measurable function~$f$, such that $f(x) \neq 0$ for a.e.\ $x \in X$, then $\mu$ and~$\nu$ are in the same measure class \csee{fmuClassOfMuEx}. 
%(The \term{Radon-Nikodym Theorem} implies that the converse is true,
%	% if the measures are $\sigma$-finite, 
%but we do not need this fact.)
\end{defn}

\begin{thm}[(\thmindex{Radon-Nikodym}Radon-Nikodym Theorem)] \label{RadonNikodym}
Two $\sigma$-finite measures $\mu$ and~$\nu$ on~$X$ are in the same class if and only if there is a measurable function $D \colon X \to \real^+$, such that $\mu = D \nu$. That is, for every measurable subset~$A$ of~$X$, we have
	$ \mu(A) = \int_A D \, d\nu $.
\end{thm}

The function~$D$ is called the \defit{Radon-Nikodym derivative} $d\mu/d\nu$.



\section{Functional analysis}

\begin{defns}
Let  $\field$ be either $\real$ or~$\complex$, and let $V$ be a vector space over~$\field$.
	\begin{enumerate}
	
	\item A \defit{topological vector space} is a vector space~$V$, with a topology, such that the operations of scalar multiplication and vector addition are continuous (that is, the natural maps $\field \times V \to V$ and $V \times V \to V$ are continuous).
	
	\item A subset~$C$ of~$V$ is \defit[convex set]{convex} if, for all $v,w \in C$ and $0 \le t \le 1$, we have $t v + (1-t) w \in C$.
	
	\item A topological vector space~$V$ is \defit[locally!convex]{locally convex} if every neighborhood of~$0$ contains a convex neighborhood of~$0$.
	
	\item A locally convex topological vector space~$V$ is \defit[Fréchet space]{Fréchet} if its topology can be given by a metric that is \defit[complete metric space]{complete} (that is, such that every Cauchy sequence converges to a limit point).
	
	\item A \defit{norm} on~$V$ is a function $\| \ \| \colon V \to [0,\infty)$, such that:
	\noprelistbreak
		\begin{enumerate}
		\item $\| v + w \| \le \| v \| + \| w \|$ for all $v , w \in V$,
		\item $\| av \| = |a| \, \|v\|$ for $a \in \field$ and $v \in V$,
		and
		\item $\|v\| = 0$ if and only if $v = 0$.
		\end{enumerate}
	Note that any norm $\|\ \|$ on~$V$ provides a metric that is defined by $d(v,w) = \| v - w \|$. Thus, the norm determines a topology on~$V$.
	
	\item A \defit[Banach!space]{Banach space} is a vector space~$\Banach$, together
	\nindex{$\Banach$ = a Banach space}
	with a norm~$\| \ \|$, such that the resulting metric is complete.
	 (Banach spaces are Fréchet.)
	
	\item An \defit[inner!product]{inner product} on~$V$ is a function $\langle \, \mid \, \rangle \colon V \times V \to \field$, such that
		\begin{enumerate}
		\item $\langle a v + b w \mid x \rangle = a \langle v | x \rangle + b \langle w | x \rangle$ for $a,b \in \field$ and $v,w,x \in V$,
		\item $\langle v \mid w \rangle = \overline{\langle w \mid v \rangle}$ for $v,w \in V$, where $\overline{a}$ denotes the complex conjugate of~$a$,
		and
		\item $\langle v \mid v \rangle \ge 0$ for all $v \in V$, with equality iff $v = 0$.
		\end{enumerate}
	Note that if $\langle \, \mid \, \rangle$ is an inner product on~$V$, then a norm on~$V$ is defined by the formula $\|v\| = \sqrt{\langle v \mid v \rangle}$.
	
	\item A \defit{Hilbert space} is a vector space~$\Hilbert$, 
		\nindex{$\Hilbert$ = a Hilbert space}
	together with an inner product $\langle \, \mid \, \rangle$, such that the resulting normed vector space is complete. (Hence, every Hilbert space is a Banach space.)
	
	\item An \defit[isomorphism!of Hilbert spaces]{isomorphism} between Hilbert spaces $\bigl( \Hilbert_1, \langle \, \mid \, \rangle_1 \bigr)$ and $\bigl( \Hilbert_2, \langle \, \mid \, \rangle_2 \bigr)$ is an invertible linear transformation~$T \colon \Hilbert_1 \to \Hilbert_2$, such that 
		$$ \text{$\langle Tv \mid Tw \rangle_2 = \langle v \mid w \rangle_1$ 
		\ for all $v,w \in \Hilbert_1$} .$$
		An isomorphism from~$\Hilbert$ to itself is called a \defit[unitary!operator]{unitary operator} on~$\Hilbert$.

	\end{enumerate}
\end{defns}

\begin{eg}
If $\mu$ is a measure on~$X$, then the $\LL{p}$-norm makes $\LL{p}(X,\mu)$ into a Banach space (for $1 \le p \le \infty$). Furthermore, $\LL2(X,\mu)$ is a Hilbert space, with the inner product
	$$ \langle \varphi \mid \psi \rangle = \int_X \varphi(x) \, \overline{\psi(x)} \, d\mu(x) .$$
\end{eg}

\begin{defns} \label{WeakStarDefn}
Let $\Banach$ be a Banach space (over $\field \in \{\real, \complex\}$).
	\begin{enumerate}
	\item A \defit[linear functional!continuous]{continuous linear functional} on~$\Banach$ is a continuous function $\lambda \colon \Banach \to \field$ that is linear (which means $\lambda (a v + b w) = a \lambda(v) + b \lambda(w)$ for $a,b \in \field$ and $v,w \in \Banach$).
	
	\item \nindex{$\Banach^*$ = dual of~$\Banach$}%
	$\Banach^* = \{\, \text{continuous linear functionals on~$\Banach$} \,\}$ is the \defit[dual of a Banach space]{dual} of~$\Banach$. This is a Banach space: the norm of a linear functional~$\lambda$ is
		$$ \| \lambda \| = \sup \{\, |\lambda(v)| \mid v \in \Banach, \ \|v\| \le 1 \,\} .$$
	
	\item For each $v \in \Banach$, there is a linear function $e_v \colon \Banach^* \to \field$, defined by $e_v(\lambda) = \lambda(v)$. The \defit{weak$^*$ topology} on $\Banach^*$ is the coarsest topology for which every $e_v$ is continuous. 
	
	In other words, the basic open sets in the weak$^*$ topology are of the form 
		$ %\mathcal{O}_v^{U} = 
		\{\, \lambda \in \Banach^* \mid \lambda(v) \in U \,\}$, for some $v \in \Banach$ and some open subset~$U$ of~$\field$. A set in $\Banach^*$ is open if and only if it is a union of sets that are finite intersections of basic open sets.

	\item Any continuous, linear transformation from~$\Banach$ to itself is called a \defit{bounded operator} on~$\Banach$.

	\item The set of bounded operators on~$\Banach$ is itself a Banach space, with the \defit{operator norm}
	\nindex{$\| T \|$ = operator norm}
	$$ \| T \| = \sup \{\, \|T(v)\| \mid \| v \| \le 1 \,\} .$$

	\end{enumerate}
\end{defns}

\begin{prop}[(\thmindex{Banach-Alaoglu}{Banach-Alaoglu Theorem})]  \label{BanachAlaogluThm}
If $\Banach$ is any Banach space, then the closed unit ball in $\Banach ^*$ is compact in the weak$^*$ topology.
%\hint{Let $D = B_1(\complex)$ be the unit disk in~$\complex$. There is a natural embedding of the closed unit ball $B_1(\Banach^*)$ in the infinite Cartesian product $\bigtimes_{v \in B_1(\Banach)} D$. This product is compact (by Tychonoff's Theorem~\pref{TychonoffThm}), and the map is a homeomorphism onto its image, which is closed.}
\end{prop}

\begin{prop}[(\thmindex{Hahn-Banach}Hahn-Banach Theorem)]
Suppose 
	\begin{itemize}
	\item $\Banach$ is a Banach space over~$\field$,
	\item $W$ is a subspace of~$\Banach$  \textup(not necessarily closed\/\textup),
	and
	\item $\lambda \colon W \to \field$ is linear.
	\end{itemize}
If\/ $|\lambda(w)| \le \|w\|$ for all $w \in W$, then $\lambda$~extends to a linear functional $\widehat \lambda \colon \Banach \to \field$, such that $|\widehat \lambda(v)| \le \|v\|$ for all $v \in \Banach$.
\end{prop}

\begin{prop}[(\thmindex{Open Mapping}Open Mapping Theorem)] \label{OpenMappingThm}
Assume $X$ and~$Y$ are Fréchet spaces, and $f \colon X \to Y$ is a continuous, linear map.
	\begin{enumerate}
	\item \label{OpenMappingThm-surj}
	If $f$ is surjective, and $\mathcal{O}$ is any open subset of~$X$, then $f(\mathcal{O})$ is open.
	\item \label{OpenMappingThm-bij}
	If $f$ is bijective, then the inverse $f^{-1} \colon Y \to X$ is continuous.
	\end{enumerate} 
\end{prop}

\begin{assump} \label{HilbertSpaceSeparable}
Hilbert spaces are always assumed to be separable.
\end{assump}

This has the following consequence:

\begin{prop}
There is only one infinite-dimensional Hilbert space\/ \textup(up to isomorphism\/\textup). In other words, every infinite-dimensional Hilbert space is isomorphic to $\LL2(\real, \mu)$, where $\mu$~is Lebesgue measure.
\end{prop}

\begin{defns} \ 
\noprelistbreak
	\begin{enumerate}
	\item If $\Hilbert_1$ and~$\Hilbert_2$ are Hilbert spaces, then the \defit[direct sum!of Hilbert spaces]{direct sum} $\Hilbert_1 \oplus \Hilbert_2$ is a Hilbert space, under the inner product 
	$$ \bigl\langle (\varphi_1,\varphi_2) \mid (\psi_1,\psi_2)  \bigr\rangle = \langle \varphi_1 \mid \psi_1 \rangle + \langle \varphi_2 \mid \psi_2 \rangle .$$
By induction, this determines the direct sum of any finite number of Hilbert spaces; see \cref{HilbertDirSumInfty} for the direct sum of infinitely many.

	\item We use 
	\nindex{$\perp$ = ``is orthogonal to''}%
	``$\perp$'' as an abbreviation for ``is orthogonal to\zz.'' Therefore, if $\varphi,\psi \in \Hilbert$, then $\varphi \perp \psi$ means $\langle \varphi \mid \psi \rangle = 0$.  For subspaces $\mathcal{K}, \mathcal{K}'$ of~$\Hilbert$, we write $\mathcal{K} \perp \mathcal{K}'$ if $\varphi \perp \varphi'$ for all $\varphi \in \mathcal{K}$ and $\varphi' \in \mathcal{K}'$.

	\item The \defit[orthogonal!complement]{orthogonal complement} of a subspace~$\mathcal{K}$ of~$\Hilbert$ is%
		\nindex{$\mathcal{K}^\perp$ = orthogonal complement of the subspace~$\mathcal{K}$}
	$$\mathcal{K}^\perp = \{\, \varphi \in \Hilbert \mid \varphi \perp \mathcal{K} \,\} .$$
	This is a closed subspace of~$\Hilbert$. We have $\Hilbert = \mathcal{K} + \mathcal{K}^\perp$ and $\mathcal{K} \perp \mathcal{K}^\perp$, so $\Hilbert = \mathcal{K} \oplus \mathcal{K}^\perp$.
	
	\item The \defit[orthogonal!projection]{orthogonal projection} onto a closed subspace~$\mathcal{K}$ of~$\Hilbert$ is the (unique) bounded operator $P \colon \Hilbert \to \mathcal{K}$, such that 
		\begin{itemize}
		\item $P(\varphi) = \varphi$ for all $\varphi \in \mathcal{K}$,
		and
		\item $P(\psi) = 0$ for all $\varphi \in \mathcal{K}^\perp$.
		\end{itemize}

	\end{enumerate}
\end{defns}

\begin{defns}
Let $T \colon \Hilbert \to \Hilbert$ be a bounded operator on a Hilbert space~$\Hilbert$.%
\noprelistbreak
	\begin{enumerate}
	\item The \defit[adjoint!of a linear transformation]{adjoint} of~$T$ is the bounded operator~$T^*$ on~$\Hilbert$, such that 
		$$ \text{$\langle T \varphi \mid \psi \rangle = \langle \varphi \mid T^* \psi \rangle$ \ for all $\varphi,\psi \in \Hilbert$} .$$
It does not always exist, but $T^*$ is unique if it does exist.
	
	\item $T$ is \defit[self-adjoint operator]{self-adjoint} (or \defit[Hermitian!operator]{Hermitian}) if $T = T^*$.
	
	\item $T$ is \defit[normal operator]{normal} if $T T^* = T^* T$.
	
	\item $T$ is \defit[compact!linear operator]{compact} if there is a nonempty, open subset~$\open$\, of~$\Hilbert$, such that $T(\open\,)$ is precompact.
	
	\end{enumerate}
\end{defns}

\begin{prop} \label{CpctOpBasics}
Let $T$ be a bounded operator on a Hilbert space~$\Hilbert$.
	\begin{enumerate}
	\item If $T(\Hilbert)$ is finite-dimensional, then $T$ is compact.
	\item The set of compact operators on~$\Hilbert$ is closed\/ \textup(in the topology defined by the operator norm\/\textup).
	\end{enumerate}
\end{prop}

\begin{prop}[(\thmindex{Spectral}Spectral Theorem)] \label{SpectralThm}
If $T$ is any bounded, normal operator on any Hilbert space~$\Hilbert$, then there exist
	\begin{itemize}
	\item a finite measure~$\mu$ on $[0,1]$, 
	\item a bounded, measurable function $f \colon [0,1] \to \complex$,
	and
	\item an isomorphism\/ $U \colon \Hilbert \to \LL2([0,1],\mu)$,
	\end{itemize}
such that $U( T \varphi ) = f \, U(\varphi)$, for all $\varphi \in \Hilbert$ \ \textup(where $f \, U(\varphi)$ denotes the pointwise multiplication of the functions $f$ and~$U(\varphi)$.
 %\in \LL2(X,\mu)$ is defined by $\bigl( f \, U(\varphi) \bigr)(x) = f(x) \, \bigl( \psi(v) \bigr)(x)$\textup).

Furthermore:
	\begin{enumerate} 
	\item $T$ is unitary if and only if $|f(x)| = 1$ for a.e.\ $x \in [0,1]$.
	\item $T$ is self-adjoint if and only if $f(x) \in \real$ for a.e.\ $x \in [0,1]$.
	\end{enumerate}
\end{prop}

\begin{defn} \label{SpectralMeasDefn}
In the situation of \cref{SpectralThm}, the \defit{spectral measure} of~$T$ is $f_*\mu$.
\end{defn}

\begin{cor}[(\thmindex{Spectral}Spectral Theorem for compact, self-adjoint operators)] \label{SpectralThmCpct}
Let $T$ be a bounded operator on any Hilbert space~$\Hilbert$. Then $T$ is both self-adjoint and compact if and only if there exists an orthonormal basis $\{e_n\}$ of~$\Hilbert$, such that
\noprelistbreak
	\begin{enumerate}
	\item each $e_n$ is an eigenvector of~$T$, with eigenvalue~$\lambda_n$,
	\item $\lambda_n \in \real$,
	and
	\item $\lim_{n \to \infty} \lambda_n = 0$.
	\end{enumerate}
\end{cor}

\begin{prop}[(\thmindex{Fréchet-Riesz}Fréchet-Riesz Theorem)]
If $\lambda$ is any continuous linear functional on a Hilbert space~$\Hilbert$, then there exists $\psi \in \Hilbert$, such that $\lambda(\varphi) = \langle \varphi \mid \psi \rangle$ for all $\varphi \in \Hilbert$.
\end{prop}



%\begin{notes}
%
%\end{notes}







