%!TEX root = IntroArithGrps.tex




\standassumpfalse

\mychapter{A Quick Look at \texorpdfstring{$S$}{S}-Arithmetic Groups}
\label{SarithChap}

%Much of the importance of arithmetic groups comes from being lattices in semisimple Lie groups.
Classically, and in the main text of this book, the Lie groups under consideration were manifolds over the field~$\real$ of real numbers. However, in some areas of modern mathematics, especially Number Theory and Geometric Group Theory, it is important to understand the lattices in Lie groups not only over the classical field~$\real$ (or~$\complex$), but also over ``nonarchimedean'' fields of $p$-adic numbers. The natural analogues of arithmetic groups in this setting are called ``$S$-arithmetic groups\zz.'' Roughly speaking, this generalization is obtained by replacing the ring~$\integer$ with a slightly larger ring.

\begin{defn}
For any finite set \nindex{$S$ = a finite set of prime numbers}$S = \{p_1,p_2,\ldots,p_n\}$ of prime numbers, let%
\nindex{$\integer_S = \integer \bigl[ 1/p_1, 1/p_2, \ldots, 1/p_n \bigr]$ is the ring of $S$-integers}
% no page break here !!!
	\begin{align*}
	\integer_S 
	&= \bigset{ \frac{p}{q} \in \rational }{ 
		\begin{matrix}
		\text{every prime factor} \\
		\text{of~$q$ is in~$S$} 
		\end{matrix}} 
	 =  \integer \bigl[ 1/p_1, 1/p_2, \ldots, 1/p_n \bigr]
	. \end{align*}
This is called the ring of \defit[S-@$S$-!integers]{$S$-integers}.
\end{defn}

\begin{eg} \label{SArithPrototype} \ 
\noprelistbreak
	\begin{enumerate}
	\item The prototypical example of an arithmetic group is $\SL(\ell, \integer)$.
	\item The corresponding example of an $S$-arithmetic group is $\SL \bigl( \ell, \integer_S \bigr)$ (where $S$ is a finite set of prime numbers).
	\end{enumerate}
That is, while arithmetic groups do not allow their matrix entries to have denominators, $S$-arithmetic groups allow their matrix entries to have denominators that are products of certain specified primes.
\end{eg}

Most of the results in this book can be generalized in a natural way to $S$-arithmetic groups. (The monographs \cite{MargulisBook} and~\cite{PlatonovRapinchukBook} treat $S$-arithmetic groups alongside arithmetic groups throughout.) We will now give a very brief description of these more general results.

\begin{rem}
The monograph \cite{MargulisBook} of Margulis deals with a more general class of $S$-arithmetic groups that allows $\real$ to be replaced with certain ``local'' fields of characteristic~$p$, but we discuss only the fields of characteristic~$0$. 
\end{rem}





\section{Introduction to \texorpdfstring{$S$}{S}-arithmetic groups} \label{IntroSArithSect}

Most of the theory in this book (and much of the importance of the theory of arithmetic groups) arises from the fundamental fact that $G_{\integer}$ is a lattice in~$G$. Since the ring $\integer_S$ is not discrete (unless $S = \emptyset$), the group $G_{\integer_S}$ is usually not discrete, so it is usually not a lattice in~$G$. Instead, it is a lattice in a group~$G_S$ that will be defined in this section. % \lcnamecref{IntroSArithSect}. @@@

The construction of~$\real$ as the completion of~$\rational$ can be generalized as follows: 

\begin{defn}[($p$-adic numbers)]
Let $p$ be a prime number. 
\noprelistbreak
	\begin{enumerate}
	\item If $x$ is any nonzero rational number, then there is a unique integer $v = v_p(x)$, such that we may write
		$$ x = p^v \, \frac{a}{b} ,$$
	where $a$ and~$b$ are relatively prime to~$p$. (We let $v_p(0) = \infty$.) Then $v_p(x)$ is called the \defit[p-adic@$p$-adic!valuation]{$p$-adic valuation} of~$x$.
%	The \defit[p-adic@$p$-adic!valuation]{$p$-adic valuation} $v_p(k)$ of a nonzero integer~$k$ is defined by the equation $k = p^{v_p(k)} k'$, where $k'$~is relatively prime to~$k$. This extends to the nonzero rational numbers by letting
%		$$ v_p(a/b) = v_p(a) - v_p(b) .$$
	\item Let
		$$ d_p(x,y) = p^{-v_p(x-y)} .$$
	It is easy to verify that $d_p$ is a metric on~$\rational$. It is called the \defit[p-adic@$p$-adic!metric]{$p$-adic metric}.
	\item Let \nindex{$\rational_p$ = field of $p$-adic numbers}$\rational_p$ be the completion of~$\rational$ with respect to this metric. (That is, $\rational_p$ is the set of equivalence classes of convergent Cauchy sequences.) This is a field that naturally contains~$\rational$. It is called the \defit[p-adic@$p$-adic!field]{field of $p$-adic numbers}.
	\item If $\GG$ is an algebraic group over~$\rational$, we can define the group $\GG(\rational_p)$ of $\rational_p$-points of~$\GG$.
	\end{enumerate}
\end{defn}

\begin{notation}
To discuss real numbers and $p$-adic numbers uniformly, it is helpful to let 
\nindex{$\rational_\infty = \real$}$\rational_\infty = \real$.
\end{notation}

The construction of arithmetic subgroups by restriction of scalars \csee{RestrictScalarsSect} is based on the fact that the ring~$\ints$ of integers in a number field~$F$ embeds as a cocompact, discrete subring in $\bigoplus_{v \in S_\infty} F_v$. Using this fact, it was shown that $\GG(\ints)$ is a lattice in $\bigtimes_{v \in S_{\infty}} \GG(F_v)$. 

Similarly, to obtain a lattice in a $p$-adic group $\GG(\integer_p)$, or, more generally, in a product $\bigtimes_{v \in S \cup \{\infty\}} \F_v$ of $p$-adic groups and real  groups, we note that 
	$$ \text{$\integer_S$ embeds as a cocompact, discrete subring in $\bigoplus_{p \in S \cup \{\infty\}} \rational_p$} .$$
Using this fact, it can be shown that 
	$$ \text{$\GG(\ints_S)$ is a lattice in $\GG_S = \bigtimes_{p \in S \cup \{\infty\}} \GG(\rational_p)$} . $$% no page break here !!!
\nindex{$\GG_S = \bigtimes_{p \in S \cup \{\infty\}} \GG(\rational_p)$}%
We call $\GG(\ints_S)$ an \defit[S-@$S$-!arithmetic]{$S$-arithmetic subgroup}.

\begin{eg}
Let $\GG$ be the special linear group $\SL_n$.
\noprelistbreak
	\begin{enumerate}
	\item Letting $S = \emptyset$, we have $\integer_S = \integer$ and $G_S = \SL(n,\real)$. So $\SL(n,\integer)$ is an $S$-arithmetic lattice in $\SL(n,\real)$. This is a special case of the fact that every arithmetic lattice is an $S$-arithmetic lattice (with $S = \emptyset$).
	\item Letting $S = \{p\}$, where $p$ is a prime, we see that $\SL \bigl( n, \integer[1/p] \bigr)$ is a lattice in $\SL(n,\real) \times \SL(n,\rational_p)$.
	\item More generally, letting $S =  \{p_1,p_2,\ldots,p_r\}$, where $p_1,\ldots,p_r$ are primes, we see that $\SL \bigl( n, \integer_S \bigr)$ is a lattice in 
		$$\SL(n,\real) \times \SL(n,\rational_{p_1}) \times \SL(n,\rational_{p_2}) \times \cdots  \times \SL(n,\rational_{p_r}) .$$
	\end{enumerate}
This is an elaboration of our previous comment that $\SL(\ell, \integer_S)$ is the prototypical example of an $S$-arithmetic group \csee{SArithPrototype}.
\end{eg}

\begin{rem}[{\cite[Chap.~7]{Brown-BuildingsBook}}]
In the study of arithmetic subgroups of a Lie group~$G$, the symmetric space $G/K$ is a very important tool.
In the theory of $S$-arithmetic subgroups of $\GG_S$, this role is taken over by a space called the \defit{Bruhat-Tits building} of~$\GG_S$.
It is a Cartesian product
	$$ X_S = (G/K) \times \bigtimes_{p \in S} X_p , $$
where $X_p$ is a contractible simplicial complex on which $\GG(\rational_p)$ acts properly (but not transitively).
\end{rem}


%\begingroup \smaller \baselineskip=10pt
\subsection*{Optional:}
Readers familiar with the basic facts of Algebraic Number Theory will realize that the above discussion has the following natural generalization:

\begin{defn}[({\cite[p.~61]{MargulisBook}, \cite[p.~267]{PlatonovRapinchukBook}})]
\label{SarithFDefn} 
Let
\noprelistbreak
	\begin{itemize}
	\item $\ints$ be the ring of integers of an algebraic number field~$F$, 
	\item $S$~be a finite set of finite places of~$F$, 
	and
	\item $\GG$ be a semisimple algebraic group over~$F$,
	and
	\item $\GG_S = \bigtimes_{v \in S \cup S_\infty} \GG(F_v)$.
	\end{itemize}
Then $\GG(\ints_S)$ is an \defit[S-@$S$-!arithmetic]{$S$-arithmetic subgroup} of~$\GG_S$.
\end{defn}

\begin{rem}
More generally, much as in \cref{ArithDefn}, if 
\noprelistbreak
	\begin{itemize}
	\item $\Gamma'$ is an $S$-arithmetic subgroup of~$\GG'_S$,
	and
	\item $\varphi \colon \GG'_S \to \GG_S$ is a surjective, continuous homomorphism, with compact kernel,
	\end{itemize}
then any subgroup of~$\GG_S$ that is commensurable to $\varphi(\Gamma')$ may be called an $S$-arithmetic subgroup of~$\GG_S$.
\end{rem}


\begin{thm}[{}{\cite[Thm.~5.7, p.~268]{PlatonovRapinchukBook}}]
Every $S$-arithmetic subgroup of\/~$\GG_S$ is a lattice in\/~$\GG_S$.
\end{thm}

%\endgroup




\section{List of results on \texorpdfstring{$S$}{S}-arithmetic groups}

\begin{warn}
The Standing Assumptions \pref{standassump} do \textbf{not} apply in this \lcnamecref{SarithChap}, because $G$ is not assumed to be a \emph{real} Lie group.
\end{warn}

Instead:

\begin{assump}
Throughout the remainder of this \lcnamecref{SarithChap}:
\noprelistbreak
	\begin{itemize}
	\item $\GG$ is a semisimple algebraic group over~$\rational$,
	\item $S$ is a finite set of prime numbers,
	and
	\item $\Gamma$ is an $S$-arithmetic lattice in $\GG_S = \bigtimes_{p \in S \cup \{\infty\}} \GG(\rational_p)$.
	\end{itemize}
To avoid trivialities, we assume $\GG_S$ is not compact, so $\Gamma$ is infinite.
\end{assump}

\begin{defn}
As a substitute for real rank in this setting, let%
\nindex{$\Srank \GG = \sum_{p \in S \cup \{\infty\}} \Qprank \GG$} % no page break here !!!
		$$\Srank \GG = \sum_{p \in S \cup \{\infty\}} \Qprank \GG .$$
\end{defn}

\begin{rem}
All of these results generalize to the setting of \cref{SarithFDefn}, but we restrict our discussion to~$\rational$ for simplicity.
\end{rem}

The following theorems on $S$-arithmetic groups are all stated without proof, but each result is provided with a reference for further reading. The reader should be aware that these references are almost always secondary sources, not the original appearance of the result in the literature.





%\subsection*{Definitions related to \cref{RrankChap} (Real Rank)} 
%\label{RrankForSArith}
%
%	\begin{itemize}
%	\sitem[\spref{TorusDefn}] Let $L$ be any field containing~$F$. A Zariski-closed, Zariski-connected subgroup~$T$ of $\GG$ is a \defit[torus subgroup]{torus} if $T$ is diagonalizable
%over the algebraic closure of~$L$; that is, if there exists $g \in
%\GL(n,\algclosure{L})$, such that $g^{-1} T g$ consists entirely of diagonal matrices.
%	\sitem[\spref{RsplitDefn}] For any $v \in S$, a torus~$T$ in~$\GG$ is \defit[Qp-split torus]{$F_v$-split}
%if $T$ is diagonalizable over~$F_v$.
% 	\sitem[\spref{RrankDefn}] For $v \in S$, 
% $\Fvrank(\GG)$ is the dimension of any maximal $F_v$-split
%torus of~$\GG$. (This does not depend on the choice of the
%maximal torus, because all maximal $F_v$-split tori
%of~$\GG$ are conjugate.)
%	\item As a substitute for $\Rrank$ in the setting of $S$-arithmetic groups, let 
%		$$\Srank G = \sum_{v \in S} \Fvrank(\GG) .$$
%	\end{itemize}


%\begin{notation}
%For convenience, we use $G$ as an abbreviation for $\GG_S$.
%\end{notation}

\subsection*{Results related to \cref{BasicLatticesChap} (Basic Properties of Lattices)} 

	\begin{slist}
	\sitem[\spref{G/GammaCpct<>NoAccPt}]
 $\Gamma \backslash \GG_S$ is compact if and only if the
identity element~$e$ is \textbf{not} an accumulation point
of $\Gamma^{\GG_S}$
\cite[Thm.~1.12, p.~22]{RaghunathanBook}.
	\sitem[\spref{GammaUnip->notcpct}]
 If $\Gamma$ has a nontrivial, unipotent element, then
$\Gamma \backslash \GG_S$ is not compact. In fact, Godement's Criterion \spref{GodementCriterion} tells us that the converse is also true.
	\sitem[\spref{BDT}] The Borel Density Theorem holds, for any continuous homomorphism $\rho \colon \GG_S \to \GL(V)$, where $V$ is a vector space over $\real$, $\complex$, or any $p$-adic field~$\rational_p$
\cite[Thm.~II.2.5 (and Lem.~II.2.3), p.~84]{MargulisBook}.
	\sitem[\spref{GammaFinPres}] $\Gamma$ is finitely presented 
	\cite[Thm.~5.11, p.~272]{PlatonovRapinchukBook}.
	\sitem[\spref{torsionfree}] (Selberg Lemma) $\Gamma$ has a torsion-free
subgroup of finite index
\cite[Thm.~6.11, p.~93]{RaghunathanBook}.
	\sitem[\spref{FreeInGamma}] (Tits Alternative) $\Gamma$ has a nonabelian
free subgroup
\cite[App.~B, pp.~351--353]{MargulisBook}.
	\end{slist}



\subsection*{Results related to \cref{ArithGrpsChap} (What is an Arithmetic Group?).}

	\begin{slist}
	\sitem[\spref{MargulisArith}] (Margulis Arithmeticity Theorem) If $\Srank \GG \ge 2$, then every irreducible lattice in~$\GG_S$ is $S$-arithmetic
\cite[Thm.~IX.1.10, p.~298, and Rem.~(vi) on p.~290]{MargulisBook}. 
(Note that our definition of irreducibility is stronger than the one used in \cite{MargulisBook}.)
	\sitem[\spref{GodementCriterion}] (Godement Criterion) $\Gamma \backslash \GG_S$ is compact if and only if $\Gamma$ has no nontrivial unipotent elements
\cite[Thm.~5.7(2), p.~268]{PlatonovRapinchukBook}.
	\end{slist}

\begin{rem*} \Cref{CpctOpenSubgrp->LattsCocpct} (easily) implies:
	\begin{enumerate}
	\item {\cite[Thm.~1]{Tamagawa-DiscSubgrpsPAdic}} If $v$~is any nonarchimedean place of~$F$, then every lattice in $\GG(F_v)$ is cocompact.
	\item If $\GG(S_\infty)$ is compact, then every lattice in~$G$ is cocompact.
	\end{enumerate}
\end{rem*}

\begin{warn*}
We know that if $\ints$ is the ring of integers of~$F$, then $\GG(\ints)$ embeds as an arithmetic lattice in $\bigtimes_{v \in S_\infty} \GG(F_v)$, but that restriction of scalars allows us to realize this same lattice as the $\integer$-points of an algebraic group defined over~$\rational$ \ccf{ResScal->Latt}. This means that all arithmetic groups can be found by using only algebraic groups that are defined over~$\rational$, not other number fields. It is important to realize that the same cannot be said for $S$-arithmetic groups: most extensions of~$\rational$ provide many $S$-arithmetic groups that cannot be obtained from~$\rational$. 

For example, suppose $p$~is a prime in~$\integer$, but $p$ factors in the integers~$\ints$ of an extension field, and $a$ is a prime factor of~$p$ in~$\ints$. Then the subgroup $\SL \bigl( 2, \ints[1/p] \bigr)$ can be obtained by restriction of scalars, but $\SL \bigl( 2, \ints[1/a] \bigr)$ is an $\{a\}$-arithmetic subgroup that cannot be obtained by this method.
\end{warn*}



\subsection*{A result related to \cref{AmenableChap} (Amenable Groups)}
	\begin{slist}
	\sitem[\spref{GNotAmen}]
For $v \in S$, if $\GG(F_v)$ is not compact, then $\GG(F_v)$ is not amenable
	 \cite[Rem.~8.7.11, p.~260]{ReiterStegeman-HarmAnalLocCpctGrps}.
	\end{slist}


\subsection*{Results related to \cref{KazhdanTChap} (Kazhdan's Property ($T$))}
	\begin{slist}
	\sitem[\spref{WhichGKazhdan}]
If $\Fvrank G \ge 2$, for every simple factor~$G$ of $\GG(F_v)$, and every $v \in S$, then
 $\GG_S$ has Kazhdan's property
 \cite[Cor.~III.5.4, p.~130]{MargulisBook}.
	\sitem[\spref{Kazhdan:G->Gamma}]
If $\GG_S$ has Kazhdan's property, then $\Gamma$ also has Kazhdan's property
\cite[Thm.~III.2.12, p.~117]{MargulisBook}.
	\sitem[\spref{KazhdanlatticeCor}]
 If $\Gamma$ has Kazhdan's property, then $\Gamma/[\Gamma ,\Gamma ]$ is finite
 \cite[Thm.~III.2.5, p.~115]{MargulisBook}.
	\end{slist}


\subsection*{A result related to \cref{MargulisSuperChap} (Margulis Superrigidity Theorem)}

\begin{assump*}
Assume 
	\begin{itemize}
	\item $\Srank G \ge 2$, 
	\item $\Gamma$ is irreducible,
	and
	\item $w$ is a place of some algebraic number field~$F'$.
	\end{itemize}
\end{assump*}

	\begin{slist}
	\sitem[\spref{MargSuperG'}] (Margulis Superrigidity Theorem
	\cite[Prop.~VII.5.3, p.~225]{MargulisBook})
	If
		\begin{itemize} \leftskip=1.25\parindent % !!!
		\item $\GG'$ is a Zariski-connected, noncompact, simple algebraic group over~$F'_w$, with trivial center, 
		and
		\item $\varphi \colon \Gamma \to \GG'(F'_w)$ is a homomorphism, such that $\varphi(\Gamma)$ is:
			\begin{itemize}  \leftskip = 2em % !!!
			\item Zariski dense in~$\GG'$,
			and \par
			\item not contained in any compact subgroup of~$\GG'(F'_w)$,
			\end{itemize}
		\end{itemize}
\leftskip=\sindent % slist lost its hangindent !!!
	then $\varphi$ extends to a continuous homomorphism $\widehat\varphi \colon \GG_S \to \GG'(F'_w)$. \par

\smallskip

Furthermore,  there is some $v \in S$, such that $F_v$ is isomorphic to a subfield of a finite extension of~$F'_w$.
	
%	\sitem[\spref{MargImg}]
%	If $\varphi \colon \Gamma \to \GL(n,F'_w)$ is any homomorphism, then the Zariski closure of $\varphi(\Gamma)$ is semisimple. 
%See \cite[Cor.~VII.6.18, pp.~249--150]{MargulisBook} for cocompact case

	\end{slist}

\begin{warn*}
\Cref{MargNoncpct} does not extend to the setting of $S$-arithmetic groups: for example,
the lattice $\SL(n,\integer)$ is not cocompact, but the image of the natural inclusion $\SL(n,\integer) \hookrightarrow \SL(n,\rational_p)$ is precompact.
\end{warn*}




\subsection*{Results related to \cref{NormalSubgroupChap} (Normal Subgroups of $\Gamma$)}

	\begin{slist}
	\sitem[\spref{MargNormalSubgrpsThm}] (Margulis Normal Subgroups Theorem \cite[Thm.~VIII.2.6, p.~265]{MargulisBook})
	Assume 
		\begin{itemize} \leftskip=1.25\parindent % !!!
		\item $\Srank \GG \ge 2$,
		\item $\Gamma$ is irreducible,
		and
		\item $N$ is a normal subgroup of~$\Gamma$.
		\end{itemize}
\leftskip=\sindent % slist lost its hangindent !!!
	Then either $N$ is finite, or $\Gamma/N$ is finite.

\leftskip=0pt % end of the hangindent problems !!!

	\sitem[\spref{Rrank1->GammaNotAlmSimple}]
	If $\Srank \GG = 1$, then $\Gamma$ has (many) normal subgroups~$N$, such that neither $N$ nor $\Gamma/N$ is finite
		 \cite[Cor.~7.6]{Lubotzky-LattRank1LocalFlds}.
	\end{slist}




\subsection*{Results related to \cref{ArithClassicalChap} (Arithmetic Subgroups of Classical Groups)}

	\begin{slist}
	
	\sitem[\spref{AlmostAllOverC}]
	Let $\overline{\rational_p}$ be the algebraic closure of~$\rational_p$. Then all but finitely many of the simple Lie groups over\/~$\overline{\rational_p}$ are isogenous to either\/ $\SL(n,\overline{\rational_p})$, $\SO(n,\overline{\rational_p})$, or\/ $\Sp(2n,\overline{\rational_p})$, for some~$n$ \cite[Thm.~11.4, pp.~57--58, and Thm.~18.4, p.~101]{Humphreys-LieAlg}.

	\sitem[\spref{RformsComplete}, \spref{ArithLattsAreClassical}] Every $\rational$-form or $\rational_p$-form of $\SL(n,\overline{\rational_p})$, $\SO(n,\overline{\rational_p})$, or $\Sp(n,\overline{\rational_p})$ is of classical type, except for some ``triality'' forms of $\SO(8,\overline{\rational_p})$ \ccf{D4weird} \cite[\S2.3]{PlatonovRapinchukBook}.

	\sitem[\spref{D4weird}] Unlike $\real$, the field $\rational_p$ has extensions of degree~$3$, so some $\rational_p$-forms of $\SO(8,\overline{\rational_p})$ are \term{triality} groups, even though there are no such $\real$-forms of $\SO(8,\complex)$.
	
	\sitem[\spref{GHasCpctLatt}]
	 $\GG_S$ has a cocompact, $S$-arithmetic lattice \cite{BorelHarder-exist}.

	\sitem[\spref{Isotypic->irred}]
	If $\GG$ is isotypic, then $\GG_S$ has a cocompact, irreducible lattice that is $S$-arithmetic \cite{BorelHarder-exist}.

	\sitem[\spref{Irred->Isotypic}]
	If $\GG_S$ has an irreducible, $S$-arithmetic lattice, then $\GG$ is isotypic.

	\end{slist}



\subsection*{A result related to \cref{ReductionChap} (Construction of a Coarse Fundamental Domain)}

 If $\fund$ is any coarse fundamental domain for $\GG(\integer)$ in $\GG(\real)$, then there is a compact subset~$C$ of $\bigtimes_{p \in S} \GG(\rational_p)$, such that $\fund \times C$ is a coarse fundamental domain for $\GG(\integer_S)$ in~$\GG_S$
\cite[Prop.~5.11, p.~267]{PlatonovRapinchukBook}.
% Let $\ints$ be the ring of integers in~$F$. If $\fund$ is any coarse fundamental domain for $\GG(\ints)$ in $\GG(S_\infty)$, then there is a compact subset~$C$ of $\bigtimes_{v \in S \smallsetminus S_\infty} \GG(F_v)$, such that $\fund \times C$ is a coarse fundamental domain for $\GG(\ints_S)$ in~$G$
%\cite[Prop.~5.11, p.~267]{PlatonovRapinchukBook}.

% L.\,Ji (arith grps: what, why, how) gives references, he gives:
% 
%A.Borel, J.P.Serre, Cohomologie dÕimmeubles et groups S-arithm«etiques, Topology 15
%(1976), 211-232.
% 
%A.Borel, Some finiteness properties of adele groups over number fields, Inst. Hautes «Etudes
%Sci. Publ. Math. 16 (1963) 5-30.
% 
%A.Borel, Arithmetic properties of linear algebraic groups, 1963 Proc. Internat. Congr.
%Mathematicians (Stockholm, 1962) pp.~10-22.

This implies that every $S$-arithmetic subgroup of~$\GG_S$ is a lattice \cite[Thm.~5.7, p.~268]{PlatonovRapinchukBook}, but the short proof outlined in \cref{SLNZISLATTSlickSect} does not seem to generalize to this setting.



\subsection*{Results related to \cref{RatnerChap} (Ratner's Theorems on Unipotent Flows)}
Ratner's three main theorems (\ref{Ratner-OrbitClosure}, \ref{Ratner-Equidistribution}, and \ref{Ratner-MeasClass})
%Orbit-Closure Theorem \pref{Ratner-OrbitClosure}, Equidistribution Theorem \pref{Ratner-Equidistribution}, and Classification of Invariant Measures \pref{Ratner-MeasClass} 
have all been generalized to the $S$-arithmetic setting by Ratner \cite{Ratner-CartProd,Ratner-Sarith} and Margulis-Tomanov \cite{MargulisTomanov-LocField,MargulisTomanov-AlmLinear} (independently).





\begin{references}{99}


\bibitem{BorelHarder-exist}
 A.\,Borel and G.\,Harder:
 Existence of discrete cocompact subgroups of reductive
groups over local fields,
 \emph{J. Reine Angew. Math.} 298 (1978) 53--64.
 \MR{0483367},
 \maynewline
 \url{http://eudml.org/doc/151965}
% \url{http://www.digizeitschriften.de/dms/resolveppn/?PPN=GDZPPN00219452X}

\bibitem{Brown-BuildingsBook}
K.\,Brown: 
\emph{Buildings}. 
Springer, New York, 1989. 
ISBN 0-387-96876-8,
\MR{0969123}.
\url{http://www.math.cornell.edu/~kbrown/buildings/brown1988.pdf}

\bibitem{Humphreys-LieAlg}
 J.\,E.\,Humphreys:
 \emph{Introduction to Lie Algebras and Representation
Theory.}
 Springer, {Berlin Heidelberg New York}, 1972.
 \MR{0323842}

\bibitem{Lubotzky-LattRank1LocalFlds}
 A.\,Lubotzky:
 Lattices in rank one Lie groups over local fields,
 \emph{Geom. Funct. Anal.} 1 (1991), no.~4, 406--431.
\MR{1132296},
\maynewline
\url{http://dx.doi.org/10.1007/BF01895641}

\bibitem{MargulisBook}
 G.\,A.\,Margulis:
 \emph{Discrete Subgroups of Semisimple Lie Groups.}
 Springer, {Berlin Heidelberg New York}, 1991.
 ISBN 3-540-12179-X,
\MR{1090825}

\bibitem{MargulisTomanov-LocField}
G.\,A.\,Margulis and G.\,M.\,Tomanov:
Invariant measures for actions of unipotent groups over local fields on homogeneous spaces,
\emph{Invent. Math.}  116  (1994),  no.~1--3, 347--392.
 \MR{1253197},
\maynewline
\url{http://eudml.org/doc/144192}
%\url{http://www.digizeitschriften.de/dms/resolveppn/?PPN=GDZPPN002111802}
 
 \bibitem{MargulisTomanov-AlmLinear}
G.\,A.\,Margulis and G.\,M.\,Tomanov:
Measure rigidity for almost linear groups and its applications,
\emph{J. Anal. Math.}  69  (1996), 25--54.
 \MR{1428093},
 \maynewline
 \url{http://dx.doi.org/10.1007/BF02787100}

\bibitem{PlatonovRapinchukBook}
 V.\,Platonov and A.\,Rapinchuk: 
 \emph{Algebraic Groups and Number Theory.}
 Academic Press, Boston, 1994.
 ISBN 0-12-558180-7,
 \MR{1278263}

\bibitem{RaghunathanBook}
 M.\,S.\,Raghunathan: 
 \emph{Discrete Subgroups of Lie Groups.}
 Springer, {New York}, 1972.
 ISBN 0-387-05749-8,
\MR{0507234}

\bibitem{Ratner-CartProd}
M.\,Ratner:
Raghunathan's conjectures for Cartesian products of real and $p$-adic Lie groups,
\emph{Duke Math. J.} 77 (1995), no.~2, 275--382. 
\MR{1321062},
\maynewline
\url{http://dx.doi.org/10.1215/S0012-7094-95-07710-2}

\bibitem{Ratner-Sarith}
M.\,Ratner:
On the $p$-adic and $S$-arithmetic generalizations of Raghunathan's conjectures,
in S.\,G.\,Dani, ed.:
\emph{Lie Groups and Ergodic Theory (Mumbai, 1996).}
%Tata Inst. Fund. Res. Stud. Math., 14, 
Tata Inst. Fund. Res., Bombay, 1998, pp.~167--202.
\MR{1699365}

\bibitem{ReiterStegeman-HarmAnalLocCpctGrps}
H.\,Reiter and J.\,D.\,Stegeman:
\emph{Classical Harmonic Analysis and Locally Compact Groups}.
%Second edition. London Mathematical Society Monographs. New Series, 22. The Clarendon Press, 
Oxford University Press, New York, 2000. 
ISBN 0-19-851189-2,
\MR{1802924}

\bibitem{Tamagawa-DiscSubgrpsPAdic}
T.\,Tamagawa:
On discrete subgroups of $p$-adic algebraic groups,
in O.\,F.\,G.\,Schilling, ed.:
\emph{Arithmetical Algebraic Geometry (Proc. Conf. Purdue Univ., 1963)}.
Harper \& Row, New York, 1965, pp.~11--17.
\MR{0195864}

\end{references}
