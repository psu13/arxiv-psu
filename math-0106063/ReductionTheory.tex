%!TEX root = IntroArithGrps.tex

\mychapter{Construction of a Coarse\texorpdfstring{\\}{ }Fundamental Domain} 
\label{ReductionChap}

\prereqs{$\rational$-rank (\cref{QrankChap}). \emph{Recommended:} Siegel sets for $\SL(n,\integer)$ (\cref{IwasawaSLnZ,SiegelSLnZSect,SLNZISLATTSiegelPfSect}).}


The ordinary $2$-torus is often depicted as a square with opposite sides identified, and it would be useful to have a similar representation of $\Gamma  \backslash G$, so we would like to construct a fundamental domain for $\Gamma$ in~$G$. 
Unfortunately, it is usually not feasible to do this explicitly, so, as in \cref{SLnZLattChap}, where we showed that $\SL(n,\integer)$ is a lattice in $\SL(n,\real)$, we will make do with a nice set that is close to being a fundamental domain:

\begin{defn}[\ccf{CoarseFundDomDefn}] \label{CoarseFundDomDefnRedux}
 A subset $\fund$ of~$G$ is called a \defit[fundamental!domain!coarse]{coarse fundamental
domain} for~$\Gamma$ in~$G$ if 
 \begin{enumerate}
 \item $\Gamma \fund = G$, and
 \item \label{CoarseFundDomDefnRedux-finite}
 $\{\, \gamma \in \Gamma \mid \fund \cap \gamma \fund \neq \emptyset \,\}$ is finite.
 \end{enumerate}
\end{defn}

The main result is \cref{ReductThyArithGrps}, which states that the desired set~$\fund$ can be constructed as a finite union of (translates of) ``Siegel sets'' in~$G$. 
Applications of the construction are described in \cref{ReductionAppsSect}.

%We assume familiarity with the special case where $G = \SL(n,\real)$ and $\Gamma = \SL(n,\integer)$, which was discussed in \Cref{SLnZLattChap}.


\section{What is a Siegel set?}

Before defining Siegel sets in every semisimple group, we recall the following special case:

\begin{defn}[\ccf{SiegelSLnZDefn}] \label{WhatIsSiegelSLnROnly}
A \defit{Siegel set} for $\SL(n,\integer)$ is a set of the form $\Siegel_{\overline{N},c} = \overline{N} \, A_c \, K \subseteq \SL(n,\real)$, where 
	\begin{itemize}
	\item $\overline{N}$ is a compact subset of the group~$N$ of upper-triangular unipotent matrices,
%		$$ N = \begin{bmatrix} 1\\ 
%		\BigSymbol{0}{0}{-12}& \BigSymbol{*}{15}{9}\ddots& \\ && 1 \end{bmatrix} $$
	\item $ A_c = \{\, a \in A \mid \text{$a_{i-1,i-1}   \ge c \, a_{i,i}$ for $i = 1,\ldots,n-1$} \,\} $, where $A$ is the group of positive-definite diagonal matrices (and $c > 0$),
%		$$ A = \begin{bmatrix} *\\ 
%		\BigSymbol{0}{0}{-12}& \BigSymbol{0}{15}{7}\ddots& \\ && * \end{bmatrix} 
%		\qquad \begin{pmatrix} \text{with all diagonal} \\ \text{entries positive} \end{pmatrix} ,$$
	and
	\item $K = \SO(n)$.
	\end{itemize}
\end{defn}

In this section, we generalize this notion by replacing $\SL(n,\integer)$ with any arithmetic subgroup (or, more generally, any lattice) in any semisimple Lie group~$G$.
To this end, note that the subgroups $N$, $A$, and~$K$ above are the components of the Iwasawa decomposition $G = KAN$ (or $G = NAK$), which can be defined for any semisimple group \csee{IwasawaDecomp}:
	\noprelistbreak
	\begin{itemize}
	\item %A subgroup of $\SL(n,\real)$ is \defit[unipotent!subgroup]{unipotent} iff it is conjugate to a subgroup of~$N$ \fullcsee{unipEx}{SubgrpInN}.	Therefore, 
	$N$ is a {maximal} unipotent subgroup of~$G$.
	\item %A connected subgroup of $\SL(n,\real)$ is an \defit[torus!R-split@$\real$-split]{$\real$-split torus} iff it is conjugate to a subgroup of~$A$ \csee{RsplitDefn}. Therefore, 
	$A$ is a {maximal} $\real$-split torus of~$G$ that normalizes~$N$,
	and
	\item %Every compact subgroup of $\SL(n,\real)$ is conjugate to a subgroup of $\SO(n)$ \csee{ConjToSOn}. Therefore, 
	$K$ is a {maximal} compact subgroup of~$G$.
	\end{itemize}
%Hence, we have natural characterizations of the subgroups $N$, $A$, and~$K$ in any semisimple group, and the following result generalizes the Iwasawa decomposition of $\SL(n,\real)$ that was stated in \cref{IwasawaDecompSLnR}:
%
%\begin{thm}[(Iwasawa Decomposition)] %% moved to Rrank
% Let 
% \noprelistbreak
% \begin{itemize}
% \item $K$ be a maximal compact subgroup of~$G$,
% \item $A$ be a maximal $\real$-split torus of~$G$, 
% and
% \item $N$ be a maximal unipotent subgroup of~$G$,
% \end{itemize}
% such that $AN$ is a subgroup of~$G$. 
%
%Then $G = K A N$.
% In fact, every $g \in G$ has a \bemph{unique} representation of the form $g = k a u$ with $k \in K$, $a \in A$, and $u \in N$. 
% \end{thm}

Now, to construct Siegel sets in the general case, we will do two things. First, we rephrase \cref{WhatIsSiegelSLnROnly} in a way that does not refer to any specific realization of~$G$ as a matrix group. To this end, recall that, for $G = \SL(n,\real)$, the \emph{\term[Weyl chamber, positive]{positive Weyl chamber}} is
	$$ A^+ = \{\, a \in A \mid \text{$a_{i,i} > a_{i+1,i+1}$ for $i = 1,\ldots,n-1$} \,\} . $$
Therefore, in the notation of \cref{WhatIsSiegelSLnROnly}, we have
	$ A^+ = A_1 $, and, for any $c > 0$, it is not difficult to see that there exists some $a \in A$, such that $A_c =  a A^+$ \csee{Ac=aA+}.
Therefore, letting $C = \overline{N} a$, we see that 
	$$ \text{$\Siegel_{\overline{N},c} = C A^+ K$, and $C$ is a compact subset of~$NA$.} $$
This description of Siegel sets can be generalized in a natural way to any semisimple group~$G$. 

However, all of the above is based entirely on the structure of~$G$, with no mention of~$\Gamma$, but a coarse fundamental domain~$\fund$ needs to be constructed with a particular arithmetic subgroup~$\Gamma$ in mind.
For example, if $\Gamma \backslash G$ is compact (or, in other words, if $\Qrank \Gamma = 0$), then our coarse fundamental domain needs to be compact, so none of the factors in the definition of a Siegel set can be unbounded. 
Therefore, we need to replace the maximal $\real$-split torus~$A$ with a smaller torus~$S$ that reflects the choice of a particular subgroup~$\Gamma$. In fact, $S$ will be the trivial torus when $G/\Gamma$ is compact. In general, $S$ is a maximal $\rational$-split torus of~$G$ (hence, $S$~is compact if and only if $\Gamma \backslash G$ is compact).

Now, if $S$ is properly contained in~$A$, then $NSK$ is not all of~$G$. Hence, $NS$ will usually not be the appropriate replacement for the subgroup $NA$. Instead, if we note that $NA$ is the identity component of a minimal parabolic subgroup of $\SL(n,\real)$ \fullcsee{ParabEgs}{SLn}, and that $NA$ is obviously defined over~$\rational$, then it is natural to replace $NA$ with a minimal parabolic $\rational$-subgroup~$P$ of~$G$.

The following definition implements these considerations. 

\begin{defn} \label{ArithSiegelDefn} 
Assume
\noprelistbreak
	\begin{itemize}
	\item $G$ is defined over~$\rational$,
	\item $\Gamma$ is commensurable to~$G_{\integer}$,
	\item $P$ is a minimal parabolic $\rational$-subgroup of~$G$,
	\item $S$ is a maximal $\rational$-split torus that is contained in~$P$,
	\item $S^+$ is the positive Weyl chamber in~$S$ (with respect to~$P$), 
	\item $K$ is a maximal compact subgroup of~$G$,
	and 
	\item $C$ is any nonempty, compact subset of~$P$.
	\end{itemize}
%We fix the choice of $S$, $S^+$, $P$, and~$K$ for the remainder of this chapter.
Then 
	\ $\Siegel = \Siegel_C = C \, S^+ K $ \ 
is a \defit{Siegel set} for~$\Gamma$ in~$G$.
\end{defn}

\begin{warn}
Our definition of a Siegel set is slightly more general than what is usually found in the literature, because other authors place some restrictions on the compact set~$C$.
For example, it is often assumed that $C$ has nonempty interior.
\end{warn}



\begin{exercises}

\item \label{SiegelinS^+C}
Show that if $\Siegel = C \, S^+ K$ is a Siegel set, then there is a compact subset~$C'$ of~$G$, such that $\Siegel \subseteq S^+ C'$.
\hint{Conjugation by any element of~$S^+$ centralizes $MS$ and contracts~$N$ (where $P = MSN$ is the Langlands decomposition).}

\item \label{CpctInSiegel}
For every compact subset~$C$ of~$G$, show there is a Siegel set that contains~$C$.
\hint{\Cref{G=KP}.}

\end{exercises}








\section{Coarse fundamental domains made from Siegel sets} \label{FundFromSiegelSect}

%It is well-known that every Riemann surface of finite volume has only finitely many cusps. In particular, since the Siegel set pictured in \cref{WeakFunDomSL2ZFig} is a coarse fundamental domain for the action of $\SL(2,\integer)$ on~$\hyperbolic$, it is clear that $\SL(2,\integer) \backslash \hyperbolic $ has only one cusp. Indeed, if $\Gamma$ is any lattice in $G = \SL(2,\real)$, and some Siegel set is a coarse fundamental domain for~$\Gamma$ in~$G$, then $\Gamma \backslash\hyperbolic$ must have only one cusp.
%Since there are examples of (noncompact) hyperbolic surfaces with more than one cusp, this implies that 
%not every lattice has a coarse fundamental domain that consists of a Siegel set.

\begin{eg} \label{SiegelNotFundEg}
Let
\noprelistbreak
	\begin{itemize}
	\item $G = \SL(2,\real)$,
	\item $\Siegel$ be a Siegel set that is a coarse fundamental domain for $\SL(2,\integer)$ in~$G$ \csee{WeakFunDomSL2ZFig},
	and
	\item $\Gamma$ be a subgroup of finite index in~$\SL(2,\integer)$.
	\end{itemize}
Then $\Siegel $ may not be a coarse fundamental domain for~$\Gamma$, because $\Gamma \Siegel$ may not be all of~$G$. In fact, if the hyperbolic surface $\Gamma \backslash \hyperbolic$ has more than one cusp, then no Siegel set is a coarse fundamental domain for~$\Gamma$.

However, if we let $F$ be a set of coset representatives for~$\Gamma$ in $\SL(2,\integer)$, then
	$F \Siegel $
is a coarse fundamental domain for~$\Gamma$ \fullcsee{WeakFundDomFinInd}{IsFund}.
\end{eg}

From the above example, we see that a coarse fundamental domain can sometimes be the union of several translates of a Siegel set, even in cases where it cannot be a single Siegel set.
In fact, this construction always works (if $\Gamma$ is arithmetic):

\begin{namedthm}[\thmindex{Reduction Theory for Arithmetic Groups}Reduction Theory for Arithmetic Groups]
\label{ReductThyArithGrps}
If\/ $\Gamma$ is commensurable to~$G_{\integer}$,
then there exist a Siegel set\/~$\Siegel$ and a finite subset~$F$ of~$G_{\rational}$, such that $\fund = F \, \Siegel$ is a coarse fundamental domain for\/ $\Gamma$ in~$G$.
\end{namedthm}

The proof will be given in \cref{RedThyPfSect}.

\medbreak

Although the statement of this result only applies to arithmetic lattices, it can be generalized to the non-arithmetic case. However, this extension requires a notion of Siegel sets in groups that are not defined over~$\rational$. The following definition reduces this problem to the case where $\Gamma$ is irreducible.

\begin{defn}
If $\Siegel_i$ is a Siegel set for~$\Gamma_i$ in~$G_i$, for $i = 1,2,\ldots,n$, then 
	$$ \Siegel_1 \times \Siegel_2 \times \cdots \times \Siegel_n $$ 
is a \defit{Siegel set} for the lattice $\Gamma_1 \times \cdots \times \Gamma_n$ in $G_1 \times \cdots \times G_n$.
\end{defn}

Then, by the Margulis Arithmeticity Theorem \pref{MargulisArith}, all that remains is to define Siegel sets for lattices in $\SO(1,n)$ and $\SU(1,n)$, but we can use the same definition for all simple groups of real rank one:

\begin{defn} \label{rank1SiegelDefn}
Assume $G$ is simple, $\Rrank G = 1$, and $K$ is a maximal compact subgroup of~$G$.
\noprelistbreak
	\begin{enumerate} \setcounter{enumi}{-1}
	\item If $\Qrank \Gamma = 0$, and $C$ is any compact subset of~$G$, then 
	$ \Siegel = C K $ is a \defit{Siegel set} in~$G$.
	\item Assume now that $\Qrank \Gamma = 1$. Let
		$P$ be a minimal parabolic subgroup of~$G$, with Langlands decomposition $P = MAN$, such that
			\begin{align} \tag{\ref{rank1SiegelDefn}$N$} \label{rank1SiegelDefnN}
			\text{$\Gamma \cap N$ is a maximal unipotent subgroup of~$\Gamma$.}
			\end{align}
If 
	\begin{itemize}
	\item $C$ is any compact subset of~$P$, 
	and 
	\item $A^+$ is the positive Weyl chamber of~$A$ \textup(with respect to~$P$\textup), 
	\end{itemize}
then
	\ $ \Siegel = C A^+ K $ \ 
	is a \defit[Siegel set!generalized]{generalized Siegel set} in~$G$.
	\end{enumerate}
\end{defn}

\begin{rem} \label{Rank1NLattIffQ}
If $\Gamma$ is commensurable to~$G_\integer$ (and $G$ is defined over~$\rational$), then \pref{rank1SiegelDefnN} holds if and only if $P$~is defined over~$\rational$ (and is therefore a minimal parabolic $\rational$-subgroup).
\end{rem}

We can now state a suitable generalization of \cref{ReductThyArithGrps}:

\begin{thm} \label{ReductThyNonarith}
If $G$ has no compact factors, then there exist a generalized Siegel set\/~$\Siegel$ and a finite subset~$F$ of~$G$, such that $\fund = F \, \Siegel$ is a coarse fundamental domain for\/~$\Gamma$ in~$G$.
\end{thm}

The proof is essentially the same as for \cref{ReductThyArithGrps}.

\begin{exercises}

\item \label{GZCpctSiegel=G}
Without using any of the results in this chapter (other than the definitions of ``Siegel set'' and ``coarse fundamental domain''), show that if $\Qrank \Gamma = 0$, then some Siegel set is a coarse fundamental domain for~$\Gamma$ in~$G$.
%\hint{There is a compact subset~$C_0$ of~$G$ with $\Gamma C_0 = G$, and $C_0 K$ is a Siegel set.}

\item Suppose $\fund_1$ and~$\fund_2$ are coarse fundamental domains for $\Gamma_1$ and~$\Gamma_2$ in $G_1$ and~$G_2$, respectively. Show that $\fund_1 \times \fund_2$ is a coarse fundamental domain for $\Gamma_1 \times \Gamma_2$ in $G_1 \times G_2$.

\item Suppose $N$ is a compact, normal subgroup of~$G$, and let $\overline{\Gamma}$ be the image of~$\Gamma$ in $\overline{G} = G/N$. Show that if $\overline{\fund}$ is a coarse fundamental domain for~$\overline{\Gamma}$ in~$\overline{G}$, then 
	$$ \fund = \{\, g \in G \mid gN \in \overline{\fund} \,\} $$
is a coarse fundamental domain for~$\Gamma$ in~$G$.

\item If $G$ is simple, $\Rrank G = 1$, and $G$ is defined over~$\rational$, then \cref{ArithSiegelDefn,rank1SiegelDefn} give two different definitions of the Siegel sets for~$G_\integer$. Show that \cref{rank1SiegelDefn} is more general: any Siegel set according to \cref{ArithSiegelDefn} is also a Siegel set by the other definition.
\hint{\cref{Rank1NLattIffQ}.}

\end{exercises}











\section{Applications of reduction theory} \label{ReductionAppsSect}

Having a coarse fundamental domain is very helpful for understanding the geometry and topology of $\Gamma \backslash G$. Here are a few examples of this (with only sketches of the proofs).





\subsection{$\Gamma$ is finitely presented} \label{FinPresSect}
\Cref{FundDom->FinPres} tells us that if $\Gamma$ has a coarse fundamental domain that is a connected, open subset of~$G$, then $\Gamma$ is finitely presented. The coarse fundamental domains constructed in \cref{ReductThyArithGrps,ReductThyNonarith} are closed, rather than open, but it is easy to deal with this minor technical issue: 

\begin{defn}
A subset $\openSiegel$ of~$G$ is an \defit[Siegel set!open]{open Siegel set} if
$\openSiegel = \open \, S^+ K$, where $\open$\, is a nonempty, precompact, open subset of~$P$.
\end{defn}

Choose a maximal compact subgroup~$K$ of~$G$ that contains a maximal compact subgroup of $\czer_G(S)$. Then we may let:
	\begin{itemize}
	\item $\fund = F \, \Siegel$ be a coarse fundamental domain, with $\Siegel = C \, S^+ K$, such that $C \subseteq P^\circ$ and $\fund$ is connected \csee{CanHaveCinP}, 
	\item $\open$\, be a connected, open, precompact subset of~$P^\circ$ that contains~$C$,
	\item $\openSiegel = \open \,S^+ K$ be the corresponding open Siegel set,
	and
	\item $\openfund = F \, \openSiegel$.
	\end{itemize}
Then $\openfund$ is a coarse fundamental domain for~$\Gamma$ \csee{openfundIsFund}, and $\openfund$ is both connected and open.

This establishes \cref{GammaFinPres}, which stated (without proof) that $\Gamma$~is finitely presented.



\subsection{Mostow Rigidity Theorem}
When $\Qrank \Gamma_1 = 1$, G.\,Prasad constructed a quasi-isometry $\varphi \colon G_1/K_1 \to G_2/K_2$ from an isomorphism $\rho \colon \Gamma_1 \to \Gamma_2$, by using the Siegel-set description of the coarse fundamental domain for $\Gamma_i \backslash G_i$. % would be good to give more explanation @@@
This completed the proof of the Mostow Rigidity Theorem \pref{MostowRigidity}.


\subsection{Divergent torus orbits} \label{DivTorusSect}

\begin{defn}
Let $T$ be an $\real$-split torus in~$G$, and let $x \in G/\Gamma$. We say the $T$-orbit of~$x$ is \defit[divergent torus orbit]{divergent} if the natural map $T \to Tx$ is proper.
\end{defn}

\begin{thm} \label{Qrank=DivTorus}
$\Qrank \Gamma$ is the maximal dimension of an $\real$-split torus that has a divergent orbit on $G/\Gamma$.
\end{thm}

We start with the easy half of the proof:

\begin{lem}[\ccf{SplitDiv}]
Assume $G$ is defined over~$\rational$, and let $S$ be a maximal $\rational$-split torus in~$G$ \textup(so $\dim S = \Qrank G_\integer$\textup). Then the $S$-orbit of $e G_\integer$ is divergent.
\end{lem}

Now, the other half:

\begin{thm}
If $T$ is an\/ $\real$-split torus, and\/ $\dim T > \Qrank \Gamma$, then no $T$-orbit in\/ $G/\Gamma$ is divergent.
\end{thm}

\begin{proof}[Proof \normalfont(assuming $\Qrank \Gamma = 1$)]
Let $T$ be a $2$-dimensional, $\real$-split torus~$T$ of~$G$, and define $\pi \colon T \to G/\Gamma$, by $\pi(t) = t \Gamma$. Suppose $\pi$ is proper. (This will lead to a contradiction.)

Let $P$ be a minimal parabolic $\rational$-subgroup of~$G$, and let $S$ be a maximal $\rational$-split torus in~$P$. For simplicity, let us assume that $\Gamma = G_{\integer}$, and also that a single open Siegel set $\openSiegel = K S^- \open$\, provides a coarse fundamental domain for~$\Gamma$ in~$G$. (Note that, since we are considering $G/\Gamma$, instead of $\Gamma \backslash G$, we have reversed the order of the factors in the definition of the Siegel set, and we use the opposite Weyl chamber.)

Choose a large, compact subset~$C$ of $G/\Gamma$, and let $T_R$ be a large circle in~$T$ that is centered at~$e$. Since $\pi$ is proper, we may assume $T_R$ is so large that $\pi(T_R)$ is disjoint from~$C$. Since $T_R$ is connected, this implies $\pi(T_R)$ is contained in a connected component of the complement of~$C$. So there exists $\gamma \in \Gamma$, such that $T_R \subseteq \Siegel P_{\integer}  \gamma$ (cf.\ \cref{CuspGroup} below). 

Let $t \in T_R$, and assume, for simplicity, that $\gamma = e$. Then $t \in \Siegel P_{\integer}$, and, since $T_R$ is closed under inverses, we see that $\Siegel P_{\integer}$ also contains~$t^{-1}$. However, it is not difficult to see that conjugation by any large element of $\Siegel P_{\integer}$ expands the volume form on~$P$ \csee{SiegelContracts}. Since the inverse of an expanding element is a contracting element, not an expanding element, this is a contradiction.
\end{proof}

\Cref{Qrank=DivTorus} can be restated in the following geometric terms:

\begin{thm}[\csee{QrankFlats}]
$\Qrank \Gamma$ is the largest natural number~$r$, such
that some finite cover of the locally symmetric space\/ $\Gamma \backslash G/K$ contains a closed, simply connected, $r$-dimensional flat.
 \end{thm}




\subsection{The large-scale geometry of locally symmetric spaces} \label{LargeScaleSect}

If we let $\pi \colon G \to \Gamma \backslash G / K$ be the natural map,
then it is not difficult to see that the restriction of~$\pi$ to any Siegel set is proper \csee{ProperOnSiegel}. 
In fact, with much more work (which we omit), it can be shown that the restriction of~$\pi$ is very close to being an isometry:

\begin{thm} \label{AlmIsomOnSiegel}
If\/ $\Siegel = C \, S^+ K$ is any Siegel set, and
	$$ \text{$\pi \colon G \to \Gamma \backslash G / K$ is the natural map,} $$
then there exists $c \in \real^+$, such that, for all $x,y \in \Siegel$, we have
	$$ d \bigl( \pi(x), \pi(y) \bigr) \le d( x,y) \le d \bigl( \pi(x), \pi(y) \bigr) + c .$$
\end{thm}

This allows us to describe the precise shape of the the locally symmetric space associated to~$\Gamma$, up to quasi-isometry:

\begin{thm}
Let 
	\begin{itemize}
	\item $X = \Gamma \backslash G/K$ be the locally symmetric space associated to~$\Gamma$, 
	and 
	\item $r = \Qrank \Gamma$.
	\end{itemize}
Then $X$ is quasi-isometric to the cone on a certain $(r-1)$-dimensional simplicial complex at~$\infty$.
\end{thm}

\begin{proof}[Idea of proof]
Modulo quasi-isometry, any features of bounded size in~$X$ can be completely ignored. Note that:
	\begin{itemize}
	\item \cref{AlmIsomOnSiegel} tells us that, up to a bounded error, $\Siegel$ looks the same as its image in~$X$. 	
	\item There is a compact subset~$C'$ of~$G$, such that $\Siegel \subseteq S^+C'$ \csee{SiegelinS^+C}, so every element of~$\Siegel$ is within a bounded distance of~$S^+$. 
Therefore, $\Siegel$ and~$S^+$ are indistinguishable, up to quasi-isometry.
	\end{itemize}
Then, since $\fund = F \, \Siegel$ covers all of~$X$, we conclude that $X$ is quasi-isometric to $ \bigcup_{f \in F} f S^+ $.

The Weyl chamber $S^+$ is a cone; more precisely, it is the cone on an $(r-1)$-simplex at~$\infty$. Therefore, up to quasi-isometry, $X$~is the union of these finitely many cones, so it is the cone on some $(r-1)$-dimensional simplicial complex at~$\infty$.
\end{proof}
 
\begin{rems} \label{LargeScaleRems} \ 
\noprelistbreak
	\begin{enumerate}

	\item The same argument shows that we get the same picture if, instead of looking at~$X$ modulo quasi-isometry, we look at it from farther and farther away, as in the definition of the asymptotic cone of~$X$ in \cref{AsympConeDefn}. Therefore, the asymptotic cone of~$X$ is the cone on a certain $(r-1)$-dimensional simplicial complex at~$\infty$. This establishes \cref{HattoriThm}.

	\item \label{LargeScaleRems-Tits}
	For a reader familiar with ``Tits buildings\zz,'' the proof (and the construction of~$\fund$) shows that this simplicial complex at~$\infty$ can be constructed by taking the \term{Tits building} of parabolic $\rational$-subgroups of~$G$, and modding out by the action of~$\Gamma$.

	\end{enumerate}
\end{rems}

\begin{exercises}

\item  \label{FinManyFinSubgrps}
Show that
$\Gamma$ has only finitely many conjugacy classes of finite subgroups.
\hint{If $H$ is a finite subgroup of~$\Gamma$, then $H^g \subseteq K$, for some $g \in G$. Write $g = \gamma x$, with $\gamma \in \Gamma$ and $x \in \fund$. Then
	$ H^\gamma x 
%	=  \gamma^{-1} H \gamma x
%	= \gamma^{-1} H g
%	= (\gamma^{-1} g) (g^{-1} H g) 
	= x \cdot H^g
%	\subseteq \fund \cdot K
%	= \fund
	\subseteq \fund$,
so $H$ is conjugate to a subset of
	$ \{\, \gamma \in \Gamma \mid \fund \cap \gamma\fund \neq \emptyset \,\} $.}

\item \label{SplitDiv}
Let $G = \SL(n,\real)$, $\Gamma = \SL(n,\integer)$, and $S$~be the group of positive-definite diagonal matrices. Show the $S$-orbit of $\Gamma e$ is proper.
\hint{If $s_{j,j}/s_{i,i}$ is large, then conjugation by~$s$ contracts a unipotent matrix~$\gamma$ whose only off-diagonal entry is~$\gamma_{i,j}$.}

%\item Show that if $\fund = F \Siegel$, where $\Siegel$ is a Siegel set and $F$ is a finite set, then $\fund = \fund K$.

\item Show that every open Siegel set is an open subset of~$G$ (so the terminology is consistent).

\item \label{CanHaveCinP}
Assume 
	\begin{itemize}
	\item $K$ contains a maximal compact subgroup of $\czer_G(S)$,
	\item $C$ is a compact subset of~$P$, 
	and 
	\item $F$ is a finite subset of~$G$.
	\end{itemize}
Show there is a compact subset $C_\circ$ of~$P^\circ$, such that $C S^+ K \subseteq C_\circ S^+ K$ and $F C_\circ S^+ K$ is connected.
\hint{Show $P^\circ \bigl( K \cap \czer_P(S) \bigr) = P$.}

\item \label{openfundIsFund}
Show the set~$\openfund$ constructed in \cref{FinPresSect} is indeed a coarse fundamental domain for~$\Gamma$ in~$G$.
\hint{\Cref{F1inFinF2}.}

\item \label{ProperOnSiegel}
Let $\pi \colon G \to \Gamma \backslash G$ be the natural map.
Show that if $\Siegel = C S^+ K$ is a Siegel set for~$\Gamma$, then the restriction of~$\pi$ to~$\Siegel$ is proper.
\hint{Let $v$ be a nontrivial element of $N \cap \Gamma$. If $g$ is a large element of $\Siegel$, then $v^g \approx e$.}

\item \label{SiegelContracts}
Show that if $\open$\, is contained in a compact subset of~$P$, then conjugation by any large element of $K S^- \open\, P_{\integer}$ expands the Haar measure on~$P$.
\hint{Conjugation by any element of $M \cup N$ preserves the measure, conjugation by an element of~$\open$\, is bounded, and $S^-$~centralizes $SM$ and expands~$N$. Also note that $P_\integer \doteq M_\integer N_\integer$ \csee{MNZinPZ}.}

\end{exercises}











\section{Outline of the proof of reduction theory} \label{RedThyPfSect}

\begin{notation}
Throughout this \lcnamecref{RedThyPfSect}, we assume 
	\begin{itemize}
	\item $G$~is defined over~$\rational$, 
	\item $\Gamma$~is commensurable to~$G_\integer$,
	and 
	\item $P$ is a minimal parabolic $\rational$-subgroup of~$G$, with Langlands decomposition $P = MSN$.
	\end{itemize}
\end{notation}

In order to use Siegel sets to construct a coarse fundamental domain, a bit of care needs to be taken when choosing a maximal compact subgroup~$K$. Before stating the precise condition, we recall that the \defit[Cartan!involution]{Cartan involution} corresponding to~$K$ is an automorphism~$\tau$ of~$G$, such that $\tau^2$ is the identity, and $K$~is the set of fixed points of~$\tau$. (For example, if $G = \SL(n,\real)$ and $K = \SO(n)$, then $\tau(g) = (g^\transpose)^{-1}$ is the transpose-inverse of~$g$.)

\begin{defn}
A Siegel set $\Siegel = C \, S^+ K$ is \defit[Siegel set!normal]{normal} if $S$ is invariant under the Cartan involution corresponding to~$K$.
\end{defn}

Fix a normal Siegel set $\Siegel = C \, S^+ K$, and some finite $F \subseteq G_\rational$. Then, letting $\fund = F \, \Siegel$, the proof of \cref{ReductThyArithGrps} has two parts, corresponding to the two conditions in the definition of coarse fundamental domain \pref{CoarseFundDomDefnRedux}:
	\begin{enumerate} \renewcommand{\theenumi}{\roman{enumi}}
	\item $\Siegel$ and~$F$ can be chosen so that $\Gamma \fund = G$ \csee{GZFSiegel=G,GammaG/P}, 
	and
	\item for all choices of $\Siegel$ and~$F$, the set 
		$\{\, \gamma \in \Gamma \mid \fund \cap \gamma \fund \neq \emptyset \,\}$ is finite \csee{SiegelProperty}.
	\end{enumerate}
We will sketch proofs of both parts (assuming $\Qrank \Gamma = 1$).
However, as a practical matter, the methods of proof are not as important as understanding the construction of the coarse fundamental domain as a union of Siegel sets \csee{FundFromSiegelSect}, and being able to use this in applications (as in \cref{ReductionAppsSect}). 
%So this material should be considered optional.

\subsection{Proof that $\Gamma \fund = G$}
Here is the rough idea: Fix a base point in $\Gamma \backslash G$.  A Siegel set can easily cover all of the nearby points \csee{CpctInSiegel}, so consider a point $\Gamma g$ that is far away. Godement's Criterion \pref{GodementCriterion} implies there is some nontrivial unipotent $v \in \Gamma$, such that $v^g \approx e$. Replacing $g$ with a different representative of the coset replaces $v$ with a conjugate element. If we assume all the maximal unipotent $\rational$-subgroups of~$G$ are conjugate under~$\Gamma$, this implies that we may assume $v \in N$. If we furthermore assume, for simplicity, that the maximal $\rational$-split torus~$S$ is actually a maximal $\real$-split torus, then the Iwasawa decomposition \pref{IwasawaDecomp} tells us $G = NSK$. The compact group~$K$ is contained in our Siegel set~$\Siegel$, and the subgroup~$N$ is contained in $\Gamma \Siegel$ if $\Siegel$ is sufficiently large, so let us assume $g \in S$. Since $g$~contracts the element~$v$ of~$N$, and, by definition, $S^+$~consists of the elements of~$S$ that contract~$N$, we conclude that $g \in S^+ \subseteq \Siegel$.

\medbreak

We now explain how to turn this outline into a proof.

Recall that all minimal parabolic $\rational$-subgroups of~$G$ are conjugate under~$G_{\rational}$ \fullcsee{parab/Q}{conj}. 
The following technical result from the algebraic theory of arithmetic groups asserts that there are only finitely many conjugacy classes under the much smaller group~$G_{\integer}$. In geometric terms, it is a generalization of the fact that hyperbolic manifolds of finite volume have only finitely many cusps.

\begin{thm} \label{GammaG/P}
There is a finite subset~$F$ of~$G_{\rational}$, such that\/ $\Gamma \, F \, P_{\rational} = G_{\rational}$.
\end{thm}

The finite subset~$F$ provided by the theorem can be used to construct the coarse fundamental domain~$\fund$:

\begin{thm} \label{GZFSiegel=G}
If $F$ is a finite subset of~$G_{\rational}$, such that\/ $\Gamma F P_{\rational} = G_{\rational}$, then there is a\/ \textup(normal\/\textup) Siegel set $\Siegel = C \, S^+ K$, such that\/ $\Gamma F \Siegel = G$.
\end{thm}

\begin{proof}[Idea of proof \normalfont (assuming $\Qrank \Gamma \le 1$)]
For simplicity, assume $\Gamma = G_{\integer}$, and that $F = \{e\}$ has only one element \csee{GZFSiegel=GBigFEx}, so
	\begin{align} \label{AssumeGzPQ=GQ}
	\Gamma \, P_{\rational} = G_{\rational}
	. \end{align}
The theorem is trivial if $\Gamma$ is cocompact \csee{GZCpctSiegel=G}, so let us assume $\Qrank \Gamma = 1$. 

From the proof of the Godement Compactness Criterion \pref{GodementCriterion}, we have a compact subset~$C_0$ of~$G$, such that, for each $g \in G$, either $g \in \Gamma C_0$, or there is a nontrivial unipotent element~$v$ of~$\Gamma$, such that $v^g \approx e$. By choosing $C$ large enough, we may assume $C_0 \subseteq \Siegel$ \csee{CpctInSiegel}.

Now suppose some element $g$ of~$G$ is not in $\Gamma \Siegel$.
Then $g \notin \Gamma C_0$, so there is a nontrivial unipotent element~$v$ of~$\Gamma$, such that 
	\begin{align} \label{Siegelvsmall}
	v^g \approx e 
	. \end{align}
From \pref{AssumeGzPQ=GQ}, we see that we may assume $v \in N$, after multiplying~$g$ on the left by an element of~$\Gamma$ \csee{SiegelvInN}.

We have $G = PK$ \ccf{G=KP}. Furthermore, $P = MSN$, and $\Gamma$ intersects both~$M$ and~$N$ in a cocompact lattice (see \fullcref{QrankEg}{0}, 
\fullcref{LanglandsDecompQ}{M=0}, and \cref{U/UZcpct}).
Therefore, if we multiply $g$ on the left by an element of $\Gamma \cap P$, and ignore a bounded error, we may assume $g \in S$ \csee{SiegelgInS}. Then, since $\Qrank \Gamma = 1$, we have either $g \in S^+$ or $g^{-1} \in S^+$ \csee{S+orS-}. From \pref{Siegelvsmall}, we conclude it is $g$ that is in~$S^+$. So $g \in \Siegel$, which contradicts the fact that $g \notin \Gamma \Siegel$.
\end{proof}

\begin{rem} \label{SiegelMReductive}
The above proof overlooks a technical issue: in the Langlands decomposition $P = MSN$, the subgroup~$M$ may be reductive, rather than semisimple. However, the maximality of~$S$ implies that the central torus~$T$ of~$M$ has no $\real$-split subtori \fullccf{LanglandsDecompQ}{M=0}, so it can be shown that this implies $T/T_{\integer}$ is compact.
Therefore $M/M_\integer$ is compact, even if $M$ is not semisimple.
\end{rem}



\subsection{Proof that $\Siegel$ intersects only finitely many $\Gamma$-translates} \label{SiegelPropertySect}
We know that $\Gamma$ is commensurable to~$G_\integer$. Therefore, if we make the minor assumption that $G_\complex$ has trivial center, then $\Gamma \subseteq G_\rational$ \csee{Comm=GC}. Hence, the following result establishes Condition~\fullref{CoarseFundDomDefnRedux}{finite} for $\fund = F \Siegel$:

\begin{thm}[(``Siegel property'')] \label{SiegelProperty}
If
\noprelistbreak
	\begin{itemize}
%	\item $\Gamma$ is commensurable to $G_{\integer}$,
%	\item $S$ is a maximal $\rational$-split torus in~$G$,
	\item $\Siegel = C S^+ K$ is a normal Siegel set,
	and
	\item $q \in G_{\rational}$,
	\end{itemize}
then 
	$\{\, \gamma \in G_\integer \mid q \Siegel \cap  \gamma \Siegel \neq \emptyset \,\}$ 
is finite.
\end{thm}

\begin{proof}[Proof \normalfont (assuming $\Qrank \Gamma \le 1$)]
The desired conclusion is obvious if $\Siegel$ is compact, so we may assume $\Qrank \Gamma = 1$. To simplify matters, let $\Gamma = G_{\integer}$, and
	$$ \text{assume $q = e$ is trivial.} $$

The proof is by contradiction: assume 
	$$\sigma = \gamma \sigma'  , $$
for some large element $\gamma$ of~$\Gamma$, and some $\sigma, \sigma' \in \Siegel$.
Since $\gamma$ is large, we may assume $\sigma$~is large (by interchanging $\sigma$ with $\sigma'$ and  replacing $\gamma$ with~$\gamma^{-1}$, if necessary).
Let 
	$$ \text{$u$ be an element of $N_{\integer}$ of bounded size.} $$
Since $\sigma \in \Siegel = C S^+ K$, we may write 
	$$ \text{$\sigma = c s k$ with $c \in C$, $s \in S^+$, and $k \in K$. } $$
Then $s$ must be large (since $K$ and~$C$ are compact), so conjugation by~$s$ performs a large contraction on~$N$. 
Since $u^c$ is an element of~$N$ of bounded size, and $K$~is compact, this implies that 
	$u^\sigma \approx e $.
In other words, 
	$$u^{\gamma \sigma'} \approx e .$$
In addition, we know that $u^{\gamma} \in G_{\integer}$. Since $\sigma' \in \Siegel$, we conclude that $u^{\gamma} \in N$ \csee{OnlyUnipPShrinks}.

Now we use the assumption that $\Qrank \Gamma = 1$: since 
	$$u^\gamma \in N \cap N^\gamma ,$$
\cref{Qrank1UniqMaxUnip} tells us that $N = N^\gamma$, so \fullcref{parab/Q}{nzer} implies
	$$ \gamma \in \nzer_G(N) = P .$$
Then, since $(MN)_{\integer}$ has finite index in~$P_{\integer}$ \csee{MNZinPZ}, we may assume $\gamma \in (MN)_{\integer}$.

This implies that we may work inside of~$MN$: if we choose a compact subset $\overline{C} \subseteq MN$, such that 
	$C S^+ \subseteq \overline{C} S$, 
then we have
	\begin{align} \label{SiegelInP}
	\overline{C} (K \cap P) \cap \gamma \overline{C} (K \cap P) \neq \emptyset 
	\end{align}
\csee{SiegelInPEx}. 
Since $\overline{C} (K \cap P)$ is compact (and $\Gamma$ is discrete), we conclude that there are only finitely many possibilities for~$\gamma$.
%
%\begin{case}
%The general case. (large $\rational$-rank)
%\end{case}
%We argue roughly as in \cref{SiegelPropertyPf-Qrank1}, but somewhat more is involved. we do not provide all of the details \csee{SiegelPropertyFullPf}.
%
%Suppose 
%	$$ \text{$\{\, \gamma \in \Gamma \mid \Siegel \cap \gamma \Siegel \neq \emptyset \,\}$ contains an infinite subset $\Gamma_{\Siegel}$.} $$ 
%For $\gamma \in \Gamma_{\Siegel}$, there exist $\sigma, \sigma' \in \Siegel$ with $\sigma = \gamma \sigma'$, and we may write $\sigma = c s k$, with $c \in C$, $s \in S^+$, and $k \in K$. By replacing $\Gamma_{\Siegel}$ with a subset, we may assume, for each simple root~$\alpha$, 
%	\begin{itemize}
%	\item[] either $\{\alpha(s)\}$ is bounded,
%	\item[] or $\alpha(s)$ is large for all but finitely many $\gamma \in \Gamma_{\Siegel}$.
%	\end{itemize}
%Let 
%	\begin{itemize}
%	\item $\Sigma = \Bigl\{\, \alpha \in \Delta \Big| \text{$\{ \alpha(s) \}$ is unbounded} \,\Bigr\} $,
%	\item $Q = P_\Sigma$ be the corresponding parabolic $\rational$-subgroup that contains~$P$,
%	and
%	\item $Q = MNT$ be a Langlands decomposition, with $T \subseteq S$.
%	\end{itemize}
%For any small neighborhood~$W$ of~$e$ in~$G$, we have
%	$$ \bigl\langle (G_{\integer})^\sigma \cap W  \bigr\rangle = (N_{\integer})^\sigma $$
%for all but finitely many $\gamma \in \Gamma_{\Siegel}$.
%The same is true for~$\sigma'$, so
%	$$ \bigl\langle (G_{\integer})^{\gamma \sigma} \cap W  \bigr\rangle =  (N_{\integer})^{\gamma \sigma}.$$
%Therefore, $N$ and~$N^\gamma$ must both be contracted by~$\sigma$, so, since $\sigma \in \Siegel$, we conclude that both $N$ and $N^\gamma$ are contained in $\unip P$. Therefore \cref{unipQ1ConjTounipQ2} tells us that $Q = Q^\gamma$. In other words, $\gamma \in \nzer_G(Q) = Q$.
%
%Hence 
%	$$ \text{$\Siegel_Q \cap \gamma \Siegel_Q \neq \emptyset$, where 
%	$\Siegel_Q = \Siegel \cap Q = C S^+ (K \cap Q)$.} $$
%Choose a compact subset $C'$ of~$MN$ with $C' T = CT$, let
%	$$ \Siegel_{MN} = C' (S \cap M)^+ (K \cap M) ,$$
%and note that \Cref{NormalSiegel->KcapQ} tells us that $K \cap Q = K \cap M \subseteq \czer_G(T)$. 
%Then, since $S^+ \subseteq T (S \cap M)^+$, we have
%	\begin{align*}
%	 \Siegel_Q 
%	&\subseteq C T (S \cap M)^+ (K \cap Q)
%	= C' T (S \cap M)^+ (K \cap M)
%	\\&= C' (S \cap M)^+ (K \cap M) \cdot T
%	= \Siegel_{MN} T 
%	. \end{align*}
%Since $(MN)_{\integer}$ has finite index in~$Q_{\integer}$, there is no harm in assuming $\gamma \in MN$. Then we must have
%	$$ \Siegel_{MN} \cap \gamma \Siegel_{MN} \neq \emptyset .$$
%Now, let $\overline{\phantom{x}}$ denote the natural homomorphism $MN \to M/N \iso M$. Then:
%	\begin{itemize}
%	\item By induction on $\rational$-rank (and ignoring the technicality that $M$ is reductive, rather than semisimple), we see that there are only finitely many possibilities for~$\overline{\gamma}$. 
%	\item For each choice of~$\overline{\gamma}$, there are only finitely many possibilities for~$\gamma$, because $\Siegel_{MN} \cap N$ is compact, and $\Gamma$ is discrete.
%	\end{itemize}
\end{proof}


\begin{rem} \label{CuspGroup}
When $\Qrank \Gamma = 1$, the first part of the proof establishes the useful fact that there is a compact subset~$C_0$ of~$G$, such that if $\gamma \in \Gamma$, and $\Siegel \cap \gamma \Siegel \not\subseteq C_0$, then $\gamma \in P$.
\end{rem}

\begin{exercises}

\item \label{CanChooseNormalSiegel}
Show that every $\rational$-split torus of~$G$ is invariant under some Cartan involution of~$G$. (Therefore, for any maximal $\rational$-split torus~$S$, there exists a maximal compact subgroup~$K$, such that the resulting Siegel sets $C S^+ K$ are normal.)
\hint{If $\tau$ is any Cartan involution, then there is a maximal $\real$-split torus~$A$, such that $\tau(a) = a^{-1}$ for all $a \in A$. Any $\real$-split torus is contained in some conjugate of~$A$.}

\item \label{SiegelvInN}
In the proof of \cref{GZFSiegel=G}, explain why it may be assumed that $v \in N$.
\hint{Being unipotent, $v$ is contained in the unipotent radical of some minimal parabolic $\rational$-subgroup \fullcsee{parab/Q}{U}. Since $\Gamma \, P_{\rational} = G_{\rational}$, we know that all minimal parabolic $\rational$-subgroups are conjugate under~$\Gamma$.}
%Let $P_0$ be a minimal parabolic $\rational$-subgroup of~$G$, such that $v \in \unip P_0$ \fullsee{parab/Q}{U}. From \pref{AssumeGzPQ=GQ} and the fact that all minimal parabolic $\rational$-subgroups of~$G$ are conjugate under~$G_\rational$ \fullcsee{parab/Q}{conj}, we know that all minimal parabolic $\rational$-subgroups of~$G$ are conjugate under~$\Gamma$. Hence, there is some $\gamma \in \Gamma$ with $P^\gamma = P_0$. Then, by replacing $g$ with $\gamma g$ and $v$ with $v^{\gamma^{-1}}$, we may assume $P_0 = P$. 

\item \label{SiegelgInS}
In the proof of \cref{GZFSiegel=G}, complete the proof without assuming that $g \in S$.
\hint{Write $g = pk \in PK$. If $C$ is large enough that $MN \subseteq (\Gamma \cap P) C$, then we have $g \in \Gamma cs k \subseteq \Gamma CSK$, so $v^s \approx e$.}

\item \label{MNZinPZ}
Show that if $P = MAN$ is a Langlands decomposition of a parabolic $\rational$-subgroup of~$G$, then $(MN)_{\integer}$ contains a finite-index subgroup of~$P_{\integer}$.
\hint{A $\rational$-split torus can have only finitely many integer points.}

\item \label{S+orS-}
Show that if $\Qrank \Gamma = 1$, then $S = S^+ \cup (S^+)^{-1}$.
\hint{If $s \in S^+$, then $s^t \in S^+$ for all $t \in \rational^+$ (and, hence, for all $t \in \real^+$).}

\item \label{GZFSiegel=GBigFEx}
Prove \cref{GZFSiegel=G} in the case where $\Qrank \Gamma = 1$. 
\hint{Replace the imprecise arguments of the text with rigorous statements, and do not assume $F$ is a singleton. (In order to assume $v \in N$, multiply $g$ on the left by an element~$x$ of $F\Gamma$. Then $x \Gamma x^{-1}$ contains a finite-index subgroup of $\Gamma \cap P$.)}

\item \label{OnlyUnipPShrinks}
Let $\Siegel = C S^+ K$ be a Siegel set, and let $P$ be the minimal parabolic $\rational$-subgroup corresponding to~$S^+$. Show there is a neighborhood~$W$ of~$e$ in~$G$, such that if $\gamma \in G_{\integer}$ and $\gamma^\sigma \in W$, for some $\sigma \in \Siegel$, then $\gamma \in \unip P$.

\item \label{NormalSiegel->KcapP}
Suppose 
	\begin{itemize}
%	\item $P$ is a minimal parabolic $\rational$-subgroup with Langlands decomposition $P = MSN$,
	\item $\tau$ is the Cartan involution corresponding to the maximal compact subgroup~$K$, 
	and
	\item $S$ is a $\tau$-invariant.
	\end{itemize}
Show $K \cap P \subseteq M \subseteq \czer_G(S)$.
\hint{Since $\czer_G(S)$ is $\tau$-invariant, the restriction of~$\tau$ to the semisimple part of~$M$ is a Cartan involution. 
Therefore $K \cap M$ contains a maximal compact subgroup of~$M$, which is a maximal compact subgroup of~$P$. The second inclusion is immediate from the definition of the Langlands decomposition.}

\item  \label{SiegelInPEx}
Establish \pref{SiegelInP}.
\hint{$\Siegel \cap P \cap \gamma(\Siegel \cap P) \neq \emptyset$ (since $\gamma \in P$)
and $\Siegel \cap P = C S^+ (K \cap P) = C (K \cap P) S^+$ \csee{NormalSiegel->KcapP}.}

\item \label{SiegelPropertyFullPf}
Give a complete proof of \cref{SiegelProperty}.

\item Show that $\Gamma$ has only finitely many conjugacy classes of
maximal unipotent subgroups.
\hint{You may assume, for simplicity, that $\Gamma = G_{\integer}$ is arithmetic. 
Use \fullCref{parab/Q}{U} and \cref{GammaG/P}.}
%Suppose $U$ is a maximal unipotent subgroup
%of~$G_{\integer}$. Then the Zariski closure~$\Zar{U}$
%of~$U$ is a unipotent $\rational$-subgroup of~$G$. Hence,
%there is a parabolic $\rational$-subgroup $P = MAN$, such
%that $U \subseteq N$. Then $U \subseteq \Zar{U}_{\integer}
%\subseteq N_{\integer}$, so the maximality of~$U$ implies that
%$U = N_{\integer}$. 
%
%The converse is similar (and uses \cref{U/UZcpct}).


\end{exercises}





\begin{notes}

The main results of this chapter were obtained for many classical groups by L.\,Siegel (see, for example, \cite{Siegel-DiscGrps}), and the general results are due to A.\,Borel and Harish-Chandra \cite{BorelHarishChandra-ArithSubgrps}.

The book of A.\,Borel \cite{Borel-IntroArithGrps} is the standard reference for this material; see \cite[Thm.~13.1, p.~90]{Borel-IntroArithGrps} for the construction of a fundamental domain for $G_\integer$ as a union of Siegel sets. The proof there does not assume $G_\integer$ is a lattice, so this provides a proof of the fundamental fact that every arithmetic 
	\thmindex{arithmetic subgroups are lattices}
subgroup of~$G$ is a lattice \csee{arith->latt}. See \cite[\S4.6]{PlatonovRapinchukBook} for an exposition of Borel and Harish-Chandra's original proof of this fact (using the Siegel set $\Siegel _{c_1,c_2,c_3}$ for $\SL(n,\integer)$ from \cref{SiegelSLnZDefn}).

\Cref{AlmIsomOnSiegel} was conjectured by C.\,L.\,Siegel in 1959, and was proved by L.\,Ji \cite[Thm.~7.6]{Ji-MetricCompact}. Another proof is in E.\,Leuzinger \cite[Thm.~B]{Leuzinger-TitsGeometry}.

The construction of coarse fundamental domains for non-arithmetic lattices in groups of real rank one is due to Garland and Raghunathan \cite{GarlandRaghunathan-Rrank1}.
(An exposition appears in \cite[Chap.~13]{RaghunathanBook}.) Combining this result with \cref{ReductThyArithGrps} yields \cref{ReductThyNonarith}.

\Cref{FinManyFinSubgrps} can be found in \cite[Thm.~4.3, p.~203]{PlatonovRapinchukBook}.

%\Cref{HattoriThm,HattoriTits} are due to T.\,Hattori \cite{Hattori-AsympGeom} (and they are also proved in \cite{Leuzinger-TitsGeometry}).

Regarding \cref{SiegelMReductive}, see \cite[Prop.~8.5, p.~55]{Borel-IntroArithGrps} or \cite[Thm.~4.11, p.~208]{PlatonovRapinchukBook} for a proof that if $T$ is a $\rational$-torus that has no $\rational$-split subtori, then $T/T_\integer$ is compact.
\end{notes}


\begin{references}{9}

\bibitem{Borel-IntroArithGrps}
A.\,Borel:
\emph{Introduction aux Groupes Arithm\'etiques}.
%Publications de l'Institut de MathŽmatique de l'UniversitŽ de Strasbourg, XV. ActualitŽs Scientifiques et Industrielles, No. 1341 
Hermann, Paris, 1969.
\MR{0244260}

\bibitem{BorelHarishChandra-ArithSubgrps}
A.\,Borel and Harish-Chandra:
 Arithmetic subgroups of algebraic groups,
 \emph{Ann. Math.} (2) 75 (1962) 485--535.
\MR{0147566},
\maynewline
\url{http://www.jstor.org/stable/1970210}

%\bibitem{ChatterjeeMorris}
%P.\,Chatterjee and D.\,W.\,Morris:
%Divergent torus orbits in homogeneous spaces of $\rational$-rank two,
%\emph{Israel J. Math.} 152 (2006), 229--243. 
%\MR{2214462}

\bibitem{GarlandRaghunathan-Rrank1}
H.\,Garland and M.\,S.\,Raghunathan:
Fundamental domains for lattices in ($\real$-)rank 1 semisimple Lie groups
\emph{Ann. Math.} (2) 92 (1970) 279--326.
\MR{0267041},
\maynewline
\url{http://www.jstor.org/stable/1970838}

%\bibitem{Hattori-AsympGeom}
%T.\,Hattori:
%Asymptotic geometry of arithmetic quotients of symmetric spaces,
%\emph{Math. Z.} 222 (1996), no.~2, 247--277. 
%\MR{1429337}

%\bibitem{HelgasonBook} 
%S.\,Helgason:
%\emph{Differential Geometry, Lie Groups, and Symmetric Spaces.}
%% Pure and Applied Mathematics, 80. 
%Academic Press, New York, 1978.
%ISBN 0-12-338460-5,
%\MR{0514561}

%\bibitem{Humphreys-ArithGrps}
% J.\,E.\,Humphreys:
% \emph{Arithmetic Groups.}
% %Lecture Notes in Mathematics \#789. 
% Springer, Berlin, 1980. 
% ISBN 3-540-09972-7,
% \MR{0584623}
 

%\bibitem{Iwasawa-SomeTypes}
%K.\,Iwasawa:
%On some types of topological groups,
%\emph{Ann. of Math.} (2) 50 (1949) 507--558. 
%\MR{0029911}

\bibitem{Ji-MetricCompact}
L.\,Ji:
Metric compactifications of locally symmetric spaces,
\emph{Internat. J. Math.} 9 (1998), no.~4, 465--491. 
\MR{1635185},
\maynewline
\url{http://dx.doi.org/10.1142/S0129167X98000208}

\bibitem{Leuzinger-TitsGeometry}
E.\,Leuzinger:
Tits geometry, arithmetic groups, and the proof of a conjecture of Siegel,
\emph{J. Lie Theory} 14 (2004), no.~2, 317--338. 
\MR{2066859},
\maynewline
\url{http://www.emis.de/journals/JLT/vol.14_no.2/6.html}

\bibitem{PlatonovRapinchukBook}
 V.\,Platonov and A.\,Rapinchuk: 
 \emph{Algebraic Groups and Number Theory.}
 Academic Press, Boston, 1994.
 ISBN 0-12-558180-7,
 \MR{1278263}

\bibitem{RaghunathanBook}
 M.\,S.\,Raghunathan: 
 \emph{Discrete Subgroups of Lie Groups.}
 Springer, {New York}, 1972.
 ISBN 0-387-05749-8,
\MR{0507234}

\bibitem{Siegel-DiscGrps}
C.\,L.\,Siegel:
Discontinuous groups,
\emph{Ann. of Math.} (2) 44 (1943) 674--689. 
\MR{0009959},
\url{http://www.jstor.org/stable/1969104}

%\bibitem{Weiss-Qrank}
%B.\,Weiss:
%Divergent trajectories and $\rational$-rank,
%\emph{Israel J. Math.} 152 (2006), 221--227. 
%\MR{2214461}

\end{references}

