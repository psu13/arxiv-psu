%\chapter{Higher homotopy pushouts}

\section{$2$-spans and $2$-pushouts}

\section{Applications}

\begin{thm}
The join is an associative operation.
\end{thm}

\begin{thm}
the 3-by-3-lemma
\end{thm}

\begin{exercises}
\item In this exercise we study the \define{reflexive coequalizer}. Let $R:A\to A\to\UU$ be a relation, and $\rho:\prd{x:A}R(x,x)$ be a witness of reflexivity.  
\begin{subexenum}
\item Formulate the induction principle and computation rules for the higher inductive type $\mathsf{rcoeq}(A,R,\rho)$ with constructors
\begin{align*}
\pts{\eta} &: A \to \mathsf{rcoeq}(A,R,\rho) \\
\edg{\eta} &: \prd*{x,y:A} R(x,y)\to \id{\pts{\eta}(x)}{\pts{\eta}(y)} \\
\rfx{\eta} &: \prd{x:A} \edg{\eta}(\rho(x))=\refl{\pts{\eta}(x)}.
\end{align*}
\item Show that
\begin{equation*}
\begin{tikzcd}
\sm{x,y:A} R(x,y) \arrow[r,"\pi_2"] \arrow[d,swap,"\pi_1"] & A \arrow[d,"\pts{\eta}"] \\
A \arrow[r,swap,"\pts{\eta}"] & \mathsf{rcoeq}(A,R,\rho)
\end{tikzcd}
\end{equation*}
commutes, and is a pushout square.
\item Compute $\mathsf{rcoeq}(A,\idtypevar{A},\refl{})$ and $\mathsf{rcoeq}(A,(\lam{x}{y}\unit),(\lam{x}\ttt))$. Furthermore, consider a pointed type $\pairr{X,x_0}$ as a reflexive relation on the unit type, and compute $\mathsf{rcoeq}(\unit,X,x_0)$.
\end{subexenum}
\end{exercises}
