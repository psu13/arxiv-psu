\section{The image of a map and the replacement axiom}\label{chap:image}

The idea of the image of a map $f:A\to X$ is that it is, in a way, the least subtype of $X$ that contains all the values of $f$. More precisely, the image of $f$ is an embedding $i:\im(f)\hookrightarrow X$ that fits in a commuting triangle
\begin{equation*}
  \begin{tikzcd}[column sep=tiny]
    A \arrow[rr,"q"] \arrow[dr,swap,"f"] & & \im(f) \arrow[dl,hook,"i"] \\
    \phantom{\im(f)} & X
  \end{tikzcd}
\end{equation*}
and satisfies the \emph{universal property} of the image of $f$. The universal property of the image of $f$ asserts that if a subtype $B\hookrightarrow X$ contains all the values of $f$, then it contains the image of $f$.
%In other words, for  asserts that there is a unique map $h:\im(f)\to B$ for which the tetrahedron
%\begin{equation*}
%  \begin{tikzcd}[column sep=large]
%    A \arrow[rr] \arrow[dr,"q"] \arrow[dddr,swap,"f"] & & B \arrow[dddl,"m"] \\
%    & \im(f) \arrow[ur,densely dotted,"h"] \arrow[dd,"i"] \\ \\
%    & X
%  \end{tikzcd}
%\end{equation*}
%commutes.
The image of a map can be constructed using the propositional truncation operation. In fact, we can also go the other way around: The propositional truncation of a type $A$ is the image of the map $A\to\unit$.

The final topic of this section is the type theoretic replacement axiom. A specific instance of the replacement axiom asserts that the image of any map $f:A\to\UU$ is equivalent to a type in $\UU$, provided that $A$ is equivant to a type in $\UU$. This property will be used to construct quotients in type theory, much in the same way as quotients are constructed in set theory.

We should note that the existence of the propositional truncation operation and the replacement axiom will be assumed for now. However, once we assume that universes are closed under pushouts, we will be able to construct the propositional truncations and we will be able to prove the replacement axiom. These constructions will be given in \cref{sec:join-construction}.

\subsection{The image of a map}\label{sec:image-construction}
 %Note that there is quite a lot of information in this diagram: not only are there the three small commuting triangles; there is also the large commuting triange in the back, and there is a three-dimensional solid filling the space between the four triangles. We make the following definition, in order to express the universal property of the image efficiently.

\begin{defn}
  Let $f:A\to X$ and $g:B\to X$ be maps. A \define{morphism} from $f$ to $g$ over $X$ consists of a map $h:A\to B$ equipped with a homotopy $H:f\htpy g\circ h$ witnessing that the triangle
\begin{equation*}
\begin{tikzcd}[column sep=tiny]
A \arrow[rr,"h"] \arrow[dr,swap,"f"] & & B \arrow[dl,"g"] \\
& X
\end{tikzcd}
\end{equation*}
commutes. Thus, we define the type
\begin{equation*}
\mathrm{hom}_X(f,g)\defeq\sm{h:A\to B}f\htpy g\circ h.
\end{equation*}
Composition of morphisms over $X$ is defined by
\begin{equation*}
  (k,K)\circ (h,H) \defeq (k\circ h,\ct{H}{(K\cdot h)}).
\end{equation*}
\end{defn}

\begin{defn}
Consider a commuting triangle
\begin{equation*}
\begin{tikzcd}[column sep=tiny]
A \arrow[rr,"q"] \arrow[dr,swap,"f"] & & I \arrow[dl,"i"] \\
& X
\end{tikzcd}
\end{equation*}
with $H:f\htpy i\circ q$, where $i$ is an embedding\index{embedding}.
We say that $i$ has the \define{universal property of the image of $f$}\index{universal property!of the image} if the map
\begin{equation*}
\blank\circ(q,H) : \mathrm{hom}_X(i,m)\to\mathrm{hom}_X(f,m)
\end{equation*}
is an equivalence for every embedding $m:B\to X$. 
\end{defn}

\begin{rmk}
  Consider a commuting triangle
\begin{equation*}
\begin{tikzcd}[column sep=tiny]
A \arrow[rr,"q"] \arrow[dr,swap,"f"] & & I \arrow[dl,"i"] \\
& X
\end{tikzcd}
\end{equation*}
with $H:f\htpy i\circ q$, where $i$ is an embedding. Then it is not hard to see that the embedding $i$ satisfies the universal property of the image inclusion if and only if for every commuting triangle
\begin{equation*}
  \begin{tikzcd}[column sep=tiny]
    A \arrow[dr,swap,"f"] \arrow[rr,"g"] & & B \arrow[dl,"m"] \\
    & X
  \end{tikzcd}
\end{equation*}
with $G:f\htpy m\circ g$, where $m$ is an embedding, the type of quadruples $(h,K,L,M)$ consisting of
\begin{enumerate}
\item a map $h:I\to B$,
\item a homotopy $K:i\htpy m\circ h$ witnessing that the triangle
  \begin{equation*}
    \begin{tikzcd}[column sep=tiny]
      I \arrow[rr,"h"] \arrow[dr,swap,"i"] & & B \arrow[dl,"m"] \\
      & X
    \end{tikzcd}
  \end{equation*}
  commutes,
\item a homotopy $L:g\htpy h\circ q$ witnessing that the triangle
  \begin{equation*}
    \begin{tikzcd}[column sep=tiny]
      A \arrow[rr,"q"] \arrow[dr,swap,"g"] & & I \arrow[dl,"h"] \\
      & B
    \end{tikzcd}
  \end{equation*}
  commutes,
\item a homotopy $M:\ct{H}{(K\cdot q)}\htpy\ct{G}{(m\cdot L)}$ witnessing that the square
  \begin{equation*}
    \begin{tikzcd}
      f \arrow[d,swap,"H"] \arrow[r,"G"] & m\circ g \arrow[d,"m\cdot L"] \\
      i\circ q \arrow[r,swap,"K\cdot q"] & m\circ h\circ g
    \end{tikzcd}
  \end{equation*}
  commutes,
\end{enumerate}
is contractible. However, the situation is in fact much simpler, because the type $\mathrm{hom}_X(f,m)$ is a proposition whenever $m$ is an embedding.
\end{rmk}

\begin{rmk}
  Suppose that the map $f:A\to X$ has a section. Then the identity function
  \begin{equation*}
    \idfunc:X\to X
  \end{equation*}
  satisfies the universal property of the image of $f$. 
\end{rmk}

\begin{rmk}
  Suppose that $f:A\to X$ is already an embedding. Then $f$ itself satisfies the universal property of the image of $f$.
\end{rmk}

\begin{lem}
For any $f:A\to X$ and any embedding\index{embedding} $m:B\to X$, the type $\mathrm{hom}_X(f,m)$ is a proposition.
\end{lem}

\begin{proof}
  Recall from \cref{ex:triangle_fib} that the type $\mathrm{hom}_X(f,m)$ is equivalent to the type
  \begin{equation*}
    \prd{x:X}\fib{f}{x}\to\fib{m}{x}.
  \end{equation*}
  Therefore it suffices to show that this type is a proposition. Recall from \cref{cor:prop_emb} that a map is an embedding if and only if its fibers are propositions.
  Thus we see that the type $\prd{x:X}\fib{f}{x}\to\fib{m}{x}$ is a product of propositions, hence it is a proposition by \cref{thm:trunc_pi}.
\end{proof}

\begin{prp}\label{prp:simplifly-universal-property-image}
  Consider a commuting triangle
  \begin{equation*}
    \begin{tikzcd}[column sep=tiny]
      A \arrow[rr,"q"] \arrow[dr,swap,"f"] & & I \arrow[dl,"i"] \\
      & X
\end{tikzcd}
  \end{equation*}
  with $H:f\htpy i\circ q$, where $i$ is an embedding. Then the following are equivalent:
  \begin{enumerate}
  \item The embedding $i$ satisfies the universal property of the image inclusion of $f$.
  \item For every embedding $m:B\to X$ there is a map
    \begin{equation*}
      \mathrm{hom}_X(f,m)\to\mathrm{hom}_X(i,m).
    \end{equation*}
  \end{enumerate}
\end{prp}

\begin{proof}
Since $\mathrm{hom}_X(f,m)$ is a proposition for every every embedding $m:B\to X$, the claim follows immediately by \cref{ex:prop_equiv}.
\end{proof}

Just as in the cases for pullbacks and pushouts, the universal property of the image implies that the image is determined uniquely. We will show here that the type of image factorizations of any map is a proposition. In \cref{sec:image-construction} we will construct the image, after constructing the propositional truncation.

\begin{prp}
  Let $f$ be a map, and consider two commuting triangles
  \begin{equation*}
    \begin{tikzcd}[column sep=tiny]
      A \arrow[dr,swap,"f"] \arrow[rr,"q"] & & B \arrow[dl,"i"] &[2em] A \arrow[dr,swap,"f"] \arrow[rr,"{q'}"] & & B' \arrow[dl,"{i'}"] \\
      & X & & \phantom{B'} & X
    \end{tikzcd}
  \end{equation*}
  with $I:f\htpy i\circ q$ and $I':f\htpy i'\circ q'$, in which $i$ and $i'$ are assumed to be embeddings. Moreover, consider
  \begin{equation*}
    (h,H):\mathrm{hom}_X(i,i')
  \end{equation*}
  equipped with an identification $(h,H)\circ(q,I)=(q',I')$ in $\mathrm{hom}_X(f,i')$. Then, if any two of the following properties hold, so does the third:
  \begin{enumerate}
  \item The embedding $i$ satisfies the universal property of the image inclusion of $f$.
  \item The embedding $i'$ satisfies the universal property of the image inclusion of $f$.
  \item The map $h$ is an equivalence.
  \end{enumerate}
\end{prp}

\begin{proof}
  Consider an embedding $m:C\to X$. Then we have a commuting triangle
  \begin{equation*}
    \begin{tikzcd}[column sep=-1em]
      \mathrm{hom}_X(i',m) \arrow[rr,"{\blank\circ(h,H)}"] \arrow[dr,swap,"{\blank\circ(q',I')}"] & & \mathrm{hom}_X(i,m) \arrow[dl,"{\blank\circ(q,I)}"] \\
      & \mathrm{hom}_X(f,m), & \phantom{\mathrm{hom}_X(i',m)}
    \end{tikzcd}
  \end{equation*}
  so it follows that if any two of these maps are equivalences, then so is the third. The claim now follows by the observation that $\blank\circ(h,H)$ is an equivalence for every embedding $m:C\to X$ if and only if $h$ is an equivalence.
\end{proof}

\begin{cor}\label{cor:uniqueness-image}
  Consider two image factorizations
  \begin{equation*}
    \begin{tikzcd}[column sep=tiny]
      A \arrow[dr,swap,"f"] \arrow[rr,"q"] & & B \arrow[dl,"i"] &[2em] A \arrow[dr,swap,"f"] \arrow[rr,"{q'}"] & & B' \arrow[dl,"{i'}"] \\
      & X & & \phantom{B'} & X
    \end{tikzcd}
  \end{equation*}
  of a map $f$, with $I:f\htpy i\circ q$ and $I':f\htpy i'\circ q'$. Then the type of $(e,H):\mathrm{hom}_X(i,i')$ in which $e$ is an equivalence, equipped with an identification
  \begin{equation*}
    (e,H)\circ(q,I)=(q',I')
  \end{equation*}
  in $\mathrm{hom}_X(f,i')$, is contractible.
\end{cor}

The image of a map $f:A\to X$ can now be defined using the propositional truncation:

\begin{defn}
For any map $f:A\to X$ we define the \define{image}\index{image} of $f$ to be the type
\begin{equation*}
\im(f) \defeq \sm{x:X}\brck{\fib{f}{x}}.
\end{equation*}
Furthermore, we define:
\begin{enumerate}
\item The \define{image inclusion}
  \begin{equation*}
    i_f:\im(f)\to X
  \end{equation*}
  to be the projection $\proj 1$.
\item The map
  \begin{equation*}
    q_f:A\to\im(f)
  \end{equation*}
  to be the map given by $q_f(x)\defeq(f(x),\eta(x,\refl{f(x)}))$.
\item The homotopy $I_f:f\htpy i_f\circ q_f$ witnessing that the triangle
  \begin{equation*}
    \begin{tikzcd}[column sep=tiny]
      A \arrow[rr,"q_f"] \arrow[dr,swap,"f"] & & \im(f) \arrow[dl,"i_f"] \\
      \phantom{\im(f)} & X
    \end{tikzcd}
  \end{equation*}
  commutes, to be given by $I_f(x)\defeq\refl{f(x)}$.
\end{enumerate}
\end{defn}

\begin{prp}
  The image inclusion $i_f:\im(f)\to X$ of any map $f:A\to X$ is an embedding.
\end{prp}

\begin{proof}
  The fiber of $i_f$ at $x:X$ is equivalent to the type $\brck{\fib{f}{x}}$. In particular we see that the fibers are propositions, so $i_f$ is an embedding.
\end{proof}

\begin{thm}
  The image inclusion $i_f:\im(f)\to X$ of any map $f:A\to X$ satisfies the universal property of the image inclusion of $f$.
\end{thm}

\begin{proof}
  Consider an embedding $m:B\to X$. Note that we have a commuting square
  \begin{equation*}
    \begin{tikzcd}[column sep=6em]
      \mathrm{hom}_X(i_f,m) \arrow[d] \arrow[r] & \mathrm{hom}_X(f,m) \arrow[d] \\
      \Big(\prd{x:X}\fib{i_f}{x}\to\fib{m}{x}\Big) \arrow[r,swap,"h\mapsto{\lam{x}h_x\circ\varphi_x}"] & \Big(\prd{x:X}\fib{f}{x}\to\fib{m}{x}\Big)
    \end{tikzcd}
  \end{equation*}
  The vertical maps are of the form
  \begin{equation*}
    (h,H) \mapsto \lam{x}{(y,p)}(h(y),\ct{H(y)^{-1}}{p}),
  \end{equation*}
  and they are both equivalences. The map
  \begin{equation*}
    \varphi_x:\fib{f}{x}\to\fib{i_f}{x}
  \end{equation*}
  given by $\varphi_x(a,p)\defeq((h(a),\eta(a,p)),p)$ is a propositional truncation for every $x:X$. Therefore it follows that the map
  \begin{equation*}
    (\fib{i_f}{x}\to\fib{m}{x})\to(\fib{f}{x}\to\fib{m}{x})
  \end{equation*}
  is an equivalence, for every $x:X$. Thus we conclude that the bottom map in the above square is an equivalence, which implies that the top map is an equivalence. 
\end{proof}

\begin{eg}
  An important special case of the homotopy image of a map is the image of the terminal projection
\begin{equation*}
  \const_\ttt : A \to \unit,
\end{equation*}
which results in an embedding $I\hookrightarrow \unit$. Embeddings into the unit type are in fact just propositions. To see this, note that
\begin{align*}
\sm{A:\UU}{f:A\to\unit}\isemb(f)
& \eqvsym \sm{A:\UU}\isemb(\const_\ttt) \\
& \eqvsym \sm{A:\UU}\prd{x:\unit}\isprop(\fib{\const_\ttt}{x}) \\
& \eqvsym \sm{A:\UU}\isprop(\fib{\const_\ttt}{\ttt}) \\
& \eqvsym \sm{A:\UU}\isprop(A).
\end{align*}
Therefore, the universal property of the image of the map $A\to\unit$ is equivalently described as a proposition $P$ satisfying the universal property of the propositional truncation.
\end{eg}

\subsection{Surjective maps}

Another application of the propositional truncation is the notion of surjective map.

\begin{defn}
A map $f:A\to B$ is said to be \define{surjective} if there is a term of type
\begin{equation*}
\issurj(f)\defeq \prd{y:B}\brck{\fib{f}{b}}.
\end{equation*}
\end{defn}

\begin{eg}
Any equivalence is a surjective map, and so is any map that has a section (those are sometimes called \define{split epimorphisms}). Other examples include the base point inclusion $\unit\to\sphere{n}$ for any $n\geq 1$. 
\end{eg}

\begin{prp}\label{prp:surjective}
  Consider a map $f:A\to B$. Then the following are equivalent:
  \begin{enumerate}
  \item The map $f:A\to B$ is surjective.
  \item For any family $P$ of propositions over $B$, the precomposition map
    \begin{equation*}
      \blank\circ f : \Big(\prd{y:B}P(y)\Big)\to\Big(\prd{x:A}P(f(x))\Big)
    \end{equation*}
    is an equivalence.
  \end{enumerate}
\end{prp}

\begin{proof}
  Suppose first that $f$ is surjective, and consider the commuting square
  \begin{equation*}
    \begin{tikzcd}[column sep=6em]
      \Big(\prd{y:B}P(y)\Big) \arrow[r,"\blank\circ f"] \arrow[d,swap,"h\mapsto\lam{y}\const_{h(y)}"] & \Big(\prd{x:A}P(f(x))\Big)  \\
      \Big(\prd{y:B}\brck{\fib{f}{y}}\to P(y)\Big) \arrow[r,swap,"h\mapsto\lam{y}h(y)\circ\eta"] & \Big(\prd{y:B}\fib{f}{y}\to P(y)\Big) \arrow[u,swap,"{h\mapsto\lam{x}h(f(x),(x,\refl{f(x)}))}"]
    \end{tikzcd}
  \end{equation*}
  In this square, the bottom map is an equivalence by the universal property of the propositional truncation of $\fib{f}{y}$. The map on the right is also easily seen to be an equivalence. Furthermore, the map on the left is an equivalence by the assumption that $f$ is surjective, from which it follows that the types $\brck{\fib{f}{y}}$ are contractible. Therefore it follows that the top map is an equivalence, which completes the proof that (i) implies (ii).

  For the converse, it follows immediately from the assumption (ii) that
  \begin{equation*}
    \blank\circ f : \Big(\prd{y:B}\brck{\fib{f}{y}}\Big)\to\Big(\prd{x:A}\brck{\fib{f}{f(x)}}\Big)
  \end{equation*}
  is an equivalence. Hence it suffices to construct a term of type $\brck{\fib{f}{f(x)}}$ for each $x:A$. This is easy, because we have
  \begin{equation*}
    \eta(x,\refl{f(x)}):\brck{\fib{f}{f(x)}}.\qedhere.
  \end{equation*}
\end{proof}

\begin{thm}\label{thm:surjective}
Consider a commuting triangle
\begin{equation*}
\begin{tikzcd}[column sep=tiny]
A \arrow[rr,"q"] \arrow[dr,swap,"f"] & & B \arrow[dl,"m"] \\
& X
\end{tikzcd}
\end{equation*}
in which $m$ is an embedding. Then the following are equivalent:
\begin{enumerate}
\item The embedding $m$ satisfies the universal property of the image inclusion of $f$.
\item The map $q$ is surjective.
\end{enumerate}
\end{thm}

\begin{proof}
  First assume that $m$ satisfies the universal property of the image inclusion of $f$, and consider the composite function
  \begin{equation*}
    \begin{tikzcd}
      \Big(\sm{y:B}\brck{\fib{q}{y}}\Big) \arrow[r,"\proj 1"] & B \arrow[r,"m"] & X.
    \end{tikzcd}
  \end{equation*}
  Note that $m\circ\proj 1$ is a composition of embeddings, so it is an embedding. By the universal property of $m$ there is a unique map $h$ for which the triangle
  \begin{equation*}
    \begin{tikzcd}[column sep=0]
      B \arrow[dr,swap,"m"] \arrow[rr,densely dotted,"h"] & & \sm{y:B}\brck{\fib{q}{y}} \arrow[dl,"m\circ\proj 1"] \\
      \phantom{\sm{y:B}\brck{\fib{q}{y}}} & X
    \end{tikzcd}
  \end{equation*}
  commutes. Now note that $\proj 1\circ h$ is a map such that $m\circ (\proj 1\circ h)\htpy m$. The identity function is another map for which we have $m\circ\idfunc\htpy m$, so it follows by uniqueness that $\proj 1\circ h\htpy \idfunc$. In other words, the map $h$ is a section of the projection map. Therefore we obtain by \cref{ex:pi_sec} a dependent function
  \begin{equation*}
    \prd{b:B}\brck{\fib{q}{b}},
  \end{equation*}
  showing that $q$ is surjective.

  For the converse, suppose that $q$ is surjective. To prove that $m$ satisfies the universal property of the image factorization of $f$, it suffices to construct an equivalence
  \begin{equation*}
    \mathrm{hom}_X(f,m')\to\mathrm{hom}_X(m,m'),
  \end{equation*}
  for any embedding $m':B'\to X$. To see that there is such an equivalence, we make the following calculation
  \begin{align*}
    \mathrm{hom}_X(m,m') & \simeq \prd{x:X}\fib{m}{x}\to\fib{m'}{x} \\
                         & \simeq \prd{b:B}\fib{m'}{m(b)} \\
                         & \simeq \prd{a:A}\fib{m'}{m(q(a))} \\
                         & \simeq \prd{a:A}\fib{m'}{f(a)} \\
                         & \simeq \prd{x:X}\fib{f}{x}\to\fib{m'}{x} \\
                         & \simeq \mathrm{hom}_X(f,m').
  \end{align*}
  In this calculation, the first and last equivalence hold by \cref{ex:triangle_fib}. The second and second to last equivalences hold by \cref{ex:pi-fib}. The third equivalence holds by \cref{prp:surjective}, since $q$ is assumed to be surjective, and the fourth equivalence holds since we have a homotopy $f\htpy m\circ f$.
\end{proof}

\begin{cor}
  Every map factors uniquely as a surjective map followed by an embedding.
\end{cor}

\begin{proof}
  Consider a map $f:A\to X$, and two factorizations
  \begin{equation*}
    \begin{tikzcd}[column sep=tiny]
      A \arrow[rr,"q"] \arrow[dr,swap,"f"] & & B \arrow[dl,"i"] &[3em] A \arrow[rr,"{q'}"] \arrow[dr,swap,"f"] & & B' \arrow[dl,"{i'}"] \\
      & X & & & X
    \end{tikzcd}
  \end{equation*}
  of $f$ where $m$ and $m'$ are embeddings, and $q$ and $q'$ are surjective. Then both $m$ and $m'$ satisfy the universal property of the image factorization of $f$ by \cref{thm:surjective}. Now it follows by \cref{cor:uniqueness-image} that the type of $(e,H):\mathrm{hom}_X(i,i')$ in which $e$ is an equivalence, equipped with an identification
  \begin{equation*}
    (e,H)\circ(q,I)=(q',I')
  \end{equation*}
  in $\mathrm{hom}_X(f,i')$, is contractible.
\end{proof}

\subsection{Type theoretic replacement}

\begin{comment}
We have constructed the set quotient $A/R$ as the image of the equivalence relation
\begin{equation*}
  R:A\to \UU^A.
\end{equation*}
However, the type $\UU^A$ is itself in the next universe $\UU^+$. Hence the quotient is also in the universe $\UU^+$. We prove in this section that $A/R$ is nevertheless equivalent to a type in $\UU$. In other words, we show that $A/R$ is \emph{essentially} small.
\end{comment}

\begin{defn}\label{defn:ess_small}
\begin{enumerate}
\item A type $A$ is said to be \define{essentially small}\index{essentially small!type} if there is a type $X:\UU$ and an equivalence $\eqv{A}{X}$. We write\index{ess_small(A)@{$\mathsf{ess\usc{}small}(A)$}}
\begin{equation*}
\mathsf{ess\usc{}small}(A)\defeq\sm{X:\UU}\eqv{A}{X}.
\end{equation*}
\item A map $f:A\to B$ is said to be \define{essentially small}\index{essentially small!map} if for each $b:B$ the fiber $\fib{f}{b}$ is essentially small.
We write\index{ess_small(f)@{$\mathsf{ess\usc{}small}(f)$}}
\begin{equation*}
\mathsf{ess\usc{}small}(f)\defeq\prd{b:B}\mathsf{ess\usc{}small}(\fib{f}{b}).
\end{equation*}
\item A type $A$ is said to be \define{locally small}\index{locally small!type} if for every $x,y:A$ the identity type $x=y$ is essentially small.
We write\index{loc_small(A)@{$\mathsf{loc\usc{}small}(A)$}}
\begin{equation*}
\mathsf{loc\usc{}small}(A)\defeq \prd{x,y:A}\mathsf{ess\usc{}small}(x=y).
\end{equation*}
\end{enumerate}
\end{defn}

\begin{eg}
  \begin{enumerate}
  \item Any essentially $\UU$-small type is also locally $\UU$-small.
  \item Any univalent universe $\UU$ is locally $\UU$-small, because by the univalence axiom we have equivalences
    \begin{equation*}
      (A=B)\simeq (A\simeq B)
    \end{equation*}
    for each $A,B:\UU$, and the type $A\simeq B$ is in $\UU$.
  \item Any proposition is locally small with respect to any universe $\UU$.
  \item For any family $P$ of locally $\UU$-small types over a essentially $\UU$-small type $A$, the dependent product $\prd{x:A}P(x)$ is locally $\UU$-small. In particular, any type $A\to B$ of functions from an essentially small type into a locally small type is again locally small.
  \end{enumerate}
\end{eg}

\begin{lem}\label{lem:isprop_ess_small}
The type $\mathsf{ess\usc{}small}(A)$ is a proposition for any type $A$.\index{essentially small!is a proposition}
\end{lem}

\begin{proof}
Let $A$ be a type, not necessarily in $\UU$. In order to show that $\mathsf{ess\usc{}small}(A)$ is a proposition, we will use \cref{lem:isprop_eq} and show that for any $X:\UU$ and any equivalence $e:A\simeq X$, the type
\begin{equation*}
\sm{Y:\UU}\eqv{A}{Y}
\end{equation*}
is contractible. Note that we have an equivalence
\begin{equation*}
\eqv{\Big(\sm{Y:\UU}\eqv{X}{Y}\Big)}{\Big(\sm{Y:\UU}\eqv{A}{Y}\Big)}
\end{equation*}
because precomposing with the equivalence $e:A \simeq X$ is an equivalence. However, the type $\sm{Y:\UU}\eqv{X}{Y}$ is contractible by \cref{thm:univalence}. This shows that $\mathsf{ess\usc{}small}(A)$ is equivalent to a contractible type, assuming that $A$ is essentially small.
\end{proof}

\begin{cor}
For each function $f:A\to B$, the type $\mathsf{ess\usc{}small}(f)$ is a proposition, and for each type $X$ the type $\mathsf{loc\usc{}small}(X)$ is a proposition.
\end{cor}

\begin{proof}
This follows from the fact that propositions are closed under dependent products, established in \cref{thm:trunc_pi}.
\end{proof}

Recall that in set theory, the replacement axiom asserts that for any family of sets $\{X_i\}_{i\in I}$ indexed by a set $I$, there is a set $X[I]$ consisting of precisely those sets $x$ for which there exists an $i\in I$ such that $x\in X_i$. In other words: the image of a set-indexed family of sets is again a set. Without the replacement axiom, $X[I]$ would be a class. In the following corollary we establish a type-theoretic analogue of the replacement axiom: the image of a family of small types indexed by a small type is again (essentially) small.

\begin{axiom}\label{axiom:replacement}
  For any map $f:A\to B$ from an essentially small type $A$ into a locally small type $B$, the image of $f$ is again essentially small.
\end{axiom}

\begin{eg}
  For any type $A:\UU$, the image of the constant map $\const_A:\unit\to \UU$ is essentially small. This image is called the \define{connected component} of the universe at $A$. To see why, let us calculate
  \begin{align*} 
    \im(\const_A) & \jdeq \sm{X:\UU}\Brck{\fib{\const_A}{X}} \\
                  & \jdeq \sm{X:\UU}\Brck{\sm{t:\unit}A=X} \\
                  & \simeq \sm{X:\UU}\brck{A=X}.
  \end{align*}
  We see that the image of $\const_A:\unit\to\UU$ is the type of all types that are \emph{merely} equal to $A$. In other words, they are equal to $A$ in an unspecified way.
\end{eg}

\begin{eg}
  The type $\F$ of all finite types is defined to be the image of the map
  \begin{equation*}
    \Fin : \N\to\UU_0
  \end{equation*}
  By the replacement axiom, this type is essentially small. 
\end{eg}

\begin{exercises}
  \exercise Consider a map $f:A\to P$ into a proposition $P$. Show that the following are equivalent:
  \begin{enumerate}
  \item The map $f$ is a propositional truncation of $A$.
  \item The map $f$ is surjective.
  \end{enumerate}
  \exercise Consider a map $f:A\to B$. Show that the following are equivalent:
  \begin{enumerate}
  \item $f$ is an equivalence.
  \item $f$ is both surjective and an embedding.
  \end{enumerate}
  \exercise Consider a commuting triangle
  \begin{equation*}
    \begin{tikzcd}[column sep=tiny]
      A \arrow[rr,"h"] \arrow[dr,swap,"f"] & & B \arrow[dl,"g"] \\
      & X
    \end{tikzcd}
  \end{equation*}
  with $H:f\htpy g\circ h$, and assume that $h$ is surjective. Show that the following are equivalent:
  \begin{enumerate}
  \item The map $f$ is surjective.
  \item The map $g$ is surjective.
  \end{enumerate}
  \exercise \label{ex:surjective-precomp}Consider a map $f:A\to B$. Show that the following are equivalent:
  \begin{enumerate}
  \item The map $f$ is surjective.
  \item For every set $C$, the precomposition function
    \begin{equation*}
      \blank\circ f:(B\to C)\to (A\to C)
    \end{equation*}
    is an embedding.
  \end{enumerate}
  Hint: To show that (ii) implies (i), use the assumption with the set $C\jdeq\prop_\UU$, where $\UU$ is a univalent universe containing both $A$ and $B$.
  \exercise Let us say that a type family $B$ over $A$ is \define{univalent} if the map
  \begin{equation*}
    (x=y)\to (B(x)\simeq B(y))
  \end{equation*}
  is an equivalence, for every $x,y:A$.
  \begin{subexenum}
  \item Show that a family $B:A\to\UU$ is univalent if and only if the map $B:A\to\UU$ is an embedding.
  \item For any family $B:A\to\UU$, show that the type family $\hat{B}:\hat{A}\to\UU$ defined by
    \begin{align*}
      \hat{A} & \defeq \im(B) \\
      \hat{B}(X,p) & \defeq X
    \end{align*}
    is univalent.
  \item For any two families $B:A\to\UU$ and $D:C\to\mathcal{V}$, define the type of \define{cartesian morphisms}
    \begin{equation*}
      \carthomFam((A,B),(C,D)) \defeq \sm{f:A\to C}\prd{x:A}B(x)\simeq D(f(x)).
    \end{equation*}
    Construct a cartesian morphism
    \begin{equation*}
      (\eta,\alpha) : \carthomFam((A,B),(\hat{A},\hat{B})).
    \end{equation*}
  \item Show that for any family $B:A\to\UU$ and any \emph{univalent} family $D:C\to\mathcal{V}$, the map
    \begin{equation*}
      \carthomFam((\hat{A},\hat{B}),(C,D))\to\carthomFam((A,B),(C,D))
    \end{equation*}
    given by
    \begin{equation*}
      (f,e)\mapsto (f\circ\eta,\lam{x}e_x\circ \alpha_x)
    \end{equation*}
    is an equivalence. This is the \define{universal property} of the univalent completion of $A$.
  \end{subexenum}
  %\exercise \label{also}(Mart\'in Escard\'o) For any two propositions $P$ and $Q$, define
  %\begin{equation*}
  %P\boxplus Q \defeq ((P\to Q)\to Q)\times ((Q\to P)\to P).
  %\end{equation*}
  %\begin{subexenum}
  %\item Show that $P\lor Q\to P\boxplus Q$ and $P\boxplus Q\to\neg(\neg P\land \neg Q)$.
  %\end{subexenum}
  %\item \label{ex:brck_comp} Formulate the computation rule corresponding to the path constructor $\mu$. That is, compute the type of $\apd{\rec{\brck{\blank}}(f,g)}{\mu(x,y)}$, and find a canonical element in it.
  %\exercise Let $f:A\to X$ be a map. Construct an equivalence
  %\begin{equation*}
  %\eqv{\Big(\sm{y:\mathsf{join\usc{}power}_X(n,A)}f(x)=f^{\ast n}(y)\Big)}{\Big(\sm{y:A}f(x)=f(y)\Big)^{\ast n}}
  %\end{equation*}
  %for any $x:A$.
\end{exercises}

\endinput

\begin{thm}
Consider a commuting triangle
\begin{equation*}
\begin{tikzcd}[column sep=small]
A \arrow[rr,"i"] \arrow[dr,swap,"f"] & & B \arrow[dl,"m"] \\
& X
\end{tikzcd}
\end{equation*}
with $I:f\htpy m\circ i$, where $m$ is an embedding. The following are equivalent:
\begin{enumerate}
\item $m$ satisfies the universal property of the image of $f$.
\item for each $x:X$, the proposition $\fib{m}{x}$ satisfies the universal property of the propositional truncation of $\fib{f}{x}$.
\end{enumerate}
\end{thm}
