% !TEX root = hott_intro.tex

\section{Propositions, sets, and the higher truncation levels}
\sectionmark{Truncation levels}\label{chap:hierarchy}

\index{truncated type|(}
\index{truncation level|(}
%Not all types have interesting higher groupoid structure. For example, we will see below that two natural numbers can only be equal in at most one way. Voevodsky articulated a useful notion to detect the homotopical complexity of types, which allows us to distinguish between contractible types (also called \emph{$(-2)$-types}), \emph{propositions} (also called \emph{$(-1)$-types}), \emph{sets} (\emph{$0$-types}), and \emph{$k$-types} for higher $k$.

%We will see [later] that there are types that are not $k$-types for any $k$.

\subsection{Propositions and subtypes}

\index{proposition|(}
\begin{defn}
A type $A$ is said to be a \define{proposition} if there is a term of type\index{is-prop(A)@{$\isprop(A)$}}
\begin{equation*}
\isprop(A)\defeq\prd{x,y:A}\iscontr(x=y).
\end{equation*}
Given a universe $\UU$, we define $\prop_\UU$\index{Prop@{$\prop$}} to be the type of all small propositions, i.e.,
\begin{equation*}
  \prop_\UU\defeq\sm{X:\UU}\isprop(X).
\end{equation*}
\end{defn}

\begin{eg}\label{eg:prop_contr}
  Any contractible type is a proposition by \cref{ex:prop_contr}\index{contractible type!is a proposition}\index{is a proposition!contractible type}. In particular, the unit type is a proposition.

  However, propositions do not need to be inhabited: the empty type is also a proposition, since\index{empty type!is a proposition}\index{is a proposition!empty type}
\begin{equation*}
\prd{x,y:\emptyt}\iscontr(x=y)
\end{equation*}
follows from the induction principle of the empty type.
\end{eg}

In the following lemma we prove that in order to show that a type $A$ is a proposition, it suffices to show that any two terms of $A$ are equal. In other words, propositions are types with \define{proof irrelevance}.

\begin{thm}\label{lem:isprop_eq}
  Let $A$ be a type. Then the following are equivalent:
  \begin{enumerate}
  \item The type $A$ is a proposition.
  \item Any two terms of type $A$ can be identified, i.e., there is a dependent function\index{is-prop'(A)@{$\isprop'(A)$}}\index{is-prop(A)@{$\isprop(A)$}!is-prop(A) iff is-prop'(A)@{$\isprop(A)\leftrightarrow\isprop'(A)$}}
    \begin{equation*}
      \isprop'(A)\defeq\prd{x,y:A}\id{x}{y}.
    \end{equation*}
  \item The type $A$ is contractible as soon as it is inhabited, i.e., there is a function\index{is-prop(A)@{$\isprop(A)$}!is-prop(A) iff A to is-contr(A)@{$\isprop(A)\leftrightarrow(A\to\iscontr(A))$}}
    \begin{equation*}
      A \to \iscontr(A).
    \end{equation*}
  \item The map $\const_\ttt : A\to\unit$ is an embedding.\index{is-prop(A)@{$\isprop(A)$}!is-prop(A) iff is-emb(const star)@{$\isprop(A)\leftrightarrow\isemb(\const_\ttt)$}}
  \end{enumerate}
\end{thm}

\begin{proof}
  To show that (i) implies (ii), let $A$ be a proposition. Then its identity types are contractible, so the center of contraction of $\id{x}{y}$ is identification $\id{x}{y}$, for each $x,y:A$. 

  To show that (ii) implies (iii), suppose that $A$ comes equipped with $p:\prd{x,y:A}\id{x}{y}$. Then for any $x:A$ the dependent function $p(x):\prd{y:A}\id{x}{y}$ is a contraction of $A$. Thus we obtain the function
  \begin{equation*}
    \lam{x}(x,p(x)):A\to\iscontr(A).
  \end{equation*}

  To show that (iii) implies (iv), suppose that $A\to\iscontr(A)$. We first make the simple observation that
  \begin{equation*}
    (X\to \isemb(f))\to \isemb(f)
  \end{equation*}
  for any map $f:X\to Y$, so it suffices to show that $A\to\isemb(\const_\ttt)$. However, assuming we have $x:A$, it follows by assumption that $A$ is contractible. Therefore, it follows by \cref{ex:contr_equiv} that the map $\const_\ttt:A\to\unit$ is an equivalence. Since it is an equivalence, it is an embedding by \cref{cor:emb_equiv}.

  To show that (iv) implies (i), note that if $A\to\unit$ is an embedding, then the identity types of $A$ are equivalent to contractible types and therefore they must be contractible.
\end{proof}

In the following lemma we show that propositions are closed under equivalences.

\begin{lem}\label{lem:prop_equiv}
Let $A$ and $B$ be types, and let $e:\eqv{A}{B}$. Then we have\index{proposition!closed under equivalences}
\begin{equation*}
\isprop(A)\leftrightarrow\isprop(B).
\end{equation*}
\end{lem}

\begin{proof}
We will show that $\isprop(B)$ implies $\isprop(A)$. This suffices, because the converse follows from the fact that $e^{-1}:B\to A$ is also an equivalence. 

Since $e$ is assumed to be an equivalence, it follows by \cref{cor:emb_equiv} that
\begin{equation*}
\apfunc{e} : (x=y)\to (e(x)=e(y))
\end{equation*}
is an equivalence for any $x,y:A$. If $B$ is a proposition, then in particular the type $e(x)=e(y)$ is contractible for any $x,y:A$, so the claim follows from \cref{thm:contr_equiv}.
\end{proof}

  In set theory, a set $y$ is said to be a subset\index{subset} of a set $x$, if any element of $y$ is an element of $x$, i.e., if the condition
  \begin{equation*}
    \forall_z (z\in y)\to (z\in x)
  \end{equation*}
  holds. We have already noted that type theory is different from set theory in that terms in type theory come equipped with a \emph{unique} type. Moreover, in set theory the proposition $x\in y$ is well-formed for any two sets $x$ and $y$, whereas in type theory the judgment $a:A$ is only well-formed if it is derived using the postulated inference rules. Because of these differences we must find a different way to talk about subtypes.

  Note that in set theory there is a correspondence between the subsets of a set $x$, and the \emph{predicates} on $x$. A predicate on $x$ is just a proposition $P(z)$ that varies over the elements $z\in x$. Indeed, if $y$ is a subset of $x$, then the corresponding predicate is the proposition $z\in y$. Conversely, if $P$ is a predicate on $x$, then we obtain the subset
  \begin{equation*}
    \{z\in x\mid P(z)\}
  \end{equation*}
  of $x$. Now we have the right idea of subtypes in type theory: they are families of propositions.

\begin{defn}
A type family $B$ over $A$ is said to be a \define{subtype}\index{subtype} of $A$ if for each $x:A$ the type $B(x)$ is a proposition. When $B$ is a subtype of $A$, we also say that $B(x)$ is a \define{property}\index{property} of $x:A$.
\end{defn}

We will show in \cref{thm:subtype} that a type family $B$ over $A$ is a subtype of $A$ if and only if the projection map $\proj 1:\big(\sm{x:A}B(x)\big)\to A$ is an embedding.

\index{proposition|)}

\subsection{Sets}

\index{set|(}
\begin{defn}
  A type $A$ is said to be a \define{set} if it comes equipped with a term of type
  \index{is-set(A)@{$\isset(A)$}}\index{is a set}
\begin{equation*}
\isset(A)\defeq \prd{x,y:A}\isprop(\id{x}{y}).
\end{equation*}
\end{defn}

\begin{lem}
A type $A$ is a set if and only if it satisfies \define{axiom K}\index{axiom K}, i.e., if and only if it comes equipped with a term of type\index{is-set(A)@{$\isset(A)$}!is-set(A) iff axiom-K(A)@{$\isset(A)\leftrightarrow\axiomK(A)$}}
\begin{equation*}
\axiomK(A)\defeq\prd{x:A}\prd{p:\id{x}{x}}\id{\refl{x}}{p}.
\end{equation*}
\end{lem}

\begin{proof}
If $A$ is a set, then $\id{x}{x}$ is a proposition, so any two of its elements are equal. 
This implies axiom $K$. 

For the converse, if $A$ satisfies axiom $K$, then for any $p,q:\id{x}{y}$ we have $\id{\ct{p}{q^{-1}}}{\refl{x}}$, and hence $\id{p}{q}$. This shows that $\id{x}{y}$ is a proposition, and hence that $A$ is a set.
\end{proof}

\begin{thm}\label{lem:prop_to_id}
Let $A$ be a type, and let $R:A\to A\to\UU$ be a binary relation on $A$ satisfying
\begin{enumerate}
\item Each $R(x,y)$ is a proposition,
\item $R$ is reflexive, as witnessed by $\rho:\prd{x:A}R(x,x)$,
\item There is a map
  \begin{equation*}
    R(x,y)\to (x=y)
  \end{equation*}
  for each $x,y:A$.
\end{enumerate}
Then any family of maps
\begin{equation*}
\prd{x,y:A}(\id{x}{y})\to R(x,y)
\end{equation*}
is a family of equivalences. Consequently, the type $A$ is a set.
\end{thm}

\begin{proof}
Let $f:\prd{x,y:A}R(x,y)\to(\id{x}{y})$. 
Since $R$ is assumed to be reflexive, we also have a family of maps
\begin{equation*}
\pathind_x(\rho(x)):\prd{y:A}(\id{x}{y})\to R(x,y).
\end{equation*}
Since each $R(x,y)$ is assumed to be a proposition, it therefore follows that each $R(x,y)$ is a retract of $\id{x}{y}$. Therefore it follows that $\sm{y:A}R(x,y)$ is a retract of $\sm{y:A}x=y$, which is contractible. We conclude that $\sm{y:A}R(x,y)$ is contractible, and therefore that any family of maps
\begin{equation*}
  \prd{y:A}(x=y)\to R(x,y)
\end{equation*}
is a family of equivalences.

Now it also follows that $A$ is a set, since its identity types are equivalent to propositions, and therefore they are propositions by \cref{lem:prop_equiv}. 
\end{proof}

\begin{defn}
  A map $f:A\to B$ is said to be \define{injective}\index{injective function} if for any $x,y:A$ there is a map
  \begin{equation*}
    (f(x)=f(y))\to (x=y).
  \end{equation*}
\end{defn}

\begin{cor}\label{cor:is-emb-is-injective}
  Any injective map into a set is an embedding.\index{is an embedding!injective map into a set}
\end{cor}

\begin{proof}
  Let $f:A\to B$ be an injective map between sets. Now consider the relation
  \begin{equation*}
    R(x,y)\defeq (f(x)=f(y)).
  \end{equation*}
  Note that $R$ is reflexive, and that $R(x,y)$ is a proposition for each $x,y:A$. Moreover, by the assumption that $f$ is injective, we have
  \begin{equation*}
    R(x,y)\to (x=y)
  \end{equation*}
  for any $x,y:A$. Therefore we are in the situation of \cref{lem:prop_to_id}, so it follows that the map $\apfunc{f} : (x=y)\to (f(x)=f(y))$ is an equivalence.
\end{proof}

\begin{thm}\label{thm:eq_nat}
The type of natural numbers is a set.\index{is a set!natural numbers}\index{natural numbers!is a set}
\end{thm}

\begin{proof}
We will apply \cref{lem:prop_to_id}. Note that the observational equality $\EqN:\N\to(\N\to\UU)$ on $\N$ (\cref{defn:obs_nat}) is a reflexive relation by \cref{ex:obs_nat_eqrel}, and moreover that $\EqN(n,m)$ is a proposition for every $n,m:\N$ (proof by double induction).
Therefore it suffices to show that
\begin{equation*}
\prd{m,n:\N}\EqN(m,n)\to (\id{m}{n}).
\end{equation*}
This follows from the fact that observational equality is the \emph{least} reflexive relation, which was shown in \cref{ex:obs_nat_least}.
\end{proof}

\begin{comment}
\begin{thm}[Hedberg]\label{thm:dec_eq}
Any type with decidable equality is a set.
\end{thm}

\begin{proof}
Let $A$ be a type, and let $d:\prd{x,y:A}(\id{x}{y})+\neg(\id{x}{y})$ be the witness that $A$ has decidable equality.
We first construct a reflexive binary relation $E:A\to A\to\type$ such that each $E(x,y)$ is a proposition.
For every $x,y:A$, we first define a type family $E'(x,y):((\id{x}{y})+\neg(\id{x}{y}))\to\type$ by
\begin{align*}
E'(x,y,\inl(p)) & \defeq \unit \\
E'(x,y,\inr(p)) & \defeq \emptyt.
\end{align*}
Note that $E'(x,y,q)$ is a proposition for each $x,y:A$ and $q:(\id{x}{y})+\neg(\id{x}{y})$. 
Now we set $E(x,y)\defeq E'(x,y,d(x,y))$. Then $E$ is clearly reflexive, and a family of propositions.
Therefore it remains to show that $E$ implies identity. 

Since $E$ is defined as an instance of $E'$, it suffices to construct a term of type
\begin{equation*}
\prd{x,y:A}\prd{q:(\id{x}{y})+\neg(\id{x}{y})} E'(q)\to (\id{x}{y}). 
\end{equation*}
By induction of disjoint sums, it suffices to construct terms of types
\begin{align*}
& \prd{x,y:A}\prd{p:\id{x}{y}} \unit\to (\id{x}{y}) \\
& \prd{x,y:A}\prd{p:\neg(\id{x}{y})} \emptyt\to (\id{x}{y}).
\end{align*}
In the first case, we take $\lam{x}\lam{y}\lam{p}{t}p$, and the second case is by induction on the empty type.
\end{proof}
\end{comment}
\index{set|)}

\subsection{General truncation levels}

\begin{defn}
We define $\istrunc{} : \Z_{\geq-2}\to\UU\to\UU$ by induction on $k:\Z_{\geq -2}$, taking\index{is-trunc k(A)@{$\istrunc{k}(A)$}}
\begin{align*}
\istrunc{-2}(A) & \defeq \iscontr(A) \\
\istrunc{k+1}(A) & \defeq \prd{x,y:A}\istrunc{k}(\id{x}{y}).\qedhere
\end{align*}
For any type $A$, we say that $A$ is \define{$k$-truncated}\index{k-truncated type@{$k$-truncated type}|see {truncated type}}, or a \define{$k$-type}\index{k-type@{$k$-type}}, if there is a term of type $\istrunc{k}(A)$. We say that a map $f:A\to B$ is $k$-truncated if its fibers are $k$-truncated.\index{k-truncated map@{$k$-truncated map}|see {truncated map}}\index{truncated map}
\end{defn}

%For the rest of this section, let $k:\Z_{\geq-2}$.

\begin{thm}\label{thm:istrunc_next}
If $A$ is a $k$-type, then $A$ is also a $(k+1)$-type.\index{is-trunc k(A)@{$\istrunc{k}(A)$}!is-trunc k(A) to is-trunc k+1(A)@{$\istrunc{k}(A)\to\istrunc{k+1}(A)$}}
\end{thm}

\begin{proof}
We have seen in \cref{eg:prop_contr} that contractible types are propositions. This proves the base case.
For the inductive step, note that if any $k$-type is also a $(k+1)$-type, then any $(k+1)$-type is a $(k+2)$-type, since its identity types are $k$-types and therefore $(k+1)$-types.
\end{proof}

\begin{thm}\label{thm:ktype_eqv}
If $e:\eqv{A}{B}$ is an equivalence, and $B$ is a $k$-type, then so is $A$.\index{truncated type!closed under equivalences}
\end{thm}

\begin{proof}
We have seen in \cref{ex:contr_equiv} that if $B$ is contractible and $e:\eqv{A}{B}$ is an equivalence, then $A$ is also contractible. This proves the base case.

For the inductive step, assume that the $k$-types are stable under equivalences, and consider $e:\eqv{A}{B}$ where $B$ is a $(k+1)$-type. In \cref{cor:emb_equiv} we have seen that
\begin{equation*}
\apfunc{e}:(\id{x}{y})\to(\id{e(x)}{e(y)})
\end{equation*}
is an equivalence for any $x,y$. Note that $\id{e(x)}{e(y)}$ is a $k$-type, so by the induction hypothesis it follows that $\id{x}{y}$ is a $k$-type. This proves that $A$ is a $(k+1)$-type.
\end{proof}

\begin{cor}\label{cor:emb_into_ktype}
If $f:A\to B$ is an embedding, and $B$ is a $(k+1)$-type, then so is $A$.\index{truncated type!closed under embeddings}
\end{cor}

\begin{proof}
By the assumption that $f$ is an embedding, the action on paths
\begin{equation*}
\apfunc{f}:(\id{x}{y})\to (\id{f(x)}{f(y)})
\end{equation*}
is an equivalence for every $x,y:A$. Since $B$ is assumed to be a $(k+1)$-type, it follows that $f(x)=f(y)$ is a $k$-type for every $x,y:A$. Therefore we conclude by \cref{thm:ktype_eqv} that $\id{x}{y}$ is a $k$-type for every $x,y:A$. In other words, $A$ is a $(k+1)$-type.
\end{proof}

\begin{thm}
Let $B$ be a type family over $A$. Then the following are equivalent:\index{truncated family of types}
\begin{enumerate}
\item For each $x:A$ the type $B(x)$ is $k$-truncated. In this case we say that the family $B$ is \define{$k$-truncated}.
\item The projection map
\begin{equation*}
\proj 1 : \Big(\sm{x:A}B(x)\Big)\to A
\end{equation*}
is $k$-truncated.
\end{enumerate}
\end{thm}

\begin{proof}
By \cref{ex:proj_fiber} we obtain equivalences
\begin{equation*}
\eqv{\fib{\proj 1}{x}}{B(x)}
\end{equation*}
for every $x:A$. Therefore the claim follows from \cref{thm:ktype_eqv}.
\end{proof}

\begin{thm}\label{thm:trunc_ap}
Let $f:A\to B$ be a map. The following are equivalent:
\begin{enumerate}
\item The map $f$ is $(k+1)$-truncated.
\item For each $x,y:A$, the map
\begin{equation*}
\apfunc{f} : (x=y)\to (f(x)=f(y))
\end{equation*}
is $k$-truncated. 
\end{enumerate}
\end{thm}

\begin{proof}
First we show that for any $s,t:\fib{f}{b}$ there is an equivalence
\begin{equation*}
\eqv{(s=t)}{\fib{\apfunc{f}}{\ct{\proj 2(s)}{\proj 2(t)^{-1}}}}
\end{equation*}
We do this by $\Sigma$-induction on $s$ and $t$, and then we calculate
\begin{align*}
(\pairr{x,p}=\pairr{y,q}) & \eqvsym \Eqfib_f((x,p),(y,q)) \\
  & \jdeq \sm{\alpha:x=y} p=\ct{\ap{f}{\alpha}}{q} \\
  & \eqvsym \sm{\alpha:x=y} \ct{\ap{f}{\alpha}}{q}=p \\
& \eqvsym \sm{\alpha:x=y} \ap{f}{\alpha}=\ct{p}{q^{-1}} \\
& \jdeq \fib{\apfunc{f}}{\ct{p}{q^{-1}}}.
\end{align*}
By these equivalences, it follows that if $\apfunc{f}$ is $k$-truncated, then for each $s,t:\fib{f}{b}$ the identity type $s=t$ is equivalent to a $k$-truncated type, and therefore we obtain by \cref{thm:ktype_eqv} that $f$ is $(k+1)$-truncated.

For the converse, note that we have equivalences
\begin{align*}
\fib{\apfunc{f}}{p} & \eqvsym ((x,p)=(y,\refl{f(y)})).
\end{align*}
It follows that if $f$ is $(k+1)$-truncated, then the identity type $(x,p)=(y,\refl{f(y)})$ in $\fib{f}{f(y)}$ is $k$-truncated for any $p:f(x)=f(y)$. We conclude by \cref{thm:ktype_eqv} that the fiber $\fib{\apfunc{f}}{p}$ is $k$-truncated. 
\end{proof}

\begin{cor}\label{cor:prop_emb}
A map is an embedding if and only if its fibers are propositions.\index{is an embedding!-1-truncated map@{$(-1)$-truncated map}}
\end{cor}

\begin{cor}\label{thm:subtype}
A type family $B$ over $A$ is a subtype if and only if the projection map
\begin{equation*}
\proj 1 : \Big(\sm{x:A}B(x)\Big)\to A
\end{equation*}
is an embedding.
\end{cor}

\begin{thm}
Let $f:\prd{x:A}B(x)\to C(x)$ be a family of maps. Then the following are equivalent:
\begin{enumerate}
\item For each $x:A$ the map $f(x)$ is $k$-truncated.
\item The induced map 
\begin{equation*}
\tot{f}:\Big(\sm{x:A}B(x)\Big)\to\Big(\sm{x:A}C(x)\Big)
\end{equation*}
is $k$-truncated.
\end{enumerate}
\end{thm}

\begin{proof}
This follows directly from \cref{lem:fib_total,thm:ktype_eqv}.
\end{proof}

\begin{exercises}
\exercise
  \begin{subexenum}
  \item Show that $\succN:\N\to\N$ is an embedding.\index{succ N@{$\succN$}!is an embedding}\index{is an embedding!succ N@{$\succN$}}
  \item Show that $n\mapsto m+n$ is an embedding, for each $m:\N$.\index{add N@{$\addN$}!add N(m) is an embedding@{$\addN(m)$ is an embedding}}\index{is an embedding!add N(m)@{$\addN(m)$}} Moreover, conclude that there is an equivalence
    \begin{equation*}
      \fib{\addN(m)}{n}\simeq (m\leq n).
    \end{equation*}
  \item Show that $n\mapsto mn$ is an embedding\index{mul N@{$\mulN$}!mul N(m) is an embedding if m>0@{$\mulN(m)$ is an embedding if $m>0$}}\index{is an embedding!mul N(m) for m > 0@{$\mulN(m)$ for $m>0$}}, for each $m>0$ in $\N$. Conclude that the divisibility relation\index{d {"| n}@{$d\mid n$}!is a proposition if d>0@{is a proposition if $d>0$}}\index{is a proposition!d {"| n} for d>0@{$d\mid n$ for $d>0$}}
    \begin{equation*}
      d\mid n
    \end{equation*}
    is a proposition for each $d,n:\N$ such that $d>0$. 
  \end{subexenum}
\exercise \label{ex:diagonal}Let $A$ be a type, and let the \define{diagonal} of $A$ be the map $\delta_A:A\to A\times A$ given by $\lam{x}(x,x)$. 
\begin{subexenum}
\item Show that
\begin{equation*}
{\isequiv(\delta_A)}\leftrightarrow{\isprop(A)}.
\end{equation*}
\item Construct an equivalence $\eqv{\fib{\delta_A}{(x,y)}}{(x=y)}$ for any $x,y:A$.
\item Show that $A$ is $(k+1)$-truncated if and only if $\delta_A:A\to A\times A$ is $k$-truncated.
\end{subexenum}
\exercise \label{ex:istrunc_sigma}
\begin{subexenum}
\item Let $B$ be a type family over $A$. Show that if $A$ is a $k$-type, and $B(x)$ is a $k$-type for each $x:A$, then so is $\sm{x:A}B(x)$. Conclude that for any two $k$-types $A$ and $B$, the type $A\times B$ is also a $k$-type. Hint: for the base case, use \cref{ex:contr_in_sigma,ex:contr_equiv}.
\item Show that for any $k$-type $A$, the identity types of $A$ are also $k$-types.
\item Show that any maps $f:A\to B$ between $k$-types $A$ and $B$
is a $k$-truncated map.
\item Use \cref{ex:proj_fiber} to show that for any type family $B:A\to \UU$, if $A$ and $\sm{x:A}B(x)$ are $k$-types, then so is $B(x)$ for each $x:A$. 
\end{subexenum}
\exercise \label{ex:eq_bool}Show that $\bool$ is a set by applying \cref{lem:prop_to_id} with the observational equality on $\bool$ defined in \cref{ex:obs_bool}.
\exercise \label{ex:set_coprod}Show that for any two $(k+2)$-types $A$ and $B$, the disjoint sum $A+B$ is again a $(k+2)$-type. Conclude that $\mathbb{Z}$ is a set.
\exercise Use \cref{ex:contr_retr,ex:retr_id} to show that if $A$ is a retract of a $k$-type $B$, then $A$ is also a $k$-type.
\exercise Show that a type $A$ is a $(k+1)$-type if and only if the map $\const_x:\unit\to A$ is $k$-truncated for every $x:A$.
\exercise Consider a commuting triangle
\begin{equation*}
\begin{tikzcd}[column sep=tiny]
A \arrow[rr,"h"] \arrow[dr,swap,"f"] & & B \arrow[dl,"g"] \\
& X
\end{tikzcd}
\end{equation*}
with $H: f \htpy g \circ h$, and suppose that $g$ is $k$-truncated. Show that $f$ is $k$-truncated if and only if $h$ is $k$-truncated.
% Suppose that $h$ is $k$-truncated and surjective. Show that $f$ is $(k+1)$-truncated if and only if $g$ is $(k+1)$-truncated.
\end{exercises}
\index{truncated type|)}
\index{truncation level|)}
