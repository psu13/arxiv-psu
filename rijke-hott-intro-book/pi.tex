\section{Dependent function types}
\label{ch:pi}

\index{Pi-type@{$\Pi$-type}|see {dependent function type}}
\index{dependent function type|(}
A fundamental concept in dependent type theory is that of a dependent function. A dependent function is a function of which the type of the output may depend on the input. They are a generalization of ordinary functions, because an ordinary function $f:A\to B$ is a function of which the output $f(x)$ has type $B$ regardless of the value of $x$.

\subsection{The rules for dependent function types}
Consider a section $b$ of a family $B$ over $A$ in context $\Gamma$, i.e.,
\begin{equation*}
  \Gamma,x:A\vdash b(x):B(x).
\end{equation*}
From one point of view, such a section $b$ is an operation, or a program\index{program}, that takes as input $x:A$ and produces a term $b(x):B(x)$. From a more mathematical point of view we see $b$ as a choice of an element of each $B(x)$. In other words, we may see $b$ as a function that takes $x:A$ to $b(x):B(x)$. Note that the type $B(x)$ of the output is dependent on $x:A$. In this section we postulate rules for the \emph{type} of all such dependent functions: whenever $B$ is a family over $A$ in context $\Gamma$, there is a type
\begin{equation*}
  \prd{x:A}B(x)
\end{equation*}
in context $\Gamma$, consisting of all the dependent functions of which the output at $x:A$ has type $B(x)$. There are four principal rules for $\Pi$-types:
\begin{enumerate}
\item The formation rule, which tells us how we may form dependent function types.
\item The introduction rule, which tells us how to introduce new terms of dependent function types.
\item The elimination rule, which tells us how to use arbitrary terms of dependent function types.
\item The computation rules, which tell us how the introduction and elimination rules interact. These computation rules guarantee that every term of a dependent function type behaves as expected: as a dependent function.
\end{enumerate}
In the cases of the formation rule, the introduction rule, and the elimination rule, we also need rules that assert that all the constructions respect judgmental equality. Those rules are called \define{congruence rules}.

\subsubsection{The $\Pi$-formation rule}
\index{dependent function type!formation rule}
The \define{$\Pi$-formation rule} tells us how $\Pi$-types are constructed. The idea of $\Pi$-types is that for any type family $B$ of types over $A$, there is a type of dependent functions $\prd{x:A}B(x)$, so the $\Pi$-formation rule is as follows:\index{rules!for dependent function types!formation}
\begin{prooftree}
\AxiomC{$\Gamma,x:A\vdash B(x)~\textrm{type}$}
\RightLabel{$\Pi$.}
\UnaryInfC{$\Gamma\vdash \prd{x:A}B(x)~\type$}
\end{prooftree}
This rule simply states that in order for the type $\prd{x:A}B(x)$ to be well-formed in context $\Gamma$, the type $B$ must be a well-formed type in context $\Gamma,x:A$.

We also require that the the operation of forming dependent function types respects judgmental equality. This is postulated in the \define{congruence rule} for $\Pi$-types:
\index{rules!for dependent function types!congruence}
\index{dependent function type!congruence rule}
\begin{prooftree}
\AxiomC{$\Gamma\vdash A\jdeq A'~\type$}
\AxiomC{$\Gamma,x:A\vdash B(x)\jdeq B'(x)~\textrm{type}$}
\RightLabel{$\Pi$-eq.}
\BinaryInfC{$\Gamma\vdash \prd{x:A}B(x)\jdeq\prd{x:A'}B'(x)~\type$}
\end{prooftree}

There is one last rule that we need about the formation of $\Pi$-types, asserting that it does not matter what name we use for the variable $x$ that appears in the expression $\prd{x:A}B(x)$.
More precisely, when $x'$ is a variable that does not occur in the context $\Gamma$, then we postulate that
\index{rules!for dependent function types!change of bound variable}
\index{dependent function type!change of bound variable}
\begin{prooftree}
\AxiomC{$\Gamma,x:A\vdash B(x)~\textrm{type}$}
\RightLabel{$\Pi$-$x'/x$.}
\UnaryInfC{$\Gamma\vdash \prd{x:A}B(x)\jdeq \prd{x':A}B(x')~\type$}
\end{prooftree}
This rule is also known as \define{$\alpha$-conversion} for $\Pi$-types.

\subsubsection{The $\Pi$-introduction rule}
The introduction rule for dependent functions tells us how we may construct dependent functions of type $\prd{x:A}B(x)$. The idea is that a dependent function $f:\prd{x:A}B(x)$ is an operation that takes an $x:A$ to $f(x):B(x)$. Hence the introduction rule of dependent functions postulates that, in order to construct a dependent function one has to construct a term $b(x):B(x)$ in context $x:A$, i.e.:
\begin{prooftree}
  \AxiomC{$\Gamma,x:A \vdash b(x) : B(x)$}
  \RightLabel{$\lambda$.}
  \UnaryInfC{$\Gamma\vdash \lam{x}b(x) : \prd{x:A}B(x)$}
\end{prooftree}
This introduction rule%
\index{dependent function type!introduction rule|see {$\lambda$-abstraction}}
for dependent functions is also called the \define{$\lambda$-abstraction rule}%
\index{lambda-abstraction@{$\lambda$-abstraction}}%
\index{rules!for dependent function types!lambda-abstraction@{$\lambda$-abstraction}}%
\index{dependent function type!lambda-abstraction@{$\lambda$-abstraction}}, and we also say that the $\lambda$-abstraction $\lam{x}b(x)$ \define{binds} the variable $x$ in $b$. Just like ordinary mathematicians, we will sometimes write $x\mapsto b(x)$ for a function $\lam{x}b(x)$. The map $n\mapsto n^2$ is an example.

The $\lambda$-abstraction is also required to respect judgmental equality. Therefore we postulate the \define{congruence rule} for $\lambda$-abstraction,
\index{rules!for dependent function types!lambda-congruence@{$\lambda$-congruence}}
\index{lambda-congruence@{$\lambda$-congruence}}
\index{dependent function type!lambda-congruence@{$\lambda$-congruence}}
which asserts that
\begin{prooftree}
  \AxiomC{$\Gamma,x:A \vdash b(x)\jdeq b'(x) : B(x)$}
  \RightLabel{$\lambda$-eq.}
  \UnaryInfC{$\Gamma\vdash \lam{x}b(x)\jdeq \lam{x}b'(x) : \prd{x:A}B(x)$}
\end{prooftree}

\subsubsection{The $\Pi$-elimination rule}

\index{dependent function type!elimination rule|see {evaluation}}
The elimination rule for dependent function types provides us with a way to \emph{use} dependent functions. The way to use a dependent function is to apply it to an argument of the domain type. The $\Pi$-elimination rule is therefore also called the \define{evaluation rule}\index{evaluation}\index{rules!for dependent function types!evaluation}\index{dependent function type!evaluation}. It asserts that given a dependent function $f:\prd{x:A}B(x)$ in context $\Gamma$ we obtain a term $f(x)$ of type $B(x)$ in context $\Gamma,x:A$. More formally:
\begin{prooftree}
\AxiomC{$\Gamma\vdash f:\prd{x:A}B(x)$}
\RightLabel{$ev$}
\UnaryInfC{$\Gamma,x:A\vdash f(x) : B(x)$}
\end{prooftree}
Again we require that evaluation respects judgmental equality:
\begin{prooftree}
  \AxiomC{$\Gamma\vdash f\jdeq f':\prd{x:A}B(x)$}
  \UnaryInfC{$\Gamma,x:A\vdash f(x)\jdeq f'(x):B(x)$}
\end{prooftree}

\subsubsection{The $\Pi$-computation rules}

\index{dependent function type!computation rules|see {$\beta$- and $\eta$-rules}}
We now postulate rules that specify the behavior of functions. First, we have a rule that asserts that a function of the form $\lam{x}b(x)$ behaves as expected: when we evaluate it at $x:A$, then we obtain the value $b(x):B(x)$. This rule is called the \define{$\beta$-rule}\index{beta-rule@{$\beta$-rule}!for Pi-types@{for $\Pi$-types}}\index{rules!for dependent function types!beta-rule@{$\beta$-rule}}\index{dependent function type!beta-rule@{$\beta$-rule}}
\begin{prooftree}
\AxiomC{$\Gamma,x:A \vdash b(x) : B(x)$}
\RightLabel{$\beta$.}
\UnaryInfC{$\Gamma,x:A \vdash (\lambda y.b(y))(x)\jdeq b(x) : B(x)$}
\end{prooftree}
Second, we postulate a rule that asserts that all elements of a $\Pi$-type are (dependent) functions. This rule is known as the \define{$\eta$-rule}\index{eta-rule@{$\eta$-rule}!for Pi-types@{for $\Pi$-types}}\index{rules!for dependent function types!eta-rule@{$\eta$-rule}}\index{dependent function type!eta-rule@{$\eta$-rule}}
\begin{prooftree}
\AxiomC{$\Gamma\vdash f:\prd{x:A}B(x)$}
\RightLabel{$\eta$.}
\UnaryInfC{$\Gamma \vdash \lam{x}f(x) \jdeq f : \prd{x:A}B(x)$}
\end{prooftree}
In other words, the computation rules ($\beta$ and $\eta$) for dependent function types postulate that $\lambda$-abstraction rule and the evaluation rule are mutual inverses. This completes the specification of dependent function types.

\subsection{Ordinary function types}
Given two types $A$ and $B$ in context $\Gamma$, we can use the rules for $\Pi$-types to form the type $A\to B$ of \emph{ordinary} functions from $A$ to $B$. The type of ordinary functions is obtained by first weakening $B$ by $A$ and subsequently applying the $\Pi$-formation rule, as in the following derivation:
\begin{prooftree}
\AxiomC{$\Gamma\vdash A~\textrm{type}$}
\AxiomC{$\Gamma\vdash B~\textrm{type}$}
\RightLabel{$W$}
\BinaryInfC{$\Gamma,x:A\vdash B~\textrm{type}$}
\RightLabel{$\Pi$}
\UnaryInfC{$\Gamma\vdash \prd{x:A}B~\textrm{type}$}
\end{prooftree}
A term $f:\prd{x:A}B$ is a function that takes an argument $x:A$ and returns $f(x):B$. In other words, terms of type $\prd{x:A}B$ are indeed ordinary functions from $A$ to $B$. Therefore we will write $A\to B$\index{A arrow B@{$A\to B$}|see {function type}} for the \define{type of functions}\index{function type} from $A$ to $B$. Sometimes we will also write $B^A$\index{B^A@{$B^A$}|see {function type}} for the type $A\to B$.

We give a brief summary of the rules specifying ordinary function types, omitting the congruence rules. All of these rules can be derived easily from the corresponding rules for $\Pi$-types.\index{rules!for function types}
\begin{prooftree}
\AxiomC{$\Gamma\vdash A~\textrm{type}$}
\AxiomC{$\Gamma\vdash B~\textrm{type}$}
\RightLabel{$\to$}
\BinaryInfC{$\Gamma\vdash A\to B~\textrm{type}$}
\end{prooftree}%
\begin{center}
\begin{minipage}{.45\textwidth}
\begin{prooftree}
\AxiomC{$\Gamma\vdash B~\textrm{type}$}
\AxiomC{$\Gamma,x:A\vdash b(x):B$}
\RightLabel{$\lambda$}
\BinaryInfC{$\Gamma\vdash \lam{x}b(x):A\to B$}
\end{prooftree}%
\end{minipage}
\begin{minipage}{.45\textwidth}
\begin{prooftree}
\AxiomC{$\Gamma\vdash f:A\to B$}
\RightLabel{$ev$}
\UnaryInfC{$\Gamma,x:A\vdash f(x):B$}
\end{prooftree}%
\end{minipage}
\end{center}
\begin{center}
\begin{minipage}{.45\textwidth}
\begin{prooftree}
\AxiomC{$\Gamma\vdash B~\textrm{type}$}
\AxiomC{$\Gamma,x:A\vdash b(x):B$}
\RightLabel{$\beta$}
\BinaryInfC{$\Gamma,x:A\vdash(\lam{y}b(y))(x)\jdeq b(x):B$}
\end{prooftree}%
\end{minipage}
\begin{minipage}{.45\textwidth}
\begin{prooftree}
\AxiomC{$\Gamma\vdash f:A\to B$}
\RightLabel{$\eta$}
\UnaryInfC{$\Gamma\vdash\lam{x} f(x)\jdeq f:A\to B$}
\end{prooftree}
\end{minipage}
\end{center}

\subsection{The identity function, composition, and their laws}

First, we use the rules of dependent type theory to construct the identity function on an arbitrary type. 

\begin{defn}
For any type $A$ in context $\Gamma$, we define the \define{identity function}\index{identity function}\index{function type!identity function} $\idfunc[A]:A\to A$\index{id A@{$\idfunc[A]$}} using the variable rule:
\begin{prooftree}
\AxiomC{$\Gamma\vdash A~\textrm{type}$}
\UnaryInfC{$\Gamma,x:A\vdash x:A$}
\UnaryInfC{$\Gamma\vdash \idfunc[A]\defeq \lam{x}x:A\to A$}
\end{prooftree}
\end{defn}

Note that we have used the symbol $\defeq$ in the conclusion to define the identity function. A judgment of the form $\Gamma\vdash a\defeq b:A$ should be read as "$b$ is a well-defined term of type $A$ in context $\Gamma$, and we will refer to it as $a$".

By the above construction of the identity function we see that the identity function can be introduced with the following rule
\begin{prooftree}
  \AxiomC{$\Gamma\vdash A~\type$}
  \UnaryInfC{$\Gamma\vdash\idfunc[A]:A\to A$}
\end{prooftree}
Moreover, by the $\beta$-rule we see that the identity function satisfies the following computation rule:
\begin{prooftree}
  \AxiomC{$\Gamma\vdash A~\type$}
  \UnaryInfC{$\Gamma,x:A\vdash \idfunc[A](x)\jdeq x:A$}
\end{prooftree}

Next, we define the composition of functions. We will introduce the composition operation itself as a function $\comp$ that takes two arguments: the first argument is a function $g:B\to C$, and the second argument is a function $f:A\to B$. The output is a function $\comp(g,f):A\to C$, for which we often write $g\circ f$.

Types of functions with multiple arguments can be formed by iterating the $\Pi$-formation rule or the $\to$-formation rule. For example, a function
\begin{equation*}
  f:A\to (B\to C)
\end{equation*}
takes two arguments: first it takes an argument $x:A$, and the output $f(x)$ has type $B\to C$. This is again a function type, so $f(x)$ is a function that takes an argument $y:B$, and its output $f(x)(y)$ has type $C$. We will usually write $f(x,y)$ for $f(x)(y)$. With this idea of iterating function types, we see that type of the composition operation $\comp$ should be
\begin{equation*}
  (B\to C)\to ((A\to B)\to (A\to C)).
\end{equation*}
It is the type of functions, taking a function $g:B\to C$, to the type of functions $(A\to B)\to (A\to C)$. Thus, $\comp(g)$ is again a function, mapping a function $f:A\to B$ to the type of functions $B\to C$. 

\begin{defn}
For any three types $A$, $B$, and $C$ in context $\Gamma$, there is a \define{composition}\index{function type!composition}\index{composition!of functions} operation
\begin{equation*}
\comp:(B\to C)\to ((A\to B)\to (A\to C)),
\end{equation*}
i.e., we can derive
\begin{prooftree}
\AxiomC{$\Gamma\vdash A~\textrm{type}$}
\AxiomC{$\Gamma\vdash B~\textrm{type}$}
\AxiomC{$\Gamma\vdash C~\textrm{type}$}
\TrinaryInfC{$\Gamma\vdash\comp:(B\to C)\to ((A\to B)\to (A\to C))$}
\end{prooftree}
We will usually write $g\circ f$\index{g composed with f@{$g\circ f$}} for $\comp(g,f)$\index{comp(g,f)@{$\comp(g,f)$}}. Moreover, the composition operation satisfies the following computation rule:
\begin{prooftree}
  \AxiomC{$\Gamma\vdash A~\textrm{type}$}
  \AxiomC{$\Gamma\vdash B~\textrm{type}$}
  \AxiomC{$\Gamma\vdash C~\textrm{type}$}
  \TrinaryInfC{$\Gamma,g:B\to C,f:A\to B,x:A \vdash\comp(g,f,x)\jdeq g(f(x)) :C$}
\end{prooftree}
\end{defn}

\begin{constr}
  The idea of the definition is to define $\comp(g,f)$ to be the function $\lam{x}g(f(x))$. The derivation we use to construct $\comp$ is as follows:
  \begin{prooftree}
    \AxiomC{$\Gamma\vdash A~\type$}
    \AxiomC{$\Gamma\vdash B~\type$}
    \BinaryInfC{$\Gamma,f:B^A,x:A\vdash f(x):B$}
    \UnaryInfC{$\Gamma,g:C^B,f:B^A,x:A\vdash f(x):B$}
    \AxiomC{$\Gamma\vdash B~\type$}
    \AxiomC{$\Gamma\vdash C~\type$}
    \BinaryInfC{$\Gamma,g:C^B,y:B\vdash g(y):C$}
    \UnaryInfC{$\Gamma,g:C^B,f:B^A,y:B\vdash g(y):C$}
    \UnaryInfC{$\Gamma,g:C^B,f:B^A,x:A,y:B\vdash g(y):C$}
    \BinaryInfC{$\Gamma,g:C^B,f:B^A,x:A\vdash g(f(x)) : C$}
    \UnaryInfC{$\Gamma,g:C^B,f:B^A\vdash \lam{x}g(f(x)):C^A$}
    \UnaryInfC{$\Gamma,g:B\to C\vdash \lam{f}\lam{x}g(f(x)):B^A\to C^A$}
    \UnaryInfC{$\Gamma\vdash\comp\defeq \lam{g}\lam{f}\lam{x}g(f(x)):C^B\to (B^A\to C^A)$}
  \end{prooftree}
  It is immediate by the $\beta$-rule that the composition operation satisfies the asserted computation rule.
\end{constr}

The rules of function types can be used to derive the laws of a category\index{category laws!for functions} for functions, i.e., we can derive that function composition is associative and that the identity function satisfies the unit laws. In the remainder of this section we will give these derivations.

\begin{lem}
Composition of functions is associative\index{associativity!of function composition}\index{composition!of functions!associativity}, i.e., we can derive
\begin{prooftree}
\AxiomC{$\Gamma\vdash f:A\to B$}
\AxiomC{$\Gamma\vdash g:B\to C$}
\AxiomC{$\Gamma\vdash h:C\to D$}
\TrinaryInfC{$\Gamma \vdash (h\circ g)\circ f\jdeq h\circ(g\circ f):A\to D$}
\end{prooftree}
\end{lem}

\begin{proof}
  The main idea of the proof is that both $((h\circ g)\circ f)(x)$ and $(h\circ (g\circ f))(x)$ evaluate to $h(g(f(x))$, and therefore $(h\circ g)\circ f$ and $h\circ(g\circ f)$ must be judgmentally equal. This idea is made formal in the following derivation:
  \begin{prooftree}
    \AxiomC{$\Gamma\vdash f:A\to B$}
    \UnaryInfC{$\Gamma,x:A\vdash f(x):B$}
    \AxiomC{$\Gamma\vdash g:B\to C$}
    \UnaryInfC{$\Gamma,y:B\vdash g(y):C$}
    \UnaryInfC{$\Gamma,x:A,y:B\vdash g(y):C$}
    \BinaryInfC{$\Gamma,x:A\vdash g(f(x)):C$}
    \AxiomC{$\Gamma\vdash h:C\to D$}
    \UnaryInfC{$\Gamma,z:C\vdash h(z):D$}
    \UnaryInfC{$\Gamma,x:A,z:C\vdash h(z):D$}
    \BinaryInfC{$\Gamma,x:A\vdash h(g(f(x))):D$}
    \UnaryInfC{$\Gamma,x:A\vdash h(g(f(x)))\jdeq h(g(f(x))):D$}
    \UnaryInfC{$\Gamma,x:A\vdash (h\circ g)(f(x))\jdeq h((g\circ f)(x)):D$}
    \UnaryInfC{$\Gamma,x:A\vdash ((h\circ g)\circ f)(x)\jdeq (h\circ (g \circ f))(x):D$}
    \UnaryInfC{$\Gamma\vdash (h\circ g)\circ f\jdeq h\circ(g\circ f):A\to D$.}
  \end{prooftree}
\end{proof}

\begin{lem}\label{lem:fun_unit}
Composition of functions satisfies the left and right unit laws\index{left unit law|see {unit laws}}\index{right unit law|see {unit laws}}\index{unit laws!for function composition}\index{composition!of functions!unit laws}, i.e., we can derive
\begin{prooftree}
\AxiomC{$\Gamma\vdash f:A\to B$}
\UnaryInfC{$\Gamma\vdash \idfunc[B]\circ f\jdeq f:A\to B$}
\end{prooftree}
and
\begin{prooftree}
\AxiomC{$\Gamma\vdash f:A\to B$}
\UnaryInfC{$\Gamma\vdash f\circ\idfunc[A]\jdeq f:A\to B$}
\end{prooftree}
\end{lem}

\begin{proof}
The derivation for the left unit law is
%\begin{prooftree}
%\AxiomC{$\Gamma\vdash f:A\to B$}
%\UnaryInfC{$\Gamma,x:A\vdash f(x):B$}
%\AxiomC{$\Gamma\vdash B~\type$}
%\UnaryInfC{$\Gamma,y:B\vdash \idfunc[B](y)\jdeq y:B$}
%\UnaryInfC{$\Gamma,x:A,y:B\vdash \idfunc[B](y)\jdeq y:B$}
%\BinaryInfC{$\Gamma,x:A\vdash \idfunc[B](f(x))\jdeq f(x):B$}
%\UnaryInfC{$\Gamma,x:A\vdash (\idfunc[B]\circ f)(x)\jdeq f(x):B$}
%\UnaryInfC{$\Gamma\vdash \idfunc[B]\circ f\jdeq f:A\to B$}
%\end{prooftree}
\begin{prooftree}
  \AxiomC{$\Gamma\vdash f:A\to B$}
  \UnaryInfC{$\Gamma,x:A\vdash f(x):B$}
  \AxiomC{$\Gamma\vdash A~\type$}
  \AxiomC{$\Gamma\vdash B~\type$}
  \UnaryInfC{$\Gamma,y:B\vdash\idfunc(y)\jdeq y:B$}
  \BinaryInfC{$\Gamma,x:A,y:B\vdash\idfunc(y)\jdeq y:B$}
  \BinaryInfC{$\Gamma,x:A\vdash\idfunc(f(x))\jdeq f(x):B$}
  \UnaryInfC{$\Gamma\vdash\lam{x}\idfunc(f(x))\jdeq\lam{x}f(x):A\to B$}
  \AxiomC{$\Gamma\vdash f:A\to B$}
  \UnaryInfC{$\Gamma\vdash\lam{x}f(x)\jdeq f:A\to B$}
  \BinaryInfC{$\Gamma\vdash\idfunc\circ f\jdeq f:A\to B$}
\end{prooftree}
The right unit law is left as \cref{ex:fun_right_unit}.
\end{proof}

\begin{exercises}
  \exercise \label{ex:eta_ext}The $\eta$-rule is often seen as an extensionality principle. Use the $\eta$-rule to show that if $f$ and $g$ take equal values, then they must be equal, i.e., give a derivation for the rule
  \begin{prooftree}
    \AxiomC{$\Gamma\vdash f:\prd{x:A}B(x)$}
    \AxiomC{$\Gamma\vdash g:\prd{x:A}B(x)$}
    \AxiomC{$\Gamma,x:A\vdash f(x)\jdeq g(x):B(x)$}
    \TrinaryInfC{$\Gamma\vdash f\jdeq g:\prd{x:A}B(x)$}
  \end{prooftree}
  \exercise \label{ex:fun_right_unit}Give a derivation for the right unit law of \cref{lem:fun_unit}.\index{unit laws!for function composition}
  \exercise Show that the rule
  \begin{prooftree}
    \AxiomC{$\Gamma,x:A \vdash b(x) : B(x)$}
    \RightLabel{$\lambda$-$x'/x$}
    \UnaryInfC{$\Gamma\vdash \lam{x}b(x)\jdeq \lam{x'}b(x') : \prd{x:A}B(x)$}
  \end{prooftree}
  is derivable for any variable $x'$ that does not occur in the context $\Gamma,x:A$.
  \exercise 
  \begin{subexenum}
  \item Construct the \define{constant function}\index{constant function}\index{function!constant function}\index{const x@{$\const_x$}}\index{function!const@{$\const$}}
    \begin{prooftree}
      \AxiomC{$\Gamma\vdash A~\textrm{type}$}
      \UnaryInfC{$\Gamma,y:B\vdash \const_y:A\to B$}
    \end{prooftree}
  \item Show that
    \begin{prooftree}
      \AxiomC{$\Gamma\vdash f:A\to B$}
      \UnaryInfC{$\Gamma,z:C\vdash \const_z\circ f\jdeq\const_z : A\to C$}
    \end{prooftree}
  \item Show that
    \begin{prooftree}
      \AxiomC{$\Gamma\vdash A~\textrm{type}$}
      \AxiomC{$\Gamma\vdash g:B\to C$}
      \BinaryInfC{$\Gamma,y:B\vdash g\circ\const_y\jdeq \const_{g(y)}:A\to C$}
    \end{prooftree}
  \end{subexenum}
%  \exercise In this exercise we generalize the composition operation of non-dependent function types\index{composition!of dependent functions}:
%  \begin{subexenum}
%  \item Define a composition operation for dependent function types
%    \begin{prooftree}
%      \AxiomC{$\Gamma\vdash f:\prd{x:A}B(x)$}
%      \AxiomC{$\Gamma\vdash g:\prd{x:A}\prd{y:B(x)} C(x,y)$}
%      \BinaryInfC{$\Gamma\vdash g\circ' f:\prd{x:A} C(x,f(x))$}
%    \end{prooftree}
%    and show that this operation agrees with ordinary composition when it is specialized to non-dependent function types.
%  \item Show that composition of dependent functions agrees with ordinary composition of functions:
%    \begin{prooftree}
%      \AxiomC{$\Gamma\vdash f:A\to B$}
%      \AxiomC{$\Gamma\vdash g:B\to C$}
%      \BinaryInfC{$\Gamma\vdash (\lam{x}g)\circ' f\jdeq g\circ f:A \to C$}
%    \end{prooftree}
%  \item Show that composition of dependent functions is associative.\index{associativity!of dependent function composition}\index{composition!of dependent functions!associativity}
%  \item Show that composition of dependent functions satisfies the right unit law\index{unit laws!dependent function composition}:
%    \begin{prooftree}
%      \AxiomC{$\Gamma\vdash f:\prd{x:A}B(x)$}
%      \UnaryInfC{$\Gamma\vdash (\lam{x}f)\circ'\idfunc[A]\jdeq f :\prd{x:A}B(x)$}
%    \end{prooftree}
%  \item Show that composition of dependent functions satisfies the left unit law\index{unit laws!dependent function composition}\index{composition!of dependent functions!unit laws}:
%    \begin{prooftree}
%      \AxiomC{$\Gamma\vdash f:\prd{x:A}B(x)$}
%      \UnaryInfC{$\Gamma\vdash (\lam{x}\idfunc[B(x)])\circ' f\jdeq f:\prd{x:A}B(x)$}
%    \end{prooftree}
%  \end{subexenum}
  \exercise \label{ex:swap}
  \begin{subexenum}
  \item Given two types $A$ and $B$ in context $\Gamma$, and a type $C$ in context $\Gamma,x:A,y:B$, define the \define{swap function}\index{function!swap}\index{swap function}
    \begin{equation*}
      \Gamma\vdash \sigma:\Big(\prd{x:A}\prd{y:B}C(x,y)\Big)\to\Big(\prd{y:B}\prd{x:A}C(x,y)\Big)
    \end{equation*}
    that swaps the order of the arguments.
  \item Show that
    \begin{equation*}
      \Gamma\vdash \sigma\circ\sigma\jdeq\idfunc:\Big(\prd{x:A}\prd{y:B}C(x,y)\Big)\to \Big(\prd{x:A}\prd{y:B}C(x,y)\Big).
    \end{equation*}
  \end{subexenum}
\end{exercises}
\index{dependent function type|)}
