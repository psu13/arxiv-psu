\section{Truncations}

Truncation is a universal way of turning an arbitrary type into a $k$-truncated type. We have already seen the propositional truncation of a type $X$ in \cref{chap:image}, which is the proposition that $X$ is merely inhabited, and the set truncation of $X$ in \cref{sec:set-truncation}, which is the set of connected components of $X$. The $k$-truncation is a generalization of the propositional truncation and the set truncation to an arbitrary truncation level $k$.

We construct the truncations by recursion on $k$. The base case $k\jdeq -2$ is just the operation that sends a type $X$ to the unit type $\unit$, because up to equivalence there is only one contractible type. For the inductive step, we need to construct the $(k+1)$-truncation assuming that the $k$-truncation of an arbitrary type in a fixed universe $\UU$ exists. Our construction of the $(k+1)$-truncation is a direct generalization of the construction of the set truncation as a set quotient, where we quotient out the equivalence relation
\begin{equation*}
  (x\sim y)\defeq \trunc{-1}{x=y}.
\end{equation*}
The idea is simple: if $\trunc{k+1}{X}$ is to be the universal $(k+1)$-truncated type equipped with a map $\tproj{k+1}{\blank}:X\to \trunc{k+1}{X}$, then it has to be the case that
\begin{equation*}
  (\tproj{k+1}{x}=\tproj{k+1}{x'})\simeq \trunc{k}{x=y}.
\end{equation*}
We prove that this is indeed the case in \cref{thm:trunc_id}.

The construction of the $(k+1)$-truncation as a quotient is different than the construction of the $(k+1)$-truncation that appears in \cite{hottbook} as a higher inductive type. This construction is based on the observation that a type $X$ is $k$-truncated if and only if every map $\sphere{k+1}\to X$ is constant. In other words, for every map $f:\sphere{k+1}\to X$ into a $k$-type $X$, there is a point $x:X$ and a family of paths $p(t):x=f(t)$. If we think of $f$ as a `wheel' in $X$, then $x$ is the hub at the center of the wheel, and the paths $p(t)$ are the spokes. This leads to defining the $k$-truncation of a type $X$ by the \emph{hubs-and-spokes method}. In \cref{sec:hubs-and-spokes} we show that the $k$-truncation of a type is such a higher inductive type.

\subsection{The universal property of the truncations}

\begin{defn}\label{defn:is_truncation}
Let $X$ be a type. A map $f:X\to Y$ into an $k$-type $Y$ is said to satisfy the \define{universal property of the $k$-truncation of $X$} if the precomposition map
\begin{equation*}
\blank\circ f: (Y\to Z)\to (X\to Z)
\end{equation*}
is an equivalence for every $k$-type $Z$.
\end{defn}

\begin{rmk}
A map $f:X\to Y$ into an $k$-type $Y$ satisfies the universal property of $k$-truncation if of for every $g:X\to Z$ the type of extensions
\begin{equation*}
\begin{tikzcd}
X \arrow[dr,"g"] \arrow[d,swap,"f"] \\
Y \arrow[r,densely dotted] & Z
\end{tikzcd}
\end{equation*}
is contractible. Indeed, the type of such extensions is the type
\begin{equation*}
\sm{h:Y\to Z} h\circ f\htpy g,
\end{equation*}
which is equivalent to the fiber of the precomposition map $\blank\circ f$ at $g$. 
\end{rmk}

In the following proposition we show that if a map $f:X\to Y$ into a $k$-type $Y$ satisfies the universal property of the $k$-truncation of $X$, then $f$ also satisfies the dependent elimination property.

\begin{prp}\label{thm:trunc_dup}
  Suppose the map $f:X\to Y$ into an $k$-type $Y$. The following are equivalent:
  \begin{enumerate}
  \item The map $f$ satisfies the universal property of $k$-truncation.
  \item For any family $P$ of $k$-types over $Y$, the precomposition map
    \begin{equation*}
      \blank\circ f : \Big(\prd{y:Y}P(y)\Big)\to \Big(\prd{x:X}P(f(x))\Big)
    \end{equation*}
    is an equivalence. This property is also called the \define{dependent universal property} of the $k$-truncation.
%  \item For any family $P$ of $k$-types over $Y$, the precomposition map
%    \begin{equation*}
%      \blank\circ f : \Big(\prd{y:Y}P(y)\Big)\to\Big(\prd{x:X}P(f(x))\Big)
%    \end{equation*}
%    has a section.
  \end{enumerate}
\end{prp}

\begin{proof}
  % The fact that (ii) implies (i) and (iii) is immediate, so we only have to show that both (i) and (iii) imply (ii).
  The fact that (ii) implies (i) is immediate, so we only have to prove the converse.

  % To prove that (i) implies (ii),
  Suppose $P$ is a family of $k$-truncated types over $Y$.  
  Then we have a commuting square
  \begin{equation*}
    \begin{tikzcd}
      \Big(Y\to\sm{y:Y}P(y)\Big) \arrow[r,"\blank\circ f"] \arrow[d,swap,"\proj 1 \circ\blank"] & \Big(X\to \sm{y:Y}P(y)\Big) \arrow[d,"\proj 1\circ \blank"] \\
      \Big(Y\to Y\Big) \arrow[r,swap,"\blank\circ f"] & \Big(X\to Y)
    \end{tikzcd}
  \end{equation*}
  Since the total space $\sm{y:Y}P(y)$ is again $k$-truncated by \cref{ex:istrunc_sigma}, it follows by the universal property of the $k$-truncation that the top map is an equivalence, and by the universal property the bottom map is an equivalence too. It follows from \cref{cor:pb_equiv} that this square is a pullback square, so it induces equivalences on the fibers by \cref{cor:pb_fibequiv}. In particular we have a commuting square
  \begin{equation*}
    \begin{tikzcd}
      \Big(\prd{y:Y}P(y)\Big) \arrow[r] \arrow[d] & \Big(\prd{x:X}P(f(x))\Big) \arrow[d] \\
      \fib{(\proj 1\circ \blank)}{\idfunc[Y]} \arrow[r] & \fib{(\proj 1 \circ \blank)}{f}
    \end{tikzcd}
  \end{equation*}
  in which the left and right maps are equivalences by \cref{ex:pi_sec}, and the bottom map is an equivalence as we have just established. Therefore the top map is an equivalence, so we conclude that $f$ satisfies the dependent universal property.
%
%  To see that (iii) implies (ii), consider a family $P$ of $k$-types over $Y$. Our goal is to show that the precomposition function
%  \begin{equation*}
%    \blank\circ f : \Big(\prd{y:Y}P(y)\Big)\to\Big(\prd{x:X}P(f(x))\Big)
%  \end{equation*}
%  is an equivalence. It has a section by assumption, for which we will write $\varphi$. It therefore remains to show that $\varphi(h\circ f)\htpy h$ for any dependent function $h:\prd{y:Y}P(y)$. Note that the precomposition function
%  \begin{equation*}
%    \blank\circ f : \Big(\prd{y:Y}\varphi(h\circ f)(y)=h(y)\Big)\to\Big(\prd{x:X}\varphi(h\circ f)(f(x))=h(x)\Big)
%  \end{equation*}
%  has a section, so it suffices to show that $\varphi(h\circ f)\circ f\htpy h\circ f$. This follows again from the assumption that $\varphi$ is a section of $\blank\circ f$.
\end{proof}

Just as for pullbacks, pushouts, and the many other types characterized by a universal property, the $k$-truncation of a type is unique once it exists. We prove this in the following proposition and its corollary.

\begin{prp}
  Consider a commuting triangle
  \begin{equation*}
    \begin{tikzcd}[column sep=tiny]
      \phantom{Y'} & X \arrow[dl,swap,"f"] \arrow[dr,"{f'}"] \\
      Y \arrow[rr,swap,"h"] & & Y'
    \end{tikzcd}
  \end{equation*}
  where $Y$ and $Y'$ are assumed to be $k$-types. If any two of the following three properties hold, so does the third:
  \begin{enumerate}
  \item The map $f:X\to Y$ satisfies the universal property of the $k$-truncation of $X$.
  \item The map $f':X\to X'$ satisfies the universal property of the $k$-truncation of $X$.
  \item The map $h$ is an equivalence.
  \end{enumerate}
\end{prp}

\begin{proof}
  The claim follows by the 3-for-2 property of equivalences, since we have a commuting triangle
  \begin{equation*}
    \begin{tikzcd}[column sep=tiny]
      Z^{Y'} \arrow[rr,"\blank\circ h"] \arrow[dr,swap,"{\blank\circ f'}"] & & Z^Y \arrow[dl,"\blank\circ f"] \\
      & Z^X & \phantom{Z^{Y'}}
    \end{tikzcd}
  \end{equation*}
  for any $k$-type $Z$.
\end{proof}

\begin{cor}
  Consider two maps $f:X\to Y$ and $f':X\to Y'$ into $k$-types $Y$ and $Y'$, and suppose that both $f$ and $f'$ satisfy the universal property of the $k$-truncation of $X$. Then the type of equivalences $e:Y\simeq Y'$ equipped with a homotopy witnessing that the triangle
  \begin{equation*}
    \begin{tikzcd}[column sep=tiny]
      \phantom{Y'} & X \arrow[dl,swap,"f"] \arrow[dr,"{f'}"] \\
      Y \arrow[rr,swap,"e"] & & Y'
    \end{tikzcd}
  \end{equation*}
  commutes is contractible.
\end{cor}

\subsection{The construction of the  \texorpdfstring{$(k+1)$}{(k+1)}-truncation as a quotient}

\begin{defn}
  Consider a universe $\UU$. We say that $\UU$ \define{has $k$-truncations} if for every type $X:\UU$ there is a map $f:X\to Y$ into a $k$-type $Y:\UU$ that satisfies the universal property of the $k$-truncation of $X$.
\end{defn}

\begin{rmk}
  Note that the universal property of $k$-truncations is formulated with respect to all $k$-types, and not only with respect to the $k$-types in $\UU$.
\end{rmk}

We will use the following proposition to prove the universal property of the $(k+1)$-truncation. In fact, the converse of the following proposition also holds, and we prove it below in \cref{thm:is-truncation}.

\begin{prp}\label{prp:is-truncation}
  Consider a map $f:X\to Y$ into a $(k+1)$-type $Y$. If $f$ is surjective, and its action on paths
  \begin{equation*}
    \apfunc{f}:(x=x')\to(f(x)=f(x'))
  \end{equation*}
  satisfies the universal property of the $k$-truncation of $x=x'$, then $f$ satisfies the universal property of the $(k+1)$-truncaton of $X$.
\end{prp}

\begin{proof}
  Consider a map $g:X\to Z$ into a $(k+1)$-type $Z$. Our goal is to show that $g$ extends uniquely along $f$ to a map $h:Y\to Z$. We claim that for any $y:Y$, the type of extensions
  \begin{equation*}
    \begin{tikzcd}
      \fib{f}{y} \arrow[d] \arrow[dr,"g\circ\proj 1"] \\
      \unit \arrow[r,densely dotted] & Z
    \end{tikzcd}
  \end{equation*}
  is contractible. In other words, on each of the fibers of $f$, the map $g$ is constant in a unique way. Since $f$ is assumed to be surjective, it follows by \cref{prp:surjective} that it suffices to prove the above extension property for $y\jdeq f(x)$, for each $x:X$. In other words, we have to show that the type
  \begin{equation*}
    \sm{z:Z}\prd{x':X}(f(x)=f(x'))\to (z=g(x'))
  \end{equation*}
  is contractible for each $x:X$.

  Note that the type $z=g(x')$ is $k$-truncated, and that the map $\apfunc{f}$ is assumed to satisfy the universal property of the $k$-truncation of $x=x'$. Therefore it is equivalent to show that the type
  \begin{equation*}
    \sm{z:Z}\prd{x':X}(x=x')\to (z=g(x'))
  \end{equation*}
  is contractible. This is immediate by the universal property of the identity type (\cref{thm:yoneda}), and the fact that $\sm{z:Z}z=g(x')$ is contractible (\cref{cor:contr_path}).

  It follows by \cref{thm:funext_wkfunext} that the product
  \begin{equation*}
    \prd{y:Y}\sm{z:Z}\prd{x:X}(y=f(x))\to (z=g(x))
  \end{equation*}
  is contractible. Since $\Pi$ distributes over $\Sigma$ by \cref{thm:choice}, we obtain that the type of functions $h:Y\to Z$ equipped with a homotopy $h\circ f\htpy g$ is contractible.
\end{proof}

Before we show that any universe has $k$-truncations for arbitrary $k$, we prove a truncated version of the type theoretic Yoneda lemma under the assumption that $\UU$ has $k$-truncations for a given $k$. 

\begin{lem}\label{lem:truncated-yoneda}
  Suppose $\UU$ is a universe that has $k$-truncations
  \begin{equation*}
    \tproj{k}{\blank}:X\to\trunc{k}{X}
  \end{equation*}
  for a given $k\geq-2$, and consider a family $P$ of types over $X$. We make two claims:
  \begin{enumerate}
  \item The evaluation function
    \begin{equation*}
      \Big(\prd{y:X}\trunc{k}{x=y}\to\trunc{k}{P(y)}\Big)\to \trunc{k}{P(x)}
    \end{equation*}
    given by $h\mapsto h_x(\tproj{k}{\refl{x}})$, is an equivalence.
  \item If the total space of $P$ is contractible, then the evaluation function
    \begin{equation*}
      \Big(\prd{y:X}\trunc{k}{x=y}\simeq\trunc{k}{P(y)}\Big)\to \trunc{k}{P(x)}
    \end{equation*}
    given by $e\mapsto e_x(\tproj{k}{\refl{x}})$, is an equivalence.
  \end{enumerate}
\end{lem}

\begin{proof}
  The first claim follows immediately by the universal property of the $k$-truncation of $x=y$ and the type theoretical Yoneda lemma (\cref{thm:yoneda}).
  
  To prove the second claim, we first observe that the inclusion of equivalences into all maps induces an embedding that fits in a commuting triangle
  \begin{equation*}
    \begin{tikzcd}[column sep=0]
      \Big(\prd{y:X}\trunc{k}{x=y}\simeq\trunc{k}{P(y)}\Big) \arrow[rr,hook] \arrow[dr,swap,"\ev_{\tproj{k}{\refl{x}}}"] & & \Big(\prd{y:X}\trunc{k}{x=y}\to\trunc{k}{P(y)}\Big) \arrow[dl,"\ev_{\tproj{k}{\refl{x}}}"] \\
      & \trunc{k}{P(x)}.
    \end{tikzcd}
  \end{equation*}
  The evaluation map on the right is an equivalence, and we have to show that if the total space $\sm{y:X}P(y)$ is contractible, then the evaluation map on the left is an equivalence. We do this by showing that the top map is an equivalence.

  To see this, note that we have a commuting diagram
    \begin{equation*}
    \begin{tikzcd}
      \Big(\prd{y:X}(x=y)\simeq P(y)\Big) \arrow[d,swap,"e\mapsto\lam{y}\trunc{k}{e_y}"] \arrow[r,hook] & \Big(\prd{y:X}(x=y)\to P(y)\Big) \arrow[r,"\evrefl"] \arrow[d,swap,"h\mapsto\lam{y}\trunc{k}{h_y}"] &[2em] P(x) \arrow[d] \\
      \Big(\prd{y:X}\trunc{k}{x=y}\simeq\trunc{k}{P(y)}\Big) \arrow[r,hook] & \Big(\prd{y:X}\trunc{k}{x=y}\to\trunc{k}{P(y)}\Big) \arrow[r,swap,"\ev_{\tproj{k}{\refl{x}}}"] & \trunc{k}{P(x)}
    \end{tikzcd}
  \end{equation*}
  In the top row of this diagram we have a concatenation of equivalences: the first map is an equivalence by the fundamental theorem of identity types, and the second map is an equivalence by \cref{thm:yoneda}. The second map in the bottom row is an equivalence by the first claim of this lemma. Therefore it follows that the vertical map in the middle satisfies the universal property of the $k$-truncation. Since the type at the bottom left is $k$-truncated, we obtain by the universal property of the $k$-truncation a section of the embedding in the bottom row, which proves the claim.
\end{proof}

\begin{thm}
  Any univalent universe $\UU$ that is closed under pushouts has $k$-truncations, for every truncation level $k$. We will write
  \begin{equation*}
    \tproj{k}{\blank}:X\to\trunc{k}{X}
  \end{equation*}
  for the $k$-truncation of $X$.
\end{thm}

\begin{proof}
  It is easy to see that the terminal projection $X\to\unit$ is a $(-2)$-truncation, for any type $X$. Thus, any universe has $(-2)$-truncations.

  We will proceed by induction on $k$. Our inductive hypothesis is that $\UU$ has $k$-truncations, and our goal is to show that $\UU$ has $(k+1)$-truncations. The idea of the construction is very similar to the construction of the set quotient by an equivalence relation. Consider the type-valued relation $R_k:X\to (X\to \UU)$ given by
  \begin{equation*}
    R_k(x,x') \defeq \trunc{k}{x=x'}.
  \end{equation*}
  Analogous to the definition of set quotients, we define
  \begin{equation*}
    \trunc{k+1}{X}\defeq \im(R_k),
  \end{equation*}
  which comes equipped with a surjective map $q:X\to\trunc{k+1}{X}$. Note that the image of $R:X\to(X\to\UU)$ is (essentially) small by \cref{thm:replacement}. To see that $q:X\to\trunc{k+1}{X}$ satisfies the universal property of the $(k+1)$-truncation of $X$, we apply \cref{prp:is-truncation}. Therefore it remains to show that the action on paths
  \begin{equation*}
    \apfunc{q}:(x=x')\to (q(x)=q(x'))
  \end{equation*}
  satisfies the universal property of the $k$-truncation of $x=x'$. Since $q$ is the surjective map in the image factorization of $R_k$, it is equivalent to show that the action on paths
  \begin{equation*}
    \apfunc{R}:(x=x')\to (R_k(x)=R_k(x'))
  \end{equation*}
  satisfies the universal property of the $k$-truncation of $x=x'$. Note that we have a commuting triangle
  \begin{equation*}
    \begin{tikzcd}[column sep=-2em]
      \phantom{\prd{y:X}\trunc{k}{x=y}\simeq\trunc{k}{x'=y},} & (x=x') \arrow[dl,swap,"\apfunc{R_k}"] \arrow[dr] \\
      \Big(R_k(x)=R_k(x')\Big) \arrow[rr,swap,"\simeq"] & & \Big(\prd{y:X}\trunc{k}{x=y}\simeq\trunc{k}{x'=y}\Big),
    \end{tikzcd}
  \end{equation*}
  where the bottom map is an equivalence by function extensionality and the univalence axiom. Therefore it suffices to show that the map on the right of this triangle, which is the unique map that sends $\refl{x}$ to the family of identity equivalences, satisfies the universal property of the $k$-truncation of $x=x'$.

  This map fits in a commuting square
  \begin{equation*}
    \begin{tikzcd}[column sep=huge]
      (x=x') \arrow[d] \arrow[r,"\tproj{k}{\blank}"] & \trunc{k}{x=x'} \arrow[d,"\trunc{k}{\invfunc}"] \\
      \Big(\prd{y:X}\trunc{k}{x=y}\simeq\trunc{k}{x'=y}\Big) \arrow[r,swap,"\ev_{\tproj{k}{\refl{x}}}"] & \trunc{k}{x'=x}.
    \end{tikzcd}
  \end{equation*}
  The map on the right is an equivalence because $\invfunc:(x=x')\to(x'=x)$ is an equivalence. The bottom map is an equivalence by \cref{lem:truncated-yoneda}. The top map satisfies the universal property of the $k$-truncation of $x=x'$, hence so does the map on the left, which completes the proof.
\end{proof}


\begin{thm}\label{thm:is-truncation}
  Consider a map $f:X\to Y$ into a $(k+1)$-truncated type $Y$. Then the following are equivalent:
  \begin{enumerate}
  \item The map $f$ satisfies the universal property of the $(k+1)$-truncation of $X$.
  \item The map $f$ is surjective, and for each $x,x':X$ the map
    \begin{equation*}
      \apfunc{f} : (x=x')\to(f(x)=f(x'))
    \end{equation*}
    satisfies the universal property of the $k$-truncation of $x=x'$.
  \end{enumerate}
\end{thm}

\begin{proof}
  The fact that (ii) implies (i) was established in \cref{prp:is-truncation}, so it suffices to show that (i) implies (ii).
  
  Suppose first that the map $f$ satisfies the universal property of the $(k+1)$-truncation, and let $x:X$. Recall from \cref{ex:istrunc_UUtrunc} that the universe of $k$-truncated types is itself $(k+1)$-truncated. Therefore it follows that the map $x'\mapsto \trunc{k}{x=x'}$ has a unique extension
  \begin{equation*}
    \begin{tikzcd}
      X \arrow[dr,"{x'\mapsto\trunc{k}{x=x'}}"] \arrow[d,swap,"f"] \\
      Y \arrow[r,densely dotted,swap,"P"] & \UU_{\leq k}.
    \end{tikzcd}
  \end{equation*}
  In other words, we obtain a unique family $P$ of $k$-types over $Y$ equipped with equivalences
  \begin{equation*}
    e_{x'}:P(f(x'))\simeq \trunc{k}{x=x'}
  \end{equation*}
  indexed by $x':X$. In particular, $P$ comes equipped with a point $p_0:P(f(x))$ such that $e_x(p_0)=\tproj{k}{\refl{x}}$. Hence we obtain a family of maps
  \begin{equation*}
    \prd{y:Y} (f(x)=y)\to P(y).
  \end{equation*}
  We claim that this is a family of equivalences. By the fundamental theorem of identity types, \cref{thm:id_fundamental}, it suffices to show that the total space
  \begin{equation*}
    \sm{y:Y}P(y)
  \end{equation*}
  is contractible. We have $(f(x),p_0)$ at the center of contraction, so we have to construct a contraction
  \begin{equation*}
    \prd{y:Y}{p:P(y)} (f(x),p_0)=(y,p).
  \end{equation*}
  Now we observe that the type $\sm{y:Y}P(y)$ is $(k+1)$-truncated, using the fact that any $\Sigma$-type of a family of $k$-types over a $(k+1)$-type is again $(k+1)$-truncated (\cref{ex:istrunc_sigma}). It follows that the type $(f(x),p_0)=(y,p)$ is $k$-truncated for each $y:Y$ and each $p:P(y)$. Now we use the dependent universal property of the $k$-truncation of $X$, which was proven in \cref{thm:trunc_dup}, so it suffices to show that
  \begin{equation*}
    \prd{x':X}{p:P(f(x'))} (f(x),p_0)=(f(x'),p).
  \end{equation*}
  Since we have an equivalence $e_{x'}:P(f(x'))\simeq \trunc{k}{x=x'}$ for each $x':X$, it is equivalent to show that
  \begin{equation*}
    \prd{x':X}{p:\trunc{k}{x=x'}}(f(x),p_0)=(f(x'),e_{x'}^{-1}(p)).
  \end{equation*}
  Again, we use that the type of paths $(f(x),p_0)=(f(x'),e_{x'}^{-1}(p))$ is a $k$-type, so we use \cref{thm:trunc_dup} to conclude that it suffices to show that
  \begin{equation*}
    \prd{x':X}{p:x=x'}(f(x),p_0)=(f(x'),e_{x'}^{-1}(p)).
  \end{equation*}
  This is immediate by path induction on $p:x=x'$. This proves the claim that the canonical map
  \begin{equation*}
    h_y:(f(x)=y)\to P(y)
  \end{equation*}
  is an equivalence for each $y:Y$. Now observe that we have a commuting triangle
  \begin{equation*}
    \begin{tikzcd}[column sep=0]
      \phantom{\trunc{k}{x=x'}} & (x=x') \arrow[dl,swap,"\apfunc{f}"] \arrow[dr,"\tproj{k}{\blank}"] & \phantom{(f(x)=f(x'))} \\
      (f(x)=f(x')) \arrow[rr,swap,"h_{f(x')}"] & & \trunc{k}{x=x'}
    \end{tikzcd}
  \end{equation*}
  for each $x':X$. Therefore it follows that $\apfunc{f}:(x=x')\to(f(x)=f(x'))$ satisfies the universal property of the $k$-truncation of $x=x'$.
\end{proof}

\begin{cor}\label{thm:trunc_id}
For any $x,y:X$, there is an equivalence
\begin{equation*}
\eqv{(\tproj{k+1}{x}=\tproj{k+1}{y})}{\trunc{k}{x=y}}.
\end{equation*}
\end{cor}

\subsection{The truncations as recursive higher inductive types}\label{sec:hubs-and-spokes}

Recall from \cref{thm:trunc_ap} that a map $f:A\to B$ is $(k+1)$-truncated if and only if the action on paths
\begin{equation*}
  \apfunc{f}:(x=y)\to(f(x)=f(y))
\end{equation*}
is a $k$-truncated map, for each $x,y:A$. Moreover, in \cref{ex:trunc_diagonal_map} we established that the fibers of the diagonal map $\delta_f:A\to A\times_BA$ are equivalent to the fibers of the maps $\apfunc{f}$, so it is also the case that $f$ is $(k+1)$-truncated if and only if the diagonal $\delta_f$ is $k$-truncated.

In the following theorem, we add yet another equivalent characterization to the truncatedness of a map. We will use this theorem in two ways. First, a simple corollary gives a useful characterization of $k$-truncated types. Second, we will use this theorem to derive an elimination principle of the $(k+1)$-sphere that can be applied to families of $k$-types

\begin{thm}
  Consider a map $f:A\to B$. Then the following are equivalent:
  \begin{enumerate}
  \item The map $f$ is $k$-truncated.
  \item The commuting square
    \begin{equation*}
      \begin{tikzcd}[column sep=large]
        A \arrow[r,"f"] \arrow[d,swap,"\lam{x}\const_x"] & B \arrow[d,"\lam{y}\const_y"] \\
        A^{\sphere{k+1}} \arrow[r,swap,"f^{\sphere{k+1}}"] & B^{\sphere{k+1}}
      \end{tikzcd}
    \end{equation*}
    is a pullback square.
  \end{enumerate}
\end{thm}

\begin{proof}
  We prove the claim by induction on $k\geq-2$. The base case is clear, because the map $A^{\sphere{-1}}\to B^{\sphere{-1}}$ is a map between contractible types, hence an equivalence. Therefore the square
  \begin{equation*}
    \begin{tikzcd}
        A \arrow[r] \arrow[d] & B \arrow[d] \\
        A^{\sphere{-1}} \arrow[r] & B^{\sphere{-1}}
    \end{tikzcd}
  \end{equation*}
  is a pullback square if and only if $A\to B$ is an equivalence.

  For the inductive step, assume that for any map $g:X\to Y$, the map $g$ is $k$-truncated if and only if the square
  \begin{equation*}
    \begin{tikzcd}
      X \arrow[r] \arrow[d] & Y \arrow[d] \\
      X^{\sphere{k+1}} \arrow[r] & Y^{\sphere{k+1}}
    \end{tikzcd}
  \end{equation*}
  is a pullback square, and consider a map $f:A\to B$. Then $f$ is $(k+1)$-truncated if and only if $\apfunc{f}:(x=y)\to(f(x)=f(y))$ is $k$-truncated for each $x,y:A$. By the inductive hypothesis this happens if and only if the square
  \begin{equation*}
    \begin{tikzcd}
      (x=y) \arrow[r] \arrow[d] & (f(x)=f(y)) \arrow[d] \\
      (x=y)^{\sphere{k+1}} \arrow[r] & (f(x)=f(y))^{\sphere{k+1}}
    \end{tikzcd}
  \end{equation*}
  is a pullback square, for each $x,y:A$. Now we observe that this is the case if and only if the square on the left in the diagram
  \begin{equation*}
    \begin{tikzcd}[column sep=small]
      \sm{x,y:A}x=y \arrow[r] \arrow[d] & \sm{x,y:A}(f(x)=f(y)) \arrow[d] \arrow[r] & \sm{x,y:B}x=y \arrow[d] \\
      \sm{x,y:A}(x=y)^{\sphere{k+1}} \arrow[r] & \sm{x,y:A}(f(x)=f(y))^{\sphere{k+1}} \arrow[r] & \sm{x,y:B}(x=y)^{\sphere{k+1}}
    \end{tikzcd}
  \end{equation*}
  is a pullback square. The square on the right is a pullback square, so the square on the left is a pullback if and only if the outer rectangle is a pullback. By the universal property of $\sphere{k+2}$ it follows that the outer rectangle is a pullback if and only if the square
  \begin{equation*}
    \begin{tikzcd}
      A \arrow[r] \arrow[d] & B \arrow[d] \\
      A^{\sphere{k+2}} \arrow[r] & B^{\sphere{k+2}}
    \end{tikzcd}
  \end{equation*}
  is a pullback.
\end{proof}

\begin{thm}\label{thm:truncated}
  Consider a type $A$. Then the following are equivalent:
  \begin{enumerate}
  \item The type $A$ is $k$-truncated.
  \item The map
    \begin{equation*}
      \lam{x}\const_x:A\to (\sphere{k+1}\to A)
    \end{equation*}
    is an equivalence.
  \end{enumerate}
\end{thm}

\begin{proof}
  We prove the claim by induction on $k\geq-2$. The base case is clear, because the map $A^{\sphere{-1}}$ is contractible.

  For the inductive step, assume that any type $X$ is $k$-truncated if and only if the map
  \begin{equation*}
    \lam{x}\const_x:X\to (\sphere{k+1}\to X)
  \end{equation*}
  is an equivalence.
  Then $A$ is $(k+1)$-truncated if and only if its identity types $x=y$ are $k$-truncated, for each $x,y:A$. By the inductive hypothesis this happens if and only if
  \begin{equation*}
    (x=y)\to (\sphere{k+1}\to (x=y))
  \end{equation*}
  is a family of equivalences indexed by $x,y:A$. This is a family of equivalences if and only if the induced map on total spaces
  \begin{equation*}
    \Big(\sm{x,y:A}x=y\Big)\to\Big(\sm{x,y:A}(x=y)^{\sphere{k+1}}\Big)
  \end{equation*}
  is an equivalence. Note that we have a commuting square
  \begin{equation*}
    \begin{tikzcd}
      A \arrow[r] \arrow[d] & A^{\sphere{k+2}} \arrow[d] \\
      \Big(\sm{x,y:A}x=y\Big) \arrow[r] & \Big(\sm{x,y:A}(x=y)^{\sphere{k+1}}\Big)
    \end{tikzcd}
  \end{equation*}
  in which both vertical maps are equivalences. Therefore the top map is an equivalence if and only if the bottom map is an equivalence, which completes the proof.
\end{proof}

\begin{proof}
  Immediate from the fact that $A$ is $k$-truncated if and only if the map $A\to\unit$ is $k$-truncated.
\end{proof}

\begin{defn}
  Consider a type $X$. A \textbf{$k$-truncation} of $X$ consist of a $k$-type $Y$, and a map $f:X\to Y$ satisfying the \textbf{universal property of $k$-truncation}, that for every $k$-type $Z$ the precomposition map
  \begin{equation*}
    \blank\circ f: (Y\to Z)\to (X\to Z)
  \end{equation*}
  is an equivalence.
\end{defn}

We define $\trunc{k}{X}$ by the `hubs-and-spokes' method, as a higher inductive type. The idea is to force any map $\sphere{k}{X}\to \trunc{k}{X}$ to be homotopic to a constant function by including enough points (the hubs) for the values of these constant functions, and enough paths (the spokes) for the homotopies to these constant functions.

\begin{defn}
  For any type $X$ we define a type $\trunc{k}{X}$ as a higher inductive type, with constructors
  \begin{align*}
    \eta & : X \to \trunc{k}{X}. \\
    \mathsf{hub} & : (\sphere{k+1}\to\trunc{k}{X})\to\trunc{k}{X} \\
    \mathsf{spoke} & : \prd{f:\sphere{k+1}\to\trunc{k}{X}}\prd{t:\sphere{k+1}}f(t)=\mathsf{hub}(f).
  \end{align*}
\end{defn}

\begin{rmk}
  The induction principle for $\trunc{k}{X}$ asserts that for any family $P$ of types over $\trunc{k}{X}$, if we have a dependent function $\alpha : \prd{x:X}P(\eta(x))$ and a dependent function
  \begin{equation*}
    \beta : \prd{f:\sphere{k+1}\to\trunc{k}{X}}\Big(\prd{t:\sphere{k+1}}P(f(t))\Big)\to P(\mathsf{hub}(f))
  \end{equation*}
  equipped with an identification
  \begin{equation*}
    \gamma(f,g,t):\tr_P(\mathsf{spoke}(f,t),g(t))=\beta(f,g),
  \end{equation*}
  for every $f:\sphere{k+1}\to\trunc{k}{X}$, $g:\prd{t:\sphere{k+1}}P(f(t))$, and every $t:\sphere{k+1}$, then we obtain a dependent function
  \begin{equation*}
    h:\prd{x:\trunc{k}{X}}P(\eta(x))
  \end{equation*}
  equipped with an identification $H(x):h(\eta(x))=\alpha(x)$ for any $x:X$.
\end{rmk}

\begin{prp}
  For any type $X$, the type $\trunc{k}{X}$ is $k$-truncated.
\end{prp}

\begin{proof}
  By \cref{thm:truncated} it suffices to show that the map
  \begin{equation*}
    \delta\defeq\lam{x}\const_x:\trunc{k}{X}\to (\sphere{k+1}\to\trunc{k}{X})
  \end{equation*}
  is an equivalence. Note that the inverse of this map is simply the map
  \begin{equation*}
    \mathsf{hub}: (\sphere{k+1}\to\trunc{k}{X})\to \trunc{k}{X},
  \end{equation*}
  which is a section of $\delta$ by the homotopy $\mathsf{spoke}$. Therefore it remains to show that
  \begin{equation*}
    \mathsf{hub}(\const_x)=x.
  \end{equation*}
  for every $x:\trunc{k}{X}$. Note that $\mathsf{spoke}(\const_x,\mathsf{hub}(\const_x))^{-1}$ is such an identification.
\end{proof}

Recall that the $(k+1)$-sphere is $k$-connected in the following sense.

\begin{lem}\label{lem:sphere-connected}
  For any family $P$ of $k$-types over $\sphere{k+1}$, the evaluation map at the base point
  \begin{equation*}
    \ev_\ast : \Big(\prd{t:\sphere{k+1}}P(t)\Big)\to P(\ast)
  \end{equation*}
  is an equivalence.
\end{lem}

\begin{thm}
  For any family $P$ of $k$-types over $\trunc{k}{X}$, the function
  \begin{equation*}
    \blank\circ\eta:\Big(\prd{x:\trunc{k}{X}}P(x)\Big)\to\Big(\prd{x:X}P(\eta(x))\Big)
  \end{equation*}
  is an equivalence.
\end{thm}

\begin{proof}
  We first show that for any family $P$ of $k$-types over $\trunc{k}{X}$, the function
  \begin{equation*}
    \blank\circ\eta:\Big(\prd{x:\trunc{k}{X}}P(x)\Big)\to\Big(\prd{x:X}P(\eta(x))\Big)
  \end{equation*}
  has a section. To see this, we apply the induction principle of $\trunc{k}{X}$. For any function $\alpha:\prd{x:X}P(\eta(x))$ we need to construct a function $h:\prd{x:\trunc{k}{X}}P(x)$ such that $h\circ\eta\htpy\alpha$, so it suffices to show that the $k$-truncatedness of the types in the family $P$ imply the existence of the terms $\beta$ and $\eta$ of the induction principle of $\trunc{k}{X}$. In other words, we need to show that for every $f:\sphere{k+1}\to\trunc{k}{X}$ and every $g:\prd{t:\sphere{k+1}}P(f(t))$ there are
  \begin{align*}
    \beta(f,g) & : P(\mathsf{hub}(f)) \\
    \gamma(f,g) & : \prd{t:\sphere{k+1}}\tr_P(\mathsf{spoke}(f,t),g(t))=\beta(f,g).
  \end{align*}
  Since we have already shown that $\trunc{k}{X}$ is $k$-truncated, it suffices to show the above for $f\defeq\const_x$, for any $x:\trunc{k}{X}$. Now the type of $g$ is just the function type $\sphere{k+1}\to P(x)$, so by the truncatedness of $P(x)$ it suffices to construct
  \begin{align*}
    \beta(\const_x,\const_y) & : P(\mathsf{hub}(\const_x)) \\
    \gamma(\const_x,\const_y) & : \prd{t:\sphere{k+1}}\tr_P(\mathsf{spoke}(\const_x,t),y)=\beta(\const_x,\const_y)
  \end{align*}
  for any $x:X$ and $y:P(x)$. Now we simply define
  \begin{equation*}
    \beta(\const_x,\const_y) \defeq \tr_P(\mathsf{spoke}(\const_x,\ast),y).
  \end{equation*}
  Then it remains to construct an identification
  \begin{equation*}
    \tr_P(\mathsf{spoke}(\const_x,t),y)=\tr_P(\mathsf{spoke}(\const_x,\ast),y)
  \end{equation*}
  for any $t:\sphere{k+1}$, but this follows at once from \cref{lem:sphere-connected}, because the identity types of a $k$-truncated type is again $k$-truncated. This completes the proof that the precomposition function
  \begin{equation*}
    \blank\circ\eta:\Big(\prd{x:\trunc{k}{X}}P(x)\Big)\to\Big(\prd{x:X}P(\eta(x))\Big)
  \end{equation*}
  has a section $s$ for every family $P$ of $k$-types over $\trunc{k}{X}$.

  To show that it is an equivalence, we have to show that $s$ is also a retraction of the precomposition function $\blank\circ\eta$, i.e., we have to show that
  \begin{equation*}
    s(h\circ\eta)= h
  \end{equation*}
  for any $h:\prd{x:\trunc{k}{X}}P(x)$. By function extensionality, it is equivalent to show that
  \begin{equation*}
    \prd{x:\trunc{k}{X}}s(h\circ\eta)(x)=h(x).
  \end{equation*}
  Now we observe that the type $s(h\circ\eta)(x)=h(x)$ is a $k$-type, and therefore we already know that the function
  \begin{equation*}
    \blank\circ\eta:\Big(\prd{x:\trunc{k}{X}}s(h\circ\eta)(x)=h(x)\Big)\to \Big(\prd{x:X}s(h\circ\eta)(\eta(x))=h(\eta(x))\Big)
  \end{equation*}
  has a section. In other words, it suffices to construct a dependent function of type
  \begin{equation*}
    \prd{x:X}s(h\circ\eta)(\eta(x))=h(\eta(x)).
  \end{equation*}
  Here we simply use that $s$ is a section $\blank\circ\eta$, and we are done.
\end{proof}

\begin{cor}\label{cor:k-type-is-reflective-subuniverse}
  For any type $X$, the map $\eta:X\to\trunc{k}{X}$ satisfies the universal property of $k$-truncation. 
\end{cor}

\begin{comment}
\subsection{The join extension and connectivity theorems}

\begin{defn}\label{defn:local}
For a given type $M$, a type $A$ is said to be \define{$M$-null} if the map
\begin{equation*}
\lam{a}{m}a : A \to (M \to A)  
\end{equation*}
is an equivalence.
\end{defn}

In other words, the type $A$ is $M$-null if each $f:M\to A$ has a unique extension along the
map $M\to\unit$, as indicated in the diagram
\begin{equation*}
\begin{tikzcd}
M \arrow[r,"f"] \arrow[d] & A \\
\unit. \arrow[ur,densely dotted]
\end{tikzcd}
\end{equation*}
Note that being $M$-null in the above sense is a proposition, so that the
type of all $M$-null types in $\UU$ is a subuniverse of $\UU$. 

\begin{eg}
By \cref{ex:sphere_null}, a type is $\sphere{n+1}$-null precisely when it is $n$-truncated,
for each $n\geq -2$ (taking the $(-1)$-sphere to be the empty type).
\end{eg}

The notion of $M$-connected type is in a sense dual to the notion of $M$-null types.

\begin{defn}
A type $A$ is said to be \define{$M$-connected} if every $M$-null
type is $A$-null. That is, if for every $M$-null type $B$, the map
\begin{equation*}
\lam{b}{a}{b} : B \to (A \to B)
\end{equation*}
is an equivalence. A map is said to be \define{$M$-connected} if its fibers are $M$-connected.
\end{defn}

Thus in particular, $M$ itself is $M$-connected, and the unit type $\unit$ is $M$-connected for every $M$. 

\begin{defn}
Let $M$ be a type. We say that a type $X$ has the \define{$M$-extension property}
with respect to a map $F:A\to B$, if the map
\begin{equation*}
\lam{g}{a} g(F(a)) : (B\to X)\to (A\to X)
\end{equation*}
is $M$-null. In the case $M\jdeq\unit$, we say that $X$ is \define{$F$-local}.
\end{defn}

\begin{lem}\label{lem:equivalent-extension-problems}
For any three types $A$, $A'$ and $B$, the type $B$ is $(\join{A}{A'})$-null
if and only if for any any $f:A\to B$, the type
\begin{equation*}
\sm{b:B}\prd{a:A}f(a)=b
\end{equation*}
is $A'$-null.
\end{lem}

\begin{proof}
To give $f:A\to B$ and $(f',H):A'\to\sm{b:B}\prd{a:A}f(a)=b$ is equivalent to giving a map $g:\join{A}{A'}\to B$. Concretely, the equivalence is given by substituting in $g:\join{A}{A'}\to B$ the constructors of the join, to obtain $\pairr{g\circ\inl,g\circ\inr,\apfunc{g}\circ\glue}$. 

Now observe that the fiber of precomposing with the unique map $!_{\join{A}{A'}} : \join{A}{A'}\to\unit$ at $g : \join{A}{A'}\to B$, is equivalent to
\begin{equation*}
\sm{b:B}\prd{t:\join{A}{A'}}g(t)=b.
\end{equation*}
Similarly, the fiber of precomposing with the unique map $!_{A'} : A'\to\unit$ at $\pairr{g\circ\inr,\apfunc{g}\circ\glue} : A'\to\sm{b:B}\prd{a:A}f(a)=b$ is equivalent to
\begin{equation*}
\sm{b:B}{h:\prd{a:A}g(\inl(a))=b}\prd{a':A'}\pairr{g(\inr(a')),\apfunc{g}(\glue(a,a'))}=\pairr{b,h}.
\end{equation*}
By the universal property of the join, these types are equivalent.
\end{proof}

\begin{lem}\label{lem:join-null}
Suppose $A$ is an $M$-connected type, and that $B$ is an $(\join{M}{N})$-null type. Then $B$ is $(\join{A}{N})$-null.
\end{lem}

\begin{proof}
Let $B$ be a $(\join{M}{N})$-null type. Our goal of showing that $B$ is
$(\join{A}{N})$-null is equivalent to showing that for any $f:N\to B$, 
the type 
\begin{equation*}
\sm{b:B}\prd{a:A}f(a)=b
\end{equation*}
is $A$-null. 
Since $B$ is assumed to be $(\join{M}{N})$-null, we know that this type is 
$M$-null. Since $A$ is $M$-connected, this type is also $A$-null.
\end{proof}

\begin{lem}\label{lem:N-extension-simple}
Let $A$ be $M$-connected and let $B$ be $(\join{M}{N})$-null. Then the map
\begin{equation*}
\lam{b}{a}b:B\to B^A
\end{equation*}
is $N$-null. 
\end{lem}

\begin{proof}
The fiber of $\lam{b}{a}b$ at a function $f:A\to B$ is equivalent to the type $\sm{b:B}\prd{a:A}f(a)=b$. Therefore, it suffices to show that this type is $N$-null. By \cref{lem:equivalent-extension-problems}, it is equivalent to show that $B$ is $(\join{A}{N})$-null. This is solved in \cref{lem:join-null}.
\end{proof}

\begin{thm}[Join extension theorem]\label{thm:join-extension}
Suppose $f:X\to Y$ is $M$-connected, and let $P:Y\to\UU$ be a family of
$(\join{M}{N})$-null types for some type $N$. Then precomposition by $f$, i.e.
\begin{equation*}
\lam{s}s\circ f : \Big(\prd{y:Y}P(y)\Big)\to\Big(\prd{x:X}P(f(x))\Big),
\end{equation*}
is an $N$-null map.
\end{thm}

\begin{proof}
Let $g:\prd{x:X}P(f(x))$. Then we have the equivalences
\begin{align*}
\hfib{(\blank\circ f)}{g} 
& \eqvsym \sm{s:\prd{y:Y}P(y)}\prd{x:X}s(f(x))=g(x) \\
& \eqvsym \sm{s:\prd{y:Y}P(y)}\prd{y:Y}{(x,p):\hfib{f}{y}} s(y)= \trans{p}{g(x)} \\
& \eqvsym \prd{y:Y}\sm{z:P(y)}\prd{(x,p):\hfib{f}{y}} \trans{p}{g(x)}=z \\
& \eqvsym \prd{y:Y}\hfib{\lam{z}{(x,p)}z}{\lam{(x,p)}\trans{p}{g(x)}}.
\end{align*}
Therefore, it suffices to show for every $y:Y$, that $P(y)$ has the $N$-extension property with respect to the unique map of type $\hfib{f}{y}\to\unit$. This is a special case of \cref{lem:N-extension-simple}.
\end{proof}

\begin{thm}\label{thm:simple-join}
Suppose $X$ is an $M$-connected type and $Y$ is an $N$-connected type. Then $\join{X}{Y}$ is an $(\join{M}{N})$-connected type.
\end{thm}

\begin{proof}
It suffices to show that any $(\join{M}{N})$-null type is $(\join{X}{Y})$-null.
Let $Z$ be an $(\join{M}{N})$-null type.
Since $Z$ is assumed to be $(\join{M}{N})$-null, it follows by \cref{lem:join-null} that $Z$ is $(\join{X}{N})$-null. By symmetry of the join, it also follows that $Z$ is $(\join{X}{Y})$-null.
\end{proof}

\begin{thm}[Join connectivity theorem]\label{thm:join-connectivity}
Consider an $M$-connected map $f:A\to X$ and an $N$-connected map $g:B\to X$. Then $\join{f}{g}$ is $(\join{M}{N})$-connected.
\end{thm}

\begin{proof}
This follows from \cref{thm:simple-join} and \cref{defn:join-fiber}.
\end{proof}

\begin{thm}\label{thm:joinconstruction-connectivity}
Consider the factorization
\begin{equation*}
\begin{tikzcd}
A_n \arrow[dr,swap,"f^{\ast n}"] \arrow[r,"q_n"] & \im(f) \arrow[d] \\
& X
\end{tikzcd}
\end{equation*}
of $f^{\ast n}$ through the image $\im(f)$. 
Then the map $q_n$ is $(n-2)$-connected, for each $n:\N$.
\end{thm}

\begin{proof}
We first show the assertion that, given a commuting diagram of the form
\begin{equation*}
\begin{tikzcd}
A \arrow[r,"q"] \arrow[dr,swap,"f"] & Y \arrow[d,"m"] & A' \arrow[l,swap,"{q'}"] \arrow[dl,"{f'}"] \\
& X
\end{tikzcd}
\end{equation*}
in which $m$ is an embedding, then $\join{f}{f'}=\join{(m\circ q)}{(m\circ q')}=m\circ (\join{q}{q'})$.
In other words, postcomposition with embeddings distributes over 
the join operation.

Note that, since $m$ is assumed to be an embedding, we have an equivalence of
type $\eqv{f(a)=f'(a)}{q(a)=q'(a)}$, for every $a:A$. Hence the pullback of
$f$ and $f'$ is equivalent to the pullback of $q$ along $q'$. Consequently, the
two pushouts
\begin{equation*}
\begin{tikzcd}
A\times_X A' \arrow[r,"\pi_2"] \arrow[d,swap,"\pi_1"] & A' \arrow[d] \\
A \arrow[r] & \join[X]{A}{A'}
\end{tikzcd}
\qquad\text{and}\qquad
\begin{tikzcd}
A\times_Y A' \arrow[r,"\pi_2"] \arrow[d,swap,"\pi_1"] & A' \arrow[d] \\
A \arrow[r] & \join[Y]{A}{A'}
\end{tikzcd}
\end{equation*}
are equivalent. Hence the claim follows.

As a corollary, we get that $q_n=q_f^{\ast n}$. Note that $q_f$ is surjective,
in the sense that $q_f$ is $\bool$-connected, where $\bool$ is the type of booleans%
\footnote{Recall that the $\bool$-null types are precisely the mere propositions.}.
Hence it follows that $q_n$ is $\bool^{\ast n}$-connected. 

Now recall that the $n$-th join power of $\bool$ is the $(n-1)$-sphere $\Sn^{n-1}$, and that
a type is $(\Sn^{n-1})$-connected if and only if it is $(n-2)$-connected.
\end{proof}
\end{comment}

\begin{comment}
\subsection{The construction of the $n$-truncation}\label{sec:truncation}

Our goal in this section is to prove the following theorem. Its proof will take up the entire section.

\begin{thm}\label{thm:truncation}
For every $k\geq -2$, there is a $k$-truncation operation
\begin{equation*}
\trunc{k}{\blank} : \UU\to\UU
\end{equation*}
equipped with a fiberwise transformation
\begin{equation*}
\tproj{k}{\blank}:\prd{X:\UU}X\to\trunc{k}{X},
\end{equation*}
such that for each $X:\UU$ the type $\trunc{k}{X}$ is a $k$-type satisfying the (dependent) universal property of $k$-truncation.
\end{thm}

We will define the $k$-truncation operation by induction on $k\geq-2$,
with the trivial operation as the base case. For $k\geq -2$, suppose we have
a $k$-truncation operation as described in the statement of the theorem.

\cref{thm:trunc_id} suggests that we can think of the type $\trunc{k+1}{X}$ is as the quotient of $X$ modulo the
`$(k+1)$-equivalence relation' given by $\trunc{k}{a=b}$. 

\begin{defn}
We define the reflexive relation $I_k(A) : A \to (A \to \UU)$ by
\begin{equation*}
I_k(A)(a,b) \defeq \trunc{k}{a=b},
\end{equation*}
and then we define
\begin{align*}
\trunc{k+1}{A} & \defeq \im(I_k(A)) \\
\tproj{k+1}{\blank} & \defeq q_{I_k(A)},
\end{align*}
where $q_{I_k(A)}:A\to \im(I_k(A))$ is the map with which the image comes equipped.
\end{defn}

Note that the codomain $(A\to\UU)$ of $I_k(A)$ is locally small since it is the exponent of
the locally small type $\UU$ by a small type $A$. 
Therefore the image of $I_k(A)$ is essentially small by \cref{thm:replacement}.
Since we want the $(k+1)$-truncation to be an operation $\UU\to\UU$, it would be more precise to define $\trunc{k+1}{A}$ as the (unique) type in $\UU$ that is equivalent to $\im(I_k(A))$. Of course, this makes no substantial difference.

\begin{lem}\label{lem:modal_contr}
For every $a,b:A$, we have an equivalence
\begin{equation*}
\eqv{(I_k(A)(a)=I_k(A)(b))}{\trunc{k}{a=b}}.
\end{equation*}
\end{lem}

\begin{proof}
Since $\im(I_k(A))$ is a subtype of $\UU^A$, there is for any $b:A$ a `tautological' family $E_b$ of types over $\im(I_k(A))$, given by
\begin{equation*}
E_b(P) \defeq P(b).
\end{equation*}
Note that $E_b(I_k(A)(a))\jdeq \trunc{k}{a=b}$. Therefore we can prove the claim by showing that the canonical map
\begin{equation*}
\prd{P:\im(I_k(A))} (I_k(A)(b)=P)\to P(b)
\end{equation*}
is a fiberwise equivalence. By \cref{thm:id_fundamental} it suffices to show that for each $b:A$, the total space
\begin{equation*}
\sm{P:\im(I_k(A))}P(b)
\end{equation*}
is contractible. 

For the center of contraction we take the pair
$\pairr{I_k(A)(b),\tproj{k}{\refl{b}}}$.
For the contraction we construct a term of type
\begin{equation*}
\prd{P:\im(I_k(A))}{y:P(b)} \pairr{I_k(A)(b),\tproj{k}{\refl{b}}}=\pairr{P,y}.
\end{equation*}
Since $I_k(A)(b,a)\jdeq\trunc{k}{b=a}$, it is equivalent to construct a term of type
\begin{equation*}
\prd{P:\im(I_k(A))}{y:P(b)}\sm{\alpha:\prd{a:A} \eqv{\trunc{k}{b=a}}{P(a)}} \alpha_b(\tproj{k}{\refl{b}})=y.
\end{equation*}
Let $P:\im(I_k(A))$ and $y:P(b)$. Then $P(a)$ is $n$-truncated for any $a:A$. Therefore, to construct a map
$\alpha(P,y)_a:\trunc{k}{b=a}\to P(a)$, it suffices to construct a map of type $(b=a)\to P(a)$. This may be done by
path induction, using $y:P(b)$. Since it follows that $\alpha(P,y)_b(\tproj{k}{\refl{b}})=y$, it only remains to show that each $\alpha(P,y)_a$ is an equivalence.  

Note that the type of those $P:\im(I_k(A))$ such that for all $y:P(b)$ and all $a:A$ the map $\alpha(P,y)_a$ is an equivalence, is a subtype of $\im(I_k(A))$, we may use the universal property of the image of $I_k(A)$: it suffices to lift
\begin{equation*}
\begin{tikzcd}
& \sm{P:\im(I_k(A))}\prd{y:P(b)}{a:A}\isequiv(\alpha(P,y)_a) \arrow[d] \\
A \arrow[ur,densely dotted] \arrow[r,swap,"I_k(A)"] & \im(I_k(A)).
\end{tikzcd}
\end{equation*}
In other words, it suffices to show that 
\begin{equation*}
\prd{x:A}{y:I_k(A)(x,b)}{a:A}\isequiv(\alpha(I_k(A)(x),y)_a).
\end{equation*}
Thus, we want to show that for any $y:\trunc{k}{x=b}$, the map $\trunc{k}{a=b}\to\trunc{k}{x=b}$ constructed above is an equivalence.
Since the fibers of this map are $n$-truncated, and $\iscontr(X)$ of an $n$-truncated type $X$ is always $n$-truncated, we may assume that $y$ is of the form $\tproj{k}{p}$ for $p:x=b$. 
Now it is easy to see that our map of type $\trunc{k}{b=a}\to\trunc{k}{x=a}$ is the unique map which
extends the path concatenation $\ct{p}{\blank}$, as indicated in the diagram
\begin{equation*}
\begin{tikzcd}[column sep=8em]
(b=a) \arrow[r,"\ct{p}{\blank}"] \arrow[d] & (x=a) \arrow[d] \\
\trunc{k}{b=a} \arrow[r,densely dotted,swap,"{\alpha(I_k(A)(x),y)_a}"] & \trunc{k}{x=a}.
\end{tikzcd}
\end{equation*}
Since the top map is an equivalence, it follows that the map $\alpha(I_k(A)(x),y)_a$ is an equivalence.
\end{proof}

\begin{cor}\label{cor:truncated}
The image $\im(I_k(A))$ is an $(n+1)$-truncated type. 
\end{cor}


\begin{proof}[Construction]
We will show that $\trunc{n+1}{A}$ is indeed $(n+1)$-truncated in \cref{cor:truncated} of \cref{lem:modal_contr} below. Once this fact is established, it remains to verify the dependent universal property of $(n+1)$-truncation.
By the join extension theorem \cref{thm:join-extension} (using $N\defeq \emptyt$), it suffices to show that the map $\tproj{n+1}{\blank}:A\to\trunc{n+1}{A}$ is $\sphere{n+2}$-connected. Note that $\tproj{n+1}{\blank}$ is surjective, so the claim that $\tproj{n+1}{\blank}$ is $\sphere{n+2}$-connected follows from \cref{lem:ap_connectivity}, where we show that for any surjective map $f:A\to X$, if the action on paths is $M$-connected for any two points in $A$, then $f$ is $\susp(M)$-connected. To apply this lemma, we also need to know that $\tproj{k}{\blank}:A\to\trunc{k}{A}$ is $\sphere{n+1}$-connected. This is shown in Corollary 7.5.8 of \cite{hottbook}.
\end{proof}


Before we are able to show that for any surjective map $f:A\to X$, if the action on paths is $M$-connected for any two points in $A$, then $f$ is $\susp(M)$-connected, we show that a type is $\susp(M)$-connected precisely when its identity types are $M$-connected.

\begin{lem}\label{lem:local_id}
Let $M$ be a type. Then a type $X$ is $(\join{\bool}{M})$-null
if and only if all of its identity types are $M$-null. 
\end{lem}

\begin{proof}
The map
\begin{equation*}
\lam{p}{m}p : (x=y)\to (M\to (x=y))
\end{equation*}
is an equivalence if and only if the induced map on total spaces
\begin{equation*}
\lam{\pairr{x,y,p}}\pairr{x,y,\lam{m}p} : \Big(\sm{x,y:X}x=y\Big)\to\Big(\sm{x,y:X}M\to (x=y)\Big)
\end{equation*}
is an equivalence. 
Since the map $\lam{x}\pairr{x,x,\refl{x}}:X\to\sm{x,y:X}x=y$ is an equivalence,
the above map is an equivalence if and only if the map
\begin{equation*}
\lam{x}\pairr{x,x,\lam{m}\refl{x}} : X\to\Big(\sm{x,y:X}M\to (x=y)\Big)
\end{equation*}
is an equivalence. For every $x:X$, the triple $\pairr{x,x,\lam{m}\refl{x}}$
induces a map $\susp(M)\to X$. By uniqueness of the universal property,
it follows that this map is the constant map $\lam{m}x$.
Thus we see that $\lam{x}\pairr{x,x,\lam{m}\refl{x}}$ is an equivalence if
and only if the map
\begin{equation*}
\lam{x}{m}x : X \to (\susp(M)\to X)
\end{equation*}
is an equivalence. 
\end{proof}

\begin{lem}\label{lem:ap_connectivity}
Suppose $f:A\to X$ is a surjective map, with the property that for every
$a,b:A$, the map
\begin{equation*}
\mapfunc{f}(a,b):(a=b)\to (f(a)=f(b))
\end{equation*}
is $M$-connected. Then $f$ is $\susp(M)$-connected. 
\end{lem}

\begin{proof}
We have to show that $\fib{f}{x}$ is $\susp(M)$-connected for each $x:X$. 
Since this is a mere proposition, and we assume that $f$ is surjective, it
is equivalent to show that $\fib{f}{f(a)}$ is $\susp(M)$-connected for each $a:A$. 
Let $Y$ be a $\susp(M)$-null type. 
For every $g:\fib{f}{f(a)}\to Y$ be a map we have the point $\theta(g)\defeq g(a,\refl{f(a)})$ in $Y$,
so we obtain a map
\begin{equation*}
\theta : (\fib{f}{f(a)}\to Y)\to Y
\end{equation*}
It is clear that $\theta(\lam{\pairr{b,p}}y)=y$, so it remains to show that
for every $g:\fib{f}{f(a)}\to Y$ we have $\lam{\pairr{b,p}}\theta(g)=g$.
That is, we must show that
\begin{equation*}
\prd{b:A}{p:f(a)=f(b)} g(a,\refl{f(a)})=g(b,p).
\end{equation*}
Using the assumption that $Y$ is $\susp(M)$-connected, it follows from
\cref{lem:local_id} that the type $g(a,\refl{f(a)})=g(b,p)$ is $M$-connected,
for every $b:A$ and $p:f(a)=f(b)$.
Therefore it follows, since the map $\mapfunc{f}(a,b):(a=b)\to(f(a)=f(b))$ is connected, that our goal is equivalent to
\begin{equation*}
\prd{b:A}{p:a=b} g(a,\refl{f(a)})=g(b,\mapfunc{f}(a,b,p)).
\end{equation*}
This follows by path induction. 
\end{proof}
\end{comment}

\subsection{Theorems not to forget}

\begin{thm}
  Consider a type $X$ and a family $P$ of $(k+n)$-truncated types over $\trunc{k}{X}$. Then the precomposition map
  \begin{equation*}
    \blank\circ\eta : \Big(\prd{y:\trunc{k}{X}}P(y)\Big)\to\Big(\prd{x:X}P(\eta(x))\Big)
  \end{equation*}
  is $(n-2)$-truncated.
\end{thm}

\begin{exercises}
\exercise Consider an equivalence relation $R:A\to (A\to\prop)$. Show that the map $\tproj{0}{\blank}\circ \inl:A\to \trunc{0}{A\sqcup^{R} A}$ satisfies the universal property of the quotient $A/R$, where $A\sqcup^{R} A$ is the canonical pushout
\begin{equation*}
\begin{tikzcd}
\sm{x,y:A}R(x,y) \arrow[r,"\pi_2"] \arrow[d,swap,"\pi_1"] & A \arrow[d,"\inr"] \\
A \arrow[r,swap,"\inl"] & A\sqcup^{R} A.
\end{tikzcd}
\end{equation*}
\exercise Consider the trivial relation $\unit\defeq\lam{x}{y}\unit:A\to (A\to\prop)$. Show that the set quotient $A/\unit$ is a proposition satisfying the universal property of the propositional truncation.
\exercise Show that the type of pointed $2$-element sets
\begin{equation*}
\sm{X:\UU_{\bool}}X
\end{equation*}
is contractible.
\exercise Define the type $\mathbb{F}$ of finite sets by
\begin{equation*}
\mathbb{F}\defeq \im(\fin),
\end{equation*}
where $\fin:\N\to\UU$ is defined in \cref{defn:fin}. 
\begin{subexenum}
\item Show that $\eqv{\mathbb{F}}{\sm{n:\N}\UU_{\fin(n)}}$. 
\item Show that $\mathbb{F}$ is closed under $\Sigma$ and $\Pi$. 
\end{subexenum}
\exercise
\begin{subexenum}
\item A type $Y$ is called \define{$k$-separated} if for every type $X$ the map
  \begin{equation*}
    (\trunc{k}{X}\to Y)\to(X\to Y)
  \end{equation*}
  is an embedding. Show that $Y$ is $k$-separated  if and only if it is $(k+1)$-truncated.
\item A type $Y$ is called \define{$n$-fold $k$-separated} if for every type $X$ the map
  \begin{equation*}
    (\trunc{k}{X}\to Y)\to (X\to Y)
  \end{equation*}
  is $(n-2)$-truncated. Show that $Y$ is $n$-fold $k$-separated if and only if it is $(k+n)$-truncated.
\end{subexenum}
\exercise Consider a map $f:A\to B$. Show that the square
\begin{equation*}
  \begin{tikzcd}[column sep=large]
    A \arrow[r,"f"] \arrow[d,swap,"\lam{x}\const_x"] & B \arrow[d,"\lam{y}\const_y"] \\
    A^{\sphere{k+1}} \arrow[r,swap,"f^{\sphere{k+1}}"] & B^{\sphere{k+1}}
  \end{tikzcd}
\end{equation*}
is a pullback square if and only if its gap map has a section.
\exercise Consider a map $f:X\to Y$ into a $k$-truncated type $Y$. Show that the following are equivalent:
\begin{enumerate}
\item For any family $P$ of $k$-types over $Y$, the precomposition map
  \begin{equation*}
    \blank\circ f : \Big(\prd{y:Y}P(y)\Big)\to\Big(\prd{x:X}P(f(x))\Big)
  \end{equation*}
  is an equivalence.
\item For any family $P$ of $k$-types over $Y$, the precomposition map
  \begin{equation*}
    \blank\circ f : \Big(\prd{y:Y}P(y)\Big)\to\Big(\prd{x:X}P(f(x))\Big)
  \end{equation*}
  has a section.
\end{enumerate}
\exercise Show that for each type $X$, the map
\begin{equation*}
  \trunc{k+1}{X}\to\UU^X
\end{equation*}
given by $y\mapsto\lam{x}(y=\eta'(x))$ is an embedding.
\end{exercises}
