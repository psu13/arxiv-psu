\section{Notions of fibred structure}
\label{sec:nfs}

We now introduce the notions of ``structured morphism'' that our universes will classify.
Let \E, \Ehat, and $\cE$ be as in \cref{sec:2cat}.

\begin{defn}\label{defn:fcos}
  A \textbf{\nfs} on \E is a strict discrete fibration $\phi:\F\to\cE$ with codomain \cE in $\Ehat$ that has small fibers.
\end{defn}

Explicitly, this means that for any morphism $f:X\to Y$ we have a set (possibly empty, but not a proper class) of \textbf{\F-structures} on $f$, and for any pullback square
\[
  \begin{tikzcd}
    X' \ar[d,"f'"'] \ar[r] \ar[dr,phantom, near start,"\lrcorner"] & X \ar[d,"f"]\\
    Y' \ar[r] & Y
  \end{tikzcd}
\]
we have a function from \F-structures on $f$ to \F-structures on $f'$, which is pseudofunctorial in pullback squares.
When $f$ is equipped with a chosen \F-structure we call it an \textbf{\F-algebra}, and if in the above pullback square $f'$ has the \F-structure induced from $f$ we call the square an \textbf{\F-morphism}.

\begin{eg}\label{eg:full-fcos}
  If $\F\into\cE$ is the inclusion of a subfunctor, then \F is just a pullback-stable class of morphisms in \E.
  We call this a \textbf{full} \nfs.
  This includes most previous work on universes, e.g.~\cite{klv:ssetmodel,shulman:elreedy,cisinski:elegant,stenzel:thesis}.

  In particular, when $\F=\cE$ we have the \textbf{trivial} \nfs.
\end{eg}

\begin{eg}\label{eg:pb-fcos}
  If $\F_1$ and $\F_2$ are \nfss, then so is the pullback $\F_1\times_\cE \F_2$.
  An $(\F_1\times_\cE \F_2)$-structure on a morphism is just a pair consisting of an $\F_1$-structure and an $\F_2$-structure.
\end{eg}

\begin{eg}\label{eg:retr-fcos}
  The forgetful morphism $\varpi:\cEp \to \cE$ from \cref{thm:univrep} makes $\cEp$ into a \nfs: a $\cEp$-structure on a morphism is a section of it.
\end{eg}

% \begin{rmk}\label{rmk:mor-classif}
%   Note that $\cEp \to \cE$ is the \textbf{morphism classifier} of \E.
%   Namely, by the bicategorical Yoneda lemma we have $\Ehat(Y,\cE) \simeq \cE(Y)$, where the latter is the core of $\E/Y$.
%   Explicitly, any morphism $f:X\to Y$ arises as the strict pullback % (\cref{thm:dfib-pb})
%   \[
%     \begin{tikzcd}
%       \E(-,X) \ar[d,"{\E(-,f)}"'] \ar[r] \drpullback & \cEp \ar[d]\\
%       \E(-,Y) \ar[r,"\chi_f"'] & \cE
%     \end{tikzcd}
%   \]
%   for an essentially unique morphism $\chi_f$.
% \end{rmk}

\begin{eg}\label{eg:preimage-fcos}
  Let $G:\E_1\to\E_2$ be a pullback-preserving functor and \F a \nfs on $\E_2$.
  Then there is a \textbf{preimage} \nfs $G^{-1}(\F)$ on $\E_1$, where a $G^{-1}(\F)$-structure on $f$ is by definition an \F-structure on $G(f)$.
\end{eg}

\begin{eg}\label{eg:fend-fcos}
  Let $H$ be a \emph{fibred core-endofunctor} of \E, i.e.\ a family of endofunctors $H_Y$ of the core of $\E/Y$ that commute with pullback up to coherent isomorphism.
  Then $H$ induces a map $\cE\to\cE$ in $\Ehat$, and any \nfs $\F\to\cE$ yields a \textbf{pullback} \nfs $H^*(\F)$, where an $H^*(\F)$-structure on a morphism $X\to Y$ is an \F-structure on $H_Y(X) \to Y$.
\end{eg}

\begin{eg}\label{eg:ff-fcos}
  Suppose given a \textbf{functorial factorization} on \E, i.e.\ a functor $E:\E^\dtwo \to \E^\dthree$ sending each morphism $f:X\to Y$ (regarded as an object of $\E^\dtwo$) to a composable pair $X \xto{\lambda_f} E f \xto{\rho_f} Y $ such that $f = \rho_f \lambda_f$.
  The functorial aspect factors every commutative square on the left as a pair of such on the right:
  \begin{equation}\label{eq:commsqfact}
    \begin{tikzcd}
      X' \ar[d,"f'"'] \ar[r,"g"] & X \ar[d,"f"]\\
      Y' \ar[r,"h"] & Y
    \end{tikzcd}
    \qquad = \qquad
    \begin{tikzcd}
      X' \ar[d,"{\lambda_{f'}}"'] \ar[r,"g"] & X \ar[d,"{\lambda_f}"]\\
      E f' \ar[d,"{\rho_{f'}}"'] \ar[r,"{\fact g h}"] & E f \ar[d,"{\rho_f}"]\\
      Y' \ar[r,"h"] & Y.
    \end{tikzcd}
  \end{equation}
  Define an \textbf{$\dR_E$-structure} on $f:X\to Y$ to be a retraction $r_f:E f \to X$ such that $r_f \lambda_f = \id_X$ and $f r_f = \rho_f$, exhibiting $f$ as a retract of $\rho_f$ in $\slice{\E}{Y}$.
  For example, if $E$ underlies a weak factorization system, then the morphisms that admit $\dR_E$-structures are those in the right class (but a given morphism in the right class will generally admit more than one $\dR_E$-structure).

  If the left square in~\eqref{eq:commsqfact} is a pullback and $f$ is an $\dR_E$-algebra, then the pullback universal property gives a unique $r_{f'} : E f' \to X'$ such that $f' r_{f'} = \rho_{f'}$ and $g r_{f'} = r_f \fact g h$; then $r_{f'} \lambda_{f'} = \id_{X'}$ follows by uniqueness.
  This defines the functorial action making $\dR_E$ a \nfs.
\end{eg}

\begin{eg}\label{eg:rep-fcos}
  For any morphism $\pi:\Util\to U$ in \E, its classifying map $\E(-,U) \to \cE$ %as in \cref{rmk:mor-classif}
  is faithful, hence equivalent to a strict discrete fibration $\dRep_\pi \to U$, which
  we call its \textbf{represented} \nfs.
  A $\dRep_\pi$-structure on $f:X\to Y$ is a morphism $Y\to U$ exhibiting $f$ as a pullback of \pi.
\end{eg}

It turns out that the property of being ``local on the base'', which we need to construct universes, is captured exactly by representability.
(Representable morphisms also have other uses in modeling type theory; see~\cite{awodey:natmodels} and \cref{sec:coherence}.)

\begin{defn}\label{defn:local}
  % A strict discrete fibration $\phi:\dW\to\dY$ is \textbf{representable} if for any map $\E(-,Z) \to \dY$ and strict pullback
  % \begin{equation}
  %   \begin{tikzcd}
  %     \dP \ar[r] \ar[d,"{\phi^\F_X}"'] \ar[dr,phantom,near start,"\lrcorner"] & \dW\ar[d,"{\phi}"] \\
  %     \E(-,Z) \ar[r,"X"'] & \dY
  %   \end{tikzcd}\label{eq:local-pb}
  % \end{equation}
  % the object \dP is representable, i.e.\ isomorphic to $\E(-,X)$ for some $X$.
  A \nfs \F is \textbf{\local}
  % \footnote{I struggled to find a non-confusing name for this property.
  %   The definition suggests ``representable'', but this could be confused with \cref{eg:rep-fcos}.
  %   And \cref{thm:local} suggests ``local'' (by analogy to e.g.~\cite[Definition 6.1.3.8]{lurie:higher-topoi}), but this says nothing about the fiberwise smallness condition; plus in \cref{sec:lex-loc} we will also be talking about ``local objects'' as in Bousfield localization.} 
  if the strict discrete fibration $\phi:\F\to\cE$ is representable (\cref{defn:rep}).
\end{defn}

% (The usual definition of representable morphism in a 2-category of pseudofunctors involves weak bicategorical pullbacks and representations up to equivalence, but for strict discrete fibrations this is equivalent to the stricter condition above.
% Since every representable morphisms in the weaker sense is faithful, it is equivalent to a strict discrete fibration, so there is no essential loss of generality in our definition.)

Explicitly, this means that given any map $X\to Z$, there is a map $\phi^\F_X : \F_X \to Z$ in \E such that for any $g:Y\to Z$, there is a natural bijection between \F-structures on $g^*X$ and lifts of $g$ to $\F_X$.

\begin{eg}
  Representable morphisms are closed under pullback and composition in \Ehat.
  Thus, if $\F_1$ and $\F_2$ are \local so is $\F_1\times_\cE \F_2$; and if \F is \local so is $H^*(\F)$ for any fibred core-endofunctor $H$.
\end{eg}

\begin{eg}
  If $G:\E_1\to\E_2$ preserves pullbacks and has a right adjoint $H$, and $\F$ is a \local \nfs on $\E_2$, then $G^{-1}(\F)$ is a \local \nfs on $\E_1$.
  For given $X\to Z$ in $\E_1$ and $g:Y\to Z$, to give $g^*X$ a $G^{-1}(\F)$-structure is equivalently (since $G$ preserves pullbacks) to give $(G g)^*(G X)$ an \F-structure, i.e.\ to lift $G g : G Y \to G Z$ to the representing object $\F_{G X}$; but this is the same as to lift $g$ to the pullback
  \[
    \begin{tikzcd}
      (G^{-1}(\F))_X \ar[r] \ar[d]\drpullback & H (\F_{G X}) \ar[d]\\
      Z \ar[r] & H G Z.
    \end{tikzcd}
  \]
\end{eg}

\begin{eg}\label{thm:ep-local}
  By \cref{thm:univrep}, the \nfs $\cEp$ is \local.
  % since if $\E(-,Z) \to \cE$ classifies a morphism $X\to Z$, then the strict pullback of $\cEp$ to $\E(-,Z)$ is represented by $X$.
  Moreover, every \local \nfs arises as $H^*(\cEp)$ for some fibred core-endofunctor $H$, namely $H(X) = \F_X$ in the above notation.
\end{eg}

\begin{eg}\label{eg:cart-ff-local}
  Let \E be locally cartesian closed, and call a functorial factorization \textbf{cartesian} if it preserves pullback squares, in that if the left-hand square in~\eqref{eq:commsqfact} is a pullback, so is the lower right-hand square (hence so also is the upper one).
  % The factorizations in a model category are not usually cartesian.
  %Our main example will appear in \cref{sec:universe}; for now note that any pullback-stable \emph{unique} (a.k.a.\ orthogonal) factorization system (such as (epi, mono) in a topos) yields a cartesian functorial factorization.
  Under these assumptions, the \nfs $\dR_E$ from \cref{eg:ff-fcos} is \local.
  To see this, given $f:X\to Z$ with factorization $X\xto{\lambda_f} E f\xto{\rho_f} Z$, using the locally cartesian closed structure we can build an object $\Retr_Z(\lambda_f)$ of $\E/Z$ such that for any $g:Y\to Z$, lifts of $g$ to $\Retr_Z(\lambda_f)$ are naturally bijective with retractions of $g^*(\lambda_f)$ in $\E/Y$.
  Since $E$ is cartesian, $g^*(\lambda_f) = \lambda_{g^*(f)}$, so these are $\dR_E$-structures on $g^*X \to Y$, i.e.\ $\Retr_Z(\lambda_f)$ is the desired representing object.
  % To see this, fix the notation as in \cref{thm:local}; then cartesianness and pullback-stability of colimits imply $E f \cong \colim_i E f_i}$ with coprojections $\fact {p_i}{q_i}$.
  % Thus, given compatible $\dR_E$-algebra structures $r_i : E f_i \to X_i$, we have
  % \[ p_{i} r_{i} \fact{p_{j,i}}{q_{j,i}} = p_{i} p_{j,i} r_{j} = p_{j} r_{j}.
  % \]
  % Thus, there is a unique $r : E f \to X$ such that $r \fact {p_i}{q_i} = p_i r_i$.
  % Since
  % \[ f r \fact{p_i}{q_i} = f p_i r_i = q_i f_i r_i = q_i \rho_{f_i} = \rho_f \fact{p_i}{q_i},\]
  % uniqueness gives $f r = \rho_f$, and similarly since
  % \[ r \lambda_f p_i = r \fact{p_i}{q_i} \lambda_{f_i} = p_i r_i \lambda_{f_i} = p_i, \]
  % uniqueness of the colimit $X = \colim X_i$ gives $r \lambda_f = \id_X$.
  % Thus $r$ is an $\dR_E$-structure, and the unique one such that $p_i,q_i$ are an $\dR_E$-morphism.
\end{eg}

\begin{eg}\label{eg:rep-local}
  If \E is locally cartesian closed, then the represented \nfs (\cref{eg:rep-fcos}) determined by any map $\pi:\Util\to U$ is \local.
  To see this, let $\E(-,Z) \to \cE$ classify a map $f:X\to Z$.
  Using local cartesian closure, there is an object $\Iso(X,\Util)$ of $\E/(Z\times U)$ such that for any $g:Y\to Z$ and $h:Y\to U$, lifts of $(g,h)$ to $\Iso(X,\Util)$ are naturally bijective with isomorphisms $g^*X \cong h^*\Util$.
  Put differently, for any $g:Y\to Z$, lifts of $g$ along the composite $\Iso(X,\Util) \to Z\times U \to Z$ are naturally bijective with pullback squares
  \[
    \begin{tikzcd}
      g^* X \ar[d]\ar[r] \drpullback & \Util \ar[d] \\
      Y \ar[r] & U.
    \end{tikzcd}
  \]
  But this says exactly that $\Iso(X,\Util)$ is the desired representing object $(\dRep_\pi)_X$.
  % which is to say that $\E(-,\Iso(X,\Util))$ is the pullback $\E(-,U) \times_\cE \E(-,Z)$.
\end{eg}


\begin{eg}\label{eg:pshf-can}
  Let \F be the full \nfs on a pullback-stable class (\cref{eg:full-fcos}) in a presheaf category $\E=\prcs$.
  Then \F is \local if and only if a morphism $f:X\to Y$ belongs to \F as soon as all its pullbacks to representables $\C(-,c)$ do.
  (Thus our \locality includes the \emph{strongly proper} classes of fibrations from~\cite[Definition 3.7]{cisinski:elegant}.)

  For ``only if'', fullness of \F means that $\F_X\to Z$ is a monomorphism for any $f:X\to Z$.
  But if all pullbacks of $f$ to representables are in \F, all maps from representables into $Z$ factor through $\F_X$; thus $\F_X\cong Z$, hence $f\in \F$.

  For ``if'', given $f:X\to Z$, let $\F_X$ be the sub-presheaf of $Z$ containing all $z\in Z(c)$ such that the pullback of $f$ along $z:\C(-,c)\to Z$ is in \F.
  Then $g:Y\to Z$ factors through $\F_X$ if and only if the pullback of $g^*X$ along all $y:\C(-,c) \to Y$ lies in \F, which by assumption is the same as $g^*X \in \F$.
  %In particular, the trivial \nfs is always \local.
\end{eg}

\begin{eg}\label{eg:rep-cod}
  Let $\cI$ be a set of morphisms in $\prcs$ with representable codomains.
  Then by \cref{eg:pshf-can}, the class of morphisms with the right lifting property with respect to \cI is \local, since any lifting problem against $i\in \cI$ can be solved by first pulling back to the codomain of $i$, which is representable.
  For instance, % since the horn inclusions $\Lambda^n_k \to \Delta^n$ in simplicial sets have representable codomains,
  the class of \emph{Kan fibrations} in simplicial sets is \local.

  Another way to prove \locality in presheaf categories can be found in~\cite[Theorem 3.14]{cisinski:elegant}.
\end{eg}

% \subsection{2-categorical lifting and orthogonality}
% \label{sec:lift-orth}

Since \locality is a representability property, we expect that in good cases it can be ensured by an adjoint functor theorem.

\begin{prop}\label{thm:local}
  Let $\phi:\F\to\cE$ be a \nfs on a locally presentable category \E, and let $\sQ$ denote the class of morphisms $q:\Yhat \to \E(-,Y)$ from \cref{eg:colim-orth} for all small colimits $Y = \colim_i Y_i$ in \E.
  Then the following are equivalent.
  \begin{enumerate}
  \item \F is \local.\label{item:local1}
  \item $\sQ \perp \phi$.\label{item:local2}
  \item For any morphism $f:X\to Y$ and small colimit $Y \cong \colim_i Y_i$ with coprojections $q_i : Y_i \to Y$ and pullbacks\label{item:local3}
  \begin{equation}
    \begin{tikzcd}
      X_{j} \ar[r,"{p_{j,i}}"] \ar[d,"{f_{j}}"'] \drpullback &
      X_i \ar[d,"{f_i}"'] \ar[r,"{p_i}"] \drpullback & X \ar[d,"f"]\\
      Y_{j} \ar[r,"{q_{j,i}}"'] &
      Y_i \ar[r,"{q_i}"'] & Y,
    \end{tikzcd}\label{eq:colim-to-pb}
  \end{equation}
  if each morphism $f_i: X_i \to Y_i$ is given an \F-structure such that the squares on the left in~\eqref{eq:colim-to-pb} are \F-morphisms, then $f$ has a unique \F-structure such that the squares on the right in~\eqref{eq:colim-to-pb} are \F-morphisms.
  \end{enumerate}
\end{prop}
\begin{proof}
  Recall that \locality of \F means that in any pullback
  \begin{equation}
    \begin{tikzcd}
      \dP \ar[r] \ar[d,"{\phi^\F_X}"'] \ar[dr,phantom,near start,"\lrcorner"] & \F\ar[d,"{\phi}"] \\
      \E(-,Z) \ar[r,"X"'] & \cE.
    \end{tikzcd}\label{eq:local-pb-2}
  \end{equation}
  the object \dP is representable.
  Since $\phi$ is a small discrete fibration, so is $\phi^\F_X$; and since $\E(-,Z)$ is a small discrete object, \dP is also small and discrete, i.e.\ a presheaf $\E\op\to\nSet$.
  Thus, by the adjoint functor theorem, \dP is representable if and only if preserves small colimits, i.e.\  if and only if $\sQ \perp \dP$.
  And as noted in \cref{eg:colim-orth} representables preserve colimits, so $\sQ\perp \E(-,Z)$.
  Thus by \cref{eg:orth-cancel}, $\sQ \perp \dP$ if and only if $\sQ \perp \phi^\F_X$.

  Since right orthogonality is preserved by pullback, this is implied by~\ref{item:local2}.
  But the converse also holds, since each morphism in \sQ has a representable codomain, so that any lifting problem relating it to $\phi$ factors through some $\phi^\F_X$.
  Finally,~\ref{item:local3} is just an unraveling of~\ref{item:local2} according to \cref{thm:dfib-perp}\ref{item:dp4}.
\end{proof}

\begin{verbose}
It may be helpful to unravel the ``only if'' direction of \cref{thm:local} more explicitly.
If \F is \local, then given $f:X\to Y$ and $Y \cong \colim_i Y_i$ as in \cref{thm:local}, let $\F_X$ be the classifying object for \F-structures on $f$.
Then each given \F-structure on $f_i$ induces a map $Y_i \to \F_X$, and the left-hand squares in~\eqref{eq:colim-to-pb} being \F-morphisms mean that these maps assemble into a cone under the diagram $\{Y_i\}$.
Thus they induce a unique morphism $Y\to \F_X$, corresponding to a unique \F-structure on $f$ making the right-hand squares in~\eqref{eq:colim-to-pb} \F-morphisms.
\end{verbose}

\begin{rmk}
  For full \nfss on presheaf categories, the equivalence between the conditions of \cref{thm:local,eg:pshf-can} was observed in~\cite[Remark 4.4]{sattler:eqvext}.
  More generally, by \cref{thm:local} our \locality includes the \emph{locality} condition of~\cite[(A.2)]{sattler:eqvext}.
\end{rmk}

%%% Local Variables:
%%% mode: latex
%%% TeX-master: "univinj"
%%% End:
