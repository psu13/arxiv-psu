\section{Left exact localizations}
\label{sec:lex-loc}

For a fibration $f:A\fib B$ in \E between fibrant objects, by its \textbf{fibrant diagonal} we mean a replacement of its strict diagonal $A\to A\times_B A$ by a fibration $\diag f : A' \fib A\times_B A$.
Strictly speaking this depends on the choice of fibrant replacement, but since the weak equivalence $A\to A'$ is between fibrant (and cofibrant) objects, it is a simplicial homotopy equivalence.
Thus the ambiguity is irrelevant for localization, in that whether an object is $\diag f$-local is independent of the choice of $\diag f$.

We write $\diag^n f$ for the $n$-fold iterate of $\diag$, with $\diag^0 f = f$ and $\diag^{n+1} f = \diag(\diag^n f)$.
And we write $S\dia$ for the class of iterated fibrant diagonals of morphisms in $S$:
\[ S\dia = \setof{ \diag^n f | f\in S,\; n\in \dN}.\]
By the remarks above, $X^*(S\dia)$-locality coincides with $(X^*S)\dia$-locality.
And when $S$ is a set, so is $S\dia$, so we can also localize at $(S\dia)\pb$.

We say that \textbf{$S$-localization is left exact} if it preserves homotopy pullbacks.
This implies %(and, indeed, is equivalent to saying)
that $S$-local equivalences are stable under pullback along fibrations; hence in particular a left exact localization is again right proper.
We will rely on the following characterization of left exact localizations from~\cite{abfj:lexloc}:

\begin{thm}[\cite{abfj:lexloc}]\label{thm:abfj}
  For any set $S$ of fibrations between fibrant objects:
  \begin{enumerate}
  \item $(S\dia)\pb$-localization is left exact.
  \item If $S$-localization is left exact, then it coincides with $(S\dia)\pb$-localization.
  \end{enumerate}
  In particular, every accessible left exact localization of \E is a $(S\dia)\pb$-localization.
\end{thm}

Since the $(Y^*(S\dia))\pb$-local objects of $\E/Y$ underlie a good \nfs by \cref{thm:flf-nfs}, it remains only to show that these coincide with the $(S\dia)\pb$-local fibrations over $Y$.
The idea of this proof is that the families of $(Y^*(S\dia))\pb$-localizations form an \emph{accessible lex modality} in the sense of~\cite{rss:modalities}, and the \io-categorical stable factorization system induced by any lex modality is automatically a \emph{reflective factorization system} in the sense of~\cite{chk:reflocfact}, hence coincides with the (acyclic cofibration, fibration) factorization system of the localized model structure.
%However, rather than rely on the internal type-theoretic reasoning of~\cite{rss:modalities} we will prove all the facts we need explicitly using model-categorical language.
We use~\cref{thm:abfj} to bridge a gap that was left open in~\cite{rss:modalities}, by relating the internal and external notions of accessibility for lex modalities.

\begin{lem}[{cf.~\cite[Theorem 3.1(vii)]{rss:modalities}}]\label{thm:conn-modal-pb}
  Let $S$ be a set of fibrations between fibrant objects, and
  suppose given a commutative square
  \[
    \begin{tikzcd}
      A \ar[r] \ar[d,two heads] & X \ar[d,two heads] \\
      B \ar[r] & Y
    \end{tikzcd}
  \]
  such that
  \begin{itemize}
  \item $X\to Y$ is $(Y^*(S\dia))\pb$-local in $\E/Y$.
  \item $A\to B$ is $(B^*(S\dia))\pb$-local in $\E/B$.
  \item $B\to Y$ is a $(Y^*(S\dia))\pb$-local equivalence in $\E/Y$.
  \item $A\to X$ is an $(X^*(S\dia))\pb$-local equivalence in $\E/B$.
  \end{itemize}
  Then the square is a homotopy pullback.
\end{lem}
\begin{proof}
  First factor the square in the Reedy fashion:
  \[
    \begin{tikzcd}
      A \ar[r,"\sim"] \ar[d,two heads] & A' \ar[dr,two heads] \ar[r,two heads] &
      P \ar[d,two heads] \ar[r] \drpullback & X \ar[d,two heads] \\
      B \ar[rr,"\sim"'] & & B' \ar[r,two heads] & Y
    \end{tikzcd}
  \]
  Since local equivalences are invariant under weak equivalence, as is locality by \cref{thm:fl-we}, if we replace $A$ and $B$ by $A'$ and $B'$ the hypotheses still hold.
  Thus, we may assume that the maps $B\to Y$ and $A\to P$, hence also $A\to X$, are fibrations.

  Now consider the following $3\times 3$ square in $\E/P$:
  \[
    \begin{tikzcd}
      A \ar[r] \ar[d] & P\times_B A \ar[d] \ar[r] & P\times_{Y} X \ar[d] \\
      P \times_X A \ar[r] \ar[d] & P\times A \ar[d] \ar[r] & P\times X \ar[d]\\
      P\times_Y B \ar[r] & P\times B \ar[r] & P\times Y
    \end{tikzcd}
  \]
  Here the rows and columns are defined by the following pullbacks:
  \[
    \begin{tikzcd}
      P\times_Y B \ar[r] \ar[d,two heads] \drpullback &
      P\times B \ar[d,two heads] \ar[r] \drpullback & B \ar[d,two heads]\\
      P \ar[r] & P\times Y \ar[r] & Y
    \end{tikzcd}
    \qquad
    \begin{tikzcd}
      P\times_X A \ar[r] \ar[d,two heads] \drpullback &
      P\times A \ar[d,two heads] \ar[r] \drpullback & A \ar[d,two heads]\\
      P \ar[r] & P\times X \ar[r] & X
    \end{tikzcd}
  \]
  \[
    \begin{tikzcd}
      P\times_Y X \ar[r] \ar[d,two heads] \drpullback &
      P\times X \ar[d,two heads] \ar[r] \drpullback & X \ar[d,two heads]\\
      P \ar[r] & P\times Y \ar[r] & Y
    \end{tikzcd}
    \qquad
    \begin{tikzcd}
      P\times_B A \ar[r] \ar[d,two heads] \drpullback &
      P\times A \ar[d,two heads] \ar[r] \drpullback & A \ar[d,two heads]\\
      P \ar[r] & P\times B \ar[r] & B
    \end{tikzcd}
  \]
  Since the right-hand maps in these rectangles are fibrations, by \cref{thm:pb-fl-fle} and the assumptions we have that
  \begin{itemize}
  \item $P\times_Y X\to P$ is $(P^*(S\dia))\pb$-local in $\E/P$.
  \item $P\times_B A\to P$ is $(P^*(S\dia))\pb$-local in $\E/P$.
  \item $P\times_Y B\to P$ is a $(P^*(S\dia))\pb$-local equivalence.
  \item $P\times_X A \to P$ is a $(P^*(S\dia))\pb$-local equivalence.
  \end{itemize}
  However, we also have pullbacks
  \[
    \begin{tikzcd}
      A \ar[d,two heads] \ar[r] \drpullback &
      P\times_B A \ar[d,two heads] \ar[rr] \drpullback && A \ar[d,two heads] \\
      P \ar[r] & P\times_Y X \ar[r] & B\times_Y X \ar[r,equals] & P
    \end{tikzcd}
  \]
  \[
    \begin{tikzcd}
      A \ar[d,two heads] \ar[r] \drpullback &
      P\times_X A \ar[d,two heads] \ar[rr] \drpullback && A \ar[d,two heads] \\
      P \ar[r] & P\times_Y B \ar[r] & X\times_Y B \ar[r,equals] & P
    \end{tikzcd}
  \]
  exhibiting $A$ as the simultaneous fiber in $\E/P$ of the fibrations $P\times_B A \fib P\times_Y X$ and $P\times_X A \fib P\times_Y B$.
  The former is a fibration between $(P^*(S\dia))\pb$-local objects, hence a $(P^*(S\dia))\pb$-local fibration, so that $A$ is also $(P^*(S\dia))\pb$-local.
  And the latter is a fibration between $(P^*(S\dia))\pb$-acyclic objects;
  % (equivalent to the terminal object in the $(P^*(S\dia))\pb$-local model structure on $\E/P$)
  thus since $(P^*(S\dia))\pb$-localization is left exact by \cref{thm:abfj} (and the fact that it coincides with $((P^*S)\dia)\pb$-localization), its fiber is also $(P^*(S\dia))\pb$-acyclic.
  So $A$ is both $(P^*(S\dia))\pb$-local and $(P^*(S\dia))\pb$-acyclic, hence the map $A\to P$ is a weak equivalence in \E, i.e.\ the given square is a homotopy pullback.
\end{proof}

\begin{prop}\label{thm:locfib}
  Let $S$ be a set of fibrations between fibrant objects and $p:Z\fib Y$ a fibration; the following are equivalent.
  \begin{enumerate}
  \item $p:Z\fib Y$ is a fibration in the $(S\dia)\pb$-local model structure on \E.\label{item:lf1}
  \item $p:Z\fib Y$ is $(Y^*(S\dia))\pb$-local in $\E/Y$.\label{item:lf2}
  \end{enumerate}
\end{prop}
\begin{proof}
  Since the $(S\dia)\pb$-local model structure is right proper,~\cite[Proposition 3.4.8]{hirschhorn:modelcats} tells us that~\ref{item:lf1} is equivalent to the $(S\dia)\pb$-localization square:
  \begin{equation}
    \begin{tikzcd}
      Z \ar[d,two heads,"f"'] \ar[r] & \Zhat \ar[d,two heads,"\fhat"]\\
      Y \ar[r] & \Yhat
    \end{tikzcd}\label{eq:locfib}
  \end{equation}
  being a homotopy pullback.

  Thus, if we assume~\ref{item:lf1}, then since pullback takes $(\Yhat^*(S\dia))\pb$-local objects to $(Y^*(S\dia))\pb$-local ones by \cref{thm:pb-fl-fle}, and locality is preserved by weak equivalences of fibrant objects, it will suffice to show that $\Zhat$ is $(\Yhat^*(S\dia))\pb$-local in $\E/\Yhat$.
  Now by assumption $\Zhat$ and $\Yhat$ are $S\pb$-local objects, so $\fhat$ is a fibration in the $S\pb$-local model structure $\E_{S\pb}$, hence $\Zhat$ is a fibrant object of the slice model structure $\E_{S\pb}/\Yhat$.
  On the other hand, any morphism $f:A\to B$ in $(\Yhat^*(S\dia))\pb$ is a pullback (in \E) of a morphism in $S\dia$, hence lies in $(S\dia)\pb$.
  Thus in particular it is an $(S\dia)\pb$-local equivalence in $\E$, hence a weak equivalence in $\E_{S\pb}/\Yhat$, and so the induced map
  \( \ehom{\E/\Yhat}(B,\Zhat) \to \ehom{\E/\Yhat}(A,\Zhat) \)
  is a weak equivalence for any such $f$; hence~\ref{item:lf2} holds.

  Conversely, suppose~\ref{item:lf2}.
  Note that $\fhat$, being a fibration between $(S\dia)\pb$-local objects, is a fibration in the $(S\dia)\pb$-local model structure; so by what we just proved, it is $(\Yhat(S\dia))\pb$-local in $\E/\Yhat$.
  On the other hand, by \cref{thm:unit-connected}, $Y\to \Yhat$ is a $(\Yhat^*(S\dia))\pb$-local equivalence and $Z\to \Zhat$ is a $(\Zhat^*(S\dia))\pb$-local equivalence.
  So by \cref{thm:conn-modal-pb}, the square~\eqref{eq:locfib} is a homotopy pullback, i.e.~\ref{item:lf1} holds.
\end{proof}

\begin{thm}\label{thm:lexloc-ttmt}
  For any set $S$ of morphisms in a \ttmt \E such that $S$-localization is left exact, the localized model structure $\E_S$ is again a \ttmt.
\end{thm}
\begin{proof}
  Since $\E_S$ has the same underlying category and cofibrations as \E, it is a \slcc Grothendieck 1-topos with cofibrations being the monomorphisms.
  It is combinatorial and simplicial by the usual constructions (e.g.~\cite[Theorem 4.1.1]{hirschhorn:modelcats}), and right proper since the localization is left exact.

  Let $S'$ be the result of replacing each morphism of $S$ by a weakly equivalent fibration between fibrant objects.
  Then $\E_S = \E_{S'}$, and by \cref{thm:abfj} $\E_{S'} = \E_{(S'{}\dia)\pb}$.
  Let $\dL_{S'{}\dia}$ be the \local and \stratified \nfs from \cref{thm:flf-nfs} applied to $S'{}\dia$;
  then by \cref{thm:locfib}, the morphisms admitting $\dL_{S'{}\dia}$-structure are precisely the fibrations of $\E_S$.
\end{proof}

% Local Variables:
% TeX-master: "univinj"
% End:
