\section{Relatively \ka-presentable morphisms}
\label{sec:relpres}

For size reasons, we cannot expect a universe to classify \emph{all} fibrations, only those with ``bounded cardinality''.
In~\cite{klv:ssetmodel,shulman:elreedy,cisinski:elegant} such a bound was imposed by explicit reference to the underlying sets of presheaves.
We will work more abstractly, and thus more generally, with ``internal'' categorical notions of size.

In this section \E will be a locally presentable category (often locally cartesian closed), and $\ka,\la,\mu,\nu$ will be regular cardinals.
Recall that $X\in\E$ is \textbf{\ka-presentable} (also called \textbf{\ka-compact}) if $\E(X,-):\E\to\nSet$ preserves \ka-filtered colimits.
A category \sC is \textbf{\ka-small} if $\ka>\card\C$ (the cardinality of the set of arrows of \C).

\begin{eg}\label{eg:pshf-pres}
  By~\cite[Example 1.31]{ar:loc-pres}, if $\E=\prcs$ is a presheaf category where \C is \ka-small, then $X\in\E$ is \ka-presentable if and only if it is a \ka-small colimit of representables, if and only if $\sum_c \card{X_c} < \ka$, and if and only if each $\card{X_c}<\ka$.
  (The hypothesis that \C is \ka-small is essential, however.)
  In~\cite{klv:ssetmodel,shulman:elreedy} these are called \textbf{\ka-small} objects.
\end{eg}

More generally, we have the following.

\begin{lem}[{\cite[Proposition 2.23]{low:univ-ct}}]\label{thm:pw-pres}
  If \E is locally \ka-presentable and \C is a \ka-small category, then the functor category $\func\C\E$ is locally \ka-presentable, and an object of $\func\C\E$ is \ka-presentable if and only if it is pointwise \ka-presentable in \E.
\end{lem}
\begin{proof}
  The first statement is~\cite[Corollary 1.54]{ar:loc-pres}.

  For the ``only if'' direction of the second, note that for any $c_0\in \C$, the ``evaluation at $c_0$'' functor $\mathrm{ev}_{c_0} : \func\C\E \to \E$ has a right adjoint given by $X \mapsto \{ X^{\C(c,c_0)} \}_{c\in \C}$.
  Since $\C(c,c_0)$-fold powers are \ka-small limits (as \C is \ka-small), they commute with \ka-filtered colimits, so this right adjoint is \ka-accessible.
  Hence its left adjoint $\mathrm{ev}_{c_0}$ preserves \ka-presentable objects.

  For the ``if'' direction, note that each $\mathrm{ev}_{c_0}$ also has a left adjoint given by $X \mapsto \{ \C(c_0,c) \cdot X \}_{c\in \C}$.
  Since $\mathrm{ev}_{c_0}$ preserves all colimits, this left adjoint preserves \ka-presentable objects.
  Now we note that any $A\in \func\C\E$ can be written as $A = \int^{c_0} \C(c_0,-) \cdot A_{c_0}$.
  Hence if $A$ is pointwise \ka-presentable, it is a \ka-small colimit of \ka-presentable objects in $\func\C\E$, hence \ka-presentable.
\end{proof}

We might hope to construct a universe of \ka-presentable objects for all sufficiently large regular cardinals \ka.
However, like other facts about locally presentable categories, this seems to only be possible in general when \ka has a certain ``large cofinality'' property.
For the reader's convenience we recall the basic characterizations of that property, and also add a new one that appears not to be in the literature.

\begin{prop}\label{thm:shrp}
  For regular cardinals $\la<\mu$, the following are equivalent.
  \begin{enumerate}
  \item Every \la-accessible category is \mu-accessible.\label{item:shrp1}
  \item For any set $X$ with $\card X <\mu$, the poset $P_\la(X)$ of subsets of $X$ of cardinality $<\la$ has a \mu-small cofinal subset.\label{item:shrp2}
  % \item Every \mu-presentable object of a \la-accessible category is a \mu-small \la-filtered colimit of \la-presentable objects.\label{item:shrp3}
  \item Every \mu-presentable object of a locally \la-presentable category is a \mu-small \la-filtered colimit of \la-presentable objects.\label{item:shrp3a}
  \end{enumerate}
\end{prop}
\begin{proof}
  \ref{item:shrp1}$\Leftrightarrow$\ref{item:shrp2} is~\cite[Theorem 2.11]{ar:loc-pres} and~\cite[\sect 2.3]{mp:accessible}, and~\ref{item:shrp2}$\Rightarrow$\ref{item:shrp3a} is~\cite[Proposition 2.3.11]{mp:accessible}. %, and~\ref{item:shrp3}$\Rightarrow$\ref{item:shrp3a} is obvious.
  To show~\ref{item:shrp3a}$\Rightarrow$\ref{item:shrp2}, note that $\nSet$ is locally \la-presentable, and its \ka-presentable objects are those of cardinality $<\ka$.
  Thus if $\card X<\mu$, then $X$ is \mu-presentable in \nSet, so by assumption we have $X \cong \colim_{i\in I} X_i$ with $I$ \mu-small and \la-filtered and each $\card {X_i}<\la$.
  Let $\cA = \{ q_i(X_i) \mid i\in I \}$ be the set of images of the coprojections $q_i:X_i\to X$; then $\cA \subseteq P_\la(X)$ and $\card\cA<\mu$.
  And for any $Y\subseteq X$ with $\card Y<\la$, the set $Y$ is \la-presentable, so the inclusion $Y\into X$ factors through some object $X_i$ in the \la-filtered colimit.
  But then $Y \subseteq q_i(X_i) \in \cA$; so \cA is cofinal.
\end{proof}

\noindent
When these conditions hold, one writes %$\la\shle\mu$.
% If in addition $\la<\mu$ we write
$\la\shlt\mu$. % and says that \la is \textbf{sharply smaller} than \mu.
Then:
\begin{itemize}
\item The relation $\shlt$ is transitive (by \cref{thm:shrp}\ref{item:shrp1}).
\item % The relation $\shlt$ is set-directed, i.e.\
  For any set of regular cardinals $\{\la_i\}$ there is a regular cardinal $\mu$ such that $\la_i\shlt\mu$ for all $i$ (and hence there are arbitrarily large such \mu).
  If \ka is inaccessible and each $\la_i<\ka$, the class $\{\mu\mid \forall i .(\la_i\shlt\mu)\}$ is unbounded below \ka.
\item If $\la<\ka$ and \ka is inaccessible, then $\la\shlt\ka$.
\item By~\cite[Theorem 2.4.9]{mp:accessible} or~\cite[Theorem 2.19]{ar:loc-pres},
for any accessible functor $F$ there is a \la such that for any $\mu\shgt\la$, the functor $F$ is \mu-accessible and preserves \mu-presentable objects.
\end{itemize}

\begin{rmk}
  Since $\aleph_1\nshlt\aleph_{\om+1}$ by~\cite[Example 2.13(8)]{ar:loc-pres}, \cref{thm:shrp}\ref{item:shrp3a} also fails in this case.
  In particular, a set of cardinality $\aleph_\om$ is $\aleph_{\om+1}$-presentable in \nSet, but is not a $\aleph_{\om+1}$-small $\aleph_1$-filtered colimit of $\aleph_1$-presentable objects.
  In addition, there exist accessible functors $F$ (e.g.\ the endofunctor of \nSet defined by $F(X) = X^I$ for some infinite set $I$, cf.~\cite[Remark 3.2(4)]{br:aec-acc}) for which there are arbitrarily large regular cardinals \mu such that $F$ does not preserve \mu-presentable objects.
  Thus, it seems we cannot avoid the relation $\shlt$ or something like it.
  The relation $\ll$ used in~\cite{lurie:higher-topoi} is \textit{a priori} stronger than $\shlt$, but coincides with it if the Generalized Continuum Hypothesis holds~\cite[Fact 2.5]{lrv:intsize}.
\end{rmk}

\begin{prop}\label{thm:pres-pb}
  For any locally presentable category \E, there is a \la such that for any $\ka\shgt \la$, $\E$ is locally \ka-presentable and the \ka-presentable objects in \E are closed under finite limits.%\footnote{The proof of~\cite[Proposition 6.1.6.7]{lurie:higher-topoi} claims this for all sufficiently large \ka, but the argument only shows it to be true for a \shrp class.}
\end{prop}
\begin{proof}
  By \cref{thm:pw-pres}, if \E is locally \ka-presentable, then the \ka-presentable objects of the category $\E^{(\to\ot)}$ of cospans are the pointwise \ka-presentable ones.
  Thus, as soon as the pullback functor $\E^{(\to\ot)} \to \E$ preserves \ka-presentable objects, the \ka-presentable objects of \E are closed under pullbacks.
  But since this functor is a right adjoint, it is accessible, so there is a \la such that this occurs for all $\ka\shgt\la$; and we can choose \la large enough that the terminal object is also \la-presentable.
\end{proof}

\begin{prop}\label{thm:pres-sub}
  For any locally presentable category \E, there is a \la such that for any $\ka\shgt \la$, $\E$ is locally \ka-presentable and the \ka-presentable objects in \E are closed under finite limits and subobjects.
\end{prop}
\begin{proof}
  We first prove the result when \E is a Grothendieck 1-topos.
  In this case it suffices to take \la satisfying \cref{thm:pres-pb} and such that the subobject classifier $\Omega$ is \la-presentable, since any subobject of $X$ occurs as a pullback to $X$ of the universal subobject $1\to \Omega$.

  Now an arbitrary locally presentable category \E is a reflective subcategory of some Grothendieck (indeed presheaf) 1-topos \sT, say with inclusion functor $U$ and reflector $L\adj U$.
  Both $L$ and $U$ are accessible, so there is a \la such that for any $\ka\shgt \la$, the statement holds for \sT and $L$ and $U$ preserve \ka-presentable objects.
  Now if $A$ is a subobject of a \ka-presentable $X\in \E$, then $UA$ is a subobject of the \ka-presentable $UX\in \sT$.
  Hence $U A$ is \ka-presentable in \sT, so $A \cong L U A$ is \ka-presentable in \E.
\end{proof}

For constructing universes, we also need a ``fiberwise'' notion of size for morphisms.

\begin{defn}[{\cite[Definition 6.1.6.4]{lurie:higher-topoi}}]
  A morphism $X\to Y$ in \E is \textbf{relatively \ka-presentable} if $Z\times_Y X$ is \ka-presentable for any morphism $Z\to Y$ where $Z$ is \ka-presentable.
\end{defn}

Of course, a relatively \ka-presentable morphism with \ka-presentable codomain also has \ka-presentable domain.
Conversely, if \ka-presentable objects are closed under finite limits in \E (cf. \cref{thm:pres-pb}), then every morphism between \ka-presentable objects is relatively \ka-presentable, and an object $X$ is \ka-presentable just when the map $X\to 1$ is relatively \ka-presentable.

\begin{prop}\label{thm:relpres-detect}
  Let \E be locally \la-presentable and locally cartesian closed, with \cG a strong generating set of \la-presentable objects, and let $\ka\ge\la$.
  Then $f:X\to Y$ in \E is relatively \ka-presentable if and only if $Z\times_Y X$ is \ka-presentable for any morphism $g:Z\to Y$ where $Z\in\cG$.
\end{prop}
\begin{proof}
  Every \la-presentable object is \ka-presentable, so ``only if'' is trivial.
  Conversely, the \ka-presentable objects are the closure of \cG under \ka-small colimits; for as in~\cite[Theorem 1.11]{ar:loc-pres}, the \la-small colimits of \cG form a dense generator whose canonical diagrams are \la-filtered, and then as in~\cite[Remark 1.30]{ar:loc-pres} every \ka-presentable object is (a retract of) the colimit of a \ka-small subdiagram of its canonical diagram with respect to these.
  Thus, it suffices to show that if $Y = \colim_i Y_i$ is a \ka-small colimit, with each $Y_i$ (hence also $Y$) being \ka-presentable, and $f:X\to Y$ is such that each $Y_i \times_Y X$ is \ka-presentable, then $X$ is \ka-presentable.
  But since \E is locally cartesian closed, colimits are stable under pullback; so $X = \colim_i (Y_i \times_Y X)$ is a \ka-small colimit of \ka-presentable objects, hence \ka-presentable.
\end{proof}

\begin{cor}
  For any morphism $f:X\to Y$ in a locally presentable and locally cartesian closed category, there exists a regular cardinal \ka such that $f$ is relatively \ka-presentable.
\end{cor}
\begin{proof}
  Let \E be locally \la-presentable and \cG a set of representatives for isomorphism classes of \la-presentable objects.
  Then there are only a small set of morphisms $Z\to Y$ for objects $Z\in \cG$, hence there is a $\ka\ge \la$ such that all objects $Z\times_Y X$ are \ka-presentable.
  Hence by \cref{thm:relpres-detect}, $f$ is relatively \ka-presentable.
\end{proof}

\begin{eg}\label{eg:pshf-relpres}
  As noted in \cref{eg:pshf-pres}, in a presheaf category $\E=\prcs$ where \C is \ka-small, the \ka-presentable objects are the \ka-small colimits of representables.
  Since the representables are a strong generating set of \om-presentable objects, in this case $f:X\to Y$ is relatively \ka-presentable if and only if its pullback to any representable is \ka-presentable, hence if and only if all its fibers are \ka-small sets.
  Thus, in a presheaf category, for sufficiently large \ka the relatively \ka-presentable morphisms coincide with the \textbf{\ka-small morphisms} of~\cite{klv:ssetmodel,shulman:elreedy}.
\end{eg}

We now study the preservation properties of relatively \ka-presentable morphisms, which will yield the closure of universes under type forming operations.

\begin{lem}\label{thm:relpres-comp}
  For any regular cardinal \ka, the composite of relatively \ka-presentable morphisms is relatively \ka-presentable.
\end{lem}
\begin{proof}
  If $X\to Y\to Z$ are relatively \ka-presentable and we have $W\to Z$ with $W$ \ka-presentable, then $W\times_Z X \cong (W\times_Z Y) \times_Y X$ is also \ka-presentable.
\end{proof}

\begin{lem}\label{thm:pres-slice}
  If \E is a locally \la-presentable category, then for any $A\in \E$, a morphism $X\to A$ is a \la-presentable object of $\E/A$ if and only if $X$ is a \la-presentable object of \E.
\end{lem}
Note that $A$ is not required to be \la-presentable.
If $A$ \emph{is} \la-presentable, and \la-presentable objects are closed under pullbacks, then the \la-presentable objects of $\E/A$ will also coincide with the relatively \la-presentable morphisms of \E with codomain $A$; but in the general case this is not true.
\begin{proof}[Proof of \cref{thm:pres-slice}]
  First suppose $X$ is \la-presentable in \E, and $Y = \colim_i Y_i$ is a \la-filtered colimit in $\E/A$.
  Then any map $X\to Y$ in $\E/A$ is in particular a map $X\to Y$ in \E, hence factors through some $Y_i$, and the factorization lies in $\E/A$.
  Similarly, any two maps $X\to Y_i$ and $X\to Y_j$ in $\E/A$ are in particular maps in \E, hence coincide in some $Y_k$, and so the same is true in $\E/A$.
  Thus, $X\to A$ is \la-presentable in $\E/A$.

  Conversely, suppose $X\to A$ is \la-presentable in $\E/A$, and write $X = \colim_i X_i$ as a \la-filtered colimit of \la-presentable objects in \E.
  Then we can make each $X_i$ into an object of $\E/A$ by the composite $X_i \to X\to A$, and the colimit $X = \colim_i X_i$ then lies in $\E/A$ too.
  Thus, since $X\to A$ is \la-presentable, the identity $\id_X : X\to X$ factors through some $X_i$ in $\E/A$ and hence also in \E.
  So $X$ is a retract of some $X_i$, and is therefore \la-presentable like it.
\end{proof}

\begin{lem}\label{thm:relpres-pow}
  If \E is locally presentable and enriched with powers and copowers over some category \sV, then for any $K\in \sV$ there is a \la such that for any $\ka\shgt\la$ and $Y\in \E$ the power functor $(K \pow_Y -):\E/Y \to \E/Y$ preserves relatively \ka-presentable morphisms.
\end{lem}
\begin{proof}
  Let \E be locally \mu-presentable; by \cref{thm:relpres-detect}, it suffices to ensure that $Z\times_Y (K\pow_Y X)$ is \ka-presentable for any \mu-presentable $Z$ and relatively \ka-presentable $X\to Y$.
  But $Z\times_Y (K\pow_Y X) \cong K\pow_Z (Z\times_Y X)$, so for this it suffices to ensure that $(K\pow_Z -):\E/Z\to \E/Z$ preserves objects with \ka-presentable domain for all \mu-presentable $Z$.
  But by \cref{thm:pres-slice}, this is the same as preserving \ka-presentable objects of $\E/Z$.
  And each functor $(K\pow_Z -)$ is accessible (being a right adjoint), so there is a $\la_Z$ such that it preserves \ka-presentable objects for any $\ka\shgt\la_Z$.
  Finally, there is only a set of isomorphism classes of \mu-presentable objects $Z$, so there is a $\la$ with $\la\shgt\la_Z$ for all such $Z$.
\end{proof}

\begin{lem}\label{thm:relpres-mono-dp}
  Let \E be locally presentable and locally cartesian closed.
  Then there is a \la such that for any monomorphism $i:A\mono B$ and any $\ka\shgt\la$, the functor $i_* : \E/A \to \E/B$ preserves relatively \ka-presentable morphisms.
\end{lem}
\begin{proof}
  By \cref{thm:pres-sub}, there is a \mu such that \E is locally \mu-presentable and \mu-presentable objects are closed under subobjects.
  For each morphism $j$ between \mu-presentable objects, the functor $j_*$ is accessible, and there is only a set of such functors; thus there is a \la such that $\la\ge\mu$ and for any $\ka\shgt\la$ all these functors preserve \ka-presentable objects.

  Now let $i:A\mono B$ be any monomorphism and $f:X\to A$ be relatively \ka-presentable, where $\ka\shgt\la$.
  By \cref{thm:relpres-detect}, to show that $i_*(X)$ is relatively \ka-presentable it suffices to show that $Z\times_B i_*(X)$ is \ka-presentable for any morphism $Z\to B$ where $Z$ is \mu-presentable.
  Let $Y$ be the pullback $Z\times_B A$; then $j:Y\mono Z$ is a monomorphism, so $Y$ is also \mu-presentable by our choice of \mu.
  And by the Beck-Chevalley condition, we have $Z\times_B i_*(X) \cong j_*(Y\times_A X)$.
  But $Y \times_A X$ is \ka-presentable since $X\to A$ is relatively \ka-presentable, while $j_*$ preserves \ka-presentable objects since $\ka\shgt\la$; thus $Z\times_B i_*(X)$ is \ka-presentable.
\end{proof}

\begin{lem}\label{thm:relpres-dp}
  Let \E be locally presentable and locally cartesian closed.
  Then there is a regular cardinal $\la$ such that for any inaccessible cardinal $\ka>\la$ and any relatively \ka-presentable morphisms $X \xto{g} Y \xto{f} Z$, the dependent product $f_*(X) \to Z$ is relatively \ka-presentable.
\end{lem}
\begin{proof}
  Let \la satisfy \cref{thm:pres-pb}; then the regular cardinals \mu such that the \mu-presentable objects of \E are closed under finite limits are then unbounded below any inaccessible $\ka>\la$.

  Let \ka be such an inaccessible and $X \xto{g} Y \xto{f} Z$ be relatively \ka-presentable; we must show that for any morphism $W\to Z$ with $W$ \ka-presentable, the pullback $W\times_Z f_*(X)$ is \ka-presentable.
  As in \cref{thm:relpres-mono-dp}, let $V = W\times_Z Y$ with projection $h:V\to W$; then $V$ and $V\times_Y X$ are \ka-presentable, while $W\times_Z f_*(X) \cong g_*(V\times_Y X)$.
  Thus, it suffices to prove that $g_*$ preserves \ka-presentable objects for any morphism $g:V\to W$ between \ka-presentable objects.

  Since \ka is a limit cardinal, any \ka-presentable object is \mu-presentable for some $\mu<\ka$.
  Indeed, every \ka-presentable object is a \ka-small colimit of \la-presentable objects; but this diagram has some cardinality $<\ka$, hence is \mu-small for some $\mu<\ka$, and so its colimit is \mu-presentable.
  %
  In particular, for any \ka-presentable object $Q$ and morphism $Q\to V$, there is a $\mu<\ka$ such that $Q$, $V$, and $W$ are all \mu-presentable and \mu-presentable objects are closed under pullback.
  Thus, $g^*:\E/W \to \E/V$ preserves \mu-presentable objects, hence by the proof of~\cite[Proposition 2.23]{ar:loc-pres} its right adjoint $g_*$ is \mu-accessible.
  Therefore, there is a $\nu\ge \mu$ with $\nu<\ka$ such that $g_*$ preserves \nu-presentable objects; hence $g_*(Q)$ is \nu-presentable and thus \ka-presentable.
\end{proof}

\begin{rmk}
  The asymmetry in hypotheses between \cref{thm:relpres-comp,thm:relpres-dp} (corresponding to $\Sigma$- and $\Pi$-types respectively) is due to our adherence to to the traditional use of only \emph{regular} cardinals to bound the size of objects in a locally presentable category.
  However, it should also be possible to study objects and morphisms bounded in size by singular cardinals (cf.~\cite{lrv:intsize}), enabling \cref{thm:relpres-dp} to apply to any strong limit \ka satisfying a $\shlt$-like property.
\end{rmk}

Finally, the following two facts show that relatively \ka-presentable morphisms serve our desired purpose.

\begin{prop}\label{thm:relpres-small}
  Let \E be locally \ka-presentable and locally cartesian closed.
  Then for any $Y\in\E$ the subcategory of $\E/Y$ determined by the relatively \ka-presentable morphisms is essentially small.
\end{prop}
\begin{proof}
  Write $Y = \colim_i Y_i$ as a colimit of \ka-presentable objects.
  Since colimits are stable under pullback, any $X\in \E/Y$ is the colimit in \E of the diagram of pullbacks $X \cong \colim_i (Y_i\times_Y X)$.
  Thus, if for $X$ and $X'$ the corresponding diagrams of pullbacks are isomorphic over $\{Y_i\}$, then $X\cong X'$.
  And if $X\to Y$ is relatively \ka-presentable, then each object $Y_i \times_Y X$ must be \ka-presentable, so the result follows since the full subcategory of \ka-presentable objects is essentially small.
\end{proof}

\begin{prop}[{cf.~\cite[6.1.6.5--6.1.6.7]{lurie:higher-topoi}}]\label{thm:relpres-local}
  Let \E be locally presentable and locally cartesian closed.
  Then there is a \la such that for any $\ka\shgt\la$, the relatively \ka-presentable morphisms are a \local full \nfs.
\end{prop}
\begin{proof}
  Let \la satisfy \cref{thm:pres-pb}, and $\ka\shgt\la$.
  By \cref{thm:local}, we must show that if $Y = \colim_{i\in I} Y_i$, then $f:X\to Y$ is relatively \ka-presentable as soon as its pullback to each $Y_i$ is.
  Since every colimit is a \ka-filtered colimit of \ka-small colimits, it will suffice to consider those two cases separately.

  When $I$ is \ka-filtered, any morphism $Z\to Y$ where $Z$ is \ka-presentable must factor through some $Y_i$.
  Then $Z\times_Y X$ is isomorphic to $Z\times_{Y_i} (Y_i \times_Y X)$, which is \ka-presentable since $Y_i \times_Y X \to Y_i$ is relatively \ka-presentable.

  When $I$ is \ka-small, note that by~\cite[Corollary 1.54]{ar:loc-pres}, $\E^I$ is also locally \ka-presentable.
  Thus we can write $\{Y_i\}\in\E^I$ as a \ka-filtered colimit of \ka-presentable diagrams, $Y_i = \colim_{j\in J} W_{j i}$ where $J$ is \ka-filtered and each $W_j\in \E^I$ is \ka-presentable.
  By \cref{thm:pw-pres}, each $W_{j i}\in \E$ is also \ka-presentable.
  But by assumption, the pullback of $f$ to each $Y_i$ is relatively \ka-presentable, and hence each $W_{j i} \times_Y X$ is \ka-presentable.

  Since $Y \cong \colim_i \colim_j W_{j i} \cong \colim_j \colim_i W_{j i}$, and $J$ is \ka-filtered, it will suffice to show that the pullback of $X$ to each $\colim_i W_{j i}$ is relatively \ka-presentable.
  But $\colim_i W_{j i}$ is a \ka-small colimit of \ka-presentable objects, hence \ka-presentable, so since \ka-presentable objects are closed under pullbacks it will suffice to show that the object $(\colim_i W_{j i}) \times_Y X$ is \ka-presentable.
  Finally, since \E is locally cartesian closed, colimits are stable under pullback, so this object is isomorphic to $\colim_i (W_{j i}\times_Y X)$, which is a \ka-small colimit of \ka-presentable objects, hence also \ka-presentable.
\end{proof}

We will denote this \nfs by $\cEka$.
More generally, for any \nfs \F we write $\Fka = \F \times_\cE \cEka$.


%%% Local Variables:
%%% mode: latex
%%% TeX-master: "univinj"
%%% End:
