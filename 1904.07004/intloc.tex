\section{Internal localizations}
\label{sec:int-loc}

Let \E be a \ttmt.
In the next section we will show that any left exact localization of \E is again a \ttmt; in this section we begin by studying the wider class of \emph{internal} localizations.

For an object $A\in\E$, we write $A^*$ for the pullback functor $\E \to \E/A$, taking $X$ to the product $A\times X$.
Recall that $f_*$ denotes the right adjoint of pullback $f^*$.

\begin{defn}
  Let $S$ be a class of fibrations between fibrant objects in \E.
  \begin{itemize}
  \item $S\pb$ denotes the class of all pullbacks of morphisms in $S$:
    \[ S\pb = \setof{ g^*f | (f:A\to B)\in S,\; g:C\to B } \]
  \item For $X\in\E$, $X^*S$ denotes the class of pullbacks of morphisms in $S$ to $\E/X$:
    \[ X^*S = \setof{ X^*f | f\in S} \]
  \end{itemize}
\end{defn}

Of course even when $S$ is a set, $S\pb$ is a proper class, but it is nevertheless ``generated by a set'' in the relevant sense (cf.~~\cite[Proposition 6.2.1.2]{lurie:higher-topoi}):

\begin{lem}
  For any set $S$ of fibrations between fibrant objects, there is a set of morphisms $S'$ such that an object is $S'$-local if and only if it is $S\pb$-local.
\end{lem}
\begin{proof}
  By~\cite[Proposition 4.7]{dug:pres}, there is a regular cardinal $\la$ such that all canonical maps $\displaystyle\hocolim_{\E_\la\ni W\to X}(W) \to X$ are weak equivalences, where $\E_\la$ is the full subcategory of \la-compact objects.
  Let $S'$ be a set of representatives for the pullbacks of morphisms in $S$ to \la-compact objects, and consider some $f:A\fib B$ in $S$ and some $g:X\to B$.
  Then we have a commutative square as on the left:
  \[
    \begin{tikzcd}
      \displaystyle\hocolim_{\E_\la\ni W\to X}(W \times_B A) \ar[r,"\sim"] \ar[d] &
      X\times_B A \ar[d,two heads,"g^*f"'] \ar[r] \drpullback & A \ar[d,two heads,"f"]\\
      \displaystyle\hocolim_{\E_\la\ni W\to X}(W) \ar[r,"\sim"'] & X \ar[r] & B
    \end{tikzcd}
  \]
  inducing for any fibrant object $Z$ a commutative diagram:
  \[
    \begin{tikzcd}
      \ehom\E(X,Z) \ar[r,"\sim"'] &
      \displaystyle\ehom\E\left(\hocolim_{\E_\la\ni W\to X}(W),Z\right) \ar[r,"\sim"'] &
      \displaystyle\holim_{\E_\la\ni W\to X} \ehom\E(W,Z)\\
      \ehom\E(X\times_B A, Z) \ar[r,"\sim"] \ar[from=u,"{\ehom\E(g^*f,Z)}"'] &
      \displaystyle\ehom\E\left(\hocolim_{\E_\la\ni W\to X}(W \times_B A),Z\right) \ar[r,"\sim"] \ar[from=u] &
      \displaystyle\holim_{\E_\la\ni W\to X} \ehom\E(W \times_B A,Z) \ar[from=u]
    \end{tikzcd}
  \]
  Now if $Z$ is $S'$-local, each map $\ehom\E(W,Z) \to \ehom\E(W\times_B A,Z)$ is a weak equivalence, and homotopy limits preserve weak equivalences.
  Thus the right-hand map above is a weak equivalence, hence by 2-out-of-3 so is the left-hand map; so $Z$ is $S\pb$-local.
  The converse is obvious.
\end{proof}

Since \E is combinatorial, its left Bousfield localization at $S'$, hence at $S\pb$, exists.

\begin{lem}\label{thm:stloc}
  For a fibrant object $Z$ and a class $S$ of fibrations between fibrant objects, the following are equivalent.
  \begin{enumerate}
  \item $Z$ is $S\pb$-local.\label{item:sl1}
  \item For all $f:A\fib B$ in $S$, the induced morphism
    \begin{equation}
      \ftil_Z : B^* Z \to f_* A^* Z\label{eq:stloc}
    \end{equation}
    is a weak equivalence in $\E/B$.\label{item:sl2}
  \end{enumerate}
\end{lem}
\begin{proof}
  Since $f$ is a fibration and $Z$ is fibrant, both $B^*Z$ and $f_* A^* Z$ are fibrant in $\E/B$.
  Thus~\ref{item:sl2} is equivalent to saying that for any $(f:A\to B)\in S$ and $g:X\to B$ in $\E/B$ the induced map of simplicial hom-spaces
  \[ \ehom{\E/B}(X,B^*Z) \to \ehom{\E/B}(X,f_* A^* Z) \]
  is an equivalence.
  However, by the simplicial adjunction $\E/B \toot \E$, we have $\ehom{\E/B}(X,B^*Z) \cong \ehom\E(X,Z)$.
  Similarly, since \E is \slcc, we have a simplicial adjunction $f^* : \E/B \toot \E/A : f_*$, so that
  \[\ehom{\E/B}(X,f_* A^* Z) \cong \ehom{\E/A}(f^*X, A^* Z) \cong \ehom\E(X\times_B A, Z). \]
  Under these isomorphisms~\eqref{eq:stloc} is identified with
  \[ \ehom\E(g^*f,Z) : \ehom\E(X,Z) \to \ehom\E(X\times_B A,Z) \]
  and to say that this is an equivalence for all $f\in S$ is precisely~\ref{item:sl1}.
\end{proof}

\begin{lem}\label{thm:fl-we}
  For any set $S$ of fibrations between fibrant objects and any square
  \[
    \begin{tikzcd}
      Z \ar[d,two heads] \ar[r,"\sim"] & W \ar[d,two heads] \\
      X \ar[r,"g","\sim"'] & Y
    \end{tikzcd}
  \]
  such that the vertical maps are fibrations and the horizontal maps are weak equivalences, $W$ is $(Y^*S)\pb$-local in $\E/Y$ if and only if $Z$ is $(X^*S)\pb$-local in $\E/X$.
\end{lem}
\begin{proof}
  By right properness, the induced map $Z \to g^*W$ is a weak equivalence between fibrant objects of $\E/X$.
  Since $B^*$, $A^*$, and $f_*$ in~\eqref{eq:stloc} are right Quillen functors, they preserve weak equivalences between fibrant objects.
  Thus we have
  \[
    \begin{tikzcd}
      B^*Z \ar[r] \ar[d,"\sim"'] & f_*A^* Z \ar[d,"\sim"] \\
      B^*g^*W \ar[r] & f_* A^* g^* W 
    \end{tikzcd}
  \]
  in $\E/X$, so that by 2-out-of-3 $Z$ is $(X^*S)\pb$-local if and only if $g^* W$ is.
  On the other hand, we have $g^*(Y^*S) = X^*S$, and the construction of~\eqref{eq:stloc} commutes with pullback. % (using the Beck-Chevalley condition for dependent products).
  Thus we have a triangle of pullback squares
  \[
    \begin{tikzcd}[row sep=small,column sep=small]
      B^*g^*W \ar[ddr,two heads] \ar[drr] \ar[rrr,"\sim"] &&& B^*W \ar[drr]\ar[ddr,two heads] \\
      && f_* A^* g^* W \ar[dl,two heads] \ar[rrr,near start,"\sim",crossing over] &&& f_* A^* W \ar[dl,two heads]\\
      & X \ar[rrr,"g","\sim"'] &&& Y
    \end{tikzcd}
  \]
  in which all the rightwards-pointing arrows are weak equivalences; so by 2-out-of-3 again, $g^*W$ is $(X^*S)\pb$-local if and only if $W$ is $(Y^*S)\pb$-local.
\end{proof}

\begin{prop}\label{thm:flf-nfs}
  For any set $S$ of fibrations between fibrant objects in a \ttmt \E, there is a \local, \stratified, and homotopy invariant \nfs $\dL_S$ such that the maps admitting an $\dL_S$-structure are the fibrations $Z\fib Y$ that are $(Y^*S)\pb$-local in $\E/Y$.
\end{prop}
\begin{proof}
Since $B^*Z$ and $f_* A^* Z$ in~\eqref{eq:stloc} are fibrations over $\E/B$, by~\cite[Lemma 4.3]{shulman:elreedy} %{For self-containedness, we could reproduce that argument here.}
for each $f$ and $Z$ there is a fibrant object
\( \iLoc_f Z \coloneqq B_* \isequiv_{B}(\ftil_Z) \)
such that $\ftil_Z$ is an equivalence if and only if $\iLoc_f Z$ has a global element, in which case it is acyclic (i.e.\ $\iLoc_f Z \to 1$ is an equivalence).
This is because inside homotopy type theory the property of ``being an equivalence'' is a proposition.
%since $S$ is a set (rather than a proper class) and
Hence, if we define $\iLoc_S Z \coloneqq \prod_{f\in S} \iLoc_f Z$, then $Z$ is $S\pb$-local if and only if $\iLoc_S Z$ has a global element, in which case it is also acyclic. % (since acyclic fibrations are stable under products).

Furthermore, the construction of $\iLoc_S Z$ is stable under pullback: we have $X^*(\iLoc_S Z) \cong \iLoc_{X^* S}(X^* Z)$ coherently.
Thus $\iLoc_S$ is a fibred core-endofunctor of \E, so \cref{thm:sec-afib-strat} applies; hence $\iLoc_S^*(\cEp)$ is a \local and \stratified \nfs.
  So if we define $\dL_S = \F \times_\cE \iLoc_S^*(\cEp)$, where \F is the given \nfs for fibrations in \E, then $\dL_S$ is a \local and \stratified \nfs.
  %, and an $\dL_S$-structure on $Z\fib Y$ is an \F-structure and a section of $\iLoc_{Y^*S} Z$.
  The previous paragraph, applied in $\E/Y$, shows that the $\uly{\dL_S}$-algebras over $Y$ are the $(Y^*S)\pb$-local fibrations.
  Finally, homotopy invariance follows from \cref{thm:fl-we}.
\end{proof}

\begin{rmk}\label{rmk:modal-univ}
  In particular, by \cref{thm:uf-fibrant}, $\dL_S$ admits fibrant and univalent universes whether or not $S$-localization is left exact.
  These are an external version of the ``universes of modal types'' from~\cite{rss:modalities} (for modalities only, although it should be possible to generalize them to reflective subuniverses), and are also a strict model-categorical version of the object classifiers for stable factorization systems in \io-categories obtainable from~\cite[\sect3-4]{gk:univlcc}.
\end{rmk}

We now record a few more properties of these localizations, paralleling facts proven internally in type theory in~\cite{rss:modalities}.

\begin{lem}\label{thm:pb-fl-fle}
  For any set $S$ of fibrations between fibrant objects and any $g:X\to Y$, the functor $g^*: \E/Y\to \E/X$ takes $(Y^*S)\pb$-local objects to $(X^*S)\pb$-local objects and takes $(Y^*S)\pb$-local equivalences between fibrations to $(X^*S)\pb$-local equivalences.
\end{lem}
\begin{proof}
  The first statement follows since $g^*(Y^*S) = X^*S$ and~\eqref{eq:stloc} commutes with pullback, as in \cref{thm:fl-we}.
  For the second, by factoring $g$ we can assume separately that it is a weak equivalence and that it is a fibration.

  If $g$ is a weak equivalence, then the adjunction $g_! \adj g^*$ is a Quillen equivalence by~\cite[Proposition 2.5]{rezk:proper}.
  Thus, given a $(Y^*S)\pb$-local equivalence $f:A\to B$ in $\E/Y$, to show that $\ehom{\E/X}(g^*B,Z) \to \ehom{\E/X}(g^*A,Z)$ is an equivalence for all $(X^*S)\pb$-local objects $Z\in\E/X$, it suffices to show that $\ehom{\E/Y}(B,W) \to \ehom{\E/Y}(A,W)$ is an equivalence for all fibrant objects $W\in \E/X$ such that $g^*W$ is $(X^*S)\pb$-local.
  But by \cref{thm:fl-we}, the latter is equivalent to $W$ being $(Y^*S)\pb$-local, so this is true since $f$ is a $(Y^*S)\pb$-local equivalence.

  If $g$ is a fibration, then $g^*:\E/Y\to \E/X$ is a left Quillen functor since it preserves cofibrations and weak equivalences, and $g^*(Y^*S) = X^*S$ implies $g^*((Y^*S)\pb) \subseteq (X^*S)\pb$.
  Thus, by~\cite[Proposition 3.3.18]{hirschhorn:modelcats}, $g^*$ is also a left Quillen functor from $(\E/Y)_{(Y^*S)\pb}$ to $(\E/X)_{(X^*S)\pb}$, hence (since all objects are cofibrant) it takes all $(Y^*S)\pb$-local equivalences to $(X^*S)\pb$-local equivalences.
\end{proof}

\begin{lem}[{cf.~\cite[Theorem 2.17]{rss:modalities}}]\label{thm:sigma-closed}
  For any set $S$ of fibrations between fibrant objects, any $S\pb$-local object $Y$, and any fibration $g:X\fib Y$ that is $(Y^*S)\pb$-local in $\E/Y$, the object $X$ is $S\pb$-local.
\end{lem}
\begin{proof}
  We factor the naturality square for~\eqref{eq:stloc} at $g$ through its pullback:
  \[
    \begin{tikzcd}
      \mathllap{B^*X =\;} (Y^*B)^*X \ar[r,"\sim"] \ar[d,two heads] &
      (Y^*f)_*(Y^*A)^*X \ar[r,"\sim"] \ar[d,two heads] \drpullback & f_* A^* X\ar[d,two heads]\\
      B^*Y \ar[r,equals] & B^*Y \ar[r,"\sim"'] & f_* A^* Y.
    \end{tikzcd}
  \]
  Comparing universal properties, we see that this pullback is $(Y^*f)_*(Y^*A)^*X$ as shown, and the upper-left horizontal map is~\eqref{eq:stloc} for $X$ in $\E/Y$, hence a weak equivalence.
  Since the lower-right horizontal map is~\eqref{eq:stloc} for $Y$, it is also a weak equivalence, and thus so is its pullback.
  Therefore, the top composite $B^*X \to f_* A^* X$, which is~\eqref{eq:stloc} for $X$, is also a weak equivalence; so $X$ is $S\pb$-local.
\end{proof}

\begin{lem}[{cf.~\cite[Theorem 1.32]{rss:modalities}}]\label{thm:unit-connected}
  Let $S$ be a set of fibrations between fibrant objects and let $\eta: X\to \Xhat$ be an $S\pb$-localization, i.e.\ an $S\pb$-local equivalence to an $S\pb$-local object.
  Then $\eta$ is also an $(\Xhat^*S)\pb$-local equivalence in $\E/\Xhat$.
\end{lem}
\begin{proof}
  Let $p:Z\fib \Xhat$ be an $(\Xhat^*S)\pb$-local object of $\E/\Xhat$.
  Then by \cref{thm:sigma-closed}, $Z$ is an $S\pb$-local object of \E, hence $p$ is a fibration in the $S\pb$-local model structure $\E_{S\pb}$.
  Therefore, $Z$ is a fibrant object of the slice model structure $\E_{S\pb}/\Xhat$ on $\E/\Xhat$.
  But $\eta$ is a weak equivalence (between cofibrant objects) in that same model structure, so the induced map $\ehom{\E/\Xhat}(\Xhat,Z) \to\ehom{\E/\Xhat}(X,Z)$ is a weak equivalence.
  Since this is true for any $(\Xhat^*S)\pb$-local object $Z$, $\eta$ is a $(\Xhat^*S)\pb$-local equivalence.
\end{proof}


% Local Variables:
% TeX-master: "univinj"
% End:
