\section{Coherent transformations and bar constructions}
\label{sec:coh-bar}

In preparation for our treatment of injective model structures using cobar constructions in \cref{sec:injmodel}, in this section we give some intuition for how (co)bar constructions arise and review some of their formal properties.
Let \E be a model category, and suppose for simplicity that \D is an ordinary small category.
The most obvious notions of weak equivalence, fibration, and cofibration in $\pr\D\E$ are induced pointwise from \E, and functorial factorizations in \E can also be applied pointwise.
However, there is a problem with the lifting properties: suppose we have a square in $\pr\D\E$
\[
  \begin{tikzcd}
    A \ar[r,"f"] \ar[d,"i"'] & X \ar[d,"v"] \\
    B \ar[r,"g"'] & Y
  \end{tikzcd}
\]
in which $i$ is a pointwise cofibration and $v$ a pointwise acyclic fibration.
For each $d\in\D$ we have a lift $h_d : B_d \to X_d$ with $h_d \circ i_d = f_d$ and $v_d \circ h_d = g_d$, but it may not be possible to choose these \emph{naturally} in $d$: for $\delta : d_1\to d_2$ we may not have $X_{\delta} \circ h_{d_2} = h_{d_1}\circ B_\delta$.
However, both $X_{\delta} \circ h_{d_2}$ and $h_{d_1}\circ B_\delta$ are lifts in the square
\[
  \begin{tikzcd}[column sep=large]
    A_{d_2} \ar[rr,"{X_\delta \circ f_{d_2} = f_{d_1}\circ A_\delta}"] \ar[d,"{i_{d_2}}"'] & & X_{d_1} \ar[d,"{v_{d_1}}"] \\
    B_{d_2} \ar[rr,"{Y_\delta \circ g_{d_2} = g_{d_1}\circ B_\delta}"'] & & Y_{d_1},
  \end{tikzcd}
\]
and the space of such lifts is contractible, being a fiber of the acyclic fibration
\[\ehom\E(B_{d_2},X_{d_1}) \to \ehom\E(A_{d_2},X_{d_1}) \times_{\ehom\E(A_{d_2},Y_{d_1})} \ehom\E(B_{d_2},Y_{d_1}).\]
Thus we have a \emph{homotopy} $X_{\delta} \circ h_{d_2} \sim h_{d_1}\circ B_\delta$ over $v_{d_1}$ and under $i_{d_3}$.
Similarly, given $\delta' : d_2\to d_3$ there is a 2-simplex in $\ehom\E(B_{d_3},X_{d_1})$ relating these homotopies for $\delta$, $\delta'$, and $\delta'\circ\delta$, and so on, yielding a \emph{homotopy coherent natural transformation} $h:B\cohto X$ such that $v \circ h = g$ and $h\circ i = f$.

There is an entire theory of homotopy coherent transformations (see e.g.~\cite{cp:hcct}), but as our purpose at the moment is motivational we only sketch it.
For $X,Y\in\pr\D\E$, a homotopy coherent transformation $h:X\cohto Y$ consists of:
\begin{itemize}
\item For every $d\in \D$, a morphism $h_d:X_d \to Y_d$.
\item For every $d_1\xto{\delta} d_2$ in \D, a homotopy $h_\delta: \Delta[1]\cpw X_{d_2} \to Y_{d_1}$ between $Y_{\delta} \circ h_{d_2}$ and $h_{d_1}\circ X_\delta$, such that $h_{\id_d}$ is constant.
\item For every $d_1\xto{\delta} d_2 \xto{\delta'} d_3$ in \D, a 2-simplex $h_{\delta,\delta'}:\Delta[2]\cpw X_{d_3} \to Y_{d_1}$ whose boundaries involve $h_\delta$, $h_{\delta'}$, and $h_{\delta'\circ\delta}$, satisfying similar constancy conditions.
\item And so on.
\end{itemize}
Let $U:\pr\D\E \to \E^{\ob\D}$ denote the forgetful functor, with $F$ its left adjoint defined by $(F W)_d = \coprod_{d'} \D(d,d') \cpw W_{d'}$; note both preserves simplicial copowers.
Thus:
\begin{itemize}
\item A collection of morphisms $h_d:X_d \to Y_d$ constitutes a morphism $U X \to U Y$, or equivalently $F U X \to Y$.
\item A collection of morphisms $h_\delta: \Delta[1]\cpw X_{d_2} \to Y_{d_1}$ constitutes a morphism $\Delta[1] \cpw U F U X \to U Y$, or equivalently $\Delta[1] \cpw F U F U X \to Y$.
\item A collection of morphisms $h_{\delta,\delta'}:\Delta[2]\cpw X_{d_3} \to Y_{d_1}$ constitutes a morphism $\Delta[2]\cpw U F U F U X \to U Y$, or equivalently $\Delta[2]\cpw F U F U F U X \to Y$.
\item And so on.
\end{itemize}
The fact that all these transformations have the right boundaries and constancy conditions means precisely that they assemble into a single \emph{strict} natural tranformation $\bar(F,UF,UX) \to Y$, where $\bar(F,UF,UX)$ is the geometric realization of the ``two-sided simplicial monadic bar construction''~\cite{may:goils,meyer:bar_i} $\sbar(F,UF,UX)$:
\[\small
  \begin{tikzcd}
    \cdots \ar[r,-] \ar[r,-,shift left=1] \ar[r,-,shift left=2] \ar[r,-,shift left=3]
     \ar[r,-,shift right=1] \ar[r,-,shift right=2] \ar[r,-,shift right=3]
    &
    FUFUFUX \ar[r] \ar[r,shift left=4] \ar[r,shift right=4] &
    FUFUX \ar[r,shift left=2] \ar[r, shift right=2] \ar[l,shift left=2] \ar[l,shift right=2] &
    FUX \ar[l]
  \end{tikzcd}
\]
whose face and degeneracy maps are defined by the unit and counit of the adjunction $F\adj U$.
In other words, $\bar(F,UF,UX)$ is a ``classifier'' for homotopy coherent transformations with domain $X$: we have a natural bijection between homotopy coherent transformations $X\cohto Y$ and strict transformations $\bar(F,UF,UX)\to Y$.

In particular, the identity map of $\bar(F,UF,UX)$ corresponds to a universal homotopy coherent transformation $p_X:X\cohto \bar(F,UF,UX)$, which is universal in that any homotopy coherent transformation $h:X\cohto Y$ can be written as\footnote{To compose two homotopy coherent transformations we need fibrancy and cofibrancy conditions to obtain horn-fillers in hom-spaces, but there is no trouble composing a homotopy coherent transformation with a strict one on either side.} $h = \overline{h}\circ p_X$ for a unique strict transformation $\overline{h}:\bar(F,UF,UX)\to Y$.
The identity map of $X$ corresponds to a canonical strict transformation $q_X = \overline{\id_X}:\bar(F,UF,UX)\to X$ such that $q_X \circ p_X = \id_X$, and it is a fact (see \cref{thm:bar-repl}) that we also have a homotopy $p_X \circ q_X \sim \id_{\bar(F,UF,UX)}$, so that $p_X$ and $q_X$ are inverse simplicial homotopy equivalences.
Moreover, this correspondence is natural with respect to strict transformations $v:Y\to Z$, i.e.\ we have $\overline{v\circ h} = v\circ \overline{h}$, and in particular for a strict $w:X\to Z$ we have $\overline{w} = w\circ q_X$.

\begin{rmk}\label{rmk:2mnd}
  This description of the bar construction as a classifier for homotopy coherent transformations is a generalization to homotopy theory of the \emph{pseudomorphism classifiers} of 2-monad theory.
  As shown in~\cite{bkp:2dmonads}, for any suitable 2-monad $T$, the inclusion $T\algs \into T\alg$ of the 2-category of $T$-algebras and strict morphisms into the category of $T$-algebras and pseudomorphisms has a left adjoint, traditionally denoted $A \mapsto \pscl A$.
  Thus pseudo $T$-morphisms $A\cohto B$ are in bijection with strict $T$-morphisms $\pscl A\to B$.
  In~\cite{lack:codescent-coh} the pseudomorphism classifier $\pscl A$ is constructed as a codescent object, which is really just a 2-truncated bar construction.
  (At present we are thinking only about the monad $U F$ on $\E^{\ob\D}$ whose category of algebras is $\pr\D\E$, but like the pseudomorphism classifier, the bar construction makes sense for any monad.)
\end{rmk}

The above discussion suggests that to obtain lifting properties in $\pr\D\E$, we need to be able to ``rectify'' homotopy coherent natural transformations to strict ones.
But by universality, if $p_X$ is homotopic to a strict transformation $s:X\to \bar(F,UF,UX)$ then the same is true of every homotopy coherent transformation with domain $X$.
And since $q_X$ is strict and a simplicial homotopy inverse of $p_X$, this is equivalent to saying that $q_X$ has a simplicial homotopy inverse in $\pr\D\E$.

If $q_X$ has a homotopy inverse in $\pr\D\E$ and $X$ is also pointwise cofibrant, we may call it \emph{projectively semi-cofibrant}.
In this case, if $v:Y\to Z$ is a pointwise acyclic fibration and we have $g:X\to Z$, then by choosing lifts and homotopies as above we can produce a homotopy coherent lift $h:X\cohto Y$ such that $v \circ h = g$, and then rectify it to an equivalent strict transformation $k:X\to Y$ with $k\sim h$.
But then we have only a homotopy $v\circ k \sim v\circ h = g$, i.e.\ although $k$ is a strict transformation, it is only a lift up to homotopy.

Thus we actually need a stronger property: that $s:X\to \bar(F,UF,UX)$ is a \emph{strict} section of $q_X$, i.e.\ $q_X \circ s = \id_X$.
(It is then automatically a simplicial homotopy inverse.)
%This then implies $s\circ q_X \sim p_X \circ q_X \circ s \circ q_X = p_X \circ q_X \sim \id_{\bar(F,UF,UX)}$, so $s$ is also a homotopy inverse of $q_X$.)
In this case, the strict transformation $k$ is defined by $\overline{h} \circ s$, and we have
\[v \circ k = v\circ \overline{h}\circ s = \overline{v\circ h} \circ s = \overline{g} \circ s = g\circ q_X \circ s = g \]
so that $k$ is also a strict lift.
This discussion has been somewhat informal, but we will show more carefully in \cref{thm:injfib} below that indeed, under suitable hypotheses, $\emptyset\to X$ has left lifting for pointwise fibrations if and only if it is pointwise cofibrant and $q_X$ has a strict section.
Thus these are the \emph{projectively cofibrant} objects, which indeed are the cofibrant objects in a model structure whose weak equivalences and fibrations are pointwise.

\begin{rmk}\label{rmk:flexible}
  Continuing \cref{rmk:2mnd}, in~\cite{bkp:2dmonads} an algebra $A$ for a 2-monad $T$ was defined to be \emph{semi-flexible} if the map $q : \pscl A\to A$ is an equivalence in the 2-category $T\algs$ of $T$-algebras and strict morphisms, and \emph{flexible} if this $q$ has a section (making it automatically also an equivalence).
  In~\cite{lack:htpy-2monads} it was shown that for a suitable 2-monad $T$, the category $T\algs$ admits a model structure whose ``homotopy 2-category'' is $T\alg$; whose weak equivalences and fibrations are the strict $T$-algebra morphisms that are equivalences and fibrations, respectively, in the underlying 2-category; and in which the cofibrant objects are precisely the flexible ones.
  Thus, when $T$ is the monad whose algebras are presheaves, this model structure is the projective one.
  Our characterization of projectively cofibrant objects is directly inspired by, and generalizes, this result of~\cite{lack:htpy-2monads}.
\end{rmk}

In fact this entire discussion works for any suitable adjunction $F\adj U$, with the case of most interest being when the adjunction is monadic (as in the case of presheaves).
However, in the case of presheaves, the forgetful functor $U$ is also \emph{comonadic}, which means that the entire discussion can be dualized, producing a characterization of the \emph{injectively fibrant} objects.
And with a little extra work we can generalize this to a characterization of all the injective fibrations using a \local and \stratified \nfs.

In our formal treatment, however, it is easier not to explicitly discuss homotopy coherent natural transformations at all, but rather focus on the bar construction.
In the rest of this section we give an abstract formulation of the bar construction, following~\cite{rv:hcadj-ftm,rv:elements}, including proofs of its (well-known) basic properties.

By~\cite{ss:free-adj,rv:hcadj-ftm}, the \textbf{free adjunction} $\cAdj$ is a 2-category freely generated by two objects $\zero$ and $\one$ (called $+$ and $-$ in~\cite{rv:hcadj-ftm}) and a morphism $f:\zero\to \one$ with right adjoint $u:\one\to \zero$.
Its hom-categories are:
\begin{itemize}
\item $\cAdj(\zero,\zero) = \dDelta_+$, the augmented simplex category, which can be identified with the category of (possibly-empty) finite ordinals and monotone maps.
\item $\cAdj(\one,\one) = \dDelta_+\op$, which can be identified with the category of non-empty finite ordinals and monotone maps that preserve the top and bottom elements.
  This description of its full subcategory $\dDelta\op$, corresponding to the ordinals with \emph{distinct} top and bottom elements, is well-known.
\item $\cAdj(\zero,\one) = \dDelta_\top$, the category of non-empty finite ordinals and monotone maps that preserve the top element.\footnote{This is the convention of~\cite{ss:free-adj}, which is the opposite of that of~\cite{rv:hcadj-ftm}.
  The choice is essentially arbitrary, since $\dDelta_\top\cong\dDelta_\bot$ by reversing the order of finite ordinals.}
\item Similarly, $\cAdj(\one,\zero) = \dDelta_\bot$, the category of non-empty finite ordinals and monotone maps that preserve the bottom element.
\end{itemize}
Similarly to the identification of $\dDelta_+\op$ as the subcategory of $\dDelta_+$ consisting of maps preserving the top and bottom elements, we have an isomorphism $\dDelta_\top \cong \dDelta_\bot\op$; see~\cite[Observation 3.3.6]{rv:hcadj-ftm} for a detailed discussion.

\begin{defn}
  Given an adjunction $F:\N \toot \M:U$, determining a 2-functor $V : \cAdj\to\cCat$ with $V(\zero)=\N$ and $V(\one)=\M$, the \textbf{two-sided simplicial bar construction} is the composite
  \[ \dDelta\op \into \dDelta_+\op = \cAdj(\one,\one) \xto{V} \cCat(\M,\M) \]
  rearranged into a functor
  \[ \sbar(F,UF,U-) : \M \to \pr\dDelta\M \]
  sending each object $X\in \M$ to a simplicial object $\sbar(F,UF,UX)\in \pr\dDelta\M$.
\end{defn}

By inspection,
\( \bar_n(F,UF,UX) = \overbrace{(FU)\dotsm(FU)}^{n+1} X \)
with faces and degeneracies obtained respectively from the counit $FU\to\Id_\M$ and unit $\Id_\N\to UF$.
Thus our bar construction specializes to the classical ones for monads~\cite{may:goils,meyer:bar_i}, enriched categories~\cite{cp:hcct,shulman:htpylim}, and internal categories~\cite{may:csf,horel:model-intsscat}.
The extension to $\dDelta_+\op$ gives an augmentation that is just the counit $FUX \to X$.

Now recall that for any category \N, the category $\pr\dDelta\N$ of simplicial objects in \N has a \textbf{canonical enrichment} over the category $\S=\pr\dDelta\nSet$ of simplicial sets, %  with hom-objects
% \[ \underline{\pr\dDelta\N}(X_\bullet, Y_\bullet)_n =
%   \int_{k\in \dDelta} \N(X_k,Y_k)^{\dDelta(k,n)}.
% \]
% This enrichment is
defined most easily in terms of its powers and copowers, which exist whenever \N is complete and cocomplete:
\[
  (K\cpw X_\bullet)_n = K_n \cdot X_n \qquad
  (K\pow Y_\bullet)_n = Y_n^{K_n}.
\]

\begin{lem}\label{thm:bar-simpcontr}
  For any adjunction $F\adj U$, the composite
  \[ \M \xto{\sbar(F,UF,U-)} \pr\dDelta\M \xto{U} \pr\dDelta\N \]
  is naturally simplicially contractible to $U$.
  In other words, the augmentation
  \[ \ep : U\sbar(F,UF,U-) = \sbar(UF,UF,U-) \too U \]
  has a natural section $\sigma:U\to \sbar(UF,UF,U-)$ (so that $\ep\sigma=\id$), with a homotopy $H:\sigma\ep \sim \id$ in the canonical enrichment of $\pr\dDelta\N$ such that $H\sigma$ is constant.
\end{lem}
\begin{proof}
  This is a classical fact (e.g.~\cite[Proposition 9.8]{may:goils} or~\cite[\sect6--7]{meyer:bar_i}), traditionally proven by exhibiting an ``extra degeneracy'', defining generators for a homotopy, and checking some simplicial identities (or leaving them to the reader).
  We repackage this a bit more abstractly, following~\cite{rv:hcadj-ftm} and~\cite[Chapters 9 and 10]{rv:elements} but working with simplicially enriched categories rather than quasicategories.

  We start by applying the 2-functor $V : \cAdj\to\cCat$ to a different hom-category:
  \[ \cAdj(\one,\zero) \xto{V} \cCat(\M,\N). \]
  This extends the augmented simplicial object $\sbar(UF,UF,U-) \to U$ to a diagram on the category $\cAdj(\one,\zero) = \dDelta_\bot$ of non-empty finite ordinals and bottom-preserving monotone maps.
  (The maps that also preserve the top element are those of the original augmented simplicial object; the rest are the ``extra degeneracy''.
  The whole $\dDelta_\bot$-diagram is the image of the comparison functor of~\cite[\sect 7]{rv:hcadj-ftm} sending each $X\in\M$ to the ``homotopy coherent $U F$-algebra''~\cite[eq.~(6.1.12)]{rv:hcadj-ftm} associated to the strict $U F$-algebra $U X$.)
  % Note that this category $\cAdj(\one,\zero)$ can also be identified with opposite of the category $\dDelta_*$ whose objects are the standard simplices $\Delta[n] \in \S = \pr\dDelta\nSet$, considered as pointed by their zeroth vertex, and whose morphisms are required to preserve these basepoints.
  Thus it suffices to prove the following lemma.
\end{proof}

\begin{lem}
  The underlying simplicial object of \emph{any} $X\in \func{\dDelta_\bot}{\N}$ is naturally simplicially contractible to its augmentation $X_{-1}$.
\end{lem}
\begin{proof}
  Again, this is a classical fact in homotopy theory (more recently stated \io-categorically as~\cite[Lemma 6.1.3.16]{lurie:higher-topoi} and~\cite[Theorem 5.3.1]{rv:2cat-quasi}); we simply repackage it abstractly.
  Let $\cR$ be the simplicially enriched category generated by a strong deformation retraction.
  It has two objects $\iI$ and $\iR$ with hom-objects:
  \begin{itemize}
  \item $\cR(\iI,\iR) = \Delta[0]$, the discrete simplicial set with one vertex $s$.
  \item $\cR(\iR,\iI) = \Delta[0]$, discrete on one vertex $e$.
  \item $\cR(\iI,\iI) = \Delta[0]$, discrete on one vertex $\id_\iI$ (so in particular $e s = \id_{\iI}$).
  \item $\cR(\iR,\iR) = \Delta[1]$, a 1-simplex from $\id_\iR$ to the composite $s e$.
  \end{itemize}
  Thus the objects of the simplicially enriched presheaf category $\pr\cR\S$ are pairs of simplicial sets related by a strong deformation retraction.
  In particular, the representable $\cR(-,\iI)$ is the identity retraction from $\Delta[0]$ to itself, while the representable $\cR(-,\iR)$ is the retraction of $\Delta[1]$ onto its right-hand vertex.

  Now, as for any simplicially enriched category \cR (see e.g.~\cite[C2.5.3]{ptj:elephant}), $\pr\cR\S$ can be identified with an \emph{unenriched} presheaf category $\pr{(\drr)}\nSet$.
  In our case, $\drr$ is the full subcategory of $\pr\cR\S$ on the objects $\cR(-,\iI)\times \Delta[n]$ (the identity retraction from the $n$-simplex $\Delta[n]$ to itself) and $\cR(-,\iR)\times \Delta[n]$ (the retraction of the prism $\Delta[n]\times\Delta[1]$ onto its right-hand $n$-simplex face).

  Here we use the combinatorial input: $\Delta[n]\times\Delta[1]$ contains $\Delta[n+1]$ as a retract, where the left-hand face of the prism is also a face of $\Delta[n+1]$, and the final vertex of $\Delta[n+1]$ lies in the right-hand face.
  % Moreover, this retraction is natural with respect to maps of prisms that preserve the deformation retraction onto the right-hand face.
  Thus, we can define a functor from $\drr$ to the category of based simplices $(\Delta[n],n)$ whose basepoint is their last vertex, by
  \begin{align*}
    \cR(-,\iR)\times \Delta[n] &\;\mapsto\; (\Delta[n+1],n+1)\\
    \cR(-,\iI)\times \Delta[n] &\;\mapsto\; (\Delta[0],0).
  \end{align*}
  However, this category of based simplices is isomorphic to $\dDelta_\top$, which as we noted above is isomorphic to $\dDelta_\bot\op$.
  Thus by composition with it we obtain a functor
  \[ \func{\dDelta_\bot}{\nSet} \simeq \func{\dDelta_\top\op}{\nSet} \to \pr{(\drr)}{\nSet} \simeq \pr\cR\S \]
  extracting from any $\dDelta_\bot$-diagram of sets a strong deformation retraction of simplicial sets.
  Finally, by composing with the Yoneda embedding of \N we obtain
  \begin{multline*}
  \func{\dDelta_\bot}{\N} \to \func{\dDelta_\bot}{\pr\N\nSet}
    \simeq \\ \pr\N{\func{\dDelta_\bot}{\nSet}}
    \to \pr\N{\pr\cR\S} \simeq \pr\cR{\pr\N\S}
  \end{multline*}
  sending any $X\in\func{\dDelta_\bot}{\N}$ to a strong deformation retraction of \S-valued presheaves on \N.
  The two \S-valued presheaves involved in this retraction are respectively representable by the underlying simplicial object of $X$ and the constant simplicial object at $X_{-1}$; thus the strong deformation retraction lies entirely in $\pr\dDelta\N$.
\end{proof}

If \M is simplicially enriched and cocomplete with copowers, the \textbf{geometric realization} of $X_\bullet \in \pr\dDelta\M$ is the coend
\( |X_\bullet| = \int^{n\in\dDelta} \Delta[n] \cpw X_n \),
and the \textbf{(realized) two-sided bar construction} is the geometric realization of the simplicial one:
\[ \bar(F,UF,U-) = |\sbar(F,UF,U-)|. \]
By~\cite[Proposition 5.4]{rss:simp}, geometric realization is a simplicially enriched functor $\pr\dDelta\M \to \M$ when \M has its given enrichment while $\pr\dDelta\M$ has its above {canonical} enrichment. % (which is independent of the given enrichment of \M)
Thus we obtain:

\begin{cor}\label{thm:bar-repl}
  For any simplicially enriched adjunction $F\adj U$ such that $U$ preserves geometric realizations, the two-sided bar construction defines a functor
  \[ \bar(F,UF,U-) : \M \to \M \]
  with a natural augmentation $\bar(F,UF,UX)\to X$ whose image under $U$ is a simplicial strong deformation retraction in \N, and hence a weak equivalence if \N is a simplicial model category.
\end{cor}
\begin{proof}
  Since $U$ preserves geometric realization, $U\bar(F,UF,U-)$ is the geometric realization of $U\sbar(F,UF,U-)$.
  By \cref{thm:bar-simpcontr} the latter is simplicially contractible to $U$ in $\pr\dDelta\N$.
  Thus, since geometric realization is a simplicial functor, $U\bar(F,UF,U-)$ is simplicially contractible to $U$ in $\N$.
  The last statement follows since simplicial homotopy equivalences in a simplicial model category are weak equivalences.
\end{proof}

\begin{rmk}
  If the adjunction is monadic, then $U$ preserves geometric realizations just when the corresponding monad does.
  For us this will usually be because it is a simplicial left adjoint, but there are non-left-adjoint monads that preserve geometric realizations; notably, those associated to operads, as in~\cite[Theorem 12.2]{may:goils} where the monadic two-sided bar construction was first introduced.
\end{rmk}

%%% Local Variables:
%%% mode: latex
%%% TeX-master: "univinj"
%%% End:
