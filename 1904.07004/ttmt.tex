\section{Type-theoretic model toposes}
\label{sec:ttmt}

%~\cite{ak:htmtt,shulman:invdia,gk:univlcc,ls:hits}
Numerous authors have defined classes of model categories that are well-adapted to type theory; \cref{thm:uf-fibrant} suggests a particularly strong one.

\begin{defn}\label{defn:ttmt}
  A \textbf{\ttmt} is a model category $\E$ such that:
  \begin{enumerate}
  \item \E is a Grothendieck 1-topos.
  \item The model structure is right proper, %\footnote{Since the cofibrations are the monomorphisms, all objects are cofibrant, so left properness is automatic; thus the content is in right properness.}
    simplicial, and combinatorial, and its cofibrations are the monomorphisms (hence it is also left proper).
    That is, \E is a right proper simplicial Cisinsiki model category.
  \item \E is \slcc.\footnote{That is, each pullback functor $f^* :\E/Y \to \E/X$ has a \emph{simplicially enriched} right adjoint.
      Since it is always a simplicial functor and has an ordinary right adjoint (since toposes are locally cartesian closed), this is equivalent to its preserving simplicial copowers.}
  \item There is a \local and \stratified \nfs \F on \E such that $\uly\F$ is the class of all fibrations.
  \end{enumerate}
\end{defn}

By \cref{thm:uf-fibrant}, for any \ttmt \E there is a \la such that \E has a fibrant univalent universe of relatively \ka-presentable fibrations for any $\ka\shgt\la$.
It also has all the other requisite structure to model type theory:
\begin{itemize}
\item % Pullback along any fibration has a right adjoint (because \E is locally cartesian closed) and preserves acyclic cofibrations (by right properness together with the fact that cofibrations are always stable under pullback).
  % Thus
  \E is a \emph{logical model category} in the sense of~\cite{ak:htmtt}, hence models $\Sigma$- and $\Pi$-types (and also a unit type and identity types, although these were not discussed in~\cite{ak:htmtt}).
  Categorically, $\Sigma$-types correspond to composition of fibrations, the unit type corresponds to the identity map as a fibration, $\Pi$-types correspond to dependent product of one fibration along another, and identity types correspond to path objects as in~\cite{aw:htpy-idtype}.

  One also needs a coherence theorem to strictify \E into an actual model of type theory.
  With inaccessible universes, we can use the method of~\cite{klv:ssetmodel}; otherwise we can use the ``local universes'' technique of~\cite{lw:localuniv} (see also~\cite{awodey:natmodels}).
\item %Since monomorphisms are closed under arbitrary limits,
  \E is a \emph{type-theoretic model category} in the sense of~\cite{shulman:invdia}, hence in particular its $\Pi$-types satisfy function extensionality.
  Categorically, function extensionality means that for any fibration $f$, the adjunction $f^*\adj f_*$ is a Quillen adjunction, with $f_*$ preserving both fibrations and acyclic fibrations.
\item \E is a \emph{good model category} as in~\cite{ls:hits}, hence models higher inductive pushouts (obtained as fibrant replacements of explicit simplicial homotopy pushouts) and other non-recursive (higher) inductive types such as the empty type, the boolean type, coproduct types, circles, spheres, tori, and so on.
\item \E is also an \emph{excellent model category}\footnote{No relation to the ``excellent model categories'' of~\cite[A.3.2.16]{lurie:higher-topoi}.} in the sense of~\cite{ls:hits}, hence models many other higher inductive types, including the natural numbers, $W$-types, truncations, and localizations.
  These are obtained by mixing a fibrant replacement monad with a ``cell monad'' built from polynomial endofunctors.
  Strictification for these types is discussed in~\cite{ls:hits} using the method of~\cite{lw:localuniv}, but adapts easily to the universe method of~\cite{klv:ssetmodel}.
\end{itemize}

We also expect universes in type theory to be \emph{closed} under all the relevant type constructors.
Since our universes classify all relatively \ka-presentable fibrations, as in~\cite{klv:ssetmodel} this will be true if the corresponding categorical operations preserve relatively \ka-presentable fibrations.
\begin{itemize}
\item By \cref{thm:relpres-comp}, $\Sigma$-types preserve relatively \ka-presentable morphisms for any regular cardinal \ka.
\item Likewise, identity maps (i.e.\ the unit type) are always relatively \ka-presentable.
\item If we define identity types as powers by $\Delta[1]$, then by \cref{thm:relpres-pow} there is a \la such that they preserve relatively \ka-presentable morphisms for any $\ka\shgt\la$.
\item By \cref{thm:relpres-dp}, $\Pi$-types preserve relatively \ka-presentable morphisms if $\ka$ is sufficiently large and inaccessible.
\item We can also choose \ka large enough that the universe will contain any fixed collection of (higher) inductive types, such as the empty type, the boolean type, the natural numbers, circles, spheres, tori, and so on.
\end{itemize}
It is not yet known how to obtain universes closed under parametrized (higher) inductive types such as $W$-types and pushouts, since fibrant replacement need not preserve relatively \ka-presentable morphisms.
However, in the special case of binary coproducts, once we have universes we can use the trick of defining $A+B = \sum_{x:\bool} \mathsf{rec}_\bool(U,A,B)$, where $\bool$ is the boolean type:\footnote{I am indebted to Bas Spitters for pointing this out.}
%Categorically, the argument becomes the following.

\begin{prop}\label{thm:relpres-coprod}
  In a \ttmt, there exists a regular cardinal \la such that for any $\ka\shgt\la$, if $X\fib Z$ and $Y\fib Z$ are relatively \ka-presentable fibrations with fibrant codomain, then their copairing $X\amalg Y \to Z$ factors as an acyclic cofibration $X\amalg Y \acof P$ followed by a fibration $P\fib Z$ that is again relatively \ka-presentable.
\end{prop}
\begin{proof}
  Let $\bool$ denote a fibrant replacement of $1\amalg 1$, so we have an acyclic cofibration $1\amalg 1\acof \bool$.
  We let \la be such that $\bool$ is \la-presentable and \E has a fibrant univalent universe of relatively \ka-presentable fibrations for any $\ka\shgt\la$.
  Let $\pi:\Util\fib U$ be such a universe, and let $x,y:Z \toto U$ be classifying maps for $X$ and $Y$ respectively.
  Then since $Z$ is fibrant, the map $Z\amalg Z \cong Z\times (1\amalg 1) \acof Z\times \bool$ is again an acyclic cofibration, so the map $[x,y] : Z\amalg Z \to U$ extends to a map $w:Z\times \bool \to U$.

  Let $P = w^* \Util$; then the composite $P \fib Z\times \bool \fib Z$ is a fibration (since $\bool$ is fibrant).
  And it is relatively \ka-presentable, since $\pi$ is relatively \ka-presentable, \bool is \ka-presentable, and relatively \ka-presentable morphisms are closed under pullback and composition.
  Moreover, in the following diagram:
  \[
    \begin{tikzcd}
      X\amalg Y \ar[d,two heads] \ar[r] \drpullback & P \ar[d,two heads] \ar[r]\drpullback & \Util \ar[d,two heads,"\pi"]\\
      Z\amalg Z \ar[r,tail,"\sim"'] &  Z\times \bool \ar[r,two heads,"w"'] & U
    \end{tikzcd}
  \]
  the outer rectangle and right-hand square are pullbacks (the former since \E is extensive), hence so is the left-hand square.
  Since $P\to Z\times \bool$ is a fibration, this implies that $X\amalg Y \to P$ is, like $Z\amalg Z\acof Z\times \bool$, an acyclic cofibration.
\end{proof}

In summary, we have the following.

\begin{thm}\label{thm:ttmt-models}
  For any \ttmt \E, there is a regular cardinal \la such that \E interprets Martin-L\"{o}f type theory with the following structure:
  \begin{enumerate}
  \item $\Sigma$-types, a unit type, $\Pi$-types with function extensionality, identity types, and binary sum types.\label{item:sigpiid}
  \item The empty type, the natural numbers type, the circle type $S^1$, the sphere types $S^n$, and other specific ``cell complex'' types such as the torus $T^2$.\label{item:unparam-hits}
  \item As many universe types as there are inaccessible cardinals larger than $\la$, all closed under the type formers~\ref{item:sigpiid} and containing the types~\ref{item:unparam-hits}, and satisfying the univalence axiom.\label{item:univ}
  \item $\mathsf{W}$-types, pushout types, truncations, localizations, James constructions, and many other recursive higher inductive types.\label{item:hits}
  \end{enumerate}
\end{thm}
\begin{proof}
  We have already noted that \E has all the structure in~\ref{item:sigpiid}, \ref{item:unparam-hits}, and~\ref{item:hits}.
  Let \la satisfy \cref{thm:uf-fibrant,thm:relpres-pow,thm:relpres-coprod} for \E, and also be such that the unparametrized higher inductive types in~\ref{item:unparam-hits} are \la-presentable.
  Then if $\ka>\la$ is inaccessible (hence in particular $\ka\shgt\la$), the universe for relatively \ka-presentable fibrations is univalent and closed under~\ref{item:sigpiid} by the above remarks.
  Moreover, since each type $X$ in~\ref{item:unparam-hits} is \la-presentable, hence \ka-presentable, and the \ka-presentable objects are closed under finite limits, the morphism $X\to 1$ is relatively \ka-presentable and hence can be classified by all these universes.

  To complete the proof it is necessary to apply a coherence theorem to replace \E by a strict model of type theory (such as a category with families or contextual category).
  Unfortunately, the coherence theorem of~\cite{klv:ssetmodel} deals with only one internal universe and requires an additional inaccessible outside the model, while that of~\cite{lw:localuniv} does not mention universes at all.
  Thus, in \cref{sec:coherence} we extend the latter to handle an arbitrary family of universes.
\end{proof}

A \ttmt is also a \emph{combinatorial type-theoretic model category} as in~\cite{gk:univlcc}, hence the \io-category it presents is locally presentable and locally cartesian closed.
In fact, it is a Grothendieck \io-topos, as we now show.

\begin{thm}\label{thm:descent}
  A \ttmt has descent in the sense of~\cite{rezk:homotopy-toposes}.
\end{thm}
\begin{proof}
  As defined in~\cite{rezk:homotopy-toposes}, descent consists of two conditions.
  Condition (P1) says that homotopy colimits are stable under homotopy pullback.
  Since \E is right proper, it suffices to show that for any fibration $f:X\fib Y$, the pullback functor $f^* : \E/Y \to \E/X$ preserves homotopy colimits.
  But it preserves cofibrations (as these are the monomorphisms) and weak equivalences (by right properness), and has a right adjoint (since toposes are locally cartesian closed).
  Thus it is a left Quillen functor, hence preserves homotopy colimits.

  Condition (P2) says that if $f:X\to Y$ is a map between homotopy colimits $X = \hocolim_i X_i$ and $Y = \hocolim_i Y_i$ induced by a natural transformation $f_i : X_i\to Y_i$ such that each square on the left below is a homotopy pullback (i.e.\ the transformation is ``cartesian'' or ``equifibered''):
  \begin{equation}
    \begin{tikzcd}
      X_i \ar[r] \ar[d,"f_i"'] & X_j \ar[d,"f_j"]\\
      Y_i \ar[r] & Y_j
    \end{tikzcd}
    \hspace{2cm}
    \begin{tikzcd}
      X_i \ar[r] \ar[d,"f_i"'] & X \ar[d,"f"]\\
      Y_i \ar[r] & Y,
    \end{tikzcd}\label{eq:descent}
  \end{equation}
  then the squares on the right above are also homotopy pullbacks.
  To prove this, it suffices to consider coproducts and pushouts. %, since all homotopy colimits can be constructed from these.

  For coproducts, we may assume given fibrations $f_i : X_i \fib Y_i$; equifiberedness is vacuous, and all coproducts are homotopy colimits since all objects are cofibrant.
  Choosing an \F-structure on each $f_i$, \locality of \F induces an \F-structure on $f$; hence it is a fibration.
  And the squares on the right in~\eqref{eq:descent} are strict pullbacks in \E since it is a 1-topos; thus they are also homotopy pullbacks.

  For pushouts, suppose given the solid arrows below, where the vertical maps $f_i:X_i\to Y_i$ are fibrations, and the maps $X_0\cof X_2$ and $Y_0\cof Y_2$ are cofibrations.
  \[ \begin{tikzcd}[row sep=small,column sep=small]
      X_0 \arrow[rr, tail] \arrow[dd, two heads,"f_0"'] \arrow[rd] &  & X_2 \arrow[dd, two heads,near end,"f_2"] \arrow[rd, dashed] &  \\
      & X_1 \arrow[rr, dashed, crossing over]  &  & X \arrow[dd, two heads, dashed,"f"] \\
      Y_0 \arrow[rr, tail] \arrow[rd] &  & Y_2 \arrow[rd, dashed] &  \\
      & Y_1 \arrow[rr, dashed] \arrow[from=uu,crossing over, two heads,near start,"f_1"] &  & Y
    \end{tikzcd}
  \]
  Equifiberedness means the two squares of solid arrows are homotopy pullbacks, i.e.\ the maps $X_0 \to X_1\times_{Y_1} Y_0$ and $X_0 \to X_2\times_{Y_2} Y_0$ are equivalences.
  Up to homotopy, we can therefore replace $X_0$ with $X_1\times_{Y_1} Y_0$ to make the left-hand face of the cube a strict pullback, and then by \cref{thm:u4p} we can replace $f_2$ by an equivalent fibration making the back face of the cube also a strict pullback.

  Now we take the strict pushouts, which are also homotopy pushouts since $X_0\cof X_2$ and $Y_0\cof Y_2$ are cofibrations.
  Since toposes are adhesive~\cite{ls:topadh}, the front and right-hand faces of the resulting cube are also strict pullbacks.

  Choose an \F-structure on $f_1$, inducing one on $f_0$ making the left-hand face an \F-morphism.
  Since \F is \stratified, we can give $f_2$ an \F-structure making the back face also an \F-morphism.
  Thus, since \F is \local, by \cref{thm:local} there is an induced \F-structure on $f$, so that in particular $f$ is a fibration.
  So the front and right-hand faces of the cube, being strict pullbacks of $f$, are homotopy pullbacks.
\end{proof}

Recall from~\cite{rezk:homotopy-toposes} that a \textbf{model topos} is a model category that is Quillen equivalent to a left exact left Bousfield localization of the projective model structure on a category of simplicial presheaves.
This implies that the \io-category it presents is a Grothendieck \io-topos.

\begin{cor}\label{thm:ttmt-mt}
  Every \ttmt is a model topos.
\end{cor}
\begin{proof}
  Since it is combinatorial, by~\cite{dug:pres} it has a small presentation, and by \cref{thm:descent} it has descent.
  Thus we can apply~\cite[Theorem 6.9]{rezk:homotopy-toposes} or~\cite[Theorem 6.1.0.6]{lurie:higher-topoi}.
\end{proof}

\begin{rmk}
  On one hand, we have seen that a \ttmt \E has univalent universes classifying the relatively \ka-presentable fibrations for arbitrarily large regular cardinals \ka.
  On the other hand, since the \io-category $\Ho_\oo(\E)$ presented by \E is a Grothendieck \io-topos, by~\cite[Theorem 6.1.6.8]{lurie:higher-topoi} it contains object classifiers for the relatively \ka-presentable morphisms for arbitrarily large regular cardinals \ka.
  % \footnote{This is claimed in~\cite[Theorem 6.1.6.8]{lurie:higher-topoi} for all sufficiently large \ka, but the proof (which uses the \io-categorical version of \cref{thm:pres-pb}) only shows it for a \shrp class.}
  One expects the univalent universes in \E to present the object classifiers in $\Ho(\E)$, but this is a subtle question because the relationship between \ka-presentability in \E and in $\Ho_\oo(\E)$ is nontrivial; see~\cite{MO:cptobjio,stenzel:thesis}.
\end{rmk}

\cref{thm:ttmt-mt} shows that \ttmts are, as the term suggests, a subclass of model toposes that are particularly well-adapted to model type theory.
Our main goal is to show that up to homotopy, this subclass involves no loss of generality: every model topos is Quillen equivalent to a type-theoretic one.
We begin with some easy cases.

\begin{prop}\label{thm:ss-ttmt}
  The category \S of simplicial sets, with its Kan-Quillen model structure, is a \ttmt.
\end{prop}
\begin{proof}
  It is a presheaf topos and its tmodel structure is right proper, simplicial, and combinatorial, with its cofibrations the monomorphisms.
  It is \slcc because its copowers are just cartesian products, which are preserved by pullback.
  Finally, $\F=\dFib$ is \local by \cref{eg:rep-cod}, and trivially \stratified.
\end{proof}

Combining \cref{thm:ss-ttmt,thm:uf-fibrant} reproduces Voevodsky's construction~\cite{klv:ssetmodel} of a univalent universe in simplicial sets.

\begin{rmk}
  Specializing the argument for pushouts in \cref{thm:descent} to $\E=\S$, we obtain a proof of the ``cube theorem''~\cite{puppe:cube} for \S.
  A similar proof appears in~\cite[Lemma 6.1.3.12]{lurie:higher-topoi} using minimal fibrations; \cref{thm:u4p} avoids these % (which are hard to generalize to categories other than \S)
  by providing a different way to turn homotopy pullbacks into strict ones.
\end{rmk}

\begin{prop}\label{thm:slice-ttmt}
  If \E is a \ttmt and $X\in \E$, then $\E/X$ is also a \ttmt.
\end{prop}
\begin{proof}
  A slice of a topos is again a topos, and the slice model structure inherits properness, simplicial-ness, combinatoriality, and cofibrations being the monomorphisms.
  A slice of a slice is a slice, so it inherits \slcclosure.
  And if \F is the \local and \stratified \nfs for the fibrations of \E, then since the forgetful functor $U: \E/X\to \E$ creates pullbacks, colimits, and fibrations, the preimage \nfs $U^{-1}(\F)$ is a \local and \stratified \nfs for the fibrations of $\E/X$.
\end{proof}

\begin{prop}\label{thm:prod-ttmt}
  If $\{\E_i\}_{i\in I}$ is a small family of \ttmts, then the product category $\prod_i \E_i$ is also a \ttmt.
\end{prop}
\begin{proof}
  A product of toposes is a topos; all the structure is inherited pointwise.
\end{proof}

In the rest of the paper we study two more basic constructions of \ttmts that together suffice to obtain our desired generality.

\begin{rmk}\label{rmk:design-space}
  I would not claim that \ttmts are the last word in ``model categories that interpret type theory'', but they do occupy a fairly stable point in the design space: they have a nearly maximal set of good properties one can assume of a model category, are sufficient to model all \io-toposes, interpret (most of) type theory, and are closed under many constructions.

  Note that not every \ttmt is a cartesian \emph{monoidal} model category (e.g.\ this fails already for slices of simplicial sets), but every \io-topos is presented by some \ttmt that does have this property (e.g.\ a left exact localization of an injective model structure on enriched simplicial presheaves).
  Allowing \E to be an arbitrary 1-topos is also somewhat unnecessary generality; all the examples we will construct in this paper are in fact \emph{presheaf} 1-toposes.

  Finally, while I have chosen to stick with simplicial enrichments for simplicity and to facilitate the connection with classical homotopy theory, it should be possible to formulate a more general notion of \ttmt that is enriched over some other monoidal model category, such as some variety of cubical sets.
\end{rmk}

\begin{verbose}
\begin{rmk}
  I would not claim that \ttmts are the last word in definitions of ``model categories that interpret type theory'', but they do seem to occupy a fairly stable point in the design space.
  
  On one hand, they already include a nearly maximal set of the good properties that can be assumed of a model category.
  The one axiom I could imagine adding is that \E is a cartesian \emph{monoidal} model category.
  This is satisfied by left exact localizations of injective model structures on enriched simplicial presheaf categories (which suffice to model all \io-toposes).
  But it fails for slice categories and presheaves on internal categories, and I have not found any particular use for it yet.
  %(As we will note in \cref{thm:enrpre-ttmt}, it is always ``almost'' true: acyclicity of pushout products requires the objects to be sharp.)

  On the other hand, most properties of the definition are necessary for one or another of our arguments.
  \begin{itemize}
  \item Of course, the \local and \stratified \nfs is the whole point, which ensures the existence of strict univalent universes.
  \item The requirement that \E is a 1-topos whose cofibrations are the monomorphisms (a ``Cisinski model category''~\cite{cisinski:presheaves,cisinski:local-acyc}) is extremely useful to relate homotopy theory to 1-categorical descent properties; it appears in \cref{thm:u4p,thm:descent,eg:wo-strat,eg:cof-ff-fcos,eg:span}.
  \item To model $\Pi$-types by 1-categorical dependent products (which appears necessary if we want them to satisfy $\eta$-conversion as well as $\beta$-reduction) we seem to need $f_*$ to preserve fibrations whenever $f$ is a fibration, which by adjointness means that pullback along fibrations preserves acyclic cofibrations.
    Since acyclic fibrations are always stable under pullback, by factorization this implies that pullback along any fibration preserves weak equivalences, i.e.\ that \E is right proper; and the converse also holds if the cofibrations are stable under pullback (as if they are the monomorphisms).
    Note that the \io-categories presentable by right proper Cisinski model categories are precisely the locally cartesian closed locally presentable ones~\cite{cisinski:lccc-rpcmc,gk:univlcc}.
  \item Simplicial enrichment of the model structure is of course very convenient (it is used in \cref{thm:u4p}, for instance), and in \cref{sec:injmodel} it appears necessary to obtain good behavior of cobar constructions.
  \item \Slcclosure appears necessary to control cobar constructions for presheaves on internal categories, and is also used in~\cite{ls:hits} to construct higher inductive types.
  \item  Combinatoriality is used in~\cite{ls:hits} for algebraic fibrant replacements, and also implies the existence of the Bousfield localizations used in \cref{sec:lex-loc}.
  \end{itemize}
  One might imagine weakening combinatoriality to accessibility~\cite{rosicky:acc-model}, but in fact any accessible model structure on a 1-topos whose cofibrations are the monomorphisms is automatically combinatorial.
    For the monomorphisms in a 1-topos are always cofibrantly generated by a set (cf.~\cite[Proposition 1.2.27]{cisinski:presheaves},~\cite[Proposition 1.2.2]{cisinski:local-acyc}, and~\cite[Proposition 1.12]{beke:sheafifiable}); so the weak equivalences are accessible and accessibly embedded in $\E^\dtwo$, being the preimage of the small-injectivity class of acyclic fibrations~\cite[Proposition 3.3]{rosicky:comb-model} under an accessible fibration-replacement functor.
    The other hypotheses of Smith's theorem (\cite[Theorem 1.7]{beke:sheafifiable} or~\cite[Proposition 1.7]{barwick:enr-localization}) or~\cite[Theorem 1.4.3]{cisinski:presheaves} are automatically satisfied if we already have a model structure.
\end{rmk}
\end{verbose}


%%% Local Variables:
%%% mode: latex
%%% TeX-master: "univinj"
%%% End:
