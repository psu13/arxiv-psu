\section{Universes in model categories}
\label{sec:univalence}

Now let \E be a model category and \Fib the full \nfs determined by its fibrations, so that $\Fibka = \Fib\times_\cE \cEka$ denotes the relatively \ka-presentable fibrations.
In an ideal world, the object $\Fibka\in\Ehat$ would be (pseudonaturally equivalent to) a representable presheaf $\E(-,U)$.
But since \E is itself a 1-category rather than an \io-category, this is unreasonable to expect. % (unless all fibrations are monomorphisms, as in the case of the subobject classifier in a 1-topos).

Instead, we will replace \Fibka by a representable presheaf that is ``weakly equivalent'' in some sense.
We do not have a model structure on $\Ehat$ with which to make sense of this, but we can at least use the Yoneda embedding $\E\to\Ehat$ to lift the weak factorization systems of \E.

\begin{defn}\label{defn:afib}
  Let \E be a model category.
  A morphism $\dX\to\dY$ in $\Ehat$ is an \textbf{acyclic fibration} if it has the right lifting property (\cref{defn:2liftorth}) for all morphisms $\E(-,j) : \E(-,A) \to \E(-,B)$, where $j:A\to B$ is a cofibration in \E.
\end{defn}

\begin{rmk}
  If $f:\dX\to\dY$ is a representable morphism, then it is an acyclic fibration in the sense of \cref{defn:afib} if and only if in any pullback
  \begin{equation*}
    \begin{tikzcd}
      \E(-,W) \ar[r] \ar[d] \ar[dr,phantom,near start,"\lrcorner"] & \dX\ar[d,"{f}"] \\
      \E(-,Z) \ar[r] & \dY
    \end{tikzcd}
  \end{equation*}
  the induced map $W\to Z$ is an acyclic fibration in \E.
  This fits a standard pattern for extending pullback-stable properties of morphisms in \E to properties of representable morphisms in \Ehat.
  
  Note that this notion of (acyclic) fibration based on the model structure of \E is unrelated to the 2-categorical notions of (strict, discrete) fibration defined in \cref{sec:2cat}.
\end{rmk}

\begin{defn}
  If \F is a \nfs on \E, a \textbf{universe} for \F is a cofibrant object $U\in\E$ equipped with an acyclic fibration $\E(-,U) \to \F$ in \Ehat.
\end{defn}

That is, a universe is a sort of ``cofibrant replacement'' of $\Fibka$.
(This perspective was introduced informally in~\cite[\sect 3]{shulman:elreedy}.)

\begin{rmk}\label{rmk:universe}
  When $U$ is a universe, the morphism $\E(-,U) \to \F$ corresponds by the Yoneda lemma to an \F-algebra $\pi:\Util\to U$.
  And the fact that the morphism $\E(-,U) \to \F$ is an acyclic fibration means that given the solid arrows below, where $i:A\mono B$ is a cofibration, $f:X\to B$ is an \F-algebra, and both squares of solid arrows are pullbacks and \F-morphisms:
  \begin{equation*}
    \begin{tikzcd}[row sep=small, column sep=small]
      i^*(X) \arrow[rr] \arrow[dd, "g"'] \arrow[rd] &  & \Util \arrow[dd, "\pi"] \\
      & X \arrow[ru, dashed] &  \\
      A \arrow[rd, "i"', tail] \arrow[rr, near start, "h" description] &  & U \\
      & B \arrow[ru, dashed] \arrow[from=uu, near start, "f" description, crossing over] & 
    \end{tikzcd}%\label{eq:u2p}
  \end{equation*}
  there exist the dashed arrows rendering the diagram commutative and the third square also a pullback and an \F-morphism.
  This property of a universe was first noted in the proof of~\cite[Theorem 2.2.1]{klv:ssetmodel} and isolated more abstractly in~\cite[(2$'$)]{shulman:elreedy}, \cite[Corollary 3.11]{cisinski:elegant}, and~\cite{stenzel:thesis} under varying names.

  If the initial object $\emptyset$ is strict (i.e.\ every morphism with codomain $\emptyset$ is an isomorphism) and $\id_\emptyset$ has a unique \F-structure, this implies that every \F-algebra with cofibrant codomain is a pullback of $\pi$ (though not in a unique way).
\end{rmk}

As usual, when \E is cofibrantly generated we can hope to produce such a cofibrant replacement by the small object argument.
However, the colimits in \E used to build cell complexes are no longer colimits in \Ehat; thus we have to restrict to the objects of \Ehat that preserve these particular colimits.

\begin{defn}
  Let \E be a model category.
  We say $\dX\in\Ehat$ is a \textbf{stack for cell complexes} if as a pseudofunctor $\dX:\E\op\to\cGPD$ it preserves (in the weak bicategorical sense) coproducts, pushouts of cofibrations, and transfinite composites of cofibrations.
\end{defn}

\begin{eg}\label{eg:topos-stack}
  If \E is a Grothendieck 1-topos and all cofibrations are monomorphisms, then the trivial \nfs \cE is a stack for cell complexes.
  This because any topos is infinitary extensive~\cite{clw:ext-dist}, adhesive~\cite{ls:adhesive,ls:topadh}, and exhaustive~\cite{nlab:exhaustive,shulman:elreedy}; see~\cite[\sect3]{shulman:elreedy} and~\cite[Lemma 7.5]{sattler:eqvext}.
  More generally,
  \cE being a stack for cell complexes is one of the conditions for the (cofibration, acyclic fibration) weak factorization system of \E to be \emph{suitable} as in~\cite[Definition 3.2]{sattler:eqvext}.
\end{eg}

\begin{lem}\label{thm:nfs-stack}
  If \cE is a stack for cell complexes and $\phi:\F\to\cE$ is a \local \nfs, then \F is also a stack for cell complexes.
\end{lem}
\begin{proof}
  Let \sQ be the class of morphisms $q:\Yhat \to \E(-,Y)$ from \cref{eg:colim-orth} where $Y= \colim_i Y_i$ ranges over coproducts, pushouts of cofibrations, and transfinite composites of cofibrations.
  Since \cE is a stack for cell complexes, $\sQ \perp \cE$; and since \F is \local, by \cref{thm:local} we have $\sQ\perp\phi$.
  Hence $\sQ\perp \F$.
\end{proof}

\begin{lem}\label{thm:icell-acyc}
  If \E is cofibrantly generated with \cI a set of generating cofibrations, \dX and \dY are stacks for cell complexes, and $f:\dX\to\dY$ has the right lifting property in \Ehat against all morphisms $\E(-,j):\E(-,A)\to\E(-,B)$ where $j:A\to B$ is in \cI, then $f$ is an acyclic fibration.
\end{lem}
\begin{proof}
  As usual, any cofibration is a retract of a transfinite composite of pushouts of coproducts of elements of \cI.
  Thus, given that \dX and \dY are stacks for cell complexes, the lifting property carries through all these operations in the usual way.
\end{proof}

We call a pseudofunctor $\dZ\in\Ehat$ \textbf{small-groupoid-valued} if each groupoid $\dZ(A)$ is essentially small.
Note that by definition, for any \nfs \F the map $\phi:\F\to\cE$ has small \emph{fibers}, i.e.\ any given morphism $f:X\to Y$ has a small set of \F-structures; but \F is only small-groupoid-valued if any given object $Y\in\E$ there is a small set of isomorphism classes of \F-algebras with codomain $Y$.
In general we will achieve this by considering $\Fka = \dF\times_\cE \cEka$ as in \cref{sec:relpres}.

\begin{thm}\label{thm:2cat-soa}
  Let \E be a combinatorial model category, and $\dZ\in\Ehat$ a small-groupoid-valued stack for cell complexes.
  Then any morphism $f:\E(-,X) \to\dZ$ in \Ehat factors, up to isomorphism, as $\E(-,X) \xto{\E(-,j)} \E(-,Y) \xto{p} \dZ$, where $j$ is a cofibration in \E and $p$ is an acyclic fibration in \Ehat.
\end{thm}
\begin{proof}
  This is just a bicategorical adaptation of the small object argument.
  Let \cI be a set of generating cofibrations for \E; we will define an \cI-cell complex sequence $X_0 \to X_1 \to \cdots$ in \E, along with maps $f_n : \E(-,X_n) \to \dZ$ and coherent isomorphisms $f_n \circ j_{m,n} \cong f_m$.
  We start with $X_0 = X$ and $f_0 = f$.
  For limit $n$ we let $X_n = \colim_{m<n} X_m$, with $f_n : \E(-,X_n) \to \dZ$ and attendant isomorphisms induced by the fact that \dZ preserves this colimit.

  At a successor stage $n+1$, we let $S_n$ be a set of representatives for isomorphism classes of pseudo-commutative squares
  \[
    \begin{tikzcd}
      \E(-,A) \ar[r] \ar[d,"i"'] \ar[dr,phantom,"\scriptstyle\Downarrow\cong"] & \E(-,X_n) \ar[d] \\
      \E(-,B) \ar[r] & \dZ
    \end{tikzcd}
  \]
  where $i\in \cI$.
  This is a small set, since \dZ is small-groupoid-valued and \E is locally small.
  Now let $X_{n+1}$ be the pushout
  \[
    \begin{tikzcd}
      \coprod_{s\in S_n} A_s \ar[r] \ar[d] \drpushout & X_n \ar[d]\\
      \coprod_{s\in S_n} B_s \ar[r] & X_{n+1}
    \end{tikzcd}
  \]
  Since \dX preserves these coproducts and pushouts, there is an essentially unique induced map $f_{n+1} : X_{n+1}\to Z$ with attendant isomorphisms.

  Finally, since \E is locally presentable, there is a regular cardinal \la such that all domains of morphisms in \cI are \la-presentable.
  Thus, in any square
  \[
    \begin{tikzcd}
      \E(-,A) \ar[r] \ar[d,"i"'] \ar[dr,phantom,"\scriptstyle\Downarrow\cong"] & \E(-,X_\la) \ar[d] \\
      \E(-,B) \ar[r] & \dZ
    \end{tikzcd}
  \]
  the top morphism $A\to X_\la$ factors through $X_n$ for some $n<\la$, and hence there is a lift $B\to X_{n+1} \to X_\la$.
  Therefore, the map $f_{\la} : X_\la \to \dZ$ has right lifting for \cI, and is thus an acyclic fibration by \cref{thm:icell-acyc}.
\end{proof}

\begin{cor}\label{thm:nfs-universe}
  If \E is a Grothendieck 1-topos with a combinatorial model structure in which all cofibrations are monomorphisms, then any small-groupoid-valued \local \nfs \F on \E has a universe.
\end{cor}
\begin{proof}
  By \cref{eg:topos-stack,thm:nfs-stack}, \F is a stack for cell complexes; thus we can apply \cref{thm:2cat-soa} to factor the map $\E(-,\emptyset) \to \F$.
\end{proof}

In some cases such as \cref{eg:pshf-can,eg:rep-cod}, \Fib is \local and hence so is \Fibka.
This includes the universes constructed in~\cite{klv:ssetmodel,shulman:elreedy,cisinski:elegant}.
However, in the general case we need a different approach: we will suppose given a non-full \nfs \F that \emph{is} \local, and an acyclic fibration $\F\to\Fib$. Thus we will be able to apply \cref{thm:nfs-universe} to \F instead.

More generally, for a \nfs \F, let $\uly\F$ denote the image of the map $\phi:\F\to \cE$.
Thus $\uly\F$ is a full \nfs (though not generally \local, even if \F is), and the $\uly\F$-algebras are the morphisms that admit some \F-structure.

\begin{defn}\label{defn:stratified}
  A \nfs \F on a model category \E is \textbf{\stratified} if the map $\F \to \uly\F$ is an acyclic fibration.
  That is, for any pullback
  \[
    \begin{tikzcd}
      X' \ar[d,"f'"'] \ar[r,"g"] \ar[dr,phantom, near start,"\lrcorner"] & X \ar[d,"f"]\\
      Y' \ar[r,"i",tail] & Y
    \end{tikzcd}
  \]
  with $f$ and $f'$ \F-algebras and $i$ a cofibration, there exists a new \F-structure on $f$ making the square an \F-morphism.
\end{defn}

\begin{prop}\label{thm:pre-u2p}
  Let \F be a \local, \stratified, small-groupoid-valued \nfs on a Grothendieck 1-topos that is a combinatorial model category whose cofibrations are monomorphisms.
  Then $\uly\F$ has a universe.
\end{prop}
\begin{proof}
  Apply \cref{thm:nfs-universe} to \F, and observe that the composite $\E(-,U) \to \F \to \uly\F$ of acyclic fibrations is again an acyclic fibration.
\end{proof}

\begin{rmk}
  By \cref{rmk:universe}, if all objects are cofibrant, $\emptyset$ is strict, and $U$ is a universe for a full \nfs \F such that $\id_\emptyset\in\F$, then in fact $\F = \uly{\dRep_\pi}$ (with $\dRep_\pi$ as in \cref{eg:rep-fcos}).
  Conversely, if \E is locally cartesian closed and $\pi:\Util\to U$ is a universe for $\uly{\dRep_\pi}$, then $\dRep_\pi$ is \local (by \cref{eg:rep-local}), \stratified, and small-groupoid-valued.
  Thus the hypotheses of \cref{thm:pre-u2p} are basically optimal.
\end{rmk}

\begin{eg}
  Full \nfss are always \stratified, as is $\F_1 \times_\cE \F_2$ if $\F_1$ and $\F_2$ are.
\end{eg}

\begin{eg}
  If \F is a \stratified \nfs on $\E_2$ and $G:\E_1\to\E_2$ preserves pullbacks (hence also monomorphisms), then $G^{-1}(\F)$ is also \stratified.
\end{eg}

\begin{eg}\label{thm:sec-afib-strat}
  Let \E be a model category and $H$ be a fibred core-endofunctor of \E such that whenever $H_Y(X)\to Y$ has a section, it is an acyclic fibration.
  Then the \nfs $H^*(\cEp)$ is \stratified.
  For given a pullback square
  \begin{equation}
    \begin{tikzcd}
      X' \ar[d,"f'"'] \ar[r,"g"] \ar[dr,phantom, near start,"\lrcorner"] & X \ar[d,"f"]\\
      Y' \ar[r,"i",tail] & Y
    \end{tikzcd}\label{eq:sec-strat-sq}
  \end{equation}
  with $i$ a cofibration, along with sections $s'$ and $s$ of $H_{Y'}(X')\to Y'$ and $H_Y(X)\to Y$, the assumption implies $H_Y(X)\to Y$ is an acyclic fibration.
  Thus we can find a lift in the square
  \[
    \begin{tikzcd}
      Y' \ar[d,"i"',tail] \ar[r,"s"] & H_{Y'}(X') \ar[r,"{H(g,i)}"] & H_Y(X) \ar[d,"\sim",two heads]\\
      Y \ar[rr,equals] & & Y
    \end{tikzcd}
  \]
  giving an $H^*(\cEp)$-structure on $f$ making~\eqref{eq:sec-strat-sq} an $H^*(\cEp)$-morphism.
\end{eg}

Recall that a \textbf{Cisinski model category}~\cite{cisinski:topos,cisinski:local-acyc} is a Grothendieck 1-topos with a combinatorial model structure whose cofibrations are \emph{precisely} the monomorphisms.

\begin{prop}\label{eg:cof-ff-fcos}
  Let \E be a Cisinski model category, \F a \stratified \nfs on \E, and $E$ a cartesian functorial factorization on \E that factors every \F-algebra as an acyclic cofibration followed by a fibration.
  Then the \nfs $\F\times_\cE \dR_E$ (where $\dR_E$ is as in \cref{eg:ff-fcos}) is also \stratified.
\end{prop}
\begin{proof}
  Suppose given the pullback square on the left:
  \[
    \begin{tikzcd}
      X' \ar[d,"f'"'] \ar[r,"g"] \ar[dr,phantom, near start,"\lrcorner"] & X \ar[d,"f"]\\
      Y' \ar[r,"i",tail] & Y\\
      \phantom{E f}
    \end{tikzcd}
    \hspace{2cm}
    \begin{tikzcd}
      X' \ar[r,"g"] \ar[d,tail,"{\lambda_{f'}}"'] \ar[dr,phantom,near end,"\ulcorner"] & X \ar[ddr,tail,"{\lambda_f}"] \ar[d] \\
      E f' \ar[r] \ar[drr,tail,"{\fact g i}"'] & P \ar[dr,"j" description]\\
      && E f.
    \end{tikzcd}
  \]
  where $f$ and $f'$ are \F-algebras with $\dR_E$-structures $r_{f} : E f \to X$ and $r_{f'} : E f' \to X'$.
  Since \F is \stratified, $f$ has a new \F-structure making $(g,i)$ an \F-morphism; so it remains to find a new $\dR_E$-structure $\rtil_{f} : E f \to X$ (so that $f \circ \rtil_f = \rho_f$ and $\rtil_f \circ \lambda_f = \id_X$) such that $(g,i)$ is also an $\dR_E$-morphism, i.e.\ $\rtil_f \circ \fact g i = g \circ r_{f'}$.

  Define $P$ and $j$ by the pushout as on the right above.
  Since $f$ and $f'$ are \F-algebras, $\lambda_{f}$ and $\lambda_{f'}$ are acyclic cofibrations, and in particular monomorphisms.
  By cartesianness, $\fact g i$ is also a monomorphism (being a pullback of $i$) and $X'\cong E f' \times_{E f}X$.
  Thus, $P$ is a union of subobjects of $E f$ in the 1-topos \E, hence $j:P\to E f$ is also a monomorphism.
  Moreover, since $X\to P$ is a pushout of $\lambda_{f'}$, it is also an acyclic cofibration; hence by 2-out-of-3 $j$ is also acyclic.

  Now since $f$ is an \F-algebra, $\rho_f$ is a fibration.
  But $f$ is a retract of $\rho_f$ (by its $\dR_E$-structure $r_f$), hence also a fibration.
  Thus we can find a lift in the square:
  \[
    \begin{tikzcd}
      P \ar[d,"j"'] \ar[r] & X \ar[d,"f"] \\
      E f \ar[r,"{\rho_f}"'] \ar[ur,dotted,"{\rtil_f}" description] & Y
    \end{tikzcd}
  \]
  where the top arrow is induced by $g \circ r_{f'}$ and $\id_X$.
  Such a lift is then an $\dR_E$-structure on $f$ such that $(g,i)$ is an $\dR_E$-morphism, as desired.
\end{proof}

It remains to ensure that our universes are fibrant and univalent.

\begin{defn}\label{defn:nfs-hoinvar}
  Let \E be a locally presentable category with a model structure.
  A \nfs \F is \textbf{homotopy invariant} if every \F-algebra is a fibration, and given any commutative square
  \[
    \begin{tikzcd}
      X' \ar[d,"f'"',two heads] \ar[r,"\sim"] & X \ar[d,"f",two heads]\\
      Y' \ar[r,"\sim"'] & Y
    \end{tikzcd}
  \]
  where $f$ and $f'$ are fibrations and the horizontal maps are weak equivalences, $f$ admits an \F-structure if and only if $f'$ does.
\end{defn}

Of course, if $\uly\F=\Fib$ is the class of all fibrations, then \F is homotopy invariant.
More generally, homotopy invariance is a condition only on $\uly\F$.

We now recall the fundamental ``equivalence extension'' property.
To my knowledge, a version of this property first appeared in~\cite[Theorem 3.4.1]{klv:ssetmodel} in the case of simplicial sets.
It was observed in~\cite[Theorem 3.1]{shulman:elreedy} and~\cite[Remark 2.19]{cisinski:elegant} that the proof generalizes to any simplicial Cisinski model category.
A similar construction for cubical sets appeared under the name ``gluing'' in~\cite{cchm:cubicaltt}, which was then placed in a more abstract setting by~\cite{sattler:eqvext}.

\begin{thm}\label{thm:u4p}
  Let $\E$ be a simplicial Cisinski model category and \F a homotopy invariant \nfs on \E.
  Then there is a \la such that for any $\ka\shgt\la$ and any cofibration $i: A\cof B$, relatively \ka-presentable $\uly\F$-algebras $D_2 \fib B$ and $E_1 \fib A$, and weak equivalence $w: E_1 \toiso E_2$ over $A$, where $E_2 \coloneqq i^* D_2$:
%  Suppose given also a map $g:C\to D_2$ over $B$ and a factorization of $i^*(g) : i^* C \to E_2$ as $w\circ k$ for some $k:i^*(C) \to E_1$, filling out the solid arrows below:
  \begin{equation}
  \begin{tikzcd}[column sep=small]
      % & i^*C \ar[dl,"k"'] \ar[rrr] &&& C \ar[dl,dashed,"h"] \ar[ddr,"g"] \\
      E_1 \arrow[rdd, two heads] \arrow[rrd, "w"] \arrow[rrr, dashed] &  &  & D_1 \arrow[rdd, two heads, dashed] \arrow[rrd, "v", dashed] &  &  \\
      &  & \mathllap{E_2=\;}i^* D_2 \arrow[ld, two heads] \arrow[rrr,crossing over]
      % \ar[from=uul, crossing over]
      &  &  & D_2 \arrow[ld, two heads] \\
      & A \arrow[rrr, tail,"i"'] &  &  & B & 
    \end{tikzcd}\label{eq:u4p}
  \end{equation}
  there exists a relatively \ka-presentable $\uly\F$-algebra $D_1\fib B$ and an equivalence $v: D_1 \toiso D_2$ over $B$ such that $i^*(v)=w$.
  % If \F is \stratified, we can choose the \F-structure of $D_1$ so that the back square is an \F-morphism.
  %, and a map $h:C\to D_1$ with $i^*(h) = k$ and $v \circ h = g$.
\end{thm}
\begin{proof}
  Largely identical to that of~\cite[Theorem 3.1]{shulman:elreedy}.
  The latter statement assumes that \E is a presheaf category, but this is only used to obtain a notion of ``\ka-small morphism'' that is preserved by $i_*$ and by pullback; using \cref{thm:pres-pb,thm:relpres-mono-dp} instead allows \E to be any Grothendieck 1-topos.\footnote{The author claimed in~\cite{shulman:elreedy} that when $\E=\prcs$ it suffices to take $\ka > \card\C$, but Raffael Stenzel has pointed out that this is not enough to ensure that $i_*$ preserves \ka-small morphisms; even in the presheaf case we need some analogue of the relation $\shlt$.}
  (The proof uses that \E is a simplicial model category and has effective unions, to extend deformation retractions along $i$.)
  We conclude $D_1$ is an $\uly\F$-algebra by homotopy invariance, since it is equivalent to $D_2$ over $B$.
  % The application of \stratification is immediate.
\end{proof}

Univalence of our universes will follow from \cref{thm:u4p} as in~\cite{klv:ssetmodel,shulman:elreedy,cisinski:elegant}.
To show that $U$ is a fibrant object,~\cite[Theorem 2.2.1]{klv:ssetmodel} and~\cite[Proposition 2.21]{cisinski:elegant} use minimal fibrations, while~\cite[Lemma 6.3]{shulman:elreedy} uses a Reedy induction; but in fact fibrancy of $U$ is almost immediate from \cref{thm:u4p}.
A similar fact in the restricted situation of cubical-type model structures (where an explicit description of the fibrations is available) appears in~\cite{cchm:cubicaltt,sattler:eqvext}, while the general case was observed in~\cite{stenzel:thesis}.

\begin{thm}\label{thm:uf-fibrant}
  Let $\E$ be a right proper simplicial Cisinski model category, and \F a \local, \stratified, and homotopy invariant \nfs on \E.
  Then there is a regular cardinal \la such that for any regular cardinal $\ka\shgt\la$, there exists a morphism $\pi:\Util \to U$ such that:
  \begin{enumerate}
  \item The \ka-presentable objects in \E are closed under finite limits.\label{item:uf0}
  \item $\pi:\Util \to U$ is a relatively \ka-presentable $\uly\F$-algebra (in particular, a fibration).\label{item:uf1}
  \item Every relatively \ka-presentable $\uly\F$-algebra is a pullback of $\pi$.\label{item:uf2}
  \item The object $U$ is fibrant.\label{item:uf4}
  \item $\pi$ satisfies the univalence axiom.\label{item:uf3}
  \end{enumerate}
\end{thm}
\begin{proof}
  Let $\la_0$ satisfy \cref{thm:u4p}, let $\la_1$ be such that \E has a generating set of acyclic cofibrations with $\la_1$-presentable domains and codomains, let $\la_2$ be such that \E has functorial factorizations that preserve \ka-presentable objects for any $\ka\shgt\la_2$ (such exists since these factorizations are accessible functors), and let $\la_3$ be such that for any $\ka\shgt\la_3$ the \ka-presentable objects are closed under finite limits (which exists by \cref{thm:pres-pb}).
  % \footnote{It is claimed in~\cite[Propositions 7.2]{dug:pres} that a combinatorial model category satisfies this for all sufficiently large \ka, but the proof only shows it for a \shrp class.}
  Let $\la$ be such that $\la\shgt \la_j$ for $j=0,1,2,3$, and assume $\ka\shgt \la$; then $\ka\shgt\la_j$ for all $j$ as well, and in particular~\ref{item:uf0} holds.

  Since \F and $\cEka$ are \local and \stratified, so is $\Fka = {\F\times_\cE \cEka}$.
  Let $\pi:\Util\to U$ be the universe for $\uly\Fka$ obtained from \cref{thm:pre-u2p}; then~\ref{item:uf1} holds trivially.
  And since \cE and \F are stacks for cell complexes, in particular they preserve the initial object; so by \cref{rmk:universe} we have~\ref{item:uf2}.

  Since $\ka\gt\la_1$, to show that $U$ is fibrant~\ref{item:uf4} it suffices to show that it has right lifting for all acyclic cofibrations between \ka-presentable objects.
  Let $i:A\acof B$ be an acyclic cofibration with $A$ and $B$ \ka-presentable, let $h:A\to U$ be a map, and let $E_1 \fib A$ be the pullback of $\pi$ along $h$.
  Since $\pi$ is relatively \ka-presentable, $E_1$ is \ka-presentable.
  Thus since $\ka\shgt\la_2$, we can factor the composite $E_1\fib A \xto{i} B$ as an acyclic cofibration $E_1 \acof D_2$ followed by a fibration $D_2 \fib B$, where $D_2$ is \ka-presentable.
  So since $\ka\shgt\la_3$, $D_2\fib B$ is a relatively \ka-presentable fibration, and by homotopy invariance it is an $\uly\F$-algebra, hence an $\uly\Fka$-algebra.

  Let $E_2\coloneqq i^*(D_2)$; then by right properness the map $E_2 \to D_2$ is a weak equivalence, hence by 2-out-of-3 so is the induced map $E_1 \to E_2$.
  Thus since $\ka\shgt\la_0$, by \cref{thm:u4p} there is an $\uly\Fka$-algebra $D_1 \fib B$ with $i^*(D_1)\cong E_1$.
  Finally, since $U$ is a universe for $\uly\Fka$, by \cref{rmk:universe} there is a map $k:B\to U$ pulling $\pi$ back to $D_1$ such that $k i = h$; so $U$ has right lifting for $i$.

  For univalence~\ref{item:uf3}, we follow~\cite[Theorem 3.4.1]{klv:ssetmodel},~\cite[\sect 2]{shulman:elreedy}, and~\cite[Theorem 3.12]{cisinski:elegant}.
  Let $\Eq(\Util)$ be the universal space of auto-equivalences of $\pi$, as in~\cite[\sect 4]{shulman:elreedy}; it suffices to show that the composite projection
  \(\Eq(\Util) \to U\times U \to U\)
  is an acyclic fibration.
  Now a square
  \[
    \begin{tikzcd}
      A \ar[d,tail,"i"'] \ar[r] & \Eq(\Util) \ar[d]\\
      B \ar[r] & U
    \end{tikzcd}
  \]
  with $i$ a monomorphism yields a diagram of solid arrows~\eqref{eq:u4p} where all fibrations are $\uly\Fka$-algebras.
  Thus, since $\ka\shgt\la_0$ we can fill out the dashed arrows in~\eqref{eq:u4p} with $D_1 \fib B$ also a $\uly\Fka$-algebra; so by \cref{rmk:universe} we can classify it by a map to $U$ extending the given classifying map of $E_1$.
  But this is precisely what we need to specify a lift $B\to \Eq(\Util)$.
\end{proof}

Thus, to build fibrant and univalent universes for relatively \ka-presentable fibrations in a right proper simplicial Cisinski model category, it suffices to find a \local and \stratified \nfs \F such that $\uly\F=\Fib$ is the class of all fibrations.
We have essentially already seen one way to do this: if \E has a set of generating acyclic cofibrations with representable codomains, then $\Fib$ itself has these properties.
This was the approach of~\cite{klv:ssetmodel,shulman:elreedy}; but to deal with the general case we will have to use non-full \nfss.

\begin{rmk}
  Although our primary interest is in constructing univalent universes for \emph{all} (relatively \ka-presentable) fibrations, it is potentially useful that \cref{thm:uf-fibrant} also yields univalent universes for subclasses of fibrations.
  For instance, the \emph{left fibrations} in bisimplicial sets~\cite{vk:yoneda-css,pbb:groth-segal,rasekh:yoneda-ss} are a subclass of the Reedy fibrations, which by~\cite[Remark 2.1.4(a)]{vk:yoneda-css} are characterized by right lifting against a generating set with representable codomains; thus they admit fibrant and univalent universes.
  Such a universe of left fibrations is essentially the ``\oo-category of spaces'' constructed in~\cite{vk:yoneda-css}, although they do not explain how to make it a strict presheaf.
  In fact it is a complete Segal space (this is shown in~\cite[Theorem 2.2.11]{vk:yoneda-css}, and can also be deduced from~\cite[Theorem 4.8]{rasekh:yoneda-ss}), and could be useful for the programme of~\cite{rs:stt} to use that model for ``synthetic \io-category theory''.
  
  We will see another class of examples in \cref{thm:flf-nfs,rmk:modal-univ}.
\end{rmk}


%%% Local Variables:
%%% mode: latex
%%% TeX-master: "univinj"
%%% End:
