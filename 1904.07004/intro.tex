\section{Introduction}
\label{sec:introduction}

%\skiptoc
\subsection{Background}
\label{sec:background}

In the 1960s Grothendieck introduced \emph{toposes} (categories of sheaves on sites) as a powerful tool in the study of all kinds of geometry.
Part of their usefulness comes from the fact that they share most properties of the category of sets, so that all sorts of mathematics can be done ``internal'' to an arbitrary topos.
For instance, internal group theory yields the theory of sheaves of groups, and so on.

The direct way to express mathematics internally in a topos is to rewrite it manually in terms of objects, morphisms, and commutative diagrams; but this is tedious and verbose.
A more elegant method was found by Mitchell, Benabou, and Joyal in the early 1970s, building on the ``elementary'' toposes of Lawvere and Tierney.
Namely, the formal language of \emph{intuitionistic higher-order logic (IHOL)} can be interpreted algorithmically in any topos; thus almost any mathematics can be internalized in toposes simply by writing it in IHOL (or observing that it can be so written).
This generally requires very little modification,\footnote{Except that ``non-constructive'' principles such as the axiom of choice and the law of excluded middle must be avoided.} since the ``types'' of IHOL have ``elements'' and behave otherwise like sets.

In the 21\textsuperscript{st} century it has become clear that generalizing Grothendieck's toposes to \emph{higher categories} yields an even more powerful tool for geometry.
A \emph{Grothendieck \io-topos}~\cite{tv:hag-i,rezk:homotopy-toposes,lurie:higher-topoi} contains a subcategory that is an ordinary Grothendieck topos (a.k.a.\ 1-topos), but also many higher-dimensional objects that share the \io-categorical properties of the ``spaces'' in homotopy theory.
Thus not only set-based mathematics, but homotopy-theoretic and higher-categorical mathematics, can be done ``internal'' to any \io-topos.\footnote{One can also do homotopy-theoretic mathematics in a 1-topos, e.g.\ with internal simplicial objects or chain complexes, as indeed Grothendieck himself did.
  %plenty of this, and it is quite good enough for many purposes; otherwise it probably wouldn't have taken four decades to invent the notion of \io-topos.
  But compared to the \io-topos version, this approach sometimes fails to have the desired properties or to capture the desired information even about classical topological spaces; see~\cite[\sect 6.5.4]{lurie:higher-topoi} (the connection with internal simplicial objects is established by~\cite[Proposition 6.5.2.14]{lurie:higher-topoi}, \cite[Theorem 5]{jardine:fields-spre}, and~\cite{beke:sheafifiable,jardine:bool-loc}, while internal chain complexes are discussed in~\cite[Remark 6.5.4.3]{lurie:higher-topoi}).
  Moreover, there are important \io-toposes carrying homotopical information that is invisible to their underlying 1-topos, such as those whose objects are parametrized spaces~\cite{maysig:pht,abghr:thom-oocat}, parametrized spectra~\cite{joyal:logoi,hoyois:topoi-param}, generalized orbispaces~\cite{rezk:global-cohesion}, or excisive functors for Goodwillie calculus~\cite{abfj:goodwillie-htt}.}

Unsurprisingly, directly internalizing mathematics in an \io-topos is even more tedious and verbose than in a 1-topos, since there are many more coherences to keep track of.
Fortunately, an \io-categorical analogue of the Mitchell--Benabou--Joyal language emerged at about the same time as the theory of Grothendieck \io-toposes.
Building on~\cite{hs:gpd-typethy}, Awodey and Warren~\cite{aw:htpy-idtype} and Voevodsky (published a decade later as~\cite{klv:ssetmodel}) discovered that \emph{Martin-L\"{o}f intensional dependent type theory (MLTT)}~\cite{martinlof:itt,martinlof:itt-pred} has a interpretation in which its ``types'' behave like the spaces of homotopy theory, and hence like the objects of an \io-topos.
Voevodsky then formulated the \emph{univalence axiom} for MLTT, which internalizes the ``object classifiers'' of an \io-topos using a \emph{universe type}, just as the type of propositions in IHOL internalizes the subobject classifier of a 1-topos.

The study of formal systems with this sort of homotopical interpretation is now known as \emph{homotopy type theory (HoTT)}\footnote{Voevodsky's closely related ``univalent foundations'' program~\cite{vv:unimath} is, roughly, the use of such formal systems as a computer-verifiable foundation for mathematics.}.
The specific system of MLTT with univalence and also ``higher inductive types'', which internalize homotopy colimits and recursive constructions, is sometimes known as \emph{Book HoTT}\footnote{Although since~\cite{hottbook} did not formulate a general notion of ``higher inductive type'', the phrase ``Book HoTT'' does not have a fully precise meaning.} with reference to the book~\cite{hottbook} which popularized it.
In recent years it has been shown that large amounts of homotopy theory can be done inside Book HoTT, including homotopy groups of spheres~\cite{ls:pi1s1,lb:pinsn,brunerie:thesis}, ordinary and generalized cohomology~\cite{lf:emspaces,bf:cellcoh-hott,cavallo:cohom-hott,br:rp-hott}, the Freudenthal suspension and Blakers--Massey theorems~\cite{ffll:blakers-massey}, the Serre and Atiyah--Hirzebruch spectral sequences~\cite{floris:thesis}, and localization at primes~\cite{cors:loc-hott,scoccola:nilpfrac-hott}.
These proofs, which constitute the subfield of \emph{synthetic homotopy theory}, are sometimes simply rewritings of classical ones, but often they contribute new ideas even to classical homotopy theory.
More importantly, their expected validity internal to any \io-topos should lead to many new applications, some of which have already been realized (e.g.~\cite{abfj:gen-blakers-massey,abfj:goodwillie-htt}).

Unfortunately, rigorous proofs of the interpretation of type theories into \io-toposes have lagged behind these internal developments.
The problem\footnote{In addition to this coherence problem, it is necessary to relate the type-theoretic syntax (e.g.\ with variable binding) to the fully strictified ``algebraic'' version.
  Known as the ``initiality principle'', this is essentially a straightforward but tedious bookkeeping problem.
  It is well-known to be true for simpler type theories, but has not been written down completely yet for more complicated ones like MLTT and Book HoTT, though it is universally expected to be unproblematic.} is that, unlike for 1-toposes, many equalities that hold strictly in type theory hold only up to homotopy in an \io-topos, giving rise to a \emph{coherence problem}.
The straightforward solution (though one may imagine others) is to replace an \io-topos by a strict model that satisfies all the same strict equalities as the type theory.

The insight of~\cite{aw:htpy-idtype,klv:ssetmodel} was that the existing notion of \emph{Quillen model category}~\cite{quillen:htpical-alg} is already almost sufficient\footnote{In fact,~\cite{gg:idtypewfs,lumsdaine:hit-model-strux} showed that it is almost \emph{necessary} as well: any model of Book HoTT has the structure of at least the cofibrant-fibrant objects in a model category.}.
In fact, Voevodsky~\cite{klv:ssetmodel} constructed a model of all of MLTT, with univalence, in the model category of simplicial sets, which is a strictification of the \io-topos of \oo-groupoids (the \io-categorical analogue of the 1-topos of sets).
Later work such as~\cite{gb:topsimpid,ak:htmtt,shulman:invdia,gk:univlcc,ls:hits} showed that any \io-topos (and indeed any locally cartesian closed, locally presentable \io-category) can be strictified to a Quillen model category that models \emph{almost} all of Book HoTT,\footnote{To be precise, even after finding an appropriate Quillen model category there is an additional step of making all the structure strictly stable under pullback; but this was solved quite generally by~\cite{lw:localuniv} (see also~\cite{awodey:natmodels}), generalizing the technique of~\cite{klv:ssetmodel}.
In \cref{sec:coherence} we will extend this method to deal with universes.} including most higher inductive types --- but \emph{not} including the all-important univalence axiom.
In~\cite{shulman:invdia,shulman:elreedy,cisinski:elegant,shulman:eiuniv} some very special classes of \io-toposes were shown to model univalence, but a construction applicable to all \io-toposes proved elusive.

Our main goal in this paper is to solve this problem, showing that every Grothendieck \io-topos can be strictified to a Quillen model category that models MLTT with univalence, and hence nearly all of Book HoTT.
(The principal remaining open question is whether the higher inductive types can be chosen such that the univalent universes are closed under them.)

\begin{rmk}\label{rmk:cubical}
  We work only with model categories, relying on previous work such as~\cite{aw:htpy-idtype,klv:ssetmodel,lw:localuniv,shulman:invdia,shulman:elreedy} to establish the connection with type-theoretic syntax.
  (With one exception: in \cref{sec:coherence} we sketch an extension of these results to universe types, since this does not appear to be in the literature yet.)
  And we will use classical homotopy theory (notably, simplicial sets), classical logic (including the axiom of choice)\footnote{In the ``metatheory'' where we construct models, that is.  The type theory being modeled is intuitionistic, as befits the internal language of an arbitrary topos.}, and Book HoTT with univalence as an axiom (as stated by Voevodsky).

  Another popular class of formal systems used in homotopy type theory are \emph{cubical type theories}~\cite{cchm:cubicaltt,ahw:chtt-i,ah:chtt-ii,abchhl:cart-cube,cm:unif-cartcube}, which enjoy computational advantages (e.g.\ univalence is a theorem rather than an axiom), and \emph{cubical set} models, which can be studied in constructive metatheories~\cite{bch:tt-cubical} and contain univalent universes~\cite{bch:univalence-cubical,ahh:chtt-iii} closed under higher inductive types~\cite{chm:cubical-hits,ch:chtt-iv}.
  However, the class of \io-categories presented by cubical models remains unclear.
  Indeed, it seems not even to be known which kinds of cubical sets can present the standard \io-topos of \oo-groupoids.
  Thus, there are still good reasons to study Book HoTT and its models in classical homotopy theory.
\end{rmk}

\begin{rmk}\label{rmk:pfthy}
  As in~\cite{klv:ssetmodel}, we are only able to obtain universes closed under the basic type formers of MLTT by assuming that inaccessible cardinals exist in our background set theory.
  It is true that classical ZFC set theory with inaccessibles is much stronger, proof-theoretically, than \emph{predicative} MLTT % Martin-L\"{o}f type theory
  with universes (even with univalence); the latter is only as strong as Kripke-Platek set theory with recursive inaccessibles~\cite{rathjen:pfthy-ua}.
  However, we will show in \cref{thm:propresizing} that our models also validate the \emph{propositional resizing principle}, which is impredicative and probably increases the consistency strength of type theory to nearly that of ZFC with inaccessibles.
  Thus our metatheory is not really much stronger than necessary.

  Of course, ZFC itself does not prove that any inaccessible cardinals exist.
  Their consistency is fairly uncontroversial among modern set theorists, who routinely study much stronger large cardinal axioms.
  However, for the reader who nevertheless prefers to avoid them, we note that our construction also produces univalent universes within ZFC, albeit with weaker closure properties in the internal type theory.
\end{rmk}



%\skiptoc
\subsection{Overview}
\label{sec:overview}

The model categories we will use are familiar from \io-topos theory~\cite{lurie:higher-topoi}: they are left exact left Bousfield localizations of injective model structures on categories of simplicial presheaves.
These have always seemed a promising choice for modeling type theory, and indeed they are a subclass of those previously known to model the rest of MLTT; all they are missing is universes.

The model-categorical version of a universe is a classifier for small fibrations: a fibration $\pi:\Util\to U$ with the property that every fibration satisfying some cardinality bound on its fibers occurs as a (strict, 1-categorical) pullback of $\pi$.
The problem is that until now no explicit description of the fibrations in a general injective model structure, let alone a localization thereof, has been known.
Thus, our main task will be to establish new characterizations of these fibrations.

For a long time the dominant tool in model category theory for describing classes of fibrations was \emph{cofibrant generation}, whereby a set $\cI$ of ``generating acyclic cofibrations'' determines the class of fibrations $f:X\to Y$ having the right lifting property against all morphisms in \cI, i.e.\ for any commutative square
\begin{equation}
  \begin{tikzcd}
    A \ar[d,"i"'] \ar[r,"g"] & X \ar[d,"f"]\\
    B \ar[r,"h"'] & Y
  \end{tikzcd}\label{eq:intro-sq}
\end{equation}
with $i\in \cI$, there exists a lifting $k:B\to X$ with $k i = g$ and $f k = h$.
It was shown in~\cite[A.3.3.3]{lurie:higher-topoi} that injective model structures are cofibrantly generated, but the generating acyclic cofibrations are very inexplicit, consisting essentially of all pointwise acyclic cofibrations whose domain and codomain satisfy some cardinality bound.
This makes it difficult to say anything concrete about the injective fibrations.

However, recently a more general \emph{algebraic} approach to model category theory has arisen~\cite{gt:nwfs,garner:soa,riehl:nwfs-model,bg:awfs-i,rosicky:acc-model}, in which the fibrations are characterized by admitting the structure of an algebra for some pointed endofunctor on the category of arrows.
The algebraic approach has proven particularly useful for modeling type theory~\cite{gb:topsimpid,cchm:cubicaltt,gs:uniform,awodey:qmc-cube}.
Our characterization of the injective fibrations mixes these two approaches: they are the pointwise fibrations (these are determined by a right lifting property that is fairly explicit) that are additionally algebras for a certain pointed endofunctor.

The endofunctor in question is nothing new: in homotopy theory it is called a \emph{cobar construction}~\cite{may:goils,meyer:bar_i}, and in 2-category theory the analogous construction is called a \emph{pseudomorphism coclassifier}~\cite{bkp:2dmonads,lack:codescent-coh}.
Given a square~\eqref{eq:intro-sq} in which $i$ is a pointwise acyclic cofibration and $f$ a pointwise fibration, we can choose pointwise diagonal lifts; and while these will not generally form a natural transformation, they do always form a \emph{homotopy coherent natural transformation} $B\cohto X$ which satisfies naturality ``up to all higher homotopies''.
Thus, to obtain an actual lift we need $f$ to have the property that such homotopy coherent transformations can be ``rectified'' to strictly natural ones.
But the cobar construction of $f$, which we denote $E f$, is a representing object for such homotopy coherent transformations, so that any such $B\cohto X$ corresponds bijectively to some strictly natural $B\to E f$.
Thus, to obtain a strict map $B\to X$ we simply need a suitable strict map $E f\to X$, i.e.\ $f$ must be an algebra for the pointed endofunctor $E$.

This idea is inspired by~\cite{lack:htpy-2monads}, who characterized the cofibrant objects in the dual \emph{projective} model structure, in the simpler 2-categorical case, as those admitting a \emph{coalgebra} structure for the dual pseudomorphism classifier.
In 2-category theory the latter are known as \emph{flexible} objects; thus the injectively fibrant objects may be called \emph{coflexible}.
Informally, a coflexible presheaf $X$ is equipped with a rectification of any ``pseudo-element'' --- meaning an element $x\in X_d$ equipped with a coherent family of homotopical replacements for its restrictions along all morphisms in the domain category --- into an ordinary element of $X_d$, in a coherent and natural way.

With this characterization in hand, there are two remaining problems to be solved.
The first is to actually use this description to construct a universe of injective fibrations.
The existing techniques for constructing universes from~\cite{klv:ssetmodel,streicher:ttuniv,shulman:elreedy,cisinski:elegant} are designed for the cofibrantly generated case, but can be generalized to the algebraic case by considering the algebra operation as a structure, rather than its mere existence as a property.

More specifically, the ``unstructured'' techniques require the fibrations to be ``local on the base'', in the sense that a morphism $f:X\to Y$ into a colimit $Y = \colim_i Y_i$ is a fibration as soon as its pullback to each $Y_i$ is.
This corresponds to the generating acyclic cofibrations having representable codomains, which is what fails for general injective model structures.
But the generalized ``structured'' technique instead allows us to assume that the pullback of $f$ to each $Y_i$ is equipped with an algebra structure, and moreover that each square relating these pullbacks along a transition morphism $Y_i\to Y_j$ is a \emph{morphism} of algebras.
This makes it much easier to show that the algebra structures can be ``glued together'' into an algebra structure on $f$, yielding a universe of injective fibrations.
(Similar phenomena have been observed for the ``uniform fibrations'' used in cubical set models, which are cofibrantly generated with representable codomains only in an algebraic sense~\cite{bch:tt-cubical,gs:uniform,awodey:qmc-cube}.)

The last remaining problem is to deal with left exact localizations.
Here the basic idea was already sketched in~\cite[Remarks 3.24 and A.29]{rss:modalities}: since the local objects can be described internally in the type theory of the un-localized model category, we can construct internally a ``universe of local objects''.
In the left exact case this universe is itself local, and thus supplies a universe for the localized model structure.
At the time of~\cite{rss:modalities} we could not quite carry this out, since we were unable to show that every left exact localization of an \io-topos yields an \emph{internal} left exact localization in its internal type theory.
Fortunately,~\cite{abfj:lexloc} have recently given an explicit characterization of left exact localizations of \io-toposes, which passes to the internal type theory and thus allows us to prove this result, and hence construct universes for such localizations.
(For consistency in exposition, in this paper we will express the construction entirely at the model-categorical level without using the internal type theory, but the underlying idea is as described above.)

%\skiptoc
\subsection{\Ttmts}
\label{sec:ttmts-intro}

This much would be sufficient to prove our main result that homotopy type theory with univalence can be interpreted into Grothendieck \io-toposes.
However, with an eye towards the future, we do not restrict attention only to simplicial presheaf categories, but introduce a further abstraction.
Inspecting the proof, we see that the fact that we are working in a presheaf category is never really used (at least if we use the method of~\cite{shulman:elreedy} to construct universes by ``cofibrant replacement'', rather than the explicit methods of~\cite{klv:ssetmodel,streicher:ttuniv}).
What we do frequently need is the fact that the cofibrations satisfy certain exactness properties characteristic of the monomorphisms in a 1-topos.

Thus, we will define a \emph{\ttmt} to be a model category $\E$ whose underlying category is a \slcc Grothendieck 1-topos\footnote{Thus there is something of a pun in the name ``\ttmt'': it is a (1-)topos with a model structure that presents an (\io-)topos.}, whose model structure is proper, simplicial, and combinatorial, with its cofibrations being the monomorphisms, and that is furthermore equipped with a good notion of ``algebraic structured fibration''.
The above constructions can then be factored into the following results:
\begin{enumerate}[label=(\arabic*)]
\item Simplicial sets are a \ttmt (this is~\cite{klv:ssetmodel}).\label{item:it1}
\item \Ttmts are closed under passage to presheaves on simplicial categories, with injective model structures.\label{item:it2}
\item \Ttmts are closed under left exact localizations.\label{item:it3}
\item Every \ttmt has the structure to model homotopy type theory with univalent universes.
\end{enumerate}
Steps~\ref{item:it1}--\ref{item:it3} imply that every Grothendieck \io-topos can be strictified into some \ttmt, namely a left exact localization of an injective model structure on simplicial presheaves.
And conversely, every \ttmt presents a Grothendieck \io-topos, indeed it is a \emph{model topos}~\cite{rezk:homotopy-toposes}.

This additional step of abstraction is analogous to the step from defining toposes themselves via presentations (a category equivalent to a left exact localization of a presheaf category) to defining them intrinsically (a locally presentable category satisfing Giraud's exactness axioms).
%It allows us to consider a much wider class of model categories under one umbrella, some of which have already been studied (e.g.~\cite{shulman:eiuniv}), but others of which are relatively unexplored (e.g.\ injective model structures on presheaves over \emph{internal} categories).
\Ttmts also admit analogues of the basic constructions of elementary topos theory: slice categories, presheaves on internal as well as enriched categories, left exact localizations, coalgebras for suitable comonads, and Artin gluing.
In particular, they are closed under passage to ``internal sheaves on internal sites'', so that we are free to use any \ttmt as the ``ambient base topos'' instead of simplicial sets.
%This argues further for the naturalness of the general definition, and has potential applications to modeling modal type theories.

We have no use for this extra generality as yet, but the type theory we consider here has various extensions that we would also like to interpret in \io-toposes, and it might happen that some ``nonstandard'' \ttmts are more convenient for some such purpose.
This includes:
\begin{itemize}
\item \emph{Two-level type theories} such as~\cite{ack:2ltt}, especially with axioms such as fibrancy of the natural numbers object.
  The latter holds in enriched simplicial presheaf categories, but fails in general in their left exact localizations.
  However, it might hold in a model structure on simplicial \emph{sheaves} (though it seems doubtful that such models would suffice to present all \io-toposes).
\item \emph{Modal type theories} as in~\cite{ls:1var-adjoint-logic,lsr:multi,lsr:depdep}, which represent a collection of toposes connected by functors.
  Greater freedom to change the model categories seems likely to help in finding sufficiently strict models of such.
\item Our universes are closed under the standard type formers of MLTT, but it remains to be shown that they are closed under higher inductive types (the construction of the latter in~\cite{ls:hits} does not preserve smallness).
  It could be that some \ttmts make this easier to achieve than others.
  For instance, the method used to build higher inductive types in cubical set models~\cite{chm:cubical-hits,ch:chtt-iv} may be adaptable to simplicial sets, and perhaps to some other \ttmts with good properties.
\end{itemize}

\begin{rmk}
  The discussion above, and the rest of the paper, deals only with \emph{Grothendieck} \io-toposes, i.e.\ left exact localizations of presheaf \io-categories.
  Eventually one would also like to also model HoTT in ``elementary \io-toposes'', a hypothetical notion with proposed definitions~\cite{shulman:eleminf-talk,rasekh:eleminf} but essentially no known examples (other than Grothendieck ones).
  Also, sometimes the ``internal language correspondence'' refers to the even stronger conjecture that some ``homotopy theory of type theories''~\cite{kl:hot-tt} is \emph{equivalent} to that of elementary \io-toposes.

  The tools developed here for the Grothendieck case may be useful for proving these stronger conjectures, e.g.\ by embedding an elementary \io-topos in its presheaf category (as done in~\cite{ks:intlang-lex} to prove the strong internal language correspondence for \io-categories with finite limits).
  However, while these stronger conjectures are theoretically interesting, for most practical applications all that is needed is the interpretation of type theory into Grothendieck \io-toposes, which we establish here.
  For instance, this interpretation implies that the type-theoretic proof of the Blakers-Massey theorem~\cite{ffll:blakers-massey} \emph{is already} a proof of the corresponding \io-topos-theoretic theorem, without needing to be manually translated into \io-categorical language as in~\cite{abfj:gen-blakers-massey}.
\end{rmk}

%\skiptoc
\subsection{Outline}
\label{sec:outline}

We begin in \cref{sec:2cat} with 2-categorical preliminaries.
Then in \cref{sec:nfs} we study classes of structured morphisms, and in \cref{sec:relpres} we investigate how to restrict the cardinality of their fibers in an abstract way.
In \cref{sec:univalence} we construct universes for suitable such classes, and prove that in a suitable model category they are fibrant and univalent.
In \cref{sec:ttmt} we collect the accumulated hypotheses of these theorems into the notion of \ttmt.

We then move on to injective model structures and left exact localizations.
In \cref{sec:coh-bar} we motivate the (co)bar construction in more detail and review its basic properties, and then in \cref{sec:injmodel} we use it to give our new description of injective fibrations and prove that injective model structures are \ttmts.
Finally, in \cref{sec:int-loc} we prove some preliminary results about internal localizations (a generalization of left exact ones) and in \cref{sec:lex-loc} we show that \ttmts are preserved by left exact localization.
In \cref{sec:coherence} we deal with coherence for universes, which does not appear in any extant references. %  is fairly straightforward but

The main prerequisite for the reader is familiarity with model category theory; good modern references include~\cite{hovey:modelcats,hirschhorn:modelcats,mp:more-concise,riehl:cht}.
We will also use basic notions of 2-category theory, for which~\cite{steve:companion} is an excellent introduction, and some theory of locally presentable categories, whose standard reference is~\cite{ar:loc-pres}.

%\skiptoc
\subsection{Acknowledgments}
\label{sec:acknowledgments}

I am very grateful to Mathieu Anel, Steve Awodey, Dan Christensen, Eric Finster, Jonas Frey, Andr\'{e} Joyal, Peter LeFanu Lumsdaine, Emily Riehl, Bas Spitters, and Raffael Stenzel for helpful conversations, feedback, and detailed and careful reading of drafts.
And I would like to once again thank Peter May, for the two-sided simplicial bar construction.

%%% Local Variables:
%%% mode: latex
%%% TeX-master: "univinj"
%%% End:
