\section{Injective model structures}
\label{sec:injmodel}

In this section we will show that \ttmts are closed under passage to presheaf categories with injective model structures.
Although for our main theorem it would suffice to use \emph{enriched} presheaf categories $\pr\D\E$, % (and indeed even just $\pr\D\S$),
where \D is a small simplicially enriched category, we will also include \emph{internal} presheaf categories $\pr\cD\E$, where \cD is an internal category in \E.
Combined with \cref{sec:lex-loc}, this will imply that \ttmts are closed under passage to ``internal sheaves on internal sites''.

\begin{defn}
  For any functor $U:\M\to\N$ where \N is a model category, we define a morphism in \M to be:
  \begin{itemize}
  \item A \textbf{$U$-weak equivalence}, \textbf{$U$-cofibration}, or \textbf{$U$-fibration} if its image under $U$ belongs to the respective class in \N.
  \item A \textbf{projective cofibration} if it has left lifting for all $U$-acyclic $U$-fibrations.
  \item A \textbf{projective acyclic cofibration} if it has left lifting for all $U$-fibrations.
  \item An \textbf{injective fibration} if it has right lifting for all $U$-acyclic $U$-cofibrations.
  \item An \textbf{injective acyclic fibration} if it has right lifting for all $U$-cofibrations.
  \end{itemize}
  The \textbf{projective} or \textbf{right-lifted model structure}, if it exists, consists of the $U$-weak equivalences, $U$-fibrations, and projective cofibrations.
  Similarly, the \textbf{injective} or \textbf{left-lifted model structure}, if it exists, consists of the $U$-weak equivalences, $U$-cofibrations, and injective fibrations.
\end{defn}

The following is a compilation of some well-known and more recent results.

\begin{prop}\label{thm:projmodel-gen}
  Let \N be a model category and $U:\M\to\N$ a functor with both adjoints $F\adj U \adj G$, such that the adjunction $U F \adj U G$ is Quillen.
  Then:
  \begin{enumerate}
  \item If \N is cofibrantly generated, the projective model structure exists and is cofibrantly generated.
  \item If \N is an accessible model category and \M is locally presentable, then both projective and injective model structures exist and are accessible.
  \item If \N is a combinatorial model category and \M is locally presentable, then both projective and injective model structures exist and are combinatorial.
  \end{enumerate}
  Moreover:
  \begin{itemize}
  \item Every projective cofibration is a $U$-cofibration, and every injective fibration is a $U$-fibration.
  \item The projective and injective model structures are right or left proper if \N is.
  \item If \N is a \V-model category for some monoidal model category \V, and \M is a \V-category and the adjunctions are \V-adjunctions, then the projective and injective model structures are also \V-model categories.
  \end{itemize}
\end{prop}
\begin{proof}
  We begin by constructing factorizations.
  The cofibrantly generated case is standard: we take the generating cofibrations and acyclic cofibrations of \M to be the $F$-images of those of \N, whose domains are small in \M because $U$ preserves all colimits.
  The accessible case follows from~\cite{gkr:liftacc-modelstr}, and the combinatorial case from~\cite[Remark 3.8]{mr:cellular}.

  We sketch how to complete the proof in the projective case (the injective is dual).
  By retract arguments, it suffices to show every projective acyclic cofibration $f$ is a $U$-weak equivalence.
  We show $U f$ is an acyclic cofibration, i.e.\ that it has left lifting for any fibration $h$.
  This is equivalent to $f$ having left lifting for $G h$, so it will suffice to show $G h$ is a $U$-fibration, i.e.\ that $U G h$ is a fibration; but $U G$ preserves fibrations since it is right Quillen.

  A similar argument shows that projective cofibrations are $U$-cofibrations.
  Both properness claims now follow since $U$ preserves pullbacks, pushouts, cofibrations, and fibrations, and creates weak equivalences.
  Finally, enrichment of the model structure is easy using the characterization in terms of powers, since $U$ preserves powers and pullbacks and creates fibrations and acyclic fibrations.
\end{proof}

\begin{rmk}
  \cite[Remark 3.8]{mr:cellular} uses intricate calculations as in~\cite[A.3.3.3]{lurie:higher-topoi}, but in our case there is a simpler argument.
  If the cofibrations of \N are the monomorphisms, $U$ is faithful, and \M is a topos, then the $U$-cofibrations are also the monomorphisms, hence cofibrantly generated by~\cite[Proposition 1.12]{beke:sheafifiable}.
  And the $U$-weak equivalences are accessible and accessibly embedded, being the $U$-preimage of the weak equivalences of \N, so Smith's theorem applies.
\end{rmk}

Often $U$ will be both monadic and comonadic, so that \M is the category of $UF$-algebras and also of $UG$-coalgebras.

\begin{eg}\label{eg:enrcat}
  If \V is a monoidal category, \E a complete and cocomplete \V-category, and \D a small \V-category, then the forgetful functor $U : \pr\D\E \to \E^{\ob\D}$ is both monadic and comonadic. %, where $\pr\D\E$ is the enriched \V-presheaf category.
  In the accessible case this specializes to~\cite{moser:injproj}, and we have an analogous result in the combinatorial case: the projective and injective model structures exist if the copowers $(\D(x,y)\cpw -)$ preserves cofibrations and acyclic cofibrations.
  For instance, this occurs if \V is a monoidal model category, \E is a \V-model category, and each $\D(x,y)$ is cofibrant.
  The earliest reference I know for both model structures (when $\E=\S$) is~\cite{heller:htpythys}; later references for projective model structures include~\cite{piacenza:hodia,hirschhorn:modelcats,dro:enrfunc} and for injective ones~\cite{lurie:higher-topoi,moser:injproj}.
\end{eg}

\begin{eg}\label{eg:intcat}
  If \E is a locally cartesian closed category and $\cD = (D_1 \toto D_0)$ an internal category in \E, then the forgetful functor $\pr \cD \E \to \E/D_0$ is monadic and comonadic, where $\pr \cD \E$ is the ``internal presheaf category''.
  Thus, if \E is also an accessible model category, then $\pr \cD \E$ has projective and injective model structures if $(D_1 \times_{D_0} -)$ preserves cofibrations and acyclic cofibrations.
  In particular, this occurs if the cofibrations in \E are the monomorphisms and the target map $D_1 \to D_0$ is a \textbf{sharp morphism}~\cite{rezk:sharp-maps}, i.e.\ the pullback functor $\E/D_0 \to \E/D_1$ preserves weak equivalences (e.g.\ \E is right proper and it is a fibration).
  For $\E=\S$, this projective model structure was constructed in~\cite[Proposition 6.6]{horel:model-intsscat}.
\end{eg}

Now a bar construction has always been thought of as a ``cofibrant resolution'' of a sort; we will show that it is in fact a \emph{projective} cofibrant replacement.
Related results include~\cite[Proposition 14.8.8]{hirschhorn:modelcats} and~\cite{gambino:wgtlim} as well as~\cite[Propositions 6.9 and 6.10]{horel:model-intsscat}.
However, we need to strengthen the hypotheses of \cref{thm:projmodel-gen} a little.

Following~\cite[\sect 4]{rv:reedy}, for a two-variable functor $\Asterisk:\M\times\N\to\P$ we write its \textbf{Leibniz} (or ``pushout-product'') two-variable functor as $\widehat{\Asterisk} :\M^\dtwo\times\N^\dtwo\to\P^\dtwo$.
In order to use this notation for the ``application'' functor $\func\M\N \times \M\to \N$, we denote the latter by $\app$; thus $F\app X = F(X)$.

\begin{defn}
  Let $S,T:\M\to \N$ be functors between a pair of model categories, and $\alpha:S\to T$ a natural transformation.
  We say $\alpha$ is a \textbf{\qcof} if for any cofibration $i:A\to B$ in \M, the ``Leibniz application'' $\alpha \lapp i$:
  \begin{equation}
    \begin{tikzcd}
      S A \ar[r,"\alpha_A"] \ar[d,"S i"'] \ar[dr,phantom,near end,"\ulcorner"] & T A \ar[d] \ar[ddr,"T i"]\\
      S B \ar[r] \ar[drr,"\alpha_B"'] & \bullet % S B \amalg_{S A} T A
      \ar[dr,dashed,"{\alpha\lapp i}" description] \\
      & & T B
    \end{tikzcd}\label{eq:lapp}
  \end{equation}
  is a cofibration that is acyclic if $i$ is.
\end{defn}

\begin{lem}\label{thm:leibcomp}
  Given \qcofs $\al:S\to T$ and $\be:P\to Q$ between pushout-preserving functors $S,T:\M\to\N$ and $P,Q:\N\to\P$, their ``Leibniz composite''
  \[ \be\lcirc\al :  (Q\circ S) \amalg_{P\circ S} (P\circ T) \too (Q\circ T) \]
  is again a \qcof.
\end{lem}
\begin{proof}
  By the usual arguments (cf.~\cite[Observation 4.7]{rv:reedy}), we have
  $(\be\lcirc\al)\lapp i \cong \be \lapp (\al \lapp i)$,
  so we can apply the assumptions on $\al$ and $\be$ in succession.
\end{proof}

\begin{rmk}
  There is a dual notion of a \textbf{\qfib}.
  If $S$ and $T$ have right adjoints $S^*$ and $T^*$, then $\alpha:S\to T$ is a \qcof if and only if its mate $\alpha^* : T^* \to S^*$ is a \qfib.
  \qcofs can also be defined by a left lifting property in $\func\M\N$ against ``Leibniz right Kan extensions'' of a cofibration and a fibration one of which is acyclic, using the fact that the two-variable application functor
  \( \func\M\N \times \M \to \N \)
  has an adjoint on one side.
\end{rmk}

\begin{eg}
  A functor is \textbf{\qcoft} (i.e.\ $\emptyset \to T$ is a \qcof, where $\emptyset$ is the functor constant at the initial object) if and only if it preserves cofibrations and acyclic cofibrations.
  Hence \qcoft left adjoints are precisely left Quillen functors (hence the name ``\qcof'').
\end{eg}

\begin{defn}
  If $T$ is a monad on a model category whose unit $\Id\to T$ is a \qcof, we say that $T$ is \textbf{\qucoft}.
  Dually, we have the notion of a \textbf{\qufibt} comonad.
\end{defn}

Since the identity functor is \qcoft, any \qucoft monad is also a \qcoft functor, but not conversely.

\begin{eg}\label{thm:monmodel-lcof}
  If \V is a monoidal model category, \N is a \V-model category, and $j:K\to L$ is a cofibration in \V, then $(j\cpw -) : (K\cpw -) \to (L\cpw -)$ is a \qcof.
  In particular, the monad on \N induced by a monoid $D$ in \V, whose algebras are $D$-modules, is \qucoft if the unit map $\unit\to D$ is a cofibration in \V.
\end{eg}

\begin{eg}\label{eg:mat}
  Let \V be a monoidal model category and \E a \V-model category, and let $\sO$ be a small set.
  Let $\W=\V^{\sO\times\sO}$ with the ``matrix multiplication'' monoidal structure, $(A\otimes B)(x,y) = \coprod_z A(x,z) \otimes B(z,y)$ with unit $\unit_\W(x,y) = \unit_\V$ if $x=y$ and $\emptyset$ otherwise.
  Let $\N = \E^\sO$ with the usual product model structure, with $\W$-enriched hom-objects $\ehom{\N}(X,Y)(x,y) = \ehom{\E}(X(x),Y(y))$, so that the copower is ``matrix multiplication'' $(A\cpw X)(x) = \coprod_y A(x,y) \otimes X(y)$.
  Then \W is a % (non-symmetric)
  monoidal model category and \N is a \W-model category, so \cref{thm:monmodel-lcof} applies.

  In particular, note that a small \V-category \D can be regarded as a monoid in \W, and the algebras for the resulting monad on \N are enriched presheaves on \D.
  This monad is \qucoft if each hom-object $\D(x,y)$ is cofibrant and each unit map $\unit \to\D(x,x)$ is a cofibration in \V.
\end{eg}

\begin{eg}\label{eg:span}
  Let \E be a simplicial model category that is a Grothendieck topos and whose cofibrations are the monomorphisms, and $O\in \E$ an object.
  Let $\W= \E/(O\times O)$, regarded as the category of spans $O \xot{s} A \xto{t} O$ with the span-composition monoidal structure (pullback over $O$).
  Let $\N = \E/O$ with the usual slice model structure, with $\W$-enrichment $\ehom{\N}(X,Y)$ the local exponential in $\W$ of $\pi_1^* X$ and $\pi_2^* Y$, so that the copower is again pullback over $O$:
  \[
    \begin{tikzcd}[row sep=small,column sep=small]
      && A\cpw X\ar[dl] \ar[dr] \ar[dd,near start,phantom,"{\rotatebox{-45}{$\lrcorner$}}"] \\
      & A \ar[dl,"s"] \ar[dr,"t"'] && X \ar[dl]\\
      O & {} & O
    \end{tikzcd}
  \]
  A monoid in \W is an internal category in \E, and the algebras for the induced monad on \N are the internal presheaves.

  Now \W is \emph{not} in general a monoidal model category.
  But if $i:A_1\to A_2$ is a monomorphism for which the target morphisms $A_1 \to O$ and $A_2\to O$ are sharp morphisms, then the induced morphism of endofunctors of \N is a \qcof.
  To see this, let $j:B_1\to B_2$ be a cofibration in \N and construct the map:
  \[
    \begin{tikzcd}
      A_1\times_O B_1 \ar[d,"{i\times_O B_1}"'] \ar[r,"{A_1 \times_O j}"] \ar[dr,phantom, near end,"\ulcorner"] &
      A_1\times_O B_2 \ar[d] \ar[ddr,"{i\times_O B_2}"] \\
      A_2 \times_O B_1 \ar[r] \ar[drr,"{A_2 \times_O j}"'] & \bullet \ar[dr,"{i \mathbin{\widehat{\times_O}} j}" description]\\
      && A_2\times_O B_2
    \end{tikzcd}
  \]
  But the outer square is a pullback, hence the pushout is a union of subobjects in the topos $\E/O$; so $i \mathbin{\widehat{\times_O}} j$ is a monomorphism.
  If $j$ is acyclic, then so are $A_1 \times_O j$ and $A_2 \times_O j$ since $A_1$ and $A_2$ are sharp over $O$; hence the pushout of the former is again an acyclic cofibration, and so by 2-out-of-3 $i \mathbin{\widehat{\times_O}} j$ is also acyclic.

  Since identity maps are sharp, and the inclusion of the identities $O\to A$ of an internal category is monic, any internal category with sharp target-morphism $A\to O$ induces a \qucoft monad on $\E/O$.
  % Note in particular that if \E is right proper, any fibration is sharp.
  % \emph{Under some conditions I'm not sure what}, any morphism whose codomain is discrete (i.e.\ a coproduct of copies of the terminal object) is also sharp.
  % Thus, if we regard an \E-enriched category as an internal category with discrete object-of-objects in this sense, it is automatically flat.
\end{eg}

\begin{thm}\label{thm:cobar-injfib}
  Let \N be a simplicial model category, \M a simplicial category, and $F : \N \toot \M : U$ a simplicial adjunction such that $U$ preserves pushouts
  and $U F$ is \qucoft.
  If $g:A\to B$ is a $U$-cofibration in \M, then
  \[\bar(F,UF,Ug):\bar(F,UF,UA)\to \bar(F,UF,UB)\]
  is a projective cofibration, and a projective acyclic cofibration if $g$ is $U$-acyclic.
\end{thm}
\begin{proof}
  We will consider the $U$-cofibration case; the acyclic one is entirely analogous.
  Thus, let $g$ be a $U$-cofibration; we must show that $\bar(F,UF,Ug)$ has the left lifting property against any $U$-acyclic $U$-fibration $p:X\to Y$.
  Since geometric realization is left adjoint to the ``total singular object'' functor $Y \mapsto \Delta[\bullet] \pow Y$, this means we must lift in any square
  \[
    \begin{tikzcd}
      \sbar(F,UF,UA) \ar[r]\ar[d] & \Delta[\bullet] \pow X \ar[d] \\
      \sbar(F,UF,UB) \ar[r] & \Delta[\bullet] \pow Y
    \end{tikzcd}
  \]
  in $\pr\dDelta\M$.
  Using the Reedy structure on $\dDelta$, this means we must inductively find lifts in squares of the form
  \[\small
    \begin{tikzcd}
      \bar_n(F,UF,UA) \amalg_{L_n\bar(F,UF,UA)} {L_n\bar(F,UF,UB)} \ar[r]\ar[d] & \Delta[n] \pow X \ar[d] \\
      \bar_{n}(F,UF,UB) \ar[r] & (\Delta[n] \pow Y) \times_{(\partial\Delta[n] \pow Y)} (\partial\Delta[n] \pow X).
    \end{tikzcd}
  \]
  in \M.
  Now, none of the degeneracy maps involved in building the colimits on the left-hand side of this square involve the outermost copy of $F$ in
  \[ \bar_n(F,UF,UZ) = \overbrace{(FU)\dotsm(FU)}^{n+1} Z = F \overbrace{(UF)\dotsm(UF)}^{n} U Z. \]
  Since $F$ preserves colimits, this left-hand morphism is therefore $F$ applied to an analogous colimit construction.
  Specifically, let $\ddup$ be the category of degeneracy maps in $\dDelta\op$; then the restriction of $\sbar(F,UF,UZ)$ to $\ddup$ can be written as $F$ applied pointwise to a diagram $\sbarp(UZ) \in \func{\ddup}{\N}$, where
  \[\barp_n(W) = \overbrace{(UF)\dotsm(UF)}^{n} W.\]
  Thus, since $F$ preserves colimits, by the adjunction $F\adj U$ it suffices to find a lift in the adjunct square
  \[\small
    \begin{tikzcd}
      \barp_n(UA) \amalg_{L_n \barp(UA)} {L_n \barp(UB)} \ar[r]\ar[d] & \Delta[n] \pow U X \ar[d] \\
      \barp_n(UB) \ar[r] & (\Delta[n] \pow U Y) \times_{(\partial\Delta[n] \pow U Y)} (\partial\Delta[n] \pow U X).
    \end{tikzcd}
  \]
  where on the right we have also used the fact that $U$ preserves pullbacks and simplicial powers, since it is a simplicial right adjoint.
  And this right-hand morphism is an acyclic fibration, since $U p: U X \to U Y$ is an acyclic fibration by assumption and \N is a simplicial model category; thus it will suffice to show the left-hand map is a cofibration in \N.

  Since $U g$ is a cofibration in \N, it will suffice to show that the functor $\sbarp : \N \to \func{\ddup}{\N}$ takes cofibrations to Reedy cofibrations.
  Inspecting the definition, this is equivalent to saying that each latching map transformation $L_n\barp \to \barp_n$ is a \qcof between endofunctors of $\N$, i.e.\ that $\barp$ is ``Reedy \qcoft''.
  At this point the argument essentially reduces to a classical one showing that bar constructions are Reedy cofibrant (a.k.a.\ ``good'' or ``proper'') when the ``unit maps'' are cofibrations (e.g.~\cite[Proposition A.10]{may:goils} and~\cite[Proposition 23.6]{shulman:htpylim}); again we simply package it abstractly.

  We can express $\ddup$ as the category of finite ordinals with distinct endpoints and injective endpoint-preserving monotone maps.
  But injectivity means we can also discard the endpoints and identify $\ddup$ with the category of finite (possibly empty) ordinals and injective monotone maps.
  In particular, the reduced slice category $\ddup\sslash [n]$, over which the colimit defining $L_n \barp$ is taken, is isomorphic to the poset of proper subsets of an $n$-element set.

  Let $\cA$ be the category $(a \ot b \to c)$, and regard $[n+1]\in\ddup$ as the $(n+1)$-element ordinal $\{\underline{0},\dotsc,\underline{n}\}$.
  There is a functor $q:(\ddup\sslash [n+1]) \to \cA$ which sends the subset $\{\underline{1},\dotsc,\underline{n}\}$ to $a$, all of its proper subsets to $b$, and all other proper subsets of $[n+1]$ (those containing $\underline{0}$) to $c$.
  This functor is an opfibration, so that colimits over $\ddup\sslash [n+1]$ (such as that defining $L_{n+1} \barp$) can be computed by first taking colimits over the fibers of $q$ and then a colimit over \cA (i.e.\ a pushout).

  The induced functor $q^{-1}(b) \to q^{-1}(c)$ is an isomorphism (given by adding $\underline{0}$ to a subset), and these fibers $q^{-1}(b)$ and $q^{-1}(c)$ are both isomorphic to $\ddup\sslash[n]$.
  Moreover, when the diagram whose colimit is $L_{n+1} \barp$ is restricted to these fibers, it becomes the diagrams whose colimits are $L_n \barp$ and $UF \circ L_n\barp$ respectively.
  Thus we obtain an expression of $L_{n+1} \barp$ as a pushout:
  \[
    \begin{tikzcd}
      L_n \barp \ar[r] \ar[d] \ar[dr,phantom,near end,"\ulcorner"] & UF \circ L_n \barp  \ar[d] \ar[ddr] \\
      \barp_n \ar[r] \ar[drr] & L_{n+1} \barp \ar[dr,dashed] \\
      && (UF\circ \barp_{n}) \mathrlap{\;\cong \barp_{n+1}}
    \end{tikzcd}
  \]
  which identifies the latching map $L_{n+1}\barp \to \barp_{n+1}$ as a Leibniz composite of the latching map $L_n \barp \to \barp_n$ with the unit $\Id \to UF$.
  Since by assumption all the functors preserve pushouts and $UF$ is \qucoft, it follows by \cref{thm:leibcomp} and induction on $n$ that each map $L_n\barp \to \barp_n$ is a \qcof.
  (The base case $L_0\barp \to \barp_0$ is $\emptyset \to \Id$, a \qcof as remarked above.)
\end{proof}

\begin{cor}
  Let \N be a simplicial model category, \M a simplicial category, and $F : \N \toot \M : U$ a simplicial adjunction such that $U$ preserves pushouts and geometric realizations and $U F$ is \qucoft.
  If $X\in\M$ is $U$-cofibrant, then the augmentation $\bar(F,U F,U X) \to X$ is a projective cofibrant replacement (i.e.\ a weak equivalence from a projective cofibrant object).\qed
\end{cor}

Since our main interest is in injective model structures, we henceforth dualize everything.
Thus, a forgetful simplicial functor $U:\M\to\N$ preserving totalizations (the dual of geometric realization) and having a simplicial right adjoint $G$ yields a \textbf{cobar construction} $\cobar(G,UG,U-)$ with a natural coaugmentation $\nu_X:X \to \cobar(G,UG,UX)$ such that $U\nu_X$ is a simplicial strong deformation coretraction (hence $\nu_X$ is a $U$-weak equivalence).
If $U$ also preserves pullbacks and the counit $U G \to \Id$ is a \qfib, then $\cobar(G,UG,U-)$ takes $U$-fibrations to injective fibrations and $U$-acyclic $U$-fibrations to injective acyclic fibrations.
In particular, if $X$ is $U$-fibrant, then $\cobar(G,UG,UX)$ is injectively fibrant; so if \M has a $U$-fibrant replacement functor $R$, we obtain an injective fibrant replacement $X \to R X \to \cobar(G,UG,URX)$.

Importantly, we can also extend this to a factorization of morphisms.

\begin{defn}\label{defn:rel-cobar}
  Given a morphism $f:X\to Y$ in \M, its \textbf{relative cobar construction} is the pullback $E f$ shown below:
  \begin{equation}
    \begin{tikzcd}
      X \ar[dr,"{\lambda_f}" description] \ar[drr,"{\nu_X}"] \ar[ddr,"f"'] \\
      & E f \ar[r,"{\nu_f}"'] \ar[d,"{\rho_f}"] \ar[dr,phantom,"\lrcorner"] & \cobar(G,UG,UX) \ar[d,"{\cobar(G,UG,Uf)}"]\\
      & Y \ar[r,"{\nu_Y}"'] & \cobar(G,UG,UY)
    \end{tikzcd}\label{eq:Ef}
  \end{equation}
\end{defn}

We call the functorial factorization $f = \rho_f \circ \lambda_f$ the \textbf{cobar factorization}.

\begin{lem}\label{thm:cobarff-cart}
  If $U:\M\to\N$ preserves pullbacks and has a right adjoint $G$, then the cobar functorial factorization is cartesian in the sense of \cref{eg:cart-ff-local}.
\end{lem}
\begin{proof}
  The cobar construction $\cobar(G,UG,U-)$ preserves pullbacks, since $U$ and $G$ do, as does totalization of cosimplicial objects (being a limit construction).
  Thus any pullback square
  \[
    \begin{tikzcd}
      X' \ar[r] \ar[d,"f'"'] \ar[dr,phantom,near start,"\lrcorner"] & X \ar[d,"f"]\\
      Y' \ar[r] & Y
    \end{tikzcd}
  \]
  gives rise to a commutative cube:
  \[\begin{tikzcd}[row sep=small]
      E f' \arrow[dd, "{\rho_{f'}}"'] \arrow[rr] \arrow[rd] &  & {\cobar(G,UG,UX')} \arrow[dd] \arrow[rd] &  \\
      & E f \arrow[rr,crossing over] &  & {\cobar(G,UG,UX)} \arrow[dd] \\
      Y' \arrow[rr] \arrow[rd] &  & {\cobar(G,UG,UY')} \arrow[rd] &  \\
      & Y \arrow[rr] \arrow[from=uu, near start, "{\rho_f}"', crossing over] &  & {\cobar(G,UG,UY)}
    \end{tikzcd}\]
  in which the right-hand face as well as the front and back faces are pullbacks.
  Thus the left-hand face is also a pullback, i.e.\ $E$ is cartesian.
\end{proof}

\begin{lem}\label{thm:rho-injfib}
  Suppose $U:\M\to \N$ preserves pullbacks and has a simplicial right adjoint $G$ such that $U G$ is \qufibt.
  Then whenever $f$ is a $U$-fibration, $\rho_f$ is an injective fibration.
\end{lem}
\begin{proof}
  It is a pullback of $\cobar(G,UG,Uf)$, which is an injective fibration by the dual of \cref{thm:cobar-injfib}.
\end{proof}

\begin{lem}\label{thm:lambda-defret}
  If $U:\M\to\N$ has a simplicial right adjoint $G$ and preserves totalizations and pullbacks, then for any $f$ with cobar factorization $f = \rho_f \lambda_f$, the map $U\lambda_f$ in \N is the inclusion of a simplicial strong deformation retract over $U Y$.
\end{lem}
\begin{proof}
  The strong deformation retraction of \cref{thm:bar-repl} is natural (on functors with codomain \N, not \M).
\begin{concise}
  Thus, these retractions for $U\nu_Y$ and $U\nu_X$ induce one for $U\lambda_f$, using the enriched universal property of the pullback $U (E f)$ (which is also a pullback since $U$ preserves pullbacks).
\end{concise}
\begin{verbose}
  \fxwarning{Would need updating to general adjunctions.}
  That is, for fixed $d\in \D$, we have
  \begin{mathpar}
    r_X(d) : C(\D(d,-),\D,X) \to X(d) \and r_{Y}(d) : C(\D(d,-),\D,Y) \to Y(d)
  \end{mathpar}
  such that $f \circ r_X(d) = r_Y(d) \circ C(\D(d,-),\D,f)$, and simplicial homotopies $H_X(d)$ and $H_{Y}(d)$ from identities to $r_X(d) \circ \nu_X(d)$  and $r_Y(d) \circ \nu_Y(d)$ respectively, such that $C(\D(d,-),\D,f) \circ H_X(d) = H_Y(d) \circ C(\D(d,-),\D,f)$, and such that $H_X(d) \circ \nu_X(d)$ and $H_Y(d) \circ \nu_Y(d)$ are constant homotopies.

  Now the composite $r_X(d) \circ \nu_f : E f \to X$ is over $Y$ and satisfies $r_X(d) \circ \nu_f \circ \lambda_f = r_X(d) \circ \nu_X = 1_X$, so it suffices to give a homotopy from the identity of $E f$ to $\lambda_f \circ r_X(d) \circ \nu_f$.
  And by the universal property of the pullback $E f$, for this it suffices to give homotopies
  \begin{mathpar}
    \nu_f \sim \nu_f \circ \lambda_f \circ r_X(d) \circ \nu_f
    \and \rho_f \sim \rho_f \circ \lambda_f \circ r_X(d) \circ \nu_f
  \end{mathpar}
  that agree in $C(\D(d,-),\D,Y)$.
  But
  \[ \nu_f \circ \lambda_f \circ r_X(d) \circ \nu_f = \nu_X \circ r_X(d) \circ \nu_f \sim \nu_f\]
  by $H_X(d)\circ \nu_f$, while
  \begin{multline*}
  \rho_f \circ \lambda_f \circ r_X(d)\circ \nu_f
    = f \circ r_X(d)\circ \nu_f
  \\  = r_Y(d) \circ C(\D(d,-),\D,f) \circ \nu_f
    = r_Y(d) \circ \nu_Y \circ \rho_f
    = \rho_f
  \end{multline*}
  so we have the constant homotopy.
  And to see that these agree in $C(\D(d,-),\D,Y)$, we note that
  \[ C(\D(d,-),\D,f) \circ H_X(d)\circ \nu_f
    = H_Y(d) \circ C(\D(d,-),\D,f)\circ \nu_f
    = H_Y(d) \circ \nu_Y \circ \rho_f \]
  which is constant since $H_Y(d) \circ \nu_Y$ is constant.
  Finally, this homotopy from the identity of $E f$ to $\lambda_f \circ r_X(d) \circ \nu_f$ becomes constant when precomposed with $\lambda_f$ since $H_X(d)$ becomes so when composed with $\nu_X = \nu_f \circ \lambda_f$.
\end{verbose}
\end{proof}

We can now deduce our desired characterization of injective fibrations.

\begin{thm}\label{thm:injfib}
  Let $U:\M\to\N$ be a simplicial functor with both simplicial adjoints $F\adj U\adj G$, where \N is a simplicial model category, and suppose that:
  \begin{enumerate}
  \item $U G$ is \qufibt (equivalently, $U F$ is \qucoft).\label{item:im1}
  \item Inclusions of simplicial strong deformation retracts are cofibrations in \N (hence automatically acyclic cofibrations).\label{item:im3}
  \end{enumerate}
  Then a map $f:X\to Y$ in \M is an injective fibration if and only if it is a $U$-fibration and the map $\lambda_f : X\to E f$ has a retraction over $Y$.

  In particular, an object $X$ is injectively fibrant if and only if it is $U$-fibrant and the map $\nu_X : X \to \cobar(G,UG,UX)$ has a retraction.
\end{thm}
\begin{proof}
  Since $U$ is a simplicial right adjoint, it preserves totalizations and pullbacks, so \cref{thm:cobar-injfib,thm:lambda-defret} apply.
  In one direction, if $f$ is a $U$-fibration, then by \cref{thm:rho-injfib} $\rho_f$ is an injective fibration.
  Thus, if $f$ is a retract of it, then $f$ is also an injective fibration.

  Conversely, suppose $f:X\to Y$ is an injective fibration.
  Since~\ref{item:im1} implies that $U G$ is also \qfibt, the argument of \cref{thm:projmodel-gen} tells us that $f$ is a $U$-fibration.
  Moreover, by \cref{thm:lambda-defret} and~\ref{item:im3} $\lambda_f$ is a $U$-acyclic $U$-cofibration; thus $f$ has right lifting for it, yielding the desired retraction.
\end{proof}

Note that \cref{thm:injfib}\ref{item:im3} is automatically satisfied if the cofibrations in \N are the monomorphisms.
If in addition \N is right proper, an alternative proof of \cref{thm:lambda-defret} is to observe that the composite of $U\lambda_f$ with the weak equivalence $U\nu_f$ is the acyclic cofibration $U\nu_X$, hence it is an acyclic cofibration.

\begin{concise}
\begin{eg}\label{eg:cat}
  The dual of \cref{thm:injfib}\ref{item:im3} seems harder to satisfy in general, but one example where it holds is the ``trivial model structure''~\cite{lack:htpy-2monads} on a 2-category.
  In particular, if in \cref{eg:mat} we let $\V=\cCat$ with its canonical model structure and \E be a 2-category with its trivial model structure, then for any small 2-category \D, the dual of \cref{thm:injfib} applies to the projective model structure on the category $\pr\D\E$ of 2-functors and strict 2-natural transformations.
  In this case, as mentioned in \cref{rmk:2mnd,rmk:flexible}, the simplicial bar construction coincides with the \emph{codescent data} of~\cite{lack:codescent-coh}, and its realization with the corresponding \emph{pseudomorphism classifier} $\pscl A$.
  Thus we recover~\cite[Theorem 4.12]{lack:htpy-2monads} for the projective model structure: the cofibrant presheaves are the flexible ones.
  %, i.e.\ those that are a retract of their pseudomorphism classifier.
  % In particular, every pseudonatural transformation with projectively cofibrant (i.e.\ flexible) domain is isomorphic to a strict one.

  Dually, when \E has its ``cotrivial'' model structure, the injectively fibrant objects are the \emph{coflexible} presheaves: those that are retracts of their pseudomorphism coclassifier $\cocl A$, whose universal property is that strict 2-natural transformations $B \to \cocl A$ are bijective to pseudonatural ones $B\cohto A$.
  Explicitly, an element of $\cocl A_x$ is an element $a\in A_x$ together with, for each nonidentity $\xi:y\to x$ in \D, an object $a_\xi\in A_{y}$ and an isomorphism $a_\xi \cong \xi^*(a)$.
  (This construction is familiar to type theorists as the ``right adjoint splitting'' of a comprehension category~\cite{hofmann:ttinlccc,lw:localuniv}.)
  Thus, a presheaf $A$ is injectively fibrant if any such ``section with specified pseudo-restrictions'' can be ``rectified'' to a single section, in a way that is strictly natural and fixes the image of $A\to \cocl A$ (which is defined by taking $a_\xi=\xi^*(a)$ for all $\xi$).
  % In particular, any pseudonatural transformation with injectively fibrant (i.e.\ coflexible) \emph{codomain} is isomorphic to a strict one.
  A similar explicit interpretation can be given for arbitrary \V. %; see e.g.~\cite{cp:hcct} and~\cite[\sect 10]{shulman:htpylim}.
\end{eg}
\end{concise}

\begin{concise}
\begin{rmk}\label{rmk:reedy}
  When \D is a direct category, a generalized direct category~\cite{bm:extn-reedy}, or an elegant Reedy category~\cite{br:reedy}, the injective model structure coincides with the Reedy one.
  It is an interesting exercise to work out explicitly how in such cases \cref{thm:injfib} ends up describing the Reedy fibrations.
  Intuitively, in a Reedy fibration we can \emph{inductively} construct ``rectifications'' for families of pseudo-restrictions as described above, by successively applying the path lifting property for each matching object fibration $A_x \fib M_x A$.
  That is, in these special cases the ``natural rectifiability'' property of \cref{thm:injfib} can be reduced to a family of \emph{non-interacting} path lifting properties, one for each object of \D; but in the general case the naturality must be included in the operation.

  The only explicit characterization of a non-Reedy injective model structure that I am aware of is~\cite{bordg:thesis,bordg:injective} for \D the 2-element group.
  We leave it to the interested reader to relate this characterization directly to \cref{thm:injfib}.
\end{rmk}
\end{concise}

\begin{verbose}
\begin{eg}
  The injective fibrations in $\pr\D\E$ have previously known characterizations in special cases, such as when \D is a direct category, a generalized direct category~\cite{bm:extn-reedy}, an elegant Reedy category~\cite{br:reedy}, or the 2-element group~\cite{bordg:thesis,bordg:injective} (in all cases but the last, the characterization being that the injective model structure coincides with the Reedy one).
  Of course these characterizations must be equivalent to ours, but it is an interesting exercise to work out explicitly how that occurs.

  For example, consider the simplest case, when \D is the direct category $\dtwo = (0\to 1)$; the Reedy approach says that a presheaf $A\in\E^\dtwo$ is injectively fibrant if $A_0$ is fibrant and the map $A_1\to A_0$ is a fibration (hence $A_1$ is also fibrant).
  By contrast, \cref{thm:injfib} starts by requiring that $A_0$ and $A_1$ are separately fibrant and then imposes another condition.
  The simplicial cobar construction
  \[\scobar(G,U G,U A) = (\scobar(G,U G,U A)_1 \to \scobar(G,U G,U A)_0)\]
  has $\scobar(G,U G,U A)_0$ constant at $A_0$, while $\scobar(G,U G,U A)_1$ is the cosimplicial object
  \[
    \begin{tikzcd}[column sep=huge]
      A_0 \times A_1 \ar[r,shift left=2] \ar[r,shift right=2] \ar[r,<-] &
      A_0 \times A_0 \times A_1 \ar[r,shift left=4] \ar[r,shift right=4] \ar[r] \ar[r,shift left=2,<-] \ar[r,shift right=2,<-] &
      A_0 \times A_0 \times A_0 \times A_1 \mathrlap{\; \cdots}
    \end{tikzcd}
  \]
  in which the leftwards-pointing codegeneracy arrows discard factors of $A_0$ in the middle, and each rightwards-pointing arrow duplicates a different copy of $A_0$, except for the last one in each group which duplicates $A_1$ and then applies the given map $A_1\to A_0$.
  The totalization of this involves, at the beginning, $A_0\times A_1$ together with a simplicial homotopy between its two images in $A_0\times A_0\times A_1$; but this homotopy must become constant when the middle copy of $A_0$ is projected away, so it is really just a homotopy between the projection and the action $A_0\times A_1\toto A_0$.
  For similar reasons, the higher simplices are forced to be entirely degenerate, so that $\cobar(G,UG,UA)_1$ is the pullback
  \[
    \begin{tikzcd}
      \cobar(G,UG,UA)_1 \ar[d] \ar[r] \drpullback & \Delta[1] \pow A_0\ar[d] \\
      A_0\times A_1 \ar[r] & A_0\times A_0,
    \end{tikzcd}
  \]
  i.e.\ the mapping path space of the morphism $A_1\to A_0$.
  Since $A_0$ and $A_1$ are fibrant, the projection $\cobar(G,UG,UA)_1 \to \cobar(G,UG,UA)_0 = A_0$ is indeed a fibration; thus if $A$ is a retract of $\cobar(G,UG,UA)$ then $A_1\to A_0$ must also be a fibration.
  The converse is just the path lifting property of a fibration.
\end{eg}
\end{verbose}

\begin{verbose}
We end this section by considering some special cases of \cref{thm:injfib} which reproduce known results.

\begin{eg}\label{eg:cat}
  Let $\V=\cCat$ with its canonical model structure, in which the weak equivalences are the equivalences of categories, the cofibrations are the injective-on-objects functors, and the fibrations (``isofibrations'') are the functors with the isomorphism-lifting property.
  Let \E be a 2-category with its \emph{trivial model structure} in the sense of~\cite{lack:htpy-2monads}, whose weak equivalences are the internal equivalences and whose fibrations are the internal isofibrations.
  The factorizations in the latter are obtained from 2-categorical limits and colimits, hence are 2-functorial; thus \cref{thm:injmodel}\ref{item:im2} holds, while~\ref{item:im1} is clear.
  The dual of condition~\ref{item:im3} is straightforward to check, so \cref{thm:injmodel} yields the projective model structure on $\pr\D\E$.
  Note that $\pr\D\E$ is 2-monadic over $\E^{\ob\D}$ via the 2-monad
  \begin{equation}
    T A(x) = \coprod_{y\in\ob\D} \D(x,y)\times A(y),\label{eq:2mnd}
  \end{equation}
  and the projective model structure coincides with the one constructed in~\cite{lack:htpy-2monads} for algebras over a 2-monad.

  Now the geometric realization of a simplicial object in a 2-category coincides with the ``codescent object'' of its 2-truncation.
  In the case of $B(\D,\D,A)$, this codescent object is precisely the one used in~\cite{lack:codescent-coh} to construct the \emph{pseudo morphism classifier} $\pscl{A}$ for the 2-monad $T$, i.e.\ a left adjoint to the inclusion $T\algs \to T\alg$ of algebras and strict morphisms into algebras and pseudo morphisms.
  Thus \cref{thm:injfib} recovers the result of~\cite[Theorem 4.12]{lack:htpy-2monads} for the projective model structure: the cofibrant objects are those for which the map $q:\pscl{A}\to A$ has a section in $T\algs$, known in 2-category theory as \textbf{flexible algebras}.

  Dually, starting from the ``cotrivial model structure'' on \E, \cref{thm:injmodel} yields the injective model structure.
  The cobar construction $C(\D,\D,A)$ specializes to the descent object that constructs a pseudo morphism \emph{coclassifier} $\cocl{A}$ for the 2-\emph{comonad} right adjoint to $T$, whose coalgebras and their strict and pseudo morphisms coincide with the $T$-algebras and their morphisms.
  Now \cref{thm:injfib} yields a dual of~\cite[Theorem 4.12]{lack:htpy-2monads}: the fibrant objects in the injective model structure are the \textbf{coflexible coalgebras}, whose inclusion into their pseudo morphism coclassifier has a retraction.
  (Note that the construction of a model structure on $T\algs$ in~\cite{lack:htpy-2monads} does not dualize in general, as it appeals to local presentability.)

  Now suppose $\D$ is a 1-category and $\E=\cCat$; then we can describe $\pscl{A}$ and $\cocl{A}$ more explicitly.
  By~\cite[Theorem 4.10]{lack:codescent-coh}, since the 2-monad $T$ preserves bijective-on-objects morphisms, the pseudo morphism classifier $\pscl{A}$ can also be constructed by factoring the action map $T A \to A$ as a bijective-on-objects morphism followed by a fully-faithful one.
  Thus, the objects of $\pscl{A}(x)$ are those of $B_0(\D,\D,A) = T A$, i.e.\ pairs consisting of a morphism $\xi :x\to y$ in $\D$ and an object $a\in A(y)$; while its morphisms are the morphisms in $A$ between their images, i.e.\ a morphism $(\xi,a) \to (\ze,b)$ is a morphism $\xi^*(a) \to \ze^*(b)$ in $A(x)$.

  For the pseudo morphism coclassifier $\cocl{A}$ we can inspect its definition as a descent object to see that an object of $\cocl{A}(x)$ consists of the following:
  \begin{enumerate}[label=(\arabic*)]
  \item For every morphism $\xi:y\to x$ in \D, an object $a_\xi \in A(y)$.
  \item For every composable pair $z \xto{\ze} y \xto{\xi} x$ in \D, an isomorphism $\ze^*(a_\xi) \cong a_{\xi\ze}$.
  \item When $\ze=1_y$, this isomorphism is the identity of $a_\xi$.
  \item For every composable triple $w\xto{\chi} z \xto{\ze} y \xto{\xi} x$ in \D, the following square commutes:
    \[
      \begin{tikzcd}
        \chi^* \ze^*(a_\xi) \ar[r] \ar[d, equals] & \chi^*(a_{\xi\ze}) \ar[d]\\
        (\ze\chi)^*(a_\xi) \ar[r] & a_{\xi\ze\chi}.
      \end{tikzcd}
    \]
  \end{enumerate}
  We leave it to the reader to describe the morphisms in $\cocl{A}(x)$.
  The action of a morphism $\chi : w\to x$ in \D is the functor $\cocl{A}(x) \to \cocl{A}(w)$ sending the family of objects $(a_\xi)_{\xi : y\to x}$ to the family $(a_{\chi\xi})_{\xi:y\to w}$, with corresponding family of isomorphisms.

  However, this can be further simplified: taking $\xi=1_x$ in the above square shows that the isomorphisms $\ze^*(a_\xi) \cong a_{\xi\ze}$ are uniquely determined by the cases when $\xi=1_x$, in which case the squares impose no further condition.
  Thus, an object of $\cocl{A}(x)$ can be more simply described as:
  \begin{enumerate}[label=(\arabic*$'$)]
  \item For every morphism $\xi:y\to x$ in \D, an object $a_\xi \in A(y)$.
  \item For every morphism $\xi:y\to x$ in \D, an isomorphism $\xi^*(a_1) \cong a_\xi$.
  \item When $\xi=1_y$, this isomorphism is the identity of $a_1$.
  \end{enumerate}
  The presheaf $A$ is then injectively fibrant, i.e.\ coflexible, if the evident inclusion $A\to \cocl{A}$ has a retraction.
  Note that the obvious family of retractions $\cocl{A}(x) \to A(x)$ picking out $a_{1}$ is only \emph{pseudonatural} in $x$; for $A$ to be coflexible there must be a \emph{strictly} natural retraction.

  Note that these two constructions $\pscl{A}$ and $\cocl{A}$ are familiar to categorical type theorists under another name: they are the two ways to replace a comprehension category by a split one, called the \emph{left and right adjoint splittings} in~\cite{lw:localuniv}.
\end{eg}

\begin{eg}
  The primary special case in which an explicit description of injective fibrations is known is when \D is a \emph{direct category}, i.e.\ admits an identity-reflecting ``degree'' functor to the partially ordered class of ordinals.
  In this case the injective model structure coincides with the Reedy model structure, hence the injective fibrations coincide with the Reedy fibrations.
  The most direct proof of this is to simply observe that the two model structures have the same weak equivalences and cofibrations, but this conveys little insight into why it happens and how Reedy fibrations are related to injective ones in general.

  The underlying idea is clearest in the simple case $\E=\cCat$ described above.
  When \D is a direct category, to say that $A\in \pr\D\cCat$ is Reedy fibrant is to say that each matching map $p_x:A_x\to M_x A$ is an isofibration, i.e.\ that given $a\in A_x$ and $b\in M_x A$ and $\phi : p_x(a) \cong b$, there is an $\ahat \in A_x$ and $\phihat : a\cong \ahat$ with $p_x(\ahat) = b$ and $p_x(\phihat) = \phi$.
  Now $b\in M_x A$ consists of, for each nonidentity morphism $\xi : y\to x$ in \D, an object $b_\xi \in A(y)$, such that for any $\ze : z\to y$ we have $\ze^*(b_\xi) = b_{\xi\ze}$.
  Similarly, $\phi : p_x(a) \cong b$ consists of, for each nonidentity $\xi:y\to x$, an isomorphism $\phi_\xi : \xi^*(a) \cong b_\xi$, such that for any $\ze:z\to y$ we have $\ze^*(\phi_\xi) = \phi_{\xi\ze}$.
  But given these data, the objects $b_1 \coloneqq a$ and $b_\xi$ with the equalities $\ze^*(b_\xi) = b_{\xi\ze}$ and isomorphisms $\phi_\xi : \xi^*(a) \cong b_\xi$ can be regarded as an object of $\cocl{A}_x$.
  Thus, if $A$ is coflexible, the retraction $\cocl{A} \to A$ supplies an object $\ahat \in A_x$.
  Moreover, the isomorphisms $\phi$ also exhibit an isomorphism between this object and the image of $a$ in $\cocl{A}$, so the retraction also yields $\phihat : a\cong \ahat$.

  Conversely, if $A$ is Reedy fibrant and we are given an object of $\cocl{A}_x$, by inducting up the degree function we can successively transfer its objects along its isomorphisms to end up with an object of $A_x$ using the isofibration property of each matching map.
  For instance, when $\D = (z\xto{\ze} y\xto{\xi} x)$, an object of $\cocl{A}_x$ consists of objects $a_1\in A_x$, $a_\xi \in A_y$, and $a_{\xi\ze} \in A_z$ with isomorphisms $\xi^*(a_1) \cong a_\xi$ and $\ze^*\xi^*(a_1) \cong a_{\xi\ze}$.
  We can then lift $a_{\xi\ze}$ along the composite isomorphism $\ze^*(a_\xi) \cong \ze^*\xi^*(a_1) \cong a_{\xi\ze}$ to an object $\ahat_{\xi}$ with an isomorphism $a_\xi \cong \ahat_\xi$, and lift $\ahat_\xi$ along the composite isomorphism $\xi^*(a_1) \cong a_\xi \cong \ahat_\xi$ to an object $\ahat_1$ with an isomorphism $a_1\cong \ahat_1$, and define the retraction to be $\ahat_1\in A_x$.
  By choosing lifts of identity morphisms to be identities, this becomes a strict retraction, and by the careful choice of degreewise lifts it is strictly natural.
  (The most natural way to formulate this precisely for a general direct category \D is probably to construct the map $\cocl{A} \to A$ degree by degree in the usual Reedy way, using the lifting property of the matching isofibration against the levelwise acyclic cofibration $A_x \to \cocl{A}_x$ at each step.)

  A similar, but more complicated, analysis is possible for more general \E, as well as when \D is one of the more general \emph{elegant Reedy categories} of~\cite{br:reedy} which also have the property that their Reedy model structures coincide with injective ones (at least for some model categories \E).
\end{eg}

\begin{eg}
  The only other previously known explicit characterization of injective fibrations I am aware of was given in~\cite{bordg:thesis} in the case $\E=\cGpd$ and $\D$ the group $\dZ/2$ regarded as a 1-object groupoid.
  Thus, the objects of $\pr\D\E$ are groupoids equipped with a (strict) involution.
  Proposition 5.3.5 of~\cite{bordg:thesis} states that a map $f:(A,\alpha)\to (B,\beta)$ of groupoids with involution is an injective fibration if and only if it is an isofibration on underlying groupoids (i.e.\ a projective fibration), and given any $a\in A$ equipped with an isomorphism $\phi : \alpha(a) \cong a$ from its involute such that $\alpha(\phi) = \phi^{-1}$, and isomorphism $\psi : f(a) \cong b$ such that $\beta(b) = b$ and $\beta(\psi) = \psi\circ f(\phi)$, there exists an object $\ahat\in A$ with $\alpha(\ahat)=\ahat$ and $f(\ahat) = b$ and an isomorphism $\phihat:a\cong \ahat$ such that $\alpha(\phihat) = \phihat\circ \phi$ and $f(\phihat)=\psi$.
  This is certainly a lifting property against an injective acyclic cofibration, so any injective fibration must have this property; the hard part is showing that it is sufficient.
  In~\cite{bordg:thesis} this is done by decomposing any injective acyclic cofibration as a cell complex in an explicit way; we can instead derive it from \cref{thm:injfib} as follows.

  Given $(A,\alpha)$, an object of $C(\D,\D,A) = \cocl{A}$ is, by the explicit description in \cref{eg:cat}, two objects $a,a'\in A$ equipped with an isomorphism $\phi : \alpha(a) \cong a'$.
  The morphisms are pairs of morphisms in $A$ that commute with the isomorphisms, and the involution $\cocl{\alpha}$ maps $(a,a',\phi)$ to $(a',a,\alpha(\phi)^{-1})$.
  The injection $A\to \cocl{A}$ sends each object $a$ to the pair $(a,\alpha(a))$ with $\phi = 1_{\alpha(a)}$.

  Let $f:A\to B$ be a projective fibration satisfying the single explicit lifting property described above.
  We must exhibit $A$ as a retract of $E f$ over $B$, where the objects of $E f$ are triples $(a,a',\phi) \in \cocl{A}$ as above such that $f(a)=f(a')$ and $f(\phi)$ is an identity.
  We define $E f \to A$ on objects by three cases:
  \begin{enumerate}
  \item If $\phi$ is an identity, then $(a,\alpha(a),1)$ is the image of $a\in A$, so to have a retraction we must send it back to $a$.
    This respects the involution.
  \item If $\phi$ is not an identity and $\alpha(\phi)^{-1} \neq \phi$, then the same is true of $\cocl{\alpha}(a,a',\phi) = (a',a,\alpha(\phi)^{-1})$, which is unequal to $(a,a',\phi)$.
    Thus, objects of this sort come in pairs.
    In each pair we choose one object $(a,a',\phi)$ to send to $a\in A$, and send the other object $(a',a,\alpha(\phi)^{-1})$ to $\alpha(a)$ (as we must, to respect the involution).
  \item Finally, if $\phi$ is not an identity but $\alpha(\phi)^{-1} \neq \phi$, then the object $(a,a',\phi)$ is fixed by $\cocl{\alpha}$, so we must send it to a fixed object of $A$.
    However, since $f(a)=f(a')$ and $f(\phi)$ is an identity, we can let $\psi$ be an identity in the explicit lifting property; thus there is a fixed object $\ahat$ and an isomorphism $\phihat:a\cong \ahat$ such that $\alpha(\phihat) = \phihat\circ \phi$ and $f(\phihat)$ is an identity.
    We can then send $(a,a',\phi)$ to $\ahat\in A$.
  \end{enumerate}
  We leave it to the reader to extend this definition to morphisms, by composing with the given isomorphisms as needed.
\end{eg}
\end{verbose}

As an aside, if pointwise factorizations exist we can use \cref{thm:injfib} to \emph{construct} injective model structures, not requiring local presentability.

\begin{cor}\label{thm:injmodel}
  In the situation of \cref{thm:injfib}, suppose furthermore that
  \begin{enumerate}\setcounter{enumi}{2}
  \item Every morphism in \M factors as a $U$-cofibration followed by a $U$-acyclic $U$-fibration, and as a $U$-acyclic $U$-cofibration followed by a $U$-fibration.\label{item:im2}
  \end{enumerate}
  Then the injective model structure on \M exists.
\end{cor}
\begin{proof}
  Given a map $f$ to factor, first write $f = g i$ with $i$ a $U$-acyclic $U$-cofibration and $g$ a $U$-fibration, then factor $g = \rho_g \lambda_g$ as above.
  Then $\rho_g$ is an injective fibration, while $\lambda_g$ is (by \cref{thm:lambda-defret} and~\ref{item:im3}) a $U$-acyclic $U$-cofibration, hence so is $\lambda_g i$.
  The other factorization is analogous.
\end{proof}

\begin{cor}\label{thm:injfunctor}
  Suppose \V is a symmetric monoidal simplicial model category, \M is a \V-model category, and \D is a small \V-category such that each $\D(x,y)$ is cofibrant and each $\unit \to \D(x,y)$ is a cofibration in \V.
  Suppose moreover that inclusions of simplicial strong deformation retracts are cofibrations in \M, and that one of the following holds:
  \begin{enumerate}[label=(\alph*)]
  \item \M is cofibrantly generated.\label{item:if1}
  \item \M has \V-enriched functorial factorizations.\label{item:if2}
  \item \D is the free \V-category generated by an ordinary category.\label{item:if3}
  \end{enumerate}
  Then the injective model structure on $\pr\D\M$ exists.
\end{cor}
\begin{proof}
  \cref{thm:injfib}\ref{item:im1} holds by \cref{eg:mat} and~\ref{item:im3} by assumption, so it suffices to verify \cref{thm:injmodel}\ref{item:im2}.
  In cases~\ref{item:if2} and~\ref{item:if3} we can simply apply the factorizations of \M pointwise, while in case~\ref{item:if1} the projective model structure exists and its factorizations are in particular pointwise ones.
\end{proof}

We now return to our main goal of constructing \ttmts.

\begin{thm}\label{thm:mnd-ttmt}
  Let \E be a \ttmt and $T$ a simplicially enriched \qucoft monad on \E having a simplicial right adjoint $S$.
  Then the category $\E^T$ of $T$-algebras is again a \ttmt.
\end{thm}
\begin{proof}
  Since $T$ is a simplicially enriched monad, $\E^T$ is a simplicially enriched category.
  And since $T$ has a simplicial right adjoint $S$, there is an induced simplicial comonad structure on $S$ such that the $T$-algebras coincide with the $S$-coalgebras, and thus the forgetful functor $U:\E^T\to\E$ has both simplicial adjoints.
  Hence by \cref{thm:projmodel-gen}, the injective model structure on $\E^T$ exists and is proper, combinatorial, and simplicial.
  Moreover, $S$ is a simplicially enriched comonad that preserves finite limits, and the simplicial enrichment carries through the proof in~\cite[Theorem A4.2.1]{ptj:elephant} that the category of coalgebras for a finite-limit-preserving comonad inherits local cartesian closure; thus $\E^T$ is \slcc.
  And as $S$ is a right adjoint, it is accessible and left exact; hence its category of coalgebras is again a Grothendieck topos (it is an elementary topos by~\cite[Theorem A4.2.1]{ptj:elephant}, and locally presentable by the limit theorem of~\cite{mp:accessible}).

  Finally, if \F is the \local and \stratified \nfs for the fibrations of \E, let $\F^T= U^{-1}(\F) \times_\cE \dR_E$, where $\dR_E$ is as in \cref{eg:ff-fcos} for the cobar functorial factorization $E$ of \cref{defn:rel-cobar}.
  By \cref{thm:cobarff-cart,eg:cart-ff-local}, $\dR_E$ and hence $\F^T$ are \local, and by \cref{thm:injfib,eg:cof-ff-fcos} $\F^T$ is \stratified and the morphisms with $\F^T$-structure are the injective fibrations.
\end{proof}

\begin{rmk}
  The category of algebras for a cocontinuous monad on a presheaf category is again a presheaf category.
  Thus if \E in \cref{thm:mnd-ttmt} is a presheaf topos, so is $\E^T$ (cf.~\cref{rmk:design-space}).

  In fact, however, the proof of \cref{thm:mnd-ttmt} really relies more on the comonad $S$.
  The fact that $S$ has a left adjoint $T$ is only used to show that $S$ preserves finite limits and totalizations, that injective fibrations are $U$-fibrations, and that $S$-coalgebras satisfy the acyclicity condition to lift the model structure.
  The latter property could also be proven with the dual of ``Quillen's path object argument'', while the others could be assumed explicitly of a comonad, yielding a closer analogue of~\cite[Theorem A4.2.1]{ptj:elephant} for \ttmts. 
\end{rmk}

\begin{cor}\label{thm:simppre-ttmt}
  If \E is a \ttmt and \D is a small simplicially enriched category, then the injective model structure on $\pr\D\E$ is a \ttmt.
\end{cor}
\begin{proof}
  By \cref{eg:enrcat,eg:mat}, the monad on $\E^{\ob\D}$ whose category of algebras is $\pr\D\E$ is \qucoft, and has a simplicial right adjoint.
  % Thus \cref{thm:mnd-ttmt} applies.
\end{proof}

\cref{thm:simppre-ttmt} is the main result we want: combined with left exact localizations (which we discuss in the next section) it will show that \ttmts suffice to present all \io-toposes.
(Note that it includes the models of~\cite[\sect 11]{shulman:invdia} and~\cite{shulman:elreedy}.)
However, \cref{thm:mnd-ttmt} yields many other examples as well.

\begin{cor}\label{thm:intpre-ttmt}
  If \E is a \ttmt and $\cD$ is an internal category in \E whose target map $t:D_1\to D_0$ is sharp, then the category $\pr \cD\E$ of internal presheaves on $\cD$ is a \ttmt.
\end{cor}
\begin{proof}
  By \cref{eg:intcat,eg:span}, the monad on $\E/D_0$ whose category of algebras is $\pr \cD\E$ is \qucoft, and since \E is \slcc it is a simplicial monad with a simplicial right adjoint.
  % Thus \cref{thm:mnd-ttmt} applies.
\end{proof}

\begin{cor}\label{thm:enrpre-ttmt}
  If \E is a \ttmt and \D is a small \E-enriched category such that every object $\D(x,y)$ is sharp, then the injective model structure on $\pr\D\E$ is a \ttmt.
\end{cor}
\begin{proof}
  Although \E may not be a cartesian monoidal \emph{model} category, monomorphisms in a topos are always preserved by Leibniz products, and acyclicity of one factor is preserved if the other has sharp domain and codomain, similarly to \cref{eg:span}.
  Thus the monad on $\E^{\ob\D}$ whose category of algebras is $\pr\D\E$ is again \qucoft, and has a simplicial right adjoint.
  % Thus \cref{thm:mnd-ttmt} also applies here.
\end{proof}

\begin{cor}\label{thm:glue-ttmt}
  If $\E_1$ and $\E_2$ are \ttmts and $F:\E_1 \toot \E_2 : G$ is a simplicial Quillen adjunction, then the comma category $(\E_1\dn G)$ (a.k.a.\ the ``Artin gluing'') is also a \ttmt.
\end{cor}
\begin{proof}
  By \cref{thm:prod-ttmt}, $\E_1\times \E_2$ is a \ttmt.
  Define a simplicial monad $T$ on $\E_1\times \E_2$ by $T(X_1,X_2) = (X_1, F X_1 \amalg X_2)$, with right adjoint $S(Y_1,Y_2) = (Y_1 \times G Y_2, Y_2)$.
  The Leibniz application of the unit $\Id \to T$ to a pointwise cofibration $i = (i_1,i_2) : (A_1,A_2) \cof (B_1,B_2)$ is
  \[ (\id, F i_1 \amalg \id) : (B_1, F A_1 \amalg B_2) \to (B_1, F B_1 \amalg B_2 ) \]
  which is a pointwise cofibration that is acyclic if $i$ is, since $F$ is left Quillen.
  Thus $T$ is \qucoft, so \cref{thm:mnd-ttmt} applies.
\end{proof}

We can obtain the models of~\cite{shulman:eiuniv} by applying \cref{thm:glue-ttmt} iteratively with $\E_1=\S$; hence they are \ttmts.
(In fact they are also instances of \cref{thm:intpre-ttmt}.)
More generally, \emph{colax limits} of diagrams of right Quillen functors can be constructed analogously, yielding \ttmts like the models of~\cite[\sect12]{shulman:invdia}.
As in~\cite[B3.4]{ptj:elephant} and~\cite[\sect6.3.2]{lurie:higher-topoi}, these should present \emph{colimits} of \io-toposes.

%   Can probably allow colax limits indexed by simplicial categories if the diagrams factor through their local $\pi_0$, as in~\cite{harpaz:laxlim-model}, and some kind of weighted limits too.
%   Maybe even allow some kind of $(\infty,2)$-categorical indexing and weighting (categories enriched over simplicial categories --- not a model for all $(\infty,2)$-categories, though).
%   Is the (or a) ``cartesian'' localization of a colax limit left exact, to obtain homotopy limits?

% Local Variables:
% TeX-master: "univinj"
% End:
