\section{2-categorical preliminaries}
\label{sec:2cat}

We begin with some 2-categorical observations.
A morphism $f:\dX\to\dY$ in a 2-category \sK is an \textbf{internal fibration} if each induced functor $\sK(\dZ,\dX) \to \sK(\dZ,\dY)$ is a Grothendieck fibration.
Explicitly, this means for any $g:\dZ\to \dX$ and 2-cell $\alpha: h \to f\circ g$ there is a cartesian lift $\beta : k\to g$ with $f\circ k = h$ and  $f\circ \beta = \alpha$.
Similarly we have \textbf{internal discrete fibrations}, for which $\beta$ must be unique.

Let \E be a (large, locally small) category and write $\Ehat = \cPsFun(\E\op,\cGPD)$ for the (very large) 2-category of contravariant pseudofunctors from \E to large groupoids, and pseudonatural transformations between them.

\begin{defn}\label{defn:dfib}
  A \textbf{strict fibration} in \Ehat is a \emph{strictly} natural transformation $\dX\to\dY$ such that each component $\dX(A)\to \dY(A)$ is a fibration of groupoids.
  Similarly we define a \textbf{strict discrete fibration}.
\end{defn}

Of course, if \dX and \dY are discrete (i.e.\ functors $\E\op\to\nSET$), then any morphism $\dX\to\dY$ is a strict discrete fibration.
More generally, we have:

\begin{lem}\label{thm:fib-eqv}
  Any morphism $f:\dZ\to\dY$ in \Ehat is equivalent to a strict fibration over \dY, which is discrete if and only if $f$ is faithful.
\end{lem}
\begin{proof}
  Given $f:\dZ\to\dY$, each component $\dZ(A) \to \dY(A)$ is equivalent to a fibration $\dX(A) \to \dY(A)$, and we can transfer the functorial action of \dZ across these equivalences to make $\dX$ an object of $\Ehat$ and factor $f$ as an equivalence $\dZ\to\dX$ followed by a pseudonatural transformation $\dX\to\dY$ whose components are fibrations.
  Then we can use isomorphism-lifting for these fibrations to modify the pseudofunctorial action of $\dX$ to make this second transformation strictly natural.
  Finally, a fibration of groupoids is discrete if and only if it is faithful, which is invariant under equivalence.
\end{proof}

\begin{lem}
  A strict fibration $f:\dX\to\dY$ is an internal fibration in $\Ehat$, and similarly in the discrete case.
\end{lem}
\begin{proof}
  Given $g:\dZ\to \dX$ and $h:\dZ\to\dY$ with $\alpha : h\cong f \circ g$, for each $A$ there is a functor $k_A : \dZ(A) \to \dX(A)$ and isomorphism $\beta_A : k_A \cong g_A$ (unique in the discrete case) such that $f_A \circ k_A = h_A$ and $f_A \circ \beta_A = \alpha_A$.
  Transferring the pseudonaturality constraints of $g$ along the isomorphisms $\beta$ makes $k$ a pseudonatural transformation and $\beta$ a modification such that $f\circ k = h$ and $f\circ \beta = \alpha$.
\end{proof}

\begin{lem}\label{thm:dfib-pb}
  Given a strict fibration $f:\dX\to\dY$ and a morphism $g:\dZ\to\dY$, for each $A\in\E$ define $\dW(A)$ by the strict pullback of groupoids on the left:
  \[
    \begin{tikzcd}
      \dW(A) \ar[r,"h_A"] \ar[d,"k_A"'] \drpullback & \dX(A) \ar[d,"f_A"] \\
      \dZ(A) \ar[r,"g_A"'] & \dY(A)
    \end{tikzcd}
    \hspace{2cm}
    \begin{tikzcd}
      \dW \ar[r,"h"] \ar[d,"k"'] \drpullback & \dX \ar[d,"f"] \\
      \dZ \ar[r,"g"'] & \dY.
    \end{tikzcd}
  \]
  Then \dW can be made into a pseudofunctor and $h$ a pseudonatural transformation such that $k$ is a strict fibration and the square on the right commutes strictly in \Ehat and is a weak bicategorical pullback there.
\end{lem}
We will refer to this construction as a \textbf{strict pullback}.
\begin{proof}
  We define the pseudofunctorial actions of \dW and the pseudonaturality constraints of $h$ by lifting the pseudonaturality constraints of $g$ along the components of $f$.
  All the claims are straightforward to verify.
\end{proof}

Henceforth we suppose \E has pullbacks.
The following notion is fairly standard.

\begin{defn}\label{defn:rep}
  A morphism $f:\dW\to\dY$ in \Ehat is \textbf{representable} if for any $Z\in\E$ and weak bicategorical pullback
  \begin{equation*}
    \begin{tikzcd}
      \dP \ar[r] \ar[d] \ar[dr,phantom,near start,"\lrcorner"] & \dW\ar[d,"{f}"] \\
      \E(-,Z) \ar[r] & \dY
    \end{tikzcd}
  \end{equation*}
  the object \dP is equivalent to a representable $\E(-,X)$.
\end{defn}

Note that a morphism with representable codomain is a representable morphism if and only if it has representable domain.
In addition, every representable morphism is faithful, hence (by \cref{thm:fib-eqv}) equivalent to a strict discrete fibration.
Moreover, if $f$ is a representable strict discrete fibration, then in \cref{defn:rep} it suffices to consider strict pullbacks as in \cref{thm:dfib-pb}, and such a strict pullback \dP must be \emph{isomorphic} to a representable.
We therefore mainly consider representable strict discrete fibrations, which encompass all representable morphisms up to equivalence but are simpler to work with.

\begin{defn}
  Let $\cE\in\Ehat$ denote the \textbf{core of the self-indexing} of \E, where $\cE(Y)$ is the maximal subgroupoid of the slice category $\E/Y$, with pseudofunctorial action by pullback.
  Similarly, define $\cEp \in\Ehat$ such that $\cEp(Y)$ is the groupoid of morphisms $f:X\to Y$ equipped with $s:Y\to X$ such that $f s = \id_Y$.
\end{defn}

\begin{prop}\label{thm:univrep}
  The forgetful map $\varpi:\cEp\to\cE$ is the pseudo-universal representable morphism and the strictly universal representable strict discrete fibration.
  That is, the following are equivalent for any $\dY\in\Ehat$:
  \begin{enumerate}
  \item The hom-groupoid $\Ehat(\dY,\cE)$.\label{item:urep1}
  \item The groupoid of representable strict discrete fibrations with codomain \dY.\label{item:urep2}
  \item The 2-groupoid of representable morphisms with codomain \dY.\label{item:urep3}
  \end{enumerate}
  The functor~\ref{item:urep1}$\to$\ref{item:urep2} is by strict pullback and~\ref{item:urep2}$\to$\ref{item:urep3} is by inclusion, so that~\ref{item:urep1}$\to$\ref{item:urep3} is by weak bicategorical pullback.
\end{prop}
\begin{proof}
  Assuming that pullbacks in \E are globally chosen, $\varpi$ is strictly natural, and it is straightforward to check that it is a discrete fibration.
  A morphism $\E(-,Z) \to \cE$ corresponds, by the bicategorical Yoneda lemma, to a morphism $X\to Z$, and the pullback of $\cEp$ to $\E(-,Z)$ is then $\E(-,X)$.
  Thus $\varpi$ is a representable strict discrete fibration, hence so are its strict pullbacks.

  Note that~\ref{item:urep2} is a groupoid rather than a 2-groupoid since there are no nonidentity 2-cells between discrete fibrations, and it is equivalent to~\ref{item:urep3} by \cref{thm:fib-eqv}.
  Now for any representable strict discrete fibration $f:\dX\to\dY$, choose for each $y\in \dY(Z)$ a strict pullback square
  \[
    % \begin{tikzcd}
    %   \dX \ar[d,"f"'] \ar[r] \drpullback & \cEp \ar[d,"\varpi"] \\
    %   \dY \ar[r,"p"'] & \cE
    % \end{tikzcd}
    % \hspace{2cm}
    \begin{tikzcd}
      \E(-,P_y) \ar[d] \ar[r] \drpullback & \dX \ar[d,"f"] \\
      \E(-,Z) \ar[r,"y"'] & \dY.
    \end{tikzcd}
  \]
  Sending $y\mapsto (P_y\to Z)$ then defines a pseudonatural transformation $\dY\to\cE$, yielding a functor~\ref{item:urep2}$\to$\ref{item:urep1}, which can be verified to be an inverse equivalence to pullback.
\end{proof}

We conclude this section with bicategorical lifting and orthogonality.

\begin{defn}\label{defn:2liftorth}
  Let $i:\dA\to\dB$ and $p:\dX\to\dY$ be morphisms in a 2-category \sK.
  We say $i$ and $p$ have the \textbf{lifting property}, and write $i \lifts p$, if the functor
  \begin{equation}
    \sK(\dB,\dX) \to \sK(\dA,\dX) \bitimes_{\sK(\dA,\dY)} \sK(\dB,\dY)\label{eq:bicat-lift}
  \end{equation}
  is essentially surjective, where $\bitimes$ denotes a weak bicategorical pullback.
  In other words, $i \lifts p$ if given any $f:\dA\to\dX$ and $g:\dB\to\dY$ and isomorphism $\al : p f \cong g i$, there exists a morphism $h:\dB\to \dX$ and isomorphisms $\be:f \cong h i$ and $\gm : p h \cong g$ such that $(\gm i).(p\be) = \al$.

  If instead~\eqref{eq:bicat-lift} is an equivalence of categories, we say $i$ and $p$ are \textbf{orthogonal} and write $i\perp p$.
  In addition to the lifting property, this says that given $h,k:\dB\to\dX$ with 2-cells $\ph:h i \to k i$ and $\psi : p h \to p k$ such that $\psi i = p \ph$, there exists a unique $\chi : h\to k$ with $\chi i = \ph$ and $p \chi = \psi$.

  For an object \dX, we write $i\lifts \dX$ and $i\perp \dX$ to refer to lifting and orthogonality properties for the unique morphism $\dX\to 1$.
\end{defn}

\begin{lem}\label{thm:dfib-perp}
  If $p:\dX\to\dY$ is an internal discrete fibration in \sK, the following are equivalent for any $i:\dA\to\dB$ in \sK.
  \begin{enumerate}
  \item $i \perp p$.\label{item:dp1}
  \item The analogous functor
    \begin{equation}
      \sK(\dB,\dX) \to \sK(\dA,\dX) \times_{\sK(\dA,\dY)} \sK(\dB,\dY)\label{eq:bicat-strict-lift}
    \end{equation}
    to the strict pullback is an isomorphism.\label{item:dp2}
  \item The functor~\eqref{eq:bicat-strict-lift} is bijective on objects.\label{item:dp3}
  \item Given any $f:\dA\to\dX$ and $g:\dB\to\dY$ such that $p f = g i$, there exists a unique morphism $h:\dB\to \dX$ such that $p h = g$ and $h i = f$.\label{item:dp4}
  \end{enumerate}
\end{lem}
\begin{proof}
  If $p$ is a fibration, then so is the induced map $\sK(\dA,\dX) \to \sK(\dA,\dY)$, and thus the bicategorical pullback is equivalent to the strict one.
  Since isomorphisms are also equivalences,~\ref{item:dp2}$\Rightarrow$\ref{item:dp1}.

  On the other hand, assuming~\ref{item:dp1}, then if $p f = g i$ we can choose $\alpha = \id$ to obtain an $h:\dB\to \dX$ and isomorphisms $\be:f \cong h i$ and $\gm : p h \cong g$ such that $(\gm i).(p\be) = \id$.
  But now since $p$ is a fibration, there is an isomorphism $\delta:h\cong h'$ such that $p h' = g$ and $p \delta = \gm$.
  Then we have $(\delta i).\be : f \cong h' i$, and $p((\delta i).\be) = (p\delta i).(p\be) = (\gm i).(p\be) = \id$; thus since $p$ is a \emph{discrete} fibration $(\delta i).\be = \id$ and so $f = h' i$.
  Thus~\eqref{eq:bicat-strict-lift} is a \emph{surjective} equivalence.
  Furthermore, if $h,k :\dB\to \dY$ are such that $p h = p k$ and $h i = k i$, then~\ref{item:dp1} gives a unique isomorphism $\xi : h\cong k$ with $p \xi = \id$ and $\xi i = \id$.
  But since $p$ is a discrete fibration, $p \xi = \id$ implies $\xi=\id$ and $h=k$; thus~\ref{item:dp1}$\Rightarrow$\ref{item:dp2}.

  Now of course~\ref{item:dp2}$\Rightarrow$\ref{item:dp3}.
  But conversely, if~\ref{item:dp3} holds and $h,k:\dB\to\dY$ are given with $\ph:h i \to k i$ and $\psi : p h \to p k$ such that $\psi i = p \ph$, then since $p$ is a discrete fibration there is a unique $\chi : k'\to k$ with $p k' = p h$ and $p \chi = \psi$, and uniqueness furthermore implies $k' i = h i$ and $\chi i = \ph$.
  So~\ref{item:dp3} implies that in fact $k' = h$; hence the unique $\chi$ shows that~\eqref{eq:bicat-strict-lift} is fully faithful, i.e.~\ref{item:dp2} holds.

  Finally,~\ref{item:dp4} is just an explicit restatement of~\ref{item:dp3}.
\end{proof}

% \begin{eg}
%   Let $i:A\to B$ be a morphism in \E, and $p:\dX\to\dY$ a morphism in $\Ehat$.
%   Then by the bicategorical Yoneda lemma, $\E(-,i) \lifts p$ if and only if given any $x\in \dX(A)$ and $y\in \dY(B)$ and isomorphism $\al \cong p(x) \cong i^*(y)$, there exists a $z\in \dX(B)$ and isomorphisms $\be : x \cong i^*(z)$ and $\gm : p(z) \cong y$ such that the composite
%   \[ p(x) \xto{p(\be)} p(i^*(z)) \cong i^*(p(z)) \xto{\i^*(\gm)} i^*(y) \]
%   is equal to $\al$.
%   If $p$ is moreover a discrete fibration, then $\E(-,i) \lifts p$ if and only if given any $x\in \dX(A)$ and $y\in \dY(B)$ such that $p(x) = i^*(y)$ there exists $z\in \dX(B)$ such that $x = i^*(z)$ and $p(z) = y$.
% \end{eg}

% \begin{eg}
%   If $i:A\to B$ and $p:X\to Y$ are both morphisms in \E, then $\E(-,i) \lifts \E(-,p)$ in \Ehat if and only if $i\lifts p$ in \E in the usual sense, and similarly for orthogonality.
% \end{eg}

\begin{eg}\label{eg:orth-cancel}
  As in the 1-categorical case, morphisms having a right lifting or right orthogonality property are closed under composition and (weak bicategorical) pullback.
  Similarly, morphisms with a right orthogonality property are cancellable: if $i\perp qp$ and $i\perp q$ then $i\perp p$.
\end{eg}

\begin{eg}\label{eg:colim-orth}
  Let $Y = \colim_i Y_i$ be a diagram in a category \E, and let $\Yhat = \bicolim_i \E(-,Y_i)$ be the corresponding weak bicategorical colimit of representables in \Ehat.
  Its universal property is that
  \[\Ehat(\Yhat,\dX) \simeq \bilim_i \Ehat(\E(-,Y_i),\dX) \simeq \bilim_i \dX(Y_i)\]
  where $\bilim_i$ denotes the weak bicategorical limit: an object of $\bilim_i \dX(Y_i)$ consists of objects $x_i \in \dX(Y_i)$ and coherent isomorphisms $\dX_\iota(x_j) \cong x_i$ for all $\iota:i\to j$.

  In particular, the given colimit cocone in \E induces a map $q:\Yhat \to \E(-,Y)$.
  Then for $\dX\in\Ehat$ we have $q \perp \dX$ if and only if \dX preserves the colimit $Y = \colim_i Y_i$, in that the induced map $\dX(Y) \to \bilim_i \dX(Y_i)$ is an equivalence.
  Note that all representables $\E(-,Z) \in \Ehat$ preserve all colimits.
\end{eg}


%%% Local Variables:
%%% mode: latex
%%% TeX-master: "univinj"
%%% End:
