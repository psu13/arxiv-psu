\section{Summary}
\label{sec:summary}

Putting together our conclusions from \cref{sec:ttmt,sec:injmodel,sec:lex-loc}, we have the following.

\begin{thm}\label{thm:iotop-ttmt}
  Every model topos~\cite{rezk:homotopy-toposes} is Quillen equivalent to a \ttmt.
  Therefore, every Grothendieck \io-topos~\cite{lurie:higher-topoi} can be presented by a \ttmt.
\end{thm}
\begin{proof}
  A model topos is, by definition, Quillen equivalent to a left exact localization of a projective model structure on simplicial presheaves.
  Thus it is also equivalent to the corresponding localization of the \emph{injective} model structure, which is a \ttmt by \cref{thm:enrpre-ttmt,thm:lexloc-ttmt}.
\end{proof}

\begin{thm}\label{thm:models}
  For any Grothendieck \io-topos \fE, there is a regular cardinal \la such that \fE can be presented by a Quillen model category that interprets Martin-L\"{o}f type theory with the following structure:
  \begin{enumerate}
  \item $\Sigma$-types, a unit type, $\Pi$-types with function extensionality, identity types, and binary sum types.\label{item:sigpiid}
  \item The empty type, the natural numbers type, the circle type $S^1$, the sphere types $S^n$, and other specific ``cell complex'' types such as the torus $T^2$.\label{item:unparam-hits}
  \item As many universe types as there are inaccessible cardinals larger than $\la$, all closed under the type formers~\ref{item:sigpiid} and containing the types~\ref{item:unparam-hits}, and satisfying the univalence axiom.\label{item:univ}
  \item $\mathsf{W}$-types, pushout types, truncations, localizations, James constructions, and many other recursive higher inductive types.\label{item:hits}
  \end{enumerate}
\end{thm}
\begin{proof}
  By \cref{thm:iotop-ttmt}, $\fE$ can be presented by a \ttmt \E, to which we can apply \cref{thm:ttmt-models}.
\end{proof}

We also have the following (cf.~\cref{rmk:pfthy}):

\begin{prop}\label{thm:propresizing}
  Any \ttmt satisfies the propositional resizing principle for sufficiently large universes.
\end{prop}
\begin{proof}
  As stated in~\cite[Axiom 3.5.5]{hottbook},
  the propositional resizing principle\footnote{Not to be confused with the propositional resizing \emph{rule} of Voevodsky, which is not known to have any univalent models at all.} says that for universes $U^{\ka_1}$ and $U^{\ka_2}$ with $\ka_1<\ka_2$, the corresponding inclusion between universes of $(-1)$-truncated maps $\Prop^{\ka_1}\into \Prop^{\ka_2}$ is an equivalence.
  Since by univalence this is always a $(-1)$-truncated map, it suffices to construct a homotopy right inverse.
  For this it will suffice to show that there is a regular cardinal $\ka$ such that any $(-1)$-truncated fibration $f:X\to Y$ is equivalent over $Y$ to a relatively \ka-presentable one, since the latter will be classified by a map to $\Prop^\ka$.

  We will prove this first when \E is a left exact localization of a simplicial presheaf category, $\pr\C\S_{(S\dia)\pb}$.
  In this case it will suffice to show that any $(-1)$-truncated fibration $f:X\to Y$ is equivalent over $Y$ to a monomorphism, since by \cref{eg:pshf-relpres} a monomorphism is $\ka$-small for any sufficiently large $\ka$.
  Given such an $f$, for each $c\in \C$ let $Z(c)$ be the union of all connected components of $Y(c)$ that contain any points in the image of $X(c)$.
  The functorial action of $\C$ preserves these components, since $f$ is a natural transformation, so $Z$ becomes a presheaf as well and we have a factorization $X \xto{j} Z \xto{p} Y$ in which $Z\to Y$ is a monomorphism.

  Now each map $p_c : Z(c) \to Y(c)$ is also a Kan fibration, and each of its fibers is either empty or contractible.
  The same is true of $j_c : X(c)\to Y(c)$ since $f$ is $(-1)$-truncated, and by construction if a given fiber of $Z(c)$ is inhabited then so is the corresponding fiber of $X(c)$.
  Thus $j_c$ is a weak equivalence on each fiber, hence a weak equivalence, and so $j$ is a pointwise weak equivalence.

  It remains to show that $p:Z\to Y$ is a fibration in $\pr\C\S_{(S\dia)\pb}$.
  But the inclusion a union of connected components actually has the \emph{unique} right lifting property against all weak equivalences of simplicial sets.
  Thus $p$ is an injective fibration, since we can lift against any injective acyclic cofibration at each $c$ separately and fit the lifts together by uniqueness.
  Finally, $p$ is an $(S\dia)\pb$-local fibration since it is fiberwise equivalent to the $(S\dia)\pb$-local fibration $f$.

  This completes the proof when $\E=\pr\C\S_{(S\dia)\pb}$.
  Hence any Grothendieck \io-topos has a subobject classifier (this is also remarked after~\cite[Proposition 6.1.6.3]{lurie:higher-topoi}): a monomorphism of which every monomorphism is uniquely a (homotopy) pullback.
  In particular, the homotopy \io-category of any \ttmt \E has such a subobject classifier, which is presented by some $(-1)$-truncated fibration $\pi_{-1}:\widetilde{\Omega} \fib\Omega$ in \E.
  Let \ka be such that $\pi_{-1}$ is relatively \ka-presentable; then any $(-1)$-truncated morphism in \E will be a homotopy pullback of $\pi_{-1}$, hence equivalent to the corresponding strict pullback, which is relatively \ka-presentable.
\end{proof}



The main remaining open question is therefore whether the parametrized higher inductive types in \cref{thm:models}\ref{item:hits} can be constructed in such a way that the universes are closed under them.
Additionally,~\cite{ls:hits} does not construct all possible higher inductive types (only those without ``fibrant structure in the constructors''), and has not yet been generalized to indexed higher inductive types, inductive-inductive types, etc.


%%% Local Variables:
%%% mode: latex
%%% TeX-master: "univinj"
%%% End:
