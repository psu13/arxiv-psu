\section{Coherence with universes}
\label{sec:coherence}

As noted in \cref{thm:ttmt-models}, the existing coherence theorems in the literature~\cite{klv:ssetmodel,lw:localuniv,awodey:natmodels} do not include an arbitrary family of universes.
Thus, here we sketch an extension of the coherence theorem of~\cite{lw:localuniv} to universe types.
For simplicity, we consider only non-cumulative Tarski universes.
The material in this appendix owes a great deal to conversations with Peter Lumsdaine.

First we review the original theorem of~\cite{lw:localuniv}, reformulated to match our \crefrange{sec:2cat}{sec:nfs} using the ideas of~\cite{awodey:natmodels}.
Recall that we write $\Ehat = \cPsFun(\E\op,\cGPD)$ to denote the 2-category of pseudofunctors $\E\op\to\cGPD$, and the notions of strict discrete fibration and representability from \cref{defn:dfib,defn:rep}.

\begin{defn}
  A \textbf{natural pseudo-model} is a category \E with a terminal object and a representable strict discrete fibration $\varpi:\dTm\to\dTy$ in \Ehat.
  It is a \textbf{natural model}~\cite{awodey:natmodels} if $\dTy$ (hence also \dTm) is discrete, i.e.\ a presheaf $\E\op\to\nSET$.
\end{defn}

It is shown in~\cite{awodey:natmodels} that natural models are equivalent to \emph{categories with families}~\cite{dybjer:internal-tt}.
Analogously, we have:

\begin{lem}
  A natural pseudo-model is the same as a comprehension category~\cite{jacobs:compr-cat} whose fibers are all groupoids.
\end{lem}
\begin{proof}[Sketch of proof]
  This is most obvious when \E has all pullbacks.
  In this case, by \cref{thm:univrep} a representable strict discrete fibration $\varpi:\dTm\to\dTy$ is classified by an essentially unique morphism $\dTy\to\cE$.
  Reformulating pseudofunctors to $\cGPD$ as fibrations with groupoidal fibers, this morphism becomes exactly the usual notion of a comprehension category.
  Unwinding this argument explicitly, we see that it also works even if not all pullbacks exist in \E.
\end{proof}

The analogous reformulation of a natural model is known as a \emph{category with attributes}~\cite{cartmell:gatcc,hofmann:ssdts}, while if the morphism $\dTy\to\cE$ of a comprehension category is a full inclusion it is called a \emph{display map category}~\cite{taylor:pracfdn}.

\begin{rmk}
  By \cref{thm:fib-eqv}, any representable morphism in \Ehat is equivalent to a natural pseudo-model, and thus might be called a \textbf{pseudonatural pseudo-model}.
  As in \cref{thm:univrep}, pseudonatural pseudo-models are only ``bicategorically'' equivalent to comprehension categories, and I don't know of any naturally-occurring examples of such that are not natural pseudo-models.
\end{rmk}

Natural models are the ``fully algebraic''\footnote{To make this completely precise, a natural model must actually be equipped with \emph{specified} representing objects for each strict pullback of $\varpi$ to a representable.} models of type theory.
The objects $\Gamma\in\E$ represent contexts, the objects $A\in \dTy(\Gamma)$ represent types in context $\Gamma$, the fiber over such an $A$ in $\dTm(\Gamma)$ represent terms $a:A$ in context $\Gamma$, and the representability of $\varpi$ yields context extension:
\[
  \begin{tikzcd}
    \E(-,\Gamma\ce A) \ar[r] \ar[d] \drpullback & \dTm \ar[d,"\varpi"] \\
    \E(-,\Gamma) \ar[r,"A"] & \dTy.
  \end{tikzcd}
\]
We write $\ec$ for the terminal object of \E, regarding it as the empty context.

A natural pseudo-model is the category-theoretic input from which a coherence theorem constructs a natural model.

\begin{eg}
  The \textbf{trivial} natural pseudo-model on \E is $\cEp \to \cE$.
\end{eg}

\begin{eg}\label{eg:cc-can}
  Any model category \E gives rise to a \textbf{canonical} natural pseudo-model where $\dTy=\Fib$ with $\dTy\into\cE$ the inclusion, so that $\dTm = \dFib \times_\cE \cEp$.
\end{eg}

\begin{eg}\label{eg:rep-cwf}
  For any morphism $\pi:\Util\to U$ in \E, the induced map $\E(-,\pi) : \E(-,\Util) \to \E(-,U)$ defines a \textbf{represented} natural model.
  This is essentially the construction of a model of type theory from a universe as in~\cite[\sect1.3]{klv:ssetmodel}.
\end{eg}

\begin{eg}\label{eg:las}
  If \E is a natural pseudo-model, its \textbf{left adjoint splitting}~\cite{lw:localuniv} is the natural model defined by
  \[ \las\dTy = \coprod_{V_A\in \E \atop E_A \in \dTy(V_A)} \E(-,V_A). \]
  There is a (surjective) morphism $\las\dTy\to \dTy$ composed of the classifying maps $E_A : \E(-,V_A) \to \dTy$, and we set $\las\dTm = \las\dTy \times_{\dTy} \dTm$.
  The object $V_A$ is known as the \emph{local universe} of $(V_A,E_A,A) \in \las\dTy(\Gamma)$.
\end{eg}

\begin{rmk}
  Any \nfs (\cref{defn:fcos}) is of course a natural pseudo-model with $\dTy=\cE$.
  On the other hand, if the comprehension morphism $\dTy\to\cE$ of a natural pseudo-model is faithful (such as for a natural model or display map category), then it is equivalent to a strict discrete fibration, which is a \nfs if it has small fibers.
  A ``$\dTy$-structure'' on a morphism is ``a way to express it as the comprehension of a dependent type''.
\end{rmk}

Now we sketch the interpretation of the basic type forming operations.
For any natural pseudo-model, define $\dTy^\Pi,\dTy^\Sigma,\dTy^{\Idtype},\dTy^{+} \in\Ehat$ as follows:
\begin{itemize}
\item $\dTy^\Pi(\Gamma)$ is the groupoid of pairs $(A,B)$ with $A\in \dTy(\Gamma)$ and $B\in \dTy(\Gamma\ce A)$.
\item $\dTy^\Sigma = \dTy^\Pi$.
\item $\dTy^{\Idtype}(\Gamma)$ is the groupoid of triples $(A,x,y)$ with $A\in \dTy(\Gamma)$ and $x,y\in \dTm(\Gamma)$ with $\varpi(x)=\varpi(y)=A$. % (these strict equalities are sensible because $\varpi$ is a discrete fibration).
\item $\dTy^{+} = \dTy\times\dTy$, so $\dTy^+(\Gamma)$ is the groupoid of pairs $(A,B)$ with $A,B\in\dTy(\Gamma)$.
\end{itemize}
The following notions are mostly as in~\cite[Definition 3.4.2.8]{lw:localuniv}:
\begin{itemize}
\item A \textbf{pseudo-stable class of $\Pi$-types} in a natural pseudo-model is a morphism $\Pi:\dTy^\Pi \to \dTy$ (corresponding to the formation rule) together with appropriate extra structure on each type $\Pi(A,B)$ (corresponding to the introduction, elimination, and equality rules).
\item A \textbf{strictly stable class of $\Pi$-types} is a pseudo-stable class for which \dTy is a natural model (since then the morphism $\Pi:\dTy^\Pi \to \dTy$ must be strictly natural, and so on).
\item A \textbf{weakly stable class of $\Pi$-types} is a span $\dTy^\Pi \ot \dG \to \dTy$ in which the first leg $\dG \to \dTy^\Pi$ is a surjective strict fibration (\cref{defn:dfib}), together with appropriate extra structure on the images of the second leg $\dG\to\dTy$.
  This is closely related to~\cite[Definition 3.4.2.5]{lw:localuniv} but more functorial\footnote{I am indebted to Peter Lumsdaine for suggesting this rephrasing.}; the objects of $\dG$ are the ``good $\Pi$-types''.
\end{itemize}
Similarly, we define all three kinds of $\Sigma$-types, identity types, and binary sum types.
Formal definitions of the ``appropriate extra structure'' can be found in~\cite{lw:localuniv}, and extensions to higher inductive types can be found in~\cite{ls:hits}; here we are recalling only the parts of the construction that are relevant for our extension to universes.

\begin{eg}
  The canonical natural pseudo-model of a \ttmt has pseudo-stable $\Pi$-types and $\Sigma$-types.
  It has only weakly stable $\Idtype$-types, but even in this case the type-\emph{forming} operation can be taken to be pseudo-stable (powers with $\Delta[1]$ as in \cref{sec:ttmt}).
  The translation from category-theoretic structure to type-theoretic structure in these cases can be found in~\cite[\sect1.4 and Proposition 2.3.3]{klv:ssetmodel} (for simplicial sets), \cite[Theorem 3.1]{aw:htpy-idtype}, \cite[Theorem 2.17]{warren:thesis}, \cite[Theorem 26]{ak:htmtt}, \cite[Theorem 4.2.2]{lw:localuniv}, and~\cite[\sect3]{awodey:natmodels}.

  However, binary sums are an example where weak stability matters even for the formation rule: we take the objects of $\dG(Y)$ to consist of three fibrations $A\fib Y$, $B\fib Y$, and $C\fib Y$ together with an acyclic cofibration $A+B \acof C$ over $Y$.
  By~\cite[Theorem 3.3]{ls:hits}, such data is stable under pullback, hence defines an object $\dG\in\Ehat$.
  The projection $\dG \to \dTy^{+} = \dFib\times \dFib$ picks out $A$ and $B$; this is a surjective strict fibration since fibrant replacements exist and are stable under isomorphism.
  The other projection $\dG \to \dTy= \dFib$ picks out $C$, with the rest of the structure constructed as in~\cite[Theorem 3.3]{ls:hits}.
\end{eg}

The coherence theorem of~\cite{lw:localuniv} says that if \E is locally cartesian closed (technically, a slightly weaker condition suffices), then weakly stable structure on $\dTy$ induces strictly stable structure on its left adjoint splitting $\las\dTy$ (\cref{eg:las}).
The basic observation is that
\[ \las\dTy^\Pi(\Gamma) \cong \coprod_{V_A\in \E \atop E_A \in \dTy(V_A)} \coprod_{V_B\in \E \atop E_B \in \dTy(V_B)} \coprod_{A:\Gamma\to V_A} \E(\Gamma\ce A,V_B) \]
and that when \E is locally cartesian closed, there is a universal object $V_A \lu V_B$\footnote{As remarked in~\cite{lw:localuniv}, this notation is abusive in that $V_A\lu V_B$ depends on $E_A$ too, not just on $V_A$ and $V_B$.} such that $\coprod_{A:\Gamma\to V_A} \E(\Gamma\ce A,V_B)$ is naturally isomorphic to $\E(\Gamma,V_A\lu V_B)$.
Thus $\las\dTy^\Pi$ is again a coproduct of representables:
\begin{equation}
\las\dTy^\Pi \cong \coprod_{V_A\in \E \atop E_A \in \dTy(V_A)} \coprod_{V_B\in \E \atop E_B \in \dTy(V_B)} \E(-,V_A \lu V_B).\label{eq:lastypi}
\end{equation}
Now if $\dTy$ has weakly stable $\Pi$-types, then since representables are projective in $\Ehat$ we can lift the map $\las\dTy^\Pi \to \dTy^\Pi$ along the surjective fibration $\dG\to\dTy^\Pi$ to the dashed map below:
\begin{equation}
  \begin{tikzcd}
    \las\dTy^\Pi \ar[d] \ar[dr,dashed] \ar[rr, dotted] & \ar[dr,phantom,"\cong"] & \las\dTy \ar[d] \\
    \dTy^\Pi & \dG \ar[r] \ar[l,two heads] & \dTy.
  \end{tikzcd}\label{eq:las-lift}
\end{equation}
The composite map $\las\dTy^\Pi \to \dG \to \dTy$ assigns to each $(V_A,E_A,V_B,E_B)$ a type $P \in \dTy(V_A\lu V_B)$, so that $(V_A,E_A,V_B,E_B) \mapsto (V_A \lu V_B, P)$ defines the dotted lifting making the quadrilateral commute up to isomorphism.
The rest of the structure is treated similarly, as are $\Sigma$-types, $\Idtype$-types, and so on.

Now we define universes in natural models and pseudo-models.

\begin{defn}
  A \textbf{level structure} is a partially ordered set \cL with binary joins and a partial endofunction $\suc : \cL \rightharpoonup \cL$ (not required to respect the ordering).
\end{defn}

\begin{eg}
  Any ordinal $\mu$ is a level structure, with $\suc(\al) = \al+1$ whenever $\al+1<\mu$ (thus $\suc$ is total if $\mu$ is a limit ordinal).
  Important special cases are $\mu=0$, $1$, $\om$, and $\om+1$.
\end{eg}

\begin{defn}\label{defn:tu}
  Let $\varpi:\dTm\to\dTy$ be a natural pseudo-model on \E, and \cL a level structure.
  An \textbf{\cL-family of pseudo Tarski universes} consists of:
  \begin{enumerate}
  \item For each $\al\in\cL$, types $\U_\al \in \dTy(\ec)$ and $\El_\al \in \dTy(\ec\ce \U_\al)$.\label{item:tu1}
  \item An extension of the function $\al \mapsto (\pi_\al : \ec\ce \U_\al \ce \El_\al \to \ec\ce \U_\al)$ to a functor from \cL to the category whose objects are morphisms in \E and whose morphisms are pullback squares in \E, sending each $\al\le\be$ to a pullback square\label{item:tu2}
    \[
      \begin{tikzcd}
        \ec\ce \U_\al \ce \El_\al \ar[d,"{\pi_\al}"'] \ar[r]\drpullback &
        \ec\ce \U_\be \ce \El_\be \ar[d,"{\pi_\be}"] \\
        \ec\ce \U_\al \ar[r,"{\Lift}"'] & \ec\ce \U_\be.
      \end{tikzcd}
    \]
  \item Whenever $\suc(\al)$ is defined, a morphism $\iu_\al : \ec \to \ec\ce \U_{\suc(\al)}$ and an isomorphism $\iu_\al^*(\El_{\suc(\al)}) \cong \U_\al$ in $\dTy(\ec)$.\label{item:tu3}
  \end{enumerate}
  We call it \textbf{strict} if $\varpi$ is a natural model (hence $\iu_\al^*(\El_{\suc(\al)}) = \U_\al$).
\end{defn}

Inside of type theory,~\ref{item:tu1} says that each universe is a type $\U_\al$ equipped with a coercion $(X:\U_\al) \vdash (\El_\al(X) \type)$ allowing us to regard its elements as types.
(The presence of the coercion $\El_\al$ is what makes these ``Tarski'' universes, in constrast to ``Russell'' universes whose elements are \emph{themselves} types; see \cref{rmk:russell}.)

Condition~\ref{item:tu2} says that any type can be lifted to a larger universe, $(X:\U_\al) \vdash (\Lift(X) : U_\be)$, which is isomorphic to it via functions $\lift : \El_\al(X) \to \El_\be(\Lift(X))$ and $\low : \El_\be(\Lift(X)) \to \El_\al(X)$ such that $\lift\circ\low \jdeq \iid$ and $\low\circ\lift \jdeq \iid$ judgmentally.
Finally, condition~\ref{item:tu3} says (in the strict case) that the universe $\U_\al$ is an \emph{element} of its successor universe: we have $\iu_\al : \U_{\suc(\al)}$ such that $\El_{\suc(\al)}(\iu_\al) \jdeq \U_\al$.

\begin{eg}\label{eg:ttmt-tu}
  Let \E be a \ttmt, let \la satisfy \cref{thm:uf-fibrant} for \E, and let \cL be a set of regular cardinals \al such that $\al\shgt\la$, with $\pi_\al : \Util_\al\fib U_\al$ the fibration constructed for \al in \cref{thm:uf-fibrant}.
  The induced well-ordering on \cL makes it a level structure, with $\suc(\al)$ the least element of \cL such that $U_\al$ is $\suc(\al)$-presentable (if such exists).

  We will extend $\{\pi_\al\}$ to an \cL-family of pseudo Tarski universes for the canonical natural pseudo-model of \E.
  Let $\ka\in\cL$ and assume inductively that we have constructed pullback squares as in \cref{defn:tu}\ref{item:tu2} for all $\al<\be<\ka$, which are additionally \F-morphisms.
  Then since $\Fka$ is \local, the map
  \[\colim_{\be<\ka} \Util_\be \to \colim_{\be<\ka} U_\be
  \]
  is a relatively \ka-presentable \F-algebra which pulls back to $\Util_\be$ over each $U_\be$.
  Thus it is classified by some map to $U_\ka$, i.e.\ we have an \F-morphism
  \[
    \begin{tikzcd}
      \colim_{\be<\ka}\Util_\be \ar[d,"{\pi_\al}"'] \ar[r]\drpullback &
      \Util_\ka \ar[d,"{\pi_\be}"] \\
      \colim_{\be<\ka}U_\be \ar[r,"{\Lift}"'] & U_\ka
    \end{tikzcd}
  \]
  inducing \F-morphisms $\Lift : U_\be \to U_\ka$ for all $\be<\ka$.
  Finally, since $U_\al$ is $\suc(\al)$-presentable by assumption whenever $\suc(\al)$ is defined, it is classified by some map $1 \to U_{\suc(\al)}$ which we can take as $\iu_\al$.
\end{eg}

In general we expect universes to be closed under the type operations.
Recall from \cref{eg:rep-cwf} that each $\pi_\al$ induces a represented natural model; thus we can ask the latter to have strictly stable $\Pi$-types, $\Sigma$-types, and so on.
In fact, we have $\E(-,U)^\Pi \cong \E(-,U\lu U)$ and similarly for the domains of all the other type operations.
Thus $\Pi$-types on $\E(-,\pi)$ are given by a morphism $\Pi : U\lu U \to U$ in \E, and so on.
We also expect $\El$ to respect these operations.

This corresponds type-theoretically to an operation
\[(X:\U_\al), (Y:\El_\al(X) \to \U_\al) \vdash (\Pi_\al(X,Y): \U_\al)\]
such that (in the strict case) $\El_\al(\Pi_\al(X,Y)) \jdeq \prod_{x:\El_\al(X)} \El_\al(Y(x))$.
But especially since our universes are non-cumulative, it is more useful to have relative operations
\[(X:\U_\al), (Y:\El_\al(X) \to \U_\be) \vdash (\Pi_{\al,\be}(X,Y): \U_{\al\vee\be})\]
in which the base and fibers can lie in different universes.
This suggests the following.

\begin{defn}
  Let $\varpi:\dTm\to\dTy$ be a natural pseudo-model on \E that has weakly stable $\Pi$-types.
  An \cL-family of pseudo Tarski universes is \textbf{closed under $\Pi$-types} if for each $\al,\be\in\cL$ we have dashed and dotted morphisms making the following diagram commute (with an isomorphism in the quadrilateral):
  \begin{equation}
    \begin{tikzcd}
      \E(-,(\ec\ce\U_\al)\lu (\ec\ce\U_\be)) \ar[d] \ar[dr,dashed] \ar[rr, dotted] &
      \ar[dr,phantom,"\cong"] & \E(-,\ec\ce\U_{\al\vee\be}) \ar[d] \\
      \dTy^\Pi & \dG \ar[r] \ar[l,two heads] & \dTy.
    \end{tikzcd}\label{eq:tu-lift}
  \end{equation}
  We say it is \textbf{strictly closed} if $\varpi$ is a natural model (so that in particular the isomorphism is an identity).
\end{defn}

Of course, this is very similar to the construction of split structure on $\las\dTy$ in~\eqref{eq:las-lift}.
Note that, as there, the dashed lift can always be chosen since $\dG\to\dTy^\Pi$ is surjective and representables are projective.

We can similarly define closure under $\Sigma$-types, identity types, binary sum types, and so on.
Note that we do not have to say anything about the introduction or elimination rules, as these only happen after the coercions $\El$ are applied.

\begin{eg}
  The discussion in \cref{sec:ttmt} implies that the families of universes from \cref{eg:ttmt-tu} for the canonical natural pseudo-model of a \ttmt \E are always closed under $\Sigma$-types, can be chosen to be closed under identity types and binary sum types by taking $\la$ sufficiently large, and are closed under $\Pi$-types if each $\al\in\cL$ is inaccessible.
  
  Consider for instance the case of $\Pi$-types: we choose a dashed lift in~\eqref{eq:tu-lift}, and then the composite $\E(-,U_\al \lu U_\be) \to \dG\to\dTy$ classifies the universal dependent product of a relatively \be-presentable fibration along a relatively \al-presentable one.
  Since $\al\vee\be$ is inaccessible, this is relatively $(\al\vee\be)$-presentable, hence is classified by some map $U_\al \lu U_\be \to U_{\al\vee\be}$.
\end{eg}

Now suppose $\varpi:\dTm\to\dTy$ is a natural pseudo-model on \E with weakly stable $\Pi$-types and a family of universes closed under them; we would like $\las\dTy$ to inherit a family of universes \emph{strictly} closed under $\Pi$-types.
However, although the map $\E(-,(\ec\ce\U_\al)\lu (\ec\ce\U_\be)) \to \dTy^\Pi$ in~\eqref{eq:tu-lift} factors through $\las\dTy^\Pi$ by picking out the summand of~\eqref{eq:lastypi} corresponding to $(V_A,E_A,V_B,E_B)=(\ec\ce\U_\al, \ec\ce\U_\al\ce\El_\al, \ec\ce\U_\be, \ec\ce\U_\be\ce\El_\be)$, the square we might hope for does not commute:
\begin{equation*}
  \begin{tikzcd}
    \E(-,(\ec\ce\U_\al)\lu (\ec\ce\U_\be)) \ar[d] \ar[rr] \ar[drr,phantom,"\scriptstyle\text{(does not commute)}"] &&
    \E(-,\ec\ce\U_{\al\vee\be}) \ar[d]\\
    \las\dTy^\Pi \ar[d] \ar[dr] \ar[rr] & \ar[dr,phantom,"\cong"] & \las\dTy \ar[d] \\
    \dTy^\Pi & \dG \ar[r] \ar[l,two heads] & \dTy.
  \end{tikzcd}\label{eq:las-tu-lift}
\end{equation*}
The left-bottom composite is the inclusion into a summand of $\las\dTy$ with $V_A = (\ec\ce\U_\al)\lu (\ec\ce\U_\be)$, while the top-right composite maps into a summand with $V_A = \ec\ce\U_{\al\vee\be}$.
Thus we need to generalize the local universes construction.

\begin{defn}
  A \textbf{family of local universes} for a natural pseudo-model $\varpi : \dTm\to \dTy$ on \E is an indexed family $\cV$ of pairs $(V,E)$ with $V\in \E$ and $E\in \dTy(V)$.
  %
  If $\varpi$ has weakly stable $\Pi$-types
  \( \dTy^\Pi \ot \dG \xto{\Pi} \dTy \),
  then \cV \textbf{supports $\Pi$-types} if it is equipped with, for every $(V_A,E_A)$ and $(V_B,E_B)$ in \cV, some $G$ in the fiber of $\dG(V_A \lu V_B)$ over the universal pair of types in $\dTy^\Pi(V_A\lu V_B)$ (this is $(E_A[\pi_A],E_B[\pi_B])$ in the notation of~\cite[Lemma 3.4.2.4]{lw:localuniv}) and a pullback square
  \begin{equation}
    \begin{tikzcd}
      (V_A\lu V_B)\ce \Pi(G) \ar[d] \ar[r] \drpullback & V_{\Pi(A,B)}\ce E_{\Pi(A,B)} \ar[d] \\
      V_A\lu V_B \ar[r] & V_{\Pi(A,B)}
    \end{tikzcd}\label{eq:lu-pi-pb}
  \end{equation}
  for some $(V_{\Pi(A,B)},E_{\Pi(A,B)})\in\cV$.
\end{defn}

\begin{eg}\label{eg:lu-triv}
  The \textbf{trivial} family of local universes is the family of \emph{all} such pairs $(V,E)$, with $V_{\Pi(A,B)} = V_A \lu V_B$ and~\eqref{eq:lu-pi-pb} the identity square (for some arbitrary choice of $G$).
\end{eg}

\begin{eg}\label{eg:lu-tu}
  An \cL-family of universes (\cref{defn:tu}) induces a family of local universes with \cV the \cL-indexed family of pairs $(\ec\ce\U_\al,\El_\al)$ for $\al\in\cL$.
  If the family of universes is closed under $\Pi$-types, then \cV supports $\Pi$-types with $V_{\Pi(\al,\be)} = \ec\ce\U_{\al\vee\be}$.
\end{eg}

\begin{eg}\label{eg:lu-mixed}
  We can also consider the disjoint union of the families of local universes from \cref{eg:lu-triv,eg:lu-tu}.
  (Thus the pairs $(\ec\ce\U_\al,\El_\al)$ appear twice in the indexed family \cV, once as an arbitrary pair $(V,E)$ and once as a ``universe'' pair.)
  We define the $\Pi(A,B)$ operation to restrict to those of \cref{eg:lu-triv,eg:lu-tu} when $A$ and $B$ are of the same kind, and when they are of different kinds we ``forget'' that the universe pair $(\ec\ce\U_\al,\El_\al)$ is special and proceed as in \cref{eg:lu-triv}.
\end{eg}

Now given a family of local universes \cV, we define 
\[ \lasv\dTy = \coprod_{(V,E)\in \cV} \E(-,V) \]
with $\lasv\dTm = \lasv\dTy \times_{\dTy} \dTm$ as in \cref{eg:las}.
We have
\[ \lasv\dTy^\Pi = \coprod_{(V_A, E_A) \in\cV} \coprod_{(V_B, E_B) \in\cV} \E(-,V_A \lu V_B),\]
so if \E has weakly stable $\Pi$-types and \cV supports $\Pi$-types, the squares~\eqref{eq:lu-pi-pb} are precisely what is needed to induce the top dotted map:
\begin{equation}
  \begin{tikzcd}
    \lasv\dTy^\Pi \ar[d] \ar[dr,dashed] \ar[rr, dotted] & \ar[dr,phantom,"\cong"] & \lasv\dTy \ar[d] \\
    \dTy^\Pi & \dG \ar[r] \ar[l,two heads] & \dTy.
  \end{tikzcd}\label{eq:lasv-lift}
\end{equation}
Thus $\lasv\dTy$ again has strictly stable $\Pi$-types.
(We do need to check that the strictification of the introduction and elimination rules also works.
But here we can use the ordinary local universes construction, with the reindexing of $E_{\Pi(A,B)}$ to $V_A \lu V_B$ along~\eqref{eq:lu-pi-pb} replacing its isomorph $\Pi(G)$.)

Applied to \cref{eg:lu-triv} this reproduces the original local universes model $\las\dTy$.
On the other hand, applied to \cref{eg:lu-tu} in the case $\cL=1$, it reproduces the ``global universe'' coherence theorem of~\cite{klv:ssetmodel}.
More generally, when we apply it to \cref{eg:lu-tu,eg:lu-mixed}, we obtain exactly the commutative squares we needed:
\begin{equation*}
  \begin{tikzcd}
    \E(-,(\ec\ce\U_\al)\lu (\ec\ce\U_\be)) \ar[d] \ar[rr] \ar[drr,phantom,"\scriptstyle\text{(commutes!)}"] &&
    \E(-,\ec\ce\U_{\al\vee\be}) \ar[d]\\
    \lasv\dTy^\Pi \ar[d] \ar[dr] \ar[rr] & \ar[dr,phantom,"\cong"] & \lasv\dTy \ar[d] \\
    \dTy^\Pi & \dG \ar[r] \ar[l,two heads] & \dTy.
  \end{tikzcd}\label{eq:las-tu-lift}
\end{equation*}
We are thus most of the way to the following.

\begin{thm}\label{thm:lasv}
  Suppose $\varpi:\dTm\to\dTy$ is a natural pseudo-model on \E with weakly stable $\Pi$-types, \cL a level structure, and we are given an \cL-family of pseudo Tarski universes for $\dTy$ that is closed under $\Pi$-types.
  Then for \cV as in \cref{eg:lu-mixed}, the natural model $\lasv\varpi:\lasv\dTm\to\lasv\dTy$ has strictly stable $\Pi$-types and an \cL-family of strict Tarski universes strictly closed under $\Pi$-types.
\end{thm}
\begin{proof}
  We have not yet actually constructed an \cL-family of universes in $\lasv\dTy$ itself.
  Given $\al$, if $\suc(\al)$ is defined, we take $\lasU{\al}\in \lasv\dTy(\ec)$ to be $(\ec\ce\U_{\suc(\al)}, \El_{\suc(\al)}, \iu_\al)$, i.e.\ as the ``delayed substitution'' corresponding to the given isomorphism $\iu_\al^*(\El_{\suc(\al)}) \cong \U_\al$.
  Then the projection $\lasv\dTy \to \dTy$ maps $\lasU{\al}$ to $\U_\al$, so we can take $\lasEl{\al} = \El_\al$.
  On the other hand, if $\suc(\al)$ is not defined, we take $\lasU{\al}$ to be $(\ec,\U_{\al},\id_\ec)$.

  Since the comprehension of $\lasv\dTy$ factors through $\dTy$, the lifting isomorphisms of the latter immediately induce ones for the former, and similarly for the morphisms $\lasu{\al}$.
  And the definition of substitution in $\lasv\dTy$ implies that $\lasu{\al}^*(\lasEl{\suc(\al)}) = \lasU{\al}$ when $\suc(\al)$ is defined.
  Thus we have a family of strict Tarski universes for $\lasv\dTy$, and the observations above show that it is strictly closed under $\Pi$-types.
\end{proof}

We can deal similarly with $\Sigma$-types, identity types, and binary sum types.
Thus we have the following more precise version of \cref{thm:ttmt-models}:

\begin{thm}\label{thm:ttmt-lasv}
  For any \ttmt \E, there is a regular cardinal \la such that if the inaccessible cardinals greater than $\la$ have order type $\mu$, for some ordinal $\mu$, then the canonical natural pseudo-model of \E can be replaced by a natural model $\lasv\Fib$ with:
  \begin{enumerate}
  \item Strictly stable $\Sigma$-types, a unit type, $\Pi$-types with function extensionality, identity types, and binary sum types.\label{item:last1}
  \item A strictly stable empty type, natural numbers type, circle type $S^1$, sphere types $S^n$, and other specific ``cell complex'' types such as the torus $T^2$.\label{item:last2}
  \item A $\mu$-family of strict Tarski universes, strictly closed under the type formers~\ref{item:last1} and containing the types~\ref{item:last2}, and satisfying the univalence axiom.\label{item:last3}
  \item Strictly stable $\mathsf{W}$-types, pushout types, truncations, localizations, James constructions, and many other recursive higher inductive types.\label{item:last4}
  \end{enumerate}
\end{thm}
\begin{proof}
  As in \cref{thm:ttmt-models}, we take \la large enough that the types in~\ref{item:last2} are $\la$-presentable.
  Then \cref{thm:lasv,eg:ttmt-tu} yield~\ref{item:last1} and~\ref{item:last3}.
  Finally, the local universes coherence arguments for higher inductive types from~\cite{ls:hits} apply just as well to the modified version $\lasv\dTy$, yielding~\ref{item:last2} and~\ref{item:last4}.
\end{proof}

\begin{rmk}\label{rmk:russell}
  If the successor in \cL is totally defined (e.g.\ if $\mu$ in \cref{thm:ttmt-lasv} is a limit ordinal, such as $\om$), then in \cref{thm:lasv} we can use \cref{eg:lu-tu} instead of \cref{eg:lu-mixed}.
  This gives a model like that of~\cite{coquand:pshf-model} in which every type belongs to a unique universe, which is the closest possible approximation to (non-cumulative) Russell-type universes obtainable in a natural model.
  Probably this construction could also be adapted to a semantic structure that models Russell-type universes more closely, such as the generalized algebraic theories of~\cite{coquand:cannorm-dtt,sterling:att-universes,chs:hocan-cubical}.
\end{rmk}

\begin{rmk}
  The universes we have constructed here are \emph{non-cumulative} in that $\El(\Lift(A))$ is judgmentally isomorphic to $\El(A)$ rather than judgmentally equal to it.
  Such universes are implemented in Agda and Lean, and are easy to use as long as we formulate the type formers appropriately using $\al\vee\be$.
  (Note that the universes in Agda have order type $\om+1$, with the $\om^{\mathrm{th}}$ universe not belonging to any larger universe; this is covered by \cref{thm:ttmt-lasv} as long as $\mu\ge \om+1$.)
  Other proof assistants such as Coq use cumulative universes instead, as does~\cite{hottbook}; it should be possible to model these in a \ttmt as well (some ideas are sketched in~\cite{shulman:invdia,shulman:elreedy}), but we leave this for future work.
\end{rmk}

%%% Local Variables:
%%% mode: latex
%%% TeX-master: "univinj"
%%% End:
