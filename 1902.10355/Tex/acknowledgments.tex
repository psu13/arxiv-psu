\chapter*{Acknowledgments}
\doublespacing

\noindent\lettrine{I}{} am delighted to take this opportunity to acknowledge the people I learned from the most and those who supported me during my doctoral research at Yale. In this short Acknowledgements section, it is infeasible to properly thank everyone. I apologize in advance for any potential shortcomings. This is particularly relevant for those I worked with most closely during the initial years of my Ph.D., before I changed course by proposing and carrying out the quantum jump project in the final two years. The product of these two years constitutes the remainder of this Dissertation. 

To set the stage for the acknowledgements below, let me first briefly recount the origin and story of this work. In the summer of 2015, I traveled to Scotland to participate in the Scottish Universities Summer School in Physics (SUSSP71) by partly self-funding my participation. There I heard a cogent lecture by Howard J. Carmichael, which radically changed the direction of my doctoral scientific inquiry. Howard presented his gedanken experiment for catching and reversing a quantum jump mid-flight, which made the striking prediction that the nature of quantum jumps could be continuous and coherent. The discussion emphasized that a test of the conclusions remains infeasible, since the requisite experimental conditions remain far out of reach of atomic physics (see Chap.~1). Excited by the ideas, however, I brainstormed and ran simulations to find a possible realization of the gedanken experiment, but in a different domain --- superconducting quantum circuits. Although much was unclear in the leap from the quantum optics to the superconducting realm, I reached out to Howard and found him very receptive to my still developing ideas.  After an initial rebuff of my proposed experiment on my return to Yale, I spent the fall and early months of the next year developing the concepts and implementation in detail to prove their validity, and the feasibility of the project. Following many lessons and in-depth discussions with Michel, and the input of Howard and the people acknowledged below, the experiment was successfully realized just short of two years later. The predictions were not only experimentally confirmed but were further expanded to demonstrate the coherent, continuous, and deterministic evolution of the quantum jump even in the absence of a coherent drive on the jump transition. 

Thus, first and foremost, I would like to express my deep and heartfelt gratitude to my dissertation advisor, \textsc{Michel H. Devoret}.  Michel’s reputation as a great mentor and a brilliant physicist, agreed upon by all sources, notably preceded him as early as during my undergraduate days at Berkeley. Since I joined Michel’s group, I have only developed an ever-growing admiration for his breadth of knowledge and deep thinking. It has been a privilege and a pleasure to learn from and work closely with Michel. I am endlessly thankful for the countless hours and legions of lessons he so elegantly delivered on physics, writing, aesthetics in science, and innumerable other subjects. I am especially grateful to him for allowing me to take the unusual path of proposing and carrying out a complex, original experiment with little relation to any other project in the lab or to my previous work. I deeply appreciate this unique opportunity and I am especially thankful for his support, trust, and belief in the ideas of a young graduate student.  Michel also taught me how to communicate science clearly and to follow the highest standards, to pay attention to every detail, including font choices and color combinations. Michel continues to be my role model for a scientist of the highest caliber. I will dearly miss our inspired, in-depth conversations on science and beyond. I’ll also miss the collaborative and knowledge-rich environment of Michel’s \textsc{Quantronics Laboratory (Qlab)} on the fourth floor in Becton.

As related above, it was my good fortune to meet \textsc{Howard J. Carmichael} at SUSSP71, where I was inspired by his lecture on quantum jumps. It has been a pleasure and a privilege to work with Howard, who also inspires me with his record of pathbreaking advances in quantum optics theory and his seminal role in the foundation of quantum trajectory theory [the term was coined by him in \cite{Carmichael1993}]. His open reception of my ideas since the beginning, his generosity with his time, and his continued support has been beyond measure. I am endlessly grateful to Howard, as well as his student, \textsc{Ricardo Gutiérrez-Jáuregui}, whom I also had the pleasure of meeting at SUSSP71, for the indispensable theoretical modeling of the final experimental results. The project greatly benefitted from the discussions with and theoretical contributions of both Howard and Ricardo. I am particularly indebted to Howard for his thoughtful edits and input in the writing and revising of the paper we have submitted, and the enlightening lessons I picked up along the way. 

I deeply appreciate the theoretical discussions during the inception of the project with Professor \textsc{Mazyar Mirrahimi}, which were of significant help in navigating the landscape of cascaded non-linear parametric processes in circuit quantum electrodynamics (cQED), used in the readout of the atom.  I am grateful to Mazyar for patiently accommodating the many questions I had. I also want to express my deep gratitude to Professor \textsc{Steven M. Girvin} for the cQED and quantum trajectory discussions we had, for his detailed reading and edits of this dissertation manuscript, and for his kind manner of teaching and enlightening his students with countless deep insights, and warm encouragement. I feel indebted to Professor \textsc{Jack Harris}, who advised me early on at Yale on a project in optomechanics in his lab, and also provided careful and thoughtful comments on this dissertation manuscript. Jack has always been inspirational and supportive of my efforts. I also owe a debt to all the professors and senior scientists who taught me a great deal of what I know, \textsc{Rob Schoelkopf, Luigi Frunzio, Liang Jiang, Doug Stone}, and those who significantly broadened my horizons, and improved my understanding of physics: \textsc{Daniel Prober, Leonid Glazmann, Peter Rakich, David DeMille, Hui Cao, Yoram Alhassid, Paul Fleury, Sean E. Barrett}.

During the quantum jump project, I had the pleasure of working very closely with \textsc{Shantanu Mundhada}. Shantanu was instrumental to the success of the project by contributing a great deal with his aid in fabrication and the initial DiTransmon design of the device. \textsc{Philip Reinhold} had developed an outstanding Python platform for the control of the FPGA, and helped me a great deal in debugging the control code. \textsc{Shyam Shankar} aided in the fabrication of the device and provided general support to the lab in his kind and patient manner. I appreciate the many fruitful discussions with \textsc{Victor V. Albert, Matti P. Silveri,} and \textsc{Nissim Ofek}. Victor in particular addressed an aspect of the Lindblad theoretical modeling regarding the waiting-time distribution. It was Nissim and \textsc{Yehan Liu} who spearheaded the initial FPGA development.  In later discussions, I benefited from insightful conversations with \textsc{Howard M. Wiseman, Klaus  Mølmer, Birgitta Whaley, Juan P. Garrahan, Ananda Roy, Joachim Cohen,} and \textsc{Katarzyna Macieszczak.} 

In my earlier days at Yale, I learned much about low-temperature experimental physics from \textsc{Ioan Pop} and \textsc{Nick Masluk}. I had the good fortune to work with them and \textsc{Archana Kamal} (who inspired me with her dual mastery of experiment and theory) on the development of a superinductance with a Josephson junction array \citep{Masluk2012, Minev2012-APSMM}. Ioan and I, after spending three months repairing nearly every part of our dilution system, continued to work together for the next five years. We demonstrated the first superconducting whispering-gallery mode resonators (WGMR), which achieved the highest quality factors of planar or quasi-planar quantum structures at the time \citep{Minev2013}. These led us to demonstrate the first multi-layer (2.5D), flip-chip cQED architecture \citep{Minev2015-patent, Minev2016}, which demonstrated the successful unification of the advantages of the planar (2D) and three-dimension (3D) cQED architectures \citep{Minev2013-APSMM,Minev2014-APSMM,Minev2015-APSMM,Serniak2015-APSMM}. During the later phase of this project, I had the pleasure of working with \textsc{Kyle Serniak}, whom I thank for his many hours in the cleanroom. During these first years, I greatly benefited from Ioan’s mentorship, his energetic and cheerful character, and the pleasure of wonderful gatherings hosted by Cristina and him; additionally, our sailing lessons. While none of the work described in this paragraph is featured in this Dissertation --- as it could form an orthogonal, independent dissertation --- its results are detailed in the cited literature. 

During the development of the 2.5D architecture, I came up with an alternative idea for the quantization of black-box quantum circuits --- the energy-participation ratio (EPR) approach to the design of quantum Josephson circuits \citep{Minev2018-EPR}. I am grateful to Michel and to \textsc{Zaki Leghtas} for their support along the way for this unanticipated project. More generally, I had the extreme pleasure of working closely with Zaki and learning a great deal of physics from him in lab and over countless dinners. During the EPR project, I was privileged to coach a number of talented undergraduate students, whose enthusiasm and time I am thankful for: \textsc{Dominic Kwok, Samuel Haig, Chris Pang, Ike Swetlitz, Devin Cody, Antonio Martinez,} and \textsc{Lysander Christakis}.

Overall, many students and post-docs in \textsc{Qlab} and \textsc{RSL} contributed to the success of my time at Yale. I would like to thank them all. I have been fortunate to remain close friends and colleagues with my incoming class, \textsc{Kevin Chou, Eric Jin, Uri Vool, Theresa Brecht}, and \textsc{Jacob Blumoff}, and to learn dancing with Kevin. It has been a particular pleasure to work more closely with \textsc{Serge Rosenblum, Chan U Lei, Zhixin Wang, Vladimir Sivak, Steven Touzard}, and \textsc{Evan Zalys-Geller}. As part of \textsc{Qlab}, I had the privilege to work, although more indirectly, with a number of excellent postdoctoral researchers, including  \textsc{Michael Hatridge, Baleegh Abdo, Ioannis Tsioutsios, Philippe Campagne-Ibarcq, Gijs de Lange, Angela Kou}, and \textsc{Alexander Grimm}. I also had the pleasure to occasionally collaborate with a number of the other graduate students in \textsc{Qlab}, including \textsc{Anirudh Narla, Clarke Smith, Nick Frattini, Jaya Venkatraman, Max Hays, Xu Xiao, Alec Eickbusch, Spencer Diamond, Flavius Schackert, Katrina Sliwa,} and \textsc{Kurtis Geerlings}.

Our work was always mutually supported and very closely intertwined with that of Rob Scholekopf’s lab, and I thank \textsc{Hanhee Paik, Gerhard Kirchmair, Luyan Sun, Chen Wang, Reinier Heeres, Yiwen Chu, Brian Lester,} and \textsc{Vijay Jain}.  There are also many graduate students in Rob’s group I would like to acknowledge: \textsc{Andrei Petrenko, Matthew Reagor, Brian Vlastakis, Eric Holland, Matthew Reed,  Adam Sears, Christopher Axline, Luke Burkhart, Wolfgang Pfaff,   Yvonne Gao, Lev Krayzman, Christopher Wang, Taekwan Yoon, Jacob Curtis}, and \textsc{Sal Elder}. I benefited from a number of thoughtful theoretical discussions with \textsc{Linshu Li} and \textsc{William R Sweeney}.  

Our department would not run without the endless support and help provided to us by \textsc{Giselle M. DeVito, Maria P. Rao, Theresa Evangeliste}, and \textsc{Nuch Graves}, or that provided by \textsc{Florian Carle}and \textsc{Racquel Miller} for the \textsc{Yale Quantum Institute (YQI)}.

My time at Yale would not be what it was without \textsc{Open Labs}, a science outreach and  careers pathways not-for-profit I founded in 2012, and the innumerable, wonderful people who helped me develop it into a nation-wide organization that has reached over 3,000 young scholars and parents and coached more than several hundred graduate students. In 2015, I had the good fortune to meet two of my best friends and kindred spirits, \textsc{Darryl Seligman} and \textsc{Sharif Kronemer}. I am deeply grateful to Darryl for his immeasurable effort in spearheading the expansion of Open Labs from Yale to Princeton, Columbia, Penn, and Harvard, and to Sharif for further growing and shaping Open Labs into a long-lasting sustainable organization. I would like to thank \textsc{Maria Parente} and \textsc{Claudia Merson} for believing in my nascent idea of Open Labs and providing support through \textsc{Yale Pathways to Science}. There are far too many other key people to thank for their volunteer work with Open Labs, but I must acknowledge \textsc{Jordan Feyngold, Ian Weaver, Munazza Alam, Aida Behmard, Matt Grobis, Kirsten Blancato, Nicole Melso, Shannon Leslie, Diane Yu, Lina Kroehling, Christian Watkins, Arvin Kakekhani}, and \textsc{Charles Brown}. More recently, I wish to express my gratitude to Yale for recognizing me with the Yale-Jefferson Award for Public Service and to the American Physical Society (APS) and National Science Foundation (NSF) for supporting Open Labs with an outreach grant award. 

Beyond the world of physics, I was fortunate to meet some of my best friends whose support and good cheer I am deeply thankful for, including \textsc{Rick Yang, Brian Tennyson, Xiao Sun, Rasmus Kyng, Marius Constantin, Stafford Sheehan}, and \textsc{Luis J.P.~Lorenzo}.  I am also very grateful to \textsc{Olga Laur} for her unstinting support during the writing of this work. Finally, the people whose contributions are the greatest yet the least directly visible are my family members. There are no words to describe the incomparable debt I owe each of you, especially to my parents \textsc{Lora} and \textsc{Kris}, and to those painfully no longer among us --- my grandparents, \textsc{Angelina} and \textsc{Tzvetko}, to whom I dedicate my dissertation. Your richness of knowledge, creativity in science and art, and unconditional love remain a beacon of inspirational light. Thank you!




\begin{center}
	
	``\emph{If I have seen a little further, it is by standing on the shoulders of Giants.}''  \\ - Isaac Newton, "Letter from Sir Isaac Newton to Robert Hooke,'' \\Historical Society of Pennsylvania.
	
\end{center} % February 5, 1675



% funding and awards , Blais 

% Paul Griffin , Sir Petar Knight , Petar Zoller, Philipe Grangier ,


\singlespacing

%\iffalse 

%\fi
