\doublespacing % required by GSAS
\begin{abstract}
\begin{changemargin}{-0.25cm}{-0.25cm}  % fit on one page
A quantum system driven by a weak deterministic force while under strong continuous energy measurement exhibits quantum jumps between its energy levels \citep{Nagourney1986, Sauter1986, Bergquist1986}. This celebrated phenomenon is emblematic of the special nature of randomness in quantum physics. The times at which the jumps occur are reputed to be fundamentally unpredictable. However, certain classical phenomena, like tsunamis, while unpredictable in the long term, may possess a degree of predictability in the short term, and in some cases it may be possible to prevent a disaster by detecting an advance warning signal. 
Can there be, despite the indeterminism of quantum physics, a possibility to know if a quantum jump is about to occur or not?
In this dissertation, we answer this question affirmatively by experimentally demonstrating that the completed jump from the ground to an excited state of a superconducting artificial atom can be tracked, as it follows its predictable ``flight,'' by monitoring the population of an auxiliary level coupled to the ground state. Furthermore, the experimental results demonstrate that the jump when completed is continuous, coherent, and deterministic. Exploiting these features, we catch and reverse a quantum jump mid-flight, thus deterministically preventing its completion. 
This real-time intervention is based on a particular lull period in the population of the auxiliary level, which serves as our advance warning signal. 
Our results, which agree with theoretical predictions essentially without adjustable parameters,  support the modern quantum trajectory theory % \citep{Carmichael1993,Gardiner1992-original-traj,Dalibard1992-original-traj,Korotkov1999-original-traj}
and  provide new ground for the exploration of real-time intervention techniques in the control of quantum systems, such  as early detection of error syndromes.
\end{changemargin}
\end{abstract}
\singlespacing
