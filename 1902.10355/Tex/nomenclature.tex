%!TEX root = ../main_dissertation.tex

%TODO:
% - check for consistent capitalization across all entries


% - - - - - - - - - - - - - - - - - - - - - - - - - - - - - - - -
%%% Acronyms and abbreviations

\nomenclature[ACqed]{CQED}{Cavity quantum electrodynamics}
\nomenclature[Acqed]{cQED}{Circuit quantum electrodynamics}
\nomenclature[An]{NMR}{Nuclear magnetic resonance}
\nomenclature[Aqnd]{QND}{Quantum non-demolition}
\nomenclature[Aqec]{QEC}{Quantum error correction}


\nomenclature[Abbq]{BBQ}{Black box quantization}
\nomenclature[Ahfss]{HFSS}{High-frequency electromagnetic simulation}
\nomenclature[Aepr]{EPR}{Energy participation ratio}
\nomenclature[Aljc]{LJC}{Linearized Josephson circuit}

\nomenclature[Adac]{DAC}{Digital-to-analog converter}
\nomenclature[Aacd]{ADC}{Analog-to-digital converter}
\nomenclature[Afpga]{FPGA}{Field-programmable gate array}
\nomenclature[Aawg]{AWG}{Arbitrary waveform generator}
\nomenclature[Asnr]{SNR}{Signal-to-noise ratio}


\nomenclature[Acw]{CW}{Continuous wave}
\nomenclature[Arf]{RF}{Radio frequency}
\nomenclature[Aif]{IF}{Intermediate frequency}
\nomenclature[Alo]{LO}{Local oscillator}
\nomenclature[Asma]{SMA}{SubMiniature version A RF connector}
\nomenclature[Avna]{VNA}{Vector network analyzer}
%\nomenclature[Asa]{SA}{Spectrum analyzer}
\nomenclature[Aiq]{IQ}{In-phase/quadrature}


\nomenclature[Ahemt]{HEMT}{High electron mobility transistor}
\nomenclature[Asquid]{SQUID}{Superconducting quantum interference device}
\nomenclature[Ajpc]{JPC}{Josephson parametric converter}


\nomenclature[Adrag]{DRAG}{Derivative removal by adiabatic gate}
\nomenclature[Aspam]{SPAM}{State preparation and measurement}


\nomenclature[Apovm]{POVM}{Positive operator valued measure}
\nomenclature[Acptp]{CPTP}{Completely positive and trace preserving}
\nomenclature[Arwa]{RWA}{Rotating wave approximation}
\nomenclature[Ahc]{$\mathrm{H.c.}$}{Hermitian conjugate}
\nomenclature[Asnr]{cNOT}{Controlled-NOT gate}

% Quantum trajectory theory related
\nomenclature[Asse]{SSE}{Stochastic Schr\"{o}dinger equation}
\nomenclature[Asme]{SME}{Stochastic master equation}
\nomenclature[Asde]{SDE}{Stochastic differential equation}
\nomenclature[Aqsde]{QSDE}{Quantum stochastic differential equation}

\nomenclature[Apmma]{PMMA}{Polymethyl methacrylate}
\nomenclature[Anmp]{NMP}{$N$-Methyl-2-pyrrolidone}
\nomenclature[Aipa]{IPA}{Isopropyl alcohol}
%\nomenclature[Arpm]{r.p.m.}{Revolutions per minute}





% - - - - - - - - - - - - - - - - - - - - - - - - - - - - - - - -
%%% Constants

\nomenclature[C1]{$h$}{Planck's constant
	\nomunit{$6.63\times 10^{-34}~\mathrm{J\,Hz}$} }
\nomenclature[C2]{$\hbar$}{Reduced Planck's constant ($=h/2\pi$)
	\nomunit{$1.05\times 10^{-34}~\mathrm{J\,Hz}$} }
\nomenclature[C3]{$e$}{Electron charge
	\nomunit{$1.60\times 10^{-19}~\mathrm{C}$} }
\nomenclature[C6]{$\Phi_0$}{Magnetic flux quantum ($=h/2e$)
	\nomunit{$2.07\times 10^{-15}~\mathrm{Wb}$} }
\nomenclature[C7]{$\phi_0$}{Reduced magnetic flux quantum ($=\hbar/2e$)
	\nomunit{$3.29\times 10^{-16}~\mathrm{Wb}$} }
\nomenclature[C8]{$R_Q$}{Resistance quantum ($=\hbar/\left(2e\right)^2$)
	\nomunit{$2.58\times 10^{4}~\mathrm{\Omega}$} }
\nomenclature[C9]{$k_B$}{Boltzmann constant
	\nomunit{$1.38\times 10^{-23}~\mathrm{J\, K^{-1}}$} }





% - - - - - - - - - - - - - - - - - - - - - - - - - - - - - - - -
%%% Classical stochastic differential equations 

\nomenclature[Dd3]{$\Pr{X=x}$}{Probability that classical stochastic variable $X$ has value $x$}
\nomenclature[Dd4]{$\wp(x)$}{For discrete event outcome $x$: probability of the event $\wp(x) = \Pr{X=x}$; for continuous event outcome $x$: probability density of the event $\wp(x)\mathrm{d}x = \Pr{X\in\left(x,x+\mathrm{d}x\right)}$}
\nomenclature[Dd5]{$\E{X}$}{Expectation value of classical stochastic variable $X$}

\nomenclature[Df1]{$N(t)$}{Continuous-time stochastic point process whose value corresponds to the number of detection events (e.g., photodetections) up to time $t$}
\nomenclature[Df2]{$W(t)$}{Continuous-time stochastic Wiener process (often called standard Brownian motion process)}
\nomenclature[Df3]{$\xi(t)$}{Gaussian white noise process, completely characterized by the two moments $\E{\xi(t)\xi(t')}=\delta(t-t')$ and $\E{\xi(t)}=0$; typically found in Stratonovich SDEs}
\nomenclature[Df4]{$\delta(t)$}{Dirac delta function}
\nomenclature[Df5]{$\delta_{ij}$}{Kronecker delta}


\nomenclature[Dg1]{$\dt$}{Deterministic time increment}
\nomenclature[Dg2]{$\dN(t)$}{Stochastic point-process increment. Defined by $\dN^2 = \dN$ and  $E\left[\dN\right] = \lambda\left(X\right) \dt$, where $\lambda\left(X\right)$ is a positive function of the random variable X;  typically found in It\^{o} SDEs}

\nomenclature[Dg3]{$\dW(t)$}{Stochastic Wiener increment, which satisfies $\dW(t)^2 = \dt$ and $\E{dW(t)}=0$; typically found in It\^{o} SDEs}
\nomenclature[Dg4]{$\dZ(t)$}{Complex Wiener increment, defined by $\dZ(t) \equiv \left( \dW_x(t)+ i \dW_y(t) \right)/\sqrt{2} $, which satisfies $\dZ(t)^* \dZ(t) = \dt$ and $\dZ(t)^2 = 0$; typically found in It\^{o} SDEs}

\nomenclature[Di]{$\mathbb{I}\int, \mathbb{S}\int  $}{Integral in the It\^{o} and Stratonovich sense, respectively; in later chapters, we relax the blackboard notation to avoid undue notational complexity}



% - - - - - - - - - - - - - - - - - - - - - - - - - - - - - - - -
%%% Quantum stochastic differential equations 


% - - - - - - - - - - - - - - - - - - - - - - - - - - - - - - - -
%%% Quantum trajectory theory 

\nomenclature[Fa1]{$\ket{\Psi}$}{Pure quantum state of the system \textit{and} environment (in Schr\"{o}dinger picture)}
\nomenclature[Fa2]{$\ket{\psi}$}{Unconditioned pure quantum state of the system}
\nomenclature[Fa3]{$\rho$}{Unconditioned density matrix of the system}
\nomenclature[Fa6]{$\ket{\psi_r}$}{Pure quantum state of system conditioned on the measurement result $r$}
\nomenclature[Fa7]{$\rho_r$}{Density matrix of system conditioned on the measurement result $r$}
\nomenclature[Fa8]{$\ket{\alpha}$}{Coherent state at complex displacement $\alpha$ from the vacuum}
\nomenclature[Fa9]{$\ket{\beta}$}{Coherent state at complex displacement $\beta$ from the vacuum}



\nomenclature[Fc,1]{$\hat{I}$}{Identity operator}
\nomenclature[Fc,4]{$\hat{H}$}{Hamiltonian operator}
\nomenclature[Fc,5]{$\hat{H}_S$}{System Hamiltonian operator}
\nomenclature[Fc,6]{$\hat{H}_E$}{Environment Hamiltonian operator}
\nomenclature[Fc,7]{$\hat{H}_{SE}$}{Hamiltonian operator corresponding to the interaction between the system and environment}
\nomenclature[Fc,9]{$\hat{U}$}{Unitary operator}
\nomenclature[Fc,~a1]{$\hat{M}_r$}{Kraus map operator corresponding to measurement result $r$}

\nomenclature[Fd1]{$\hat \sigma_{x,y,z}$}{Pauli $X$, $Y$, $Z$ operators}
\nomenclature[Fd2]{$\hat a$,$\hat a^\dag$}{Creation and annihilation operators}
\nomenclature[Fd3]{$\hat c$}{Arbitrary system operator coupled to the environment, e.g., $\hat{c} = \sqrt{\Gamma} \hat{a}$, where $\Gamma$ is the coupling rate; also, arbitrary system operator subjected to monitoring}

\nomenclature[Fe1]{$\mathcal{L}$}{Liouvillian superoperator}
\nomenclature[Fe2]{$\mathcal{D}\left[\hat{c}\right]$}{Lindblad superoperator $\mathcal{D}\left[\hat{c}\right]\rho
	\equiv
	\hat{c}^\dagger\rho\hat{c}-\frac{1}{2}[\hat{c}^{\dagger}\hat{c},\rho]_{+}$}
\nomenclature[Fe3]{$\mathcal{G}\left[\hat{c}\right]$}{Photodetection superoperator 
	$\mathcal{G}\left[\hat{c}\right]\rho \equiv \frac{\hat{c}\rho\hat{c}^{\dagger}}{\Tr{\hat{c}\rho\hat{c}^{\dagger}}}-\rho$}
\nomenclature[Fe4]{$\mathcal{H}\left[\hat{c}\right]$}{Dyne detection superoperator 
$	\mathcal{H}\left[\hat{c}\right]\rho\equiv\hat{c}\rho+\rho\hat{c}^{\dagger}-\Tr{\hat{c}\rho+\rho\hat{c}^{\dagger}}\rho$}


\nomenclature[Fg1]{$r$}{Outcome resulting from the measurement}
\nomenclature[Fg2]{$N(t)$}{Total number of photodetections up to time $t$}
\nomenclature[Fg2b]{$I(t)$}{Photocurrent measurement record}
\nomenclature[Fg2c]{$J(t)$}{Dyne detection measurement record}
%\nomenclature[Fg4]{$J(t)$}{Alternative Complex heterodyne measurement record  $Z_\mathrm{rec}(t)$, alternatively }


\nomenclature[Fh]{$\eta$}{Quantum measurement efficiency}
\nomenclature[Fh6]{$T$}{Temperature}
\nomenclature[Fh7]{$n_\mathrm{th}$}{Thermal photon number}



% - - - - - - - - - - - - - - - - - - - - - - - - - - - - - - - -
%%% Quantum jumps in the three-level atom  - F
% see table
\nomenclature[Hb1]{$\G$}{Ground state of the three-level atom}
\nomenclature[Hb2]{$\B$}{Bright state of the three-level atom}
\nomenclature[Hb3]{$\D$}{Dark state of the three-level atom}

\nomenclature[Hd1a]{$\omega$}{Resonance frequency}
\nomenclature[Hd1b]{$\omega_\mathrm{C}$}{Resonance frequency of readout cavity conditioned on the atom state $\G$}
\nomenclature[Hd1c]{$\omega_\mathrm{BG}$}{Bare resonance frequency of the $\G$ to $\B$ transition}
\nomenclature[Hd1d]{$\omega_\mathrm{DG}$}{Bare resonance frequency of the $\G$ to $\D$ transition}

\nomenclature[Hd2a]{$\Omega$}{Rabi drive amplitude}
\nomenclature[Hd2b]{$\Omega_{\mathrm{BG}}$}{Rabi drive (or drives $\Omega_{\mathrm{B0}}$ and $\Omega_{\mathrm{B1}}$) between $\B$ and $\G$}
\nomenclature[Hd2c]{$\Omega_{\mathrm{B0}}$}{First Rabi drive applied to the BG transition, at $\omega_{\mathrm{BG}}$}
\nomenclature[Hd2c]{$\Omega_{\mathrm{B1}}$}{Second Rabi drive applied to the BG transition, at  $\omega_{\mathrm{BG}} - \Delta_\mathrm{B1}$}
\nomenclature[Hd2c]{$\Omega_{\mathrm{DG}}$}{Rabi drive applied to the DG transition, at $\omega_{\mathrm{DG}} - \Delta_{\mathrm{DG}}$}

\nomenclature[Hd3]{$\Delta_{\mathrm{DG}}$}{Detuning of Rabi drive $\Omega_{\mathrm{DG}}$ from the bare DG transition frequency}
\nomenclature[Hd3]{$\Delta_{\mathrm{B1}}$}{Detuning of Rabi drive $\Omega_{\mathrm{B1}}$ from the bare BG transition frequency}

\nomenclature[Hd3]{$\deltatcatch$}{Catch signal duration, corresponding to the time of the complete absence of clicks, following the last click; The duration $\deltatcatch$ is divided in two phases, one lasting $\Delta t_{\mathrm{on}}$ and the other lasting $\Delta t_{\mathrm{off}}$ }
\nomenclature[Hd3b]{$\Delta t_{\mathrm{on}}$}{Duration of the first phase of catch protocol, during which all control drives are on}
\nomenclature[Hd3d]{$\Delta t_{\mathrm{off}}$}{Duration of the second phase of catch protocol, during which $\Omega_{\mathrm{DG}}$ is turned off}
\nomenclature[Hd3f]{$\Delta t_{\mathrm{mid}}$}{Time of mid-flight, in the complete presence of $\Omega_{\mathrm{DG}}$}
\nomenclature[Hd3g]{$\Delta t_{\mathrm{mid}}'$}{Time of mid-flight, in the absence of $\Omega_{\mathrm{DG}}$}


%\nomenclature[Hd4a]{$\chi$}{Cross-Kerr coupling frequency (dispersive shift)}
\nomenclature[Hd4b]{$\chi_\mathrm{B}, \chi_\mathrm{D}$}{Cross-Kerr coupling (dispersive shift) frequency between $\B$ or $\D$ and readout cavity, respectively}
%\nomenclature[Hd4b]{$\chi_\mathrm{B}$}{Cross-Kerr coupling frequency between $\B$ and readout cavity}
%\nomenclature[Hd4c]{$\chi_\mathrm{D}$}{Cross-Kerr coupling frequency between $\D$ and readout cavity}
\nomenclature[Hd4d]{$\chi_\mathrm{DB}$}{Cross-Kerr coupling frequency between the bright and dark transmon}

\nomenclature[Hd5]{$\alpha$}{Anharmonicity of a transmon qubit; also, occasionally used as arbitrary complex prefactor of coherent state}
\nomenclature[Hd5b]{$\alpha_\mathrm{B}, \alpha_\mathrm{D}$}{Anharmonicity of the bright and dark transmon, respectively }

\nomenclature[Hd7]{$\kappa$}{Energy decay rate of readout cavity}
\nomenclature[Hd8]{$\Gamma$}{Effective measurement rate of $\B$}
\nomenclature[Hd9]{$\bar{n}$}{Number of photons in readout cavity when driven resonantly}

\nomenclature[Hf1]{$Q$}{Quality factor}
\nomenclature[Hf2]{$S_{11}$}{Reflection scattering parameter}
\nomenclature[Hf3]{$S_{21}$}{Transmission scattering parameter}

\nomenclature[Hh3]{$I_\mathrm{rec},Q_\mathrm{rec}$}{Demodulated in-quadrature and out-of-quadrature measurement outcome of the heterodyne detection, respectively}

\nomenclature[Hf6]{$R_i(\theta)$}{Rotation around the $i$ axis by angle $\theta$}
\nomenclature[Hf7]{$\theta_{\mathrm{I}}$}{Rotation angle of intervention pulse in the reverse protocol}
\nomenclature[Hf7]{$\varphi_{\mathrm{I}}$}{Angle defining the axis, X', of the intervention pulse in the reverse protocol}
\nomenclature[Hf8]{$P_{\mathrm{G}},P_{\mathrm{D}}$}{Population in the ground and dark state, respectively}




\nomenclature[Hh2]{$T_\mathrm{int}$}{Integration time}
\nomenclature[Hh5]{$T_1$}{Energy relaxation time}
\nomenclature[Hh6]{$T_\mathrm{2R}$}{Ramsey coherence time}
\nomenclature[Hh7]{$T_\mathrm{2E}$}{Hahn echo coherence time}

\nomenclature[Hi1]{$X_\mathrm{DG},Y_\mathrm{DG},Z_\mathrm{DG}$}{Bloch vector components corresponding to the DG manifold}

% - - - - - - - - - - - - - - - - - - - - - - - - - - - - - - - -
%%% Quantum circuit design
% see table

\nomenclature[Ja]{$C$}{Capacitance}
\nomenclature[Ja]{$L$}{Inductance}
\nomenclature[Jb]{$L_j$}{Linear inductance of Josephson device $j$}
\nomenclature[Jb1]{$E_j$}{Energy scale of Josephson device $j$}
\nomenclature[Jb2]{$E_C$}{Transmon charging energy}
% L and C matrix, phi ZPF 

\nomenclature[Jcb1]{$p_{mj}$}{Energy participation ratio of Josephson device $j$ in LJC mode $m$}
\nomenclature[Jcb2]{$\phi_{mj}$}{Reduced zero-point fluctuation of Josephson device $j$ in LJC mode $m$}

\nomenclature[Jd1]{$\hat H$}{Hamiltonian of complete Josephson system ($\hat H = \hat  H_\mathrm{lin} + \hat H_\mathrm{nl}$)}
\nomenclature[Jd2]{$\hat H_\mathrm{lin}$}{Linearized $\hat H$ about operating point}
\nomenclature[Jd3]{$\hat H_\mathrm{nl}$}{Purely non-linear terms in $\hat H$}

\nomenclature[Jf]{$\hat c$,$\hat c^\dag$}{Readout cavity creation and annihilation operators}
\nomenclature[Jf]{$\hat b$,$\hat b^\dag$}{Bright transmon creation and annihilation operators}
\nomenclature[Jf]{$\hat d$,$\hat d^\dag$}{Dark transmon creation and annihilation operators}

% E, H field, classical energies etc.. ? 


\printnomenclature[2cm]
