
\chapter*{Overview\label{chap:Preamble}}

\addcontentsline{toc}{chapter}{Overview}

\emph{\begin{changemargin}{0.6cm}{0.6cm}  
Can there be, despite the indeterminism of quantum physics, a possibility
to know if a quantum jump is about to occur or not?\end{changemargin}
}

Chapter~\ref{chap:Introduction-and-overview} opens by introducing
the notion of a quantum jump between discrete energy levels of a quantum
system, a theoretical idea introduced by Bohr in 1913 \citep{Bohr1913}
--- yet, one whose existence was experimentally observed only seven
decades later \citep{Nagourney1986,Sauter1986,Bergquist1986}, in
a single atomic three-level system. Section~\ref{sec:Principle-of-the}
outlines our proposal to map out the dynamics of a quantum jump from
the ground, $\G$, to an excited, $\D$, state of a three-level superconducting
system. We propose a protocol to catch quantum jumps mid-flight and,
further, to reverse them prior to their completion. The proposal critically
hinges on achieving near unit-measurement efficiency, as discussed,
and experimentally demonstrated, in Sec.~\ref{sec:Unconditioned-monitoring-of}.
Building on this, the catch and reverse experimental protocols and
measurement results are presented in Sections~\ref{sec:Catching-the-quantum}
and~\ref{sec:Reversing-the-quantum}. These results directly demonstrate
that the answer to the above-posed question can indeed be in the affirmative.
Section~\ref{sec:Reversing-the-quantum} summarizes the experimental
results that demonstrate the deterministic prevention of the completion
of jumps; this experiment thereby precludes quantum jumps from occurring
altogether. A control experiment in which the feedback intervention
does not exploit the deterministic character of the completed jumps
is presented. Before proceeding to the remainder of the thesis, Section~\ref{sec:Discussion-of-main-results}
provides a brief discussion of the main results and their implications
for the hundred-year-long debate on the nature and reality of quantum
jumps. The section concludes by providing an outlook for the implications
of the results for future experiments. The remaining chapters, whose
individually aim is described in the following paragraphs, provide
further support to the main conclusions presented in Chapter~\ref{chap:Introduction-and-overview}
and devoted special attention to explicating the theory and experimental
methodology of the work.

Chapter~\ref{chap:Quantum-Trajectory-Theory} develops the essential
background needed to gain access to the core ideas and results of
quantum measurement theory and its formulation, which lead to the
catch and reverse theoretical prediction and modeling of the experiment.
The basic notions of the formalism are introduced in view of specific
examples. Building on this background, Chapter~\ref{chap:theoretical-description-jumps}
develops the quantum trajectory description of the quantum jumps observed
in the three-level atom. The basic ideas as well as the rigorous,
quantitative description of the continuous, coherent, and deterministic
evolution of a completed quantum jump is presented. Finally, the realistic
model of the experiment including known imperfections is developed.
 Chapter~\ref{chap:Experimental-Methods} details the experimental
methods, including our approach to the design of the superconducting
quantum devices developed in \citet{Minev2018-EPR}. Section~\ref{subsec:Energy-participation-ratio}
provides a nutshell introduction to this approach, referred to as
the energy-participation-ratio (EPR) approach and used to design and
optimize both the dissipative and Hamiltonian parameters of our circuit-quantum-electrodynamic
(cQED) systems.

Chapter~\ref{chap:Experimental-results} presents the results of
control experiments that further support the conclusions reached in
Chapter~\ref{chap:Introduction-and-overview}. The comparison between
the experimental results and the predictions of the quantum trajectory
theory developed in Chapter~\ref{chap:theoretical-description-jumps}
is provided in Sec.~\ref{subsec:Comparison-between-theory}. Chapter~\ref{chap:Conclusion-and-perspective}
summarizes the results of this dissertation  and discusses future
research directions. 

