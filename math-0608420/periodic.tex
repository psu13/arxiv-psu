


%~~########################################################################
%############## Baez's diagram macros #####################################

\newcommand{\cxymatrix}[1]{\vcenter{\xymatrix{#1}}}
\newcommand{\Center}[1]{\begin{matrix}#1\end{matrix}}
\input xy \xyoption{all} \xyoption{poly} % use Xy-pic
\labelmargin-{1.2pt}  % bring the labels closer

\newcommand{\thetagraph}[3]  % 2 vertices 3 edges
{ \xymatrix{ *{\bullet} \ar@{-} @/^1.5pc/ [r] ^{#1} \ar@{-}
@/_1.5pc/ [r] _{#3} \ar@{-} [r]^{#2} & *{\bullet}
\\}
}

\newcommand{\fourtheta}[4]  % 2 vertices 4 edges
{ \cxymatrix{ *{\bullet} \ar@{-} @/^1.5pc/ [r] ^{#1} \ar@{-}
@/_1.5pc/ [r] _{#4} \ar@{-} @/^/ [r]^{#2} \ar@{-} @/_/ [r]_{#3} &
*{\bullet}
\\}
}

%tet net with labelled edges
\newcommand{\TetJ}[6]{
\def\lab{\ifcase\xypolynode\or #1 \or #2 \or #3 \fi}
\begin{xy}
\xygraph{!{<3.2pc,0pc>:}
 *{\bullet}
 !P3"A"{~><{@{{}{-}*{\bullet}}} ~>>{_{\lab}}}
 "A0" -@-_{#4} "A1"
 "A0" -@-_{#5} "A2"
 "A0" -@-^{#6} "A3"
}
\end{xy}
}



% tet net with unlabelled edges
\newcommand{\Tet}{
\begin{xy}
\xygraph{!{<3.2pc,0pc>:}
 *{\bullet}
 !P3"A"{~><{@{{}{-}*{\bullet}}}}
 "A0" -@- "A1"
 "A0" -@- "A2"
 "A0" -@- "A3"
}
\end{xy}
}




% 10j-symbols with edges labelled j_{kl} for k,l = 1,2,3,4,5
\newcommand{\TenJ}{
\def\lab{\ifcase\xypolynode\or 12 \or 23 \or 34 \or 45 \or 15 \fi}
\begin{xy}
\xygraph{!{<4pc,0pc>:}
 !P5"A"{~><{@{{}{-}*{\bullet}}} ~>>{_{j_{\lab}}}}
 "A1" -@-_{j_{13}} "A3"
 "A2" -@-_{j_{24}} "A4"
 "A3" -@-_{j_{35}} "A5"
 "A4" -@-_{j_{14}} "A1"
 "A5" -@-_{j_{25}} "A2"
}
\end{xy}
}


% 10j-symbols with edges labelled a_{kl} for k,l = 1,2,3,4,5
\newcommand{\TenA}{
\def\lab{\ifcase\xypolynode\or 12 \or 23 \or 34 \or 45 \or 15 \fi}
\begin{xy}
\xygraph{!{<4pc,0pc>:}
 !P5"A"{~><{@{{}{-}*{\bullet}}} ~>>{_{a_{\lab}}}}
 "A1" -@-_{a_{13}} "A3"
 "A2" -@-_{a_{24}} "A4"
 "A3" -@-_{a_{35}} "A5"
 "A4" -@-_{a_{14}} "A1"
 "A5" -@-_{a_{25}} "A2"
}
\end{xy}
}


% 10j-symbols with unlabelled edges
\newcommand{\Ten}{%
\def\lab{\ifcase\xypolynode\or 1,2 \or 2,3 \or 3,4 \or 4,5 \or 5,1 \fi}
\begin{xy}
\xygraph{!{<2pc,0pc>:}
 !P5"A"{~><{@{{}{-}*{\bullet}}} ~>>{_{}}}
 "A1" -@-_{} "A3"
 "A2" -@-_{} "A4"
 "A3" -@-_{} "A5"
 "A4" -@-_{} "A1"
 "A5" -@-_{} "A2"
}
\end{xy}
}

% 10j-symbols with edges labelled $\smult$
\newcommand{\TenL}{
\def\lab{\ifcase\xypolynode\or 1,2 \or 2,3 \or 3,4 \or 4,5 \or 5,1 \fi}
\begin{xy}
\xygraph{!{<4pc,0pc>:}
 !P5"A"{~><{@{{}{-}*{\bullet}}} ~>>{_{\smult}}}
 "A1" -@-_{\smult} "A3"
 "A2" -@-_{\smult} "A4"
 "A3" -@-_{\smult} "A5"
 "A4" -@-_{\smult} "A1"
 "A5" -@-_{\smult} "A2"
}
\end{xy}
}

% little 10j-symbols with edges labelled $\smult$
\newcommand{\tenl}{
\def\lab{\ifcase\xypolynode\or 1,2 \or 2,3 \or 3,4 \or 4,5 \or 5,1 \fi}
\begin{xy}
\xygraph{!{<2pc,0pc>:}
 !P5"A"{~><{@{{}{-}*{\bullet}}} ~>>{_{\smult}}}
 "A1" -@-_{\smult} "A3"
 "A2" -@-_{\smult} "A4"
 "A3" -@-_{\smult} "A5"
 "A4" -@-_{\smult} "A1"
 "A5" -@-_{\smult} "A2"
}
\end{xy}
}



\newcommand{\periodictable}{
\begin{center}
 { \textbf{
\begin{tabular}{|c|c|c|c|}  \hline
        & $\mathbf{\mathit n = 0}$ & $\mathbf{\mathit n = 1}$ &
$\mathbf{\mathit n = 2}$\\ \hline $\mathbf{\mathit k = 0}$ & sets
& categories & 2-categories     \\     \hline
$\mathbf{\mathit k = 1}$  & monoids   & monoidal   & monoidal         \\
        &           & categories & 2-categories     \\     \hline
$\mathbf{\mathit k = 2}$  &commutative& braided    & braided          \\
        & monoids   & monoidal   & monoidal         \\
        &           & categories & 2-categories     \\     \hline
$\mathbf{\mathit k = 3}$  &`'         & symmetric  & sylleptic \\
        &           & monoidal   & monoidal         \\
        &           & categories & 2-categories     \\     \hline
$\mathbf{\mathit k = 4}$  &`'         & `'         & symmetric \\
        &           &            & monoidal         \\
        &           &            & 2-categories     \\     \hline
$\mathbf{\mathit k = 5}$  &`'         &`'          & `'               \\
        &           &            &                  \\
        &           &            &                  \\     \hline
$\mathbf{\mathit k = 6}$  &`'         &`'          & `'               \\
        &           &            &                  \\
        &           &            &                  \\     \hline
\end{tabular}}}
\vskip 0.5em
\end{center}}


\newcommand{\extendedperiodictable}{
\begin{center}
 { \textbf{
\begin{tabular}{|c|c|c|c|c|c|}  \hline
        & $\mathbf{\mathit n = -2}$ & $\mathbf{\mathit n = -1}$ &
         $\mathbf{\mathit n = 0}$ & $\mathbf{\mathit n = 1}$ &
$\mathbf{\mathit n = 2}$\\ \hline 
$\mathbf{\mathit k = 0}$ &?  & ?
& sets & categories & 2-categories     \\     \hline
$\mathbf{\mathit k = 1}$  
& `' & ?
& monoids   & monoidal   & monoidal         \\
& &       &           & categories & 2-categories     \\     \hline
$\mathbf{\mathit k = 2}$  
& `'  &  `'
&commutative& braided    & braided          \\
& &        & monoids   & monoidal   & monoidal         \\
& &        &           & categories & 2-categories     \\     \hline
$\mathbf{\mathit k = 3}$  
& `'  &  `'
&`'         & symmetric  & sylleptic \\
& &        &           & monoidal   & monoidal         \\
& &        &           & categories & 2-categories     \\     \hline
$\mathbf{\mathit k = 4}$  
& `'  &  `'
&`'         & `'         & symmetric \\
& &        &           &            & monoidal         \\
& &        &           &            & 2-categories     \\     \hline
\end{tabular}}}
\vskip 0.5em
\end{center}}



\newcommand{\periodictableII}{
{
\[ %
\xy (-75,60)*{k=0}; (-75,30)*{k=1}; (-75,0)*{k=2};
(-75,-30)*{k=3}; (-75,-60)*{k=4}; (-40,75)*{n=0}; (0,75)*{n=1};
(40,75)*{n=2};
(0,0)*{ %^^^^^^^^^^^^^^^^^^^^^^^^^^^^^^^^^^^^^^^^^^^^^^^^
  \xy 0;/r.20pc/: % n=1 k=2
  (25,25)*{};
(-2,12)*{\bullet}="1"+(-2,1)*{ \scriptstyle x};
(1,16)*{\bullet}="2"+(-2,2)*{\scriptstyle x^{\ast}};
(10,14)*{\bullet}="3"+(1,3)*{\scriptstyle x};
(1,-7)*{\bullet}="4"+(-1,3)*{\scriptstyle x}; "3";"4" **\crv{}
\POS?(.3)*{\hole}="J"; ?(0)*\dir{>}; "1";"J" **\crv{(15,-5)};
?(.15)*\dir{>}; "2";"J" **\crv{} ?(.28)*\dir{<};
%*********************BUILDS BOX*****************************************************
(-14,10)*{}="TL"; (14,10)*{}="TR"; (14,-10)*{}="BR";
(-14,-10)*{}="BL"; (-6,20)*{}="xTL"; (22,20)*{}="xTR";
(22,0)*{}="xBR"; (-6,0)*{}="xBL";
    "TL";"TR" **\dir{-};
    "TR";"BR" **\dir{-};
    "BR";"BL" **\dir{-};
    "BL";"TL" **\dir{-};
    "xTL";"xTR" **\dir{-};
    "xTR";"xBR" **\dir{-};
    "xBR";"xBL" **\dir{.};
    "TL";"xTL" **\dir{-};
    "TR";"xTR" **\dir{-};
    "BL";"xBL" **\dir{.};
    "BR";"xBR" **\dir{-};
    "xTL";"xBL" **\dir{.};
%*********************END BUILD BOX***************************************************
\endxy
   };
   (0,30)*{
   %^^^^^^^^^^^^^^^^^^^^^^^^^^^^^^^^^^^^^^^^^^^^^^^^^^^^^^
    \xy 0;/r.20pc/: % n=1 k=1
(-7,10)*{\bullet}="1"+(0,3)*{x};
(-1,10)*{\bullet}="2"+(0,3)*{x^{\ast}};
(6,10)*{\bullet}="3"+(0,3)*{x}; (1,-10)*{\bullet}="4"+(0,-3)*{x};
 "1"; "2" **\crv{(-7,2) & (-1,2)}; ?(.2)*\dir{>}; ?(.9)*\dir{>};
 "3"; "4" **\crv{(9,4) & (-5,-1)}; ?(.5)*\dir{>};
(-10,10)*{}; (10,10)*{} **\dir{-}; (10,-10)*{}; (10,10)*{}
**\dir{-}; (-10,-10)*{}; (-10,10)*{} **\dir{-}; (-10,-10)*{};
(10,-10)*{} **\dir{-};
\endxy};
(40,0)*{
%^^^^^^^^^^^^^^^^^^^^^^^^^^^^^^^^^^^^^^^^^^^^^^^^^^^^^^
\xy 0;/r.20pc/:% n=2 k=2
(8.25,-1.25)*\ellipse(2,.65){-}; (6,18)*{}="b";
  (1,16)*{}="a";
 \vunder~{(1,17.5)}{(6,18)}{(3.5,15.5)}{(6,16)};
 (1,17.5)*{}; (6,16)*{} **\crv{(-5,15)& (6,13) }; \POS?(.75)*{\hole}="J1";
 "J1";(6,18)*{} **\crv{(10,14)&(11,18)};
  (6,-2)*{}="b";
  (1,-4)*{}="a";
 \vunder~{(1,-2.5)}{(6,-2)}{(3.5,-4.5)}{(6,-4)};
 (1,-2.5)*{}; (6,-4)*{} **\crv{(-5,-5)& (6,-7) }; \POS?(.75)*{\hole}="J1";
 "J1";(6,-2)*{} **\crv{(10,-6)&(11,-2)};
 (-1,16)*{}="TL";
 (9,16)*{}="TR";
 (-1,-4)*{}="BL";
 (9.25,-3.5)*{}="BR";
 (18.5,-2.5)*{}="BRR";
 (14.5,-2.5)*{}="BRR2";
 "TL";"BL" **\crv{(-2,13) & (1,6)};
 "TR";"BRR" **\crv{(9,12) & (10,6)};
 "BR";"BRR2" **\crv{(7,12) & (12.5,0)};
%*********************BUILDS BOX*****************************************************
(-14,10)*{}="TL"; (14,10)*{}="TR"; (14,-10)*{}="BR";
(-14,-10)*{}="BL"; (-6,20)*{}="xTL"; (22,20)*{}="xTR";
(22,0)*{}="xBR"; (-6,0)*{}="xBL";
    "TL";"TR" **\dir{-};
    "TR";"BR" **\dir{-};
    "BR";"BL" **\dir{-};
    "BL";"TL" **\dir{-};
    "xTL";"xTR" **\dir{-};
    "xTR";"xBR" **\dir{-};
    "xBR";"xBL" **\dir{.};
    "TL";"xTL" **\dir{-};
    "TR";"xTR" **\dir{-};
    "BL";"xBL" **\dir{.};
    "BR";"xBR" **\dir{-};
    "xTL";"xBL" **\dir{.};
%*********************END BUILD BOX***************************************************
(5,22)*{\textbf{4d}}; (25,25)*{};
\endxy};
(-40,0)*{ %^^^^^^^^^^^^^^^^^^^^^^^^^^^^^^^^^^^^^^^^
\xy  0;/r.20pc/:% n=0 k=2
(-5,5)*{\bullet}+(1,3)*{x}; (5,4)*{\bullet}+(1,3)*{x^{\ast}};
(1,-7)*{\bullet}+(1,3)*{x}; (-10,10)*{}; (10,10)*{} **\dir{-};
(10,-10)*{}; (10,10)*{} **\dir{-}; (-10,-10)*{}; (-10,10)*{}
**\dir{-}; (-10,-10)*{}; (10,-10)*{} **\dir{-};
\endxy
};(0,-30)*{ %^^^^^^^^^^^^^^^^^^^^^^^^^^^^^^^^^^^^^^^^^^^^^^^^^
\xy 0;/r.20pc/: % n=1 k=3
(-4,14)*{\bullet}="1"+(,2)*{ \scriptstyle x};
(0,14)*{\bullet}="2"+(,2)*{\scriptstyle x^{\ast}};
(10,14)*{\bullet}="3"+(1,3)*{\scriptstyle x};
(1,-7)*{\bullet}="4"+(-1,3)*{\scriptstyle x}; "3";"4"
**\crv{(10,4) & (-1,-6)}; ?(.6)*\dir{>};
%*********************BUILDS BOX*****************************************************
(-14,10)*{}="TL"; (14,10)*{}="TR"; (14,-10)*{}="BR";
(-14,-10)*{}="BL"; (-6,20)*{}="xTL"; (22,20)*{}="xTR";
(22,0)*{}="xBR"; (-6,0)*{}="xBL";
    "TL";"TR" **\dir{-}; \POS?(.35)*{\hole}="J"; \POS?(.5)*{\hole}="J1";
    "TR";"BR" **\dir{-};
    "BR";"BL" **\dir{-};
    "BL";"TL" **\dir{-};
    "xTL";"xTR" **\dir{-};
    "xTR";"xBR" **\dir{-};
    "xBR";"xBL" **\dir{.};
    "TL";"xTL" **\dir{-};
    "TR";"xTR" **\dir{-};
    "BL";"xBL" **\dir{.};
    "BR";"xBR" **\dir{-};
    "xTL";"xBL" **\dir{.};
%*********************END BUILD BOX***************************************************
 "1";"J" **\crv{}?(.9)*\dir{>};
 "2";"J1" **\crv{} ?(.3)*\dir{<};
 "J";"J1" **\crv{(-4,5) & (0,5)};
 (5,22)*{\textbf{4d}};
 (25,25)*{};
\endxy
};
 (0,-60)*{%^^^^^^^^^^^^^^^^^^^^^^^^^^^^^^^^^^^^^^^^^^^^^^^^^^^^
   \xy 0;/r.20pc/: % n=1 k=4
(-4,14)*{\bullet}="1"+(,2)*{ \scriptstyle x};
(0,14)*{\bullet}="2"+(,2)*{\scriptstyle x^{\ast}};
(10,14)*{\bullet}="3"+(1,3)*{\scriptstyle x};
(1,-7)*{\bullet}="4"+(-1,3)*{\scriptstyle x}; "3";"4"
**\crv{(10,4) & (-1,-6)}; ?(.6)*\dir{>};
%*********************BUILDS BOX*****************************************************
(-14,10)*{}="TL"; (14,10)*{}="TR"; (14,-10)*{}="BR";
(-14,-10)*{}="BL"; (-6,20)*{}="xTL"; (22,20)*{}="xTR";
(22,0)*{}="xBR"; (-6,0)*{}="xBL";
    "TL";"TR" **\dir{-}; \POS?(.35)*{\hole}="J"; \POS?(.5)*{\hole}="J1";
    "TR";"BR" **\dir{-};
    "BR";"BL" **\dir{-};
    "BL";"TL" **\dir{-};
    "xTL";"xTR" **\dir{-};
    "xTR";"xBR" **\dir{-};
    "xBR";"xBL" **\dir{.};
    "TL";"xTL" **\dir{-};
    "TR";"xTR" **\dir{-};
    "BL";"xBL" **\dir{.};
    "BR";"xBR" **\dir{-};
    "xTL";"xBL" **\dir{.};
%*********************END BUILD BOX***************************************************
 "1";"J" **\crv{}?(.9)*\dir{>};
 "2";"J1" **\crv{} ?(.3)*\dir{<};
 "J";"J1" **\crv{(-4,5) & (0,5)};
 (5,22)*{\textbf{5d}};
 (25,25)*{};
\endxy
};
(-40,30)*{ %^^^^^^^^^^^^^^^^^^^^^^^^^^^^^^^^^^^^^^^^^^^^^^^^^^^^^^
\xy   % n=0 k=1
(0,0)*{\bullet}+(1,3)*{x^{\ast}}; (5,0)*{\bullet}+(1,3)*{x};
(-5,0)*{\bullet}+(1,3)*{x}; (-10,0)*{}; (10,0)*{} **\dir{-};
\endxy
};
(-40,60)*{ %^^^^^^^^^^^^^^^^^^^^^^^^^^^^^^^^^^^^^^^^^^^^^^^^^^^^^^
% n=0 k=0
\xy (0,0)*{\bullet}+(2,3)*{x^{\ast}};
\endxy
};
(-40,-30)*{ %^^^^^^^^^^^^^^^^^^^^^^^^^^^^^^^^^^^^^^^^^^^^^^^^^^^^^^
  \xy 0;/r.20pc/: % n=0 k=3
(-2.5,4)*{\bullet}+(1,2)*{\scriptstyle x}; (5,7)*{
\bullet}+(3,1)*{\scriptstyle x^{\ast}};
(6,0)*{\bullet}+(1,2)*{\scriptstyle x};
%*********************BUILDS BOX*****************************************************
(-14,10)*{}="TL"; (14,10)*{}="TR"; (14,-10)*{}="BR";
(-14,-10)*{}="BL"; (-6,20)*{}="xTL"; (22,20)*{}="xTR";
(22,0)*{}="xBR"; (-6,0)*{}="xBL";
    "TL";"TR" **\dir{-};
    "TR";"BR" **\dir{-};
    "BR";"BL" **\dir{-};
    "BL";"TL" **\dir{-};
    "xTL";"xTR" **\dir{-};
    "xTR";"xBR" **\dir{-};
    "xBR";"xBL" **\dir{.};
    "TL";"xTL" **\dir{-};
    "TR";"xTR" **\dir{-};
    "BL";"xBL" **\dir{.};
    "BR";"xBR" **\dir{-};
    "xTL";"xBL" **\dir{.};
%*********************END BUILD BOX***************************************************
(25,25)*{};
\endxy
}; (-40,-60)*{
  \xy 0;/r.20pc/: % n=0 k=3
(-2.5,4)*{\bullet}+(1,2)*{\scriptstyle x}; (5,7)*{
\bullet}+(3,1)*{\scriptstyle x^{\ast}};
(6,0)*{\bullet}+(1,2)*{\scriptstyle x};
%*********************BUILDS BOX*****************************************************
(-14,10)*{}="TL"; (14,10)*{}="TR"; (14,-10)*{}="BR";
(-14,-10)*{}="BL"; (-6,20)*{}="xTL"; (22,20)*{}="xTR";
(22,0)*{}="xBR"; (-6,0)*{}="xBL";
    "TL";"TR" **\dir{-};
    "TR";"BR" **\dir{-};
    "BR";"BL" **\dir{-};
    "BL";"TL" **\dir{-};
    "xTL";"xTR" **\dir{-};
    "xTR";"xBR" **\dir{-};
    "xBR";"xBL" **\dir{.};
    "TL";"xTL" **\dir{-};
    "TR";"xTR" **\dir{-};
    "BL";"xBL" **\dir{.};
    "BR";"xBR" **\dir{-};
    "xTL";"xBL" **\dir{.};
%*********************END BUILD BOX***************************************************
(25,25)*{}; (5,22)*{\textbf{4d}};
\endxy
}; (0,60)*{
 \xy  0;/r.20pc/:  %k=o n=1
(0,10)*{\bullet}="a"+(2.5,1)*{x};
(0,-10)*{\bullet}="b"+(2.5,1)*{x}; "a";"b" **\dir{-};
?(.45)*\dir{>};
\endxy
}; (40,60)*{
 \xy 0;/r.20pc/: % n=2 k=0
(-10,10)*{\bullet}="TL"+(-1,3)*{x}; (10,10)*{}="2"="TR"+(1,3)*{x};
(10,-10)*{\bullet}="BR"+(1,-3)*{x};
(-10,-10)*{\bullet}="BL"+(-1,-3)*{x}; "TL";"TR" **\dir{-};
?(.5)*\dir{>}; "BL";"BR" **\dir{-}; ?(.5)*\dir{>}; "TL";"BL"
**\dir{-}; "TR";"BR" **\dir{-}; (0,3)*{}="a"; (0,-3)*{}="a'";
{\ar@{=>} "a";"a'"};
\endxy
}; (40,30)*{
 \xy 0;/r.20pc/: % n=2 k=3
(-9,10)*{\bullet}="1"+(-1,-3)*{x^{\ast}};
(-1,10)*{\bullet}="2"+(-1,-3)*{x};
 (-9,-10)*{\bullet}="b1"+(-1,-3)*{x^{\ast}};
(-1,-10)*{\bullet}="b2"+(-1,-3)*{x};
(13,20)*{\bullet}="4"+(1,3)*{x};
(7,20)*{\bullet}="3"+(1,3)*{x^{\ast}}; (13,0)*{\bullet}="b4";
(7,0)*{\bullet}="b3"+(1,3); "1";"2" **\crv{(-4,15) & (3,15)};
?(.15)*\dir{<}; "3";"4" **\crv{(3,15) & (10,15)}; ?(.85)*\dir{>};
 (-1,-5.25)*{}="M";
"1";"b1"  **\dir{-}; "2";"b2"  **\dir{-}; "b1";"M"  **\dir{-};
?(.76)*\dir{>}; "b3";"M"  **\dir{.}; "b2";"b4"  **\dir{-};
?(.4)*\dir{<}; "4";"b4"  **\dir{-}; "3";"b3"  **\dir{.};
(0,13.25)*{}="z1"; (7,16.15)*{}="z2"; "z1";"z2" **\crv{(2,-3) &
(7,3)};
%*********************BUILDS BOX*****************************************************
(-14,10)*{}="TL"; (14,10)*{}="TR"; (14,-10)*{}="BR";
(-14,-10)*{}="BL"; (-6,20)*{}="xTL"; (22,20)*{}="xTR";
(22,0)*{}="xBR"; (-6,0)*{}="xBL";
    "TL";"TR" **\dir{-};
    "TR";"BR" **\dir{-};
    "BR";"BL" **\dir{-};
    "BL";"TL" **\dir{-};
    "xTL";"xTR" **\dir{-};
    "xTR";"xBR" **\dir{-};
    "xBR";"xBL" **\dir{.};
    "TL";"xTL" **\dir{-};
    "TR";"xTR" **\dir{-};
    "BL";"xBL" **\dir{.};
    "BR";"xBR" **\dir{-};
    "xTL";"xBL" **\dir{.};
%*********************END BUILD BOX***************************************************
(25,25)*{};
\endxy
}; (40,-30)*{
 \xy 0;/r.20pc/:% n=2 k=3
(-1.5,-2)*\ellipse(3,1){-}; (3.5,-2)*\ellipse(3,1){-};
(6,18)*{}="b";
  (1,16)*{}="a";
 \vunder~{(1,17.5)}{(6,18)}{(3.5,15.5)}{(6,16)};
 (1,17.5)*{}; (6,16)*{} **\crv{(-5,15)& (6,13) }; \POS?(.75)*{\hole}="J1";
 "J1";(6,18)*{} **\crv{(10,14)&(11,18)};
 (-1,16)*{}="TL";
 (9,16)*{}="TR";
 (-6,-4)*{}="BLL";
 (0,-4)*{}="BL";
 (10,-4)*{}="BRR";
 (4,-4)*{}="BR";
 (3,5)*{}="C";
 (4.25,6.25)*{}="C2";
 (4.25,15)*{}="C3";
   "TL";"BLL" **\crv{(-2,13) & (-1,6)};
   "TR";"BRR" **\crv{(9,13) & (10,6)};
   "C";"BL" **\crv{};
   "C";"BR" **\crv{(6,10)};
    "C3";"C2" **\dir{.};
%*********************BUILDS BOX*****************************************************
(-14,10)*{}="TL"; (14,10)*{}="TR"; (14,-10)*{}="BR";
(-14,-10)*{}="BL"; (-6,20)*{}="xTL"; (22,20)*{}="xTR";
(22,0)*{}="xBR"; (-6,0)*{}="xBL";
    "TL";"TR" **\dir{-};
    "TR";"BR" **\dir{-};
    "BR";"BL" **\dir{-};
    "BL";"TL" **\dir{-};
    "xTL";"xTR" **\dir{-};
    "xTR";"xBR" **\dir{-};
    "xBR";"xBL" **\dir{.};
    "TL";"xTL" **\dir{-};
    "TR";"xTR" **\dir{-};
    "BL";"xBL" **\dir{.};
    "BR";"xBR" **\dir{-};
    "xTL";"xBL" **\dir{.};
%*********************END BUILD BOX***************************************************
(5,22)*{\textbf{5d}}; (25,25)*{};
\endxy
}; (40,-60)*{
\xy 0;/r.20pc/:% n=2 k=4
(-1.5,-2)*\ellipse(3,1){-}; (3.5,-2)*\ellipse(3,1){-};
(6,18)*{}="b";
  (1,16)*{}="a";
 \vunder~{(1,17.5)}{(6,18)}{(3.5,15.5)}{(6,16)};
 (1,17.5)*{}; (6,16)*{} **\crv{(-5,15)& (6,13) }; \POS?(.75)*{\hole}="J1";
 "J1";(6,18)*{} **\crv{(10,14)&(11,18)};
 (-1,16)*{}="TL";
 (9,16)*{}="TR";
 (-6,-4)*{}="BLL";
 (0,-4)*{}="BL";
 (10,-4)*{}="BRR";
 (4,-4)*{}="BR";
 (3,5)*{}="C";
 (4.25,6.25)*{}="C2";
 (4.25,15)*{}="C3";
   "TL";"BLL" **\crv{(-2,13) & (-1,6)};
   "TR";"BRR" **\crv{(9,13) & (10,6)};
   "C";"BL" **\crv{};
   "C";"BR" **\crv{(6,10)};
    "C3";"C2" **\dir{.};
%*********************BUILDS BOX*****************************************************
(-14,10)*{}="TL"; (14,10)*{}="TR"; (14,-10)*{}="BR";
(-14,-10)*{}="BL"; (-6,20)*{}="xTL"; (22,20)*{}="xTR";
(22,0)*{}="xBR"; (-6,0)*{}="xBL";
    "TL";"TR" **\dir{-};
    "TR";"BR" **\dir{-};
    "BR";"BL" **\dir{-};
    "BL";"TL" **\dir{-};
    "xTL";"xTR" **\dir{-};
    "xTR";"xBR" **\dir{-};
    "xBR";"xBL" **\dir{.};
    "TL";"xTL" **\dir{-};
    "TR";"xTR" **\dir{-};
    "BL";"xBL" **\dir{.};
    "BR";"xBR" **\dir{-};
    "xTL";"xBL" **\dir{.};
%*********************END BUILD BOX***************************************************
(5,22)*{\textbf{6d}}; (25,25)*{};
\endxy
}:
\endxy % END END END END END END END END END END
\]
} }
%############################################################################################




\newcommand{\feynmandiagram}{
  \xy 0;/r.22pc/:      %PICTURE OF A BUNCH OF FEYNMAN DIAGRAMS
 (-3,10)*{}="TL"; (3,10)*{}="TR";
 (0,-2)*{}="B";
 (0,-12)*{}="BB";
    "TL";"B" **\dir{-}?(.5)*\dir{>};
    "TR";"B" **\dir{~};
    "B";"BB" **\dir{-}?(.5)*\dir{>};
 \endxy
                    \; \; \; + \; \;
\xy 0;/r.22pc/:
 (-3,10)*{}="TL"; (3,10)*{}="TR";
 (0,3)*{}="B";
 (0,-12)*{}="BB";
    "TL";"B" **\dir{-}?(.5)*\dir{>};
    "TR";"B"+(0,-.6) **\dir{~};
    "B";"BB" **\dir{-}?(.5)*\dir{>} ?(.24)*\dir{}="1" ?(.84)*\dir{}="2";
    {\ar@/^.35pc/@{~} "1"+(1,0);"2"+(1,0)};
 \endxy
                    \; \; + \; \;
\xy
 (-3,10)*{}="TL"; (4,10)*{}="TR";
 (0,-2)*{}="B";
 (0,-12)*{}="BB";
    "TL";"B" **\dir{-}?(.5)*\dir{>}?(.17)*\dir{}="1";
    "TR";"B" **\dir{~} ?(.5)*\dir{}="mid";
    "B";"BB" **\dir{-}?(.5)*\dir{>} ?(.7)*\dir{}="2";
    {\ar@/^1pc/@{~}|<<<<{\hole} "1"+(1,0);"2"+(1,0)};
 \endxy
                     \; \; + \; \cdots \; + \;
 \xy
 (-3,10)*{}="TL"; (3,10)*{}="TR";
 (0,3)*{}="B";
 (0,-12)*{}="BB";
    "TL";"B" **\dir{-}?(.35)*\dir{>} ?(.75)*\dir{>};
    "TR";"B"+(0,-.6) **\dir{~};
    "B";"BB" **\dir{-}?(.4)*\dir{>}?(.7)*\dir{>} ?(.24)*\dir{}="1" ?(.85)*\dir{}="2";
    {\ar@/^.35pc/@{~} "1"+(.9,0);"2"+(.9,0)};
    %%%%
     {\ar@{~} (-5.2,-2.7);(0,-8.1)};
   {\ar@{~} (-1.9,8);(-5,2.1)};
     (-5,0)*\xycircle(2,2.8){-};
     {\ar@{~} (-3,0);(0,-5)};
 \endxy
    \; \; + \; \cdots
}

\newcommand{\bothzigzags}{
\xy
    (0,-10)*++{}="g";
    (10,0)*{}="mid";
      **\crv{(5,20)}  ?(.83)*\dir{>};
    (6.5,10)*{\scriptstyle i_{H}};
    (20,10)*++{}="f";
     "f";"mid"; **\crv{(15,-20)} ?(0)*\dir{<} ?(.76)*\dir{<};
    (15.5,-10)*{\scriptstyle e_{H}};
\endxy
\quad = \quad
 \xy
  (0,10)*{};(0,-10)*{};
  **\dir{-} ?(.47)*\dir{<}+(3,1)*{\scriptstyle 1_H}
 \endxy
\qquad  \qquad
 \xy
    (0,10)*++{}="g";
    (10,0)*{}="mid";
      **\crv{(5,-20)} ?(.02)*\dir{>} ?(.83)*\dir{>};
    (6,-10)*{\scriptstyle e_{H}};
    (20,-10)*++{}="f";
     "f";"mid"; **\crv{(15,20)}  ?(.78)*\dir{<};
    (15,10)*{\scriptstyle i_{H}};
\endxy
\quad = \quad
 \xy
  (0,10)*{};(0,-10)*{};
  **\dir{-} ?(.53)*\dir{>}+(3,1)*{\scriptstyle 1_H}
 \endxy
}

\newcommand{\orientationproblem}{
 \vcenter{\xy 0;/r.30pc/:
  (0,6)*{\bullet};
 (-10,0)*{}="L";
 (10,0)*{}="R";
 (0,16)*{}="T";
 (0,6)*{}="M";
 (0,-4)*{}="B";
 (-10,12)*{}="TL";
 (10,12)*{}="TR";
    "T";"L" **\dir{.};
    "R";"T" **\dir{.};
    "L";"R" **\dir{.};
    "TL";"M" **\dir{-}?(.5)*\dir{>};
    "TR";"M" **\dir{-}?(.5)*\dir{>};
    "M";"B" **\dir{-}?(.6)*\dir{>};
 \endxy}
\qquad = \qquad
 \vcenter{\xy 0;/r.30pc/:
 (-10,0)*{}="L";
 (10,0)*{}="R";
 (0,16)*{}="T";
 (0,6)*{}="M";
    "L";"T" **\dir{.};
    "R";"T" **\dir{.};
    "L";"R" **\dir{.};
    "T";"M" **\dir{.};
    "R";"M" **\dir{.};
    "L";"M" **\dir{.};
%%%%%%%%%%%%%%
 (0,-4)*{}="B";
 (-10,12)*{}="TL";
 (10,12)*{}="TR";
 (-3.5,8)*{}="tl";
 (3.5,8)*{}="tr";
 (0,2.5)*{}="b";
    "TL";"tl" **\dir{-}?(.5)*\dir{>};
    "TR";"tr" **\dir{-}?(.5)*\dir{>};
    "b";"B" **\dir{-}?(.6)*\dir{>};
    "tl";"tr" **\dir{-}?(.5)*\dir{>};
    "tr";"b" **\dir{-}?(.5)*\dir{>};
    "tl";"b" **\dir{-}?(.6)*\dir{}+(-2,-1)*{?};
 \endxy}
 }

 \newcommand{\trianglemultiplication}{
  \xy
  (0,6)*{\bullet};
 (-10,0)*{}="L";
 (10,0)*{}="R";
 (0,16)*{}="T";
 (0,6)*{}="M";
 (0,-4)*{}="B";
 (-10,12)*{}="TL";
 (10,12)*{}="TR";
    "T";"L" **\dir{.};
    "R";"T" **\dir{.};
    "L";"R" **\dir{.};
    "TL";"M" **\dir{-}?(.5)*\dir{>};
    "TR";"M" **\dir{-}?(.5)*\dir{>};
    "M";"B" **\dir{-}?(.6)*\dir{>};
 \endxy
\qquad \qquad
 \xy
 (0,0)*+{A }="B";
 (0,12)*+{A \tensor A}="T"; {\ar_m "T";"B"};
 \endxy
 }

 \newcommand{\triangleassoc}{
  \xy
 (-24,10)*+{A \tensor A \tensor A}="T";
 (-24,-10)*+{A}="B";
 (-24,0)*+{A \tensor A }="M";
    {\ar_{1 \tensor m} "T";"M"};
    {\ar_{ m} "M";"B"};
 (0,10)*{}="mt";
 (0,-10)*{}="mb";
 (-16,0)*{}="l";
 (16,0)*{}="r";
  (6,0)*{}="xr";
  (-6,0)*{}="xl";
  (-12,10)*{}="xlt";
  (-12,-10)*{}="xlb";
    (12,10)*{}="xrt";
  (12,-10)*{}="xrb";
  (6,0)*{\bullet};
  (-6,0)*{\bullet};
    "mt";"mb" **\dir{.};
    "mb";"l" **\dir{.};
    "mt";"l" **\dir{.};
    "mb";"r" **\dir{.};
    "mt";"r" **\dir{.};
    "xl";"xr" **\dir{-}?(.5)*\dir{<};
    "xl";"xlt" **\dir{-}?(.5)*\dir{<};
    "xl";"xlb" **\dir{-}?(.65)*\dir{>};
    "xr";"xrt" **\dir{-}?(.5)*\dir{<};
    "xr";"xrb" **\dir{-}?(.5)*\dir{<};
 \endxy
 \quad = \quad
 \xy
 (-24,10)*+{A \tensor A \tensor A}="T";
 (-24,-10)*+{A}="B";
 (-24,0)*+{A \tensor A }="M";
    {\ar_{m \tensor 1} "T";"M"};
    {\ar_{ m} "M";"B"};
 (0,10)*{}="mt";
 (0,-10)*{}="mb";
 (-16,0)*{}="l";
 (16,0)*{}="r";
  (0,-4)*{}="xr";
  (0,4)*{}="xl";
  (10,10)*{}="xlt";
  (-10,10)*{}="xlb";
    (10,-10)*{}="xrt";
  (-10,-10)*{}="xrb";
  (0,4)*{\bullet};
  (0,-4)*{\bullet};
    "l";"r" **\dir{.};
    "mb";"l" **\dir{.};
    "mt";"l" **\dir{.};
    "mb";"r" **\dir{.};
    "mt";"r" **\dir{.};
    "xl";"xr" **\dir{-}?(.58)*\dir{>};
    "xl";"xlt" **\dir{-}?(.5)*\dir{<};
    "xl";"xlb" **\dir{-}?(.5)*\dir{<};
    "xr";"xrt" **\dir{-}?(.5)*\dir{<};
    "xr";"xrb" **\dir{-}?(.65)*\dir{>};
 \endxy}

\newcommand{\defnondegenerate}{
\xy (0,11)*{}; (0,0)*\xycircle(2.65,2.65){-}="1_x"; **\dir{-}
?(.5)*\dir{<}; "1_x";(0,-11)*{}; **\dir{-} ?(.5)*\dir{>};
(0,0)*{\sharp};
\endxy
\qquad
 \xy
 (0,9)*+{A}="T";
(0,-9)*+{A^*}="B";
 {\ar^{\sharp}"T";"B"};
 \endxy
\qquad  \equiv \qquad
 \xy
    (0,12)*++{}="g";
    (10,0)*{}="mid";
      **\crv{(5,-22)} ?(.02)*\dir{>} ?(.76)*\dir{<};
    (6,-12)*{\scriptstyle g};
    (20,-12)*++{}="f";
     "f";"mid"; **\crv{(15,22)}  ?(.78)*\dir{<};
    (15,12)*{\scriptstyle i_{A}};
\endxy
}

\newcommand{\defflat}{
\xy (0,11)*{}; (0,0)*\xycircle(2.65,2.65){-}="1_x"; **\dir{-}
?(.7)*\dir{>}; "1_x";(0,-11)*{}; **\dir{-} ?(.3)*\dir{<};
(0,0)*{\flat};
\endxy
\qquad
 \xy
 (0,9)*+{A^*}="T";
(0,-9)*+{A}="B";
 {\ar^{\flat} "T";"B"};
 \endxy}

\newcommand{\traceLaLb}{
\xy
 (-3,4)*{}="B";
 (4,-3)*{}="B'";
    "B'";"B" **\crv{(12,2) & (6,10)}?(0)*\dir{>};
    "B";"B'" **\crv{(-12,-1) & (-6,-10)}?(.07)*\dir{>}?(.35)*\dir{>};
 (-14,12)*\xycircle(2,2){-}="1";
 (-6,12)*\xycircle(2,2){-}="2";
    (-14,12)*{a};
    (-6,12)*{b};
    "2";(-2,4.7) **\crv{(-6,8)}?(.75)*\dir{>};
    "1";(-6.5,1) **\crv{(-16,8)}?(.75)*\dir{>};
 \endxy
 }

 \newcommand{\gandtrace}{
 g \quad = \quad     \vcenter{
 \xy %CUP
    (-6,4)*{};(6,4)*{};
      **\crv{(5,-10) & (-5,-10)}
       ?(.15)*\dir{>}+(2,-1)
       ?(.85)*\dir{<}+(-2,-1);
\endxy}
\qquad  \equiv  \qquad \xy
 (-3,4)*{}="B";
 (4,-3)*{}="B'";
    "B'";"B" **\crv{(12,2) & (6,10)}?(0)*\dir{>};
    "B";"B'" **\crv{(-12,-1) & (-6,-10)}?(.07)*\dir{>}?(.35)*\dir{>};
 (-14,12)*{}="1";
 (-6,12)*{}="2";
    "2";(-2,4.7) **\crv{(-4,8)}?(.75)*\dir{>};
    "1";(-6.5,1) **\crv{(-12,8)}?(.75)*\dir{>};
 \endxy
 }
\newcommand{\triangulatedbubble}{
\vcenter{
 \xy 0;/r.18pc/:
    (-6,4)*{};(6,4)*{};
      **\crv{(5,-10) & (-5,-10)}
       ?(.15)*\dir{>}+(2,-1)
       ?(.85)*\dir{<}+(-2,-1);
 (0,2)*{\bullet}; (0,-14)*{\bullet} **\dir{.};
\endxy}
\qquad  =  \qquad
 \vcenter{\xy 0;/r.14pc/:
  (0,6)*{};
 (0,2)*{}="M";
 (-8,-6)*{}="B";
 (-13.5,-11)*{}="b";
 (-5.5,-16)*{}="b'";
 (0,16)*{}="TL";
 (2,4)*{}="TR";
 (10,-2)*{}="tr";
    "TL";"M" **\dir{-} ?(.75)*\dir{>};
    "M";"M" **\dir{-};
    "M";"B" **\dir{-}?(.55)*\dir{>};
(-8,16)*{}="a";
    "a";"B" **\dir{-}?(.55)*\dir{>};
   "b";"B" **\dir{-};
   "b'";"tr" **\dir{-} ?(.55)*\dir{>};
  "M";"tr" **\crv{(6,9) & (15,4)};
  "b";"b'" **\crv{ (-18,-16)&(-14,-24) };
%^^^^^^^^^^^^
 (-4,12)*{\bullet}="x1"; (-4,-8)*{\bullet}="x2" **\dir{.};
 "x2"; (-4,-28)*{\bullet}="x3" **\dir{.};
(-4,-8)*\xycircle(20,20){.};
 \endxy}}

\newcommand{\bigpentagon}{
\xy 0;/r.20pc/:
    (-24.73,8.03)*+{\big(k(hg) \big)f}="l";
    (0,26)*+{\big((kh)g \big)f}="t";
    (24.73,8.03)*+{(kh)(gf)}="r";
    (15.28,-21.03)*+{k\big((h(gf)\big)}="br";
    (-15.28,-21.03)*+{k\big((hg)f\big)}="bl";
     {\ar^{a} "t";"l"};
     {\ar_{a} "l";"bl"};
     {\ar_{a} "br";"bl"};
     {\ar_{a} "r";"br"};
     {\ar^{a} "t";"r"};
(-51.35,16.68)*+{
    \xy 0;/r.16pc/:
     (-10.6,-14.63)*{}="t1";
     (-17.1,5.56)*{}="t2";
     (0,18)*{}="t3";
     (17.1,5.56)*{}="t4";
     (10.6,-14.63)*{}="t5";
    {\ar@{-}^{f} "t1";"t2"};
    {\ar@{-}^{g} "t2";"t3"};
    {\ar@{-}^{h} "t3";"t4"};
    {\ar@{-}^{k} "t4";"t5"};
    {\ar@{-} "t1";"t5"};
    {\ar@{-} "t2";"t4"};
    {\ar@{-} "t2";"t5"};
   \endxy};
(0,54)*+{
    \xy 0;/r.16pc/:
     (-10.6,-14.63)*{}="t1";
     (-17.1,5.56)*{}="t2";
     (0,18)*{}="t3";
     (17.1,5.56)*{}="t4";
     (10.6,-14.63)*{}="t5";
    {\ar@{-}^{f} "t1";"t2"};
    {\ar@{-}^{g} "t2";"t3"};
    {\ar@{-}@{-}^{h} "t3";"t4"};
    {\ar@{-}^{k} "t4";"t5"};
    {\ar@{-} "t1";"t5"};{\ar@{-} "t2";"t5"};
    {\ar@{-} "t3";"t5"};
   \endxy};
(51.36,16.68)*+{
    \xy 0;/r.16pc/:
     (-10.6,-14.63)*{}="t1";
     (-17.1,5.56)*{}="t2";
     (0,18)*{}="t3";
     (17.1,5.56)*{}="t4";
     (10.6,-14.63)*{}="t5";
   {\ar@{-}^{f} "t1";"t2"};
   {\ar@{-}^{g} "t2";"t3"};
   {\ar@{-}^{h} "t3";"t4"};
   {\ar@{-}^{k} "t4";"t5"};
   {\ar@{-} "t1";"t5"};
   {\ar@{-} "t1";"t3"};
   {\ar@{-} "t3";"t5"};
  \endxy};
(31.73,-43.68)*+{
    \xy 0;/r.16pc/:
     (-10.6,-14.63)*{}="t1";
     (-17.1,5.56)*{}="t2";
     (0,18)*{}="t3";
     (17.1,5.56)*{}="t4";
     (10.6,-14.63)*{}="t5";
    {\ar@{-}^{f} "t1";"t2"};
    {\ar@{-}^{g} "t2";"t3"};
    {\ar@{-}^{h} "t3";"t4"};
    {\ar@{-}^{k} "t4";"t5"};
    {\ar@{-} "t1";"t5"};
    {\ar@{-} "t1";"t3"};
    {\ar@{-} "t1";"t4"};
  \endxy};
(-31.73,-43.68)*+{
    \xy 0;/r.16pc/:
     (-10.6,-14.63)*{}="t1";
     (-17.1,5.56)*{}="t2";
     (0,18)*{}="t3";
     (17.1,5.56)*{}="t4";
     (10.6,-14.63)*{}="t5";
   {\ar@{-}^{f} "t1";"t2"};
   {\ar@{-}^{g} "t2";"t3"};
   {\ar@{-}^{h} "t3";"t4"};
   {\ar@{-}^{k} "t4";"t5"};
   {\ar@{-} "t1";"t5"};
   {\ar@{-} "t2";"t4"};
   {\ar@{-} "t1";"t4"};
  \endxy};
\endxy
}


\newcommand{\tinytwothreemove}{
 \xy 0;/r.09pc/:
 (6.18,19)*{}="t1"; %2pi/5
 (-16.18,11.74)*{}="t2";
 (-16.18,-11.74)*{}="t3";
(6.18,-19)*{}="t4";
 (20,0)*{}="t5";
   {\ar@{-}"t1";"t2"};
   {\ar@{-} "t2";"t3"};
   {\ar@{-} "t3";"t4"};
   {\ar@{-} "t4";"t5"};
   {\ar@{-} "t1";"t5"};
   {\ar@{-} "t1";"t3"}; {\ar@{-} "t3";"t5"};
   {\ar@{-}|>>>>>>{ \hole \; \hole} "t1";"t4"};
   {\ar@{-}|<<<<<<{\hole}|<<<<<<<<<<{ \hole} "t2";"t4"};
\endxy
\quad   = \quad
 \xy 0;/r.09pc/:
 (6.18,19)*{}="t1"; %2pi/5
 (-16.18,11.74)*{}="t2";
 (-16.18,-11.74)*{}="t3";
(6.18,-19)*{}="t4";
 (20,0)*{}="t5";
   {\ar@{-}"t1";"t2"};
   {\ar@{-} "t2";"t3"};
   {\ar@{-} "t3";"t4"};
   {\ar@{-} "t4";"t5"};
   {\ar@{-} "t1";"t5"};
   {\ar@{-} "t1";"t3"}; {\ar@{-} "t3";"t5"};
   {\ar@{-}|<<<<<<{ \hole \; \hole}|>>>>>>{ \hole \; \hole} "t1";"t4"};
   {\ar@{-}|<<<<<<{ \hole } "t2";"t5"};
   {\ar@{-}|<<<<<<{\hole}|<<<<<<<<<<{ \hole} "t2";"t4"};
\endxy
}





\newcommand{\bigtwothreemove}{
\xy %0;/r.18pc/:
 (6.18,19)*{}="t1"; %2pi/5
 (-16.18,11.74)*{}="t2";
 (-16.18,-11.74)*{}="t3";
(6.18,-19)*{}="t4";
 (20,0)*{}="t5";
   {\ar@{-}"t1";"t2"};
   {\ar@{-} "t2";"t3"};
   {\ar@{-}|<<<<<<<<<<<<<<<{ \hole \; \hole}|>>>>>>>>>>>>>>>{ \hole \; \hole} "t2";"t4"};
   {\ar@{-} "t3";"t4"};
   {\ar@{-} "t4";"t5"};
   {\ar@{-} "t1";"t5"};
   {\ar@{-} "t1";"t3"}; {\ar@{-} "t3";"t5"};
   {\ar@{-}|>>>>>>>>>>>>>>>{ \hole \; \hole}  "t1";"t4"};
\endxy
\qquad \qquad  = \qquad  \qquad
 \xy %0;/r.18pc/:
 (6.18,19)*{}="t1"; %2pi/5
 (-16.18,11.74)*{}="t2";
 (-16.18,-11.74)*{}="t3";
(6.18,-19)*{}="t4";
 (20,0)*{}="t5";
   {\ar@{-}"t1";"t2"};
   {\ar@{-} "t2";"t3"};
   {\ar@{-} "t3";"t4"};
   {\ar@{-} "t4";"t5"};
   {\ar@{-} "t1";"t5"};
   {\ar@{-} "t1";"t3"}; {\ar@{-} "t3";"t5"};
   {\ar@{-}|<<<<<<<<<<<<<<<{ \hole \; \hole}|>>>>>>>>>>>>>>>{ \hole \; \hole} "t1";"t4"};
   {\ar@{-}|<<<<<<<<<<<<<<{ \hole \; \hole} "t2";"t5"};
   {\ar@{-}|<<<<<<<<<<<<<<<{ \hole \; \hole}|>>>>>>>>>>>>>>>{ \hole \; \hole} "t2";"t4"};
\endxy
}


\newcommand{\mediumtwothreemove}{
 \xy 0;/r.17pc/:
 (6.18,19)*{}="t1"; %2pi/5
 (-16.18,11.74)*{}="t2";
 (-16.18,-11.74)*{}="t3";
(6.18,-19)*{}="t4";
 (20,0)*{}="t5";
   {\ar@{-}"t1";"t2"};
   {\ar@{-} "t2";"t3"};
   {\ar@{-} "t3";"t4"};
   {\ar@{-} "t4";"t5"};
   {\ar@{-} "t1";"t5"};
   {\ar@{-} "t1";"t3"}; {\ar@{-} "t3";"t5"};
   {\ar@{-}|>>>>>>>>>>>{ \hole \; \hole} "t1";"t4"};
   {\ar@{-}|<<<<<<<<<<<{ \hole \; \hole}|>>>>>>>>>>>{ \hole \; \hole} "t2";"t4"};
\endxy
\qquad   =   \qquad
 \xy 0;/r.18pc/:
 (6.18,19)*{}="t1"; %2pi/5
 (-16.18,11.74)*{}="t2";
 (-16.18,-11.74)*{}="t3";
(6.18,-19)*{}="t4";
 (20,0)*{}="t5";
   {\ar@{-}"t1";"t2"};
   {\ar@{-} "t2";"t3"};
   {\ar@{-} "t3";"t4"};
   {\ar@{-} "t4";"t5"};
   {\ar@{-} "t1";"t5"};
   {\ar@{-} "t1";"t3"}; {\ar@{-} "t3";"t5"};
   {\ar@{-}|<<<<<<<<<<<{ \hole \; \hole}|>>>>>>>>>>>{ \hole \; \hole} "t1";"t4"};
   {\ar@{-}|<<<<<<<<<<<<{ \hole \; \hole} "t2";"t5"};
   {\ar@{-}|<<<<<<<<<<<{ \hole \; \hole}|>>>>>>>>>>>{ \hole \; \hole} "t2";"t4"};
\endxy
}

\newcommand{\smalltwothreemove}{
 \xy 0;/r.10pc/:
 (6.18,19)*{}="t1"; %2pi/5
 (-16.18,11.74)*{}="t2";
 (-16.18,-11.74)*{}="t3";
(6.18,-19)*{}="t4";
 (20,0)*{}="t5";
   {\ar@{-}"t1";"t2"};
   {\ar@{-} "t2";"t3"};
   {\ar@{-} "t3";"t4"};
   {\ar@{-} "t4";"t5"};
   {\ar@{-} "t1";"t5"};
   {\ar@{-} "t1";"t3"}; {\ar@{-} "t3";"t5"};
   {\ar@{-}|>>>>>>>{ \hole \; \hole} "t1";"t4"};
   {\ar@{-}|<<<<<<<{  \hole}|>>>>>>{  \hole} "t2";"t4"};
\endxy
\qquad   = \qquad
 \xy 0;/r.10pc/:
 (6.18,19)*{}="t1"; %2pi/5
 (-16.18,11.74)*{}="t2";
 (-16.18,-11.74)*{}="t3";
(6.18,-19)*{}="t4";
 (20,0)*{}="t5";
   {\ar@{-}"t1";"t2"};
   {\ar@{-} "t2";"t3"};
   {\ar@{-} "t3";"t4"};
   {\ar@{-} "t4";"t5"};
   {\ar@{-} "t1";"t5"};
   {\ar@{-} "t1";"t3"}; {\ar@{-} "t3";"t5"};
   {\ar@{-}|<<<<<<<{ \hole \; \hole}|>>>>>>>{ \hole \; \hole} "t1";"t4"};
   {\ar@{-}|<<<<<<<{ \hole } "t2";"t5"};
   {\ar@{-}|<<<<<<<{  \hole}|>>>>>>{  \hole} "t2";"t4"};
\endxy
}

\newcommand{\smallfouronemove}{
\xy 0;/r.15pc/:
 (-10,-5 )*{}="1";
 (8,-10)*{}="2";
 (15,0)*{}="3";
 (1,12)*{}="4";
    {\ar@{-} "1";"2" };
    {\ar@{-}"2";"3" };
    {\ar@{-} "4";"3" };
    {\ar@{-} "1";"4" };
    {\ar@{-} "4";"2" };
    {\ar@{.}|>>>>>>>>>>{\hole \hole} "1";"3"};
 \endxy
\qquad =\qquad
 \xy 0;/r.15pc/:
 (-10,-5 )*{}="1";
 (8,-10)*{}="2";
 (15,0)*{}="3";
 (1,12)*{}="4";
 (0,3)*{}="m";
    {\ar@{-} "1";"m" };
    {\ar@{-} "2";"m" };
    {\ar@{-} "3";"m" };
    {\ar@{-} "4";"m" };
    {\ar@{-} "1";"2" };
    {\ar@{-}"2";"3" };
    {\ar@{-} "4";"3" };
    {\ar@{-} "1";"4" };
    {\ar@{-} "4";"2" };
    {\ar@{.}|>>>>>>>>>>{\hole \hole} "1";"3"};
 \endxy
 }

\newcommand{\mediumfouronemove}{
 \xy
 (-10,-5 )*{}="1";
 (8,-10)*{}="2";
 (15,0)*{}="3";
 (1,12)*{}="4";
    {\ar@{-} "1";"2" };
    {\ar@{-}"2";"3" };
    {\ar@{-} "4";"3" };
    {\ar@{-} "1";"4" };
    {\ar@{-} "4";"2" };
    {\ar@{.}|>>>>>>>>>>{\hole \hole} "1";"3"};
 \endxy
\qquad =\qquad
 \xy
 (-10,-5 )*{}="1";
 (8,-10)*{}="2";
 (15,0)*{}="3";
 (1,12)*{}="4";
 (0,3)*{}="m";
    {\ar@{-} "1";"m" };
    {\ar@{-} "2";"m" };
    {\ar@{-} "3";"m" };
    {\ar@{-} "4";"m" };
    {\ar@{-} "1";"2" };
    {\ar@{-}"2";"3" };
    {\ar@{-} "4";"3" };
    {\ar@{-} "1";"4" };
    {\ar@{-} "4";"2" };
    {\ar@{.}|>>>>>>>>>>{\hole \hole} "1";"3"};
 \endxy
}


\newcommand{\stickassociator}{
\xy 0;/r.13pc/:
 (-30,0)*{
  \xy 0;/r.15pc/:
    (-8,10)*{}="TL";
    (8,10)*{}="TR";
    (-2,10)*{}="X'";
    (-5,7)*{}="XM'";
    (2,10)*{}="X";
    (5,7)*{}="XM";
    (0,2)*{}="M";
    (0,-10)*{}="B";
    "TL";"M" **\dir{-};
    "X";"XM" **\dir{-};
    "X'";"XM'" **\dir{-};
    "TR";"M" **\dir{-};
    "B";"M" **\dir{-};
    \endxy
    }="1";
 (30,0)*{
    \xy 0;/r.15pc/:
    (-8,10)*{}="TL";
    (8,10)*{}="TR";
    (-2,10)*{}="C";
    (2,10)*{}="X";
    (0,8)*{}="XM";
    (3,5)*{}="CM";
    (0,2)*{}="M";
    (0,-10)*{}="B";
    "TL";"M" **\dir{-};
    "X";"XM" **\dir{-};
    "C";"CM" **\dir{-};
    "TR";"M" **\dir{-};
    "B";"M" **\dir{-};
    \endxy
    }="5";
 (15,-30)*{
    \xy 0;/r.15pc/:
    (-8,10)*{}="TL";
    (8,10)*{}="TR";
    (2,10)*{}="C";
    (-2,10)*{}="X";
    (0,8)*{}="XM";
    (-3,5)*{}="CM";
    (0,2)*{}="M";
    (0,-10)*{}="B";
    "TL";"M" **\dir{-};
    "X";"XM" **\dir{-};
    "C";"CM" **\dir{-};
    "TR";"M" **\dir{-};
    "B";"M" **\dir{-};
    \endxy
    }="4";
 (0,20)*{
    \xy 0;/r.15pc/:% Far left of stick associator
    (-8,10)*{}="TL";
    (8,10)*{}="TR";
    (2,10)*{}="C";
    (-2,10)*{}="X";
    (-5,7)*{}="XM";
    (-3,5)*{}="CM";
    (0,2)*{}="M";
    (0,-10)*{}="B";
    "TL";"M" **\dir{-};
    "X";"XM" **\dir{-};
    "C";"CM" **\dir{-};
    "TR";"M" **\dir{-};
    "B";"M" **\dir{-};
\endxy
    }="2";
 (-15,-30)*{
 \xy 0;/r.15pc/: % Far left of stick associator
    (8,10)*{}="TL";
    (-8,10)*{}="TR";
    (-2,10)*{}="C";
    (2,10)*{}="X";
    (5,7)*{}="XM";
    (3,5)*{}="CM";
    (0,2)*{}="M";
    (0,-10)*{}="B";
    "TL";"M" **\dir{-};
    "X";"XM" **\dir{-};
    "C";"CM" **\dir{-};
    "TR";"M" **\dir{-};
    "B";"M" **\dir{-};
\endxy
    }="3";
    {\ar@{=>} "2";"1"};
    {\ar@{=>} "2";"5"};
    {\ar@{=>} "1";"3"};
    {\ar@{=>} "4";"3"};
    {\ar@{=>} "5";"4"};
\endxy}


\newcommand{\LTQTtable}{
\begin{tabular}{|c|c|}
  \hline
  % after \\: \hline or \cline{col1-col2} \cline{col3-col4} ...
  \textbf{2D Lattice Field Theory} & \textbf{3D Lattice Field Theory} \\
  \hline
  $\vcenter{\xy (0,0)*{\LARGE \bullet}; \endxy}$  unlabeled
&
  $\vcenter{\xy (0,0)*{\LARGE \bullet}; \endxy}$  unlabeled
\\
  $\vcenter{\xy {\ar (-6,0)*+{\bullet};
  (6,0)*+{\bullet}}; (0,5)*{};(0,-5)*{};\endxy}$  $A \in \Vect$
 & $\vcenter{\xy {\ar (-6,0)*+{\bullet};
  (6,0)*+{\bullet}}; (0,5)*{};(0,-5)*{};\endxy}$  $A \in$ 2-$\Vect$ \\
  $ \vcenter{  \xy 0;/r.16pc/:
  (0,6)*{\bullet};
 (-10,0)*{}="L";
 (10,0)*{}="R";
 (0,16)*{}="T";
 (0,6)*{}="M";
 (0,-4)*{}="B";
 (-10,12)*{}="TL";
 (10,12)*{}="TR";
    "T";"L" **\dir{.};
    "R";"T" **\dir{.};
    "L";"R" **\dir{.};
    "TL";"M" **\dir{-}?(.5)*\dir{>};
    "TR";"M" **\dir{-}?(.5)*\dir{>};
    "M";"B" **\dir{-}?(.6)*\dir{>};
 \endxy}$ $m \maps A \tensor A \to A$ &   $ \vcenter{  \xy 0;/r.16pc/:
  (0,6)*{\bullet};
 (-10,0)*{}="L";
 (10,0)*{}="R";
 (0,16)*{}="T";
 (0,6)*{}="M";
 (0,-4)*{}="B";
 (-10,12)*{}="TL";
 (10,12)*{}="TR";
    "T";"L" **\dir{.};
    "R";"T" **\dir{.};
    "L";"R" **\dir{.};
    "TL";"M" **\dir{-}?(.5)*\dir{>};
    "TR";"M" **\dir{-}?(.5)*\dir{>};
    "M";"B" **\dir{-}?(.6)*\dir{>}; (0,-12)*{};
 \endxy}$ $m \maps A \tensor A \to A$ \footnote{Care must be taken in the definition
 of this tensor product but using a basis this can be done without to much difficulty.} \\
  $\xy 0;/r.16pc/:
 (0,10)*{}="mt";
 (0,-10)*{}="mb";
 (-16,0)*{}="l";
 (16,0)*{}="r";
  (6,0)*{}="xr";
  (-6,0)*{}="xl";
  (-12,10)*{}="xlt";
  (-12,-10)*{}="xlb";
    (12,10)*{}="xrt";
  (12,-10)*{}="xrb";
  (6,0)*{\bullet};
  (-6,0)*{\bullet};
    "mt";"mb" **\dir{.};
    "mb";"l" **\dir{.};
    "mt";"l" **\dir{.};
    "mb";"r" **\dir{.};
    "mt";"r" **\dir{.};
    "xl";"xr" **\dir{-}?(.5)*\dir{<};
    "xl";"xlt" **\dir{-}?(.5)*\dir{<};
    "xl";"xlb" **\dir{-}?(.65)*\dir{>};
    "xr";"xrt" **\dir{-}?(.5)*\dir{<};
    "xr";"xrb" **\dir{-}?(.5)*\dir{<};
 \endxy
 \quad = \quad
 \xy 0;/r.16pc/:
 (0,10)*{}="mt";
 (0,-10)*{}="mb";
 (-16,0)*{}="l";
 (16,0)*{}="r";
  (0,-4)*{}="xr";
  (0,4)*{}="xl";
  (10,10)*{}="xlt";
  (-10,10)*{}="xlb";
    (10,-10)*{}="xrt";
  (-10,-10)*{}="xrb";
  (0,4)*{\bullet};
  (0,-4)*{\bullet};
    "l";"r" **\dir{.};
    "mb";"l" **\dir{.};
    "mt";"l" **\dir{.};
    "mb";"r" **\dir{.};
    "mt";"r" **\dir{.};
    "xl";"xr" **\dir{-}?(.58)*\dir{>};
    "xl";"xlt" **\dir{-}?(.5)*\dir{<};
    "xl";"xlb" **\dir{-}?(.5)*\dir{<};
    "xr";"xrt" **\dir{-}?(.5)*\dir{<};
    "xr";"xrb" **\dir{-}?(.65)*\dir{>};
 \endxy$ &  $\xy 0;/r.16pc/:
 (0,10)*{}="mt";
 (0,-10)*{}="mb";
 (-16,0)*{}="l";
 (16,0)*{}="r";
  (6,0)*{}="xr";
  (-6,0)*{}="xl";
  (-12,10)*{}="xlt";
  (-12,-10)*{}="xlb";
    (12,10)*{}="xrt";
  (12,-10)*{}="xrb";
  (6,0)*{\bullet};
  (-6,0)*{\bullet};
    "mt";"mb" **\dir{.};
    "mb";"l" **\dir{.};
    "mt";"l" **\dir{.};
    "mb";"r" **\dir{.};
    "mt";"r" **\dir{.};
    "xl";"xr" **\dir{-}?(.5)*\dir{<};
    "xl";"xlt" **\dir{-}?(.5)*\dir{<};
    "xl";"xlb" **\dir{-}?(.65)*\dir{>};
    "xr";"xrt" **\dir{-}?(.5)*\dir{<};
    "xr";"xrb" **\dir{-}?(.5)*\dir{<};
 \endxy
 \quad \xy {\ar@{=>}^{\scs \alpha} (-3,0);(3,0)}; \endxy \quad
 \xy 0;/r.16pc/:
 (0,10)*{}="mt";
 (0,-10)*{}="mb";
 (-16,0)*{}="l";
 (16,0)*{}="r";
  (0,-4)*{}="xr";
  (0,4)*{}="xl";
  (10,10)*{}="xlt";
  (-10,10)*{}="xlb";
    (10,-10)*{}="xrt";
  (-10,-10)*{}="xrb";
  (0,4)*{\bullet};
  (0,-4)*{\bullet};
    "l";"r" **\dir{.};
    "mb";"l" **\dir{.};
    "mt";"l" **\dir{.};
    "mb";"r" **\dir{.};
    "mt";"r" **\dir{.};
    "xl";"xr" **\dir{-}?(.58)*\dir{>};
    "xl";"xlt" **\dir{-}?(.5)*\dir{<};
    "xl";"xlb" **\dir{-}?(.5)*\dir{<};
    "xr";"xrt" **\dir{-}?(.5)*\dir{<};
    "xr";"xrb" **\dir{-}?(.65)*\dir{>};
 \endxy$\\
  $\vcenter{\xy 0;/r.16pc/:
  (0,6)*{\bullet};
(0,24)*{}; %SPACING
 (-10,0)*{}="L";
 (10,0)*{}="R";
 (0,16)*{}="T";
 (0,6)*{}="M";
 (0,-4)*{}="B";
 (-10,12)*{}="TL";
 (10,12)*{}="TR";
    "T";"L" **\dir{.};
    "R";"T" **\dir{.};
    "L";"R" **\dir{.};
    "TL";"M" **\dir{-};
    "TR";"M" **\dir{-};
    "M";"B" **\dir{-};
 \endxy}
\quad = \quad
 \vcenter{\xy 0;/r.16pc/:
 (0,24)*{}; %SPACING
(-10,0)*{}="L";
 (10,0)*{}="R";
 (0,16)*{}="T";
 (0,6)*{}="M";
    "L";"T" **\dir{.};
    "R";"T" **\dir{.};
    "L";"R" **\dir{.};
    "T";"M" **\dir{.};
    "R";"M" **\dir{.};
    "L";"M" **\dir{.};
%%%%%%%%%%%%%%
 (0,-4)*{}="B";
 (-10,12)*{}="TL";
 (10,12)*{}="TR";
 (-3.5,8)*{}="tl";
 (3.5,8)*{}="tr";
 (0,2.5)*{}="b";
    "TL";"tl" **\dir{-};
    "TR";"tr" **\dir{-};
    "b";"B" **\dir{-};
    "tl";"tr" **\dir{-};
    "tr";"b" **\dir{-};
    "tl";"b" **\dir{-}?(.6)*\dir{};
 \endxy}$ &  $\vcenter{\xy 0;/r.16pc/:
  (0,6)*{\bullet};
(0,24)*{}; %SPACING
 (-10,0)*{}="L";
 (10,0)*{}="R";
 (0,16)*{}="T";
 (0,6)*{}="M";
 (0,-4)*{}="B";
 (-10,12)*{}="TL";
 (10,12)*{}="TR";
    "T";"L" **\dir{.};
    "R";"T" **\dir{.};
    "L";"R" **\dir{.};
    "TL";"M" **\dir{-};
    "TR";"M" **\dir{-};
    "M";"B" **\dir{-};
 \endxy}
\quad \xy {\ar@{=>}^{} (-3,1);(3,1)};
          {\ar@{=>}^{} (3,-4);(-3,-4)};\endxy \quad
 \vcenter{\xy 0;/r.16pc/:
 (0,24)*{}; %SPACING
(-10,0)*{}="L";
 (10,0)*{}="R";
 (0,16)*{}="T";
 (0,6)*{}="M";
    "L";"T" **\dir{.};
    "R";"T" **\dir{.};
    "L";"R" **\dir{.};
    "T";"M" **\dir{.};
    "R";"M" **\dir{.};
    "L";"M" **\dir{.};
%%%%%%%%%%%%%%
 (0,-4)*{}="B";
 (-10,12)*{}="TL";
 (10,12)*{}="TR";
 (-3.5,8)*{}="tl";
 (3.5,8)*{}="tr";
 (0,2.5)*{}="b";
    "TL";"tl" **\dir{-};
    "TR";"tr" **\dir{-};
    "b";"B" **\dir{-};
    "tl";"tr" **\dir{-};
    "tr";"b" **\dir{-};
    "tl";"b" **\dir{-}?(.6)*\dir{};
 \endxy}$ \\
   & $\stickassociator$ \\
   &  \\
  \hline
\end{tabular}
}



\newcommand{\LTQTtableII}{
\begin{tabular}{|c|c|}
  \hline
  % after \\: \hline or \cline{col1-col2} \cline{col3-col4} ...
  \textbf{2D Lattice Field Theory} & \textbf{3D Lattice Field Theory} \\
  \hline
  $\vcenter{\xy (0,0)*{\LARGE \bullet}; \endxy}$  unlabeled
&
  $\vcenter{\xy (0,0)*{\LARGE \bullet}; \endxy}$  unlabeled
\\
  $\vcenter{\xy {\ar (-6,0)*+{\bullet};
  (6,0)*+{\bullet}}; (0,5)*{};(0,-5)*{};\endxy}$  $A \in \Vect$
 & $\vcenter{\xy {\ar (-6,0)*+{\bullet};
  (6,0)*+{\bullet}}; (0,5)*{};(0,-5)*{};\endxy}$  $A \in$ 2-$\Vect$ \\
  $ \vcenter{  \xy 0;/r.16pc/:
  (0,6)*{\bullet};
 (-10,0)*{}="L";
 (10,0)*{}="R";
 (0,16)*{}="T";
 (0,6)*{}="M";
 (0,-4)*{}="B";
 (-10,12)*{}="TL";
 (10,12)*{}="TR";
    "T";"L" **\dir{.};
    "R";"T" **\dir{.};
    "L";"R" **\dir{.};
    "TL";"M" **\dir{-}?(.5)*\dir{>};
    "TR";"M" **\dir{-}?(.5)*\dir{>};
    "M";"B" **\dir{-}?(.6)*\dir{>};
 \endxy}$ $m \maps A \tensor A \to A$ &   $ \vcenter{  \xy 0;/r.16pc/:
  (0,6)*{\bullet};
 (-10,0)*{}="L";
 (10,0)*{}="R";
 (0,16)*{}="T";
 (0,6)*{}="M";
 (0,-4)*{}="B";
 (-10,12)*{}="TL";
 (10,12)*{}="TR";
    "T";"L" **\dir{.};
    "R";"T" **\dir{.};
    "L";"R" **\dir{.};
    "TL";"M" **\dir{-}?(.5)*\dir{>};
    "TR";"M" **\dir{-}?(.5)*\dir{>};
    "M";"B" **\dir{-}?(.6)*\dir{>}; (0,-12)*{};
 \endxy}$ $m \maps A \tensor A \to A$ \footnote{Care must be taken in the definition
 of this tensor product but using a basis this can be done without to much difficulty.} \\
  $\xy 0;/r.16pc/:
 (0,10)*{}="mt";
 (0,-10)*{}="mb";
 (-16,0)*{}="l";
 (16,0)*{}="r";
  (6,0)*{}="xr";
  (-6,0)*{}="xl";
  (-12,10)*{}="xlt";
  (-12,-10)*{}="xlb";
    (12,10)*{}="xrt";
  (12,-10)*{}="xrb";
  (6,0)*{\bullet};
  (-6,0)*{\bullet};
    "mt";"mb" **\dir{.};
    "mb";"l" **\dir{.};
    "mt";"l" **\dir{.};
    "mb";"r" **\dir{.};
    "mt";"r" **\dir{.};
    "xl";"xr" **\dir{-}?(.5)*\dir{<};
    "xl";"xlt" **\dir{-}?(.5)*\dir{<};
    "xl";"xlb" **\dir{-}?(.65)*\dir{>};
    "xr";"xrt" **\dir{-}?(.5)*\dir{<};
    "xr";"xrb" **\dir{-}?(.5)*\dir{<};
 \endxy
 \quad = \quad
 \xy 0;/r.16pc/:
 (0,10)*{}="mt";
 (0,-10)*{}="mb";
 (-16,0)*{}="l";
 (16,0)*{}="r";
  (0,-4)*{}="xr";
  (0,4)*{}="xl";
  (10,10)*{}="xlt";
  (-10,10)*{}="xlb";
    (10,-10)*{}="xrt";
  (-10,-10)*{}="xrb";
  (0,4)*{\bullet};
  (0,-4)*{\bullet};
    "l";"r" **\dir{.};
    "mb";"l" **\dir{.};
    "mt";"l" **\dir{.};
    "mb";"r" **\dir{.};
    "mt";"r" **\dir{.};
    "xl";"xr" **\dir{-}?(.58)*\dir{>};
    "xl";"xlt" **\dir{-}?(.5)*\dir{<};
    "xl";"xlb" **\dir{-}?(.5)*\dir{<};
    "xr";"xrt" **\dir{-}?(.5)*\dir{<};
    "xr";"xrb" **\dir{-}?(.65)*\dir{>};
 \endxy$ &  $\xy 0;/r.16pc/:
 (0,10)*{}="mt";
 (0,-10)*{}="mb";
 (-16,0)*{}="l";
 (16,0)*{}="r";
  (6,0)*{}="xr";
  (-6,0)*{}="xl";
  (-12,10)*{}="xlt";
  (-12,-10)*{}="xlb";
    (12,10)*{}="xrt";
  (12,-10)*{}="xrb";
  (6,0)*{\bullet};
  (-6,0)*{\bullet};
    "mt";"mb" **\dir{.};
    "mb";"l" **\dir{.};
    "mt";"l" **\dir{.};
    "mb";"r" **\dir{.};
    "mt";"r" **\dir{.};
    "xl";"xr" **\dir{-}?(.5)*\dir{<};
    "xl";"xlt" **\dir{-}?(.5)*\dir{<};
    "xl";"xlb" **\dir{-}?(.65)*\dir{>};
    "xr";"xrt" **\dir{-}?(.5)*\dir{<};
    "xr";"xrb" **\dir{-}?(.5)*\dir{<};
 \endxy
 \quad \xy {\ar@{=>}^{\scs \alpha} (-3,0);(3,0)}; \endxy \quad
 \xy 0;/r.16pc/:
 (0,10)*{}="mt";
 (0,-10)*{}="mb";
 (-16,0)*{}="l";
 (16,0)*{}="r";
  (0,-4)*{}="xr";
  (0,4)*{}="xl";
  (10,10)*{}="xlt";
  (-10,10)*{}="xlb";
    (10,-10)*{}="xrt";
  (-10,-10)*{}="xrb";
  (0,4)*{\bullet};
  (0,-4)*{\bullet};
    "l";"r" **\dir{.};
    "mb";"l" **\dir{.};
    "mt";"l" **\dir{.};
    "mb";"r" **\dir{.};
    "mt";"r" **\dir{.};
    "xl";"xr" **\dir{-}?(.58)*\dir{>};
    "xl";"xlt" **\dir{-}?(.5)*\dir{<};
    "xl";"xlb" **\dir{-}?(.5)*\dir{<};
    "xr";"xrt" **\dir{-}?(.5)*\dir{<};
    "xr";"xrb" **\dir{-}?(.65)*\dir{>};
 \endxy$\\
  $\vcenter{\xy 0;/r.16pc/:
  (0,6)*{\bullet};
(0,24)*{}; %SPACING
 (-10,0)*{}="L";
 (10,0)*{}="R";
 (0,16)*{}="T";
 (0,6)*{}="M";
 (0,-4)*{}="B";
 (-10,12)*{}="TL";
 (10,12)*{}="TR";
    "T";"L" **\dir{.};
    "R";"T" **\dir{.};
    "L";"R" **\dir{.};
    "TL";"M" **\dir{-};
    "TR";"M" **\dir{-};
    "M";"B" **\dir{-};
 \endxy}
\quad = \quad
 \vcenter{\xy 0;/r.16pc/:
 (0,24)*{}; %SPACING
(-10,0)*{}="L";
 (10,0)*{}="R";
 (0,16)*{}="T";
 (0,6)*{}="M";
    "L";"T" **\dir{.};
    "R";"T" **\dir{.};
    "L";"R" **\dir{.};
    "T";"M" **\dir{.};
    "R";"M" **\dir{.};
    "L";"M" **\dir{.};
%%%%%%%%%%%%%%
 (0,-4)*{}="B";
 (-10,12)*{}="TL";
 (10,12)*{}="TR";
 (-3.5,8)*{}="tl";
 (3.5,8)*{}="tr";
 (0,2.5)*{}="b";
    "TL";"tl" **\dir{-};
    "TR";"tr" **\dir{-};
    "b";"B" **\dir{-};
    "tl";"tr" **\dir{-};
    "tr";"b" **\dir{-};
    "tl";"b" **\dir{-}?(.6)*\dir{};
 \endxy}$ &  $\vcenter{\xy 0;/r.16pc/:
  (0,6)*{\bullet};
(0,24)*{}; %SPACING
 (-10,0)*{}="L";
 (10,0)*{}="R";
 (0,16)*{}="T";
 (0,6)*{}="M";
 (0,-4)*{}="B";
 (-10,12)*{}="TL";
 (10,12)*{}="TR";
    "T";"L" **\dir{.};
    "R";"T" **\dir{.};
    "L";"R" **\dir{.};
    "TL";"M" **\dir{-};
    "TR";"M" **\dir{-};
    "M";"B" **\dir{-};
 \endxy}
\quad \xy {\ar@{=>}^{} (-3,1);(3,1)};
          {\ar@{=>}^{} (3,-4);(-3,-4)};\endxy \quad
 \vcenter{\xy 0;/r.16pc/:
 (0,24)*{}; %SPACING
(-10,0)*{}="L";
 (10,0)*{}="R";
 (0,16)*{}="T";
 (0,6)*{}="M";
    "L";"T" **\dir{.};
    "R";"T" **\dir{.};
    "L";"R" **\dir{.};
    "T";"M" **\dir{.};
    "R";"M" **\dir{.};
    "L";"M" **\dir{.};
%%%%%%%%%%%%%%
 (0,-4)*{}="B";
 (-10,12)*{}="TL";
 (10,12)*{}="TR";
 (-3.5,8)*{}="tl";
 (3.5,8)*{}="tr";
 (0,2.5)*{}="b";
    "TL";"tl" **\dir{-};
    "TR";"tr" **\dir{-};
    "b";"B" **\dir{-};
    "tl";"tr" **\dir{-};
    "tr";"b" **\dir{-};
    "tl";"b" **\dir{-}?(.6)*\dir{};
 \endxy}$ \\
   $\xy (0,15)*{};(0,-15)*{}; \endxy$ & $\smalltwothreemove$ \\
   $\xy (0,-10)*{}; \endxy$& $\smallfouronemove$ \\
  \hline
\end{tabular}
}




\newcommand{\bubblenattrans}{
  \vcenter{
 \xy %CUP
    (-6,4)*{};(6,4)*{};
      **\crv{(5,-10) & (-5,-10)};
\endxy}
\qquad  \xy {\ar@{=>} (-3,0);(3,0)} \endxy  \qquad
 \vcenter{\xy 0;/r.14pc/:
  (0,6)*{};
 (0,6)*{}="M";
 (-6,-4)*{}="B";
 (-9.5,-10)*{}="b";
 (-1.5,-16)*{}="b'";
 (-4,14)*{}="TL";
 (4,12)*{}="TR";
 (12,6)*{}="tr";
    "TL";"M" **\dir{-} ;
    "TR";"M" **\dir{-};
    "M";"B" **\dir{-};
(-16,14)*{}="a";
    "a";"B" **\dir{-};
   "b";"B" **\dir{-};
   "b'";"tr" **\dir{-} ;
  "TR";"tr" **\crv{(8,18) & (16,12)};
  "b";"b'" **\crv{ (-14,-16)&(-6,-22) };
 \endxy}
 }

\newcommand{\dualbubblenattrans}{
  \vcenter{
 \xy %CUP
    (-6,4)*{};(6,4)*{};
      **\crv{(5,-10) & (-5,-10)};
\endxy}
\qquad  \xy {\ar@{=>} (3,0);(-3,0)} \endxy  \qquad
 \vcenter{\xy 0;/r.14pc/:
  (0,6)*{};
 (0,6)*{}="M";
 (-6,-4)*{}="B";
 (-9.5,-10)*{}="b";
 (-1.5,-16)*{}="b'";
 (-4,14)*{}="TL";
 (4,12)*{}="TR";
 (12,6)*{}="tr";
    "TL";"M" **\dir{-} ;
    "TR";"M" **\dir{-};
    "M";"B" **\dir{-};
(-16,14)*{}="a";
    "a";"B" **\dir{-};
   "b";"B" **\dir{-};
   "b'";"tr" **\dir{-} ;
  "TR";"tr" **\crv{(8,18) & (16,12)};
  "b";"b'" **\crv{ (-14,-16)&(-6,-22) };
 \endxy}
 }



\newcommand{\psBubble}{
\psset{unit=0.5cm}
 \begin{pspicture}(15,9)
    \pscircle[fillcolor=lightgray,fillstyle=gradient,
      gradbegin=white, gradend=darkgray,gradmidpoint=0,gradangle=110](5,5){1}
    \pspolygon[fillcolor=lightgray,fillstyle=gradient,
      gradbegin=darkgray, gradend=white, gradmidpoint=1,gradangle=110](4.4,6.5)(4.4,1.5)(5.5,3)(5.5,8)
 \begin{psclip}{\pscustom{\psellipse[linestyle=dashed](5,5)(.3,1) \pswedge[linestyle=none](5,5){1}{-93}{93}} }
 \pscircle[fillcolor=lightgray,fillstyle=gradient,
         gradbegin=white, gradend=black,gradmidpoint=0,gradangle=130](5,5){1}
\end{psclip}
 \rput(10,5){=}
    \pspolygon[fillcolor=lightgray,fillstyle=gradient,
      gradbegin=darkgray, gradend=white, gradmidpoint=1,gradangle=110](14.4,6.5)(14.4,1.5)(15.5,3)(15.5,8)
 \end{pspicture}
}






\newcommand{\psYbubble}{ \psset{unit=0.5cm} \vcenter{
 \begin{pspicture}(10,10)
    \pspolygon[fillcolor=gray,fillstyle=gradient,
      gradbegin=gray, gradend=black,
      gradmidpoint=1,gradangle=290](5,8)(3,9)(3,3)(5,2)
    \pspolygon[fillstyle=gradient,
      gradbegin=lightgray, gradend=darkgray, gradmidpoint=0,gradangle=110](5,8)(5,2)(7.5,2.5)(7.5,8.5)
    \pscircle[fillcolor=lightgray,fillstyle=gradient,
      gradbegin=white, gradend=black,gradmidpoint=0,gradangle=130](5,5){1}
    \pspolygon[fillcolor=lightgray,fillstyle=gradient,
      gradbegin=darkgray, gradend=lightgray, gradmidpoint=1,gradangle=110](4.4,5)(4.4,0)(5,2)(5,8)
 \begin{psclip}{\psellipse[linestyle=dashed](5,5)(.25,1)}
 \pscircle[fillcolor=lightgray,fillstyle=gradient,
      gradbegin=white, gradend=black,gradmidpoint=0,gradangle=130](5,5){1}
\end{psclip}
\rput(12.5,5){=}
    \pspolygon[fillcolor=gray,fillstyle=gradient,
      gradbegin=gray, gradend=black,
      gradmidpoint=1,gradangle=290](20,8)(18,9)(18,3)(20,2)
    \pspolygon[fillstyle=gradient,
      gradbegin=lightgray, gradend=darkgray, gradmidpoint=0,gradangle=110](20,8)(20,2)(22.5,2.5)(22.5,8.5)
    \pspolygon[fillcolor=lightgray,fillstyle=gradient,
      gradbegin=darkgray, gradend=lightgray, gradmidpoint=1,gradangle=110](19.4,5)(19.4,0)(20,2)(20,8)
 \end{pspicture}}
}


\newcommand{\YYYA}{
\begin{pspicture}[.5](5,5)
       \pscustom[fillcolor=gray,  fillstyle=gradient,
    gradbegin=white, gradend=black, gradmidpoint=0,gradangle=140]{
        \psbezier(3,4)(3.3,4.7)(3.3,4.7)(3.8,4.8)
        \psline(3.8,2)
   }
    \pscustom[fillcolor=gray,  fillstyle=gradient,
    gradbegin=white, gradend=gray, gradmidpoint=.35,gradangle=140]{
        \psbezier(2,3)(2.6,3.4)(2.4,3.8)(3,4)
        \psbezier(3,4)(3.5,3.8)(3.7,3.5)(4.5,3.8)
        \psline(4.5,1.5)
   }
  \pscustom[fillcolor=gray,  fillstyle=gradient,
    gradbegin=white, gradend=gray, gradmidpoint=.5,gradangle=100]{
        \psline(2,1)(0,0)
        \psline(0,0)(0,2)
        \psline(0,2)(2,3)
        \psbezier(2,3)(3,3.5)(4,1.8)(5.5,2.5)
        \psline(5.5,2.5)(5.5,.5)
        \psbezier(5.5,.5)(4,-.2)(3,1.5)(2,1)
    }
   \psbezier[linestyle=dotted](4,.5)(4.3,.7)(4,1.2)(4.5,1.5)
   \psline[linestyle=dotted](4.5,1.5)(4.5,3.8)
   \psbezier[linestyle=dotted](2,1)(2.8,1.2)(2.6,2.6)(3.8,2.8)
   \psline[linestyle=dotted](3.8,2.8)(3.8,4.8)
\end{pspicture}
}

\newcommand{\YYYB}{
\begin{pspicture}[.5](5,5)
       \pscustom[fillcolor=gray,  fillstyle=gradient,
    gradbegin=white, gradend=black, gradmidpoint=0,gradangle=140]{
        \psbezier(2.8,4.2)(3.3,4.7)(3.3,4.7)(3.8,4.8)
        \psline(3.8,2)
   }
    \pscustom[fillcolor=gray,  fillstyle=gradient,
    gradbegin=white, gradend=gray, gradmidpoint=.35,gradangle=140]{
        \psbezier(2,3)(2.6,3.4)(2.4,3.8)(2.8,4.2)
        \psbezier(2.8,4.2)(3.5,3.8)(3.7,3.5)(4.5,3.8)
        \psline(4.5,1.5)
   }
  \pscustom[fillcolor=gray,  fillstyle=gradient,
    gradbegin=white, gradend=gray, gradmidpoint=.5,gradangle=100]{
        \psline(2,1)(0,0)
        \psline(0,0)(0,2)
        \psline(0,2)(2,3)
        \psbezier(2,3)(3,3.5)(4,1.8)(5.5,2.5)
        \psline(5.5,2.5)(5.5,.5)
        \psbezier(5.5,.5)(4,-.2)(3,1.5)(2,1)
    }
   \psbezier[linestyle=dotted](4,.5)(4.3,.7)(4,1.2)(4.5,1.5)
   \psline[linestyle=dotted](4.5,1.5)(4.5,3.8)
   \psbezier[linestyle=dotted](2,1)(2.8,1.2)(2.6,2.6)(3.8,2.8)
   \psline[linestyle=dotted](3.8,2.8)(3.8,4.8)
   %
   \psline[linestyle=dashed](2,3)(2,1)
   \psdots[dotscale=1.5](2,1.6)
   \psline[linestyle=dashed](2,1.6)(4,.5)
   \psline[linestyle=dashed](2,1.6)(2.8,4.2)
\end{pspicture}
}

\newcommand{\YYYC}{
\begin{pspicture}[.5](5,5)
       \pscustom[fillcolor=gray,  fillstyle=gradient,
    gradbegin=white, gradend=black, gradmidpoint=0,gradangle=140]{
        \psbezier(2.8,4.2)(3.3,4.7)(3.3,4.7)(3.8,4.8)
        \psline(3.8,2)
   }
    \pscustom[fillcolor=gray,  fillstyle=gradient,
    gradbegin=white, gradend=gray, gradmidpoint=.35,gradangle=140]{
        \psbezier(2,3)(2.6,3.4)(2.4,3.8)(2.8,4.2)
        \psbezier(2.8,4.2)(3.5,3.8)(3.7,3.5)(4.5,3.8)
        \psline(4.5,1.5)
   }
  \pscustom[fillcolor=gray,  fillstyle=gradient,
    gradbegin=white, gradend=gray, gradmidpoint=.5,gradangle=100]{
        \psline(2,1)(0,0)
        \psline(0,0)(0,2)
        \psline(0,2)(2,3)
        \psbezier(2,3)(3,3.5)(4,1.8)(5.5,2.5)
        \psline(5.5,2.5)(5.5,.5)
        \psbezier(5.5,.5)(4,-.2)(3,1.5)(2,1)
    }
   \psbezier[linestyle=dotted](4,.5)(4.3,.7)(4,1.2)(4.5,1.5)
   \psline[linestyle=dotted](4.5,1.5)(4.5,3.8)
   \psbezier[linestyle=dotted](2,1)(2.8,1.2)(2.6,2.6)(3.8,2.8)
   \psline[linestyle=dotted](3.8,2.8)(3.8,4.8)
   %
   \psline[linestyle=dashed](2,3)(2,1)
   \psdots[dotscale=1.5](2,1.6)
   \psline[linestyle=dashed](2,1.6)(4,.5)
   \psline[linestyle=dashed](2,1.6)(2.8,4.2)
   %LABELS
   \rput(5.2,2){$\scriptstyle A$}
   \rput(4.2,3.5){$\scriptstyle A$}
   \rput(3.6,4.5){$\scriptstyle A$}
   \rput(2.1,2.8){$\scriptstyle A$}
   \rput(2.15,1.25){$\scriptstyle A$}
   \rput(1.8,2.4){$\scriptstyle m$}
   \rput(1.8,1.2){$\scriptstyle m$}
   \rput(3.5,1){$\scriptstyle m$}
   \rput(2.5,2.6){$\scriptstyle m$}
   \rput(1.8,1.6){$\scriptstyle \alpha$}
\end{pspicture}
}


\newcommand{\YYYD}{
\psset{unit=0.5cm}
 \begin{pspicture}[.5](2,9)
    \pscircle[fillcolor=lightgray,fillstyle=gradient,
      gradbegin=white, gradend=darkgray,gradmidpoint=0,gradangle=110](1,5){1}
    \pspolygon[fillcolor=lightgray,fillstyle=gradient,
      gradbegin=darkgray, gradend=white, gradmidpoint=1,gradangle=110](.4,6.5)(.4,1.5)(1.5,3)(1.5,8)
 \begin{psclip}{\pscustom{\psellipse[linestyle=dashed](1,5)(.3,1) \pswedge[linestyle=none](1,5){1}{-93}{93}} }
 \pscircle[fillcolor=lightgray,fillstyle=gradient,
         gradbegin=white, gradend=black,gradmidpoint=0,gradangle=130](1,5){1}
\end{psclip}
\end{pspicture}
}


\newcommand{\YYYE}{
\psset{unit=0.5cm}
  \begin{pspicture}[.5](2,9)
    \pspolygon[fillcolor=lightgray,fillstyle=gradient,
      gradbegin=darkgray, gradend=white, gradmidpoint=1,gradangle=110](.4,6.5)(.4,1.5)(1.5,3)(1.5,8)
 \end{pspicture}
}

\newcommand{\YYYF}{
\psset{unit=0.5cm}
 \begin{pspicture}[.5](5,9)
    \pspolygon[fillcolor=gray,fillstyle=gradient,gradbegin=gray, gradend=black,
            gradmidpoint=1,gradangle=290](2,8)(0,9)(0,3)(2,2)
    \pspolygon[fillstyle=gradient,gradbegin=lightgray, gradend=darkgray,
            gradmidpoint=0,gradangle=110](2,8)(2,2)(4.5,2.5)(4.5,8.5)
    \pscircle[fillcolor=lightgray,fillstyle=gradient,gradbegin=white, gradend=black,
        gradmidpoint=0,gradangle=130](2,5){1}
    \pspolygon[fillcolor=lightgray,fillstyle=gradient,
      gradbegin=darkgray, gradend=lightgray, gradmidpoint=1,gradangle=110](1.4,5)(1.4,0)(2,2)(2,8)
 \begin{psclip}{\psellipse[linestyle=dotted,dotsep=.5pt](2,5)(.25,1)}
 \pscircle[fillcolor=lightgray,fillstyle=gradient,
      gradbegin=white, gradend=black,gradmidpoint=0,gradangle=130](2,5){1}
  \end{psclip}
\end{pspicture}
}

\newcommand{\YYYG}{
\psset{unit=0.5cm}
 \begin{pspicture}[.5](5,9)
    \pspolygon[fillstyle=gradient,
      gradbegin=lightgray, gradend=black,
      gradmidpoint=1,gradangle=290](2,8)(0,9)(0,3)(2,2)
    \pspolygon[fillstyle=gradient,
      gradbegin=white, gradend=darkgray, gradmidpoint=0,gradangle=110](2,8)(2,2)(4.5,2.5)(4.5,8.5)
    \pscircle[fillstyle=gradient,
      gradbegin=white, gradend=black,gradmidpoint=0,gradangle=130](2,5){1}
 \pspolygon[fillstyle=gradient,gradbegin=darkgray, gradend=white, gradmidpoint=1,gradangle=110]
         (1.4,5)(1.4,0)(2,2)(2,8)
 \begin{psclip}{\psellipse[linestyle=dotted,dotsep=.5pt](2,5)(.25,1)}
 \pscircle[fillstyle=gradient,gradbegin=white, gradend=black,gradmidpoint=0,gradangle=130]
        (2,5){1}
\end{psclip}
\end{pspicture}
}

\newcommand{\YYYH}{
\psset{unit=0.5cm}
 \begin{pspicture}[.5](5,9)
    \pspolygon[fillstyle=gradient,
      gradbegin=lightgray, gradend=black,
      gradmidpoint=1,gradangle=290](2,8)(0,9)(0,3)(2,2)
    \pspolygon[fillstyle=gradient,
      gradbegin=white, gradend=darkgray, gradmidpoint=0,gradangle=110](2,8)(2,2)(4.5,2.5)(4.5,8.5)
    \pspolygon[fillcolor=lightgray,fillstyle=gradient,
      gradbegin=darkgray, gradend=white, gradmidpoint=1,gradangle=110](1.4,5)(1.4,0)(2,2)(2,8)
 \end{pspicture}
}

\newcommand{\YYYI}{
\psset{unit=0.5cm}
 \begin{pspicture}[.5](5,9)
    \pspolygon[fillcolor=gray,fillstyle=gradient,
      gradbegin=gray, gradend=black,
      gradmidpoint=1,gradangle=290](2,8)(0,9)(0,3)(2,2)
    \pspolygon[fillstyle=gradient,
      gradbegin=lightgray, gradend=darkgray, gradmidpoint=0,gradangle=110](2,8)(2,2)(4.5,2.5)(4.5,8.5)
    \pspolygon[fillcolor=lightgray,fillstyle=gradient,
      gradbegin=darkgray, gradend=lightgray, gradmidpoint=1,gradangle=110](1.4,5)(1.4,0)(2,2)(2,8)
 \end{pspicture}
}

\newcommand{\YYYJ}{
\begin{pspicture}[.5](5,4)
\pspolygon[fillcolor=gray,  fillstyle=gradient,
    gradbegin=white, gradend=black,
    gradmidpoint=0,gradangle=140](1.5,0)(0,2)(3.5,2)(5,0)(1.5,0)
\pspolygon[fillcolor=gray,  fillstyle=gradient,
    gradbegin=white, gradend=black,
    gradmidpoint=0,gradangle=140](2.5,1)(3.3,2)(3.3,3.5)(2.5,2.5)(2.5,1)
\pspolygon[fillcolor=gray,  fillstyle=gradient,
    gradbegin=white, gradend=black,
    gradmidpoint=0,gradangle=140](2.5,1)(.75,1)(.75,2.5)(2.5,2.5)(2.5,1)
\pspolygon[fillcolor=gray,  fillstyle=gradient,
    gradbegin=white, gradend=black,
    gradmidpoint=0,gradangle=140](2.5,2.5)(2.5,1)(3.25,0)(3.25,1.5)(2.5,2.5)
 \psdots[dotscale=1.3](2.5,1)
 \psline[linestyle=dotted](0,2)(3.5,2)
 \psline[linestyle=dotted](2.5,1)(3.3,2)
\end{pspicture}
}

\newcommand{\YYYK}{
\begin{pspicture}[.5](6,4)
\pspolygon[fillcolor=gray,  fillstyle=gradient,
    gradbegin=lightgray, gradend=black,
    gradmidpoint=0,gradangle=180](.5,0)(.5,1.5)(5,1.5)(5,0)(.5,0)
\pspolygon[fillcolor=lightgray,  fillstyle=gradient,
    gradbegin=white, gradend=black,
    gradmidpoint=0,gradangle=140](1,.75)(0,2.25)(4.5,2.25)(5.5,.75)(1,.75)
 \psline(.5,1.5)(5,1.5)
\pspolygon[fillcolor=gray,  fillstyle=gradient,
    gradbegin=white, gradend=black,
    gradmidpoint=0,gradangle=110](2.5,2.25)(3.5,.75)(3.5,2.25)(2.5,3.75)(2.5,2.25)
 \psline[linestyle=dotted](.5,1.5)(5,1.5)
 \psline[linestyle=dotted](0,2.25)(3.5,2.25)
 \psdots[dotscale=1.3](3,1.5)
\end{pspicture}
}

\newcommand{\FAKEi}{
\psset{unit=0.5cm}
\begin{pspicture}[.5](5,3)
\pspolygon[fillcolor=gray,  fillstyle=gradient,
    gradbegin=white, gradend=darkgray,
    gradmidpoint=0,gradangle=140](1.5,1.5)(0,3.5)(3.5,3.5)(5,1.5)(1.5,1.5)
    \psdots[dotscale=.8](2.5,2.5)
\end{pspicture}
}

\newcommand{\FAKEiii}{
\psset{unit=0.5cm}
\begin{pspicture}[.5](6,4)
\pspolygon[fillcolor=gray,  fillstyle=gradient,
    gradbegin=lightgray, gradend=black,
    gradmidpoint=0,gradangle=180](.5,0)(.5,1.5)(5,1.5)(5,0)(.5,0)
\pspolygon[fillcolor=lightgray,  fillstyle=gradient,
    gradbegin=white, gradend=black,
    gradmidpoint=0,gradangle=140](1,.75)(0,2.25)(4.5,2.25)(5.5,.75)(1,.75)
 \psline(.5,1.5)(5,1.5)
\pspolygon[fillcolor=gray,  fillstyle=gradient,
    gradbegin=white, gradend=black,
    gradmidpoint=0,gradangle=110](2.5,2.25)(3.5,.75)(3.5,2.25)(2.5,3.75)(2.5,2.25)
 \psline[linestyle=dotted](.5,1.5)(5,1.5)
 \psline[linestyle=dotted](0,2.25)(3.5,2.25)
 \psdots[dotscale=.8](3,1.5)
\end{pspicture}
}


\newcommand{\YYYL}{
%\psset{gridcolor=green, subgridcolor=yellow} \psgrid
\begin{pspicture}[.5](5,6)
\pspolygon[fillcolor=darkgray,
fillstyle=solid](2.5,0)(2.5,1.5)(1,3.5)(1,2)(2.5,0)
\pspolygon[fillcolor=gray,  fillstyle=gradient,
    gradbegin=white, gradend=darkgray,
    gradmidpoint=0,gradangle=140](1.5,1.5)(0,3.5)(3.5,3.5)(5,1.5)(1.5,1.5)
\psline(2.5,1.5)(1,3.5) \pspolygon[fillcolor=gray,
fillstyle=gradient,
    gradbegin=white, gradend=black,
    gradmidpoint=0,gradangle=140](2.5,2.5)(3.3,3.5)(3.3,5)(2.5,4)(2.5,3.5)
\pspolygon[fillcolor=gray,  fillstyle=gradient,
    gradbegin=white, gradend=black,
    gradmidpoint=0,gradangle=140](2.5,2.5)(.75,2.5)(.75,4)(2.5,4)(2.5,2.5)
\pspolygon[fillcolor=gray,  fillstyle=gradient,
    gradbegin=white, gradend=black,
    gradmidpoint=0,gradangle=140](2.5,4)(2.5,2.5)(4.25,1.5)(4.25,3)(2.5,4)
 \psdots[dotscale=1](2.5,2.5)
 \psdots[dotscale=1](1.75,2.5)
 \psline[linestyle=dotted](0,3.5)(3.5,3.5)
 \psline[linestyle=dotted](2.5,2.5)(3.3,3.5)
 \psline[linestyle=dotted](2.5,1.5)(1,3.5)
 \psline[linestyle=dotted](5,1.5)(3.5,3.5)
\end{pspicture}
}


\newcommand{\YYYM}{
%\psset{gridcolor=green, subgridcolor=yellow} \psgrid
\begin{pspicture}[.5](5,6)
\pspolygon[fillcolor=darkgray,
fillstyle=solid](3.75,0)(3.75,1.5)(2.75,3.5)(2.75,2)(3.75,0)
\pspolygon[fillcolor=gray,  fillstyle=gradient,
    gradbegin=white, gradend=darkgray,
    gradmidpoint=0,gradangle=140](1.5,1.5)(0,3.5)(3.5,3.5)(5,1.5)(1.5,1.5)
 \psline(3.75,1.5)(2.75,3.5)
\pspolygon[fillcolor=gray,  fillstyle=gradient,
    gradbegin=white, gradend=black,
    gradmidpoint=0,gradangle=140](2.5,2.5)(3.3,3.5)(3.3,5)(2.5,4)(2.5,3.5)
\pspolygon[fillcolor=gray,  fillstyle=gradient,
    gradbegin=white, gradend=black,
    gradmidpoint=0,gradangle=140](2.5,2.5)(.75,2.5)(.75,4)(2.5,4)(2.5,2.5)
\pspolygon[fillcolor=gray,  fillstyle=gradient,
    gradbegin=white, gradend=black,
    gradmidpoint=0,gradangle=140](2.5,4)(2.5,2.5)(4.25,1.5)(4.25,3)(2.5,4)
 \psdots[dotscale=1](2.5,2.5)
 \psdots[dotscale=1](3.5,1.95)
 \psdots[dotscale=1](2.95,3.05)
 \psline[linestyle=dotted](0,3.5)(3.5,3.5)
 \psline[linestyle=dotted](2.5,2.5)(3.3,3.5)
 \psline[linestyle=dotted](3.75,1.5)(2.75,3.5)
 \psline[linestyle=dotted](5,1.5)(3.5,3.5)
\end{pspicture}
}


\newcommand{\YYYN}{
%\psset{gridcolor=green, subgridcolor=yellow} \psgrid
\begin{pspicture}[.5](5,5)
\pspolygon[fillcolor=darkgray,
fillstyle=solid](2.5,.5)(2.5,1.5)(4,3.5)(4,2.5)(2.5,.5)
\pspolygon[fillcolor=gray,  fillstyle=gradient,
    gradbegin=white, gradend=darkgray,
    gradmidpoint=0,gradangle=140](0,1.5)(1.5,3.5)(5,3.5)(3.5,1.5)(0,1.5)
\pspolygon[fillcolor=gray,  fillstyle=gradient,
    gradbegin=white, gradend=darkgray,
    gradmidpoint=0,gradangle=140](1.75,1.5)(3.25,3.5)(3.25,4.5)(1.75,2.5)(1.75,1.5)
 \psline(2.5,1.5)(4,3.5)
 \psline[linestyle=dotted](4,3.5)(4,2.5)
  \psline[linestyle=dotted](2.5,.5)(4,2.5)
\end{pspicture}
}


\newcommand{\YYYO}{
%\psset{gridcolor=green, subgridcolor=yellow} \psgrid
\begin{pspicture}[.5](5,5)
 \psline(2.5,1.5)(4,3.5)
\pspolygon[fillcolor=darkgray,
fillstyle=solid](2.5,.5)(2.5,1.5)(4,3.5)(4,2.5)(2.5,.5)
\pspolygon[fillcolor=gray,  fillstyle=gradient,
    gradbegin=white, gradend=darkgray,
    gradmidpoint=0,gradangle=140](0,1.5)(1.5,3.5)(5,3.5)(3.5,1.5)(0,1.5)
\psline(2.5,1.5)(4,3.5) \pscustom[fillcolor=gray,
fillstyle=gradient,
    gradbegin=white, gradend=darkgray,
    gradmidpoint=.4,gradangle=150]{
 \psbezier(1.75,1.5)(2.2,2)(2.2,2.2)(2.8,2.2)
 \psbezier(2.8,2.2)(4,2.2)(4,2.5)(3.3,2.8)
 \psbezier(3.3,2.8)(3,2.9)(3,3.2)(3.25,3.5)
 \psline(3.25,4.5)
 \psline(1.75,2.5)
 \psline(1.75,1.5)
 }
 \psline[linestyle=dotted](4,3.5)(4,2.5)
  \psline[linestyle=dotted](2.5,.5)(4,2.5)
   \psline[linestyle=dotted](2.5,1.5)(4,3.5)
  \psdots(3,2.2)(3.4,2.75)
\end{pspicture}
}


\newcommand{\YYYP}{
%\psset{gridcolor=green, subgridcolor=yellow} \psgrid
\begin{pspicture}[.6](5,4)
\pspolygon[fillcolor=gray,  fillstyle=gradient,
    gradbegin=white, gradend=darkgray,
    gradmidpoint=0,gradangle=140](1.5,1.5)(0,3.5)(3.5,3.5)(5,1.5)(1.5,1.5)
\end{pspicture}
}

\newcommand{\YYYPii}{
\begin{pspicture}[.6](5,4)
\pspolygon[fillcolor=gray,  fillstyle=gradient,
    gradbegin=white, gradend=darkgray,
    gradmidpoint=0,gradangle=140](1.5,1.5)(0,3.5)(3.5,3.5)(5,1.5)(1.5,1.5)
 \begin{psclip}{
    \pscustom{
    \psellipse(2.5,2.65)(.55,.3)
    \pswedge(2.5,2.65){.55}{0}{180}  }
    }
 \pscircle[fillcolor=lightgray,fillstyle=gradient,
      gradbegin=white, gradend=black,gradmidpoint=0,gradangle=130](2.5,2.65){.55}
\end{psclip}
\psellipse[linestyle=dotted](2.5,2.65)(.55,.3)
\end{pspicture}
}

\newcommand{\YYYQ}{
\psset{unit=0.5cm}
%\psset{gridcolor=green, subgridcolor=yellow} \psgrid
\begin{pspicture}[.6](4.5,5)
 \psline[linestyle=dashed](0,3)(4,3.5)
 \psline(0,3)(3,2)
 \psline(4,3.5)(3,2)
 \psline(0,3)(2,5)
 \psline(2,5)(3,2)
 \psline(2,5)(4,3.5)
\end{pspicture}
\quad \xy {\ar@{<->} (0,0);(6,0)}; \endxy \quad
%\psset{gridcolor=green, subgridcolor=yellow} \psgrid
\begin{pspicture}[.6](4.5,5)
\pscustom[fillstyle=gradient,
    gradbegin=white, gradend=lightgray,
    gradmidpoint=.5,gradangle=140]{
  \psbezier(0,3)(1,3.8)(3,4.3)(4,3.5)
   \psline(3,2)
   \psline(0,3)
  }
  \pscustom[fillstyle=gradient,
    gradbegin=white, gradend=gray,
    gradmidpoint=.5,gradangle=140]{
  \psbezier(0,3)(1.3,.8)(3.7,.8)(4,3.5)
   \psline(3,2)
   \psline(0,3)
  }
  %
 \psline[linestyle=dashed](0,3)(4,3.5)
 %
 \psline(0,3)(2,5)
 \psline(2,5)(3,2)
 \psline(2,5)(4,3.5)
 %
 \psline[linestyle=dashed](2.25,2.75)(0,3)
 \psline[linestyle=dashed](2.25,2.75)(4,3.5)
 \psline[linestyle=dashed](2.25,2.75)(3,2)
 %
 \psdots(2.25,2.75)
\end{pspicture}
\quad \xy {\ar@{<->} (0,0);(6,0)}; \endxy \quad
%\psset{gridcolor=green, subgridcolor=yellow} \psgrid
\begin{pspicture}[.6](4.5,5)
 \psline[linestyle=dashed](0,3)(4,3.5)
 \psline(0,3)(3,2)
 \psline(4,3.5)(3,2)
 \psline(0,3)(2,5)
 \psline(2,5)(3,2)
 \psline(2,5)(4,3.5)
  \psline(1.9,3.6)(0,3)
  \psline(1.9,3.6)(2,5)
 \psline(1.9,3.6)(4,3.5)
 \psline(1.9,3.6)(3,2)
 %
 \psdots(1.9,3.6)
\end{pspicture}
}

\newcommand{\YYYR}{
%\psset{gridcolor=green, subgridcolor=yellow} \psgrid
\begin{pspicture}[.5](3,3)
  \begin{psclip}{\pswedge[linestyle=none](1.5,1.5){1.25}{180}{0}}
          \psellipse(1.5,1.5)(1.25,.5)
  \end{psclip}
     \psellipse[linestyle=dotted](1.5,1.5)(1.25,.5)
     \pscircle(1.5,1.5){1.25}
     \psdots(.8,1.1)(2.2,1.1)(1.6,2)(1.5,1.5)
     \psline[linestyle=dotted](1.5,1.5)(.8,1.1)
     \psline[linestyle=dotted](1.5,1.5)(2.2,1.1)
     \psline[linestyle=dotted](1.5,1.5)(1.6,2)
\end{pspicture}
}

\newcommand{\YYYS}{
%\psset{gridcolor=green, subgridcolor=yellow} \psgrid
\begin{pspicture}[.5](3,3)
   \pscustom[fillcolor=lightgray,fillstyle=gradient,
      gradbegin=gray, gradend=white, gradmidpoint=1,gradangle=110]{
   \psarc(1.5,1.5){1.25}{180}{0}
   \psellipse(1.5,1.5)(1.25,.5)}
\pscustom[fillcolor=lightgray,fillstyle=gradient,
      gradbegin=lightgray, gradend=white, gradmidpoint=1,gradangle=110]{
   \psarc(1.5,1.5){1.25}{0}{180}
   \psellipse(1.5,1.5)(1.25,.5)}
  \begin{psclip}{\pswedge[linestyle=none](1.5,1.5){1.25}{180}{0}}
          \psellipse(1.5,1.5)(1.25,.5)
  \end{psclip}
     \psellipse[linestyle=dotted](1.5,1.5)(1.25,.5)
     \pscircle(1.5,1.5){1.25}
     \psdots(.8,1.1)(2.2,1.1)(1.6,2)(1.5,1.5)
     \psline[linestyle=dotted](1.5,1.5)(.8,1.1)
     \psline[linestyle=dotted](1.5,1.5)(2.2,1.1)
     \psline[linestyle=dotted](1.5,1.5)(1.6,2)
\end{pspicture}
}


\newcommand{\DiscCircle}{
 \xy
  (-15,0)*+{\begin{pspicture}(.6,.6)
            \pscircle(.3,.3){.3}
            \rput(0,.8){$S^{1}$}        %%%%LABEL FOR CIRCLE
            \end{pspicture}}="l";
  (15,0)*+{\begin{pspicture}(.6,.6)
        \pscircle(.3,.3){.3}
        \rput(.7,.8){$S^{1}$}         %%%%LABEL FOR CIRCLE
        \end{pspicture}}="r";
  (0,15)*{\begin{pspicture}(.6,.6)
            \pscircle[linewidth=0.3pt,fillcolor=lightgray,fillstyle=solid](.3,.3){.3}
            \rput(0,.8){$D^{2}$}       %%%%LABEL FOR CIRCLE
            \end{pspicture}}="t";
  {\ar^{i} "l";"t"};
   {\ar^{r} "t";"r"};
    {\ar_{1_{S^{1}}} "l";"r"};
  (0,22)*{};  %%%% Artificial vertical spacing
 \endxy
 }

 \newcommand{\DiscCircleII}{
  \xy
  (-20,0)*+{\begin{pspicture}(1.3,1.2)
            \pscircle(.6,.6){.6}
            \rput(0,1.3){$S^{1}$}        %%%%LABEL FOR CIRCLE
            \end{pspicture}}="l";
  (20,0)*+{\begin{pspicture}(1.3,1.2)
        \pscircle(.6,.6){.6}
        \rput(1.3,1.3){$S^{1}$}         %%%%LABEL FOR CIRCLE
        \end{pspicture}}="r";
  (0,20)*{\begin{pspicture}(1.3,1.3)
            \psellipse[linewidth=0.3pt, fillcolor=lightgray,fillstyle=solid](.6,.6)(.6,.3)
            \rput(0,1.3){$D^{2}$}       %%%%LABEL FOR CIRCLE
            \end{pspicture}}="t";
  {\ar^{i} "l";"t"};
   {\ar^{r} "t";"r"};
    {\ar_{1_{S^{1}}} "l";"r"};
 \endxy
 }

\newcommand{\DDDI}{
%\psset{gridcolor=green, subgridcolor=yellow} \psgrid
\psset{linewidth=0.3pt,dimen=middle,
linearc=.05pt,cornersize=absolute}
\begin{pspicture}(3.5,3.5)
%%%%
     \psline[linestyle=dashed,dash=3pt 2.5pt,linewidth=0.4pt]{c-c}(0,.5)(4,2)
%%%%
     \pspolygon[linestyle=solid,linecolor=black, fillstyle=solid,fillcolor=lightgray]
        (1.9,1.9)(2.8,2.55)(2.9,1.9)(2,1.6)(1.9,1.9)
     \pspolygon[linestyle=solid,linecolor=black, fillstyle=solid,fillcolor=gray]
        (2,.8)(2,1.5)(2.9,1.9)(3.5,1)(2,.8)
     \pspolygon[linestyle=solid,linecolor=black, fillstyle=solid,fillcolor=lightgray]
        (1.6,1.4)(1,1.9)(1.9,1.9)(2,1.6)(1.6,1.4)
     \pspolygon[linestyle=solid,linecolor=black, fillstyle=solid,fillcolor=gray]
        (1.6,1.4)(1.4,.26)(2,.8)(2,1.5)(1.6,1.4)
     \pspolygon[linestyle=solid,linecolor=black, fillstyle=solid,fillcolor=darkgray]
        (2.2,1.9)(1.6,1.4)(2,1.5)(2.9,1.9)(2.25,1.9)
%%%%
     %% draws the outer tetrahedron
     \psline[linestyle=dashed,dash=3pt 2pt, linewidth=0.4pt]{c-c}(0,.5)(3,0)
     \psline[linestyle=dashed, dash=3pt 2pt,linewidth=0.4pt]{c-c}(3,0)(4,2)
     \psline[linestyle=dashed,dash=3pt 2pt,linewidth=0.4pt]{c-c}(4,2)(1.75,3)
     \psline[linestyle=dashed,dash=3pt 2pt,linewidth=0.4pt]{c-c}(3,0)(1.75,3)
     \psline[linestyle=dashed,dash=3pt 2pt,linewidth=0.4pt]{c-c}(0,.5)(1.75,3)
\end{pspicture}
}

\newcommand{\DDDII}{
\psset{linewidth=0.3pt,dimen=middle,
linearc=.05pt,cornersize=absolute}
\begin{pspicture}(3.5,3.5)
%%%%
     \psline[linestyle=dashed,dash=3pt 2.5pt,linewidth=0.4pt]{c-c}(0,.5)(4,2)
%%%%
     \pspolygon[linestyle=solid,linecolor=black, fillstyle=solid,fillcolor=lightgray]
        (1.9,1.9)(2.8,2.55)(2.9,1.9)(2,1.6)(1.9,1.9)
     \pspolygon[linestyle=solid,linecolor=black, fillstyle=solid,fillcolor=gray]
        (2,.8)(2,1.5)(2.9,1.9)(3.5,1)(2,.8)
     \pspolygon[linestyle=solid,linecolor=black, fillstyle=solid,fillcolor=lightgray]
        (1.6,1.4)(1,1.9)(1.9,1.9)(2,1.6)(1.6,1.4)
     \pspolygon[linestyle=solid,linecolor=black, fillstyle=solid,fillcolor=gray]
        (1.6,1.4)(1.4,.26)(2,.8)(2,1.5)(1.6,1.4)
     \pspolygon[linestyle=solid,linecolor=black, fillstyle=solid,fillcolor=darkgray]
        (2.2,1.9)(1.6,1.4)(2,1.5)(2.9,1.9)(2.2,1.9)
%%%%
     %% draws the outer tetrahedron
     \psline[linestyle=dashed,dash=3pt 2pt, linewidth=0.4pt]{c-c}(0,.5)(3,0)
     \psline[linestyle=dashed, dash=3pt 2pt,linewidth=0.4pt]{c-c}(3,0)(4,2)
     \psline[linestyle=dashed,dash=3pt 2pt,linewidth=0.4pt]{c-c}(4,2)(1.75,3)
     \psline[linestyle=dashed,dash=3pt 2pt,linewidth=0.4pt]{c-c}(3,0)(1.75,3)
     \psline[linestyle=dashed,dash=3pt 2pt,linewidth=0.4pt]{c-c}(0,.5)(1.75,3)
%%%%
     %%DRAWS ASSOCIATOR
     \psline[linestyle=solid,linecolor=red, linewidth=1pt]{c-c}(1.4,.26)(1.6,1.4)
     \psline[linestyle=solid,linecolor=red, linewidth=1pt]{c-c}(1.6,1.4)(1,1.9)
     \psline[linestyle=solid,linecolor=red, linewidth=1pt]{c-c}(2.2,1.9)(1.6,1.4)
     \psline[linestyle=solid,linecolor=red, linewidth=1pt]{c-c}(2.9,1.9)(2.2,1.9)
     \psline[linestyle=solid,linecolor=red, linewidth=1pt]{c-c}(2.8,2.55)(2.9,1.9)
     \psline[linestyle=solid,linecolor=red, linewidth=1pt]{c-c}(2.9,1.9)(3.5,1)
\end{pspicture}
\qquad \qquad
%\psset{gridcolor=green, subgridcolor=yellow} \psgrid
\psset{linewidth=0.3pt,dimen=middle,
linearc=.05pt,cornersize=absolute}
\begin{pspicture}(3.5,3.5)
%%%%
     \psline[linestyle=dashed,dash=3pt 2.5pt,linewidth=0.4pt]{c-c}(0,.5)(4,2)
%%%%
     \pspolygon[linestyle=solid,linecolor=black, fillstyle=solid,fillcolor=lightgray]
        (1.9,1.9)(2.8,2.55)(2.9,1.9)(2,1.6)(1.9,1.9)
     \pspolygon[linestyle=solid,linecolor=black, fillstyle=solid,fillcolor=gray]
        (2,.8)(2,1.5)(2.9,1.9)(3.5,1)(2,.8)
     \pspolygon[linestyle=solid,linecolor=black, fillstyle=solid,fillcolor=lightgray]
        (1.6,1.4)(1,1.9)(1.9,1.9)(2,1.6)(1.6,1.4)
     \pspolygon[linestyle=solid,linecolor=black, fillstyle=solid,fillcolor=gray]
        (1.6,1.4)(1.4,.26)(2,.8)(2,1.5)(1.6,1.4)
     %%
     \psline[linestyle=solid, linecolor=red, linewidth=1pt]{c-c}(1.9,1.9)(2,1.6)
     %%
     \pspolygon[linestyle=solid,linecolor=black, fillstyle=solid,fillcolor=darkgray]
        (2.2,1.9)(1.6,1.4)(2,1.5)(2.9,1.9)(2.2,1.9)
%%%%
     %% draws the outer tetrahedron
     \psline[linestyle=dashed,dash=3pt 2pt, linewidth=0.4pt]{c-c}(0,.5)(3,0)
     \psline[linestyle=dashed, dash=3pt 2pt,linewidth=0.4pt]{c-c}(3,0)(4,2)
     \psline[linestyle=dashed,dash=3pt 2pt,linewidth=0.4pt]{c-c}(4,2)(1.75,3)
     \psline[linestyle=dashed,dash=3pt 2pt,linewidth=0.4pt]{c-c}(3,0)(1.75,3)
     \psline[linestyle=dashed,dash=3pt 2pt,linewidth=0.4pt]{c-c}(0,.5)(1.75,3)
%%%%
     %%DRAWS ASSOCIATOR
     \psline[linestyle=solid,linecolor=red, linewidth=1pt]{c-c}(1,1.9)(1.9,1.9)
     %\psline[linestyle=dotted, dotsep=2pt, linecolor=red, linewidth=1pt]{c-c}(1.9,1.9)(2,1.6)
     \psline[linestyle=solid,linecolor=red, linewidth=1pt]{c-c}(1.9,1.9)(2.8,2.55)
     \psline[linestyle=solid,linecolor=red, linewidth=1pt]{c-c}(2,.8)(2,1.47)
     \psline[linestyle=solid,linecolor=red, linewidth=1pt]{c-c}(3.5,1)(2,.8)
     \psline[linestyle=solid,linecolor=red, linewidth=1pt]{c-c}(1.4,.26)(2,.8)
\end{pspicture}
}

\newcommand{\PdualTetra}{
%\psset{gridcolor=green, subgridcolor=yellow} \psgrid
\psset{linewidth=0.3pt,dimen=middle,
linearc=.05pt,cornersize=absolute}
\begin{pspicture}(3.5,3.5)
%%%%
     \psline[linestyle=dashed,dash=3pt 2.5pt,linewidth=0.4pt]{c-c}(0,.5)(4,2)
%%%%
    %%Draws inner tetrahedron
     \pspolygon[linestyle=solid,linecolor=black, fillstyle=gradient,gradbegin=lightgray,
        gradend=gray,gradmidpoint=0,gradangle=90](1.6,1.4)(2.9,1.9)(2.4,.8)(1.6,1.4)
     \pspolygon[linestyle=solid,linecolor=black, fillstyle=gradient,gradbegin=lightgray,
        gradend=white,gradmidpoint=0,gradangle=280](1.6,1.4)(1.9,1.9)(2.9,1.9)(1.6,1.4)
    %% draws the dotted line for inner tetrahedron
    \psline[linestyle=dotted,dotsep=2pt,linecolor=black](1.9,1.9)(2.4,.8)
%%%%
     %% draws the outer tetrahedron
     \psline[linestyle=dashed,dash=3pt 2pt, linewidth=0.4pt]{c-c}(0,.5)(3,0)
     \psline[linestyle=dashed, dash=3pt 2pt,linewidth=0.4pt]{c-c}(3,0)(4,2)
     \psline[linestyle=dashed,dash=3pt 2pt,linewidth=0.4pt]{c-c}(4,2)(1.75,3)
     \psline[linestyle=dashed,dash=3pt 2pt,linewidth=0.4pt]{c-c}(3,0)(1.75,3)
     \psline[linestyle=dashed,dash=3pt 2pt,linewidth=0.4pt]{c-c}(0,.5)(1.75,3)
%%%%
    %\psdots(1.6,1.4)(1.9,1.9)(2.9,1.9)(2.4,.8)
%%%%
    \psline[linestyle=dotted, dotsep=1.5pt,linewidth=0.2pt](2.9,1.9)(1.75,3)
    \psline[linestyle=dotted, dotsep=1.5pt,linewidth=0.2pt](2.9,1.9)(3,0)
    \psline[linestyle=dotted, dotsep=1.5pt,linewidth=0.2pt](2.9,1.9)(4,2)
    %%
    \psline[linestyle=dotted, dotsep=1.5pt,linewidth=0.2pt](1.6,1.4)(1.75,3)
    \psline[linestyle=dotted, dotsep=1.5pt,linewidth=0.2pt](1.6,1.4)(3,0)
    \psline[linestyle=dotted, dotsep=1.5pt,linewidth=0.2pt](1.6,1.4)(0,.5)
    %%
    \psline[linestyle=dotted, dotsep=5pt, linewidth=0.2pt](2.4,.8)(4,2)
    \psline[linestyle=dotted, dotsep=5pt,linewidth=0.2pt](2.4,.8)(3,0)
    \psline[linestyle=dotted, dotsep=5pt,linewidth=0.2pt](2.4,.8)(0,.5)
    %%
    \psline[linestyle=dotted, dotsep=5pt, linewidth=0.2pt](1.9,1.9)(4,2)
    \psline[linestyle=dotted, dotsep=5pt,linewidth=0.2pt](1.9,1.9)(1.75,3)
    \psline[linestyle=dotted, dotsep=5pt,linewidth=0.2pt](1.9,1.9)(0,.5)
    %%
\end{pspicture}
}


%############################# BUILDING BLOCKS (INTERVALS) ##########################
%####################################################################################

\newcommand{\multl}{
  \pscustom[fillcolor=lightgray, fillstyle=solid]{
        \psbezier(1.5,2.5)(1.5,1.1)(.4,1.6)(.5,0)
        \psline(-0.5,0)
        \psbezier(-0.5,0)(-.4,1.6)(-1.5,1.1)(-1.5,2.5)
        \psline(-.5,2.5)
        \psbezier(-.5,2.5)(-.6,1.5)(0.6,1.5)(.5,2.5)
        \psline(1.5,2.5)
    }
}

\newcommand{\comultl}{
  \pscustom[fillcolor=lightgray, fillstyle=solid]{
        \psbezier(1.5,0)(1.5,1.4)(.4,.9)(.5,2.5)
        \psline(-0.5,2.5)
        \psbezier(-0.5,2.5)(-.4,.9)(-1.5,1.4)(-1.5,0)
        \psline(-.5,0)
        \psbezier(-.5,0)(-.6,1)(0.6,1)(.5,0)
        \psline(1.5,0)
    }
}


\newcommand{\ctl}{
  \begin{psclip}{
    \pscustom{
        \psline(-.58,2)(-.58,0)
        \psline(-.58,0)(.42,0)
        \psline(.42,0)(.42,2)
        \psellipse(-.08,2)(.5,.2)
    }
  }
    \pspolygon[fillcolor=lightgray,fillstyle=gradient,
    gradbegin=lightgray,gradend=gray,gradmidpoint=1,gradangle=110](-.58,0)(-.58,2.4)(.42,2.4)(.42,0)(-.58,0)
 \end{psclip}
 \pscustom[fillcolor=lightgray,fillstyle=gradient,
        gradbegin=white, gradend=gray,gradmidpoint=0,gradangle=88]{
    \psline(-.58,2)(-.58,0)
    \psbezier(-.58,0)(-.48,.5)(-.48,.7)(-.08,1)
    \psbezier(-.08,1)(.32,.7)(.32,.5)(.42,0)
    \psellipse(-.08,2)(.5,.2)
 }
 \psellipse[fillcolor=lightgray,fillstyle=gradient,
        gradbegin=lightgray, gradend=gray,gradmidpoint=1,gradangle=110](-.08,2)(.5,.2)
}

\newcommand{\ltc}{
    \pspolygon[fillcolor=lightgray,fillstyle=gradient,
    gradbegin=lightgray,gradend=gray,gradmidpoint=1,gradangle=60](.58,2)(.58,.4)(-.42,.4)(-.42,2)(.58,2)
 \pscustom[fillcolor=lightgray,fillstyle=gradient,
        gradbegin=white, gradend=gray,gradmidpoint=0,gradangle=88]{
    \psline(.58,0)(.58,2)
    \psbezier(.58,2)(.48,1.5)(.48,1.3)(.08,1)
    \psbezier(.08,1)(-.32,1.3)(-.32,1.5)(-.42,2)
    \psline(-.42,0)
    \psbezier(-.42,0)(-.32,-.25)(.48,-.25)(.58,0)
 }
 \begin{psclip}{
 \pspolygon[linestyle=none](.58,0)(.58,.3)(-.42,.3)(-.42,0)(.58,0)
 }
 \psellipse[linestyle=dotted](.08,0)(.5,0.2)
 \end{psclip}
}

 \newcommand{\birthl}{
 \pscustom[fillcolor=lightgray, fillstyle=solid]{
        \psbezier(-.5,0)(-.5,.9)(0.5,.9)(.5,0)
        \psline(-.5,0)
    }
 }

  \newcommand{\deathl}{
 \pscustom[fillcolor=lightgray, fillstyle=solid]{
        \psbezier(-.5,0)(-.5,-.9)(0.5,-.9)(.5,0)
        \psline(-.5,0)    }
 }


%############################# BUILDING BLOCKS (CIRCLES) ############################
%####################################################################################

\newcommand{\multc}{
      \pscustom[fillstyle=gradient,
    gradbegin=white, gradend=gray,gradmidpoint=0,gradangle=70]{
        \psbezier(1.5,2.5)(1.5,1.1)(.4,1.6)(.5,0)
        \psbezier(.5,0)(.4,-.25)(-.4,-.25)(-.5,0)
        \psbezier(-0.5,0)(-.4,1.6)(-1.5,1.1)(-1.5,2.5)
        \psline(-.5,2.5)
        \psbezier(-.5,2.5)(-.6,1.5)(0.6,1.5)(.5,2.5)
        \psline(1.5,2.5)
    }
 %% TOP ELLIPSES
    \psellipse[fillcolor=lightgray,fillstyle=gradient,
        gradbegin=lightgray, gradend=gray,gradmidpoint=1,gradangle=110](-1,2.5)(.5,.2)
    \psellipse[fillcolor=lightgray,fillstyle=gradient,
        gradbegin=lightgray, gradend=gray,gradmidpoint=1,gradangle=110](1,2.5)(.5,.2)
 %% DOTTED PART
     \begin{psclip}{
 \pspolygon[linestyle=none](.5,0)(.5,.3)(-.5,.3)(-.5,0)(.5,0)
 }
 \psellipse[linestyle=dotted](0,0)(.5,0.2)
 \end{psclip}
 }

\newcommand{\comultc}{
  \pscustom[fillstyle=gradient,
    gradbegin=white, gradend=gray,gradmidpoint=0,gradangle=110]{
        \psbezier(1.5,0)(1.5,1.4)(.4,.9)(.5,2.5)
        \psline(-0.5,2.5)
        \psbezier(-0.5,2.5)(-.4,.9)(-1.5,1.4)(-1.5,0)
        \psbezier(-1.5,0)(-1.4,-.25)(-.6,-.25)(-.5,0)
        \psbezier(-.5,0)(-.6,1)(0.6,1)(.5,0)
        \psbezier(.5,0)(.6,-.25)(1.4,-.25)(1.5,0)
    }
%% TOP ELLIPSE
  \psellipse[fillcolor=lightgray,fillstyle=gradient,
        gradbegin=lightgray, gradend=gray,gradmidpoint=1,gradangle=110](0,2.5)(.5,.2)
%% DOTTED ELLIPSE
\begin{psclip}{
 \pspolygon[linestyle=none](1.5,0)(1.5,.3)(-1.5,.3)(-1.5,0)(1.5,0)
 }
 \psellipse[linestyle=dotted](1,0)(.5,0.2)
 \psellipse[linestyle=dotted](-1,0)(.5,0.2)
 \end{psclip}
 }


\newcommand{\birthc}{
 \pscustom[fillstyle=gradient,
    gradbegin=white, gradend=gray,gradmidpoint=0,gradangle=110]{
        \psbezier(-.5,0)(-.5,.9)(0.5,.9)(.5,0)
        \psbezier(.5,0)(.4,-.25)(-.4,-.25)(-.5,0)
    }
 %%%%
 \begin{psclip}{
 \pspolygon[linestyle=none](.5,0)(.5,.3)(-.5,.3)(-.5,0)(.5,0)
 }
 \psellipse[linestyle=dotted](0,0)(.5,0.2)
 \end{psclip}
 }

\newcommand{\deathc}{
 \pscustom[fillstyle=gradient,
    gradbegin=white, gradend=gray,gradmidpoint=0,gradangle=70]{
        \psbezier(-.5,1)(-.5,.1)(0.5,.1)(.5,1)
        \psline(-.5,1)
 }
  \psellipse[fillcolor=lightgray,fillstyle=gradient,
        gradbegin=lightgray, gradend=gray,gradmidpoint=1,gradangle=110](0,1)(.5,.2)
 }


\newcommand{\zagc}{
   \pscustom[fillstyle=gradient,
    gradbegin=white, gradend=gray,gradmidpoint=0,gradangle=110]{
        \psbezier(1.5,0)(1.6,2)(-1.6,2)(-1.5,0)
        \psbezier(-1.5,0)(-1.4,-.25)(-.6,-.25)(-.5,0)
        \psbezier(-.5,0)(-.6,.8)(0.6,.8)(.5,0)
        \psbezier(.5,0)(.6,-.25)(1.4,-.25)(1.5,0)
    }
 %%%%
  \begin{psclip}{
 \pspolygon[linestyle=none](1.5,0)(1.5,.3)(-1.5,.3)(-1.5,0)(1.5,0)
 }
 \psellipse[linestyle=dotted](1,0)(.5,0.2)
 \psellipse[linestyle=dotted](-1,0)(.5,0.2)
 \end{psclip}
 }

\newcommand{\zigc}{
       \pscustom[fillstyle=gradient,
    gradbegin=white, gradend=gray,gradmidpoint=0,gradangle=70]{
        \psbezier(1.5,2)(1.6,0)(-1.6,0)(-1.5,2)
        \psline(-.5,2)
        \psbezier(-.5,2)(-.6,1.2)(0.6,1.2)(.5,2)
        \psline(1.5,2)
    }
 %%%%
 \psellipse[fillcolor=lightgray,fillstyle=gradient,
        gradbegin=lightgray, gradend=gray,gradmidpoint=1,gradangle=110](1,2)(.5,.2)
        \psellipse[fillcolor=lightgray,fillstyle=gradient,
        gradbegin=lightgray, gradend=gray,gradmidpoint=1,gradangle=110](-1,2)(.5,.2)
}

\newcommand{\identc}{
 \pscustom[fillcolor=lightgray,fillstyle=gradient,
        gradbegin=white, gradend=gray,gradmidpoint=0,gradangle=88]{
 \psline(.5,0)(.5,2.5)
 \psline(-.5,2.5)
 \psline(-.5,0)
 \psbezier(-.5,0)(-.4,-.25)(.4,-.25)(.5,0)
 }
\psellipse[fillcolor=lightgray,fillstyle=gradient,
        gradbegin=lightgray, gradend=gray,gradmidpoint=1,gradangle=110](0,2.5)(.5,.2)
 \begin{psclip}{
 \pspolygon[linestyle=none](.5,0)(.5,.3)(-.5,.3)(-.5,0)(.5,0)
 }
 \psellipse[linestyle=dotted](0,0)(.5,0.2)
 \end{psclip}
 }

  \newcommand{\medidentc}{
     \pscustom[fillcolor=lightgray,fillstyle=gradient,
        gradbegin=white, gradend=gray,gradmidpoint=0,gradangle=88]{
        \psline(-.5,2)(-.5,0)
        \psbezier(-.5,0)(-.4,-.25)(.4,-.25)(.5,0)
        \psline(.5,2)
        \psline(-.5,2)
    }
\psellipse[fillcolor=lightgray,fillstyle=gradient,
        gradbegin=lightgray, gradend=gray,gradmidpoint=1,gradangle=110](0,2)(.5,.2)
 \begin{psclip}{
 \pspolygon[linestyle=none](.5,0)(.5,.3)(-.5,.3)(-.5,0)(.5,0)
 }
 \psellipse[linestyle=dotted](0,0)(.5,0.2)
 \end{psclip}
}

 \newcommand{\smallidentc}{
     \pscustom[fillcolor=lightgray,fillstyle=gradient,
        gradbegin=white, gradend=gray,gradmidpoint=0,gradangle=88]{
        \psline(-.5,1)(-.5,0)
        \psbezier(-.5,0)(-.4,-.25)(.4,-.25)(.5,0)
        \psline(.5,1)
        \psline(-.5,1)
    }
\psellipse[fillcolor=lightgray,fillstyle=gradient,
        gradbegin=lightgray, gradend=gray,gradmidpoint=1,gradangle=110](0,1)(.5,.2)
 \begin{psclip}{
 \pspolygon[linestyle=none](.5,0)(.5,.3)(-.5,.3)(-.5,0)(.5,0)
 }
 \psellipse[linestyle=dotted](0,0)(.5,0.2)
 \end{psclip}
}


\newcommand{\crossc}{
   \pscustom[fillcolor=lightgray,fillstyle=gradient,
        gradbegin=white, gradend=gray,gradmidpoint=0,gradangle=125]{
 \psline(-.5,0)(1.5,2.5)
 \psline(.5,2.5)
 \psline(-1.5,0)
 \psbezier(-1.5,0)(-1.4,-.25)(-.6,-.25)(-.5,0)
 }
 \pscustom[fillcolor=lightgray,fillstyle=gradient,
        gradbegin=white, gradend=gray,gradmidpoint=0,gradangle=125]{
 \psline(.5,0)(-1.5,2.5)
 \psline(-.5,2.5)
 \psline(1.5,0)
 \psbezier(1.5,0)(1.4,-.25)(.6,-.25)(.5,0)
 }
\psellipse[fillcolor=lightgray,fillstyle=gradient,
        gradbegin=lightgray, gradend=gray,gradmidpoint=1,gradangle=110](-1,2.5)(.5,.2)
\psellipse[fillcolor=lightgray,fillstyle=gradient,
        gradbegin=lightgray, gradend=gray,gradmidpoint=1,gradangle=70](1,2.5)(.5,.2)
%%%%
 \psline[linestyle=dotted](1.5,2.5)(-.5,0)
 \psline[linestyle=dotted](.5,2.5)(-1.5,0)
%%%%
 \begin{psclip}{
 \pspolygon[linestyle=none](1.5,0)(1.5,.3)(-1.5,.3)(-1.5,0)(1.5,0)
 }
 \psellipse[linestyle=dotted](1,0)(.5,0.2)
 \psellipse[linestyle=dotted](-1,0)(.5,0.2)
 \end{psclip}
 }

\newcommand{\ucrossc}{
\pscustom[fillcolor=lightgray,fillstyle=gradient,
        gradbegin=white, gradend=gray,gradmidpoint=0,gradangle=125]{
 \psline(.5,0)(-1.5,2.5)
 \psline(-.5,2.5)
 \psline(1.5,0)
 \psbezier(1.5,0)(1.4,-.25)(.6,-.25)(.5,0)
 }
  \pscustom[fillcolor=lightgray,fillstyle=gradient,
        gradbegin=white, gradend=gray,gradmidpoint=0,gradangle=70]{
 \psline(-.5,0)(1.5,2.5)
 \psline(.5,2.5)
 \psline(-1.5,0)
 \psbezier(-1.5,0)(-1.4,-.25)(-.6,-.25)(-.5,0)
 }
\psellipse[fillcolor=lightgray,fillstyle=gradient,
        gradbegin=lightgray, gradend=gray,gradmidpoint=1,gradangle=110](-1,2.5)(.5,.2)
\psellipse[fillcolor=lightgray,fillstyle=gradient,
        gradbegin=lightgray, gradend=gray,gradmidpoint=1,gradangle=110](1,2.5)(.5,.2)
%%%%
 \psline[linestyle=dotted](-1.5,2.5)(.5,0)
 \psline[linestyle=dotted](-.5,2.5)(1.5,0)
%%%%
 \begin{psclip}{
 \pspolygon[linestyle=none](1.5,0)(1.5,.3)(-1.5,.3)(-1.5,0)(1.5,0)
 }
 \psellipse[linestyle=dotted](1,0)(.5,0.2)
 \psellipse[linestyle=dotted](-1,0)(.5,0.2)
 \end{psclip}
}

\newcommand{\curverightc}{
  \pscustom[fillstyle=gradient,
    gradbegin=white, gradend=gray,gradmidpoint=0,gradangle=65]{
        \psbezier(1.5,2.5)(1.5,1.5)(.4,1.3)(.5,0)
        \psbezier(.5,0)(.4,-.25)(-.4,-.25)(-.5,0)
        \psbezier(-.5,0)(-.6,1.3)(.5,1.5)(.5,2.5)
        \psline(1.5,2.5)
    }
     \psellipse[fillcolor=lightgray,fillstyle=gradient,
        gradbegin=lightgray, gradend=gray,gradmidpoint=1,gradangle=110](1,2.5)(.5,.2) \begin{psclip}{
 \pspolygon[linestyle=none](.5,0)(.5,.3)(-.5,.3)(-.5,0)(.5,0)
 }
 \psellipse[linestyle=dotted](0,0)(.5,0.2)
 \end{psclip}
}

\newcommand{\curveleftc}{
  \pscustom[fillstyle=gradient,
    gradbegin=white, gradend=gray,gradmidpoint=0,gradangle=115]{
        \psbezier(-1.5,2.5)(-1.5,1.5)(-.4,1.3)(-.5,0)
        \psbezier(-.5,0)(-.4,-.25)(.4,-.25)(.5,0)
        \psbezier(.5,0)(.6,1.3)(-.5,1.5)(-.5,2.5)
        \psline(-1.5,2.5)
    }
    \psellipse[fillcolor=lightgray,fillstyle=gradient,
        gradbegin=lightgray, gradend=gray,gradmidpoint=1,gradangle=110](-1,2.5)(.5,.2)
 \begin{psclip}{
 \pspolygon[linestyle=none](.5,0)(.5,.3)(-.5,.3)(-.5,0)(.5,0)
 }
 \psellipse[linestyle=dotted](0,0)(.5,0.2)
 \end{psclip}
}

\newcommand{\stringZIGZAGi}{
\xy
    (-10,-12)*{}="1E";
    (-10,0)*{}="1";
    (0,0)*{}="2";
    (10,0)*{}="3";
    (10,12)*{}="3B";
 "2";"1" **\crv{(0,10)& (-10,10)}
     ?(.03)*\dir{>}  ?(1)*\dir{>};
 "3";"2" **\crv{(10,-10)& (0,-10)}
     ?(.03)*\dir{>}  ;
 "1";"1E" **\dir{-};
 "3B";"3" **\dir{-};
     (-5,8.5)*{\scriptstyle };
     (5,-9)*{\scriptstyle };
\endxy
\qquad = \xy (-6,10)*{}; (0,12)*{}; (0,-12)*{}; **\dir{-}
?(.47)*\dir{<}; (6,-10)*{}; (4,0)*{\scriptstyle }
\endxy
}

\newcommand{\stringZIGZAGii}{
  \xy
    (-10,12)*{}="1E";
    (-10,0)*{}="1";
    (0,0)*{}="2";
    (10,0)*{}="3";
    (10,-12)*{}="3B";
 "2";"1" **\crv{(0,-10)& (-10,-10)}
     ?(.03)*\dir{>}  ?(1)*\dir{>};
 "3";"2" **\crv{(10,10)& (0,10)}
     ?(.03)*\dir{>}  ;
 "1";"1E" **\dir{-};
 "3B";"3" **\dir{-};
      (5,10)*{\scriptstyle  };
     (-5,-10.5)*{\scriptstyle };
\endxy
\qquad = \xy (-6,10)*{}; (0,12)*{}; (0,-12)*{}; **\dir{-}
?(.53)*\dir{>}; (6,-10)*{}; (4,0)*{\scriptstyle };
\endxy
}

\newcommand{\stringDIM}{
 \xy
    (-5,8)*{}="x1";
    (5,8)*{}="x2";
    (-5,5)*{}="m1";
    (5,5)*{}="m2";
    (-5,-5)*{}="k1";
    (5,-5)*{}="k2";
    (-5,-8)*{}="y1";
    (5,-8)*{}="y2";
 \vtwist~{"m1"}{"m2"}{"k1"}{"k2"};
 "x1";"m1" **\dir{-} ?(.25)*\dir{>};
 "x2";"m2" **\dir{-} ?(0)*\dir{<};
 "k1";"y1" **\dir{-} ?(0)*\dir{<};
 "k2";"y2" **\dir{-} ?(.27)*\dir{>};
    "x1";"x2" **\crv{(-5,16) & (5,16)};
    "y1";"y2" **\crv{(-5,-14) & (5,-14)};
        (-7.5,8)*{H};
\endxy
}

\newcommand{\DIMofH}{
 \vcenter{\xy
    (-5,8)*{}="x1";
    (5,8)*{}="x2";
    (-5,5)*{}="m1";
    (5,5)*{}="m2";
    (-5,-5)*{}="k1";
    (5,-5)*{}="k2";
    (-5,-8)*{}="y1";
    (5,-8)*{}="y2";
 \vtwist~{"m1"}{"m2"}{"k1"}{"k2"};
 "x1";"m1" **\dir{-} ?(.25)*\dir{>};
 "x2";"m2" **\dir{-} ?(0)*\dir{<};
 "k1";"y1" **\dir{-} ?(0)*\dir{<};
 "k2";"y2" **\dir{-} ?(.27)*\dir{>};
    "x1";"x2" **\crv{(-5,16) & (5,16)};
    "y1";"y2" **\crv{(-5,-14) & (5,-14)};
        (-7.5,8)*{H};
\endxy}
 \qquad \qquad
 \vcenter{ \xy
    (0,14)*+{1}="1";
    (0,5)*+{e^i \tensor e_i}="2";
    (0,-5)*+{e_i \tensor e^i}="3";
    (0,-14)*+{\delta^i_i = \dim(H)}="4";
        {\ar@{|->} "1";"2"};
        {\ar@{|->} "2";"3"};
        {\ar@{|->} "3";"4"};
\endxy }
}

\newcommand{\feynmanI}{
\xy
 (0,10)*{}="1";
 (0,4)*{}="2";
 (0,-4)*{}="3";
 (0,-10)*{}="4";
    "1";"2"+(0,-1) **\dir{~};
    "3"+(0,+.9);"4" **\dir{~};
    "2";"3" **\crv{(-3,4) & (-3,-4)}?(.6)*\dir{>};
    "2";"3" **\crv{(3,4) & (3,-4)}?(.45)*\dir{<};
\endxy
}

\newcommand{\feynmanII}{
\xy
 (0,10)*{}="1";
 (6,4)*{}="2";
 (6,-4)*{}="3";
 (0,-10)*{}="4";
    "1";"4" **\dir{-}?(.55)*\dir{>};
    "2";"3" **\crv{(3,4) & (3,-4)}?(.4)*\dir{<};
    "2";"3" **\crv{(9,4) & (9,-4)}?(.55)*\dir{>};
\endxy
}

\newcommand{\binorID}{
 \xy
 (-4,8)*{}="TL"; (4,8)*{}="TR";
 (-4,-8)*{}="BL"; (4,-8)*{}="BR";
    \vtwist~{"TL"}{"TR"}{"BL"}{"BR"};
    (-6,6.5)*{\scriptstyle \frac{1}{2}};
    (6,6.5)*{\scriptstyle \frac{1}{2}};
 \endxy
\quad = \quad
 \xy
 (-4,8)*{}="TL"; (4,8)*{}="TR";
 (-4,-8)*{}="BL"; (4,-8)*{}="BR";
    "TL";"TR" **\crv{(-3,0) & (3,0)};
    "BL";"BR" **\crv{(-3,0) & (3,0)};
    (-6,6.5)*{\scriptstyle \frac{1}{2}};
    (6,6.5)*{\scriptstyle \frac{1}{2}};
    (-6,-6.5)*{\scriptstyle \frac{1}{2}};
    (6,-6.5)*{\scriptstyle \frac{1}{2}};
 \endxy
\quad + \quad
  \xy
 (-4,8)*{}="TL"; (4,8)*{}="TR";
 (-4,-8)*{}="BL"; (4,-8)*{}="BR";
    "TL";"BL" **\dir{-};
    "TR";"BR" **\dir{-};
    (-6,6.5)*{\scriptstyle \frac{1}{2}};
    (6,6.5)*{\scriptstyle \frac{1}{2}};
 \endxy
 }

 \newcommand{\binorIDII}{
 \xy
 (-4,8)*{}="TL"; (4,8)*{}="TR";
 (0,3)*{}="M1"; (0,-3)*{}="M2";
 (-4,-8)*{}="BL"; (4,-8)*{}="BR";
    "TL";"M1" **\dir{-};
    "TR";"M1" **\dir{-};
    "M1";"M2" **\dir{-};
    "M2";"BL" **\dir{-};
    "M2";"BR" **\dir{-};
    (-6,6.5)*{\scriptstyle 1};
    (6,6.5)*{\scriptstyle 1};
    (-6,-6.5)*{\scriptstyle 1};
    (6,-6.5)*{\scriptstyle 1};
 \endxy
\quad =   \quad \scriptstyle (\frac{1}{2})
  \xy
 (-4,8)*{}="TL"; (4,8)*{}="TR";
 (-4,-8)*{}="BL"; (4,-8)*{}="BR";
    \vtwist~{"TL"}{"TR"}{"BL"}{"BR"};
    (-6,6.5)*{\scriptstyle 1};
    (6,6.5)*{\scriptstyle 1};
 \endxy
\quad - \quad \scriptstyle (\frac{1}{2})
  \xy
 (-4,8)*{}="TL"; (4,8)*{}="TR";
 (-4,-8)*{}="BL"; (4,-8)*{}="BR";
    "TL";"BL" **\dir{-};
    "TR";"BR" **\dir{-};
    (-6,6.5)*{\scriptstyle 1};
    (6,6.5)*{\scriptstyle 1};
 \endxy
 }

\newcommand{\useBINOR}{
\xy
  (0,15)*{}="T";
  (0,-15)*{}="B";
  (0,7.5)*{}="T'";
  (0,-7.5)*{}="B'";
    "T";"T'" **\dir{-};
    "B";"B'" **\dir{-};
    (-4.5,0)*{}="MB";
    (-10.5,0)*{}="LB";
    "T'";"LB" **\crv{(-1.5,-6) & (-10.5,-6)}; \POS?(.25)*{\hole}="2z";
    "LB"; "2z" **\crv{(-12,9) & (-3,9)};
    "2z";"B'" **\crv{(0,-4.5)};
    (2,13)*{\scs \frac{1}{2}};
    \endxy
    \quad = \quad
    \xy
  (0,15)*{}="T";
  (0,-15)*{}="B";
  (0,4)*{}="T'";
  (0,-4)*{}="B'";
    "T";"T'" **\dir{-};
    "B";"B'" **\dir{-};
    (-4,4)*{}="t1";
    (-10,4)*{}="t2";
    (-4,-4)*{}="b1";
    (-10,-4)*{}="b2";
    "T'"; "t1" **\crv{(0,0) & (-4,0)};
    "t1"; "t2" **\crv{(-4,7) & (-10,7)};
    "t2";"b2" **\dir{-};
    "B'"; "b1" **\crv{(0,0) & (-4,0)};
    "b1"; "b2" **\crv{(-4,-7) & (-10,-7)};
    (2,13)*{\scs \frac{1}{2}};
    \endxy
    \quad + \quad
    \xy
  (0,15)*{}="T";
  (0,-15)*{}="B";
  (0,4)*{}="T'";
  (0,-4)*{}="B'";
    "T";"B" **\dir{-};
    (-4,4)*{}="t1";
    (-10,4)*{}="t2";
    (-4,-4)*{}="b1";
    (-10,-4)*{}="b2";
    "t1"; "t2" **\crv{(-4,7) & (-10,7)};
    "t2";"b2" **\dir{-};
    "t1";"b1" **\dir{-};
    "b1"; "b2" **\crv{(-4,-7) & (-10,-7)};
    (2,13)*{\scs \frac{1}{2}};
    \endxy
        \quad =\quad -\;\;
    \xy
  (0,15)*{}="T";
  (0,-15)*{}="B";
    "T";"B" **\dir{-};
    (2,13)*{\scs \frac{1}{2}};
    \endxy
    }


\newcommand{\spinNET}{
\xy
    (-4,0)*{}="b";
    (4,8)*{}="t";
    (9,-6)*{}="v";
    (14,6)*{}="u";
    (-5,8)*{}="e";
    (-20,1)*{}="q";
    (-10,-10)*{}="c";
    (10,-15)*{}="d";
    (22,4)*{}="z";
    (24,-9)*{}="w";
        {\ar@/^1pc/@{-} "q";"e"};
        {\ar@/_.1pc/@{-} "q";"b"};
        {\ar@{-} "b";"e"};
        {\ar@/^.2pc/@{-} "e";"t"};
        {\ar@/_.5pc/@{-} "u";"t"};
        {\ar@/^.5pc/@{-} "u";"t"};
        {\ar@/_.2pc/@{-} "b";"v"};
        {\ar@/_.2pc/@{-} "z";"u"};
        {\ar@/_1pc/@{-} "q";"c"};
        {\ar@/_.8pc/@{-} "c";"d"};
        {\ar@/^.6pc/@{-} "c";"d"};
        {\ar@/^.3pc/@{-} "z";"v"};
        {\ar@/^.2pc/@{-} "z";"w"};
        {\ar@/^.4pc/@{-} "w";"d"};
        {\ar@{-} "v";"w"};
 \endxy
 }

\newcommand{\spinNETII}{
\def\objectstyle{\scriptscriptstyle}
\xy
    (-4,0)*{\bullet}="b";
    (4,8)*{\bullet}="t";
    (9,-6)*{\bullet}="v";
    (14,6)*{\bullet}="u";
    (-5,8)*{\bullet}="e";
    (-20,1)*{\bullet}="q";
    (-10,-10)*{\bullet}="c";
    (10,-15)*{\bullet}="d";
    (22,4)*{\bullet}="z";
    (24,-9)*{\bullet}="w";
        {\ar@/^1pc/@{-}^1 "q";"e"};
        {\ar@/_.1pc/@{-}^1 "q";"b"};
        {\ar@{-}^1 "b";"e"};
        {\ar@/^.2pc/@{-}_1 "e";"t"};
        {\ar@/_.5pc/@{-}_1 "u";"t"};
        {\ar@/^.5pc/@{-}^1 "u";"t"};
        {\ar@/_.2pc/@{-}^1 "b";"v"};
        {\ar@/_.2pc/@{-}^1 "z";"u"};
        {\ar@/_1pc/@{-}_1 "q";"c"};
        {\ar@/_.8pc/@{-}_1 "c";"d"};
        {\ar@/^.6pc/@{-}^1 "c";"d"};
        {\ar@/^.3pc/@{-}^1 "z";"v"};
        {\ar@/^.2pc/@{-}^1 "z";"w"};
        {\ar@/^.4pc/@{-}^1 "w";"d"};
        {\ar@{-}_1 "v";"w"};
 \endxy
 }

\newcommand{\xygrid}{
%%%%
%%%%
 (-20,20)*{20};(20,20)*{} **\dir{.};
 (-20,15)*{};(20,15)*{} **\dir{.};
 (-20,10)*{10};(20,10)*{} **\dir{.};
 (-20,5)*{};(20,5)*{} **\dir{.};
 (-20,0)*{0};(20,0)*{} **\dir{.};
 (-20,-20)*{-20};(20,-20)*{} **\dir{.};
 (-20,-15)*{};(20,-15)*{} **\dir{.};
 (-20,-10)*{-10};(20,-10)*{} **\dir{.};
 (-20,-5)*{};(20,-5)*{} **\dir{.};
 %%%%
  (-20,20)*{020};(-20,-20)*{} **\dir{.};
 (-15,20)*{};(-15,-20)*{} **\dir{.};
 (-10,20)*{010};(-10,-20)*{} **\dir{.};
 (-5,20)*{};(-5,-20)*{} **\dir{.};
 (0,20)*{0};(0,-20)*{} **\dir{.};
 (20,20)*{20};(20,-20)*{} **\dir{.};
 (15,20)*{};(15,-20)*{} **\dir{.};
 (10,20)*{10};(10,-20)*{} **\dir{.};
 (5,20)*{};(5,-20)*{} **\dir{.};
 %%%%
 %%%%
}

\newcommand{\PLUGzigzag}{ \xy
    (-10,-12)*{}="1E";
    (-10,0)*{}="1";
    (0,0)*{}="2";
    (10,0)*{}="3"+(3,0)*{H};
    (10,12)*{}="3B";
 "2";"1" **\crv{(0,10)& (-10,10)}
     ?(.03)*\dir{>}  ?(1)*\dir{>};
 "3";"2" **\crv{(10,-10)& (0,-10)}
     ?(.03)*\dir{>}  ;
 "1";"1E" **\dir{-};
 "3B";"3" **\dir{-};
     (-5,8.5)*{\scriptstyle e_H};
     (5,-9)*{\scriptstyle i_H};
\endxy
 \qquad \qquad
\xy
    (0,11)*+{\psi}="1";
    (0,0)*+{e_i \tensor e^i \tensor \psi}="2";
    (0,-11)*+{e_i \tensor \psi^i = \psi}="3";
        {\ar@{|->} "1";"2"};
        {\ar@{|->} "2";"3"};
\endxy
}

\newcommand{\SIXjI}{
\psset{xunit=1.2cm,yunit=1.2cm}
\psset{linewidth=0.3pt,dimen=middle,linearc=.05pt,cornersize=absolute}
%\psset{gridcolor=green, subgridcolor=yellow} \psgrid
\begin{pspicture}[.3](3.5,3.5)
     %% draws the outer tetrahedron
     \pspolygon[linewidth=0.5pt,linestyle=solid,linecolor=black, fillstyle=gradient,gradbegin=white,
        gradend=lightgray,gradmidpoint=1,gradangle=130](0,.5)(3,0)(1.75,3)(0,.5)
     \pspolygon[linewidth=0.5pt,linestyle=solid,linecolor=black, fillstyle=gradient,gradbegin=lightgray,
        gradend=gray,gradmidpoint=0,gradangle=310](3,0)(1.75,3)(4,2)(3,0)
     %\psline{c-c}(0,.5)(3,0)
     %\psline{c-c}(3,0)(4,2)
     %\psline[linewidth=0.5pt]{c-c}(4,2)(1.75,3)
     %\psline[linewidth=0.5pt]{c-c}(3,0)(1.75,3)
     %\psline[linewidth=0.5pt]{c-c}(0,.5)(1.75,3)
%%%%
     \psline[linestyle=dotted,linewidth=0.5pt]{c-c}(0,.5)(4,2)
%%%%
%%%%
     %% LABELS
     \rput(.7,2){$a$}
     \rput(1.2,.1){$d$}
     \rput(1.2,1.2){$e$}
     \rput(3,2.7){$b$}
     \rput(2.8,1){$f$}
     \rput(3.8,.8){$c$}
\end{pspicture}
}



\newcommand{\SIXjII}{
\psset{xunit=1.2cm,yunit=1.2cm}
\psset{linewidth=0.3pt,dimen=middle,linearc=.05pt,cornersize=absolute}
\begin{pspicture}[.3](3.5,3.5)
     %% draws the outer tetrahedron
     \pspolygon[linewidth=0.5pt,linestyle=solid,linecolor=black, fillstyle=gradient,gradbegin=white,
        gradend=lightgray,gradmidpoint=1,gradangle=130](0,.5)(3,0)(1.75,3)(0,.5)
     \pspolygon[linewidth=0.5pt,linestyle=solid,linecolor=black, fillstyle=gradient,gradbegin=lightgray,
        gradend=gray,gradmidpoint=0,gradangle=310](3,0)(1.75,3)(4,2)(3,0)
%%%%
    %%Draws inner tetrahedron
     \pspolygon(1.6,1.4)(2.9,1.9)(2.4,.8)(1.6,1.4)
     \pspolygon(1.6,1.4)(1.9,1.9)(2.9,1.9)(1.6,1.4)
    %% draws the dotted line for inner tetrahedron
    \psline(1.9,1.9)(2,1.7)
    \psline(2.08,1.5)(2.4,.8)
%%%%
%%%%
     \psline[linestyle=dotted,linewidth=0.5pt]{c-c}(0,.5)(1.85,1.2)
     \psline[linestyle=dotted,linewidth=0.5pt]{c-c}(2.75,1.52)(4,2)
%%%%
    %\psdots(1.6,1.4)(1.9,1.9)(2.9,1.9)(2.4,.8)
         %% LABELS
     \rput(.7,2){$a$}
     \rput(1.2,.1){$d$}
     \rput(1.2,1.2){$e$}
     \rput(3,2.7){$b$}
     \rput(2.8,1){$f$}
     \rput(3.8,.8){$c$}
%%%%
\end{pspicture}
}

\newcommand{\SIXjIII}{
\psset{xunit=1.8cm,yunit=1.8cm}
\psset{linewidth=0.3pt,dimen=middle,linearc=.05pt,cornersize=absolute}
\begin{pspicture}(3.5,1.3)
    %%Draws inner tetrahedron
     \pspolygon(1.6,.6)(2.9,1.1)(2.4,0)(1.6,.6)
     \pspolygon(1.6,.6)(1.9,1.1)(2.9,1.1)(1.6,.6)
    %% draws the dotted line for inner tetrahedron
    \psline(1.9,1.1)(2,.9)
    \psline(2.08,.7)(2.4,0)
%%%%
    \psdots[dotscale=.7 .7](1.6,.6)(1.9,1.1)(2.9,1.1)(2.4,0)
     %% LABELS
     \rput(1.6,1){$a$}
     \rput(1.8,.2){$d$}
     \rput(2.4,1.23){$b$}
     \rput(2.8,.6){$c$}
      \rput(2.5,.8){$f$}
     \rput(2.05,.55){$e$}
\end{pspicture}
}


\newcommand{\opetopicCHART}{
\begin{center}\makebox[0pt]{
\begin{tabular}{|c|c|c|c|c|}
  \hline
  % after \\: \hline or \cline{col1-col2} \cline{col3-col4} ...
   \textbf{Objects} & \textbf{Morphisms} & \textbf{2-morphisms} & \textbf{3-morphisms} & $\cdots$ \\
  \hline \hline
   $\bullet$ & $\xy
  (-6,0)*+{\bullet}="1";
  (0,8)*+{}; %SPACE
  (0,-8)*+{}; %SPACE
  (6,0)*+{\bullet}="2";
  {\ar "1";"2"};
 \endxy$
    & $
  \def\objectstyle{\scriptstyle}
\xy %0;/r.18pc/:
 (-10,-5)*+{\bullet}="x";
 (10,-5)*+{\bullet}="y";
 (-10,1)*{\bullet}="1";
 (-7,6)*{\bullet}="2";
 (0,6)*+{\dots}="3";
 (7,6)*{\bullet}="4";
 (10,1)*{\bullet}="5";
  {\ar "x"; "y"};
  {\ar "x"; "1"};
  {\ar "1"; "2"};
  {\ar "2"; "3"};
  {\ar "3"; "4"};
  {\ar "4"; "5"};
  {\ar "5"; "y"};
  {\ar@{=>} (0,3); (0,-3)};
\endxy$
 & $\def\objectstyle{\scriptstyle} \xy 0;/r.14pc/:
 (-10,-5)*+{\bullet}="1";
 (10,-5)*+{\bullet}="2";
 (-6,6)*+{\bullet}="3";
 (6,6)*+{\bullet}="3'";
  {\ar "1"; "2"};
  {\ar "3'"; "2"};
  {\ar "1"; "3"};
  {\ar "3"; "3'"};
  {\ar "1"; "3'"};
%(-3,3)*{\alpha}; (3,-1)*{\beta};
\endxy
\;\; \xy {\ar@3{->}^{} (-2,0);(2,0)}; \endxy \; \xy 0;/r.14pc/:
 (-10,-5)*+{\bullet}="1";
 (10,-5)*+{\bullet}="2";
 (-6,6)*+{\bullet}="3";
 (6,6)*+{\bullet}="3'";
  {\ar "1"; "2"};
  {\ar "3'"; "2"};
  {\ar "1"; "3"};
  {\ar "3"; "3'"};
%(0,0)*{\gamma};
\endxy$ & Opetopes  \\
  \hline
\end{tabular}
}\end{center} }

\newcommand{\DEFprojector}{
 \xy 0;/r.16pc/:
   (0,-8)*\xycircle(1.5,1.5){-}="m";
   (0,-14)*{}="b";
   (-3,-5)*{}="l";
   (3,-5)*{}="r";
   (-3,5)*{}="l1";
   (3,5)*{}="r1";
   (0,8)*\xycircle(1.5,1.5){-}="m1";
   (0,14)*\xycircle(2.5,2.5){-}="b1";
    (0,14)*{a};
         "b";"m" **\dir{-};
         "m";"l" **\crv{(-3,-8)};
         "m";"r" **\crv{(3,-8)};
         "m1";"l1" **\crv{(-3,8)};
         "m1";"r1" **\crv{(3,8)};
         "b1";"m1" **\dir{-};
         \vtwist~{"l1"}{"r1"}{"l"}{"r"}
 \endxy
}


\newcommand{\PROJcenterI}{
  \xy 0;/r.16pc/:
   (0,-8)*\xycircle(1.5,1.5){-}="m";
   (0,-14)*{}="b";
   (-3,-5)*{}="l";
   (3,-5)*{}="r";
   (-3,5)*{}="l1";
   (3,5)*{}="r1";
   (0,8)*\xycircle(1.5,1.5){-}="m1";
   (0,14)*\xycircle(2.7,2.7){-}="b1";
    (0,14)*{a};
         "b";"m" **\dir{-};
         "m";"l" **\crv{(-3,-8)};
         "m";"r" **\crv{(3,-8)};
         "m1";"l1" **\crv{(-3,8)};
         "m1";"r1" **\crv{(3,8)};
         "b1";"m1" **\dir{-};
         \vtwist~{"l1"}{"r1"}{"l"}{"r"}
 \endxy
 \quad = \quad
 \xy 0;/r.16pc/:
   (0,-8)*\xycircle(1.5,1.5){-}="m";
   (0,-14)*{}="b";
   (-3,-4)*\xycircle(1.5,1.5){-}="l";
   (3,-2)*{}="r";
   (-3,7)*{}="l1";
   (3,7)*{}="r1";
   (8,12)*\xycircle(2.7,2.7){-}="z";
    (8,12)*{a};
   (0,10)*{}="m1";
   %(0,14)*{}="b1";
         "b";"m" **\dir{-};
         "m";"l" **\crv{(-3,-8)};
         "m";(3,-4) **\crv{(3,-8)};
         "m1";"l1" **\crv{(-3,10)};
         "m1";"r1" **\crv{(3,10)};
         "l";"z" **\crv{(10,-4)};
         %"b1";"m1" **\dir{-};
         \vtwist~{"l1"}{"r1"}{"l"}{"r"}
 \endxy
  \quad = \quad
 \xy 0;/r.16pc/:
   (0,-8)*\xycircle(1.5,1.5){-}="m";
   (0,-14)*{}="b";
   (-3,-4)*\xycircle(1.5,1.5){-}="l";
   (-8,12)*\xycircle(2.7,2.7){-}="z";
    (-8,12)*{a};
   (3,-2)*{}="r";
   (-3,7)*{}="l1";
   (3,7)*{}="r1";
   (0,10)*{}="m1";
         "b";"m" **\dir{-};
         "m";"l" **\crv{(-3,-8)};
         "m";"r" **\crv{(3,-8)};
         "m1";"l1" **\crv{(-3,10)};
         "m1";"r1" **\crv{(3,10)};
         "l";"z" **\crv{(-10,-4)};
         \vtwist~{"l1"}{"r1"}{"l"}{"r"}
 \endxy
   \quad = \quad
 \xy 0;/r.16pc/:
   (0,-8)*\xycircle(1.5,1.5){-}="m";
   (0,-14)*{}="b";
   (-3,-4)*\xycircle(1.5,1.5){-}="l";
   (3,-2)*{}="r";
   (0,10)*{}="m1";
   (-8,12)*\xycircle(2.7,2.7){-}="z";
    (-8,12)*{a};
         "b";"m" **\dir{-};
         "m";"l" **\crv{(-3,-8)};
         "m";"r" **\crv{(5,-8)};
         "l";"z" **\crv{(-10,-4)};
         "l";"r" **\crv{(-1,2) & (2,0) };
 \endxy
 \qquad = \qquad
 \xy 0;/r.16pc/:
   (0,12)*\xycircle(2.7,2.7){-}="b";
   (0,-14)*{}="m1";
   (0,12)*{a};
          "b";"m1" **\dir{-};
 \endxy
}

\newcommand{\PROJcenterII}{
 \xy 0;/r.12pc/:
   (-5,-8)*\xycircle(1.5,1.5){-}="m";
   (-5,14)*\xycircle(2.7,2.7){-}="b1"="b";
    (-5,14)*{a};
   (-8,-5)*{}="l";
   (-2,-5)*{}="r";
   (-8,5)*{}="l1";
   (-2,5)*{}="r1";
   (-5,8)*\xycircle(1.5,1.5){-}="m1";
   (0,-15)*\xycircle(1.5,1.5){-}="b1";
   (5,-8)*{}="m'";
          "b";"m1" **\dir{-};
          "m'";(5,15) **\dir{-};
          "b1";(0,-23) **\dir{-};
         "m";"l" **\crv{(-8,-8)};
         "m";"r" **\crv{(-2,-8)};
         "m1";"l1" **\crv{(-8,8)};
         "m1";"r1" **\crv{(-2,8)};
         "b1";"m" **\crv{(-5,-15)};
         "b1";"m'" **\crv{(5,-15)};
         \vtwist~{"l1"}{"r1"}{"l"}{"r"};
 \endxy
\qquad = \qquad
 \xy 0;/r.12pc/:
   (5,-8)*\xycircle(1.5,1.5){-}="m";
   (5,14)*\xycircle(2.7,2.7){-}="b1"="b";
    (5,14)*{a};
   (8,-5)*{}="l";
   (2,-5)*{}="r";
   (8,5)*{}="l1";
   (2,5)*{}="r1";
   (5,8)*\xycircle(1.5,1.5){-}="m1";
   (0,-15)*\xycircle(1.5,1.5){-}="b1";
   (-5,-8)*{}="m'";
          "b";"m1" **\dir{-};
          "m'";(-5,15) **\dir{-};
          "b1";(0,-23) **\dir{-};
         "m";"l" **\crv{(8,-8)};
         "m";"r" **\crv{(2,-8)};
         "m1";"l1" **\crv{(8,8)};
         "m1";"r1" **\crv{(2,8)};
         "b1";"m" **\crv{(5,-15)};
         "b1";"m'" **\crv{(-5,-15)};
         \vtwist~{"l1"}{"r1"}{"l"}{"r"};
 \endxy
 }

 \newcommand{\PROJcenterIII}{
 \xy 0;/r.12pc/:
   (-5,-8)*\xycircle(1.5,1.5){-}="m";
   (-5,14)*\xycircle(2.7,2.7){-}="b1"="b";
    (-5,14)*{a};
   (-8,-5)*{}="l";
   (-2,-5)*{}="r";
   (-8,5)*{}="l1";
   (-2,5)*{}="r1";
   (-5,8)*\xycircle(1.5,1.5){-}="m1";
   (0,-15)*\xycircle(1.5,1.5){-}="b1";
   (5,-8)*{}="m'";
          "b";"m1" **\dir{-};
          "m'";(5,15) **\dir{-};
          "b1";(0,-23) **\dir{-};
         "m";"l" **\crv{(-8,-8)};
         "m";"r" **\crv{(-2,-8)};
         "m1";"l1" **\crv{(-8,8)};
         "m1";"r1" **\crv{(-2,8)};
         "b1";"m" **\crv{(-5,-15)};
         "b1";"m'" **\crv{(5,-15)};
         \vtwist~{"l1"}{"r1"}{"l"}{"r"};
 \endxy
 \quad = \quad
 \xy 0;/r.12pc/:
   (5,-8)*\xycircle(1.5,1.5){-}="m";
   (-5,14)*\xycircle(2.7,2.7){-}="b";
    (-5,14)*{a};
   (8,-5)*{}="l";
   (2,-5)*{}="r";
   (-8,5)*{}="l1";
   (-2,5)*{}="r1";
   (-5,8)*\xycircle(1.5,1.5){-}="m1";
   (0,-15)*\xycircle(1.5,1.5){-}="b1";
   (-5,-8)*{}="m'";
          "b";"m1" **\dir{-};
          "l";(8,15) **\dir{-};
          "b1";(0,-23) **\dir{-};
         "m";"l" **\crv{(8,-8)};
         "m";"r" **\crv{(2,-8)};
         "m1";"l1" **\crv{(-8,8)};
         "m1";"r1" **\crv{(-2,8)};
         "b1";"m" **\crv{(5,-15)};
         "b1";"m'" **\crv{(-5,-15)};
         \vtwist~{"l1"}{"r1"}{"m'"}{"r"};
 \endxy
\quad = \quad
 \xy 0;/r.12pc/:
   (-5,14)*\xycircle(2.7,2.7){-}="b1"="b";
    (-5,14)*{a};
   (-5,8)*\xycircle(1.5,1.5){-}="m1";
   (-8,-2)*\xycircle(1.5,1.5){-}="l";
   (0,-15)*\xycircle(1.5,1.5){-}="b1";
   (3,-10)*{}="m'";
          "b";"m1" **\dir{-};
          "b1";(0,-23) **\dir{-};
         "b1";"m'" **\crv{(5,-15)};
         "m1";"l" **\crv{(-10,6)};
         "l";(-4,2) **\crv{(-5,1)};
         (-2,4);(3,15) **\crv{(3,8)};
         \vtwist~{"l"}{"m1"}{"b1"}{"m'"};
 \endxy
\quad = \quad
 \xy  0;/r.12pc/:
   (-5,14)*\xycircle(2.7,2.7){-}="b";
    (-5,14)*{a};
   (-5,8)*\xycircle(1.5,1.5){-}="m1";
   (-8,-2)*{}="l";
   (0,0)*\xycircle(1.5,1.5){-}="x";
   (0,-15)*\xycircle(1.5,1.5){-}="b1";
   (3,-10)*{}="m'";
          "b";"m1" **\dir{-};
          "b1";(0,-23) **\dir{-};
         "b1";"m'" **\crv{(5,-15)};
         "m1";"l" **\crv{(-10,6)};
         "x";"m1" **\crv{(4,8)};
        "x";(-1,-3) **\crv{(-2,7) & (-8,-5)};
        (1,-3);(6,15) **\crv{(8,-1) & (5,5)};
         \vtwist~{"l"}{"x"}{"b1"}{"m'"};
 \endxy
\quad = \quad
 \xy 0;/r.12pc/:
   (0,-15)*\xycircle(1.5,1.5){-}="m";
   (0,-23)*{}="b";
   (-5,-10)*\xycircle(1.5,1.5){-}="l";
   (3,-10)*{}="r";
   (-3,-6)*{}="l'";
   (3,-6)*{}="r'";
   (-3,5)*{}="l1";
   (3,5)*{}="r1";
   (0,8)*\xycircle(1.5,1.5){-}="m1";
   (0,14)*\xycircle(2.7,2.7){-}="b1";
    (0,14)*{a};
         "b";"m" **\dir{-};
         "m";"l" **\crv{(-3,-14)};
         "m";"r" **\crv{(3,-14)};
         "m1";"l1" **\crv{(-3,8)};
         "m1";"r1" **\crv{(3,8)};
         "b1";"m1" **\dir{-};
         "r'";"r" **\dir{-};
         "l";"l'" **\crv{(-3,-8)};
         "l";(-4,-5) **\crv{(-10,-5)};
         (-2,-5);(2,-5) **\crv{(0,-5)};
         (4,-5);(8,16) **\crv{(7,-5) & (8,1)};
         \vtwist~{"l1"}{"r1"}{"l'"}{"r'"}
 \endxy
\quad = \quad
 \xy 0;/r.12pc/:
   (0,-8)*\xycircle(1.5,1.5){-}="m";
   (0,-15)*\xycircle(1.5,1.5){-}="z";
   (0,-23)*{}="z'";
   (0,-12)*{}="b";
   (-3,-5)*{}="l";
   (3,-5)*{}="r";
   (-3,5)*{}="l1";
   (3,5)*{}="r1";
   (0,8)*\xycircle(1.5,1.5){-}="m1";
   (0,14)*\xycircle(2.7,2.7){-}="b1";
    (0,14)*{a};
         "b";"m" **\dir{-};
         "m";"l" **\crv{(-3,-8)};
         "m";"r" **\crv{(3,-8)};
         "m1";"l1" **\crv{(-3,8)};
         "m1";"r1" **\crv{(3,8)};
         "b1";"m1" **\dir{-};
         "z";"z'" **\dir{-};
         "z";"m" **\dir{-};
         "z";(-1,-12) **\crv{(-8,-12)};
         (1,-12);(10,15) **\crv{(12,-12)};
         \vtwist~{"l1"}{"r1"}{"l"}{"r"}
 \endxy
 \quad = \quad
  \xy 0;/r.12pc/:
   (5,-8)*\xycircle(1.5,1.5){-}="m";
   (5,14)*\xycircle(2.7,2.7){-}="b";
    (5,14)*{a};
   (8,-5)*{}="l";
   (2,-5)*{}="r";
   (8,5)*{}="l1";
   (2,5)*{}="r1";
   (5,8)*\xycircle(1.5,1.5){-}="m1";
   (0,-15)*\xycircle(1.5,1.5){-}="b1";
   (-5,-8)*{}="m'";
          "b";"m1" **\dir{-};
          "m'";(-5,14) **\dir{-};
          "b1";(0,-23) **\dir{-};
         "m";"l" **\crv{(8,-8)};
         "m";"r" **\crv{(2,-8)};
         "m1";"l1" **\crv{(8,8)};
         "m1";"r1" **\crv{(2,8)};
         "b1";"m" **\crv{(5,-15)};
         "b1";"m'" **\crv{(-5,-15)};
         \vtwist~{"l1"}{"r1"}{"l"}{"r"};
 \endxy
 }

 \newcommand{\BUBimpliesTOi}{
  \def\objectstyle{\scriptstyle}
\xy
 (-10,-8)*{\bullet}="L";
 (10,-8)*{\bullet}="R";
 (0,8)*{\bullet}="T";
    "L";"T" **\dir{-};
    "R";"T" **\dir{-};
    "L";"R" **\dir{-};
 \endxy
\quad \xy {\ar^{\txt\bf{2-2}} (-5,0); (5,0)}; \endxy \quad
 \xy
 (-10,-8)*{\bullet}="L";
 (10,-8)*{\bullet}="R";
 (0,8)*{\bullet}="T";
 (0,-8)*{\bullet}="M";
    "L";"T" **\dir{-};
    "R";"T" **\dir{-};
    {\ar@/_1.5pc/@{-}"L";"R"};
    {\ar@{-}@/^1.5pc/ "L";"R"};
    "R";"M" **\dir{-};
    "L";"M" **\dir{-};
 \endxy
\quad \xy {\ar^{\txt\bf{Bubble}} (-8,0); (8,0)}; \endxy \;\;
  \xy
 (-10,-8)*{\bullet}="L";
 (10,-8)*{\bullet}="R";
 (0,8)*{\bullet}="T";
 (0,-8)*{\bullet}="M";
    "L";"T" **\dir{-};
    "R";"T" **\dir{-};
    {\ar@/_1.5pc/@{-}"L";"R"};
    "T";"M" **\dir{-};
    "R";"M" **\dir{-};
    "L";"M" **\dir{-};
 \endxy
}

 \newcommand{\BUBimpliesTOii}{
\def\objectstyle{\scriptstyle}
\xy
 %%%% SPACING
 (-10,0)*{};
 (10,0)*{};
 %%%%
 (0,2)*{}="M";
 (0,-8)*{}="B";
 (-6,8)*{}="TL";
 (6,8)*{}="TR";
    "TL";"M" **\dir{-};
    "TR";"M" **\dir{-};
    "M";"B" **\dir{-};
 \endxy
\quad \xy  (-5,0)*{}; (0,0)*{=};(5,0)*{}; \endxy \quad
 \xy
 %%%% SPACING
 (-10,0)*{};
 (10,0)*{};
 %%%%
 (0,4)*{}="M";
 (0,1)*{}="M'";
 (0,-8)*{}="B";
 (0,-5)*{}="B'";
 (-6,8)*{}="TL";
 (6,8)*{}="TR";
 (-2,-2)*{}="xl";
 (2,-2)*{}="xr";
    "TL";"M" **\dir{-};
    "TR";"M" **\dir{-};
    "M";"M'" **\dir{-};
    "B";"B'" **\dir{-};
    "M'";"xl" **\crv{(-2,1)};
    "M'";"xr" **\crv{(2,1)};
    "xl";"B'" **\crv{(-2,-5)};
    "xr";"B'" **\crv{(2,-5)};
 \endxy
\quad \xy  (-8,0)*{}; (0,0)*{=};(8,0)*{}; \endxy \;\;
 \xy
 %%%% SPACING
 (-10,0)*{};
 (10,0)*{};
 %%%%
 (0,-8)*{}="B";
 (0,-3)*{}="B'";
 (-6,8)*{}="TL";
 (6,8)*{}="TR";
 (-2,4)*{}="xl";
 (2,4)*{}="xr";
    "TL";"xl" **\dir{-};
    "TR";"xr" **\dir{-};
    "xl";"xr" **\dir{-};
    "B";"B'" **\dir{-};
    "xl";"B'" **\crv{(-8,0)};
    "xr";"B'" **\crv{(8,0)};
 \endxy
 }

 \newcommand{\TOimpliesBUBi}{
 \def\objectstyle{\scriptstyle}
  \xy
 (-10,-8)*{\bullet}="L";
 (10,-8)*{\bullet}="R";
 (0,8)*{\bullet}="T";
    "L";"T" **\dir{-};
    "R";"T" **\dir{-};
    "L";"R" **\dir{-};
 \endxy
\quad \xy {\ar^{\txt\bf{3-1}} (-8,0); (8,0)}; \endxy \;\;
  \xy
 (-10,-8)*{\bullet}="L";
 (10,-8)*{\bullet}="R";
 (0,8)*{\bullet}="T";
 (0,-8)*{\bullet}="M";
    "L";"T" **\dir{-};
    "R";"T" **\dir{-};
    {\ar@/_1.5pc/@{-}"L";"R"};
    "T";"M" **\dir{-};
    "R";"M" **\dir{-};
    "L";"M" **\dir{-};
 \endxy
 \quad \xy {\ar^{\txt\bf{2-2}} (-5,0); (5,0)}; \endxy \quad
 \xy
 (-10,-8)*{\bullet}="L";
 (10,-8)*{\bullet}="R";
 (0,8)*{\bullet}="T";
 (0,-8)*{\bullet}="M";
    "L";"T" **\dir{-};
    "R";"T" **\dir{-};
    {\ar@/_1.5pc/@{-}"L";"R"};
    {\ar@{-}@/^1.5pc/ "L";"R"};
    "R";"M" **\dir{-};
    "L";"M" **\dir{-};
 \endxy
}

 \newcommand{\TOimpliesBUBii}{
\def\objectstyle{\scriptstyle}
\xy
 %%%% SPACING
 (-10,0)*{};
 (10,0)*{};
 %%%%
 (0,2)*{}="M";
 (0,-8)*{}="B";
 (-6,8)*{}="TL";
 (6,8)*{}="TR";
    "TL";"M" **\dir{-};
    "TR";"M" **\dir{-};
    "M";"B" **\dir{-};
 \endxy
\quad \xy  (-8,0)*{}; (0,0)*{=};(8,0)*{}; \endxy \quad \;
  \xy
   %%%% SPACING
 (-10,0)*{};
 (10,0)*{};
 %%%%
 (0,-8)*{}="B";
 (-6,8)*{}="TL";
 (6,8)*{}="TR";
 (-2,3)*{}="tl";
 (2,3)*{}="tr";
 (0,-1.5)*{}="b";
    "TL";"tl" **\dir{-};
    "TR";"tr" **\dir{-};
    "b";"B" **\dir{-};
    "tl";"tr" **\dir{-};
    "tr";"b" **\dir{-};
    "tl";"b" **\dir{-};
  \endxy
 \quad \xy (-5,0)*{}; (0,0)*{=}; (5,0)*{}; \endxy \quad
  \xy
 %%%% SPACING
 (-10,0)*{};
 (10,0)*{};
 %%%%
 (0,4)*{}="M";
 (0,1)*{}="M'";
 (0,-8)*{}="B";
 (0,-5)*{}="B'";
 (-6,8)*{}="TL";
 (6,8)*{}="TR";
 (-2,-2)*{}="xl";
 (2,-2)*{}="xr";
    "TL";"M" **\dir{-};
    "TR";"M" **\dir{-};
    "M";"M'" **\dir{-};
    "B";"B'" **\dir{-};
    "M'";"xl" **\crv{(-2,1)};
    "M'";"xr" **\crv{(2,1)};
    "xl";"B'" **\crv{(-2,-5)};
    "xr";"B'" **\crv{(2,-5)};
 \endxy
 }

\newcommand{\TWOTHREEi}{
 \xy
 (0,30)*+{
    \xy 0;/r.10pc/:
    (6.18,19)*{}="t1"; %2pi/5
    (-16.18,11.74)*{}="t2";
    (-16.18,-11.74)*{}="t3";
    (6.18,-19)*{}="t4";
    (20,0)*{}="t5";
        {\ar@{-}"t1";"t2"};
        {\ar@{-} "t2";"t3"};
        {\ar@{-} "t3";"t4"};
        {\ar@{-} "t4";"t5"};
        {\ar@{-} "t1";"t5"};
        {\ar@{-} "t1";"t3"}; {\ar@{-} "t3";"t5"};
   \endxy}="t";
 (-30,15)*+{
    \xy 0;/r.10pc/:
    (6.18,19)*{}="t1"; %2pi/5
    (-16.18,11.74)*{}="t2";
    (-16.18,-11.74)*{}="t3";
    (6.18,-19)*{}="t4";
    (20,0)*{}="t5";
        {\ar@{-}"t1";"t2"};
        {\ar@{-} "t2";"t3"};
        {\ar@{-} "t3";"t4"};
        {\ar@{-} "t4";"t5"};
        {\ar@{-} "t1";"t5"};
        {\ar@{-} "t1";"t3"}; {\ar@{-} "t3";"t5"};
        {\ar@{-}|<<<<<<<{ \hole } "t2";"t5"};
   \endxy}="l1";
 (-30,-15)*+{
    \xy 0;/r.10pc/:
    (6.18,19)*{}="t1"; %2pi/5
    (-16.18,11.74)*{}="t2";
    (-16.18,-11.74)*{}="t3";
    (6.18,-19)*{}="t4";
    (20,0)*{}="t5";
        {\ar@{-}"t1";"t2"};
        {\ar@{-} "t2";"t3"};
        {\ar@{-} "t3";"t4"};
        {\ar@{-} "t4";"t5"};
        {\ar@{-} "t1";"t5"};
        {\ar@{-} "t1";"t3"}; {\ar@{-} "t3";"t5"};
        {\ar@{-}|<<<<<<<{ \hole } "t2";"t5"};
        {\ar@{-}|<<<<<<<{  \hole}|>>>>>>{  \hole} "t2";"t4"};
   \endxy}="l2";
 (0,-30)*+{
    \xy 0;/r.10pc/:
    (6.18,19)*{}="t1"; %2pi/5
    (-16.18,11.74)*{}="t2";
    (-16.18,-11.74)*{}="t3";
    (6.18,-19)*{}="t4";
    (20,0)*{}="t5";
        {\ar@{-}"t1";"t2"};
        {\ar@{-} "t2";"t3"};
        {\ar@{-} "t3";"t4"};
        {\ar@{-} "t4";"t5"};
        {\ar@{-} "t1";"t5"};
        {\ar@{-} "t1";"t3"}; {\ar@{-} "t3";"t5"};
        {\ar@{-}|<<<<<<<{ \hole \; \hole}|>>>>>>>{ \hole \; \hole} "t1";"t4"};
        {\ar@{-}|<<<<<<<{ \hole } "t2";"t5"};
        {\ar@{-}|<<<<<<<{  \hole}|>>>>>>{  \hole} "t2";"t4"};
   \endxy}="b";
 (30,15)*+{
    \xy 0;/r.10pc/:
    (6.18,19)*{}="t1"; %2pi/5
    (-16.18,11.74)*{}="t2";
    (-16.18,-11.74)*{}="t3";
    (6.18,-19)*{}="t4";
    (20,0)*{}="t5";
        {\ar@{-}"t1";"t2"};
        {\ar@{-} "t2";"t3"};
        {\ar@{-} "t3";"t4"};
        {\ar@{-} "t4";"t5"};
        {\ar@{-} "t1";"t5"};
        {\ar@{-} "t1";"t3"}; {\ar@{-} "t3";"t5"};
        {\ar@{-}|<<<<<<<{ \hole } "t2";"t5"};
   \endxy}="r1";
 (30,-15)*+{
    \xy 0;/r.10pc/:
    (6.18,19)*{}="t1"; %2pi/5
    (-16.18,11.74)*{}="t2";
    (-16.18,-11.74)*{}="t3";
    (6.18,-19)*{}="t4";
    (20,0)*{}="t5";
        {\ar@{-}"t1";"t2"};
        {\ar@{-} "t2";"t3"};
        {\ar@{-} "t3";"t4"};
        {\ar@{-} "t4";"t5"};
        {\ar@{-} "t1";"t5"};
        {\ar@{-} "t1";"t3"}; {\ar@{-} "t3";"t5"};
        {\ar@{-}|<<<<<<<{ \hole } "t2";"t5"};
        {\ar@{-}|<<<<<<<{  \hole}|>>>>>>{  \hole} "t2";"t4"};
   \endxy}="r2";
    {\ar@3{->} "t";"r1" };
    {\ar@3{->} "t";"l1" };
    {\ar@3{->} "l1";"l2" };
    {\ar@3{->} "r1";"r2" };
    {\ar@3{->} "r2";"b" };
    {\ar@3{->} "l2";"b" };
 \endxy
 }


\newcommand{\WWWz}{
 \def\objectstyle{\scriptscriptstyle}
  \xy
  (6,0)*{}="xr";
  (-6,0)*{}="xl";
  (-12,10)*{}="xlt";
  (-12,-10)*{}="xlb";
    (12,10)*{}="xrt";
  (12,-10)*{}="xrb";
  (6,0)*{\bullet};
  (-6,0)*{\bullet};
    "xl";"xr" **\dir{-}?(.5)*\dir{<};
    "xl";"xlt" **\dir{-}?(.5)*\dir{<};
    "xl";"xlb" **\dir{-}?(.65)*\dir{>};
    "xr";"xrt" **\dir{-}?(.5)*\dir{<};
    "xr";"xrb" **\dir{-}?(.5)*\dir{<};
 \endxy
 \quad = \quad
 \xy
  (0,-4)*{}="xr";
  (0,4)*{}="xl";
  (10,10)*{}="xlt";
  (-10,10)*{}="xlb";
    (10,-10)*{}="xrt";
  (-10,-10)*{}="xrb";
  (0,4)*{\bullet};
  (0,-4)*{\bullet};
    "xl";"xr" **\dir{-}?(.58)*\dir{>};
    "xl";"xlt" **\dir{-}?(.5)*\dir{<};
    "xl";"xlb" **\dir{-}?(.5)*\dir{<};
    "xr";"xrt" **\dir{-}?(.5)*\dir{<};
    "xr";"xrb" **\dir{-}?(.65)*\dir{>};
 \endxy
}

\newcommand{\WWWy}{
    \vcenter{\xy
   (6,16)*{}="t1";
   (-6,16)*{}="t2";
   (0,2)*{}="m";
         "t1";"m" **\crv{(6,2)} ?(.4)*\dir{>}  ;
         "t2";"m" **\crv{(-6,2)} ?(.4)*\dir{>}  ;
 \endxy}
\qquad =  \qquad
  \vcenter{\xy
   (5,6)*{}="t1";
   (-5,6)*{}="t2";
   (0,2)*{}="m";
           \vtwist~{(-5,16)}{(5,16)}{"t2"}{"t1"};
         "t1";"m" **\crv{(5,2)} ?(0)*\dir{>}  ;
         "t2";"m" **\crv{(-5,2)} ?(0)*\dir{>}  ;
 \endxy}
 }

 \newcommand{\CATchartI}{
 \begin{center}  \makebox[0pt]{
\begin{tabular}{|p{2.2in}|p{2.4in}|}
  \hline
  % after \\: \hline or \cline{col1-col2} \cline{col3-col4} ...
  \textbf{Set-based mathematics} & \textbf{Category-based mathematics} \\
  \hline \hline
  \textbf{Sets}
 $ \xy
 (0,0)*{\includegraphics{blob30.eps}};
 (0,0)*{\scs \bullet};
 (2,4)*{\scs \bullet};
 (-5,-4)*{\scs \bullet};
 (6,-1)*{\scs \bullet};
 \endxy $ &  \textbf{Categories}
 $ \xy
 (0,0)*{\includegraphics{blob30.eps}};
 (2,0)*{\scs \bullet}="1";
 (-3,6)*{\scs \bullet}="2";
 (-5,-4)*{\scs \bullet}="3";
 (2,-7)*{\scs \bullet}="4";
  "1";"2" **\crv{(-2,6)}?(.25)*\dir{<};
  "1";"3" **\crv{(-2,-4)}?(.25)*\dir{<};
  "3";"2" **\crv{(-6,0)}?(.5)*\dir{<};
  "1";"4" **\crv{}?(.25)*\dir{<};
 \endxy $\\
 \textbf{Functions} \quad$ \xy
 (0,0)*{\includegraphics{blob30.eps}};
 (30,0)*{\includegraphics{blobII30.eps}};
 (0,-5)*{\scs \bullet}="1";
 (2,4)*{\scs \bullet}="2";
 (6,-1)*{\scs \bullet}="3";
 (33,-2)*{\scs \bullet}="1'";
 (27,-4)*{\scs \bullet}="2'";
 (28,1)*{\scs \bullet}="3'";
 (31,5)*{\scs \bullet}="4'";
 {\ar@{>} "1";"1'"};
 {\ar@{>} "2";"4'"};
 {\ar@{>} "3";"3'"};
 \endxy$  & \textbf{Functors} \quad $ \xy
 (0,0)*{\includegraphics{functor30.eps}};
 %(-10,-5)*{\scs \bullet}="1";
 (-14,2)*{\scs \bullet}="2";
 (-8,-3)*{\scs \bullet}="3";
 (21,-2)*{\scs \bullet}="1'";
 (17,-4)*{\scs \bullet}="3'";
 (18,5)*{\scs \bullet}="4'";
  "2";"3" **\crv{}?(.5)*\dir{>};
  "3'";"4'" **\crv{}?(.5)*\dir{>};
  "1'";"4'" **\crv{(22,2)}?(.55)*\dir{>};
 %{\ar@{>} "1";"1'"};
 {\ar@{>} "2";"4'"};
 {\ar@{>} "3";"3'"};
 \endxy$ \\
   &  \textbf{Natural Transformations}
 $ \xy
 (0,0)*{\includegraphics{natural40.eps}};
 (-17,1)*{\scs \bullet}="1";
 (-17,-7)*{\scs \bullet}="2";
 (12,4)*{\scs \bullet}="tl";
 (20,4)*{\scs \bullet}="tr";
 (12,-5)*{\scs \bullet}="bl";
 (20,-5)*{\scs \bullet}="br";
  "1";"2" **\crv{}?(.58)*\dir{>};
  "tl";"tr" **\crv{}?(.5)*\dir{>};
  "bl";"br" **\crv{}?(.55)*\dir{>};
  "tl";"bl" **\crv{}?(.55)*\dir{>}+(-3,0)*{\scs \alpha_y};
  "tr";"br" **\crv{}?(.55)*\dir{>}+(3,0)*{\scs \alpha_x};
 \endxy $\\
  \hline
\end{tabular} }
\end{center}
}

\newcommand{\vvvY}{
\xy
 (0,2)*{}="M";
 (0,-8)*{}="B";
 (-6,8)*{}="TL";
 (6,8)*{}="TR";
    "TL";"M" **\dir{-};
    "TR";"M" **\dir{-};
    "M";"B" **\dir{-};
 \endxy
}

\newcommand{\vvvT}{
 \xy
 (0,-8)*{}="B";
 (-6,8)*{}="TL";
 (6,8)*{}="TR";
 (-2,3)*{}="tl";
 (2,3)*{}="tr";
 (0,-1.5)*{}="b";
    "TL";"tl" **\dir{-};
    "TR";"tr" **\dir{-};
    "b";"B" **\dir{-};
    "tl";"tr" **\dir{-};
    "tr";"b" **\dir{-};
    "tl";"b" **\dir{-};
 \endxy
}


\newcommand{\vvvYB}{
\xy
 (0,4)*{}="M";
 (0,1)*{}="M'";
 (0,-8)*{}="B";
 (0,-5)*{}="B'";
 (-6,8)*{}="TL";
 (6,8)*{}="TR";
 (-2,-2)*{}="xl";
 (2,-2)*{}="xr";
    "TL";"M" **\dir{-};
    "TR";"M" **\dir{-};
    "M";"M'" **\dir{-};
    "B";"B'" **\dir{-};
    "M'";"xl" **\crv{(-2,1)};
    "M'";"xr" **\crv{(2,1)};
    "xl";"B'" **\crv{(-2,-5)};
    "xr";"B'" **\crv{(2,-5)};
 \endxy
}


\newcommand{\vvvG}{
\xy
 (0,-8)*{}="B";
 (0,-3)*{}="B'";
 (-6,8)*{}="TL";
 (6,8)*{}="TR";
 (-2,4)*{}="xl";
 (2,4)*{}="xr";
    "TL";"xl" **\dir{-};
    "TR";"xr" **\dir{-};
    "xl";"xr" **\dir{-};
    "B";"B'" **\dir{-};
    "xl";"B'" **\crv{(-8,0)};
    "xr";"B'" **\crv{(8,0)};
 \endxy
}

\newcommand{\vvvS}{
\xy
 (0,-8)*{}="B";
 (0,8)*{}="T";
 (-2,-0)*{}="xl";
 (2,0)*{}="xr";
    "T";"B" **\dir{-};
 \endxy
}

\newcommand{\vvvB}{
\xy
 (0,3)*{}="M";
 (0,-8)*{}="B";
 (0,-3)*{}="B'";
 (0,8)*{}="T";
 (-2,-0)*{}="xl";
 (2,0)*{}="xr";
    "T";"M" **\dir{-};
    "B";"B'" **\dir{-};
    "M";"xl" **\crv{(-2,3)};
    "M";"xr" **\crv{(2,3)};
    "xl";"B'" **\crv{(-2,-3)};
    "xr";"B'" **\crv{(2,-3)};
 \endxy
}
%####################################################################################

