\documentclass[12pt]{article}
\usepackage{amsthm,thmtools}

%\usepackage{amsmath,amsthm,amscd,amssymb}
\usepackage[colorlinks=true
,breaklinks=true
,urlcolor=blue
,anchorcolor=blue
,citecolor=blue
,filecolor=blue
,linkcolor=blue
,menucolor=blue
,linktocpage=true]{hyperref}
\hypersetup{
bookmarksopen=true,
bookmarksnumbered=true,
bookmarksopenlevel=10
}
\usepackage[noBBpl,sc]{mathpazo}
\linespread{1.05}
\usepackage[papersize={6.7in, 10.0in}, left=.5in, right=.5in, top=1in, bottom=.9in]{geometry}
\sloppy
\raggedbottom
\pagestyle{plain}

\input macros

% these include amsmath and that can cause trouble in older docs.
\makeatletter
\@ifpackageloaded{amsmath}{}{\RequirePackage{amsmath}}

\DeclareFontFamily{U}  {cmex}{}
\DeclareSymbolFont{Csymbols}       {U}  {cmex}{m}{n}
\DeclareFontShape{U}{cmex}{m}{n}{
    <-6>  cmex5
   <6-7>  cmex6
   <7-8>  cmex6
   <8-9>  cmex7
   <9-10> cmex8
  <10-12> cmex9
  <12->   cmex10}{}

\def\Set@Mn@Sym#1{\@tempcnta #1\relax}
\def\Next@Mn@Sym{\advance\@tempcnta 1\relax}
\def\Prev@Mn@Sym{\advance\@tempcnta-1\relax}
\def\@Decl@Mn@Sym#1#2#3#4{\DeclareMathSymbol{#2}{#3}{#4}{#1}}
\def\Decl@Mn@Sym#1#2#3{%
  \if\relax\noexpand#1%
    \let#1\undefined
  \fi
  \expandafter\@Decl@Mn@Sym\expandafter{\the\@tempcnta}{#1}{#3}{#2}%
  \Next@Mn@Sym}
\def\Decl@Mn@Alias#1#2#3{\Prev@Mn@Sym\Decl@Mn@Sym{#1}{#2}{#3}}
\let\Decl@Mn@Char\Decl@Mn@Sym
\def\Decl@Mn@Op#1#2#3{\def#1{\DOTSB#3\slimits@}}
\def\Decl@Mn@Int#1#2#3{\def#1{\DOTSI#3\ilimits@}}

\let\sum\undefined
\DeclareMathSymbol{\tsum}{\mathop}{Csymbols}{"50}
\DeclareMathSymbol{\dsum}{\mathop}{Csymbols}{"51}

\Decl@Mn@Op\sum\dsum\tsum

\makeatother

\makeatletter
\@ifpackageloaded{amsmath}{}{\RequirePackage{amsmath}}

\DeclareFontFamily{OMX}{MnSymbolE}{}
\DeclareSymbolFont{largesymbolsX}{OMX}{MnSymbolE}{m}{n}
\DeclareFontShape{OMX}{MnSymbolE}{m}{n}{
    <-6>  MnSymbolE5
   <6-7>  MnSymbolE6
   <7-8>  MnSymbolE7
   <8-9>  MnSymbolE8
   <9-10> MnSymbolE9
  <10-12> MnSymbolE10
  <12->   MnSymbolE12}{}

\DeclareMathSymbol{\downbrace}    {\mathord}{largesymbolsX}{'251}
\DeclareMathSymbol{\downbraceg}   {\mathord}{largesymbolsX}{'252}
\DeclareMathSymbol{\downbracegg}  {\mathord}{largesymbolsX}{'253}
\DeclareMathSymbol{\downbraceggg} {\mathord}{largesymbolsX}{'254}
\DeclareMathSymbol{\downbracegggg}{\mathord}{largesymbolsX}{'255}
\DeclareMathSymbol{\upbrace}      {\mathord}{largesymbolsX}{'256}
\DeclareMathSymbol{\upbraceg}     {\mathord}{largesymbolsX}{'257}
\DeclareMathSymbol{\upbracegg}    {\mathord}{largesymbolsX}{'260}
\DeclareMathSymbol{\upbraceggg}   {\mathord}{largesymbolsX}{'261}
\DeclareMathSymbol{\upbracegggg}  {\mathord}{largesymbolsX}{'262}
\DeclareMathSymbol{\braceld}      {\mathord}{largesymbolsX}{'263}
\DeclareMathSymbol{\bracelu}      {\mathord}{largesymbolsX}{'264}
\DeclareMathSymbol{\bracerd}      {\mathord}{largesymbolsX}{'265}
\DeclareMathSymbol{\braceru}      {\mathord}{largesymbolsX}{'266}
\DeclareMathSymbol{\bracemd}      {\mathord}{largesymbolsX}{'267}
\DeclareMathSymbol{\bracemu}      {\mathord}{largesymbolsX}{'270}
\DeclareMathSymbol{\bracemid}     {\mathord}{largesymbolsX}{'271}

\def\horiz@expandable#1#2#3#4#5#6#7#8{%
  \@mathmeasure\z@#7{#8}%
  \@tempdima=\wd\z@
  \@mathmeasure\z@#7{#1}%
  \ifdim\noexpand\wd\z@>\@tempdima
    $\m@th#7#1$%
  \else
    \@mathmeasure\z@#7{#2}%
    \ifdim\noexpand\wd\z@>\@tempdima
      $\m@th#7#2$%
    \else
      \@mathmeasure\z@#7{#3}%
      \ifdim\noexpand\wd\z@>\@tempdima
        $\m@th#7#3$%
      \else
        \@mathmeasure\z@#7{#4}%
        \ifdim\noexpand\wd\z@>\@tempdima
          $\m@th#7#4$%
        \else
          \@mathmeasure\z@#7{#5}%
          \ifdim\noexpand\wd\z@>\@tempdima
            $\m@th#7#5$%
          \else
           #6#7%
          \fi
        \fi
      \fi
    \fi
  \fi}

\def\overbrace@expandable#1#2#3{\vbox{\m@th\ialign{##\crcr
  #1#2{#3}\crcr\noalign{\kern2\p@\nointerlineskip}%
  $\m@th\hfil#2#3\hfil$\crcr}}}
\def\underbrace@expandable#1#2#3{\vtop{\m@th\ialign{##\crcr
  $\m@th\hfil#2#3\hfil$\crcr
  \noalign{\kern2\p@\nointerlineskip}%
  #1#2{#3}\crcr}}}

\def\overbrace@#1#2#3{\vbox{\m@th\ialign{##\crcr
  #1#2\crcr\noalign{\kern2\p@\nointerlineskip}%
  $\m@th\hfil#2#3\hfil$\crcr}}}
\def\underbrace@#1#2#3{\vtop{\m@th\ialign{##\crcr
  $\m@th\hfil#2#3\hfil$\crcr
  \noalign{\kern2\p@\nointerlineskip}%
  #1#2\crcr}}}

\def\bracefill@#1#2#3#4#5{$\m@th#5#1\leaders\hbox{$#4$}\hfill#2\leaders\hbox{$#4$}\hfill#3$}

\def\downbracefill@{\bracefill@\braceld\bracemd\bracerd\bracemid}
\def\upbracefill@{\bracefill@\bracelu\bracemu\braceru\bracemid}

\DeclareRobustCommand{\downbracefill}{\downbracefill@\textstyle}
\DeclareRobustCommand{\upbracefill}{\upbracefill@\textstyle}

\def\upbrace@expandable{%
  \horiz@expandable
    \upbrace
    \upbraceg
    \upbracegg
    \upbraceggg
    \upbracegggg
    \upbracefill@}
\def\downbrace@expandable{%
  \horiz@expandable
    \downbrace
    \downbraceg
    \downbracegg
    \downbraceggg
    \downbracegggg
    \downbracefill@}

\DeclareRobustCommand{\overbrace}[1]{\mathop{\mathpalette{\overbrace@expandable\downbrace@expandable}{#1}}\limits}
\DeclareRobustCommand{\underbrace}[1]{\mathop{\mathpalette{\underbrace@expandable\upbrace@expandable}{#1}}\limits}

\makeatother


\usepackage[small]{titlesec}
\usepackage{cite}

% make sure there is enough TOC for reasonable pdf bookmarks.
\setcounter{tocdepth}{3}

\newtheorem{thm}{Theorem}[]
\newtheorem{lemma}[thm]{Lemma}

\theoremstyle{definition}
\newtheorem{defn}{Definition}[]
\newtheorem{example}{Example}[]

\theoremstyle{remark}
\newtheorem{remark}{Remark}[]
\newtheorem{note}{Note}[]

\numberwithin{equation}{section}


%    Blank box placeholder for figures (to avoid requiring any
%    particular graphics capabilities for printing this document).
\newcommand{\blankbox}[2]{%
  \parbox{\columnwidth}{\centering
%    Set fboxsep to 0 so that the actual size of the box will match the
%    given measurements more closely.
    \setlength{\fboxsep}{0pt}%
    \fbox{\raisebox{0pt}[#2]{\hspace{#1}}}%
  }%
}

\titleformat{\section}[block]
  {\fillast}
  {\bfseries{\thesection }.}
  {1ex minus .1ex}
  {\bfseries}
  
\titleformat{\subsection}[block]
  {\fillast}
  {{\it \thesection }.}
  {1ex minus .1ex}
  {\it}

\titleformat*{\subsubsection}{\scshape}
\begin{document}

\title{The Yoneda Lemma, Top Down}
\author{Pete Su}

\maketitle
\newpage

\section{The Big Picture}

The Yoneda Lemma is a basic and beloved result in category theory. In most treatments it
shows up fairly early in the books and lecture notes on the subject. Its
statement is also deceivingly compact because all of its content is buried inside layers
of abstraction and notation that would have been built up while working through the
conceptual basis of category theory.

In particular the result ties together the following sets of ideas:

\begin{itemize}

\item Categories (objects and arrows).
\item Sets of mappings in categories
\item Functors
\item Natural transformations
\item Functor categories

\end{itemize}

\noindent
What most textbooks and notes do is to spend several chapters and sections defining and
giving examples and intuition about what all these things are, and then telling you what
the lemma says. Then you have to flip back and forth through a hundred pages of exposition
to remember what all the symbols mean.

I am going to do the following dumb thing: I am going to state the result in several
different kinds of notation and then explain the pieces from the bottom up (or top down).
Also hopefully I'll provide handy cross references and call backs so you don't have to do
all that page flipping.

Note that I am not a mathematician or a category theory expert. I'm just a guy trying to
figure out the notation. So up to half of everything in this document is probably wrong.

\section{Statement of the Lemma}

The hardest thing about stating the result is deciding what notation to use to state it.
There are a ton of choices, all of which use one of three or four different notational
templates.

\begin{lemma}[Yoneda]   
\label{yoneda1}
Let $\cat{C}$ be a locally small category.  Then
% 
\begin{equation}        
\label{eq:yoneda}
\pshf{\cat{C}}(\h_X, F)
\iso
F(C)
\end{equation}
% 
naturally in $X \in \cat{C}$ and $F \in \pshf{\cat{C}}$.  
\end{lemma}

\begin{lemma}[Yoneda]\label{yoneda2}
 Let $\cat{C}$ be a category, let $X$ be an object of $\cat{C}$, and let $F:\cat{C}^\op\to\Set$ be a presheaf on $\cat{C}$. 
 Consider the map 
 $$
 \Hom_{[\cat{C}^\op,\Set]} \big( \Hom_\cat{C} (-,X) , F \big) \longrightarrow FX
 $$
 assigning to a natural transformation $\alpha:\Hom_\cat{C} (-,X)\Rightarrow F$ the element $\alpha_X(\id_X)\in FX$, which is the value of the component $\alpha_X$ of $\alpha$ on the identity at $X$. 
\end{lemma}


\begin{lemma}[Yoneda]\label{yoneda3} Let $\cat{C}$ be a locally small category and $X \in
\cat{C}$. Then for any functor $F : \cat{C} \to \Set$ there is a bijection
$$
\Hom(\cat{C}(X,-), F) \iso Fc
$$
that associates a natural transformation $\alpha:\cat{C}(X,-) \implies F$ to the element
$\alpha_X(1_X) \in FX$. Moreover, this correspondence is natural in both $X$ and $F$.
\end{lemma}

\noindent This verison introduces the $\Nat$ notation for the set of natural transforms.

\begin{lemma}[Yoneda]\label{yoneda4} For any locally small category $\cat{C}$, object $X
\in \cat{C}$, and functor $F:\cat{C} \to \Set$ we have  $\Nat(\cat{C}(X,-),F) \iso FX$
both naturally in $X \in \cat{C}$ and $F \in [\cat{C}, \Set]$
\end{lemma}

\noindent You can also write that one the dual way:
\begin{lemma}[Yoneda]\label{yoneda4-dual} For any locally small category $\cat{C}$, object
$X \in \cat{C}$, and functor $F:\cat{C}^{\op} \to \Set$ we have  $\Nat(\cat{C}(-,X),F)
\iso FX$ both naturally in $X \in \cat{C}$ and $F \in [\cat{C}^{\op}, \Set]$
\end{lemma}
\noindent
This version peels away some of the layers:

\begin{lemma}[Yoneda]\label{yoneda5}
Let $\cat{C}$ be a locally small category, let $X$ be an object of $\cat{C}$, and let $F: \cat{C} \to \Set$ be a functor. Then

(i) There is a bijection between natural transformations $\cat{C}(X, -) \to F$ and the elements of $FX$

(ii) The bijection in (i) is natural in both $F$ and $X$.
\end{lemma}

\noindent
I've made these look a bit more uniform than they do in real life just by using the same style of typesetting for all of them. But, there is still a lot of notation flying around, so let's get into it.

\section{Categories}

\begin{defn}
A {\it category}%
%
\index{category}
%
$\cat{C}$ consists of:
% 
\begin{itemize}
\item 
A collection of objects denoted $\objc$. In general we won't bother to write out $\objc$
and instead just say $\cat{C}$ when the context makes clear what is going on.
\item 
For each $A, B \in \objc$, a collection $\Arrows_{\cat{C}}(A, B)$%
of {\it arrows} or {\it mappings} or {\it functions} or {\it morphims} from $A$ to $B$. We will usually write $\Arrows(A,B)$ when it's clear what category $A$ and $B$ come from. Or we won't bother at all and just write $f: A \to B$ to mean $f \in \Arrows_{\cat{C}}(A,B)$.
\item
For each $A, B, C \in \objc$ a function called {\it composition} that takes $f:A \to B$ and $g : B \to C$ and maps it to $g \circ f$.
\item
For each $A \in \objc$ a function $1_A: A \to A$, called the {\it identity} at $A$ that
maps $A$ to itself. \end{itemize}%
Generally we denote objects in categories using upper case letters ($A, B,C$, etc) and
arrows using lower case letters ($f, g, h$, etc).

Finally, The objects and arrows of a category satisfy the following rules:
\begin{itemize}
\item For any $f: A \to B$ we have that  $1_B \circ f$ and $f \circ 1_A$ are both equal to $f$.
\item Given $f: A \to B$, $g: B \to C$, $h: C\to D$ we have that $(h \circ g) \circ f = h \circ (g \circ f)$.%
\end{itemize}%
\end{defn}%
\noindent
Often people write $\Hom(A, B)$ or $\Hom_{\cat{C}}(A,B)$ to mean $\Arrows(A,B)$. ``Hom'' stands for homomorphism, which is a word that is used a lot in mathematics to mean mappings of functions that preserve some kind of structure. Presumably this is also where the term {\it morphism} comes from. I like {\it arrow} because it just sounds more friendly and less abstract.

At this point the category theory book will list a few dozen examples of categories that maybe help you work out what's going on and maybe don't. One thing is for sure: they will all have short names that have been truncated in a bizzarre way ($\cat{Meas}$, $\cat{Toph}$, $\cat{Grp}$, $\cat{FinOrd}$, etc), and they will all be typeset in a different font. Often you will see {\sc small caps} or {\sf sans serif}. I picked {\bf bold} for convenience.


\section{Sets of Mappings in Categories}

\section{Functors}

\section{Natural Transformations}

\section{Functor Categories}

\section{The Full Stack}

\section{Who I Stole From}



\end{document}
