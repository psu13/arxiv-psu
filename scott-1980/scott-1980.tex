\documentclass[12pt]{article}

%\usepackage{amsmath,amsthm,amscd,amssymb}
\usepackage[colorlinks=true
,breaklinks=true
,urlcolor=blue
,anchorcolor=blue
,citecolor=blue
,filecolor=blue
,linkcolor=blue
,menucolor=blue
,linktocpage=true]{hyperref}
\hypersetup{
bookmarksopen=true,
bookmarksnumbered=true,
bookmarksopenlevel=10
}
\usepackage[noBBpl,sc]{mathpazo}
\usepackage[papersize={6.5in, 10.0in}, left=.5in, right=.5in, top=1in, bottom=.9in]{geometry}
\linespread{1.05}
\sloppy
\raggedbottom
\pagestyle{plain}
\usepackage{eulervm}
\usepackage{mathpartir}

% these include amsmath and that can cause trouble in older docs.
\makeatletter
\@ifpackageloaded{amsmath}{}{\RequirePackage{amsmath}}

\DeclareFontFamily{U}  {cmex}{}
\DeclareSymbolFont{Csymbols}       {U}  {cmex}{m}{n}
\DeclareFontShape{U}{cmex}{m}{n}{
    <-6>  cmex5
   <6-7>  cmex6
   <7-8>  cmex6
   <8-9>  cmex7
   <9-10> cmex8
  <10-12> cmex9
  <12->   cmex10}{}

\def\Set@Mn@Sym#1{\@tempcnta #1\relax}
\def\Next@Mn@Sym{\advance\@tempcnta 1\relax}
\def\Prev@Mn@Sym{\advance\@tempcnta-1\relax}
\def\@Decl@Mn@Sym#1#2#3#4{\DeclareMathSymbol{#2}{#3}{#4}{#1}}
\def\Decl@Mn@Sym#1#2#3{%
  \if\relax\noexpand#1%
    \let#1\undefined
  \fi
  \expandafter\@Decl@Mn@Sym\expandafter{\the\@tempcnta}{#1}{#3}{#2}%
  \Next@Mn@Sym}
\def\Decl@Mn@Alias#1#2#3{\Prev@Mn@Sym\Decl@Mn@Sym{#1}{#2}{#3}}
\let\Decl@Mn@Char\Decl@Mn@Sym
\def\Decl@Mn@Op#1#2#3{\def#1{\DOTSB#3\slimits@}}
\def\Decl@Mn@Int#1#2#3{\def#1{\DOTSI#3\ilimits@}}

\let\sum\undefined
\DeclareMathSymbol{\tsum}{\mathop}{Csymbols}{"50}
\DeclareMathSymbol{\dsum}{\mathop}{Csymbols}{"51}

\Decl@Mn@Op\sum\dsum\tsum

\makeatother

\makeatletter
\@ifpackageloaded{amsmath}{}{\RequirePackage{amsmath}}

\DeclareFontFamily{OMX}{MnSymbolE}{}
\DeclareSymbolFont{largesymbolsX}{OMX}{MnSymbolE}{m}{n}
\DeclareFontShape{OMX}{MnSymbolE}{m}{n}{
    <-6>  MnSymbolE5
   <6-7>  MnSymbolE6
   <7-8>  MnSymbolE7
   <8-9>  MnSymbolE8
   <9-10> MnSymbolE9
  <10-12> MnSymbolE10
  <12->   MnSymbolE12}{}

\DeclareMathSymbol{\downbrace}    {\mathord}{largesymbolsX}{'251}
\DeclareMathSymbol{\downbraceg}   {\mathord}{largesymbolsX}{'252}
\DeclareMathSymbol{\downbracegg}  {\mathord}{largesymbolsX}{'253}
\DeclareMathSymbol{\downbraceggg} {\mathord}{largesymbolsX}{'254}
\DeclareMathSymbol{\downbracegggg}{\mathord}{largesymbolsX}{'255}
\DeclareMathSymbol{\upbrace}      {\mathord}{largesymbolsX}{'256}
\DeclareMathSymbol{\upbraceg}     {\mathord}{largesymbolsX}{'257}
\DeclareMathSymbol{\upbracegg}    {\mathord}{largesymbolsX}{'260}
\DeclareMathSymbol{\upbraceggg}   {\mathord}{largesymbolsX}{'261}
\DeclareMathSymbol{\upbracegggg}  {\mathord}{largesymbolsX}{'262}
\DeclareMathSymbol{\braceld}      {\mathord}{largesymbolsX}{'263}
\DeclareMathSymbol{\bracelu}      {\mathord}{largesymbolsX}{'264}
\DeclareMathSymbol{\bracerd}      {\mathord}{largesymbolsX}{'265}
\DeclareMathSymbol{\braceru}      {\mathord}{largesymbolsX}{'266}
\DeclareMathSymbol{\bracemd}      {\mathord}{largesymbolsX}{'267}
\DeclareMathSymbol{\bracemu}      {\mathord}{largesymbolsX}{'270}
\DeclareMathSymbol{\bracemid}     {\mathord}{largesymbolsX}{'271}

\def\horiz@expandable#1#2#3#4#5#6#7#8{%
  \@mathmeasure\z@#7{#8}%
  \@tempdima=\wd\z@
  \@mathmeasure\z@#7{#1}%
  \ifdim\noexpand\wd\z@>\@tempdima
    $\m@th#7#1$%
  \else
    \@mathmeasure\z@#7{#2}%
    \ifdim\noexpand\wd\z@>\@tempdima
      $\m@th#7#2$%
    \else
      \@mathmeasure\z@#7{#3}%
      \ifdim\noexpand\wd\z@>\@tempdima
        $\m@th#7#3$%
      \else
        \@mathmeasure\z@#7{#4}%
        \ifdim\noexpand\wd\z@>\@tempdima
          $\m@th#7#4$%
        \else
          \@mathmeasure\z@#7{#5}%
          \ifdim\noexpand\wd\z@>\@tempdima
            $\m@th#7#5$%
          \else
           #6#7%
          \fi
        \fi
      \fi
    \fi
  \fi}

\def\overbrace@expandable#1#2#3{\vbox{\m@th\ialign{##\crcr
  #1#2{#3}\crcr\noalign{\kern2\p@\nointerlineskip}%
  $\m@th\hfil#2#3\hfil$\crcr}}}
\def\underbrace@expandable#1#2#3{\vtop{\m@th\ialign{##\crcr
  $\m@th\hfil#2#3\hfil$\crcr
  \noalign{\kern2\p@\nointerlineskip}%
  #1#2{#3}\crcr}}}

\def\overbrace@#1#2#3{\vbox{\m@th\ialign{##\crcr
  #1#2\crcr\noalign{\kern2\p@\nointerlineskip}%
  $\m@th\hfil#2#3\hfil$\crcr}}}
\def\underbrace@#1#2#3{\vtop{\m@th\ialign{##\crcr
  $\m@th\hfil#2#3\hfil$\crcr
  \noalign{\kern2\p@\nointerlineskip}%
  #1#2\crcr}}}

\def\bracefill@#1#2#3#4#5{$\m@th#5#1\leaders\hbox{$#4$}\hfill#2\leaders\hbox{$#4$}\hfill#3$}

\def\downbracefill@{\bracefill@\braceld\bracemd\bracerd\bracemid}
\def\upbracefill@{\bracefill@\bracelu\bracemu\braceru\bracemid}

\DeclareRobustCommand{\downbracefill}{\downbracefill@\textstyle}
\DeclareRobustCommand{\upbracefill}{\upbracefill@\textstyle}

\def\upbrace@expandable{%
  \horiz@expandable
    \upbrace
    \upbraceg
    \upbracegg
    \upbraceggg
    \upbracegggg
    \upbracefill@}
\def\downbrace@expandable{%
  \horiz@expandable
    \downbrace
    \downbraceg
    \downbracegg
    \downbraceggg
    \downbracegggg
    \downbracefill@}

\DeclareRobustCommand{\overbrace}[1]{\mathop{\mathpalette{\overbrace@expandable\downbrace@expandable}{#1}}\limits}
\DeclareRobustCommand{\underbrace}[1]{\mathop{\mathpalette{\underbrace@expandable\upbrace@expandable}{#1}}\limits}

\makeatother


\usepackage[small]{titlesec}
\usepackage{cite}

% make sure there is enough TOC for reasonable pdf bookmarks.
\setcounter{tocdepth}{3}

%\usepackage[dotinlabels]{titletoc}
%\titlelabel{{\thetitle}.\quad}
%\usepackage{titletoc}
\usepackage[small]{titlesec}

\titleformat{\section}[block]
  {\fillast\medskip}
  {\bfseries{\thesection. }}
  {1ex minus .1ex}
  {\bfseries}
 
\titleformat*{\subsection}{\itshape}
\titleformat*{\subsubsection}{\itshape}

\setcounter{tocdepth}{2}

\titlecontents{section}
              [2.3em] 
              {\bigskip}
              {{\contentslabel{2.3em}}}
              {\hspace*{-2.3em}}
              {\titlerule*[1pc]{}\contentspage}
              
\titlecontents{subsection}
              [4.7em] 
              {}
              {{\contentslabel{2.3em}}}
              {\hspace*{-2.3em}}
              {\titlerule*[.5pc]{}\contentspage}

% hopefully not used.           
\titlecontents{subsubsection}
              [7.9em]
              {}
              {{\contentslabel{3.3em}}}
              {\hspace*{-3.3em}}
              {\titlerule*[.5pc]{}\contentspage}
%\makeatletter
\renewcommand\tableofcontents{%
    \section*{\contentsname
        \@mkboth{%
           \MakeLowercase\contentsname}{\MakeLowercase\contentsname}}%
    \@starttoc{toc}%
    }
\def\@oddhead{{\scshape\rightmark}\hfil{\small\scshape\thepage}}%
\def\sectionmark#1{%
      \markright{\MakeLowercase{%
        \ifnum \c@secnumdepth >\m@ne
          \thesection\quad
        \fi
        #1}}}
        
\makeatother

%\makeatletter

 \def\small{%
  \@setfontsize\small\@xipt{13pt}%
  \abovedisplayskip 8\p@ \@plus3\p@ \@minus6\p@
  \belowdisplayskip \abovedisplayskip
  \abovedisplayshortskip \z@ \@plus3\p@
  \belowdisplayshortskip 6.5\p@ \@plus3.5\p@ \@minus3\p@
  \def\@listi{%
    \leftmargin\leftmargini
    \topsep 9\p@ \@plus3\p@ \@minus5\p@
    \parsep 4.5\p@ \@plus2\p@ \@minus\p@
    \itemsep \parsep
  }%
}%
 \def\footnotesize{%
  \@setfontsize\footnotesize\@xpt{12pt}%
  \abovedisplayskip 10\p@ \@plus2\p@ \@minus5\p@
  \belowdisplayskip \abovedisplayskip
  \abovedisplayshortskip \z@ \@plus3\p@
  \belowdisplayshortskip 6\p@ \@plus3\p@ \@minus3\p@
  \def\@listi{%
    \leftmargin\leftmargini
    \topsep 6\p@ \@plus2\p@ \@minus2\p@
    \parsep 3\p@ \@plus2\p@ \@minus\p@
    \itemsep \parsep
  }%
}%
\def\open@column@one#1{%
 \ltxgrid@info@sw{\class@info{\string\open@column@one\string#1}}{}%
 \unvbox\pagesofar
 \@ifvoid{\footsofar}{}{%
  \insert\footins\bgroup\unvbox\footsofar\egroup
  \penalty\z@
 }%
 \gdef\thepagegrid{one}%
 \global\pagegrid@col#1%
 \global\pagegrid@cur\@ne
 \global\count\footins\@m
 \set@column@hsize\pagegrid@col
 \set@colht
}%

\def\frontmatter@abstractheading{%
\bigskip
 \begingroup
  \centering\large
  \abstractname
  \par\bigskip
 \endgroup
}%

\makeatother

%\DeclareSymbolFont{CMlargesymbols}{OMX}{cmex}{m}{n}
%\DeclareMathSymbol{\sum}{\mathop}{CMlargesymbols}{"50}

\usepackage[numbers]{natbib}
%\setcitestyle{numbers}
\bibliographystyle{unsrtnat}
\date{}
\def\to{\rightarrow}
\def\union{\cup}
\def\inc{\subseteq}
\def\dom{\mathop{\rm dom}}
\def\cod{\mathop{\rm cod}}
\def\id{{\rm 1}}
\def\comp{\circ}

\title{Relating Theories of the $\lambda$-Calculus}
\author{Dana Scott \\
{\small\it Merton College}\\
{\small\it Oxford}}
\begin{document}
\maketitle
\bigskip
{\centerline
{\small\it Dedicated to Professor H. B. Curry on the occasion of his 80th Birthday}}
\bigskip

\medskip
\noindent
Mathematical theories arise for many different reasons, sometimes in connection with specific applications and often owing to accidental inspiration. From time to time we ought to ask ourselves concerning our theories where should they have come from; usually the answer will have little to do with the exact historical development. The $\lambda$-calculus is, I feel, a case in point. In Scott (1980), in the Kleene Festschrift, I made up a story of where the theory of type-free $\lambda$-calculus could have come from. Any number of people who heard my lec­ture and read the manuscript were cross with me. They said ``But it didn't develop that way! And besides we doubt it ever would have.'' But this reaction misses the point of my story. I shall not, however repeat the earlier story here, for the point of the present paper is different. For those people who do not like to discuss philosophy --- even Philosophy of Mathematics --- my remarks here can be taken as a suggestion of how to group diverse models of $\lambda$-calculus rather uniformly under a general scheme. The scheme is by now rather well known and not at all original with me. What I hope can be regarded as a useful con­tribution is my putting of the ideas in a certain order. As I consider the order to be a natural one, I feel there is a philosophical significance to my activity; but I should not want to force this view on anyone.

\section{Theories of Functions}


Everyone agrees that $\lambda$-calculus is a theory of functions.
But we must ask: ``What kind of a theory?'' And also: ``Have we got the best theory?'' Personally, I think we should also in­quire: ``How does it relate to other theories?'' I certainly find many discussions far too silent on this last issue. Well, what other theories are there? Certainly set theory comes to mind at once, and no set theory would be worth its salt if it did not provide a theory of functions. Let us not try to catalogue the various known theories here but look at a theory in the style of Zermelo --- and we do not have even to be too specific, since in any case such a theory is very standard. What is ``unsatisfactory'' about Zermelo's theory is the limiti­zation-of-size view of sets: any {\it one} set $A$ is extremely small compared to the size of $V$, the class or universe of all sets. Thus, functions $f : A \to B$ mapping one set $A$ into another set $B$ tell us very little about operations on all sets, maps on $V$ into $V$. We therefore have an urge to ``improve'' our set theory by constructing a class theory. Sets are elements $A \in V$; while classes are sub-collections $B \subseteq V$. As $V$ is (by the usual assump­tions) so highly closed under so many operations, we have no difficulty in construing certain classes as maps $F: V \to V$. For example for all $X \in V$ we could have $F(X) = \{X\}$ or $F(X) = A \times X$ (where $A$ is a fixed set).

The passage from sets to classes is a familiar and useful move in the formalization of the theory: many things can be done generally for classes and then specialized to sets. And having a notation for functions defined on all sets is in many cases a great advantage. But wait. What about operations on classes? What should we say about them? Given any two classes $A$ and $B$, we can form their union, $A \union  B$. The operation, $\union\!: V \times V \to V$, of union of {\it sets} does not directly apply to {\it classes} even though there is a connection. Do we also want a theory of class operations? Do we have to go to hyperclasses (classes of classes)? Is there any end to this expansion?


[{\it An Aside}: The story of Scott (1980) was meant to suggest one answer --- the one known to Plotkin (1972). Namely, we con­sider only ``continuous'' class operations. These are objects $F$ such that $F(X)$ is defined for every class $X \subseteq V$ and $F(X)$ is a class, too. Moreover $F$ should satisfy:

\begin{enumerate}

\item $X \inc Y$ always implies $F(X)\inc F(Y)$

\item Whenever $A \inc F(X)$ and $A$ is a set, then $A\inc F(B)$ for some set $B \inc X$.

\end{enumerate}
%
We do not have time to discuss the justification of the word "continuous" here; suffice it to say that conditions (1) and (2) are not as strict as they at first might seem. Every ordinary map $f : V \to V$ determines a continuous class operator by the def­inition:

$$
F(X) = \{f(x) \mid x \in X\}.
$$
%
Furthermore, $F$ determines $f$, for we have: $y = f(x)$ if and only if $\{y\} = F(\{x\})$,
for all $x,y \in V$. In a suitable sense, then, nothing has been lost; but what has been gained?
The reply is that continuous class operators can be identi­fied with classes. We could write, for instance:

$$ F = \{(A,B) \mid A,B \in V {\textrm { and }} A \inc F(B) \},$$
where, say:

$$
(A,B) = \{\{A\}, \{A,B\}\},
$$
More in harmony with Scott (1980) would be: $F = \{(a,B) \mid  a,B \in V {\textrm { and } } a \in F(B)\}$.

Either trick reduces operator theory to class theory - in the continuous case. And the same trick could be carried over to other kinds of set theory (e.g. Quine's). What we know is that operator theory gives a model for $\lambda$-calculus; it is a quite elementary model, too. Nice as this connection is, it is not the topic of the pre­ sent paper: we do not want to make $\lambda$-calculus depend on set theory, since then we have still to explain where set theory comes from. But the connection should be borne in mind.]

Perhaps set theory brings in too many extraneous issues. $V$, after all, is a massive object closed under all manner of strange operations. What we are probably seeking is a ``purer'' view of functions: a theory of functions in themselves, not a theory of functions derived from sets. What, then, is a pure theory of functions? Answer: category theory. General category theory is a very pure theory: it is the milk-and-water theory of functions under composition. This composition operation is associative and possesses neutral ele­ments (compositions of zero terms). That is about all you can say about it except to stress that it is also a rather bland theory of types. Every function $f$ has a (unique) domain and codomain, and we write:

$$f: \dom f \to \cod f$$
%
Every possible domain is a codomain (and conversely), because if $A$ is such, then

$$\dom \id_A = A = \cod \id_A$$
%
where $\id_A$ is the neutral element of type $A$ (If we want to be especially parsimonious in entities, we can even write $\id_A = A$, because each of $\id_A$ and $A$ uniquely deter­mines the other.). 

The point of distinguishing domains and codomains is that not only do they specify the type of $f$, but a composition $g \comp f$ is
defined if, and only if, $\dom g = \cod f$. And then $\dom(g \comp f) = \dom f$ and $\cod (g \comp f) = \cod (g)$. We usually write this as a ``rule of inference'':

$$
\inferrule
  {{f: A \to B} \\ {g: B \to C}}
  {g \comp f: A \to C}
$$
with the understanding that the typing of $f \comp  g$ can only be ob­ tained by such an application of the rule. The types, then, are invoked just to type functions, and the only theory involv­ed is that of the ``transition'' of types under composition. 

Sets (and set-theoretical mappings) do of course form a cat­egory; category theory is meant to be more general than set theory. We should construe the function entities here as tri­ples of sets $(A,f,B)$ where
$$
f \inc A \times B {\textrm{ and }}\, \forall x \in A \,\, \exists! y \in B\,\, s.t. \,\, (x,y) \in f.
$$
%
The definition of composition is obvious. Sets, in this way, give us only one special example of a category.
I beg forgiveness of the reader for boring him. All of this is well known to the moderately awake undergraduate in mathe­matics. Indeed, that is the point: there is plenty of evi­dence now that category theory is a natural and useful theory of functions. I do not have to rehearse the examples as they can be found in any number of books (e.g. Mac Lane (1971]).
There is a rather important logical point to stress, however
important for anyone who has thought about $\lambda$-calculus models. Category theory is very extensional. We assume as axioms the equations:
%
$$\id_A \comp f = f \comp \id_B =  f$$
%
and 
$$h\comp (g \comp f) = (h \comp g) \comp f,$$
%
provided $A= \dom f$ and $B = \cod f$ and the double compositions are defined. These are {\it functional} equations, and they say that two functions defined in different ways are in fact {\it identical}.
Furthermore, identical things can everywhere replace one another.

This point about extensionality may not seem exciting or im­portant, but the logician should remember that, in certain intensional theories of functions, ``obvious'' definitions will not provide categories. We shall return to this point later. But is category theory the long-sought answer? No, no, not at all. Category theory {\it pure} provides nothing explicitly aside from identity functions --- and they occur only if we have some possible domain. We do get compositions if we have the neces­sary terms. Thus, as it stands, category theory has no existential import (It was not meant to.). Set theory has "too much" existential import (It was meant to.). What we seek is the middle way --- and an argument that the middle way is natural and general.

There is no need to build up unnecessary suspense: the middle way is the theory of the (so called) cartesian closed categories.
Fortunately Lambek has written extensively about the theory, and I can refer the reader to his papers for fur­ther details; I also am happy to acknowledge his writings as helping me understand what is going on. If we remark that his paper in this volume is called "From $\lambda$-calculus to cartesian closed categories'', then we might say that my present paper ought to be called ``From cartesian closed categories to $\lambda$-calculus.'' 
I am trying to find out where $\lambda$-calculus should come from, and the fact that the notion of a cartesian closed category (c.c.c) is a late developing one (Eilenberg \& Kelly (1966)), is not relevant to the argument: I shall try to explain in my own words in the next section why we should look to it {\it first}.

\section{A Theory of Types}

I say "a theory", because there are many possible theories; indeed pure category theory is one of the theories. 
Its weak­ness lies in the fact that we are given no construction principles, no way of making new types from old. From the point of view of logic what should we expect? What more do we want to say beyond relations between types which hold when a mapping statement $f : A\to	B$ obtains.

An immediate question that must come to anyone's mind con­cerns the arity of functions. The usual way of reading a map­ping statement is to take it as a statement about {\it one-place} functions, and the $\circ$ of composition is the composition of one­ place functions. This seems very restricted.

People have suggested generalizing categories to multi-place functions with concomitant compositions (e.g. the book Szabo (1978)), but it does not seem the neatest solution. Much easier is to assume that the category has cartesian products - and more specifically particular representatives of the product
domains are chosen. As a special case we will know what the cartesian power $A^n$ is for each $n=0,1,2,\cdots$, and $n$-ary functions
are then maps $f: A^n\to B$. Not much of a surprise.

We have to take care, however. In the first place, a given category may not have cartesian products (it fails to have enough types). Even if it does, the maps allowed may be too restricted - for logical purposes. Take the category of groups and homomorphisms, for example. The required products exist.
A map $f : A^2 \to B$ in this category has to be a group homomor­phism, naturally. Suppose two maps $u,v: A \to B$ were given.
Intuitively we think in terms of elements and that we are map­ping $x \mapsto u(x)$ and $y \mapsto v(y)$. The pointwise group product of the maps, namely, $x,y \mapsto  u(x) \dot v(y)$ is a very nice map $g : A^2 \to B$ in the ordinary sense --- but unless the group $B$ is abelian, $g$ is not a homomorphism. It is a ``logical'' map but not an ``algebra­ic'' map. Pure category theory applies to many algebraic situa­tions (as everyone knows that is why it is a good theory), but not all categories are ``logical'' even if they have products. In the example of groups, what was "missing" was the group 
multiplication $\mu: B^2 \to B$ as a map in the category. (Inverse is missing as well, since it reverses order.) There is an in­teresting theory of algebraic theories that address the ques­tion of the proper categorial construction of categories of al­gebras, but I do not think we should invoke that theory here.



\end{document}
