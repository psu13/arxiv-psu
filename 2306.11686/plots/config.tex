\pgfplotsset{
    enlarge y limits={rel=0.01},
    enlarge x limits={abs=0.55},
    %only marks,
    ybar,
    xtick={0,1},
    xticklabels={},
    xmin=0, xmax=0,
    %xticklabel style = {rotate=45,anchor=east,align=center,xshift=1mm,yshift=-4mm},
    ylabel={kernel speedup relative to parallel region on the CPU},
    ylabel style = {align=center,at={(-0.05,0.55)},text width=4cm},
    xmajorticks=false,
%    ymode = log,
    %log basis y = 2,
    log ticks with fixed point,
    %log origin=infty,
    %point meta=rawy,
    visualization depends on ={y \as \y},
    visualization depends on ={rawy \as \rawy},
    %visualization depends on ={rawx \as \rawx},
    nodes near coords = {%
      \begingroup%
        % this group is merely to switch to FPU locally.
        % Might be unnecessary, but who knows.
        \pgfkeys{/pgf/fpu}%
        \pgfmathparse{\pgfplotspointmeta<0.0}%
        \global\let\isneg=\pgfmathresult%
        \pgfmathparse{\pgfplotspointmeta==0}%
        \global\let\iszero=\pgfmathresult%
        \pgfmathparse{\pgfplotspointmeta==1}%
        \global\let\isone=\pgfmathresult%
      \endgroup%
      \pgfmathfloatcreate{0}{0.0}{0}%
      \let\ZERO=\pgfmathresult%
      \pgfmathfloatcreate{1}{1.0}{0}%
      \let\ONE=\pgfmathresult%
      \pgfmathfloatcreate{1}{-1.0}{0}%
      \let\MONE=\pgfmathresult%
      \color{black}%
      \global\let\ycoord=\ZERO
%      \global\let\ycoord=\rawy
    \ifx\iszero\ONE%
    \else%
        \ifx\isneg\ONE%
        {%
          \node[fill=white,yshift=-13pt,xshift=1pt,inner sep=0pt]{\color{black}\textbf{\pgfmathprintnumber[precision=2]{\rawy}}};%
        }%
        \else%
          \node[fill=white,yshift=-2pt,xshift=1pt,inner sep=0pt]{\color{black}\textbf{\pgfmathprintnumber[precision=2]{\rawy}}};%
        \fi%
    \fi%
    },
%    restrict y to domain*={
%            \pgfkeysvalueof{/pgfplots/ymin}:\pgfkeysvalueof{/pgfplots/ymax}
%    },
     extra y tick style={
           grid=major,
           tick label style={rotate=90},
           ticklabel pos=right,
     },
    width=\linewidth,
    height=64mm,
}
