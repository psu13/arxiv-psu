
\section{Evaluation}
\label{sec:evaluation}

%\subsection{System Configuration}

To evaluate the performance of our approach, we used a system comprising an NVIDIA A100 Tensor Core GPU (40GB) with AMD EPYC 7532 processors (32 cores and hyper-threading disabled) and 256 GB DDR4 RAM.
We performed all experiments using CUDA 11.8.0 and compiled all benchmarks using the \lstinline{-O3} optimization flag.
Our prototype version is based on LLVM trunk (\raisebox{-1.5pt}{\faIcon[regular]{git}} \lstinline{50f1476a}).

\subsection{Allocator Performance}

We believe an application should pick an allocator based on their specific needs.
Some of the evaluated SPEC OMP benchmarks concurrently allocate many memory regions at the beginning of a parallel, and deallocate them at the end of the parallel region again.
To best support this scheme we introduced the balanced allocator described in \Cref{sec:allocators}.
%In addition to the specialization for these balanced allocations it tracks the allocated objects and allows us to identify underlying objects at runtime if needed.
In \Cref{fig:malloc} shows how it performs on a synthetic benchmark in which all threads in all teams allocate memory at the beginning of the kernel, use it briefly, and then deallocate it again.
This design is an exaggeration of the SPEC OMP benchmarks to stress test  allocators.
On our test platform the domain specific balanced allocator is between $3.3\times$ (1 thread, 1 team) and $30\times$ (32 threads, 256 teams) faster than the default NVIDIA provided \lstinline|malloc|.

\begin{figure}[hbt]
    \centering
    \resizebox{.9\linewidth}{!}{
      \begin{tikzpicture}
\begin{axis}[%
    legend style={
      at={(0, 1.3)},
      anchor=north west,
      draw=none,
      fill=none,
      %/tikz/every even column/.append style   = {column sep= 0mm, text width=.7\linewidth},
      %/tikz/every odd column/.append style    = {column sep= 0mm, text width=0em},
    },
      legend cell align={left},
scatter/classes={%
    w1={mark=o,draw=col1,ultra thick,scale=1.6},
    w32={mark=o,draw=col3,ultra thick,scale=1.6},
    w64={mark=o,draw=col5,ultra thick,scale=1.6},
    w128={mark=o,draw=col6,ultra thick,scale=1.6},
    w256={mark=o,draw=col7,ultra thick,scale=1.6},
    c1={mark=x,draw=col1,ultra thick,scale=1.5},
    c32={mark=x,draw=col3,ultra thick,scale=1.5},
    c64={mark=x,draw=col5,ultra thick,scale=1.5},
    c128={mark=x,draw=col6,ultra thick,scale=1.5},
    c256={mark=x,draw=col7,ultra thick,scale=1.5}
    },
    ymode = log,
xtick      ={1,  2,  3,   4,   5},
xticklabels={1, 32, 64, 128, 256},
xlabel = {number of threads per team},
xtick pos=bottom,
    xtick style={line width=20pt,yshift=-2pt},
    %xmajorticks=false,
    ylabel = {execution time in seconds},
    ylabel style = {align=center,at={(-0.12,0.55)},text width=4cm},
]
    
\addlegendimage{only marks,mark=o,ultra thick,scale=1.6}
\addlegendentry{Balanced Allocator (32 thread slots, 16 team slots)}
\addlegendimage{only marks,mark=x,ultra thick,scale=1.5}
\addlegendentry{NVIDIA malloc}

\addplot[scatter,only marks,%
    scatter src=explicit symbolic]%
table[meta=label] {
x y label
0.8 0.000005943 w1 
1.8 0.000008355 w1
2.8 0.000011715 w1
3.8 0.000019677 w1
4.8 0.000040061 w1
0.9 0.000008448 w32
1.9 0.000018851 w32
2.9 0.000027302 w32
3.9 0.000045264 w32
4.9 0.000086458 w32 
1.0 0.000013222 w64
2.0 0.000027923 w64
3.0 0.000046509 w64
4.0 0.000085261 w64
5.0 0.000232615 w64 
1.1 0.000023645 w128
2.1 0.000044717 w128
3.1 0.000086179 w128
4.1 0.000234955 w128
5.1 0.000865904 w128 
1.2 0.000023645 w256
2.2 0.000044717 w256
3.2 0.000086179 w256
4.2 0.000234955 w256
5.2 0.000865904 w256
0.8 0.000020083 c1
1.8 0.000086080 c1
2.8 0.000105703 c1
3.8 0.000154804 c1
4.8 0.000213147 c1
0.9 0.000037354 c32
1.9 0.000361153 c32
2.9 0.000542578 c32
3.9 0.000923219 c32
4.9 0.001336460 c32
1.0 0.000056922 c64
2.0 0.000553829 c64
3.0 0.000865383 c64
4.0 0.001313250 c64
5.0 0.001959345 c64
1.1 0.000093300 c128
2.1 0.000877581 c128
3.1 0.001320811 c128
4.1 0.002102296 c128
5.1 0.003548110 c128
1.2 0.000165137 c256
2.2 0.001336754 c256
3.2 0.002075279 c256
4.2 0.003669188 c256
5.2 0.007008149 c256
};
\end{axis}

\begin{axis}[%
    legend style={
      at={(0.5, 1.08)},
      anchor=center,
      draw=none,
      %small,
      %cell align={left},
%      /tikz/every even column/.append style   = {column sep= 0mm, text width=.7\linewidth},
      %/tikz/every odd column/.append style    = {column sep= 0mm, text width=0em},
    },
    legend columns=5,
scatter/classes={%
    w1={mark=o,draw=col1,ultra thick,scale=1.6},
    w32={mark=o,draw=col3,ultra thick,scale=1.6},
    w64={mark=o,draw=col5,ultra thick,scale=1.6},
    w128={mark=o,draw=col6,ultra thick,scale=1.6},
    w256={mark=o,draw=col7,ultra thick,scale=1.6},
    c1={mark=x,draw=col1,ultra thick,scale=1.5},
    c32={mark=x,draw=col3,ultra thick,scale=1.5},
    c64={mark=x,draw=col5,ultra thick,scale=1.5},
    c128={mark=x,draw=col6,ultra thick,scale=1.5},
    c256={mark=x,draw=col7,ultra thick,scale=1.5}
    },
    hide axis,
    only marks,
    ymin=0, ymax=1,
    xmin=0, xmax=1,
]
\addplot+ [mark=*] coordinates { };
\addlegendimage{only marks,mark=square*,ultra thick,scale=1.6,col1}
\addlegendentry{1 Team}
\addlegendimage{only marks,mark=square*,ultra thick,scale=1.6,col3}
\addlegendentry{32 Teams}
\addlegendimage{only marks,mark=square*,ultra thick,scale=1.6,col5}
\addlegendentry{64 Teams}
\addlegendimage{only marks,mark=square*,ultra thick,scale=1.6,col6}
\addlegendentry{128 Teams}
\addlegendimage{only marks,mark=square*,ultra thick,scale=1.6,col7}
\addlegendentry{256 Teams}
\end{axis}

\end{tikzpicture}


 
    }
    \vspace{-3mm}
    \caption{
    Comparison of the performance between the NVIDIA-provided \lstinline{malloc} and our domain-specific balanced allocator with 32 thread slots and 16 team slots (refer to ~\Cref{sec:allocators}).
    The benchmark is an exaggeration of the allocation scheme in some SPEC OMP benchmarks, where memory is allocated and deallocated at the beginning and end of a parallel region with all threads.
    }
    \Description{
    Comparison of the performance between the NVIDIA-provided \lstinline{malloc} and our domain-specific balanced allocator with 32 thread slots and 16 team slots (refer to ~\Cref{sec:allocators}).
    The benchmark is an exaggeration of the allocation scheme in some SPEC OMP benchmarks, where memory is allocated and deallocated at the beginning and end of a parallel region with all threads.
    }
    \label{fig:malloc}
\vspace{-3mm}
\end{figure}

\subsection{RPC Performance}


To measure the overhead of an RPC call, we conducted a profiling experiment where we called the \lstinline{fprintf} function 1000 times with the following arguments: \lstinline{fprintf(stderr, "fread reads: %s.\n", buffer)}. 
In this example, \lstinline|buffer| points to a 128 byte array that has to be copied back and forth as the read/write behavior of \lstinline|fprintf| arguments is unknown (without inspecting the format string).
The average device time spend per RPC was 975 microseconds.
The distribution of this time spend is visualized in \Cref{fig:rpc}.
From left to right, the following stages are traversed on the device (top part):
\begin{inparaenum}[1)]
\item $0.1$\% of the overall time is spent on initializing the RPC argument information (\lstinline{RPCArgInfo} in \Cref{fig:rpc_gen_device}).
\item $9.1\%$ of the time is on identifying the underlying objects of the three pointer arguments, which includes copying the format string and \lstinline{buffer} to an RPC buffer where the host can access them (managed memory, as described in \Cref{sec:background}). 
\item The device thread spends $89$\% of the time waiting for the host to act on the request and acknowledge that it has been performed.
\item $1.8\%$ of the time is spent copying the data from the RPC buffer back to the \lstinline{buffer}.
\end{inparaenum}
On the host, the time is spent like this, again from left to right:
\begin{inparaenum}[1)]
\item $2$\% of the overall time is spent on copying the RPC information (\lstinline{RPCInfo} in \Cref{fig:rpc_gen_host}) to the host.
\item $3.5$\% of the time is spent invoking the host wrapper, which in turn calls the actual \lstinline{fprintf} function and sets up the return value.
\item $5.4$ of the time is spent on copying \lstinline{RPCInfo} object back to the device and notifying completion.
This notification is done by setting an integer value to 0 that is in managed memory and is also accessible to the device.
\item The remaining $89.1$\% of the time is the gap between when the host notifies completion and when the device receives the notification.
This gap occurs because threads in a GPU kernel that is already running are not guaranteed to see the updates to memory done by the CPU or other devices in a specific order and within a specific time interval \cite{DBLP:conf/hipc/PotluriGRNVI17,DBLP:conf/hpdc/DaoudWS16}.
\end{inparaenum}

\begin{figure}[tb]
\lstset{basicstyle=\footnotesize\lst@ifdisplaystyle\small\linespread{0.94}\fi\ttfamily,
        }
    \centering
    \resizebox{.9\linewidth}{!}{
      
\newcommand{\timebar}[8]{
\draw[fill=#5] ( 0, 0) node (#4-1bottom) {} rectangle (#1, 1) node (#4-1top) {};
\draw[fill=#6] (#1, 0) node (#4-2bottom) {} rectangle (#2, 1) node (#4-2top) {};
\draw[fill=#7] (#2, 0) node (#4-3bottom) {} rectangle (#3, 1) node (#4-3top) {};
\ifnum10>0#3\relax
\draw[fill=#8] (#3, 0) node (#4-4bottom) {}  rectangle (10, 1) node (#4-4top) {};
\fi
}

\begin{tikzpicture}

\timebar{0.01118569379}{0.917913822}{9.819808828}{dev}{col1!30}{col2!30}{col3!30}{col4!30}
\node[rotate=90, anchor=south] at (0,0.5) {device};

\tikzset{shift={(0,-2)}}

\timebar{0.000431613533}{0.355022151501}{0.542755359371}{host}{col5!30}{col6!30}{col7!30}{col8!30}
\node[rotate=90, anchor=south] at (0,0.5) {host};

%\draw[->] (dev-2) -- (host-1);
\draw[ultra thick, ->] (dev-3bottom) -- (host-1top);
\draw[ultra thick, ->] (dev-4bottom) -- (host-4top);


\end{tikzpicture}

 
    }
    \vspace{-3mm}
    \caption{Visualization of the time (avg. total 975 microseconds) spend in different staging while resolving a \lstinline|fprintf| RPC.}
    \Description{Visualization of the time (avg. total 975 microseconds) spend in different staging while resolving a \lstinline|fprintf| RPC.}
    \label{fig:rpc}
\vspace{-4mm}
\end{figure}


% \subsection{Benchmarks}

% Our approach was evaluated using two sets of benchmarks.
% The first set comprised four proxy applications: XSBench, RSBench, AMGmk, and PageRank.
% XSBench and RSBench have both OpenMP CPU and OpenMP target offloading versions, while AMGmk and PageRank only have OpenMP CPU parallelism.
% The corresponding OpenMP target offloading versions for AMGmk and PageRank were obtained from HeCBench~\cite{zjin-lcf/HeCBench}, a GitHub repository that contains benchmarks written with various programming languages and frameworks for heterogeneous computing.
% The second set consisted of three C benchmarks (\lstinline{358.botsalgn}, \lstinline{359.botsspar}, and \lstinline{372.smithwa}) from the SPEC OMP 2012 suite~\cite{DBLP:conf/iwomp/MullerBBFHHJMPRSWWK12}, which are highly optimized for CPU execution but not well-suited for efficient GPU execution due to their programming style.
% Here is a brief introduction to each benchmark.

% \textbf{XSBench}~\cite{XS_Tramm_2014} and \textbf{RSBench}~\cite{DBLP:conf/easc/TrammSFJ14} are two proxy applications for the Open Monte Carlo (OpenMC) project~\cite{romano2013openmc}
% Both proxies compute the continuous energy macroscopic neutron cross-section lookup when studying neutron transport, and both are available in multiple programming languages and frameworks.
% While XSBench extracts one of the main kernels in OpenMC, which is memory-bound, RSBench provides a compute-bound alternative implementation.

% \textbf{AMGmk} is a geometric multigrid solver for the Poisson equation in three dimensions. It has a lot of coarse-grained parallelism.

% \textbf{PageRank} is used to evaluate the performance of graph algorithms, and in particular, the PageRank algorithm used by Google.
% It involves a lot of fine-grained parallelism.

% \textbf{\lstinline{358.botsalgn}} involves a pairwise alignment of DNA sequences and is characterized by a large number of small loops with little data reuse, which makes it difficult to parallelize efficiently on GPUs.

% \textbf{\lstinline{359.botsspar}} involves a sparse matrix-vector multiplication and is characterized by indirect memory accesses and irregular data access patterns, which make it difficult to parallelize efficiently on GPUs.

% \textbf{\lstinline{372.smithwa}} involves the Smith-Waterman algorithm for sequence alignment and is characterized by a lot of nested loops with indirect memory accesses, which makes it difficult to parallelize efficiently on GPUs.

\subsection{Parallel Region Modeling}

In the following we compare the performance obtained via \emph{GPU First} compilation of CPU code against manual offload versions of various parallel regions to determine if the proposed methodology is suitable to guide porting efforts.

\begin{figure*}[htb]
\usepackage[latin1]{inputenc}
\usepackage{amsmath}
\DeclareMathSizes{12}{11}{9}{6}

\begin{minipage}[t]{.45\linewidth}\centering%
    \begin{tikzpicture}

\begin{axis}[
    enlarge y limits={abs=0.01},
    enlarge x limits={abs=0.5},
    xticklabels={},
    yticklabels={},
    legend columns=3,
    legend style={
      at={(1.33,1.13)},
      anchor=center,
      draw=none,
      /tikz/every even column/.append style   = {column sep= 0mm, text width=15em},
    },
    legend image code/.code={
      \draw [fill=#1] (0cm,-1.5mm) rectangle (4mm,1.5mm);
    },
    xmin=0, xmax=1,
    ymin=0.0, ymax=1.1,
    width=\linewidth,
    height=64mm,
    xmajorticks=false,
    axis line style={draw=none},
    tick style={draw=none},
]
\addlegendimage{col3, postaction={pattern color=gray, pattern=crosshatch}}
\addlegendentry{manual offload (event, small)}
\addlegendimage{col4, postaction={pattern color=gray, pattern=north east lines}}
\addlegendentry{GPU First (event, small)}
\addlegendimage{col8, postaction={pattern color=gray, pattern=vertical lines}}
\addlegendentry{GPU First (history, small)}
\addlegendimage{col10!40, postaction={pattern color=gray, pattern=crosshatch}}
\addlegendentry{manual offload (event, large)}
\addlegendimage{col11!40, postaction={pattern color=gray, pattern=north east lines}}
\addlegendentry{GPU First (event, large)}
\addlegendimage{col13!40, postaction={pattern color=gray, pattern=vertical lines}}
\addlegendentry{GPU First (history, large)}
\end{axis}

\begin{axis}[
    extra y ticks={1},
    extra y tick labels={CPU},
    ymin=0.0, ymax=16.5,
    name nodes near coords/.default=coordnode,
    bar width=0.10,
    name nodes near coords/.style={
      every node near coord/.append style={
        shift={(axis direction cs:0, -\ycoord)},
        name=#1-\coordindex,
        alias=#1-last,
        %rotate=90,
        %anchor=west,
        rotate=0,
        anchor=south,
        align=center,
        /pgf/number format/fixed,
      },
    },
]

\addplot+[
  name nodes near coords=offloadEventSmall,
  col3, postaction={pattern color=gray, pattern=crosshatch},
%  error bars/.cd,y dir=both,y explicit, error bar style={color=black,ultra thick},
 ] 
coordinates {
  (0, 9.757)
  };

\addplot+[
  name nodes near coords=gpuFirstEventSmall,
  col4, postaction={pattern color=gray, pattern=north east lines},
]
coordinates {
  (0, 11.772)
  };

\addplot+[
  name nodes near coords=missing1,
  white,
]
coordinates {
(0,0)
  };


\addplot+[
%  xshift=2mm,
  name nodes near coords=gpuFirstHistorySmall,
  col8, postaction={pattern color=gray, pattern=vertical lines},
]
coordinates {
  (0, 14.360)
  };

\addplot+[
]
coordinates {
  };

\addplot+[
  name nodes near coords=offloadEventLarge,
  col10!40, postaction={pattern color=gray, pattern=crosshatch},
]
coordinates {
  (0, 11.503)
  };

\addplot+[
  name nodes near coords=gpuFirstEventLarge,
  col11!40, postaction={pattern color=gray, pattern=north east lines},
]
coordinates {
  (0, 11.434)
  };

\addplot+[
  name nodes near coords=missing2,
  white,
]
coordinates {
(0,0)
  };

\addplot+[
%  xshift=2mm,
  name nodes near coords=gpuFirstHistory,
  col13!40, postaction={pattern color=gray, pattern=vertical lines},
]
coordinates {
  (0, 10.64)
  };



\addplot[black!70,dotted,thick,sharp plot,nodes near coords={},
update limits=false,shorten >=-3mm,shorten <=-3mm]
coordinates {(-1,1) (1,1)};
\addplot[black!70,dashed,thick,sharp plot,nodes near coords={},
update limits=false,shorten >=-3mm,shorten <=-3mm]
coordinates {(0,0) (0,18)};
\node[] (top) at (axis description cs:1,1) {};
\node[anchor=north] (small) at (axis description cs:0.24,1) {small};
\node[anchor=north] (large) at (axis description cs:0.76,1) {large};

\end{axis}

\node[rotate=90,anchor=west,fill=white] (test) at ($(missing1-0) + (0,0.2)$) {no offload history};
\node[rotate=90,anchor=west,fill=white] (test) at ($(missing2-0) + (0,0.2)$) {no offload history};

\end{tikzpicture}


    \subcaption{
      Performance of the compute kernel of XSBench relative to the CPU version.
    }
    \label{fig:xbench}%
\end{minipage}%
\hfill%
\begin{minipage}[t]{.45\linewidth}\centering%
    \begin{tikzpicture}
\begin{axis}[
    extra y ticks={1},
    extra y tick labels={CPU},
%    ylabel style = {align=center,at={(1.15,0.5)},text width=4cm},
    ymin=0.0, ymax=11,
    name nodes near coords/.default=coordnode,
    bar width=0.10,
    name nodes near coords/.style={
      every node near coord/.append style={
        shift={(axis direction cs:0, -\ycoord)},
        name=#1-\coordindex,
        alias=#1-last,
        %rotate=90,
        %anchor=west,
        rotate=0,
        anchor=south,
        align=center,
        /pgf/number format/fixed,
      },
    },
]

\addplot+[
  name nodes near coords=offloadEventSmall,
  col3, postaction={pattern color=gray, pattern=crosshatch},
%  error bars/.cd,y dir=both,y explicit, error bar style={color=black,ultra thick},
 ] 
coordinates {
  (0, 6.090)
  };
%\addlegendentry{Offload (event)}

\addplot+[
  name nodes near coords=gpuFirstEventSmall,
  col4, postaction={pattern color=gray, pattern=north east lines},
]
coordinates {
  (0, 8.976)
  };
%\addlegendentry{GPU First (event)}

\addplot+[
  name nodes near coords=missing1,
  white,
]
coordinates {
(0,0)
  };


\addplot+[
%  xshift=2mm,
  name nodes near coords=gpuFirstHistorySmall,
  col8, postaction={pattern color=gray, pattern=vertical lines},
]
coordinates {
  (0, 9.0669)
  };
%\addlegendentry{GPU First (history)}

\addplot+[
]
coordinates {
  };

\addplot+[
  name nodes near coords=offloadEventLarge,
  col10!40, postaction={pattern color=gray, pattern=crosshatch},
]
coordinates {
  (0, 4.5159)
  };
%\addlegendentry{Offload (event)}

\addplot+[
  name nodes near coords=gpuFirstEventLarge,
  col11!40, postaction={pattern color=gray, pattern=north east lines},
]
coordinates {
  (0, 4.6102)
  };
%\addlegendentry{GPU First (event)}

\addplot+[
  name nodes near coords=missing2,
  white,
]
coordinates {
(0,0)
  };

\addplot+[
%  xshift=2mm,
  name nodes near coords=gpuFirstHistory,
  col13!40, postaction={pattern color=gray, pattern=vertical lines},
]
coordinates {
  (0, 4.6277)
  };
%\addlegendentry{GPU First (history)}



\addplot[black!70,dotted,thick,sharp plot,nodes near coords={},
update limits=false,shorten >=-3mm,shorten <=-3mm]
coordinates {(-1,1) (1,1)};
\addplot[black!70,dashed,thick,sharp plot,nodes near coords={},
update limits=false,shorten >=-3mm,shorten <=-3mm]
coordinates {(0,0) (0,18)};
\node[] (top) at (axis description cs:1,1) {};
\node[anchor=north] (small) at (axis description cs:0.24,1) {small};
\node[anchor=north] (large) at (axis description cs:0.76,1) {large};


\end{axis}

\node[rotate=90,anchor=west,fill=white] (test) at ($(missing1-0) + (0,0.32)$) {no offload history};
\node[rotate=90,anchor=west,fill=white] (test) at ($(missing2-0) + (0,0.32)$) {no offload history};

\end{tikzpicture}


    \subcaption{
      Performance of the compute kernel of RSBench relative to the CPU version.
    }
    \label{fig:rsbench}%
\end{minipage}
    \vspace*{-3mm}

\caption{Performance of different GPU versions of the OpenMC proxy applications XSBench and RSBench compared to their respective CPU counterpart.}
\Description{Performance of different GPU versions of the OpenMC proxy applications XSBench and RSBench compared to their respective CPU counterpart.}
\vspace*{-3mm}
\label{fig:xsrsbench}
    
\end{figure*}

% \begin{figure*}[hbt]
% \begin{tabular}{c|ccc|ccc}
% \hline
% \multirow{2}{*}{Benchmark} & \multicolumn{3}{c|}{Offloading} & \multicolumn{3}{c}{\emph{GPU First}} \\ \cline{2-7} 
%                            & Registers  & Teams   & Threads  & Registers  & Teams  & Threads \\ \hline
% XSBench (event/history)    & 91/-       & 3200/-  & 128/-    & 83/93      & 1024   & 640     \\
% RSBench (event/history)    & 160/-      & 3200/-  & 128/-    & 161/167    & 1024   & 384     \\
% Interleaved                & 32         & 16      & 256      & 78         & 16     & 256     \\
% Hypterm                    & 63/72      & 3200    & 256      & 128        & 3200   & 256     \\
% AMGmk                      & 32         & 489     & 256      & 52         & 489    & 256     \\
% Page-Rank                  & 29/32      & 79      & 256      & 58/48      & 79     & 256     \\ \hline
% \end{tabular}
% \label{fig:execution-statistics}
% \caption{Comparison of kernel information for the offloading and \emph{GPU first} versions of each benchmark.
% The table includes register usage, number of teams, and number of threads per team for each benchmark.
% Note that if a benchmark has multiple parallel regions, and only one set of data is presented, the kernel information is the same for all regions.}
% \vspace*{-3mm}
% \end{figure*}

\subsubsection{XSBench and RSBench}

The two OpenMC proxy applications XSBench~\cite{XS_Tramm_2014} (v20) and RSBench~\cite{DBLP:conf/easc/TrammSFJ14} (v13) are implemented in different parallel programming models, including OpenMP threading for the CPU, and OpenMP offload for the GPU.
In the former, two alternative methods are available to perform the cross-section lookup as part of the neutron transport simulation: event-based lookup and history-based lookup.
In the offloading version history-based mode was not implemented but we can test it out with the \emph{GPU First} methodology using the CPU implementation.
The results for both benchmarks and two different input sizes are shown in \Cref{fig:xbench} and \Cref{fig:rsbench}, respectively.
For the small input size, history mode is actually outperforming the event mode on the GPU.
However, with the large input size event mode has caught up (RSBench), or even surpassed (XSBench), history mode.
These results validate the choice of event-based mode for the offloading implementation.
The second insight from the evaluation can be derived by comparing the event-based results obtained via the manually offloaded version and the \emph{GPU First} version.
For the small input the \emph{GPU First} versions are likely to benefit from cache re-use as the data initialization is also performed on the GPU.
However, with the large input the two results are a close match.
Thus, performance predictions obtained via \emph{GPU First} and the original CPU-only version would have provided accurate guidance for a potential manual port to the GPU.

\subsubsection{HeCBench: Interleaved}

The HeCBench~\cite{zjin-lcf/HeCBench} ``interleaved'' micro benchmark originated from \citet{cook2012cuda} and shows how different memory access patterns behave on the CPU and GPU.
We timed the parallel region with interleaved memory accesses (array-of-struct inputs) as well as the one with non-interleaved accesses (struct-of-array inputs) on the CPU and GPU.
The results, expressed as speedups and slowdowns of the GPU version, are shown in \Cref{fig:interleaved}.
While the \emph{GPU First} version shows the same tendency as the manually offloading version, we needed to explicitly match the number of teams to perfectly match the result with our automatically offloaded parallel regions.
%For the remaining micro benchmarks we will only show the results with matching team counts.


\begin{figure}[htbp]
\usepackage[latin1]{inputenc}
\usepackage{amsmath}
\DeclareMathSizes{12}{11}{9}{6}


\begin{minipage}[t]{\linewidth}\centering%
    \resizebox{.9\linewidth}{!}{
    \begin{tikzpicture}
\begin{axis}[
    legend columns=3,
    legend style={
      at={(0.5,1.15)},
      anchor=center,
      draw=none,
      /tikz/every even column/.append style   = {column sep= 5mm},
      /tikz/every odd column/.append style   = {column sep= 1mm},
    },
    legend image code/.code={
      \draw [fill=#1] (0cm,-1.5mm) rectangle (4mm,1.5mm);
    }, 
    legend style={cells={align=center}},
    xmin=0, xmax=1,
    ymin=0.0, ymax=1.1,
    width=\linewidth,
    height=64mm,
    hide axis,    
]
\addlegendimage{
  col12!30, postaction={pattern color=gray, pattern=crosshatch}}
\addlegendentry{manual\\ offload}
\addlegendimage{
  col13!30, postaction={pattern color=gray, pattern=north east lines}}
\addlegendentry{\emph{GPU First}\\ (1024 teams)}
\addlegendimage{
  col14!30, postaction={pattern color=gray, pattern=north east lines}}
\addlegendentry{\emph{GPU First}\\ (matching teams)}
\end{axis}
\end{tikzpicture}


}
\end{minipage}
\vspace{-2mm}

\begin{minipage}[t]{\linewidth}\centering%
    \resizebox{.9\linewidth}{!}{
      
\begin{tikzpicture}
\begin{axis}[
    extra y ticks={1},
    extra y tick labels={CPU},
%    ylabel style = {align=center,at={(1.15,0.5)},text width=4cm},
    ymin=-58, ymax=60,
    name nodes near coords/.default=coordnode,
    bar width=0.10,
    name nodes near coords/.style={
      every node near coord/.append style={
        shift={(axis direction cs:0, -\ycoord)},
        name=#1-\coordindex,
        alias=#1-last,
        %rotate=90,
        %anchor=west,
        rotate=0,
        anchor=south,
        align=center,
        /pgf/number format/fixed,
      },
    },
    ybar=8pt,
  %  ymode=log,
  %  log origin=infty,
]

\addplot+[
  name nodes near coords=niOffload,
  col12!30, postaction={pattern color=gray, pattern=crosshatch},
 ] 
coordinates {
  (0, 52.12090036)
  };

\addplot+[
  name nodes near coords=niMT1024,
  col13!30, postaction={pattern color=gray, pattern=north east lines},
 ] 
coordinates {
  (0, 35.67445016)
  };

\addplot+[
  name nodes near coords=niMT16,
  col14!30, postaction={pattern color=gray, pattern=north east lines},
 ] 
coordinates {
  (0, 52.08492271)
  };

\addplot+[
 ] 
coordinates {
  };

\addplot+[
  name nodes near coords=iOffload,
  col12!30, postaction={pattern color=gray, pattern=crosshatch},
]
coordinates {
  (0, -24.45384987)
  };
  
\addplot+[
  name nodes near coords=iMT1024,
  col13!30, postaction={pattern color=gray, pattern=north east lines},
 ] 
coordinates {
  (0, -49.40655542)
  };

\addplot+[
  name nodes near coords=iMT16,
  col14!30, postaction={pattern color=gray, pattern=north east lines},
 ] 
coordinates {
  (0, -24.74615046)
  };
  



\addplot[black!70,dotted,thick,sharp plot,nodes near coords={},
update limits=false,shorten >=-3mm,shorten <=-3mm]
coordinates {(-1,1) (1,1)};
\addplot[black!70,dashed,thick,sharp plot,nodes near coords={},
update limits=false,shorten >=-3mm,shorten <=-3mm]
coordinates {(0,-200) (0, 200)};
\node[] (top) at (axis description cs:1,1) {};
\node[] (middle) at (0,0) {};

\end{axis}

\ExtractCoordinateA{middle}
\ExtractCoordinateB{niMT1024-0}
\node[anchor=north, text width=25mm,align=center] (noninterleaved) at (\XCoordB, \YCoordA) {non-interleaved (AoS layout)};
\ExtractCoordinateB{iMT1024-0}
\node[anchor=south, text width=25mm,align=center] (interleaved) at (\XCoordB, \YCoordA) {interleaved (SoA layout)};

\end{tikzpicture}

 
    }
    \subcaption{Relative performance of the two parallel regions in the interleaved benchmark (from HeCBench) when executed on the GPU instead of the CPU. The first (non-interleaved) uses a struct-of-array layout while the second (interleaved) uses an array-of-struct layout.}
    \label{fig:interleaved}
\end{minipage}
\vspace{3mm}

\begin{minipage}[t]{\linewidth}\centering%
    \resizebox{.9\linewidth}{!}{
      \begin{tikzpicture}
\begin{axis}[
    extra y ticks={1},
    extra y tick labels={CPU},
%    ylabel style = {align=center,at={(1.15,0.5)},text width=4cm},
    ymin=0, ymax=22,
    name nodes near coords/.default=coordnode,
    bar width=0.07,
    name nodes near coords/.style={
      every node near coord/.append style={
        shift={(axis direction cs:0, -\ycoord)},
        name=#1-\coordindex,
        alias=#1-last,
        %rotate=90,
        %anchor=west,
        rotate=0,
        anchor=south,
        align=center,
        /pgf/number format/fixed,
      },
    },
    ybar=5pt,
  %  ymode=log,
  %  log origin=infty,
      nodes near coords = {%
      \begingroup%
        % this group is merely to switch to FPU locally.
        % Might be unnecessary, but who knows.
        \pgfkeys{/pgf/fpu}%
        \pgfmathparse{\pgfplotspointmeta<0.0}%
        \global\let\isneg=\pgfmathresult%
        \pgfmathparse{\pgfplotspointmeta==0}%
        \global\let\iszero=\pgfmathresult%
        \pgfmathparse{\pgfplotspointmeta==1}%
        \global\let\isone=\pgfmathresult%
      \endgroup%
      \pgfmathfloatcreate{0}{0.0}{0}%
      \let\ZERO=\pgfmathresult%
      \pgfmathfloatcreate{1}{1.0}{0}%
      \let\ONE=\pgfmathresult%
      \pgfmathfloatcreate{1}{-1.0}{0}%
      \let\MONE=\pgfmathresult%
      \color{black}%
      \global\let\ycoord=\ZERO
%      \global\let\ycoord=\rawy
    \ifx\iszero\ONE%
    \else%
        \ifx\isneg\ONE%
        {%
          \node[rotate=90,fill=white,yshift=-4pt,xshift=9pt,inner sep=1pt]{\color{black}\textbf{\pgfmathprintnumber[precision=2]{\rawy}}};%
        }%
        \else%
          \node[rotate=90,fill=white,yshift=-4pt,xshift=9pt,inner sep=1pt]{\color{black}\textbf{\pgfmathprintnumber[precision=2]{\rawy}}};%
        \fi%
    \fi%
    },
]

\addplot+[
  name nodes near coords=off1,
  col12!30, postaction={pattern color=gray, pattern=crosshatch},
 ] 
coordinates {
  (0, 14.67804854)
  };

\addplot+[
  name nodes near coords=mt1,
  col13!30, postaction={pattern color=gray, pattern=north east lines},
 ] 
coordinates {
  (0, 13.0136894)
  };
\addplot+[
  name nodes near coords=mt1match,
  col14!30, postaction={pattern color=gray, pattern=north east lines},
 ] 
coordinates {
  (0, 15.32424341)
  };
  


\addplot+[
 ] 
coordinates {
  };

\addplot+[
  name nodes near coords=off2,
  col12!30, postaction={pattern color=gray, pattern=crosshatch},
 ] 
coordinates {
  (0, 8.522244067)
  };
\addplot+[
  name nodes near coords=mt2,
  col13!30, postaction={pattern color=gray, pattern=north east lines},
]
coordinates {
  (0, 9.723230855)
  };
\addplot+[
  name nodes near coords=mt2match,
  col14!30, postaction={pattern color=gray, pattern=north east lines},
 ] 
coordinates {
  (0, 9.65147391)
  };
  

\addplot+[
 ] 
coordinates {
  };

  
\addplot+[
  name nodes near coords=off3,
  col12!30, postaction={pattern color=gray, pattern=crosshatch},
 ] 
coordinates {
  (0, 7.895095135)
  };

\addplot+[
  name nodes near coords=mt3,
  col13!30, postaction={pattern color=gray, pattern=north east lines},
 ] 
coordinates {
(0, 9.015701821)
  };
\addplot+[
  name nodes near coords=mt3match,
  col14!30, postaction={pattern color=gray, pattern=north east lines},
 ] 
coordinates {
  (0, 9.034509518)
  };
						
  



\addplot[black!70,dotted,thick,sharp plot,nodes near coords={},
update limits=false,shorten >=-3mm,shorten <=-3mm]
coordinates {(-1,1) (1,1)};

\draw[black!70,dashed,thick,] (axis description cs:0.3333,0) -- (axis description cs:0.3333, 1);
\draw[black!70,dashed,thick,] (axis description cs:0.6666,0) -- (axis description cs:0.6666, 1);
\node[] (top) at (axis description cs:1,1) {};
\node[] (middle) at (0,0) {};

\node[anchor=north] (pr1) at (axis description cs:0.15, 1) {PR1};
\node[anchor=north] (pr2) at (axis description cs:0.5, 1) {PR2};
\node[anchor=north] (pr3) at (axis description cs:0.85, 1) {PR3};

\end{axis}


\end{tikzpicture}

 
    }
    \subcaption{Relative performance of the three parallel regions (PR1, PR2, PR3) in the hypterm micro benchmark (from HeCBench) when executed on the GPU instead of the CPU.}
    \label{fig:hypeterm}
\end{minipage}
\vspace{3mm}

\begin{minipage}[t]{\linewidth}\centering%
    \resizebox{.9\linewidth}{!}{
      \begin{tikzpicture}
\begin{axis}[
    extra y ticks={1},
    extra y tick labels={CPU},
%    ylabel style = {align=center,at={(1.15,0.5)},text width=4cm},
    ymin=0.0, ymax=9,
    name nodes near coords/.default=coordnode,
    bar width=0.10,
    name nodes near coords/.style={
      every node near coord/.append style={
        shift={(axis direction cs:0, -\ycoord)},
        name=#1-\coordindex,
        alias=#1-last,
        %rotate=90,
        %anchor=west,
        rotate=0,
        anchor=south,
        align=center,
        /pgf/number format/fixed,
      },
    },
    ybar=8pt,
]

\addplot+[
  name nodes near coords=amgmkOffload,
  col12!30, postaction={pattern color=gray, pattern=crosshatch},
 ] 
coordinates {
  (0, 7.670604372)
  };

\addplot+[
  name nodes near coords=amgmkMT,
  col13!30, postaction={pattern color=gray, pattern=north east lines},
]
coordinates {
  (0, 6.410588883)
  };
  
\addplot+[
  name nodes near coords=amgmkMTmatch,
  col14!30, postaction={pattern color=gray, pattern=north east lines},
]
coordinates {
  (0, 6.601275915)
  };
  
\addplot+[
 ] 
coordinates {
  };

\addplot+[
  name nodes near coords=pageRankOffload,
  col12!30, postaction={pattern color=gray, pattern=crosshatch},
 ] 
coordinates {
  (0, 3.102152887)
  };

\addplot+[
  name nodes near coords=pageRankMT,
  col13!30, postaction={pattern color=gray, pattern=north east lines},
]
coordinates {
  (0, 2.698591605)
  };

\addplot+[
  name nodes near coords=pageRankMTMatch,
  col14!30, postaction={pattern color=gray, pattern=north east lines},
 ] 
coordinates {
  (0, 3.900568937)
  };
  



\addplot[black!70,dotted,thick,sharp plot,nodes near coords={},
update limits=false,shorten >=-3mm,shorten <=-3mm]
coordinates {(-1,1) (1,1)};
\addplot[black!70,dashed,thick,sharp plot,nodes near coords={},
update limits=false,shorten >=-3mm,shorten <=-3mm]
coordinates {(0,0) (0,18)};
\node[] (top) at (axis description cs:1,1) {};


\end{axis}

\ExtractCoordinateA{top}
\ExtractCoordinateB{amgmkMT-0}
\node[anchor=north, text width=25mm,align=center] (amg) at (\XCoordB, \YCoordA) {AMGmk};
\ExtractCoordinateB{pageRankMT-0}
\node[anchor=north, text width=25mm,align=center] (pr) at (\XCoordB, \YCoordA) {Page Rank};

\end{tikzpicture}

 
    }
    \subcaption{Relative performance results for the timed parallel regions in the AMGmk and page-rank micro benchmark (from HeCBench) when executed on the GPU instead of the CPU.}
    \label{fig:mixed}
\end{minipage}

\vspace{-2mm}
\caption{
Comparison of micro benchmarks performance results for a parallel region compiled with \emph{GPU First} to the GPU, and the manually offloaded counterpart, relative to the corresponding CPU parallel region.
The matching teams column for \emph{GPU First} uses the same number of teams as the manually offloaded version.
The legend at the top of the figure is shared among all plots.
% Micro benchmark results that compare the performance of a parallel region compiled with \emph{GPU First} to the GPU, and the manually offloaded counterpart, relative to the respective CPU parallel region.
% The \emph{GPU First} matching teams column uses the same number of teams as the corresponding manually offloaded version.
% The legend at the top is shared among all plots.
}
\Description{
Comparison of micro benchmarks performance results for a parallel region compiled with \emph{GPU First} to the GPU, and the manually offloaded counterpart, relative to the corresponding CPU parallel region.
The matching teams column for \emph{GPU First} uses the same number of teams as the manually offloaded version.
The legend at the top of the figure is shared among all plots.
% Micro benchmark results that compare the performance of a parallel region compiled with \emph{GPU First} to the GPU, and the manually offloaded counterpart, relative to the respective CPU parallel region.
% The \emph{GPU First} matching teams column uses the same number of teams as the corresponding manually offloaded version.
% The legend at the top is shared among all plots.
}
\label{fig:AAA}
    
\end{figure}

\subsubsection{HeCBench: Hypterm}

The HeCBench ``hypterm'' micro benchmark is a complex stencil operation that originated from the ExpCNS Compressible Navier-Stokes mini-application~\cite{ExaCT} and was extracted by \citet{DBLP:conf/ppopp/RawatRSPRS18}.
The GPU version in in HeCBench contains three kernels which we transformed into three parallel regions for the CPU.
The results of the GPU version and the \emph{GPU First} version relative to the CPU are shown in \Cref{fig:hypeterm}.
While the original GPU version is slightly slower, the overall performance behavior matches the \emph{GPU First} prediction.


\subsubsection{HeCBench: AMGmk, Page-Rank, }

\Cref{fig:mixed} shows the results obtained for the AMGmk, page-rank benchmarks.
The first measures only the relax kernel of the original AMGmk proxy application~\cite{CORAL}.
The second is an implementation of the page-rank algorithm for graphs in which the propagation step is measured.

\begin{figure}

\begin{minipage}[t]{\linewidth}\centering%
    \resizebox{.7\linewidth}{!}{
    \begin{tikzpicture}
\begin{axis}[
    legend columns=2,
    legend style={
      at={(0.5,1.15)},
      anchor=center,
      draw=none,
      /tikz/every even column/.append style   = {column sep= 5mm},
      /tikz/every odd column/.append style   = {column sep= 1mm},
    },
  %  legend image code/.code={
  %    \draw [fill=#1] (0cm,-1.5mm) rectangle (4mm,1.5mm);
  %  }, 
    legend style={cells={align=center}},
    xmin=0, xmax=1,
    ymin=0.0, ymax=1.1,
    width=\linewidth,
    height=64mm,
    hide axis,    
]
\addlegendimage{  only marks,
  every mark/.append style={solid, draw=black, fill=col15!30,scale=1.3}, 
  mark=square*}
\addlegendentry{end-2-end}
\addlegendimage{  only marks,
  every mark/.append style={solid, draw=black, fill=col16!30,scale=1.3}, 
  mark=triangle*}
\addlegendentry{parallel region}
\end{axis}
\end{tikzpicture}


}
\end{minipage}
\vspace{-4mm}

\begin{minipage}[t]{\linewidth}\centering%
    \resizebox{.9\linewidth}{!}{
      \begin{tikzpicture}

\begin{axis}[
    extra y ticks={1},
    extra y tick labels={CPU},
    ymin=0.0, ymax=160,
    name nodes near coords/.default=coordnode,
    bar width=0.10,
    name nodes near coords/.style={
      every node near coord/.append style={
        shift={(axis direction cs:0, -\ycoord)},
        name=#1-\coordindex,
        alias=#1-last,
        %rotate=90,
        %anchor=west,
        rotate=0,
        anchor=south,
        align=center,
        /pgf/number format/fixed,
      },
    },
%    ymode=log,
    xmin=4,  xmax=20,
    xtick distance={2},
    enlarge x limits={abs=1},
    width=\linewidth,
    height=65mm,
     extra y tick style={
           grid=major,
           tick label style={rotate=90},
           ticklabel pos=right,
     },
    ylabel={app. slowdown of the GPU version relative to the CPU (optimized) one},
    ylabel style = {align=center,at={(-0.09,0.55)},text width=5cm},
]

\addplot+[
  name nodes near coords=end2end,
  only marks,
  every mark/.append style={solid, draw=black, fill=col15!30,scale=1.3}, 
  mark=square*,
 ] 
coordinates {
(4, 23.71291305)
(6, 28.08102772)
(8, 33.31944034)
(10, 47.00821984)
(12, 62.60580386)
(14, 78.4042739)
(16, 119.6299607)
(18, 112.1189084)
(20, 154.3024754)
  };

\addplot+[
  name nodes near coords=parRegion,
  only marks,
  every mark/.append style={solid, draw=black, fill=col16!30,scale=1.3}, 
  mark=triangle*,
 ] 
coordinates {
(4,2.853754473) 
(6, 13.90612513)
(8,21.02651895)
(10, 33.02571654)
(12, 49.47597231) 
(14, 66.95620005)
(16, 109.5703832)
(18, 105.5760577)
(20, 148.6172162)
  };

\end{axis}
\end{tikzpicture}

 
    }
\vspace{-2mm}
    \subcaption{358.botsalgn. The x axis is the number of input sequences.
    This benchmark distributes sequences across multiple threads through an outer \lstinline{parallel} region, where each thread spawns several OpenMP tasks to execute the pair alignment algorithm.}
    \label{fig:spec-358}
\end{minipage}
\begin{minipage}[t]{\linewidth}\centering%
    \resizebox{.9\linewidth}{!}{
      \begin{tikzpicture}

\begin{axis}[
    extra y ticks={1},
    extra y tick labels={CPU},
    ymin=0.0, ymax=80,
    name nodes near coords/.default=coordnode,
    bar width=0.10,
    name nodes near coords/.style={
      every node near coord/.append style={
        shift={(axis direction cs:0, -\ycoord)},
        name=#1-\coordindex,
        alias=#1-last,
        %rotate=90,
        %anchor=west,
        rotate=0,
        anchor=south,
        align=center,
        /pgf/number format/fixed,
      },
    },
%    ymode=log,
    xmin=1,  xmax=7,
    xtick distance={1},
    xtick={1,2,3,4,5,6,7},
    xticklabels={
(50,25), 
(51,31),
(52,37),
(53,44),
(54,50),
(55,57),
(56,64)
},
    enlarge x limits={abs=1},
    width=\linewidth,
    height=65mm,
     extra y tick style={
           grid=major,
           tick label style={rotate=90},
           ticklabel pos=right,
     },
    ylabel={app. slowdown of the GPU version relative to the CPU (optimized) one},
    ylabel style = {align=center,at={(-0.09,0.55)},text width=5cm},
    x tick label style={rotate=45,anchor=east},
]

\addplot+[
  name nodes near coords=end2end,
  only marks,
  every mark/.append style={solid, draw=black, fill=col15!30,scale=1.3}, 
  mark=square*,
 ] 
coordinates {
(1, 21.88757774) 
(2, 30.72475864)
(3, 35.5013504)
(4, 38.75906206)
(5, 50.10372313)
(6, 58.30036732)
(7, 70.4378462)
  };

\addplot+[
  name nodes near coords=parRegion,
  only marks,
  every mark/.append style={solid, draw=black, fill=col16!30,scale=1.3}, 
  mark=triangle*,
 ] 
coordinates {
(1, 10.79370488)
(2, 20.3155239)
(3, 28.8395581)
(4, 32.71760054)
(5, 45.56800443)
(6, 54.54240534)
(7, 69.26114314)
  };

\end{axis}
\end{tikzpicture}

 
    }
\vspace{-3mm}
\subcaption{359.botsspar uses one thread creates tasks while the other threads in the parallel region execute them.
The x axis is the size of matrix and submatrix used in the benchmark.}
\label{fig:spec-359}
\end{minipage}
\begin{minipage}[t]{.9\linewidth}\centering%
    \resizebox{\linewidth}{!}{
      \begin{tikzpicture}

\begin{axis}[
    extra y ticks={1},
    extra y tick labels={CPU},
    ymin=0.0, ymax=22,
    name nodes near coords/.default=coordnode,
    bar width=0.10,
    name nodes near coords/.style={
      every node near coord/.append style={
        shift={(axis direction cs:0, -\ycoord)},
        name=#1-\coordindex,
        alias=#1-last,
        %rotate=90,
        %anchor=west,
        rotate=0,
        anchor=south,
        align=center,
        /pgf/number format/fixed,
      },
    },
%    ymode=log,
    xmin=6,  xmax=30,
    xtick distance={2},
    enlarge x limits={abs=1},
    width=\linewidth,
    height=65mm,
     extra y tick style={
           grid=major,
           tick label style={rotate=90},
           ticklabel pos=right,
     },
    ylabel={app. slowdown of the GPU version relative to the CPU (optimized) one},
    ylabel style = {align=center,at={(-0.09,0.55)},text width=5cm},
]

\addplot+[
  name nodes near coords=offloadEventSmall,
  only marks,
  every mark/.append style={solid, draw=black, fill=col15!30,scale=1.3}, 
  mark=square*,
 ] 
coordinates {
(2,10.83923661) 
(4,6.43278232)
(6,10.54924858)
(8,10.29753388)
(10,10.39138599)
(12,9.626968046)
(14,10.05446202)
(16,11.57487804)
(18,9.701438092)
(20,8.800448367)
(22,8.890264957)
(24,9.428948964)
(26,9.147011498)
(28,12.96558124)
(30,20.01402426)
  };

\addplot+[
  name nodes near coords=parRegion,
  only marks,
  every mark/.append style={solid, draw=black, fill=col16!30,scale=1.3}, 
  mark=triangle*,
 ] 
coordinates {
(2 ,1.98299648 ) 
(4 ,0.037775978)
(6 ,2.076678019)
(8 ,1.934647426)
(10,2.212845429)
(12,1.606530646)
(14,1.714703073)
(16,1.633372769)
(18,1.198302583)
(20,0.935884944)
(22,0.891361062)
(24,1.308900685)
(26,2.346619088)
(28,6.567314424)
(30,15.79918583)
  };

\end{axis}
\end{tikzpicture}

 
    }
\vspace{-5mm}
\subcaption{372.smithwa. The x axis is the sequence length.
The workload is manually distributed among multiple threads and threads communicate with each other using a producer-consumer model via shared variables followed by barriers.}
\label{fig:spec-372}
\end{minipage}

\vspace{-2mm}
\caption{
Relative performance results for the end-to-end execution and timed parallel regions in the three SPEC OMP 2012 benchmarks when executed on the GPU instead of the CPU.
The legend at the top is shared among all plots.}
\Description{
Relative performance results for the end-to-end execution and timed parallel regions in the three SPEC OMP 2012 benchmarks when executed on the GPU instead of the CPU.
The legend at the top is shared among all plots.}
\label{fig:spec}
\end{figure}

\subsubsection{SPEC OMP: 358.botsalgn and 359.botsspar}
These are two task-based benchmarks~\cite{DBLP:conf/icpp/DuranTFMA09} from the SPEC OMP 2012 suite~\cite{DBLP:conf/iwomp/MullerBBFHHJMPRSWWK12}.
The former performs sequence alignment while the latter is a sparse LU decomposition.
They are parallelized with different OpenMP tasking strategies.
\Cref{fig:spec-358,fig:spec-359} shows the performance of \emph{GPU First} relative to the CPU.
Since LLVM/OpenMP does not support tasking on GPUs, tasks are executed immediately by the encountering thread.
This limitation severely affects the GPU performance of these benchmarks.

In the case of \texttt{358.botsalgn}, sequences are distributed across multiple threads through an outer \lstinline{parallel} region.
Each thread spawns several tasks to that perform the alignment.
Since the number of sequences is smaller than the number of CPU cores, threads not involved in the work sharing can execute the spawned tasks concurrently.
However, on the GPU only a small number of threads (equal to the number of sequences) are executing concurrently.
%Moreover, GPU threads are slower than CPU threads, which results in the observed slowdown.

Similarly, in \lstinline{359.botsspar}, one thread creates tasks while the other threads in the parallel region execute them.
This pattern of execution is equivalent to serial execution in our approach.
To enable parallelism for this benchmark, we rewrote the task regions by removing the \lstinline{task} construct and adding a \lstinline{parallel for} construct on the outer parallel region.
The results shown in \Cref{fig:spec-359} represent the threaded parallelism version of the benchmark.
The observed slowdown can be attributed to the lack of sufficient sequences to fully exploit the massive parallelism that GPUs offer, similar to the issue observed in benchmark \texttt{358.botsalgn}.
Nevertheless, our \emph{GPU First} scheme allows application developers to explore different parallelism on GPUs without much burden.

It is important to note that the lack of tasking support is not a limitation of our proposed scheme, but rather a limitation of the current LLVM/OpenMP implementation for GPUs.
If tasking is properly supported on the GPU, and there are a sufficient number of sequences, the massive parallelism of a GPU has the potential to make up for the performance difference between a CPU and a GPU thread.
While this means advancements in GPU tasking support could in the future improve performance of these codes on the GPU, the current results clearly indicate that a GPU port would require a different parallelization strategy.

\subsubsection{SPEC OMP: 372.smithwa}
\texttt{372.smithwa} implements the Smith-Waterman algorithm for sequence alignment and is characterized by a large number of nested loops with indirect memory accesses.
In this benchmark, the workload is first distributed to multiple threads, which maps well to the GPU.
However, the threads communicate with each other using a producer-consumer model via shared variables followed by barriers.
This form of communication is conceptually inefficient on GPUs.
\Cref{fig:spec-372} shows the performance of the \emph{GPU First} approach relative to the CPU.
As the input size is increased the relative performance is at first stable, indicating good scalability on the GPU.
However, when the sequence length hits 26 we can observe exponentially growing slowdown compared to the CPU execution.
Consequently, this benchmark is another example of an algorithm that is inefficient on the GPU, requiring a rewrite as part of the porting effort.
It is worth to note that without the balanced allocator the performance is dominated by the massively parallel allocations and deallocations at the beginning and end of the parallel region, respectively.
%Additionally, this benchmark allocates many memory regions at the beginning and in the middle of the parallel region, and frees them at the end.
%Although we provide a balanced allocator that is faster than the one provided by NVIDIA, massive dynamic heap allocation in parallel regions is generally inefficient on GPUs.
