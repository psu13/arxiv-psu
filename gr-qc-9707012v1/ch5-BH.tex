
\chapter{Energy and Angular Momentum}

\section{Covariant Formulation of Charge Integral}

In the usual Minkowski space formulation with charge density $\rho(\vec{x},t)$, 
the charge in a volume $V$ is written as
\bea
Q  & = &  \int_V dV\,\rho = \int_V dV\, \vec{\nabla}\cdot \vec{E} \quad \mbox{by Maxwell's eqs.}  \\
Q  & = &  \oint_{\partial V} d\vec{S}\cdot \vec{E} \quad \mbox{by Gauss' law}
\eea
where surface integral is over boundary of $V$.  Note that,
\be
\vec{\nabla}\cdot\vec{E} = \frac{1}{\sqrt{\Dim{3}g}}
\partial_i\sqrt{\Dim{3}g}E^i, \quad dV =\dx{3}{x}\sqrt{\Dim{3}g}
\ee
where $\Dim{3}g$ is the determinant of the 3-metric, so
\be
\int dV\vec{\nabla}\cdot \vec{E}=\int \dx{3}{x}\partial_i
\left(\sqrt{\Dim{3}g}E^i\right) = \int dS_i E^i\, .
\ee
The Lorentz covariant formulation uses the similar result 
\be
\frac{1}{\sqrt{-\Dim{4}g}}\partial_{\mu}\left(\sqrt{-\Dim{4}g}
F^{\mu\nu}\right) = D_{\mu}F^{\mu\nu}\, .
\ee
The volume $V$ is replaced by an arbitrary spacelike hypersurface $\Sigma$
(partial Cauchy surface) with boundary $\partial\Sigma$.  The volume element on
$\Sigma$ is a \emph{non-spacelike} co-vector (1-form) $dS_{\mu}$.  Given the
current density 4-vector $j^{\mu}(x)$ we write
\be
Q=\int_{\Sigma}dS_{\mu}j^{\mu}
\ee
We can choose $\Sigma$ (at least locally) to be $t=$ constant, in which 
case $dS_{\mu}=(dV,\vec{0})$.  Since $j^0=\rho$, we recover the previous
expression for $Q$.  Now use Maxwell's equations.  $D_{\nu}F^{\mu\nu}=j^{\mu}$
to rewrite $Q$ as
\bea
Q & = & \int_{\Sigma} dS_{\mu} D_{\nu}F^{\mu\nu} \\
 & = & \half \oint_{\partial\Sigma} dS_{\mu\nu}F^{\mu\nu} \quad 
\mbox{by Gauss' law}
\eea
where $dS_{\mu\nu}$ is the area element of $\partial\Sigma$.  When $\Sigma$ 
is $t=$ constant the only non-vanishing components of $dS_{\mu\nu}$ are
\be
dS_{0i}=-dS_{i0}\equiv dS_i
\ee
in which case
\be
Q= \oint_{\partial\Sigma}dS_i\, F^{0i}
\ee
But $F^{0i}=-F^{i0}=E^i$, so we recover the previous formula.

\section{ADM energy}\index{ADM energy}

We cannot define energy in the same way because this is associated with a 
conserved \emph{symmetric tensor} $T^{\mu\nu}$, rather than a vector.  This is
not unexpected because a \emph{locally conserved energy can exist only in a
spacetime admitting a timelike Killing vector field}. \\

[Unlike photons, which do \ul{not} carry charge, gravitons\index{graviton} 
\emph{do} carry energy $\Rightarrow$ possibility of energy exchange between
matter and its gravitational field.] \\

We can still define a \emph{total} energy in asymptotically flat spacetimes 
as a surface integral at infinity because $\partial/\partial t$ is
asymptotically Killing in such spacetimes.  In this case
\be
g_{\mu\nu} \to \eta_{\mu\nu} \quad \mbox{as }r\to\infty \quad 
(\mbox{$\eta_{\mu\nu}$ Minkowski metric})
\ee
We shall assume that, \emph{in Cartesian coordinates},
\be
h_{\mu\nu}=g_{\mu\nu}-\eta_{\mu\nu}=\mathcal{O}\left(\frac{1}{r}\right)
\ee
which will justify a linearization of Einstein's equations near $\infty$.

\paragraph{Exercise}  Show that $G_{\mu\nu}=8\pi GT_{\mu\nu}$ becomes the 
Pauli-Fierz equation\index{Pauli-Fierz equation}
\bebox{
\Box h_{\mu\nu}+h_{,\mu\nu}-2h_{(\mu,\nu)}=-16\pi G\left(T_{\mu\nu}-\half 
\eta_{\mu\nu}T\right)
\label{eq:ADMdagger}
}
where
\bea
\Box & = & \eta^{\mu\nu}\partial_{\mu}\partial_{\nu} \\
h & = & \eta^{\mu\nu} h_{\mu\nu} \\
h_{\mu} & = & \eta^{\nu\rho}h_{\rho\mu,\nu} = h^{\nu}_{\I \mu,\nu} \\
T & = &  \eta^{\mu\nu}T_{\mu\nu}
\eea
Take the trace to get
\bebox{
\Box h - h^{\mu}_{\I,\mu} = 8\pi GT
\label{eq:ADMstar}
}

We shall first consider a \emphin{weak static dust}  source
\be
T_{\mu\nu} = \left( \begin{array}{c|ccc}
\rho & & 0 & \\ \hline
 & & & \\
0 & & 0 & \\
 & & & \end{array} \right) \quad \mbox{zero pressure for `dust'}
\ee
\bean
\begin{array}{ccl}
\dot{\rho}  =  0 \quad & \mbox{for} & \mbox{\emph{static}}  \\
\left.\begin{array}{rcl}
4\pi G\rho & \ll & 1 \\
T_{0i} & = & 0 \end{array} \right\} & \mbox{for} & \mbox{\emph{weak}} 
\end{array}
\eean
Since source is static we may assume static $h_{\mu\nu}$, i.e. 
$\dot{h}_{\mu\nu}=0$.  Then $\mu=\nu=0$ component of (\ref{eq:ADMdagger})
becomes 
\be
\nabla^2h_{00}=-8\pi G T_{00} 
\label{eq:ADM1}
\ee 
while (\ref{eq:ADMstar}) becomes
\be
-\nabla^2h_{00}+\underbrace{ \nabla^2h_{jj} - 
h_{ij,ij}}_{\displaystyle \partial_i\left(\partial_i h_{jj}-\partial_j
h_{ij}\right)} = -8\pi GT_{00}
\label{eq:ADM2}
\ee
Add (\ref{eq:ADM1}) and (\ref{eq:ADM2}) to get
\be
T_{00}=\frac{1}{16\pi G}\partial_i\left(\partial_jh_{ij}-
\partial_ih_{jj}\right) \quad \mbox{(Cartesian coordinates)}
\ee
Since the source is weak we can assume that the spacetime is almost 
Minkowski, i.e. we treat $h_{\mu\nu}$ as a field on Minkowski spacetime. The 
total energy is now found by integrating $T_{00}$ over all space.
\be
E = \int_{\stackrel{\subtext{$t=$ constant}}{\subtext{all space}}} 
\dx{3}{x} T_{00}
\ee
  Using Gauss' law we can rewrite result as the surface integral
\be
E=\frac{1}{16\pi G}\oint_{\infty}dS_i \left(\partial_j h_{ij}-
\partial_i h_{jj}\right) \quad \mbox{(Cartesian coordinates)}
\ee
But this depends \emph{only} on the asymptotic data, so we may now change 
the source in any way we wish in the interior without changing $E$, provided
that the asymptotic metric is unchanged.  So \emph{formula for $E$ is valid in
general}. \\

This is the ADM formula for the energy of asymptotically flat spacetimes.

\subsection{Alternative Formula for ADM Energy}

Subtract (\ref{eq:ADM2}) from (\ref{eq:ADM1}) to get
\be
\partial_i\left(\partial_j h_{ij}-\partial_i h_{jj}\right)=-2\nabla^2 h_{00}
\ee This allows us to rewrite ADM formula as
\be
E=-\frac{1}{8\pi G} \oint_{\infty} dS_i\, \partial_i h_{00}
\ee
But (Exercise)
\be
g^{ij}\Gamma_{0j}^{\I\I 0} = -\half \partial_i h_{00} +\mathcal{O}
\left(\frac{1}{r^3}\right) \quad \mbox{($\Gamma=$ affine connection)}
\ee
and hence
\bea
E & = & \frac{1}{4\pi G}\oint_{\infty} dS_i\, g^{ij}\Gamma_{0j}^{\I\I 0} \\
 & = & \frac{1}{4\pi G}\oint_{\infty} dS_{0i}\, D^ik^0 \quad \mbox{where } 
k=\pd{}{t},\; dS_i\equiv dS_{0i}
\eea
But $k$ is asymptotically Killing, i.e.
\be
D^{\mu}k^{\nu}+D^{\nu}k^{\mu} = \mathcal{O}\left(\frac{1}{r^3}\right)
\ee
so
\be
E= -\frac{1}{8\pi G}\oint_{\infty} dS_{\mu\nu}D^{\mu}k^{\nu}
\ee

\section{Komar Integrals}\index{Komar integrals}

Let $V$ be a volume of spacetime on a spacelike hypersurface $\Sigma$, with 
boundary $\partial V$.  To every \emph{Killing} vector field $\xi$ we can
associate the Komar integral
\be
Q_{\xi}(V) = \frac{c}{16\pi G}\oint_{\partial V}dS_{\mu\nu}D^{\mu}\xi^{\nu} 
\ee
for some constant $c$.  Using Gauss' law
\be
Q_{\xi}(V) = \frac{c}{8\pi G}\int_V dS_{\mu}D_{\nu}D^{\mu}\xi^{\nu}
\ee
\fbox{\parbox{6in}{
\paragraph{Lemma} $D_{\nu}D_{\mu}\xi^{\nu} = R_{\mu\nu}\xi^{\nu}$ for 
Killing vector field $\xi$.
\paragraph{Proof} By contraction of previous `Killing vector Lemma.'
}} \\

Using Lemma,
\bea
Q_{\xi}(V) & = & \frac{c}{8\pi G} \int_V dS_{\mu}R^{\mu}_{\I\nu}\xi^{\nu}  \\
 & = & c\int dS_{\mu}\left(T^{\mu}_{\I\I\nu}\xi^{\nu}-\half 
T\xi^{\mu}\right) \quad \mbox{(by Einstein's eqs.)} \\
 & = & \int dS_{\mu}J^{\mu}(\xi)
\eea
where
\be
J^{\mu}(\xi) = c\left(T^{\mu}_{\I\I\nu}\xi^{\nu}-\half T\xi^{\mu}\right)
\ee
\paragraph{Proposition} $\partial_{\mu}J^{\mu}(\xi)=0$.
\paragraph{Proof} Using $D_{\mu}T^{\mu\nu}=0$ we have
\bea
D_{\mu}J^{\mu} & = & c\underbrace{ \left(T^{\mu\nu}D_{\mu}\xi_{\nu}-
\half TD_{\mu}\xi^{\mu}\right) }_{\displaystyle \mbox{0 for Killing vector
$\xi$}} -\frac{c}{2}\xi\cdot \partial T \\
 & = & \frac{c}{2}\xi \cdot\partial R \quad \mbox{(by Einstein's eqs.)} \\
 & = & 0 \quad \mbox{for Killing vector field $\xi$}
\eea
(In this last step, choose coordinates s.t. $\xi\cdot\partial=
\partial/\partial\alpha$, then the metric is $\alpha$-independent
($\fltpd{g_{\mu\nu}}{\alpha}=0$), so $R$ is too ($\fltpd{R}{\alpha}=0$)). \\

Since $J^{\mu}(\xi)$ is a `conserved current', the charge $Q_{\xi}(V)$ is 
time-independent provided $J^{\mu}(\xi)$ vanishes on $\partial V$, 
\emph{just as
for electric charge}.
\paragraph{Exercise} $\xi=k$ (time-translation Killing vector field)
\bebox{
E(V)=-\frac{1}{8\pi G}\oint_{\partial V}dS_{\mu\nu}D^{\mu}k^{\nu}
}
i.e. $c=-2$, is fixed by comparison with previous formula derived for total 
energy, i.e. by choosing $V=$ 2-sphere at spatial $\infty$.

\fbox{\parbox{6in}{
\paragraph{Exercise} Verify that $E(V)=M$ for Schwarzschild, for any $V$ with 
$\partial V$ in exterior $(r>2M)$ spacetime.
}}

\subsection{Angular Momentum in Axisymmetric Spacetimes}

Return to Komar integral.  Let $\xi=m=\fltpd{}{\phi}$ and choose $c=1$ to get
\bebox{
J(V)=\frac{1}{16\pi G}\oint_{\partial V}dS_{\mu\nu}D^{\mu}m^{\nu}
}
\emph{Note here factor of $-1/2$ relative to Komar integral for the energy}.

To check coefficient, use Gauss' law to write $J(V)=\int_V dS_{\mu}J^{\mu}(m)$ 
where
\be
J^{\mu}(m)=T^{\mu}_{\I\I\nu}m^{\nu}-\half Tm^{\mu}
\ee
If we choose $V$ to be on $t=$ constant hypersurface, and 
$m=\partial/\partial\phi$, then $dS_{\mu}m^{\mu}=0$, so
\be
J(V) = \int_{V}dV T^0_{\I\I\nu}m^{\nu} = \int_V 
dV\left(T^0_{\I\I 2}x^1-T^0_{\I\I 1}x^2 \right)
\ee 
in Cartesian coordinates $\left\{x^i; i=1,2,3\right\}$ where 
\be
m=x^1\pd{}{x^2}-x^2\pd{}{x^1}
\ee
For a \emph{weak source}, $g\approx \eta$ and
\be
J(V) \approx \varepsilon_{3jk}\int_V \dx{3}{x}x^j T^{k0}
\ee
which is result for $3^{\subtext{rd}}$ component of angular momentum of field 
in Minkowski spacetime with stress tensor $T_{\mu\nu}$. 

So the \emph{total} angular momentum of an asymptotically flat spacetime is 
found by taking $\partial V$ to be a 2-sphere at spatial infinity
\bebox{
J=\frac{1}{16\pi G}\oint_{\infty}dS_{\mu\nu}D^{\mu}m^{\nu}
}

\section{Energy Conditions}

\fbox{\parbox{6in}{
$T_{\mu\nu}$ satisfies the \emphin{dominant energy condition} if for 
\emph{all} future-directed timelike vector fields $v$, the vector field
\be
j(v) \equiv -v^{\mu}T_{\mu}^{\I\nu}\partial_{\nu}
\ee
is future-directed non-spacelike, or zero.
}} \\

All physically reasonable matter satisfies this condition, e.g. for massless 
scalar field $\Phi$ (with $T_{\mu\nu}=
\partial_{\mu}\Phi\partial_{\nu}\Phi-\half
g_{\mu\nu}(\partial\Phi)^2$):
\bea
j^{\mu}(v) & = & -v\cdot \partial\Phi 
\partial^{\mu}\Phi+\half v^{\mu}(\partial\Phi)^2 \\
j^2(v) & = & \frac{1}{4}v^2\underbrace{ 
\left((\partial\Phi)^2\right)^2}_{\ge 0} \le 0 \quad \mbox{if $v^2<0$}
\eea
so $j(v)$ is timelike or null if $v$ is timelike.  Since $v$ is 
assumed future-directed, $j(v)$ will be too if $-v\cdot j >0$.  Allowing for
$j=0$ means that we have to prove that $-v\cdot j\ge 0$.  Now
\bea
-v\cdot j & = & (v\cdot\partial\Phi)^2-\half v^2(\partial\Phi)^2 \\
 & = & \half(v\cdot \partial\Phi)^2+\half\left(-v^2\right)
\left(\partial\Phi-\frac{v(v\cdot\partial\Phi)}{v^2}\right)^2 
\eea
But $(v\cdot\partial\Phi)^2\ge 0$ and $-v^2>0$ for timelike $v$, so we 
have to prove that 
\be
\left(\partial\Phi-\frac{v(v\cdot\partial\Phi)}{v^2}\right)^2 \ge 0
\ee
i.e. that $\left(\partial\Phi-\frac{v(v\cdot\partial\Phi)}{v^2}\right)$ is 
spacelike or zero.  This follows from
\be
v\cdot \left(\partial\Phi-\frac{v(v\cdot\partial\Phi)}{v^2}\right) = 0
\ee
since $v\cdot V < 0$ for any non-zero timelike or null vector for timelike 
$v$ (choose coordinates s.t. $v=(1,\vec{0})$).  So if $v\cdot V=0$ then $V$
cannot be timelike or null. \\

Since $-v\cdot j=v^{\mu}v^{\nu}T^{\mu\nu}$, the dominant energy condition 
implies that $v^{\mu}v^{\nu}T_{\mu\nu} \ge 0$ for all timelike $v$.  By
continuity it also implies the

\fbox{\parbox{6in}{
\emph{Weak energy condition}\index{weak energy condition}
\be
v^{\mu}v^{\nu}T_{\mu\nu}\ge 0 \quad \forall\; \mbox{non-spacelike $v$}
\ee
}}\\

There is also the 

\fbox{\parbox{6in}{
\emph{Strong energy condition}\index{strong energy condition}
\be
v^{\mu}v^{\nu}\left(T_{\mu\nu}-\half g_{\mu\nu}T\right)\ge 0 \quad 
\forall\;\mbox{non-spacelike $v$}
\ee
}}\\

Note, \emph{Dominant $\not\Leftrightarrow$ Strong}. \\


The strong energy condition is needed to prove the singularity theorems, but 
the dominant energy condition is the physically important one.  (An inflationary
universe violates the strong energy condition).  For example it is needed for 
the

\subsubsection{Positive Energy Theorem (Shoen \& Yau, Witten)}
\index{positive energy theorem}

The ADM energy of an asymptotically-flat spacetime satisfying 
$G_{\mu\nu}=8\pi GT_{\mu\nu}$ is positive semi-definite, and vanishes
\emph{only} for Minkowski spacetime with $T_{\mu\nu}=0$, provided that 
\newcounter{positenergy}
\begin{list}{\roman{positenergy})}
{\usecounter{positenergy}}
\item $\exists$ an initially non-singular Cauchy surface (otherwise 
$M<0$ Schwarzschild would be a counter-example).

\item $T_{\mu\nu}$ satisfies the dominant energy condition (clearly, 
\emph{some} condition on $T_{\mu\nu}$ is necessary).

\item Some other technical assumptions which we ignore here.
\end{list}
