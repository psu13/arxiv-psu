
\chapter{Schwarzschild Black Hole}

\section{Test particles: geodesics and affine parameterization}

Let $\mathcal{C}$ be a timelike curve with endpoints $A$ and $B$.  The action 
for a particle of mass $m$ moving on $\mathcal{C}$ is
\be
I=-mc^2\int^B_A d\tau
\ee 
where $\tau$ is proper time on $\mathcal{C}$.  Since
\be
d\tau = \sqrt{-ds^2}=\sqrt{-dx^{\mu}dx^{\nu}g_{\mu\nu}}=
\sqrt{-\dot{x}^{\mu}\dot{x}^{\nu}g_{\mu\nu}} d\lambda
\ee
where $\lambda$ is an arbitrary parameter on $\mathcal{C}$ and 
$\dot{x}^{\mu}=\frac{dx^{\mu}}{d\lambda}$, we have
\be
I\left[x\right] = -m\int^{\lambda_B}_{\lambda_A} 
d\lambda\sqrt{ -\dot{x}^{\mu}\dot{x}^{\nu}g_{\mu\nu}} \quad (c=1)
\ee
The particle worldline, $\mathcal{C}$, will be such that 
$\delta I/\delta x(\lambda)=0$.  By definition, this is a \emphin{geodesic}. 
For the purpose of finding geodesics, an equivalent action is
\be
I\left[x,e\right]= \half \int^{\lambda_B}_{\lambda_A}d\lambda 
\left[ e^{-1}(\lambda)\dot{x}^{\mu}\dot{x}^{\nu}g_{\mu\nu}-m^2e(\lambda)\right]
\ee
where $e(\lambda)$ (the `einbein'\index{einbein}) is a new independent function.
\paragraph{Proof of equivalence} (for $m\neq 0$)
\be
\frac{\delta I}{\delta e} =0 \quad \Rightarrow \quad e = \frac{1}{m}
\sqrt{ -\dot{x}^{\mu}\dot{x}^{\nu}g_{\mu\nu}} =
\frac{1}{m}\frac{d\tau}{d\lambda} 
\label{eq:einbein_one}
\ee
and (exercise)
\be
\frac{\delta I}{\delta x^{\mu}} = 0 \quad \Rightarrow 
\quad D_{(\lambda)}\dot{x}^{\mu}=(e^{-1}\dot{e})\dot{x}^{\mu}
\label{eq:einbein_two}
\ee
where
\be
D_{(\lambda)}V^{\mu}(\lambda)\equiv \frac{d}{d\lambda}V^{\mu}+
\dot{x}^{\nu}\gamsym{\mu}{\rho}{\nu}V^{\rho}
\ee
If (\ref{eq:einbein_one}) is substituted into (\ref{eq:einbein_two}) we get 
the EL equation $\delta I/\delta x^{\mu}=0$ of the original action $I[x]$
(exercise), hence equivalence. \\

The freedom in the choice of parameter $\lambda$ is equivalent to the freedom 
in the choice of function $e$.  Thus any curve $x^{\mu}(\lambda)$ for which
$t^{\mu}=\dot{x}^{\mu}(\lambda)$ satisfies
\be
D_{(\lambda)}t^{\mu}V^{\mu}=f(x)t^{\mu} \quad \mbox{(arbitrary $f$)}
\ee is a geodesic.  Note that for any vector field on $\mathcal{C}$, 
$V^{\mu}(x(\lambda))$,
\bea
t^{\nu}D_{\nu}V^{\mu} & \equiv & t^{\nu}\partial_{\nu}V^{\mu}+
t^{\nu}\gamsym{\mu}{\nu}{\rho}V^{\rho} \\
 & = & \frac{d}{d\lambda}V^{\mu} +\dot{x}^{\nu}
\gamsym{\mu}{\nu}{\rho}V^{\rho} \\
 & = & D_{(\lambda)}V^{\mu}
\eea
Since t is \emph{tangent} to the curve $\mathcal{C}$, a vector field $V$ on 
$\mathcal{C}$ for which
\be
D_{(\lambda)}=f(\lambda)V^{\mu} \quad \mbox{(arbitrary $f$)}
\ee 
is said to be \emph{parallely transported}\index{parallel transport} along the 
curve.  A geodesic is therefore \emph{a curve whose tangent is parallely
transported along it} (w.r.t. the affine connection). \\

A natural choice of parameterization is one for which
\be
D_{(\lambda)}t^{\mu}=0 \quad (t^{\mu}=\dot{x}^{\mu})
\ee
This is called \emph{affine parameterization}\index{affine parameter}.  For a 
timelike geodesic it corresponds to $e(\lambda)=$ constant, or
\be
\lambda \propto \tau +\mbox{constant}
\ee

The einbein form of the particle action has the advantage that we can take the 
$m\to 0$ limit to get the action for a massless particle.  In this case
\be
\frac{\delta I}{\delta e} = 0 \quad \Rightarrow \quad ds^2=0 \quad (m=0)
\ee
while (\ref{eq:einbein_two}) is unchanged.  We still have the freedom to 
choose $e(\lambda)$ and the choice $e=$ constant is again called affine
parameterization. \\

\noi\fbox{\parbox{\textwidth}{
\subsubsection{Summary}

Let $t^{\mu}=\bgfrac{dx^{\mu}(\lambda)}{d\lambda}$ and ${\displaystyle 
\sigma = \left\{ \begin{array}{cc} 1 & m\neq 0 \\ 0 & m=0 
\end{array}\right\} }$.

Then
\bebox{
\begin{array}{rcl} t\cdot Dt^{\mu} & \equiv & D_{(\lambda)}t^{\mu} = 0 \\
ds^2 & = & -\sigma d\lambda^2 \end{array}}
are the equations of affinely-parameterized timelike or null geodesics. \\
}}
% end of page 8
\section{Symmetries and Killing Vectors}

Consider the transformation
\be
x^{\mu}\to x^{\mu}-\alpha k^{\mu}(x), \quad (e\to e)
\ee
Then (Exercise)
\be
I\left[x,e\right] \to I\left[x,e\right]-\frac{\alpha}{2}
\int^{\lambda_B}_{\lambda_A}d\lambda\,
e^{-1}\dot{x}^{\mu}\dot{x}^{\nu}
\left(\Lie_{k}g\right)_{\mu\nu}+\mathcal{O}\left(\alpha^2\right)
\ee
where
\bea
\left(\Lie_{k}g\right)_{\mu\nu} & = & k^{\lambda}g_{\mu\nu,\lambda}+
k^{\lambda}_{\I,\mu}g_{\lambda\nu}+k^{\lambda}_{\I,\nu}g_{\lambda\mu} \\
 & = & 2D_{(\mu}k_{\nu)} \quad \mbox{(Exercise)}
\eea
Thus the action is invariant to first order if 
\be
\Lie_k g=0
\ee
A vector field $k^{\mu}(x)$ with this property is a 
\emph{Killing vector}\index{Killing!vector} field.  $k$ is associated with a
symmetry of the particle action and hence with a conserved charge. This charge
is (Exercise)
\be
Q=k^{\mu}p_{\mu}
\ee
where $p_{\mu}$ is the particle's 4-momentum.
\bea
p_{\mu} & = & \pdoL{\dot{x}^{\mu}}=e^{-1}\dot{x}^{\nu}g_{\mu\nu} \\
 & = & m\frac{dx^{\nu}}{d\tau}g_{\mu\nu} \quad \mbox{when } m\neq 0
\eea
\paragraph{Exercise} Check that the Euler-Lagrange equations imply
\bdm
\frac{dQ}{d\lambda}=0
\edm

Quantize, $p_{\mu}\to -i\partial/\partial x^{\mu}\equiv -i\partial_{\mu}$.  
Then
\be
Q \to -ik^{\mu}\partial_{\mu}
\ee
Thus the components of $k$ can be viewed as the components of a 
\emph{differential operator} in the basis $\left\{\partial_{\mu}\right\}$.
\be
k\equiv k^{\mu}\partial_{\mu}
\ee
It is convenient to identify this operator with the vector field.  Similarly 
for all other vector fields, e.g. the tangent vector to a curve
$x^{\mu}(\lambda)$ with affine parameter $\lambda$.
\be
t=t^{\mu}\partial_{\mu}=\frac{dx^{\mu}}{d\lambda}\partial_{\mu}=
\frac{d}{d\lambda}
\ee
For any vector field, $k$, local coordinates can be found such that 
\be
k=\partial/\partial\xi
\ee 
where $\xi$ is one of the coordinates. In such a coordinate system
\be
\Lie_kg_{\mu\nu}=\pd{}{\xi}g_{\mu\nu}
\ee
So $k$ is Killing if $g_{\mu\nu}$ is independent of $\xi$.

e.g. for Schwarzschild $\partial_t g_{\mu\nu}=0$, so $\partial/\partial t$ 
is a Killing vector field.  The conserved quantity is
\be
mk^{\mu}\frac{dx^{\nu}}{d\tau}g_{\mu\nu} = mg_{00}\frac{dt}{d\tau}
=-m\varepsilon \quad (\varepsilon=\mbox{ energy/unit mass})
\ee

% p11 of notes
\section{Spherically-Symmetric Pressure Free Collapse}

While it is impossible to say with complete confidence that a real star of 
mass $M\gg 3M_{\odot}$ will collapse to a BH, it is easy to invent idealized,
but physically possible, stars that definitely do collapse to black holes. One
such `star' is a spherically-symmetric ball of `dust' (i.e. zero pressure
fluid).  \emphin{Birkhoff's theorem} implies that the metric outside the star is
the \emphin{Schwarzschild metric}.  Choose units for which 
\be
G=1, \quad c=1.
\ee
Then
\be
ds^2= -\left(\Schr\right)dt^2+\left(\Schr\right)^{-1}dr^2+r^2d\Omega^2 
\ee
where 
\be
d\Omega^2=d\theta^2+\sin^2\theta d\varphi^2 \quad \mbox{(metric on a unit 
2-sphere)} 
\ee
This is valid outside the star but also, by continuity of the metric, at the 
surface.  If $r=R(t)$ on the surface we have 
\be
ds^2=-\left[ \left(\SchR\right)-\left(\SchR\right)^{-1}
\dot{R}^2\right]dt^2+R^2d\Omega^2, \quad \left(\dot{R}=\frac{d}{dt}R\right) 
\ee
On the surface zero pressure and spherical symmetry implies that a point on 
the surface follows a \emph{radial timelike geodesic}, so $d\Omega^2=0$ and
$ds^2=-d\tau^2$, so
\be
1=\left[\left(\SchR\right)-\left(\SchR\right)^{-1}\dot{R}^2\right]
\left(\frac{dt}{d\tau}\right)^2 
\label{eq:collapse_star}
\ee
But also, since $\partial/\partial t$ is a Killing vector we have 
\emph{conservation of energy}: 
\be
\varepsilon=-g_{00}\frac{dt}{d\tau}=\left(\SchR\right)\frac{dt}{d\tau} 
\quad \mbox{(energy/unit mass)}
\ee
$\varepsilon$ is \emph{constant on the geodesics}.  Using this in 
(\ref{eq:collapse_star}) gives
\be
1=\left[\left(\SchR\right)-\left(\SchR\right)^{-1}\dot{R}^2\right]
\left(\SchR\right)^{-2}\varepsilon^2
\ee
or
\bebox{
\dot{R}^2=\frac{1}{\varepsilon^2}\left(\SchR\right)^2\left(\frac{2M}{R}-1+
\varepsilon^2\right)
\label{eq:collapse_dagger}}
($\varepsilon < 1$ for gravitationally bound particles).
\begin{center}\input{p12-1.pictex}\end{center}
$\dot{R}=0$ at $R=R_{\subtext{max}}$ so we consider collapse to begin with 
zero velocity at this radius.  $R$ then decreases and approaches $R=2M$
asymptotically as $t\to\infty$.  So an observer `sees' the star contract at most
to $R=2M$ but no further. \\

However from the point of view of an observer on the surface of the star, the 
relevant time variable is proper time\index{proper time} along a radial
geodesic, so use
\be
\frac{d}{dt}=\left(\frac{dt}{d\tau}\right)^{-1}\frac{d}{d\tau}=
\frac{1}{\varepsilon}\left(\SchR\right)\frac{d}{d\tau}
\ee
to rewrite (\ref{eq:collapse_dagger}) as
\bebox{
\left(\frac{dR}{d\tau}\right)^2=\left(\frac{2M}{R}-1+\varepsilon^2\right)=
(1-\varepsilon^2)\left(\frac{R_{\subtext{max}}}{R}-1\right)}
\begin{center}\input{p13-1.pictex}\end{center}
Surface of the star falls from $R=R_{\subtext{max}}$ through $R=2M$ in 
\emph{finite proper time}.  In fact, it falls to $R=0$ in proper time
\be
\tau=\frac{\pi M}{(1-\varepsilon)^{3/2}} \quad\mbox{(Exercise)}
\ee
Nothing special happens at $R=2M$ which suggests that we investigate the 
spacetime near $R=2M$ in coordinates adapted to infalling observers.  It is
convenient to choose \emph{massless} particles. \\

On radial null geodesics in Schwarzschild spacetime
\be
dt^2=\frac{1}{\left(\Schr\right)^2}dr^2\equiv\left(dr^*\right)^2
\ee
where
\be
r^*=r+2M\ln\left| \frac{r-2M}{2M}\right| 
\ee
is the \emphin{Regge-Wheeler radial coordinate}.  As $r$ ranges from $2M$ 
to $\infty$, $r^*$ ranges from $-\infty$ to $\infty$.  Thus 
\be
d(t\pm r^*)=0 \quad  \mbox{on radial null geodesics}
\ee
Define the ingoing radial null coordinate $v$ by 
\be
v=t+r^*,\quad  -\infty<v<\infty
\ee
and rewrite the Schwarzschild metric in \emph{ingoing Eddington-Finkelstein 
coordinates}\index{Eddington-Finkelstein coordinates!ingoing}
($v,r,\theta,\phi$).
\bea
ds^2 & = & \left(\Schr\right)\left(-dt^2+d{r^*}^2\right)+r^2d\Omega^2 \\
 & = & -\left(\Schr\right)dv^2+2dr\,dv+r^2d\Omega^2 
\eea
This metric is \emph{initially} defined for $r>2M$ since the relation 
$v=t+r^*(r)$ between $v$ and $r$ is only defined for $r>2M$, but it can now be
\emph{analytically continued} to all $r>0$.  Because of the $dr\,dv$ cross-term
the metric in EF coordinates is \emph{non-singular at $r=2M$}, so the
singularity in Schwarzschild coordinates was really a coordinate singularity. 
There is nothing at $r=2M$ to prevent the star collapsing through $r=2M$.  This
is illustrated by a \emphin{Finkelstein diagram}, which is a plot of $t^*=v-r$
against $r$:
\begin{center}\input{p15-1.pictex}\end{center}
The light cones distort as $r\to 2M$ from $r>2M$, so that no future-directed 
timelike or null worldline can reach $r>2M$ from $r\le 2M$.

\paragraph{Proof}  When $r\le 2M$,
\bea
2dr\,dv & = & -\left[ -ds^2 + \left(\frac{2M}{r}-1\right)dv^2+ r^2d\Omega^2 
\right] \\
 & \le & 0 \quad\mbox{when $ds^2\le 0$}
\eea
for all timelike or null worldlines $dr\,dv\le 0$.  $dv>0$ for future-directed 
worldlines, so $dr\le 0$ with equality when $r=2M$, $d\Omega=0$ (i.e. ingoing
radial null geodesics at $r=2M$).

\subsection{Black Holes and White Holes}

No signal from the star's surface can escape to infinity once the surface has 
passed through $r=2M$.  The star has collapsed to a \emphin{black hole}.  For
the external observer, the surface never actually reaches $r=2M$, but as $r\to
2M$ the redshift of light leaving the surface increases \emph{exponentially}
fast and the star effectively disappears from view within a time $\sim MG/c^3$. 
The late time appearance is dominated by photons escaping from the unstable
photon orbit at $r=3M$. \\

The hypersurface $r=2M$ acts like a one-way membrane.  This may seem 
paradoxical in view of the time-reversibility of Einstein's equations.  Define
the \emph{outgoing} radial null coordinate $u$ by
\be
u=t-r^*,\quad -\infty<u<\infty 
\ee
and rewrite the Schwarzschild metric in \emph{outgoing Eddington-Finkelstein 
coordinates}\index{Eddington-Finkelstein coordinates!outgoing}
($u,r,\theta,\phi$). 
\be
ds^2=-\left(\Schr\right)du^2-2dr\,du+r^2d\Omega^2
\ee
This metric is initially defined only for $r>2M$ but it can be analytically 
continued to all $r>0$.  However the $r<2M$ region in outgoing EF coordinates is
\ul{not} the same as the $r<2M$ region in ingoing EF coordinates.  To see this,
note that for $r\le 2M$
\bea
2dr\,du & = & -ds^2+\left(\frac{2M}{r}-1\right)du^2+r^2d\Omega^2 \\ 
 & \ge & 0 \quad \mbox{when } ds^2\le 0
\eea
i.e. $dr\,du\ge 0$ on timelike or null worldlines.  But $du>0$ for 
future-directed worldlines so $dr\ge 0$, with equality when $r=2M$, $d\Omega=0$,
and $ds^2=0$.  In this case, a star with a surface at $r<2M$ must \emph{expand}
and explode through $r=2M$, as illustrated in the following Finkelstein
diagram\index{Finkelstein diagram}.
\begin{center}\input{p17-1.pictex}\end{center}
This is a \emphin{white hole}, the time reverse of a black hole.  Both black 
and white holes are allowed by G.R. because of the time reversibility of
Einstein's equations, but white holes require very special initial conditions
near the singularity, whereas black holes do not, so only black holes can occur
in practice (cf. irreversibility in thermodynamics).

\subsection{Kruskal-Szekeres Coordinates}\index{Kruskal-Szekeres coordinates}

The exterior region $r>2M$ is covered by both ingoing \emph{and} outgoing 
Eddington-Finkelstein coordinates, and we may write the Schwarzschild metric in
terms of $(u,v,\theta,\phi)$
\be
ds^2=-\left(\Schr\right)du\,dv +r^2d\Omega^2 
\ee
We now introduce the new coordinates $(U,V)$ defined (for $r>2M$) by
\be
U=-e^{-u/4M},\quad V=e^{v/4M}
\ee
in terms of which the metric is now 
\bebox{
ds^2=\frac{-32M^3}{r}e^{-r/2M}dU\,dV+r^2d\Omega^2
}
where $r(U,V)$ is given implicitly by $UV=-e^{r^*/2M}$ or
\bebox{
UV=-\left( \frac{r-2M}{2M}\right) e^{r/2M}
}
We now have the Schwarzschild metric in KS coordinates $(U,V,\theta,\phi)$.  
Initially the metric is defined for $U<0$ and $V>0$ but it can be extended by
analytic continuation to $U>0$ and $V<0$.  Note that $r=2M$  corresponds to
$UV=0$, i.e. \ul{either} $U=0$ \ul{or} $V=0$.  The singularity at $r=0$
corresponds to $UV=1$. \\

It is convenient to plot lines of constant $U$ and $V$ (outgoing or ingoing
radial null geodesics) at $45^{^0}$, so the spacetime diagram now looks like 
\begin{center}\input{p19-1.pictex}\end{center}
There are four regions of Kruskal spacetime, depending on the signs of $U$ and 
$V$.  Regions I and II are also covered by the ingoing Eddington-Finkelstein
coordinates.  These are the only regions relevant to gravitational collapse
because the other regions are then replaced by the star's interior, e.g. for
collapse of homogeneous ball of pressure-free fluid: 
\begin{center}\input{p19-2.pictex}\end{center}
Similarly, regions I and III are those relevant to a white hole.

\subsubsection{Singularities and Geodesic Completeness}

A singularity of the metric is a point at which the determinant of either it or
its inverse vanishes. However, a singularity of the metric may be simply due
to a failure of the coordinate system. A simple two-dimensional example is the
origin in plane polar coordiates, and we have seen that the singularity of
the Schwarzschild metric at the Schwarzschild radius is of this type. Such
singularities are removable. If no coordinate system exists for which the
singularity is removable then it is irremovable, i.e. a genuine singularity of
the spacetime. Any singularity for which some scalar constructed from the
curvature tensor blows up as it is approached is irremovable. Such singularities
are called `curvature singularities'. The singularity at $r=0$ in the
Schwarzschild metric is an example. Not all irremovable singularities are
`curvature singularities', however. Consider the singularity at the tip of a
cone formed by rolling up a sheet of paper. All curvature invariants remain
finite as the singularity is approached; in fact, in this two-dimensional
example the curvature tensor is everywhere zero. If we could assign a curvature
to the singular point at the tip of the cone it would have to be infinite but,
strictly speaking, we cannot include this point as part of the manifold since
there is no coordinate chart that covers it. 

We might try to make a virtue of this necessity: by excising the regions
containing irremovable singularities we apparently no longer have to worry about
them. However, this just leaves us with the essentially equivalent problem of
what to do with curves that reach the boundary of the excised region. There is
no problem if this boundary is at infinity, i.e. at infinite affine parameter
along all curves that reach it from some specified point in the interior, but
otherwise the inability to continue all curves to all values of their affine
parameters may be taken as the defining feature of a `spacetime singularity'.
Note that the concept of affine parameter is not restricted to geodesics, e.g.
the affine parameter on a timelike curves is the proper time on the curve
regardless of whether the curve is a geodesic. This is just as well, since there
is no good physical reason why we should consider only geodesics. Nevertheless,
it is virtually always true that the existence of a singularity as just
defined can be detected by the incompleteness of some geodesic, i.e. there is
some geodesic that cannot be continued to all values of its affine parameter.
For this reason, and because it is simpler, we shall follow the common practice
of defining a spacetime singularity in terms of `geodesic incompleteness'.
Thus, {\it a spacetime is non-singular if and only if all geodesics can be
extended to all values of their affine parameters}, changing coordinates if
necessary. 

In the case of the Schwarzschild vacuum solution, a particle on an ingoing radial
geodesics will reach the coordinate singularity at $r=2M$ at finite affine
parameter but, as we have seen, this geodesic can be continued into region II
by an appropriate change of coordinates. Its continuation will then approach the
curvature singularity at $r=0$, coming arbitrarily close for finite affine
parameter. The excision of any region containing $r=0$ will therefore lead to a
incompleteness of the geodesic. The vacuum Schwarzschild solution is therefore
singular. The singularity theorems of Penrose and Hawking show that geodesic
incompleteness is a \emph{generic feature} of gravitational collapse, and not
just a special feature of spherically symmetric collapse.

\subsubsection{Maximal Analytic Extensions}

Whenever we encounter a singularity at finite affine parameter along some
geodesic (timelike, null, or spacelike) our first task is to identify it as
removable or irremovable. In the former case we can continue through it by a
change of coordinates.  By considering all geodesics we can construct in
this way the \emphin{maximal analytic extension} of a given spacetime in which
\emph{any geodesic that does not terminate on an irremovable singularity can be
extended to arbitrary values of its affine parameter}. The Kruskal manifold
is the maximal analytic extension of the Schwarzschild solution, so no more
regions can be found by analytic continuation.

\subsection{Eternal Black Holes}

A black hole formed by gravitational collapse is not time-symmetric because it 
will continue to exist into the indefinite future but did not always exist in
the past, and vice-versa for white holes.  However, one can imagine a
time-symmetric eternal black hole that has always existed (it could equally well
be called an eternal white hole, but isn't).  In this case there is no matter
covering up part of the Kruskal spacetime and all four regions are relevant.  
In region I
\be
\frac{U}{V}=e^{-t/2M}
\ee
so hypersurfaces of constant Schwarzschild time $t$ are straight lines through 
the origin in the Kruskal spacetime.
\begin{center}\input{p21-1.pictex}\end{center}
These hypersurfaces have a part in region I and a part in region IV.  Note that 
$(U,V)\to (-U,-V)$ is an isometry of the metric so that region IV is isometric
to region I. \\

To understand the geometry of these $t=$ constant hypersurfaces it is 
convenient to rewrite the Schwarzschild metric in \emphin{isotropic coordinates}
$(t,\rho,\theta,\phi)$, where $\rho$ is the new radial coordinate
\be
r=\left(1+\frac{M}{2\rho}\right)^2\rho
\ee
Then (\bold{Exercise})
\be
ds^2 = -\left( \frac{ 1-\frac{M}{2\rho} }{ 1+\frac{M}{2\rho}} 
\right)^2 dt^2 + \left(1+\frac{M}{2\rho}\right)^4 \underbrace{ \left[
d\rho^2+\rho^2d\Omega^2 \right] }_{\mbox{flat 3-space metric}} 
\ee
In isotropic coordinates, the $t=$ constant hypersurfaces are 
\emph{conformally flat}, but to each value of $r$ there corresponds \emph{two}
values of $\rho$
\begin{center}\input{p22-1.pictex}\end{center}
The two values of $\rho$ are exchanged by the isometry, $\rho\to M^2/4\rho$ 
which has $\rho=M/2$ as its fixed `point', actually a fixed 2-sphere of radius
$2M$.  This isometry corresponds to the $(U,V)\to (-U,-V)$ isometry of the
Kruskal spacetime.  The isotropic coordinates cover only regions I and IV since
$\rho$ is complex for $r<2M$.
\begin{center}\input{p22-2.pictex}\end{center}
As $\rho\to M/2$ from either side the radius of a 2-sphere of constant 
$\rho$ on a $t=$ constant hypersurface decreases to minimum of $2M$ at
$\rho=M/2$, so $\rho=M/2$ is a \emph{minimal 2-sphere}.  It is the midpoint of
an \emphin{Einstein-Rosen bridge} connecting spatial sections of regions I and
IV.  
\begin{center}\input{p23-1.pictex}\end{center}

\subsection{Time translation in the Kruskal Manifold}

The time translation $t\to t+c$, which is an isometry of the Schwarzschild 
metric becomes
\be
U\to e^{-c/4M}U,\quad V\to e^{c/4M}V 
\ee
in Kruskal coordinates and extends to an isometry of the entire Kruskal 
manifold.  The infinitesimal version
\be
\delta U = -\frac{c}{4M}U, \quad \delta V=\frac{c}{4M}V
\ee
is generated by the Killing vector field
\be
k=\frac{1}{4M}\left(V\pd{}{V}-U\pd{}{U}\right)
\ee
which equals $\fltpd{}{t}$ in region I.  It has the following properties
\newcounter{kproperties}
\begin{list}{(\roman{kproperties})}
{\usecounter{kproperties}}
\item $k^2=-\left(1-\frac{2M}{r}\right)\quad \Rightarrow \left\{\mbox{
\begin{tabular}{lcl}
timelike & in & I \& IV \\
spacelike & in & II \& III \\
null & on & $r=2M$, i.e. $\{U=0\}\cup\{V=0\}$
\end{tabular}}\right.$

\item $\{U=0\}$ and $\{V=0\}$ are \emphin{fixed sets} on $k$. \\

On $\left\{\mbox{\begin{tabular}{cc}
$\left\{U=0\right\}$ & $k=\fltpd{}{v}$ \\ $\left\{V=0\right\}$ & 
$k=\fltpd{}{u}$ \end{tabular}}\right\}$ where $v,u$ are EF null coordinates.

$\thdots$ $v$ is the natural group parameter on $\left\{U=0\right\}$.  
Orbits of $k$ correspond to $-\infty<v<\infty$, (where $v$ is well-defined).

\item Each point on the \emphin{Boyer-Kruskal axis}, $\{U=V=0\}$ (a 2-sphere) 
is a \emphin{fixed point} of $k$.

\end{list}
The orbits of $k$ are shown below
\begin{center}\input{p24-1.pictex}\end{center}

\subsection{Null Hypersurfaces}

Let $S(x)$ be a smooth function of the spacetime coordinates $x^{\mu}$ and 
consider a family of hypersurfaces $S= $ constant.  The vector fields normal to
the hypersurface are
\be
l=\tilde{f}(x)\left(g^{\mu\nu}\partial_{\nu}S\right)\pd{}{x^{\mu}}
\ee
where $\tilde{f}$ is an arbitrary non-zero function.  If $l^2=0$ for a 
particular hypersurface, $\mcN$, in the family, then $\mcN$ is said to be a
\emphin{null hypersurface}.

\paragraph{Example} Schwarzschild in ingoing Eddington-Finkelstein coordinates 
$(r,v,\theta,\phi)$ and the surface $S=r-2M$.
\bea
l & = & \tilde{f}(r)\left[\left(1-\frac{2M}{r}\right)\pd{S}{r}\pd{}{r}+
\pd{S}{r}\pd{}{v}+\pd{S}{v}\pd{}{r}\right] \\
 & = & \tilde{f}(r)\left[\left(1-\frac{2M}{r}\right)\pd{}{r}+\pd{}{v}\right]
\eea
while
\bea
l^2 & = & g^{\mu\nu}\partial_{\mu}S\partial_{\nu}S\tilde{f}^2 \\
 & = & g^{rr}\tilde{f}^2 = \left(1-\frac{2M}{r}\right)\tilde{f}^2
\eea
so \emph{$r=2M$ is a null hypersurface}, and 
\be
\left.l\right|_{r=2M} = \tilde{f}\pd{}{v}
\ee

\subsubsection{Properties of Null Hypersurfaces}

Let $\mathcal{N}$ be a null hypersurface with normal $l$.  A vector $t$, 
tangent to $\mathcal{N}$, is one for which $t\cdot l=0$.  But, since
$\mathcal{N}$ is null, $l\cdot l=0$, so \emph{$l$ is itself a tangent vector},
i.e. 
\be
l^{\mu}=\frac{dx^{\mu}}{d\lambda}
\ee
for some null curve $x^{\mu}(\lambda)$ in $\mcN$.

\paragraph{Proposition}  The curves $x^{\mu}(\lambda)$ are \emph{geodesics}.

\paragraph{Proof}  Let $\mathcal{N}$ be the member $S=0$ of the family of 
(not necessarily null) hypersurfaces $S=$ constant.  Then
$l^{\mu}=\tilde{f}g^{\mu\nu}\partial_{\nu}S$ and hence
\bea
l\cdot Dl^{\mu} & = & \left(l^{\rho}\partial_{\rho}\tilde{f}\right)
g^{\mu\nu}\partial_{\nu}S +\tilde{f}g^{\mu\nu}l^{\rho}D_{\rho}
\partial_{\nu}S \\
 & = & \left(l\cdot \partial\ln\tilde{f}\right)l^{\mu} +
\tilde{f}g^{\mu\nu}l^{\rho}D_{\nu}\partial_{\rho}S \quad 
\mbox{(by symmetry of
$\Gamma$)} \\
 & = & \left(\frac{d}{d\lambda}\ln\tilde{f}\right)l^{\mu}+
l^{\rho}\tilde{f}D^{\mu}\left(\tilde{f}^{-1}l_{\rho}\right) \\
 & = & \left(\frac{d}{d\lambda}\ln\tilde{f}\right)l^{\mu}+ 
l^{\rho}D^{\mu}l_{\rho}  -\left(\partial^{\mu}\ln\tilde{f}\right) l^2 \\
 & = & \left(\frac{d}{d\lambda}\ln\tilde{f}\right)l^{\mu}+
\half l^{2,\mu} -\left(\partial^{\mu}\ln\tilde{f}\right) l^2
\eea
Although $\left.l^2\right|_{\mcN}=0$ it doesn't follow that 
$\left.l^{2,\mu}\right|_{\mcN}=0$ unless the whole family of hypersurfaces $S=$
constant is null.  However since $l^2$ is constant on $\mcN$,
$t^{\mu}\partial_{\mu}l^2=0$ for any vector $t$ tangent to $\mcN$.  Thus
\be
\left.\partial_{\mu}l^2\right|_N \propto l_{\mu}
\ee
and therefore
\be
\left.l\cdot Dl^{\mu}\right|_N\propto l^{\mu}
\ee
i.e. $x^{\mu}(\lambda)$ is a geodesic (with tangent $l$).  The function 
$\tilde{f}$ can be chosen such that $l\cdot Dl=0$, i.e. so that $\lambda$ is an
affine parameter.

\paragraph{Definition}  The null geodesics $x^{\mu}(\lambda)$ with affine 
parameter $\lambda$, for which the tangent vectors $dx^{\mu}/d\lambda$ are
normal to a null hypersurface $\mathcal{N}$, are the \emph{generators of
$\mathcal{N}$}.

\paragraph{Example} $\mcN$ is $U=0$ hypersurface of Kruskal spacetime.  
Normal to $U=$ constant is
\bea
l & = & -\frac{\tilde{f}r}{32M^3}e^{r/2M}\pd{}{V} \\
\left.l\right|_N & = & -\frac{ \tilde{f}e}{16 M^2}\pd{}{V} \quad 
\mbox{since $r=2M$ on $\mathcal{N}$}
\eea
Note that $l^2\equiv 0$, so $l^2$ and $l^{2,\mu}$ both vanish on 
$\mathcal{N}$; this is because $U=$ constant is null for \emph{any} constant,
not just zero.  thus $l\cdot Dl=0$ if $\tilde{f}$ is \emph{constant}.  Choose
$\tilde{f}=-16M^2e^{-1}$.  Then 
\be
l=\pd{}{V}
\ee
is normal to $U=0$ and \emph{V is an affine parameter for the generator of 
this null hypersurface}.

\subsection{Killing Horizons}

\paragraph{Definition}  A null hypersurface $\mathcal{N}$ is a Killing 
horizon\index{Killing!horizon} of a Killing vector field $\xi$ if, on
$\mathcal{N}$, $\xi$ is normal to $\mcN$. \\

Let $l$ be normal to $\mathcal{N}$ such that $l\cdot Dl^{\mu}=0$ (affine 
parameterization).  Then, since, on $\mathcal{N}$, 
\be
\xi=fl
\ee
for some function $f$, it follows that
\bebox{
\xi\cdot D\xi^{\mu}=\kappa\xi^{\mu}, \quad \mbox{on }\mathcal{N}
}
where $\kappa = \xi \cdot \partial\ln\left|f\right|$ is called the 
\emphin{surface gravity}.

\subsubsection{Formula for surface gravity}

Since $\xi$ is normal to $\mcN$, \emphin{Frobenius' theorem} implies that
\bebox{
\left.\xi_{[\mu}D_{\nu}\xi_{\rho]}\right|_{\mathcal{N}}=0
\label{eq:normal_star}
}
where `$[\quad]$' indicates total anti-symmetry in the enclosed indices, 
$\mu,\nu,\rho$.  For a Killing vector field $\xi$, $D_{\mu}\xi_{\nu} =
D_{[\mu}\xi_{\nu]}$ (i.e. symmetric part of $D_{\mu}\xi_{\nu}$ vanishes). In
this case (\ref{eq:normal_star}) can be written as
\be
\left.\xi_{\rho}D_{\mu}\xi_{\nu}\right|_{\mathcal{N}} + 
\left.\left(
\xi_{\mu}D_{\nu}\xi_{\rho}-\xi_{\nu}D_{\mu}\xi_{\rho}\right)
\right|_{\mathcal{N}}
= 0
\ee
Multiply by $D^{\mu}\xi^{\nu}$ to get
\be
\left.\xi_{\rho}\left(D^{\mu}\xi^{\nu}\right)\left(D_{\mu}\xi_{\nu}\right)
\right|_{\mathcal{N}}  = 
-\left.2\left(D^{\mu}\xi^{\nu}\right)\xi_{\mu}\left(D_{\nu}\xi_{\rho}\right)
\right|_{\mathcal{N}}
\qquad \mbox{(since $D^{\mu}\xi^{\nu}=D^{[\mu}\xi^{\nu]}$)} 
\ee
or
\bea
\left.\xi_{\rho}\left(D^{\mu}\xi^{\nu}\right)\left(D_{\mu}\xi_{\nu}\right)
\right|_{\mathcal{N}}  & = & -\left.2\left(\xi\cdot
D\xi^{\nu}\right)D_{\nu}\xi_{\rho}\right|_{\mathcal{N}} \\
 & = & -\left. 2\kappa \xi\cdot D \xi_{\rho}\right|_{\mathcal{N}} 
\qquad \mbox{(for Killing horizon)} \\
 & = & -\left.2\kappa^2\xi_{\rho}\right|_{\mathcal{N}} 
\eea
Hence, except at points for which $\xi=0$,
\be
\fbox{ ${\displaystyle \kappa^2=-\left.\half\left(D^{\mu}\xi^{\nu}\right)
\left(D_{\mu}\xi_{\nu}\right)\right|_{\mathcal{N}} }$}
\label{eq:normal_dagger}
\ee
It will turn out that all points at which $\xi=0$ are limit points of orbits 
of $\xi$ for which $\xi\neq 0$, so continuity implies that this formula is valid
even when $\xi=0$ (Note that $\xi=0 \not\Rightarrow D_{\mu}\xi_{\nu}=0$). \\

\fbox{\parbox{6in}{
\paragraph{Killing Vector Lemma}

For a Killing vector field $\xi$
\be
\fbox{ ${ \displaystyle D_{\rho}D_{\mu}\xi^{\nu}=
R^{\nu}_{\I \mu\rho\sigma}\xi^{\sigma} }$}
\ee
where $R^{\nu}_{\I \mu\rho\sigma}$ is the Riemann tensor.

\bold{Proof: Exercise} (Question II.1)
}}

\paragraph{Proposition} $\kappa$ is constant on orbits of $\xi$. \\

\paragraph{Proof} Let $t$ be tangent to $\mathcal{N}$.  Then, since 
(\ref{eq:normal_dagger}) is valid everywhere on $\mathcal{N}$
\bea
t\cdot\partial\kappa^2 & = & -\left.\left(D^{\mu}\xi^{\nu}\right)
t^{\rho}D_{\rho}D_{\mu}\xi_{\nu}\right|_{\mathcal{N}} \\
 & = & -\left(D^{\mu}\xi^{\nu}\right)t^{\rho}
R_{\nu\mu\rho}^{\I\I\I\I\sigma}\xi_{\sigma} \qquad \mbox{(using Lemma)}
\eea
Now, $\xi$ is tangent to $\mcN$ (in addition to being normal to it).  
Choosing $t=\xi$ we have
\bea
\xi\cdot \partial\kappa^2 & = & -\left(D^{\mu}\xi^{\nu}\right) 
R_{\nu\mu\rho\sigma}\xi^{\rho}\xi^{\sigma}  \\
 & = & 0 \qquad \mbox{(since $R_{\nu\mu\rho\sigma}=-R_{\nu\mu\sigma\rho}$)}
\eea
so $\kappa$ is constant on orbits of $\xi$.

\subsubsection{Non-degenerate Killing horizons $(\kappa\neq 0)$}

Suppose $\kappa\neq 0$ on one orbit of $\xi$ in $\mathcal{N}$.  Then this 
orbit coincides with only \emph{part} of a null generator of $\mathcal{N}$. To
see this, choose coordinates on $\mathcal{N}$ such that 
\be
\xi=\pd{}{\alpha} \qquad  \mbox{(except at points where $\xi=0$)}
\ee
i.e. such that the group parameter $\alpha$ is one of the coordinates.  Then 
if $\alpha=\alpha(\lambda)$ on an orbit of $\xi$ with an affine parameter
$\lambda$
\be
\left.\xi\right|_{\subtext{orbit}} =\frac{d\lambda}{d\alpha}\frac{d}
{d\lambda}=fl \quad \left\{\begin{array}{rcl} f & = & \bgfrac{d\lambda}{d\alpha}
\\ \\ l & = & \bgfrac{d}{d\lambda} =
\bgfrac{dx^{\mu}(\lambda)}{d\lambda}\partial_{\mu} \end{array}\right.
\ee
Now 
\be
\pd{}{\alpha}\ln\left|f\right|=\kappa
\ee
where $\kappa$ is \emph{constant} for orbit on $\mcN$. For such orbits, 
$f=f_0e^{\kappa\alpha}$ for arbitrary constant $f_0$.  Because of freedom to
shift $\alpha$ by a constant we can choose $f_0 = \pm \kappa$ without loss of
generality, i.e. 
\be
\frac{d\lambda}{d\alpha}=\pm\kappa e^{\kappa\alpha} \quad \Rightarrow 
\quad \lambda = \pm e^{\kappa\alpha}+\mbox{constant}
\ee
Choose constant $=0$
\bebox{ \lambda=\pm e^{\kappa\alpha} }
As $\alpha$ ranges from $-\infty$ to $\infty$ we cover the $\lambda>0$ or 
the $\lambda<0$ portion of the generator of $\mathcal{N}$ (geodesic in $\mcN$
with normal $l$).  The bifurcation point\index{bifurcation!point} $\lambda=0$ is
a fixed point of $\xi$, which can be shown to be a 2-sphere, called the
bifurcation 2-sphere\index{bifurcation!2-sphere}, (BK-axis for Kruskal).
\begin{center}\input{p31-1.pictex}\end{center}
This is called a \emphin{bifurcate Killing 
horizon}\index{Killing!horizon!bifurcate}.

\paragraph{Proposition} If $\mathcal{N}$ is a bifurcate Killing 
horizon of $\xi$, with bifurcation 2-sphere, $B$, then $\kappa^2$ is 
constant on $\mathcal{N}$.

\paragraph{Proof}  $\kappa^2$ is constant on each orbit of $\xi$.  The value 
of this constant is the value of $\kappa^2$ at the limit point of the orbit on
$B$, so $\kappa^2$ is constant on $\mcN$ if it is constant on $B$.  But we saw
previously that
\bea
t\cdot\partial \kappa^2 & = & -\left.\left(D^{\mu}\xi^{\nu}\right)
t^{\rho}R_{\nu\mu\rho}^{\I\I\I\I\sigma}\xi_{\sigma}\right|_{\mathcal{N}}  \\
 & = & 0 \quad  \mbox{on $B$ since $\left.\xi_{\sigma}\right|_B=0$ }
\eea
Since $t$ can be any tangent to $B$, $\kappa^2$ is constant on $B$, and 
hence on $\mcN$.

\paragraph{Example} $\mathcal{N}$ is $\left\{U=0\right\}
\cup\left\{V=0\right\}$ of Kruskal spacetime, and $\xi=k$, the 
time-translation Killing vector field.

On $\mcN$,
\be
k=\left\{ \begin{array}{ccc} \bgfrac{1}{4M}V\pd{}{V} & \mbox{on} 
& \{U=0\} \\ \\
	-\bgfrac{1}{4M}U\pd{}{U} & \mbox{on} & \{V=0\} \end{array}\right\}=fl
\ee
where
\be
f=\left\{ \begin{array}{ccc} \bgfrac{1}{4M}V & \mbox{on} & \{U=0\} \\ \\
	-\bgfrac{1}{4M}U & \mbox{on} & \{V=0\} \end{array}\right\}, \quad  
l=\left\{ \begin{array}{ccc} \bgpd{}{V} & \mbox{on} & \{U=0\} \\ \\ \bgpd{}{U} &
\mbox{on} & \{V=0\} \end{array} \right\} 
\ee
Since $l$ is normal to $\mathcal{N}$, \emph{$\mathcal{N}$ is a Killing 
horizon of $k$}.  Since $l\cdot Dl=0$, the surface gravity is 
\bea
\kappa=k\cdot\partial\ln\left|f\right| & = & \left\{ \begin{array}{ccc} 
\bgfrac{1}{4M}V\pd{}{V}\ln\left|V\right| & \mbox{on} & U=0 \\ \\
-\bgfrac{1}{4M}U\pd{}{U}\ln\left|U\right| & \mbox{on} & V=0 \end{array} 
\right. \\
 & = & \left\{ \begin{array}{ccc}
\bgfrac{1}{4M} & \mbox{on} & \{U=0\} \\ \\ -\bgfrac{1}{4M} & 
\mbox{on} & \{V=0\} \end{array} \right.
\eea
So $\kappa^2=1/(4M)^2$ is indeed a constant on $\mathcal{N}$.  Note that 
orbits of $k$ lie either entirely in $\{U=0\}$ or in $\{V=0\}$ or are fixed
points on $B$, which allows a difference of sign in $\kappa$ on the two branches
of $\mcN$.

[N.B. Reinstating factors of $c$ and $G$, $\left|\kappa\right|=
\bgfrac{c^3}{4GM}$]

\subsubsection{Normalization of $\kappa$}

If $\mcN$ is a Killing horizon of $\xi$ with surface gravity $\kappa$, 
then it is also a Killing horizon of $c\xi$ with surface gravity $c^2\kappa$
[from formula (\ref{eq:normal_dagger}) for $\kappa$] for any constant $c$.  
Thus surface gravity is not a property of $\mcN$ \emph{alone}, it also 
depends on the normalization of $\xi$. 

There is no natural normalization of $\xi$ on $\mcN$ since $\xi^2=0$ there, 
but in an asymptotically flat spacetime there is a natural normalization at
spatial infinity, e.g. for the time-translation Killing vector field $k$ we
choose
\be
k^2\to -1 \quad \mbox{as} \quad r\to\infty
\ee
This fixes $k$, and hence $\kappa$, up to a sign, and the sign of $\kappa$ 
is fixed by requiring $k$ to be future-directed.

\subsubsection{Degenerate Killing Horizon ($\kappa=0$)}

In this case, the group parameter on the horizon is also an affine parameter, 
so there is no bifurcation 2-sphere.  More on this case later.

\subsection{Rindler spacetime}\index{Rindler!spacetime}

Return to Schwarzschild solution
\be
ds^2=-\left(\Schr\right)dt^2+\left(\Schr\right)^{-1}dr^2+r^2d\Omega^2 
\label{eq:schw_dagger}
\ee
and let
\be
r-2M=\frac{x^2}{8M}
\ee
Then
\bea
\Schr & = & \frac{(\kappa x)^2}{1+(\kappa x)^2} \qquad 
\left(\kappa=\frac{1}{4M}\right) \\
 & \approx & (\kappa x)^2 \qquad \mbox{near $x=0$} \\
dr^2 & = & (\kappa x)^2 dx^2
\eea
so for $r\approx 2M$ we have
\be
ds^2 \approx \underbrace{ -(\kappa x)^2dt^2+dx^2 }_{\begin{array}{c} 
\subtext{2-dim Rindler} \\ \subtext{spacetime} \end{array} } + \underbrace{
\frac{1}{4\kappa^2}d\Omega^2 }_{\begin{array}{c} \subtext{2-sphere of} \\
\subtext{radius $1/(2\kappa)$} \end{array} }
\ee
so we can expect to learn something about the spacetime near the Killing 
horizon at $r=2M$ by studying the 2-dimensional \emph{Rindler spacetime}
\be
ds^2 = -(\kappa x)^2dt^2+dx^2 \qquad (x >0) 
\ee
This metric is singular at $x=0$, but this is just a coordinate singularity.  
To see this, introduce the Kruskal-type coordinates
\be
U'=-xe^{-\kappa t},\quad V'=xe^{\kappa t} 
\ee
in terms of which the Rindler metric\index{Rindler!metric} becomes
\be
ds^2=-dU'\,dV'
\ee
Now set
\be
U'=T-X, \quad V'=T+X
\ee
to get
\be
ds^2=-dT^2 +dX^2
\ee
i.e. the \emph{Rindler spacetime is just 2-dim Minkowski in unusual 
coordinates}.  Moreover, the Rindler coordinates with $x>0$ cover only the
$U'<0,\;V'>0$ region of 2d Minkowski
\begin{center}\input{p35-1.pictex}\end{center}
From what we know about the surface $r=2M$ of Schwarzschild it follows that 
the lines $U'=0,\;V'=0$, i.e. $x=0$ of Rindler is a Killing horizon of
$k=\fltpd{}{t}$ with surface gravity $\pm\kappa$.

\paragraph{Exercise}
\newcounter{Rindler_ex}
\begin{list}{(\roman{Rindler_ex})}
{\usecounter{Rindler_ex}}
\item Show that $U'=0$ and $V'=0$ are \emph{null curves}.

\item Show that 
\be
k = \kappa\left(V'\pd{}{V'}-U'\pd{}{U'}\right)
\ee
and that $\left.k\right|_{U'=0}$ is normal to $U'=0$. (So $\{U'=0\}$ is a 
Killing horizon).

\item
\be
\left.(k\cdot Dk)^{\mu}\right|_{U'=0} =\left.\kappa k^{\mu}\right|_{U'=0}
\ee

\end{list}

Note that $k^2=-(\kappa x)^2\to -\infty$ as $x\to\infty$, so \emph{there is 
no natural normalization of $k$ for Rindler}.

i.e. In contrast to Schwarzschild only the fact that $\kappa\neq 0$ is a 
property of the Killing horizon itself - the actual value of $\kappa$ depends on
an arbitrary normalization of $k$ --- so what is the meaning of the value of
$\kappa$?

\subsubsection{Acceleration Horizons}

\paragraph{Proposition}  The proper acceleration of a particle at $x=a^{-1}$ in 
Rindler spacetime (i.e. on an orbit of $k$) is constant and equal to $a$. 

\paragraph{Proof}  A particle on a timelike orbit $X^{\mu}(\tau)$ of a Killing 
vector field $\xi$ has 4-velocity
\be
u^{\mu}=\frac{\xi^{\mu}}{\left(-\xi^2\right)^{1/2}} \qquad 
\mbox{(since $u\propto \xi$ and $u\cdot u=-1$)}
\ee
Its proper 4-acceleration is
\bea
a^{\mu} & = & D_{(\tau)}u^{\mu}=u\cdot Du^{\mu} \\
 & = & \frac{\xi\cdot D\xi^{\mu}}{-\xi^2}+
\frac{\left(\xi\cdot\partial\xi^2\right)\xi^{\mu}}{2\xi^2}
\eea
But $\xi\cdot\partial\xi^2=2\xi^{\mu}\xi^{\nu}D_{\mu}\xi_{\nu}=0$ 
for Killing vector field, so
\be
a^{\mu}=\frac{\xi\cdot D\xi^{\mu}}{-\xi^2}
\ee
and `proper acceleration' is magnitude $|a|$ of $a^{\mu}$.  \\

For Rindler with $\xi=k$ we have (\bold{Exercise})
\be
a^{\mu}\partial_{\mu}=\frac{1}{U'}\pd{}{V'}+\frac{1}{V'}\pd{}{U'}
\ee
so
\bea
|a| & \equiv & \left(a^{\mu}a^{\nu}g_{\mu\nu}\right)^{1/2}=
\left(-\frac{1}{U'V'}\right)^{1/2} \\
 & = & \frac{1}{x}
\eea
so for $x=a^{-1}$ (constant) we have $|a|=a$, i.e. \emph{orbits of $k$ 
in Rindler are worldlines of constant proper acceleration}.  The acceleration
increases \emph{without bound} as $x\to 0$, so the Killing horizon at $x=0$ is
called an \emphin{acceleration horizon}.
\begin{center}\input{p38-1.pictex}\end{center}
Although the \emph{proper acceleration} of an $x=$ constant worldline 
diverges as $x\to 0$ its acceleration as measured by another $x=$ constant
observer will remain finite.  Since
\be
d\tau^2=(\kappa x)^2dt^2 \qquad \mbox{(for $x=a^{-1}$, constant)}
\ee
the acceleration as measured by \emph{an observer whose proper time is $t$} is
\be
\left(\frac{d\tau}{dt}\right)\times \frac{1}{x} = (\kappa x)\times 
\frac{1}{x}=\kappa
\ee
which has a \emph{finite} limit, $\kappa$, as $x\to 0$. \\

In Rindler spacetime such an observer is one with constant proper 
acceleration $\kappa$, but these observers are \emph{in no way `special`}
because the normalization of $t$ was arbitrary.
\be
t\to \lambda t \quad \Rightarrow \quad \kappa \to \lambda^{-1}\kappa , 
\quad (\lambda\in \R)
\ee
For Schwarzschild, however,
\be
d\tau^2=dt^2 \quad \Rightarrow \quad \left\{\begin{array}{ll} 
r=\mbox{constant} \to \infty \\ \theta,\phi\;\mbox{constant} \end{array}\right.
\ee
i.e. an observer whose proper time is $t$ is one at spatial $\infty$.  Thus

\fbox{\parbox{6in}{\emphin{surface gravity} is the acceleration of a static 
particle near the horizon as measured at spatial infinity}} \\

This explains the term `surface gravity' for $\kappa$.

\subsection{Surface Gravity and Hawking Temperature}

We can study the behaviour of QFT in a black hole spacetime using 
\emph{Euclidean path integrals}.  In Minkowski spacetime this involves setting
\be
t=i\tau
\ee
and continuing $\tau$ from imaginary to real values.  Thus $\tau$ 
is `imaginary time'\index{imaginary time} here (\ul{not} proper time on some
worldline). \\

In the black hole spacetime this leads to a continuation of the 
Schwarzschild metric to the \emph{Euclidean Schwarzschild metric}.
\be
ds^2_{\subtext{E}}=\left(\Schr\right)d\tau^2+\frac{dr^2}
{\left(\Schr\right)}+r^2d\Omega^2
\ee
This is singular at $r=2M$.  To examine the region \emph{near $r=2M$} we set
\be
r-2M=\frac{x^2}{8M}
\ee
to get
\be
ds^2_{\subtext{E}} \approx 
\underbrace{ (\kappa x)^2d\tau^2+dx^2 }_{\subtext{Euclidean Rindler}}
+\frac{1}{4\kappa^2}d\Omega^2
\ee
Not surprisingly, the metric near $r=2M$ is the product of the metric on 
$\mbox{S}^2$ and the Euclidean Rindler
spacetime\index{Rindler!spacetime!Euclidean}
\be
ds^2_{\subtext{E}} = dx^2+x^2d(\kappa\tau)^2
\ee
This is just $\bb{E}^2$ in plane polar coordinates if we make the 
\emph{periodic identification}
\be
\tau \sim \tau +\frac{2\pi}{\kappa}
\ee
i.e. the singularity of Euclidean Schwarzschild at $r=2M$ (and of 
Euclidean Rindler at $x=0$) is just a coordinate singularity provided that
imaginary time coordinate $\tau$ is periodic with period $2\pi/\kappa$.
This means that the Euclidean functional integral must be taken over fields 
$\Phi(\vec{x},\tau)$ that are periodic in $\tau$ with period $2\pi/\kappa$ 
[Why this is so is not self-evident, which is presumably why the Hawking
temperature was not first found this way. Closer analysis shows that the
non-singularity of the Euclidean metric is required for equilibrium].

Now, the Euclidean functional integral is
\be
Z =\int \left[\mcD\Phi\right]e^{-S_{\subtext{E}}\left[\Phi\right]}
\ee
where 
\be
S_{\subtext{E}} = \int dt\left(-i p\dot{q}+H\right)
\ee
is the Euclidean action. If the functional integral is taken over fields $\Phi$
that are periodic in imaginary time with period $\hbar \beta$ then it can be
written as (see QFT course)
\be
Z=\tr\, e^{-\beta H}\, ,
\ee
which is the partition function for a quantum mechanical system with 
Hamiltonian $H$ at temperature $T$ given by $\beta=(k_B T)^{-1}$ where $k_B$ is
Boltzman's constant. \\

But we just saw that $\hbar\beta=2\pi/\kappa$ for Schwarzschild, so we deduce
that a QFT  can be in equilibrium with a black hole only at the \emph{Hawking
temperature}\index{Hawking!temperature}
\be
T_H   =   \frac{\kappa}{2\pi}\frac{\hbar}{k_B} \\
\ee
i.e. in units for which $\hbar=1$, $k_B=1$
\bebox{
T_H=\frac{\kappa}{2\pi}
}
\ul{N.B.}
\newcounter{hawkingtempNB}
\begin{list}{(\roman{hawkingtempNB})}
{\usecounter{hawkingtempNB}}
\item At any other temperature, Euclidean Schwarzschild has a conical 
singularity\index{singularity!conical} $\rightarrow$ no equilibrium.

\item Equilibrium at Hawking temperature is unstable since if the black 
hole absorbs radiation its mass increases and its temperature \emph{decreases},
i.e. the black hole has \emph{negative} specific heat.

\end{list}

\subsection{Tolman Law - Unruh Temperature}\index{Tolman law}
\index{Unruh!temperature}

\paragraph{Tolman Law}  The local temperature $T$ of a static 
self-gravitating system in thermal equilibrium satisfies 
\be
\left(-k^2\right)^{1/2}T=T_0
\ee
where $T_0$ is constant and $k$ is the timelike Killing vector 
field $\fltpd{}{t}$.  If $\left(k^2\right)\to -1$ asymptotically we can identify
$T_0$ as the temperature `as seen from infinity'.  For a Schwarzschild black
hole we have
\be
T_0=T_H=\frac{\kappa}{2\pi}
\ee
Near $r=2M$ we have, in Rindler coordinates,
\be
(\kappa x)T=\frac{\kappa}{2\pi}
\ee
so
\be
T=\frac{x^{-1}}{2\pi}
\ee
is the temperature measured by a static observer (on orbit of $k$) near the 
horizon.  But $x=a^{-1}$, constant, for such an observer, where $a$ is proper
acceleration.  So
\be
T=\frac{a}{2\pi}
\ee
is the local (Unruh) temperature.  It is a general feature of quantum mechanics
(Unruh effect\index{Unruh!effect}) that an observer accelerating in Minkowski
spacetime appears to be in a heat bath at the Unruh temperature. \\

In Rindler spacetime the Tolman law states that
\be
(\kappa x)T=T_0
\ee
Since $T=x^{-1}/(2\pi)$ for $x=$ constant, we deduce that 
$T_0=\kappa/(2\pi)$, as in Schwarzschild, but this is now just the temperature
of the observer with constant acceleration $\kappa$, who is of no particular
significance.  Note that in Rindler spacetime
\be
T=\frac{x^{-1}}{2\pi} \to 0 \quad \mbox{as} \; x\to \infty
\ee
so the Hawking temperature (i.e. temperature as measured at spatial 
$\infty$) is actually zero. 

This is expected because Rindler is just Minkowski in unusual coordinates, there is nothing inside which could radiate.  But for a black hole
\be
T_{\subtext{local}}\to T_H \quad \mbox{at infinity}
\ee
$\Rightarrow$ the black hole must be radiating at this temperature. We shall
confirm this later.

\section{Carter-Penrose Diagrams}

\subsection{Conformal Compactification}

A black hole is a ``region of spacetime from which no signal can escape to 
infinity'' (Penrose).  This is unsatisfactory because `infinity' is not part of
the spacetime.  However the `definition' concerns the \emph{causal structure} of
spacetime which is unchanged by \emphin{conformal compactification}
\be
ds^2\to d\tilde{s}^2 = \Lambda^2(\vec{r},t)ds^2, \quad \Lambda\neq 0
\ee
We can choose $\Lambda$ in such a way that all points at $\infty$ in the 
original metric are at \emph{finite} affine parameter in the new metric.  For
this to happen we must choose $\Lambda$ s.t.
\be
\Lambda(\vec{r},t) \to 0 \quad \mbox{as } |\vec{r}|\to \infty \quad 
\mbox{and/or } |t|\to\infty
\ee
In this case `infinity' can be identified as those points $(\vec{r},t)$ for 
which $\Lambda(\vec{r},t)=0$.  These points are \ul{not} part of the original
spacetime but they can be added to it to yield a \emph{conformal
compactification} of the spacetime.

\subsubsection{Example 1}  Minkowski space
\be
ds^2=-dt^2+dr^2+r^2d\Omega^2
\ee
Let
\be
\left\{ \begin{array}{rcl} u & = & t-r \\ v & = & t+r \end{array}\right\} 
\to ds^2=-du\,dv +\frac{(u-v)^2}{4}d\Omega^2
\ee
Now set
\be
\left\{ \begin{array}{rclcc} u & = & \tan\tilde{U} & \quad & 
-\pi/2<\tilde{U}<\pi/2 \\ v & = & \tan\tilde{V} & \quad & -
\pi/2<\tilde{V}<\pi/2
\end{array} \right\} \begin{array}{l} \mbox{with $\tilde{V} \ge \tilde{U}$} \\
\mbox{since $r\ge0$} \end{array}
\ee
In these coordinates,
\be
ds^2=\left(2\cos\tilde{U}\cos\tilde{V}\right)^{-2}
\left[-4d\tilde{U}\,d\tilde{V}+\sin^2\left(\tilde{V}-
\tilde{U}\right)d\Omega^2\right]
\ee
To approach $\infty$ in this metric we must take 
$\left|\tilde{U}\right|\to \pi/2$ or $\left|\tilde{V}\right|\to \pi/2$, so by
choosing
\be
\Lambda = 2\cos\tilde{U}\cos \tilde{V}
\ee
we bring these points to finite affine parameter in the new metric
\be
d\tilde{s}^2=\Lambda ds^2=-4d\tilde{U}d\tilde{V}+
\sin^2\left(\tilde{V}-\tilde{U}\right)d\Omega^2 
\ee
We can now add the `points at infinity'.  Taking the restriction 
$\tilde{V}\ge \tilde{U}$ into account, these are
\bdm
\begin{array}{cccccl}
\left.\begin{array}{rcl}  \tilde{U} & = & -\pi/2 \\ \tilde{V} & = & \pi/2
\end{array} \right\} &
\Leftrightarrow & 
\left\{\begin{array}{rcl} u & \to & -\infty \\ v & \to & \infty 
\end{array}\right\} &
\Leftrightarrow & 
\left\{ \begin{array}{c} r\to\infty \\ \mbox{$t$ finite} 
\end{array} \right\} 
\mbox{\emph{spatial} $\infty$, $i_0$} \\ \\
\left.\begin{array}{rcl}  \tilde{U} & = & \pm\pi/2 \\ 
\tilde{V} & = & \pm\pi/2 \end{array} \right\} &
\Leftrightarrow & 
\left\{\begin{array}{rcl}
u & \to & \pm\infty \\ v & \to & \pm\infty \end{array}\right\} &
\Leftrightarrow & 
\left\{ \begin{array}{c} t\to\pm\infty \\ \mbox{$r$ finite} 
\end{array} \right\}
\begin{array}{l} \mbox{past and future} \\ 
\mbox{\emph{temporal} $\infty$, $i_{\pm}$} 
\end{array} \\ \\
\left.\begin{array}{rcl} \tilde{U} & = & -\pi/2 \\ 
|\tilde{V}| & \neq & \pi/2 \end{array} \right\} &
\Leftrightarrow & 
\left\{\begin{array}{c} u\to -\infty \\ 
\mbox{$v$ finite} \end{array}\right\} &
\Leftrightarrow & 
\left\{ \begin{array}{c} r\to\infty \\ t\to -\infty \\ 
\mbox{$r+t$ finite} \end{array} \right\}   
\begin{array}{l} \mbox{past null $\infty$} \\ \scri^- \end{array} \\ \\
\left.\begin{array}{rcl} |\tilde{U}| & \neq & \pi/2 \\ 
\tilde{V} & = & \pi/2 \end{array} \right\} &
\Leftrightarrow & 
\left\{\begin{array}{c} \mbox{$u$ finite} \\ 
v\to\infty \end{array}\right\} & 
\Leftrightarrow & 
\left\{ \begin{array}{c} r\to\infty \\ t\to\infty \\ 
\mbox{$r-t$ finite} \end{array} \right\} 
\begin{array}{l} \mbox{future null $\infty$} \\ \scri^+ \end{array}
\end{array}
\edm
Minkowski spacetime is conformally embedded in the new spacetime with 
metric $d\tilde{s}^2$ with boundary at $\Lambda=0$. \\

%\be
%d\tilde{s}^2=-4d\tilde{U}\,d\tilde{V}+
%\sin^2\left(\tilde{V}-\tilde{U}\right)d\Omega^2
%\ee

Introducing the new time and space coordinates $\tau,\chi$ by 
\be
\tau = \tilde{V}+\tilde{U}, \quad \chi=\tilde{V}-\tilde{U}
\ee
we have
\bebox{
\begin{array}{rcl}
d\tilde{s}^2 & = & \Lambda ds^2 = -d\tau^2+d\chi^2+\sin^2\chi d\Omega^2  \\ \\
\Lambda & = & \cos \tau+\cos\chi \end{array}
}
$\chi$ is an angular variable which must be identified modulo $2\pi$, 
$\chi\sim \chi+2\pi$.  If no other restriction is placed on the ranges of $\tau$
and $\chi$, then this metric $d\tilde{s}^2$ is that of the \emphin{Einstein
Static Universe}, of topology $\R$ (time) $\times$ $\mbox{S}^3$ (space). \\

The 2-spheres of constant $\chi\neq 0,\pi$ have radius 
$\left|\sin\chi\right|$ (the points $\chi=0,\pi$ are the poles of a 3-sphere). 
If we represent each 2-sphere of constant $\chi$ as a point the E.S.U. can be
drawn as a cylinder.
\begin{center}\input{p46-1.pictex}\end{center}
But compactified Minkowski spacetime is conformal to the triangular region
\be
-\pi \le \tau \le \pi, \quad 0 \le \chi \le \pi
\ee
\begin{center}\input{p47-1.pictex}\end{center}
Flatten the cylinder to get the \emphin{Carter-Penrose diagram} of 
Minkowski spacetime.
\begin{center}\input{p47-2.pictex}\end{center}
Each point represents a 2-sphere, except points on $r=0$ and $i_0,i_{\pm}$.  
Light rays travel at $45^{^0}$ from $\scri^-$ through $r=0$ and then out to
$\scri^+$.  [$\scri^{\pm}$ are null hypersurfaces]. \\

\emph{Spatial sections} of the compactified spacetime are topologically 
$\mbox{S}^3$ because of the addition of the point $i_0$.  Thus, they are not
only compact, but also have no boundary.  This is not true of the whole
spacetime.  Asymptotically it is possible to identify points on the boundary of
compactified spacetime to obtain a compact manifold without boundary (the group
U(2); see Question I.6).  More generally, this is not possible because $i_{\pm}$
are singular points that cannot be added (see \bold{Example 3: Kruskal}).  
%This is illustrated by ... Kruskal 

\subsubsection{Example 2: Rindler Spacetime}

\be
ds^2=-dU'\,dV'
\ee
Let
\be
\left.\begin{array}{rcl} U' & = & \tan \tilde{U} \\ V' & = & \tan \tilde{V} 
\end{array} \right\} \begin{array}{c} -\pi/2<\tilde{U} <\pi/2 \\
-\pi/2<\tilde{V}<\pi/2 \end{array} 
\ee
Then
\bea
ds^2 & = & -\left(\cos\tilde{U}\cos\tilde{V}
\right)^{-2}d\tilde{U}\,d\tilde{V} \\
 & = & \Lambda^{-2}d\tilde{s}^2 \qquad 
\left(\Lambda=\cos\tilde{U}\cos\tilde{V}\right)
\eea
i.e. conformally compactified spacetime with metric 
$d\tilde{s}^2=-d\tilde{U}\,d\tilde{V}$ is same as before but with the above
\emph{finite} ranges for coordinates $\tilde{U},\tilde{V}$.

The points at infinity are those for which $\Lambda=0$, 
$\left|\tilde{U}\right|=\pi/2$, $\left|\tilde{V}\right|=\pi/2$. 
\begin{center}\input{p50-1.pictex}\end{center}
Similar to 4-dim Minkowski, but $i_0$ is now two points.

\subsubsection{Example 3:  Kruskal Spacetime}

\be
ds^2=-\left(\Schr\right)du\,dv +r^2d\Omega^2 \qquad \mbox{in region I}
\ee
Let
\be
\left\{\begin{array}{rcl} u=\tan \tilde{U} \\ 
v=\tan \tilde{V} \end{array} \quad   
\begin{array}{c} -\pi/2<\tilde{U} <\pi/2 \\
-\pi/2<\tilde{V}<\pi/2 \end{array}  \right\}
\ee
Then
\be
ds^2=\left(2\cos\tilde{U}\cos\tilde{V}\right)^{-2}
\left[-4\left(\Schr\right)d\tilde{U}d\tilde{V}+
r^2\cos^2\tilde{U}\cos^2\tilde{V}d\Omega^2\right]
\ee
Using the fact that
\be
r^* = \half (v-u)=\frac{\sin\left(\tilde{V}-
\tilde{U}\right)}{2\cos\tilde{U}\cos\tilde{V}} 
\ee
we have
\be
d\tilde{s}^2 = \Lambda^2 ds^2=-4\left(\Schr\right)
d\tilde{U}d\tilde{V}+\left(\frac{r}{r^*}\right)^2
\sin^2\left(\tilde{V}-\tilde{U}\right)d\Omega^2
\ee
Kruskal is an example of an \emph{asymptotically flat spacetime}.  
It approaches the metric of compactified Minkowski spacetime as $r\to\infty$
(with or without fixing $t$) so $i_0$, and $\scri^{\pm}$ can be added as
before.  Near $r=2M$ we can introduce KS-type coordinates to pass through the
horizon.  In this way one can deduce that the CP diagram for the Kruskal
spacetime is 
\begin{center}\input{p49-1.pictex}\end{center}
\paragraph{Note}
\newcounter{Kruskal_note}
\begin{list}{(\roman{Kruskal_note})}
{\usecounter{Kruskal_note}}
\item All $r=$ constant hypersurfaces meet at $i_+$ including 
the $r=0$ hypersurface, which is singular, so $i_+$ is a singular point. 
Similarly for $i_-$, so these points cannot be added.

\item We can adjust $\Lambda$ so that $r=0$ is represented by a straight line.

\end{list}
 
In the case of a collapsing star, only that part of the CP diagram of 
Kruskal that is exterior to the star is relevant.  The details of the interior
region depend on the physics of the star.  For pressure-free, spherical 
collapse, all parts of the star not initially at $r=0$ reach the singularity at
$r=0$ \emph{simultaneously}, so the CP diagram is
\begin{center}\input{p49-2.pictex}\end{center}

\section{Asymptopia}

A spacetime $(M,g)$ is \emph{asymptotically simple}
\index{asymptotically!simple} if $\exists$ a manifold
$(\widetilde{M},\tilde{g})$ with boundary $\partial\widetilde{M}=\overline{M}$
and a continuous embedding $f(M):M\to\widetilde{M}$ s.t.
\newcounter{asymsimple}
\begin{list}{(\roman{asymsimple})}
{\usecounter{asymsimple}}
\item $f(M)=\widetilde{M}-\partial \widetilde{M}$ 

\item $\exists$ a smooth function $\Lambda$ on $\widetilde{M}$ with 
$\Lambda >0$ on $f(M)$ and $\tilde{g}=\Lambda^2f(g)$.

\item $\Lambda=0$ but $d\Lambda\neq 0$ on $\partial\widetilde{M}$.

\item Every null geodesic in $M$ acquires 2 endpoints on $\partial M$.


\paragraph{Example} $M=$ Minkowski, $\widetilde{M}=$ compactified 
Minkowski.  \\

Condition (iv) excludes black hole spacetime.  This motivates the 
following definition: \\

A \emph{weakly asymptotically simple}\index{asymptotically!simple!weakly} 
spacetime $(M,g)$ is one for which $\exists$ an open set $U\subset M$ that is
isometric to an open neighborhood of $\partial\widetilde{M}$, where
$\widetilde{M}$ is the `conformal compactification' of some asymptotically
simple manifold. 

\paragraph{Example}  $M=$ Kruskal, $\widetilde{M}$ its conformal 
`compactification'. 

\paragraph{Note}
\newcounter{was}
\begin{list}{(\roman{was})}
{\usecounter{was}}
\item $\widetilde{M}$ is not actually compact because 
$\partial\widetilde{M}$ excludes $i_{\pm}$.  

\item $M$ is not asymptotically simple because geodesics that enter 
$r<2M$ cannot end on $\scri^+$.
\end{list}

\subsubsection{Asymptotic flatness}

An \emph{asymptotically flat} spacetime is one that is both weakly asymptotically simple and is \emph{asymptotically empty}\index{asymptotically!empty} in the sense that

\item $R_{\mu\nu}=0$ in an open neighborhood of $\partial M$ in $\overline{M}$.

\end{list}

This excludes, for example, anti-de Sitter space.  It also excludes 
spacetimes with long range electromagnetic fields that we don't wish to exclude
so condition (v) requires modification to deal with electromagnetic fields. \\

Asymptotically flat spacetimes have the same type of structure for 
$\scri^{\pm}$ and $i_0$ as Minkowski spacetime.
\begin{center}\input{p52-1.pictex}\end{center}
In particular they admit vectors that are asymptotic to the Killing 
vectors of Minkowski spacetime near $i_0$, which \emph{allows a definition of
total mass, momentum and angular momentum on spacelike hypersurfaces}.  The
asymptotic symmetries on $\scri^{\pm}$ are much more complicated (the `BMS'
group, which will not be discussed in this course).

\section{The Event Horizon}

Assume spacetime $M$ is weakly asymptotically flat.  Define
\bdm
J^-(U)
\edm
to be the \emph{causal past} of a set of points $U\subset M$ and 
\bdm
\overline{J}^-(U)
\edm
to be the topological closure of $J^-$, i.e. including limit points.  Define 
the \emph{boundary} of $\overline{J}^{-}$ to be
\be
\dot{J}^-(U)= \overline{J}^-(U)-J^-(U)
\ee
The \emphin{future event horizon} of $M$ is
\be
\mcH^+=\dot{J}^-\left(\scri^+\right)
\ee
i.e. the \emph{boundary of the closure of the causal past of $\scri^+$}.

\paragraph{Example}  Spacetime of a spherically-symmetric collapsing star
\begin{center}\input{p53-1.pictex}\end{center}

\subsubsection{Properties of the Future Event Horizon, $\mcH^+$}
\newcounter{propfeh}
\begin{list}{(\roman{propfeh})}
{\usecounter{propfeh}}
\item $i_0$ and $\scri^-$ are contained in $J^-\left(\scri^+\right)$, so 
they are \ul{not} part of $\mcH^+$.

\item $\mcH^+$ is a null hypersurface.

\item No two points of $\mcH^+$ are timelike separated.  For nearby points 
this follows from (ii) but is also true globally.  Suppose that $\alpha$ and
$\beta$ were two such points with $\alpha\in J^-(\beta)$.  The timelike curve
between them could then be deformed to a nearby timelike curve between $\alpha'$
and $\beta'$ with $\beta'\in J^-\left(\scri^+\right)$ but $\alpha'\not\in
J^-\left(\scri^+\right)$
\begin{center}\input{p54-1.pictex}\end{center}
But $\alpha'\in J^-(\beta)\in J^-\left(\scri^+\right)$, so we have a 
contradiction.  The timelike curve between $\alpha$ and $\beta$ cannot exist.

\item The null geodesic generators of $\mcH^+$ may have \emph{past endpoints} 
in the sense that the continuation of the geodesic further into the past is no
longer in $\mcH^+$, e.g. at $r=0$ for a spherically symmetric star, as shown in
diagram above.

\item If a generator of $\mcH^+$ had a future endpoint, the future continuation 
of the null geodesic beyond a certain point would leave $\mcH^+$.  This cannot
happen.

\end{list}

\fbox{\parbox{6in}{
\bold{Theorem} (Penrose) The generators of $\mcH^+$ have no future endpoints}}

\paragraph{Proof} Consider the causal past $J^-(S)$ of some set $S$.
\begin{center}\input{p55-0.pictex}\end{center}
Consider a point $p\in\dot{J}^-(S)$, $p\not\in S,\overline{S}$.
% and a compact region $U$ containing $p$ but not containing any past 
  Endpoints of the null geodesic in $\dot{J}^-(S)$ through $p$.  
Consider also an infinite sequence of timelike curves 
$\left\{\gamma_i\right\}$ from $p_i\in$ neighborhood of $p$ and 
$\in J^-(S)$ to
$S$
%$\scri^+$ to points $\left\{p_n\in U\right\}$ 
s.t. $p$ is the limit point of $\left\{p_i\right\}$ on $\dot{J}^-(S)$.
%$\mcH^+$.  Let $q_n$ be the point at which $\gamma_i$ intersects the 
%boundary of $U$.
\begin{center}\input{p55-1.pictex}\end{center}
The points $\left\{q_i\right\}$ must have a limit point $q$ on $\dot{J}^-(S)$.
%$\mcH^+$.  
Being the limit of timelike curves, the curve $\gamma$ from $p$ to $q$ 
cannot be spacelike, but can be null (lightlike).  It cannot be timelike either
from property (iii) above, so it is a segment of the null geodesic generator of
$\mcN$
%$\mcH^+$ 
through $p$.  The argument can now be repeated with $p$ replaced by $q$ to 
find another segment from $q$ to a point, 
%$r\in\mcH^+$
$r\in\mcN$, but further in the future.  It must be a segment of the 
\emph{same} generator because otherwise there exists a deformation to a timelike
curve in $\mcN$ separating $p$ and $r$.
\begin{center}\input{p55-2.pictex}\end{center}
Choosing $S=\scri^+$, then gives Penrose's Theorem. \\

Properties (iv) and (v) show that \emph{null geodesics may enter $\mcH^+$ 
but cannot leave it}. \\

This result may appear inconsistent with time-reversibility, but is not.  
The time-reverse statement is that null geodesics may leave but cannot enter the
\emph{past event horizon}, $\mcH^-$.  $\mcH^-$ is defined as for $\mcH^+$ with
$J^-\left(\scri^+\right)$ replaced by $J^+\left(\scri^-\right)$, i.e. the causal
future of $\scri^-$.  The time-symmetric Kruskal spacetime has both a future and
a past event horizon.
\begin{center}\input{p56-1.pictex}\end{center}
The location of the event horizon $\mcH^+$ generally requires knowledge of the 
\emph{complete} spacetime.  Its location cannot be determined by observations
over a finite time interval. \\

However if we wait until the black hole settles down to a stationary spacetime 
we can invoke: \\

\fbox{\parbox{6in}{
\bold{Theorem} (Hawking)  The event horizon of a stationary asymptotically 
flat spacetime is a Killing horizon (but \ul{not} \emph{necessarily} of
$\fltpd{}{t}$).}}

This theorem is the essential input needed in the proof of the uniqueness
theorems for stationary black holes, to be considered later.

\section{Black Holes vs. Naked Singularities}

The singularity at $r=0$ that occurs in spherically symmetric collapse is 
hidden in the sense that no signal from it can reach $\scri^+$.  This is not
true of the Kruskal spacetime manifold since a signal from $r=0$ in the white
hole region \emph{can} reach $\scri^+$.
\begin{center}\input{p57-1.pictex}\end{center}
This singularity is \emph{naked}\index{naked singularity}.  Another example 
of a naked singularity is the $M<0$ Schwarzschild solution
\be
ds^2=-\left(1+\frac{2|M|}{r}\right)dt^2+
\frac{1}{\left(1+\frac{2|M|}{r}\right)}dr^2+r^2d\Omega^2
\ee
This solves Einstein's equations so we have no a priori reason to exclude 
it.  The CP diagram is
\begin{center}\input{p57-2.pictex}\end{center}
Neither of these examples is relevant to gravitational collapse, but 
consider the CP diagram:
\begin{center}\input{p58-1.pictex}\end{center}
At late times the spacetime is $M<0$ Schwarzschild but at earlier times 
it is non-singular.  Under these circumstances it can be shown that $M\ge 0$ for
physically reasonable matter (the `positive energy' theorem) so the possibility
illustrated by the above CP diagram (formation of a naked singularity in
\emph{spherically-symmetric} collapse) cannot occur.  There remains the
possibility that naked singularities could form in non-spherical collapse.  If
this were to happen the future would eventually cease to be predictable from
data given on an initial spacelike hypersurface ($\Sigma$ in CP diagram above). 
There is considerable evidence that this possibility cannot be realized for
physically reasonable matter, which led Penrose to suggest the:

\paragraph{Cosmic Censorship Conjecture}  `Naked singularities cannot form 
from gravitational collapse in an asymptotically flat spacetime that is
non-singular on some initial spacelike hypersurface (Cauchy surface).' \\

\paragraph{Notes}
\newcounter{ccc}
\begin{list}{(\roman{ccc})}
{\usecounter{ccc}}
\item Certain types of `trivial' naked singularities must be excluded.

\item  Initial, cosmological, singularities are excluded.

\item There is no proof.  This is the major unsolved problem in classical G.R.
\end{list}
