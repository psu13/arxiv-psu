\chapter{Charged Black Holes}

\section{Reissner-Nordstr\"om}

Consider the Einstein-Maxwell action
\be
S=\frac{1}{16\pi G}\int \dx{4}{x}\sqrt{-g}\left[ R-F_{\mu\nu}F^{\mu\nu}\right], 
\qquad \left(R=R_{\mu\nu}^{\I\I\I\mu\nu}\right)
\ee
The unusual normalization of the Maxwell term means that the magnitude of the 
Coulomb force between point charges $Q_1,Q_2$ at separation $r$ (large) in flat
space is
\be
\frac{G\left|Q_1Q_2\right|}{r^2} \quad \mbox{(`geometrized' units of charge)}
\ee
The source-free Einstein-Maxwell equations are
\bea
G_{\mu\nu} & = & 2\left(F_{\mu\lambda}F_{\nu}^{\I\lambda}-
\frac{1}{4}g_{\mu\nu}F_{\rho\sigma}F^{\rho\sigma}\right) \\
D_{\mu}F^{\mu\nu} & = & 0 
\eea
They have the \emph{spherically-symmetric Reissner-Nordstr\"om}
\index{Reissner-Nordstr\"om solution} (RN) solution (which generalizes
Schwarzschild)
\bea
ds^2 & = & -\left(1-\frac{2M}{r}+\frac{Q^2}{r^2}\right)dt^2+\frac{dr^2}
{\left(1-\frac{2M}{r}+\frac{Q^2}{r^2}\right)}+r^2d\Omega^2  \\
A & = & \frac{Q}{r}dt \quad \mbox{(Maxwell 1-form potential $F=dA$)} 
\eea
The parameter $Q$ is clearly the \emph{electric charge}.

The RN metric can be written as 
\be
ds^2=-\frac{\Delta}{r^2}dt^2+\frac{r^2}{\Delta}dr^2+r^2d\Omega^2 
\ee 
where
\be
\Delta = r^2-2Mr+Q^2 = \left(r-r_+\right)\left(r-r_-\right)
\ee
where $r_{\pm}$ are not necessarily real
\be
r_{\pm} = M\pm \sqrt{M^2-Q^2}
\ee

There are therefore \ul{3 cases} to consider:
\newcounter{RNcases}
\begin{list}{\roman{RNcases})}
{\usecounter{RNcases}}

\item \ul{ $M<|Q|$} \\

$\Delta$ has no real roots so there is no horizon and the singularity at 
$r=0$ is naked. 

This case is similar to $M<0$ Schwarzschild.  According to the cosmic 
censorship hypothesis\index{cosmic censorship hypothesis} this case could not
occur in gravitational collapse.  As confirmation, consider a shell of matter of
charge $Q$ and radius $R$ in Newtonian gravity but incorporating

a) Equivalence of inertial mass $M$ with total energy, from special relativity.

b) Equivalence of inertial and gravitational mass from general relativity.

\be
\underbrace{M_{\subtext{total}}}_{\stackrel{\uparrow}{\mbox{total energy}}} = 
\underbrace{M_0}_{\stackrel{\uparrow}{\mbox{rest mass
energy}}}+\underbrace{\frac{GQ^2}{R}}_{\stackrel{\uparrow}{\mbox{Coulomb
energy}}}-\underbrace{\frac{GM^2}{R}}_{\stackrel{\uparrow}
{\stackrel{\mbox{grav.
binding energy}}{\mbox{($M$=total mass)}}}}
\ee
This is a quadratic equation for $M$.  The solution with $M\to M_0$ as 
$R\to\infty$ is
\be
M(R) = \frac{1}{2G}\left[\left(R^2+4GM_0R+4G^2Q^2\right)^{1/2}-R\right]
\ee
The shell will only undergo gravitational collapse iff $M$ decreases with 
decreasing $R$ (so allowing K.E. to increase).  Now
\be
M'=\frac{ G\left(M^2-Q^2\right)}{ 2MGR+R^2}
\ee
so collapse occurs only if $M>|Q|$ as expected. \\

Now consider $M(R)$ as $R\to 0$.
\be
M \longrightarrow |Q| \quad 
\mbox{\emph{independent of $M_0$}}
\ee
So GR resolves the infinite self-energy problem of point particles in 
classical EM.  A point particle becomes an extreme ($M=|Q|$) RN black 
hole (case (iii) below).

\paragraph{Remark}  The electron has $M\ll |Q|$ (at least when probed at 
distances $\gg GM/c^2$) because the gravitational attraction is negligible
compared to the Coulomb repulsion.  But the electron is \emph{intrinsically
quantum mechanical}, since its Compton wavelength $\gg$ Schwarzschild radius. 
Clearly the applicability of GR requires
\be
\frac{\mbox{Compton wavelength}}{\mbox{Schwarzschild radius}} = 
\frac{ \hbar/Mc}{MG/c^2} = \frac{\hbar c}{M^2 G} \ll 1 
\ee
i.e.
\be
M \gg \left(\frac{\hbar c}{G}\right)^{1/2} \equiv M_P \quad 
\mbox{(Planck mass)}
\ee
This is satisfied by any macroscopic object but not by elementary particles.

More generally the domains of applicability of classical physics QFT and 
GR are illustrated in the following diagram.
\begin{center}\input{p62-1.pictex}\end{center}

\item \ul{$M>|Q|$}  \\

$\Delta$ vanishes at $r=r_+$ and $r=r_-$ real, so metric is singular there, 
but these are coordinate singularities.  To see this we proceed as for $r=2M$ in
Schwarzschild.  Define $r^*$ by
\bea
dr^* & = &\frac{r^2}{\Delta}dr= \frac{dr}{\left(1-\frac{2M}{r}+
\frac{Q^2}{r^2}\right)} \\
\Rightarrow\; r^* & = & r+\frac{1}{2\kappa_+}\ln 
\left(\frac{ \left|r-r_+\right|}{r_+}\right) +
\frac{1}{2\kappa_-}\ln\left(\frac{ \left|r-r_-\right|}{r_-}\right)+\mbox{const}
\eea
where
\bebox{\kappa_{\pm}=\frac{ \left( r_{\pm}-r_{\mp} \right)}{ 2r_{\pm}^2} }
We then introduce the radial null coordinates $u,v$ as before
\be
v=t+r^*, \quad u = t-r^*
\ee
The RN metric in ingoing Eddington-Finkelstein coordinates $(v,r,\theta,\phi)$ 
is
\be
ds^2=-\frac{\Delta}{r^2}dv^2+2dv\,dr+r^2d\Omega^2
\ee
which is non-singular everywhere except at $r=0$.  Hence the $\Delta=0$ 
singularities of RN were coordinate singularities.  The hypersurfaces of
constant $r$ are null when $g^{rr}=\Delta/r^2=0$, i.e. when $\Delta=0$, so
$r=r_{\pm}$ are null hypersurfaces, $\mathcal{N}_{\pm}$. \\

\fbox{\parbox{6in}{
\paragraph{Proposition} The null hypersurfaces $\mcN_{\pm}$ of RN are 
Killing horizons of the Killing vector field $k=\partial/\partial v$ (the
extension of $\partial/\partial t$ in RN coordinates) with surface gravities
$\kappa_{\pm}$.}}

\paragraph{Proof}  The normals to $\mcN_{\pm}$ are 
\be
l_{\pm} = \left.f_{\pm}\left(g^{rr}\pd{}{r}+g^{vr}\pd{}{v}
\right)\right|_{\mcN_{\pm}} = f_{\pm}\pd{}{v}
\ee
(note $g^{rr}=0$ on $\mcN_{\pm}$ and $g^{vr}=1$) for some arbitrary 
functions $f_{\pm}$ which we can choose s.t. \fbox{$l_{\pm}Dl_{\pm}^{\mu}=0$}
(tangent to an affinely parameterized geodesic) so
\be
\pd{}{v} = f_{\pm}^{-1}l_{\pm}
\ee
which shows that $\mcN_{\pm}$ are Killing horizons of $\pd{}{v}$ 
(This is Killing because in EF coordinates the metric is $v$-independent). We
can interpret the LHS of this equation as a derivative w.r.t the group
parameter, and the RHS as a derivative w.r.t the affine parameter.  Now
\bea
(k\cdot Dk)^r & = & \Gamma^r_{\I vv}=-\half g^{rr}g_{vv,r} = 0 
\quad \mbox{on }\mcN_{\pm} \\
\left.(k\cdot Dk)^v\right|_{r=r_{\pm}} & = & \Gamma^v_{\I vv}= 
-\half g^{vr}g_{vv,r} = \left.\frac{1}{2r^2}\pd{}{r}\Delta\right 
|_{r=r_{\pm}} \\
 & = & \frac{1}{2r_{\pm}^2}\left(r_{\pm}-r_{\mp}\right) 
\quad \mbox{on }\mcN_{\pm} \\ 
& = & \kappa_{\pm}
\eea
\bebox{
\thdots \quad k\cdot Dk^{\mu}  =  \kappa_{\pm}k^{\mu}}
Since $k=\partial/\partial t$ in static coordinates we have $k^2\to -1$ as 
$r\to \infty$.  So we identify $\kappa_{\pm}$ as the surface gravities of
$\mcN_{\pm}$.
%\end{list}

Each of the Killing horizons $\mcN_{\pm}$ will have a bifurcation 2-sphere 
in the neighborhood of which we can introduce the KS-type coordinates
\be
U^{\pm}=-e^{-\kappa_{\pm}u},\quad V^{\pm}=e^{\kappa_{\pm}v}
\ee
For the $+$ sign we have
\be
ds^2=-\frac{r_+r_-}{\kappa_+^2}\frac{e^{-2\kappa_+r}}{r^2}
\left(\frac{r_-}{r-r_-}\right)^{\left(\frac{\kappa_+}
{\kappa_-}-1\right)}dU^+\,dV^++r^2d\Omega^2
\ee
where $r\left(U^+,V^+\right)$ is determined implicitly by
\be
U^+V^+=-e^{2\kappa_+r}\left(\frac{r-r_+}{r_+}\right)
\left(\frac{r-r_-}{r_-}\right)^{\kappa_+/\kappa_-}
\ee
This metric covers four regions of the maximal analytic extension of RN,
\begin{center}\input{p65-1.pictex}\end{center}
These coordinates do not cover $r\le r_-$ because of the coordinate 
singularity at $r=r_-$ (and $U^+V^+$ is complex for $r<r_-$), but $r=r_-$ and a
similar four regions are covered by the $\left(U^-V^-\right)$ KS-type
coordinates to this case (Exercise).
\bea
ds^2 & = & -\frac{r_+r_-}{\kappa_-^2}\frac{e^{-2\kappa_-r}}{r^2}
\left(\frac{r_+}{r_+-r}\right)^{\frac{\kappa_-}
{\kappa_+}-1}dU^-\,dV^-+r^2d\Omega^2
\\ U^-V^- & = &
-e^{-2\kappa_-r}\left(\frac{r_--r}{r_-}\right)
\left(\frac{r_+-r}{r_+}\right)^{\kappa_-/\kappa_+}
\eea
This metric covers four regions around $U^-=V^-=0$.
\begin{center}\input{p66-1.pictex}\end{center}
Region II is the same as the region II covered by the 
$\left(U^+,V^+\right)$ coordinates.  The other regions are new.  Regions V and
VI contain the curvature singularity at $r=0$, which is \emph{timelike} because
the normal to $r=$ constant is spacelike for $\Delta>0$, e.g. in $r<r_-$. \\

We know that region II of the diagram is connected to an exterior spacetime 
in the past (regions I, III, and IV),  by time-reversal invariance, region III'
must be connected to another exterior region (isometric regions I', II', and
IV').
\begin{center}\input{p66-2.pictex}\end{center}
Regions I' and IV' are new asymptotically flat `exterior' spacetimes.  
Continuing in this manner we can find an infinite sequence of them.

\subsubsection{Internal Infinities} 

Consider a path of constant $r,\theta,\phi$ in any region for which 
$\Delta <0$, e.g. region II. In ingoing EF coordinates
\bea
ds^2 & = & -\frac{\Delta}{r^2}dv^2 \\
 & = & \frac{|\Delta|}{r^2}dv^2 \quad \mbox{since are considering 
$\Delta<0$ by hypothesis}
\eea
Since $ds^2>0$ the path is spacelike.  The distance along it from $v=0$ 
to $v=-\infty$ (i.e. to $V^+=0$ or $V^-=0$) is
\bea
s & = & \int^0_{-\infty}\frac{|\Delta|^{1/2}}{r}dv = 
\frac{|\Delta|^{1/2}}{r}\int^0_{-\infty}dv \quad 
\mbox{since $r$ is constant} \\
 & = & \infty
\eea
So there is an `internal' spatial infinity behind the $r=r_+$ horizon.  
(Note that one can still reach $V^{\pm}=0$ in finite proper time on a
\emph{timelike} path, so the null hypersurfaces $V^{\pm}=0$ are part of the
spacetime). \\

If all points at $\infty$, external and internal, are brought to finite 
affine parameter by a conformal transformation, one finds the following CP
diagram, which can be infinitely extended in both directions:
\begin{center}\input{p67-1.pictex}\end{center}
%\end{list} wait for third case 

\section{Pressure-Free Collapse to RN}

Consider a spherical dust ball for which each particle of dust has 
charge/mass ratio 
\be
\gamma=\frac{Q}{M},\quad |\gamma|<1
\ee
where $Q$ is the total charge and $M$ is the total mass.  The exterior 
metric is $M>|Q|$ RN.  The trajectory of a particle at the surface is the same
as that of a radially infalling particle of charge/mass ratio $\gamma$ in the RN
spacetime.  This is \ul{not} a geodesic because of the additional electrostatic
repulsion.  From the result of Question II.4, we see that the trajectory of a
point on the surface obeys
\be
\left(\frac{dr}{d\tau}\right)^2 = \varepsilon^2-V_{\subtext{eff}},
\quad (\varepsilon <1)
\ee
where
\be
V_{\subtext{eff}} = 1-\left(1-\varepsilon \gamma^2\right)\frac{2M}{r}+
\left(1-\gamma^2\right)\frac{Q^2}{r^2}
\ee
\begin{center}\input{p68-1.pictex}\end{center}
\be
r_0=\frac{ \left(1-\gamma^2\right)}{\left(1-\varepsilon\gamma^2\right)}
\frac{Q^2}{M}=\frac{
\gamma^2\left(1-\gamma^2\right)}{\left(1-\varepsilon\gamma^2\right)}M
\ee
The collapse will therefore be halted by the electrostatic repulsion.  
All timelike curves that enter $r<r_+$ must continue to $r<r_-$, so the `bounce'
will occur in region V.  The dust ball then enters region III', explodes as a
white hole into region I' and then recollapses and re-expands indefinitely.

This is illustrated by the following CP diagram
\begin{center}\input{p69-1.pictex}\end{center}
\ul{Notes}
\newcounter{RNnotes}
\begin{list}{\roman{RNnotes})}
{\usecounter{RNnotes}}
\item No singularity is visible from $\scri^+$, in agreement with cosmic 
censorship.
\item Although the dust ball never collapses to zero size and its interior 
is completely non-singular, there is nevertheless a singularity behind
$\mathcal{H}^+$ on another branch of $r=0$, in agreement with the singularity
theorems.
\item It seems that a criminal could escape justice in universe I by escaping 
on a timelike path into universe I'.  Is this science fiction?
\end{list}

\section{Cauchy Horizons}

A particle on an ingoing radial geodesic of RN (e.g. surface of 
collapsing star) will `hit' the singularity at $r=0$, but once in region V or VI
it can accelerate away from the singularity then enter the new exterior region
via the white hole region III'.  However, there is no way to ensure in advance
of entering the black hole (e.g. by programming of rockets) that it will do so
because to get to region I' it must cross a \emph{Cauchy
horizon}\index{Cauchy!horizon}, a concept that will now be elaborated. 

\paragraph{Definition}  A \emph{partial Cauchy surface}
\index{Cauchy!surface!partial}, $\Sigma$, for a spacetime $M$ is a hypersurface
which no causal curve intersects more than once. 

\paragraph{Definition}  A causal curve is \emph{past-inextendable} if 
it has no past endpoint in $M$.

\paragraph{Definition}  The \emph{future domain of dependence}, 
$D^+(\Sigma)$ of $\Sigma$, is the set of points $p\in M$ for which every
past-inextendable causal curve through $p$ intersects $\Sigma$.
\begin{center}\input{p70-1.pictex}\end{center}
The significance of $D^+(\Sigma)$ is that the behavior of solutions of 
hyperbolic PDE's \emph{outside} $D^+(\Sigma)$ is not determined by initial data
on $\Sigma$. \\

The past domain of dependence, $D^-(\Sigma)$ of $\Sigma$, is defined 
similarly and $\Sigma$ is said to be a \emphin{Cauchy surface} for $M$ if
\be
D^+(\Sigma)\cup D^-(\Sigma) = M
\ee
If $M$ has a Cauchy surface it is said to be \emph{globally hyperbolic}.  
Examples of globally hyperbolic spacetimes are
\newcounter{GHSex}
\begin{list}{\arabic{GHSex})}
{\usecounter{GHSex}}
\item Spherical, pressure-free collapse (Schwarzschild)
\begin{center}\input{p71-1.pictex}\end{center}
$\Sigma_1$ and $\Sigma_2$ are both Cauchy surfaces. \\

\item Kruskal
\begin{center}\input{p71-2.pictex}\end{center}
$\Sigma_1$ and $\Sigma_2$ are both Cauchy surfaces. \\
\end{list}
If $M$ is not globally hyperbolic then $D^+(\Sigma)$ or $D^-(\Sigma)$ will 
have a boundary in $M$, called the \emph{future or past Cauchy horizon}.

% end of p71
\subsubsection{Examples}
\newcounter{CHex}
\begin{list}{(\roman{CHex})}
{\usecounter{CHex}}
\item Gravitationally-collapsed charged dust ball.
\begin{center}\input{p72-1.pictex}\end{center}
\item Maximal analytic extension of RN
\begin{center}\input{p72-2.pictex}\end{center}
\end{list}

In example (i) a strange feature of the future Cauchy horizon is that the 
entire infinite history of the external spacetime in region I is in its causal
past, i.e. visible, so signals from I must undergo an infinite blueshift as they
approach the Cauchy horizon.  For this reason, the Cauchy horizon usually
becomes singular when subjected to any perturbation, no matter how small.  For
any physically realistic collapse, the Cauchy horizon is a \emph{singular null
hypersurface} for which new physics beyond GR is needed.

\section{Isotropic Coordinates for RN}

Let 
\be
r=\rho+M+\frac{M^2-Q^2}{4\rho}
\ee
Then
\bea
ds^2 & = &-\frac{\Delta dt^2}{r^2(\rho)}+\frac{r^2(\rho)}{\rho^2}
\underbrace{ \left(d\rho^2+\rho^2d\Omega^2\right)}_{\subtext{flat space metric}}
\\
\Delta & = &  \left[ \rho-\frac{\left(M^2-Q^2\right)}{4\rho}\right]^2 
\eea
is RN metric in isotropic coordinates $(t,\rho,\theta,\phi)$.  As in $Q=0$ case, there are \emph{two} values $\rho$ for every value of $r>r_+$, but $\rho$ is complex for $r<r_+$.
\begin{center}\input{p73-1.pictex}\end{center}
This new metric covers \emph{two} isometric regions (I\&IV) exchanged by the 
geometry.
\be
\rho \to \frac{M^2-Q^2}{4\rho}
\ee
The fixed points set at $\rho=\sqrt{M^2-Q^2}/2$ (i.e. $r=r_+$) is a minimal 
2-sphere of an ER bridge as in the $Q=0$ case.
\begin{center}\input{p74-1.pictex}\end{center}
The distance to the horizon at $r=r_+$ along a curve of constant 
$t,\theta,\phi$ from $r=R$ is
\bea
s & = & \int^R_{r_+} \frac{dr}{\sqrt{ \left(1-\frac{r_+}{r}\right)
\left(1-\frac{r_-}{r}\right)}}  \\
 & \to & \infty \quad \mbox{as }r_+-r_-\to 0,\mbox{ i.e. as }M-|Q|\to 0
\eea
so the ER bridge separating regions I \& IV becomes \emph{infinitely long} 
in the limit as $|Q|\to M$.  In this limit, the spatial sections look like:
\begin{center}\input{p74-2.pictex}\end{center}

\item \ul{$M=|Q|$ `Extreme' RN ($r_{\pm}=M$)}
\be
ds^2 = -\left(1-\frac{M}{r}\right)^2dt^2+\frac{dr^2}
{\left(1-\frac{M}{r}\right)^2}+r^2d\Omega^2
\ee
This is singular at $r=M$ so define the Regge-Wheeler coordinate
\be
r^*=r+2M\ln\left|\frac{r-M}{M}\right|-\frac{M^2}{r-M} \quad 
\Rightarrow \quad dr^*=\frac{dr}{1-\frac{M}{r}}
\ee
and introduce ingoing EF coordinates as before.  Then
\be
ds^2=-\left(1-\frac{M}{r}\right)^2 dv^2+2dv\,dr+r^2d\Omega^2
\ee
This is non-singular on the null hypersurface $r=M$.

\paragraph{Proposition}  $r=M$ is a \emph{degenerate} (i.e. surface 
gravity $\kappa=0$) Killing horizon\index{Killing!horizon!degenerate} of the
Killing vector field $k=\partial/\partial v$.

\paragraph{Proof}  From the previous calculation $l=f\partial/\partial v$ 
so $r=M$ is a Killing horizon of $k$, and $k\cdot Dk=0$ when $r_+=r_-=M$. 

Since the orbits of $k$ on $r=M$ are affinely parameterized they must go to 
infinite affine parameter in both directions $\Rightarrow$ \emph{internal
$\infty$}.  This is the same internal $\infty$ that we find down the 
infinite ER
bridge.

Note that $k$ is null on $r=2M$, but \emph{timelike everywhere else}, 
so region II has disappeared and region I now leads directly to region V.  
The CP diagram is 
\begin{center}\input{p76-1.pictex}\end{center}

\subsection{Nature of Internal $\infty$ in Extreme RN}

The asymptotic metric as $r\to\infty$ is Minkowski.  To determine the 
asymptotic metric as $r\to M$ we introduce the new coordinate $\lambda$ by
$r=M(1+\lambda)$ and keep only the leading terms in $\lambda$, to get
\bea
F & \sim & d\lambda\wedge d t \\
ds^2 & \sim & \underbrace{ \left(-\lambda^2 dt^2+M^2
\lambda^{-2}d\lambda^2\right)}_{adS_2} +
\underbrace{M^2d\Omega^2}_{\stackrel{\subtext{2-sphere}}{\subtext{ of radius
$M$}}}
\eea
This is the Robinson-Bertotti metric.  It is a kind of `Kaluza-Klein' 
vacuum\index{Kaluza-Klein vacuum} in which two directions are compactified and
the `effective' spacetime is the two-dimensional `anti-de Sitter' ($adS_2$)
spacetime of constant negative curvature.  (See Q.II.7).

\end{list}

\subsection{Multi Black Hole Solutions}

The extreme RN in isotropic coordinates is
\be
ds^2=V^{-2}dt^2+V^2\left(d\rho^2+\rho^2d\Omega^2\right)
\ee
where 
\be
V=1+\frac{M}{\rho}
\ee
This is a special case of the multi black hole solution
\be
ds^2=V^{-2}dt^2+V^2d\vec{x}\cdot d\vec{x}
\ee
where $d\vec{x}\cdot d\vec{x}$ is the Euclidean 3-metric and $V$ is any 
solution of $\nabla^2 V=0$.  In particular,
\be
V=1+\sum_{i=1}^N \frac{M_i}{\left| \vec{x}-\vec{\bar{x}}^i\right|}
\ee
yields the metric for $N$ extreme black holes of masses $M_i$ at positions 
$\bar{x}_i$.  Note that the `points' $\bar{x}_i$ are actually minimal
2-spheres.  There are no $\delta$-function singularities at $x=\bar{x}_i$
because the lines of force continue indefinitely into the asymptotically RB
regions (`charge without charge'). \\

Note that a static multi black hole solution is possible only when there 
is an exact balance between the gravitational attraction and the electrostatic
repulsion.  This occurs only for $M=|Q|$.



