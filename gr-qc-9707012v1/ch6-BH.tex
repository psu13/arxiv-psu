\chapter{Black Hole Mechanics}

\section{Geodesic Congruences}

\paragraph{Definition}  A congruence\index{congruence} is a family of curves 
such that precisely one curve of the family passes through each point.  It is a
geodesic congruence\index{congruence!geodesic}\index{geodesic!congruence} if the
curves are geodesics. \\

The equations of a geodesic congruence may be written as 
$x^{\mu}=x^{\mu}\left(y^{\alpha},\lambda\right)$ where the parameters
$y^{\alpha}, \alpha=0,1,2$ label the geodesic and $\lambda$ is an affine
parameter\index{affine parameter} on the geodesic, i.e. 
\be
t= \frac{d}{d\lambda}=\pd{x^{\mu}}{\lambda}\partial_{\mu}
\ee
is the tangent to the geodesics such that $t\cdot Dt^{\mu}=0$. Since the
parameter $\lambda$ is affine, $t^2\equiv -1$ for timelike geodesics (while
$t^2\equiv 0$ for null geodesics). The vectors
\be
\eta_{\alpha} = \frac{d}{dy^{\alpha}}=\pd{x^{\mu}}{y^{\alpha}}\partial_{\mu}
\ee
may be considered as a basis of `displacement' vectors across the congruence:
\begin{center}\input{p107-1.pictex}\end{center}
Note that $t$ and $\eta_{\alpha}$ commute (since we could choose 
coordinates $x^{\mu}$ s.t. $t=\fltpd{}{\lambda}$ and
$\eta_{\alpha}=\fltpd{}{y^{\alpha}}$), so
\bea
0 & = & t^{\nu}\partial_{\nu}\eta_{\alpha}^{\mu}-
\eta^{\nu}_{\alpha}\partial_{\nu}t^{\mu} \\
 & = & t^{\nu}\left(\partial_{\nu}\eta^{\mu}_{\alpha}+
\Gamma^{\mu}_{\I\sigma\nu}\eta^{\sigma}_{\alpha}\right)-
\eta^{\nu}_{\alpha}\left(\partial_{\nu}t^{\mu}+
\Gamma^{\mu}_{\I\sigma\nu}t^{\sigma}\right)
\\
 & = & t^{\nu}D_{\nu}\eta^{\mu}_{\alpha}-
\eta^{\nu}_{\alpha}D_{\nu}t^{\mu} \quad \mbox{(by symmetry of connection)}
\eea
or
\bebox{
t^{\nu}D_{\nu}\eta^{\mu}_{\alpha}=B^{\mu}_{\I \nu}\eta^{\nu}_{\alpha}
}
where
\be
B^{\mu}_{\I \nu}=D_{\nu}t^{\mu}
\ee
measures the failure of the displacement vectors $\eta_{\alpha}$ to be 
paralelly-transported along the geodesics, i.e. it measures \emph{geodesic
deviation}\index{geodesic!deviation}. \\

A geodesic nearby some fiducial geodesic may now be specified by a displacement
vector $\eta$, but this specification is not \emph{unique} because
$\eta'=\eta+at$ ($a=$ constant) is a displacement vector to the \emph{same}
geodesic.
\begin{center}\input{p108-1.pictex}\end{center}
For timelike geodesics we can remove this ambiguity by requiring 
$\eta$ to be orthogonal to $t$, i.e.
\bebox{
\eta\cdot t=0 }
Strictly, speaking we can only make such a choice at a given value of 
$\lambda$, by choosing the origin of $\lambda$ across the congruence. However
\bea
\frac{d}{d\lambda}(\eta\cdot t) & = & 
\left(t\cdot D\eta^{\mu}\right)t_{\mu} 
\quad \mbox{(since $t\cdot Dt_{\mu}=0$)} \\
 & = & B^{\mu}_{\I\nu} \eta^{\nu} t_{\mu} = 
\left(\eta^{\nu}D_{\nu}t^{\mu}\right)t_{\mu} \\
 & = & \half \eta\cdot \partial t^2 =0 \, ,
\eea
since $t^2\equiv -1$ for timelike congruences, so if $\eta\cdot t$ is chosen to
vanish at one value of $\lambda$ it will do so for all $\lambda$. \\

For null congruences\index{congruence!null} the condition $\eta\cdot t=0$ is
not sufficient to eliminate the ambiguity in the choice of $\eta$ because 
\bea
\eta'\cdot t & = & (\eta+at)\cdot t = \eta\cdot t + a t\cdot t \\
 & = & \eta\cdot t 
\eea
when $t^2=0$, which means that $\eta'\cdot t=0$ whenever $\eta\cdot t=0$. 
The problem is that \emph{the 3-dim space of vectors orthogonal to $t$ now
includes $t$ itself}, so the displacement vectors $\eta$ orthogonal to $t$ 
specify only a \emph{\emph{two}-parameter family of geodesics}.  Displacement
vectors to the other null geodesics in the congruence have a component in the
direction of a vector $n$ that is \ul{not} orthogonal to $t$. The choice of $n$
is otherwise arbitrary (it is analogous to the choice of gauge in
electrodynamics), but it is \emph{convenient} to choose it such that
\bebox{n^2=0, \quad n\cdot t=-1} e.g. if $t$ is tangent to an outgoing radial
null geodesic, then $n$ is  tangent to an ingoing one.
\begin{center}\input{p109-1.pictex}\end{center}
Consistency of the choice of $n$ requires that $n^2$ and $n\cdot t$ be 
independent of $\lambda$, which is satisfied if 
\bebox{t\cdot Dn^{\mu}=0}
i.e. we choose $n$ to be parallely-transported along the geodesics. \\

Having made a choice of the vector $n$, we may now uniquely specify a
two-parameter subset of geodesics of a null geodesic congruence by 
displacement vectors $\eta$ orthogonal to $t$ by requiring them to also
satisfy 
\bebox{
\eta\cdot n = 0
}
The vectors $\eta$ now span a two-dimensional subspace, $T_{\perp}$, of the
tangent space, that is orthogonal to both $t$ and $n$, i.e. $P\eta=\eta$, where
\be
P^{\mu}_{\I\nu} = \delta^{\mu}_{\I\nu}+n^{\mu}t_{\nu}+t^{\mu}n_{\nu}
\ee
projects onto $T_{\perp}$.
\smallskip

\fbox{\parbox{6in}{
\paragraph{Proposition} $P\eta=\eta\Rightarrow t\cdot D\eta^{\mu}=
\hat{B}^{\mu}_{\I\nu} \eta^{\nu}$, where
\be
\hat{B}^{\mu}_{\I \nu} = P^{\mu}_{\I\lambda}B^{\lambda}_{\I\rho}
P^{\rho}_{\I\nu}
\ee
i.e. if $\eta\in T_{\perp}$ initially, it remains in this subspace. \\
}}
\paragraph{Proof}
\bea
t\cdot D\eta^{\mu} & = & t\cdot D\left(P^{\mu}_{\I\nu}\eta^{\nu}\right) 
\quad \mbox{(if $P\eta=\eta$)} \\
 & = & P^{\mu}_{\I\nu} t\cdot D\eta^{\nu} \quad 
\mbox{(since $t\cdot Dn=t\cdot Dt=0$)} \\
 & = & P^{\mu}_{\I\nu} B^{\nu}_{\I\rho}\eta^{\rho} 
\quad \mbox{(by definition)} \\
 & = & P^{\mu}_{\I\nu} B^{\nu}_{\I\rho} P^{\rho}_{\I\lambda} 
\eta^{\lambda} \quad \mbox{(since $P\eta=\eta$)} \\
 & = & \hat{B}^{\mu}_{\I\nu} \eta^{\nu} \quad \Box.
\eea

$\hat{B}$ is effectively a $2\times 2$ matrix.  We now decompose it into its 
algebraically irreducible parts
\be
\hat{B}^{\mu}_{\I\nu}=\half \theta P^{\mu}_{\I\nu}+
\hat{\sigma}^{\mu}_{\I\nu}+\hat{\omega}^{\mu}_{\I\nu} 
\ee
where

\begin{tabular}{rclcc}
$\theta$ & = & $\hat{B}^{\mu}_{\I\mu}$ & (trace) & 
\emph{expansion} \\
$\hat{\sigma}_{\mu\nu}$ & = & $\hat{B}_{(\mu\nu)}-\half P_{\mu\nu}
\hat{B}^{\rho}_{\I\rho}$ & (symmetric, traceless) & \emph{shear} \\
$\hat{\omega}_{\mu\nu}$  & = & $\hat{B}_{[\mu\nu]}$  & 
(anti-symmetric) & \emph{twist}
\end{tabular}
\smallskip

Notation:
\bean
\hat{B}_{(\mu\nu)} & = & \half\left(\hat{B}_{\mu\nu}+\hat{B}_{\nu\mu}\right) \\
\hat{B}_{[\mu\nu]} & = & \half\left(\hat{B}_{\mu\nu}-\hat{B}_{\nu\mu}\right)
\eean

\paragraph{Lemma} $t_{[\mu}\hat{B}_{\nu\rho]} = t_{[\mu}B_{\nu\rho]}$

\paragraph{Proof} Using $t\cdot Dt=0$ and $t^2 = 0$, we have
\be
\hat{B}^{\mu}_{\I\nu} = B^{\mu}_{\I\nu}+
t^{\mu}\left(n_{\lambda}B^{\lambda}_{\I\nu}+n_{\lambda}
B^{\lambda}_{\I\rho}n^{\rho}t_{\nu}\right)+
\left(B^{\mu}_{\I\rho}n^{\rho}\right)t_{\nu}
\ee
Hence result. ($\left[\quad\right]$ indicates total anti-symmetrization 
on enclosed indices).
\smallskip

\fbox{\parbox{6in}{
\paragraph{Proposition} The tangents $t$ are normal to a family of null 
hypersurfaces iff $\hat{\omega}=0$. 
}}

\paragraph{Proof}  If $\hat{\omega}=0$, then
\bea
0 & = & t_{[\mu}\hat{\omega}_{\nu\rho]} \equiv t_{[\mu}\hat{B}_{\nu\rho]} \\
 & = & t_{[\mu}B_{\nu\rho]} \quad \mbox{(by Lemma)} \\
 & = & t_{[\mu}D_{\rho}t_{\nu]} 
\eea
so $t$ is normal to a family of hypersurfaces by Frobenius' 
theorem\index{Frobenius' theorem}.  (In this case we can take $t=l$).

Conversely, if $t$ is normal to a family of null hypersurfaces, then 
Frobenius' theorem implies $t_{[\mu}D_{\nu}t_{\rho]}=0$.  Then, reversing the
previous steps we find that,
\be
0 = t_{[\mu}\hat{\omega}_{\nu\rho]} = 
\frac{1}{3}\left(t_{\mu}\hat{\omega}_{\nu\rho}+
t_{\rho}\hat{\omega}_{\mu\nu}+t_{\nu}\hat{\omega}_{\rho\mu}
\right) 
\ee
Contract with $n$.  Since $n\cdot t=-1$ and $n\hat{\omega}=
\hat{\omega}n=0$ (because $\hat{\omega}$ contains the projection operator $P$),
we deduce that $\hat{\omega}=0$.  \\

If $\hat{\omega}=0$ we have a family of null hypersurfaces.  The family 
is parameterized by the displacement along $n$
\begin{center}\input{p112-1.pictex}\end{center}


\subsection{Expansion and Shear}

Two linearly independent vectors $\eta^{(1)}$ and $\eta^{(2)}$ orthogonal 
to $n$ and $t$ determine an area element of $T_{\perp}$.  The shear
$\hat{\sigma}$ determines the change of \emph{shape} of this area element as
$\lambda$ increases.  The \emph{magnitude} of the area element defined by
$\eta^{(1)}$ and $\eta^{(2)}$ is
\be
a = \varepsilon^{\mu\nu\rho\sigma}t_{\mu}\eta_{\nu}
\eta^{(1)}_{\rho}\eta^{(2)}_{\sigma}
\ee
Since $t\cdot Dt=0$ and $t\cdot Dn=0$, we have
\bea
\frac{da}{d\lambda} & = & t\cdot \partial a = t\cdot 
Da= \varepsilon^{\mu\nu\rho\sigma}t_{\mu}n_{\nu}\left(t\cdot
D\eta^{(1)}_{\rho}\eta^{(2)}_{\sigma}+\eta^{(1)}_{\rho}t\cdot
D\eta^{(2)}_{\sigma}\right) \\
 & = &  \varepsilon^{\mu\nu\rho\sigma}t_{\mu}n_{\nu}
\left[\hat{B}_{\rho}^{\I\lambda}\eta^{(1)}_{\lambda}
\eta^{(2)}_{\sigma}+\eta^{(1)}_{\rho}
\hat{B}_{\sigma}^{\I\lambda}\eta^{(2)}_{\lambda}\right]
\\
 & = & 2 \varepsilon^{\mu\nu\rho\sigma}t_{\mu}n_{\nu} 
\hat{B}_{\rho}^{\I\lambda} \eta^{(1)}_{[\lambda}\eta^{(2)}_{\sigma]} \\
 & = & \theta a \quad \mbox{(see Question IV.2)}
\eea
i.e. $\theta$ measures the rate of increase of the magnitude of the 
area element.  If $\theta>0$ neighboring geodesics are \emph{diverging}, if
$\theta<0$ they are \emph{converging}.

\subsubsection{Raychaudhuri's equation for null geodesic congruences}

\bea
\frac{d\theta}{d\lambda} & = & t\cdot D
\left(B^{\mu}_{\I\nu}P^{\nu}_{\I\mu}\right) \\
 & = & P^{\nu}_{\I\mu} t\cdot DB^{\mu}_{\I\nu} 
\quad \mbox{(since $t\cdot Dt=0$ and $t\cdot Dn=0$)} \\
 & = & P^{\nu}_{\I\mu} t^{\rho}D_{\rho}D_{\nu}t^{\mu} \\
 & = & P^{\nu}_{\I\mu} t^{\rho}D_{\nu}D_{\rho}t^{\mu}+
P^{\nu}_{\I\mu}t^{\rho}\left[D_{\rho},D_{\nu}\right]t^{\mu} \\
 & = & P^{\nu}_{\I\mu}\left[\underbrace{D_{\nu}
\left(t\cdot Dt^{\mu}\right)}_0
-\left(D_{\nu}t^{\rho}\right)\left(D_{\rho}t^{\mu}
\right)\right]+P^{\nu}_{\I\mu}t^{\rho}R_{\rho\nu\I\sigma}^{\I\I\mu}
t^{\sigma} \\
 & = & -P^{\nu}_{\I\mu}B^{\mu}_{\I\rho}
B^{\rho}_{\I\nu}-t^{\rho}R_{\rho\sigma}
t^{\sigma}\quad \mbox{(using symmetries
of $R$)} \\
 & = & -P^{\nu}_{\I\mu}B^{\mu}_{\I\lambda}
P^{\lambda}_{\I\rho}B^{\rho}_{\I\nu}+
P^{\nu}_{\I\mu}B^{\mu}_{\I\lambda}t^{\lambda}n_{\rho}B^{\rho}_{\I\nu}
\nn
+P^{\nu}_{\I\mu}B^{\mu}_{\I\lambda}n^{\lambda}t_{\rho}
B^{\rho}_{\I\nu}-t^{\rho}t^{\sigma}R_{\rho\sigma} 
\\
 & = & -\hat{B}^{\mu}_{\I\rho} \hat{B}^{\rho}_{\I\nu}-
t^{\rho}t^{\sigma}R_{\rho\sigma} \quad \mbox{(using $t\cdot Dt\equiv 0$ and
$t^2\equiv 0$)}
\eea
or
\bebox{
\frac{d\theta}{d\lambda}=-\half\theta^2-\hat{\sigma}^{\mu\nu}
\hat{\sigma}_{\mu\nu}+\hat{\omega}^{\mu\nu}\hat{\omega}_{\mu\nu}-
R_{\mu\nu}t^{\mu}t^{\nu}
}
This is Raychaudhuri's equation for null geodesic 
congruences.\index{Raychaudhuri equation}

\subsubsection{Some consequences of Raychaudhuri's equation for 
null hypersurfaces}

\fbox{\parbox{6in}{
\paragraph{Proposition} The expansion $\theta$ of the null geodesic 
generator of a null hypersurface, $\mcN$, obeys the differential inequality 
\be
\frac{d\theta}{d\lambda}\le -\half \theta^2 
\ee
provided the spacetime metric solves Einstein's equations 
$G_{\mu\nu}=8\pi GT_{\mu\nu}$ and $T_{\mu\nu}$ satisfies the weak energy
condition. }}

\paragraph{Proof} $\hat{\sigma}^2 \ge 0$ because the metric in the 
orthogonal subspace $T_{\perp}$ (to $l$ and $n$) is positive definite.
$\hat{\omega}^2\ge 0$ also, but this comes in with wrong sign, however
$\hat{\omega}=0$ for a hypersurface.  Thus Raychaudhuri's equation implies
\bea
\frac{d\theta}{d\lambda} & \le & -\half \theta^2-R_{\mu\nu}l^{\mu}l^{\nu} \\
 & \le & -\half \theta^2-8\pi g T_{\mu\nu}l^{\mu}l^{\nu} \quad 
\mbox{(by Einstein's eq.)} \\
 & \le & -\half \theta^2 \quad \mbox{by weak energy condition}
\eea
\fbox{\parbox{6in}{
\paragraph{Corollary} If $\theta=\theta_0 < 0$ at some point $p$ on a null 
generator $\gamma$ of a null hypersurface, then $\theta\to -\infty$ along
$\gamma$ within an affine length $2/\left|\theta_0\right|$. }}

\paragraph{Proof} Let $\lambda$ be the affine parameter, with $\lambda=0$ at 
$p$.  Now
\be
\frac{d\theta}{d\lambda} \le -\half \theta^2 \quad \Leftrightarrow \quad 
\frac{d}{d\lambda}\left(\theta^{-1}\right)>\half \quad \Rightarrow \quad
\theta^{-1} \ge \half\lambda +\mbox{constant} 
\ee
where, since $\theta=\theta_0$ at $\lambda=0$, the constant cannot exceed 
$\theta_0^{-1}$.  Thus
\be
\theta^{-1}\ge \half\lambda +\theta_0^{-1} \quad \Rightarrow \quad 
\theta \le \frac{\theta_0}{1+\half\lambda \theta_0}
\ee
If $\theta_0<0$ the right-hand-side $\to -\infty$ when $\lambda=2/\left|
\theta_0\right|$, so $\theta\to -\infty$ within that affine length.

\paragraph{Interpretation}  When $\theta<0$ neighboring geodesics are 
converging.  The attractive nature of gravitation (weak energy condition) then
implies that they must continue to converge to a focus or a caustic. \\

\fbox{\parbox{6in}{
\paragraph{Proposition} If $\mcN$ is a Killing horizon then 
$\hat{B}_{\mu\nu}=0$ and 
\be
\frac{d\theta}{d\lambda}=0
\ee
}}

\paragraph{Proof}  Let $\xi$ be the Killing vector s.t. $\xi=fl$ 
$(l\cdot Dl=0)$ on $\mcN$ for some non-zero function $f$.  Then
\bea
\hat{B}_{\mu\nu} & = & \hat{B}_{(\mu\nu)} \quad \mbox{(since 
$\hat{\omega}=0$ for family of hypersurface)} \\
 & = & P_{\mu}^{\I\lambda}B_{(\lambda\rho)}P^{\rho}_{\I\nu}
\equiv P_{\mu}^{\I\lambda} D_{(\rho}l_{\lambda)}P^{\rho}_{\I\nu} \\
 & = & P_{\mu}^{\I\lambda}\left(\partial_{(\rho}f^{-1}
\right)\xi_{\lambda)}P^{\rho}_{\nu} \quad \mbox{(since
$D_{(\rho}\xi_{\lambda)}=0$)} \\
 & = & 0 \quad \mbox{(since $P\xi=\xi  P=0$)}
\eea
In particular $\theta=0$, \emph{everywhere on $\mcN$}, so 
$d\theta/d\lambda=0$.

\paragraph{Corollary}  For Killing horizon $\mcN$ of $\xi$
\bebox{
\left.R_{\mu\nu}\xi^{\mu}\xi^{\nu}\right|_{\mcN}=0
}

\paragraph{Proof}  Using $d\theta/d\lambda=0$ and $\hat{B}_{\mu\nu}=0$ in 
Raychaudhuri's equation.

\section{The Laws of Black Hole Mechanics}

Previously we showed that $\kappa^2$ is constant on a \emph{bifurcate} 
Killing horizon.  The proof fails if we have only part of a Killing horizon,
without the bifurcation 2-sphere, as happens in \emph{gravitational collapse}. 
In this case we need the:

\subsection{Zeroth law}  

\fbox{\parbox{6in}{If $T_{\mu\nu}$ obeys the dominant energy condition then 
the surface gravity $\kappa$ is constant on the future event horizon. }}

\paragraph{Proof}  Let $\xi$ be the Killing vector normal to $\mcH^+$ (here 
we use the theorem that $\mcH^+$ \emph{is} a Killing horizon).  Then since
$R_{\mu\nu}\xi^{\mu}\xi^{\nu}=0$ and $\xi^2=0$ on $\mcH^+$, Einstein's equations
imply
\be
0 = -\left. T_{\mu\nu}\xi^{\mu}\xi^{\nu}\right|_{\mcH^+} \equiv 
\left.J_{\mu}\xi^{\mu}\right|_{\mcH^+} 
\ee
i.e. $J=\left(-T^{\mu}_{\I\I\nu}\xi^{\nu}\right)\partial_{\mu}$ is 
tangent to $\mcH^+$.  It follows that $J$ can be expanded on a basis of tangent
vectors to $\mcH^+$
\be
J=a\xi+b_1\eta^{(1)}+b_2\eta^{(2)} \quad \mbox{on $\mcH^+$}
\ee
But since $\xi\cdot\eta^{(i)}=0$ this is spacelike or null 
(when $b_1=b_2=0$), whereas it must be \emph{timelike or null} by the
\emph{dominant energy condition}.  Thus, dominant energy $\Rightarrow J\propto
\xi$ and hence that
\bea
0 & = & \left.\xi_{[\sigma}J_{\rho]}\right|_{\mcH^+} = 
-\left.\xi_{[\sigma}T_{\rho]}^{\I\lambda}\xi_{\lambda}\right|_{\mcH^+} \\
 & = & \left. \xi_{[\sigma}R_{\rho]}^{\I\I\lambda}
\xi_{\lambda}\right|_{\mcH^+} \quad \mbox{(by Einstein's eq.)} \\
 & = & \left. \xi_{[\rho}\partial_{\sigma]}\kappa
\right|_{\mcH^+} \quad \mbox{(by result of Question IV.3)} \\
\eea
$\Rightarrow \partial_{\sigma}\kappa\propto \xi_{\sigma} 
\Rightarrow t\cdot \partial \kappa=0$ for any tangent vector $t$ to $\mcH^+$ 

$\Rightarrow$ \emph{$\kappa$ is constant on $\mcH^+$}.

\subsection{Smarr's Formula}

Let $\Sigma$ be a spacelike hypersurface in a stationary exterior black 
hole spacetime with an inner boundary, $H$, on the future event horizon and
another boundary at $i_0$.
\begin{center}\input{p117-1.pictex}\end{center}
The surface $H$ is a 2-sphere that can be considered as the `boundary' of 
the black hole. \\

Applying Gauss' law to the Komar integral for $J$ we have
\bea
J & = & \frac{1}{8\pi G}\int_{\Sigma}dS_{\mu}D_{\nu}D^{\mu}m^{\nu}+
\frac{1}{16\pi G}\oint_H dS_{\mu\nu}D^{\mu}m^{\nu} \\
 & = & \frac{1}{8\pi G}\int_{\Sigma} dS_{\mu}R^{\mu}_{\I\nu} m^{\nu} +
J_H \quad \mbox{by Killing vector Lemma}
\eea
where $J_H$ is the integral over $H$.  Using Einstein's equation,
\be
J=\int_{\Sigma}dS_{\mu}\left(T^{\mu}_{\I\I\nu}m^{\nu}m^{\nu}-
\half Tm^{\mu}\right)+J_H 
\ee
In the absence of matter other than an electromagnetic field, we 
have $T_{\mu\nu}=T_{\mu\nu}(F)$, the stress tensor of the electromagnetic
field.  Since $g^{\mu\nu}T_{\mu\nu}(F)=T(F)=0$ we have
\bebox{
J=\int_{\Sigma}dS_{\mu}T^{\mu}_{\I\I\nu}(F)m^{\nu}+J_H 
\label{eq:Smarrstar}}
for an \emph{isolated} black hole (i.e. $T_{\mu\nu}=T_{\mu\nu}(F)$).

Now apply Gauss' law to the Komar integral for the total energy (= mass).
\bea
M & = & -\frac{1}{4\pi G}\int_{\Sigma}dS_{\mu}R^{\mu}_{\I\nu}k^{\nu}-
\frac{1}{8\pi G}\oint_H dS_{\mu\nu}D^{\mu}k^{\nu} \quad \mbox{(insert
$\xi=k+\Omega_H m$)} \nn \\
 & = & \int_{\Sigma}dS_{\mu}\left(-2T^{\mu}_{\I\I\nu}k^{\nu}+
Tk^{\mu}\right)-\frac{1}{8\pi G}\oint_H
dS_{\mu\nu}\left(D^{\mu}\xi^{\nu}-\Omega_H D^{\mu}m^{\nu}\right) \\
\eea
since $\Omega_H$ is constant on $H$.  For $T_{\mu\nu}=T_{\mu\nu}(F)$ 
$(T(F)=0)$we have
\be
M = -2\int_{\Sigma} dS_{\mu}T^{\mu}_{\I\I\nu}(F)k^{\nu}+2\Omega_HJ_H-
\frac{1}{8\pi G}\oint_H dS_{\mu\nu}D^{\mu}\xi^{\nu} 
\ee
for an isolated black hole.  Using (\ref{eq:Smarrstar}) we have
\be
M = -2\int_{\Sigma}dS_{\mu}T^{\mu}_{\I\I\nu}(F)\xi^{\nu}+2\Omega_H J -
\frac{1}{8\pi G}\oint_H dS_{\mu\nu}D^{\mu}\xi^{\nu}
\ee
For simplicity, we now suppose that $T_{\mu\nu}(F)=0$, i.e. the black hole 
has zero charge (see Questions III.7\&8 for general case).  Then
\be
M=2\Omega_H J-\frac{1}{8\pi G}\oint_H dS_{\mu\nu}D^{\mu}\xi^{\nu}
\ee

\paragraph{Lemma}
\be
dS_{\mu\nu}=\left(\xi_{\mu}n_{\nu}-\xi_{\nu}n_{\mu}\right)dA 
\quad \mbox{on $H$}
\ee
where $n$ is s.t. $n\cdot\xi=-1$.

\paragraph{Proof} $n$ and $\xi$ are normals to $H$, so we have to check 
coefficients.  In coordinates such that
\begin{center}\input{p119-1.pictex}\end{center}
\bea
\xi_{\mu} & = & \frac{1}{\sqrt{2}}(1,1,0,0) \\
n_{\mu} & = & \frac{1}{\sqrt{2}}(1,-1,0,0)
\eea
we should have $\left|dS_{01}\right|=dA$.  We do if $dS_{\mu\nu}$ is as 
given.  [There is still a sign ambiguity.  Fix by requiring sensible results].

Thus
\bea
-\frac{1}{8\pi G}\oint_H dS_{\mu\nu}D^{\mu}\xi^{\nu} & = & 
-\frac{1}{4\pi G}\oint_H dA\underbrace{ (\xi\cdot D\xi)^{\nu}
}_{\kappa\xi^{\nu}} n_{\nu} \\
 & = & -\frac{\kappa}{4\pi G}\oint dA \underbrace{\xi\cdot n}_{-1} 
\qquad \mbox{($\kappa$ is constant by $0^{\subtext{th}}$ law)} \\
 & = & \frac{\kappa}{4\pi G} A 
\eea
where $A$ is the ``area of the horizon'' (i.e. $H$).

Hence
\bebox{
M=\frac{\kappa A}{4\pi} +2\Omega_H J
}
This is Smarr's formula\index{Smarr's formula} for the mass of a Kerr 
black hole. [Exercise:  Check, using previous results for $\kappa$, $\Omega_H$,
and $A$].  In the $Q\neq 0$ case, this formula generalizes to 
\be
M=\frac{\kappa A}{4\pi}+2\Omega_H J+\Phi_H Q
\ee
where $\Phi_H$ is the co-rotating electric potential\index{co-rotating 
electric potential} on the horizon (see Question III.6\&7).

\subsection{First Law}  

\fbox{\parbox{6in}{If a stationary black hole of mass $M$, charge $Q$ and 
angular momentum $J$, with future event horizon of surface gravity $\kappa$,
electric surface potential $\Phi_H$ and angular velocity $\Omega_H$, is
perturbed such that it settles down to another black hole with mass $M+\delta M$
charge $Q+\delta Q$ and angular momentum $J+\delta J$, then
\bebox{
dM=\frac{\kappa}{8\pi}dA+\Omega_H dJ+\Phi_H dQ 
}
}}
\newcounter{lawtwonotes}
\begin{list}{\arabic{lawtwonotes})}
{\usecounter{lawtwonotes}}
\item Definition of $\Phi_H$ and proof for $Q\neq 0$ in Q. III.6\&7.

\item This statement of the first law uses the fact that the event horizon 
of a stationary black hole must be a Killing horizon.
\end{list}

\paragraph{`Proof' for $Q= 0$ (Gibbons)}  Uniqueness theorems imply that
\be
M=M(A,J)
\ee
But $A$ and $J$ both have dimensions of $M^2$ $(G=c=1)$ so the function 
$M(A,J)$ must be \emph{homogeneous of degree $1/2$}.  By Euler's theorem for
homogeneous functions
\bea
A\pd{M}{A}+J\pd{M}{J} & = & \half M \\
 & = & \frac{\kappa}{8\pi}A+\Omega_H J \quad \mbox{by Smarr's formula}
\eea
Therefore 
\be
A\left(\pd{M}{A}-\frac{\kappa}{8\pi}\right)+J\left(\pd{M}{J}-\Omega_H\right)=0
\ee
But $A$ and $J$ are free parameters so 
\be
\pd{M}{A}=\frac{\kappa}{8\pi}, \quad \pd{M}{J}=\Omega_H
\ee

\subsection{The Second Law (Hawking's Area Theorem)}

\fbox{\parbox{6in}{If $T_{\mu\nu}$ satisfies the weak energy condition, and 
assuming that the cosmic censorship hypothesis is true then the area of the
future event horizon of an asymptotically flat spacetime is a non-decreasing
function of time. }} \\

Technically the cosmic censorship assumption is that the spacetime is `strongly
asymptotically predictable' which requires the existence of a globally
hyperbolic submanifold of spacetime containing both the exterior spacetime
\emph{and} the horizon.  A theorem of Geroch states that in this case there
exists a family of Cauchy hypersurfaces $\Sigma(\lambda)$ such that
$\Sigma(\lambda') \subset D^+\left(\Sigma(\lambda)\right)$ if 
$\lambda'>\lambda$.
\begin{center}\input{p121-1.pictex}\end{center}
We can choose $\lambda$ to be the affine parameter on a null geodesic 
generator of $\mcH^+$.  The ``area of the horizon'' $A(\lambda)$ is the area of
the intersection of $\Sigma(\lambda)$ with $\mcH^+$.  The second law states that
$A(\lambda')\ge A(\lambda)$ if $\lambda'>\lambda$.

\paragraph{Idea of proof}  To show that $A(\lambda)$ cannot decrease with 
increasing $\lambda$ it is sufficient to show that each area element, $a$, of
$H$ has this property.  Recalling that
\be
\frac{da}{d\lambda}=\theta a
\ee
we see that the second law holds if $\theta\ge 0$ everywhere on $\mcH^+$.  
To see that this is true, recall that if $\theta<0$ the geodesics must converge
to a focus or caustic, i.e. nearby geodesics to a given one passing through a
point $p$ must intersect $\gamma$ at finite affine distance along it.  The first
point $q$ for which this happens is called the point conjugate to $p$ on
$\gamma$.
\begin{center}\input{p122-1.pictex}\end{center}
\emph{Points on $\gamma$ beyond $q$ are no longer null separated}.  They 
are \emph{timelike} separated from $p$.  An example illustrating this is light
rays in a flat 2-dim cylindrical spacetime.
\begin{center}\input{p122-2.pictex}\end{center}
The existence of a conjugate point to the future of a null geodesic generator 
in $\mcH^+$ would mean that this generator of $\mcH^+$ has a finite endpoint,
in contradiction to Penrose's theorem, so the hypothetical conjugate point
cannot exist.  Thus it must be that $\theta\ge 0$ everywhere on $\mcH^+$ and
hence the second law.  

$\theta=0$ only for stationary spacetimes.

% clearpage inserted to account for big diagram which follows
\clearpage
\paragraph{Example}  Formation of black hole from pressure-free 
spherically-symmetric gravitational collapse.  Illustrate by a 
Finkelstein diagram
\begin{center}\input{p123-1.pictex}\end{center}
$A=0$ on $\Sigma\left(\lambda_0\right)$.  $A\neq 0$ on 
$\Sigma\left(\lambda_1\right)$ and it has increased to its final value of
$A=16\pi M^2$ for a stationary Schwarzschild black hole on
$\Sigma\left(\lambda_2\right)$. \\
% clearpage inserted to account for big diagram which follows
\clearpage
\subsubsection{Consequences of $2^{\subtext{nd}}$ Law}

\newcounter{conseq}
\begin{list}{(\arabic{conseq})}
{\usecounter{conseq}}

\item Limits to efficiency of mass/energy conversion in black hole 
collisions.  Consider Finkelstein diagram of two coalescing black holes.
\begin{center}\input{p123-2.pictex}\end{center}
Then energy radiated is $M_1+M_2-M_3$, so the efficiency, $\eta$, of mass 
to energy conversion is
\be
\eta=\frac{M_1+M_2-M_3}{M_1+M_2} = 1-\frac{M_3}{M_1+M_2} 
\ee
Assuming that the two black holes are initially approximately stationary, 
so $A_1=16\pi M_1^2$ and $A_2=16\pi M_2^2$ the area theorem says that
\be
A_3 \ge 16\pi \left(M_1^2+M_2^2\right) 
\ee
But $16\pi M_3^2 \ge A_3$ (with equality at late times), so
\be
M_3\ge \sqrt{M_1^2+M_2^2}
\ee
Thus 
\be
\eta \le 1-\frac{\sqrt{M_1^2+M_2^2}}{M_1+M_2}\le 1-\frac{1}{\sqrt{2}}
\ee
The radiated energy could be used to do work, so the area theorem limits 
the useful energy that can be extracted from black holes in the same way that
the $2^{\subtext{nd}}$ law of thermodynamics limits the efficiency of heat
engines.

\item \emph{Black holes cannot bifurcate}.  Consider $M_3\to M_1+M_2$ 
(with $M_1>0$ and $M_2>0$).  The area theorem now says that
\be
M_3 \le \sqrt{M_1^2+M_2^2} \le M_1+M_2
\ee
but energy conservation requires $M_3 \ge M_1+M_2$ (with $M_3-M_1-M_2$ 
being radiated away).  We have a contradiction so the process cannot occur.

\end{list}


