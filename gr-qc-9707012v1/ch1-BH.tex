% Black Hole notes by P. Townsend
% typed by Tim Perkins

\chapter{Gravitational Collapse}

\section{The Chandrasekhar Limit}

A Star is a self-gravitating ball of hydrogen atoms supported by thermal
pressure $P \sim nkT$ where $n$ is the number density of atoms. In equilibrium,
\be
E=E_{\subtext{grav}}+E_{\subtext{kin}}
\ee
is a minimum.  For a star of mass $M$ and radius $R$
\bea
E_{\subtext{grav}} & \sim & -\frac{GM^2}{R} \\
E_{\subtext{kin}} & \sim & nR^3\left<E\right>
\eea
where $\left<E\right>$ is average kinetic energy of atoms.  Eventually, fusion 
at the core must stop, after which the star cools and contracts.  Consider the
possible final state of a star at $T=0$. The pressure $P$ does not go to zero as
$T\to 0$ because of \emph{degeneracy pressure}\index{degenerate pressure}. 
Since $m_e\ll m_p$ the electrons become degenerate first, at a number density 
of one electron in a cube of side $\sim$ Compton wavelength.
\be
n_e^{-1/3} \sim \frac{\hbar}{\left<p_e\right>}, \quad 
\left<p\right>=\mbox{average electron momentum}
\ee

\subsection*{Can electron degeneracy pressure support a star from collapse at 
$T=0$?}

Assume that electrons are \emph{non-relativistic}.  Then
\be
\left<E\right> \sim \frac{\left<p_e\right>^2}{m_e}.
\ee

So, since $n=n_e$,
\be
E_{\subtext{kin}}\sim \frac{\hbar^2 R^2 r_e^{2/3}}{m_e}.
\ee
Since $m_e \ll m_p$, $M\approx n_eR^3m_e$, so 
\fbox{$n_e\sim\bgfrac{M}{m_pR^3}$} and 
\be
E_{\subtext{kin}} \sim \underbrace{ \frac{\hbar^2}{m_e}
\left( \frac{M}{m_p}\right)^{5/3} }_{\mbox{constant for}\atop\mbox{fixed $M$}}
\frac{1}{R^2}.
\ee
Thus 
\be
E\sim -\frac{\alpha}{R}-\frac{\beta}{R^2},\quad \alpha,
\beta\mbox{independent of $R$}.
\ee
\begin{center}\input{fig1-1.pictex}\end{center}
The collapse of the star is therefore prevented. It becomes a 
\emph{White Dwarf}\index{white dwarf} or a cold, dead star supported by electron
degeneracy pressure. \\

At equilibrium
\be
n_e \sim \frac{M}{m_p R^3_{\subtext{min}}} \
\left( \frac{m_e G}{\hbar^2}\left(Mm_p^2\right)^{2/3}\right)^3.
\ee
But the validity of non-relativistic approximation requires that
$\left<p_e\right> \ll m_e c$, i.e.
%end page 2
\bea
\frac{\left<p_e\right>}{m_e} = \frac{\hbar n_e^{1/3}}{m_e}\ll c \\
\mbox{or} \quad n_e \ll \left( \frac{m_e c}{\hbar} \right)^2.
\eea

For a White Dwarf this implies
\bea
\frac{m_e G}{\hbar^2}\left(M m_p^2\right)^{2/3} \ll \frac{m_e c}{\hbar} \\
\mbox{or} \quad M \ll \frac{1}{m_p^2}\left(\frac{\hbar c}{G}\right)^{3/2}.
\eea

For sufficiently large $M$ the electrons would have to be relativistic, in which
case we must use
\bea
\lefteqn{\avg{E}  =  \avg{p_e}c=\hbar c n_e^{1/3} }\\
\Rightarrow \quad E_{\subtext{kin}} & \sim & 
n_eR^3\avg{E}\sim \hbar cR^3n_e^{4/3} \\
 & \sim & \hbar c R^3\left(\frac{M}{m_pR^3}\right)^{4/3}\sim \hbar c 
\left(\frac{M}{m_p}\right)^{4/3}\frac{1}{R}
\eea
So now,
\be
E \sim -\frac{\alpha}{R}+\frac{\gamma}{R}.
\ee
Equilibrium is possible only for
\be
\gamma=\alpha \quad \Rightarrow \quad M \sim \frac{1}{m_p^2}
\left(\frac{\hbar c}{G}\right)^{3/2}.
\ee

For smaller $M$, $R$ must increase until electrons become non-relativistic, in
which case the star is supported by electron degeneracy pressure, as we just
saw.  For larger $M$, $R$ must continue to decrease, so electron degeneracy
pressure cannot support the star. There is therefore a critical mass
$M_C$
\be
M_C \sim \frac{1}{m_p^2}\left(\frac{\hbar c}{G}\right)^{3/2} 
\quad \Rightarrow \quad R_C \sim \frac{1}{m_em_p}\left(
\frac{\hbar^3}{Gc}\right)^{1/2}
\ee
above which a star cannot end as a White Dwarf.  This is the 
\emph{Chandrasekhar limit}\index{Chandrasekhar limit}.  Detailed calculation
gives $M_C\simeq 1.4 M_{\odot}$.

\section{Neutron Stars}

The electron energies available in a White Dwarf are of the order of the 
Fermi energy.  Necessarily $E_F\stackrel{\scriptstyle <}{\scriptstyle \sim}
m_ec^2$ since the electrons are otherwise relativistic and cannot support the
star.  A White Dwarf is therefore stable against inverse $\beta$-decay
\be
e^-+p^+\to n+\nu_e
\ee
since the reaction needs energy of at least $(\Delta m_n)c^2$ where $\Delta m_n$ is the neutron-proton mass difference.  Clearly $\Delta m > m_e$ ($\beta$-decay would otherwise be impossible) and in fact $\Delta m\sim 3m_e$.  So we need energies of order of $3m_ec^2$ for inverse $\beta$-decay.  This is not available in White Dwarf stars but for $M>M_C$ the star must continue to contract until $E_F\sim (\Delta m_n)c^2$.  At this point inverse $\beta$-decay can occur.  The reaction cannot come to equilibrium with the reverse reaction
\be
n+\nu_e\to e^-+p^+
\ee 
because the neutrinos escape from the star, and $\beta$-decay, 
\be
n \to e^-+p^+\bar{\nu}_e
\ee
cannot occur because all electron energy levels below $E<(\Delta m_n)c^2$ are 
filled when $E>(\Delta m_n)c^2$.  Since inverse $\beta$-decay removes the
electron degeneracy pressure the star will undergo a catastrophic collapse to
nuclear matter density, at which point we must take \emph{neutron-degeneracy
pressure} into account.

\subsubsection{Can neutron-degeneracy pressure support the star against 
collapse?}

The ideal gas approximation would give same result as before but with 
$m_e\to m_p$.  The critical mass $M_C$ is \emph{independent} of $m_e$ and so is
unaffected, but the critical radius is now
\be
\left(\frac{m_e}{m_p}\right)R_C \sim \frac{1}{m_p^2}
\left(\frac{\hbar^3}{Gc}\right)^{1/2}\sim \frac{GM_C}{c^2}
\ee
which is the Schwarzschild radius, so the neglect of GR effects was
not justified.  Also, at nuclear matter densities the ideal gas approximation is
not justified.  A perfect fluid approximation is reasonable (since viscosity
can't help).  Assume that $P(\rho)$ ($\rho=$ density of fluid) satisfies
\bea
& \mbox{i)} & P\ge 0 \quad \mbox{(local stability).} \\
& \mbox{ii)} & P' < c^2 \quad \mbox{(causality).} 
\eea
Then the \emph{known behaviour} of $P(\rho)$ at low nuclear densities gives
\be
M_{\subtext{max}} \sim 3M_{\odot}.
\ee
More massive stars must continue to collapse either to an unknown new  
ultra-high density state of matter or to a black hole. The latter is more
likely. In any case, there must be {\it some} mass at which gravitational
collapse to a black hole is unavoidable because the density at the
Schwarzschild radius decreases as the total mass increases. In the limit of
very large mass the collapse is well-approximated by assuming the collapsing
material to be a pressure-free ball of fluid. We shall consider this
case shortly.
% end of p.5