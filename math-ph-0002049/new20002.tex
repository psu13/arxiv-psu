\section{The Quantum Leap}
The remarkable discovery of matrix mechanics by Heisenberg in 1925
is comparable to that of the theory of relativity in 1917.
Clifford had
speculated that the world might have chosen a geometry other than Euclidean.
It was agreed that it was an experimental question, and that the data
agreed with Einstein's theory. Though the classical axioms were yet to
be written down by Kolmogorov, Heisenberg, with
help of the Copenhagen interpretation, invented a generalisation of the
concept of probability, and physicists showed that this was the model of
probability chosen by atoms and molecules.

According to Einstein et al. \cite{Einstein2} a concept
is deemed to be an {\em element of reality} within
a specified theory if there is a mathematical object in the theory which
is assigned to the concept, and which takes a definite value (when the
state of the system is given). This
is now called an {\em observable}.
For example, the choice of the zero-level of a potential function, is not
an observable since it is not determined by the state of the system.
They are not here discussing
random samples, which at the time would have been described as an ensemble.
In that case, they might have conceded that a concept could be regarded as
an element of reality if, in a random selection of the system from an
ensemble, there is a definite random variable assigned to the physical
concept.
The interpretation of a theory is not complete unless it is
specified at the outset which mathematical
objects arising in the theory correspond to observables. Thus in a theory
with randomness in classical physics, there is a space $(\Omega,{\cal B})$
and an observable is a random variable, and an ensemble is a probability
measure on $\Omega$. A non-random state is given by a point-measure. In
this state any r. v. has zero variance, thus satisfying {\em EPR}.

In quantum mechanics, this is not the case; an
observable is a Hermitian matrix $A$, or in modern terms, a self-adjoint
operator on a given Hilbert space ${\cal H}$; the possible values one can
find in a measurement are the eigenvalues of $A$.
A wave-function is determined by a vector $\psi\in{\cal H}$; but only
unit vectors are used, and $e^{i\theta}\psi$ represents the same state as
$\psi$. Thus the state is the equivalence class
$\{\psi\}=\{e^{i\alpha}\psi,\alpha\in{\bf R}\}$. If
$\dim{\cal H}=n<\infty$, such equivalence classes make up the
projective space $CP^{n-1}$. An element of $CP^{n-1}$ determines the
expectation value of any observable $A$ by $\langle\psi,A\psi\rangle$,
which according to the Copenhagen interpretation, is the mean value of $A$
if measured many times in the state $\{\psi\}$. It is seen to be
independent of the representative vector $\psi\in\{\psi\}$. Such a state is
called a {\em vector state}.
The concept of state was generalised by von Neumann to
include random mixtures of vector states . Let ${\cal B}({\cal H})$
denote the set of bounded operators on ${\cal H}$; this is a complex vector
space, and also $^*$-algebra, where conjugation is given by the adjoint
and multiplication is the usual product of operators.
A state is given by a positive operator
$\rho$ of trace 1, called a density operator,
and the expectation of an observable $A$
is taken to be $m_1(A):=\mbox{Tr}\,(\rho A)$. Any density operator
determines an element of the dual space to ${\cal B}({\cal H})$
by the map $A\mapsto m_1(A)$.
We also can define
$m_n(A):=\mbox{Tr}\,\rho A^n$ to be the $n^{\rm th}$ moment of $A$, and
$\kappa_2(A):=m_2(A)-m_1(A)^2$ to be the second cumulant, the variance,
uncertainty or dispersion of $A$ in the state $\rho$. 
von Neumann showed that there are no dispersion-free states.
Thus, quantum mechanics is intrinsically random.
Heisenberg's {\em uncertainty relation}, which is a theorem, not
a postulate, is the best-known facet of this:
\begin{theorem}
Let $A,B,C\in{\cal B}({\cal H})$ be such that $[A,B]:=AB-BA=C$; then in
any state $\rho$, we have $\kappa_2(A)\kappa_2(B)\geq m_1(C)^2/4$.
\end{theorem}
There is no uncertainty relation for commuting operators $A,B$, and
such observables are said to be
{\em compatible}. If $[A,B]\neq0$, we say that $A$ and $B$ are
complementary.

Segal has emphasised that the bounded observables in any quantum theory
should form the Hermitian part of a $C^*$-algebra with identity. This is
a complex vector space ${\cal A}$ with
\begin{enumerate}
\item a product $AB$ is defined for all $A,B\in{\cal A}$, which is
distributive and associative, but not necessarily commutative;
\item a conjugation $A\mapsto A^*$, which is complex-antilinear, is
specified;
\item ${\cal A}$ is provided with a norm $\|\bullet\|$ which
obeys Gelfand's condition
\begin{equation}
\|A^*A\|=\|A\|^2;
\end{equation}
\item ${\cal A}$ is complete in the topology given by this norm.
\end{enumerate}
This concept includes all the examples we have seen so far; the set
${\cal M}_n({\bf C})$, denoting $n\times n$ matrices, with matrix addition
and product, is a $C^*$-algebra. The $^*$ operation is Hermitian conjugate,
and the norm $\|A\|$ is the maximum eigenvalue of $|A|=(A^*A)^{1/2}$.
For any Hilbert space, ${\cal B}({\cal H})$ is also a $C^*$-algebra,
and more generally, so is any von Neumann algebra, which
can be defined as any weakly closed $^*$-subalgebra of ${\cal B}({\cal H})$
containing the identity. Another notable example
is the subset ${\cal C}(n)$ of ${\cal M}(n)$ consisting of real diagonal
matrices $A=\mbox{diag}\,(a_1,\ldots,a_n)$. This is clearly commutative, and
the diagonal elements are the eigenvalues. Thus, each $A\in{\cal C}$
determines uniquely a function $i\mapsto a_i,\;1\leq i\leq n$ from the
set $\Omega_n=(1,2,\ldots,n)$ to ${\bf R}$. Conversely, any random
variable $f$ on $\Omega_n$ defines a unique diagonal matrix diag$\,(f(1),
\ldots,f(n))$. So the classical observables on $\Omega_n$ can be described
as a special type of quantum mechanics, namely, the diagonal matrices.
Moreover, the interpretation in classical theory, of the values of the
random variables $f$ as possible observed values, coincides with the
quantum interpretation of the eigenvalues. Also, each $n\times n$
density matrix $\rho$ defines a unique probability measure $p$
on $\Omega_n$, by using the diagonal elements: $p(i)=\rho_{ii},\;1\leq 
i\leq n$. Clearly, a probability $p$ can define a density matrix by the
same formula, but there are other, non-diagonal density matrices giving
the same $p$. If all the observables are contained in ${\cal C}$, then
the off-diagonal elements of the density matrix are of no relevance,
and all the information on the state of the system is contained in $p$.
A concept that captures the essentials of this idea, removing redundant
description, is due to Segal. Given the algebra of observables, ${\cal A}$,
we say a {\em state} on ${\cal A}$ is a positive, normalised linear map
$\rho:{\cal A}\rightarrow{\bf C}$. Thus
\begin{enumerate}
\item $\rho$ is complex linear;
\item $\rho(I)=1$;
\item $\rho(A^*A)\geq0$ for all $A\in{\cal A}$.
\end{enumerate}
Naturally, two constructions that lead to the same map are said to define the
same state. We should note that we only need the expectations, i.e. the first
moments, of the observables, because ${\cal A}$ itself contains all
powers of $A$, and (as it is complete), also elements such as $e^{iA}$; so
if we know the state we know the characteristic function of every
observable, and so its distribution too.

More generally, the classical
measure theory $(\Omega,{\cal B},\mu)$, where $\mu$ is a positive
measure, can be written as a (commutative)
quantum theory by using the von Neumann algebra $L^\infty(\Omega,\mu)$
acting as multiplication operators on $L^2(\Omega,{\cal B},\mu)$;
its normal states correspond to (countably additive) probability
measures, which vanish on $\mu$-null sets. Indeed, given a state $\rho$
we can define the corresponding measure of a set $B\in{\cal B}$ as
$\rho(\chi_{_B})$. In this, sets $B$ and $B^\prime$ are indistinguishable
if the differ by a $\mu$-null set; we do not really need $\Omega$ itself,
but only the $\sigma$-tribe ${\cal B}$, modulo this equivalence.

The set of states of a $C^*$-algebra ${\cal A}$ forms a convex set, which
we shall call $\Sigma({\cal A})$ or just $\Sigma$. The convex sum
\begin{equation}
\rho=\lambda\rho_1+(1-\lambda)\rho_2,\hspace{.5in}\mbox{ where }0
<\lambda<1
\label{mixed}
\end{equation}
represents the random mixing of the states $\rho_1$ and $\rho_2$ with
weights $\lambda$ and $1-\lambda$. All expectations in the state
$\rho$ are then the same mixtures of the expectations in the states
$\rho_1$ and $\rho_2$. If $\rho_1\neq\rho_2$ we say that $\rho$ is a
{\em mixed state}. If $\rho$ cannot be written as a mixed state (so that in
any relation such as eq.~(\ref{mixed}) we must have $\rho_1=\rho_2$), the
we say that $\rho$ is a {\em pure state}. Every $C^*$-algebra possesses
many pure states. For the full matrix algebra ${\bf M}_n$, every
pure state is given by a unit ray $\{\psi\}$ in the Hilbert space
${\bf C}^n$, using the usual quantum-mechanical expression;
every density operator is a mixture of such.
This is an example of the Krein-Milman theorem, which says that a
weak$^*$-compact convex set in the dual of a Banach space is generated
by its extreme points. The representation of a mixed state as 
eq.~(\ref{mixed}) is, in general, not unique. For example, if ${\cal
H}={\bf C}^2$, the fully unpolarised state is $(1/2)I$, and this the
equal mixture of the pure states, the eigen-vectors of $J_3$, the spin
operator in the direction of quantisation, as well as the equal mixture
of the eigenstates of $J_1$, or any other spin direction.
This means that all statistical properties of the observables are the
same however the state was made up. We express this by saying that the
state-space in quantum probability is in general not a simplex: in a
simplex, any mixed state has only one decomposition into pure states. In
classical probability, in contrast, the state space $\Sigma(\Omega)$ is a
simplex. This is true in quantum probability only if ${\cal A}$ is
abelian. The density matrix contains all the information there is. Our
inability to
distinguish the history of how the state was made is due to the quantum
phenomenon of {\em coherent} sums of wave-functions.

There is an important connection between states and representations
of a $C^*$-algebra. A {\em representation} $\pi$ of ${\cal A}$ is
a $^*$-homomorphism from ${\cal A}$ into ${\cal B}({\cal H})$
for some Hilbert space ${\cal H}$. Thus, $\pi(A)$ is an operator on
${\cal H}$ and the map $\pi$ satisfies, for all $A,B\in{\cal A}$,
\begin{enumerate}
\item $\pi(\lambda A+B)=\lambda\pi(A)+\pi(B),\hspace{.3in}$ for all $\lambda
\in{\bf C}$ (linearity);
\item $\pi(A^*)=(\pi(A))^*\hspace{.3in}$(hermiticity).
\end{enumerate}
A representation is said to be faithful if $\pi(A)$ is
non-zero if $A\neq 0$. A state $\rho$ is said to be faithful if $\rho
(A^*A)=0$ only for $A=0$. To each state $\rho$ there is a representation
$\pi_\rho$, on a Hilbert space ${\cal H}_\rho$, and a unit vector $\psi_\rho
\in{\cal H}_\rho$, such that $\rho$ is vector state $\psi_\rho$; that is,
\begin{equation}
\rho(A)=\langle\psi_\rho,\pi_\rho(A)\psi_\rho\rangle,\;A\in{\cal A}.
\label{GNS}
\end{equation}
If the state $\rho$ is faithful, then so is the corresponding representation
$\pi_\rho$. Moreover, $\pi$ is irreducible if and only if $\rho$ is pure.

The proof of this theorem, which asserts the existence of ${\cal H}_\rho$
and the homomorphism $\pi_\rho$, follows the common mathematical trick:
we construct these objects out of the material at hand. Let us do this
when ${\cal A}$ has an identity and $\rho$ is faithful. We start with
the vector space ${\cal A}$ and provide it with the scalar product
\[\langle A,B\rangle:=\rho(A^*B).\]
The completion of this space is then taken to be ${\cal H}_\rho$. The
operator $\pi_\rho(A)$ is taken to be left-multiplication of ${\cal A}$
by $A$, thus: $\pi_\rho(A)B:=AB$. This defines $\pi_\rho(A)$ on the
dense set ${\cal A}\subseteq{\cal H}_\rho$, and can be shown to be bounded.
We take $\psi_\rho=I$, the identity in the algebra. One can then verify
that $({\cal H}_\rho,\pi_\rho,\psi_\rho)$ satisfy eq.~(\ref{GNS}).
A slightly more elaborate construction can be given if there is no identity
or the state is not faithful.
This realisation of the algebra is called the {\em GNS} construction, based on
$\rho$.

It took some time before it was understood that quantum theory is a
generalisation of probability, rather than a modification of the laws of
mechanics. This was not helped by the term quantum {\em mechanics}; more,
the Copenhagen interpretation is given in terms of probability, meaning
as understood at the time. Bohr has said \cite{Schilpp} that the
interpretation of microscopic measurements must be done in classical
terms, because the measuring instruments are large, and are therefore
described by classical laws. It is true, that the springs and cogs making
up a measuring instrument themselves obey classical laws; but this does not
mean that the {\em information} held on the instrument, in the numbers
indicated by the dials, obey classical statistics. If the instrument
faithfully measures an atomic observable, then the numbers indicated by the
dials should be analysed by quantum probability, however large the
instrument is.

We now present Gelfand's theorem, which shows that any
commutative quantum theory can be viewed as a classical probability
theory. We give a proof in finite dimensions.
\begin{theorem}
Given a commutative $^*$-algebra ${\cal C}$ of finite dimension, there
exists a (finite) space $\Omega$ and an algebraic $^*$-isomorphism $J$ from
${\cal C}$ onto ${\cal A}(\Omega)$, such that for any state $\rho$ on
${\cal C}$ there exists a probability $p$ on $\Omega$, such that for any
element $A\in{\cal C}$ we have
\begin{equation}
\rho(A)=E_p[J(A)].
\label{av}
\end{equation}
\end{theorem}
Proof\\
Since dim$\,{\cal C}=n<\infty$, the dimension of the dual space is the same.
There is a faithful state $\omega$ on ${\cal C}$; this could be for example
a mixture of a basis of the state-space with non-zero coefficients.
We can therefore construct a faithful realisation of ${\cal C}$ as a matrix
algebra. In this, the {\em GNS} construction,
the Hilbert space is built out of ${\cal C}$ and so is of
dimension $n$. A commutative collection of normal matrices can be simultaneously
diagonalised, so there is a basis in the Hilbert space such that each
element of ${\cal C}$ is a diagonal $n\times n$ matrix. Since exactly
$n$ of these diagonal matrices make up a linearly independent set, every
diagonal matrix appears. Every element of ${\cal C}$ is therefore a sum
of multiples of {\em units} $\{e_j\}$ of the algebra, satisfying $e_j^2=e_j$ and
$e_ie_j=0,\;i\neq j$. In the above matrix realisation, $e_j$ is the matrix
with $1$  on the diagonal in position $j$, and zero elsewhere.
Thus $A=\sum a_je_j$. So let $\Omega=\{e_j\}_{j=1,\ldots n}$, and let $JA$
be the function $JA(e_j)=a_j$. Then one verifies that $J$ is an algebraic
$^*$-isomorphism. To the state $\rho$ we associate the probability
$p(e_j)=\rho(e_j)$, and see easily that eq.~(\ref{av})
holds.\hspace{\fill}$\Box$

In this proof, instead of identifying $\Omega$ with the collection of
elements $e_j$ in the algebra, we could have taken the dual, and
identified $\Omega$
with the set of {\em characters} on ${\cal C}$. This is the set of
multiplicative states, that is, states $\omega$ obeying $\omega(AB)
=\omega(A)\omega(B)$ for all $A,B\in{\cal C}$. The set of
characters of a $C^*$-algebra is called
its {\em spectrum}. Our proof shows that there are exactly $n$ of these, 
defined by $\omega_j(e_k)=\delta_{jk}$. Putting $A=B$ we see that any
character is dispersion-free. This is why the spectrum is taken by Gelfand
to be the definition of $\Omega$ in the infinite-dimensional case:
\begin{theorem}
Let ${\cal C}$ be a commutative $C^*$-algebra with identity. Then the set of
characters can be given a topology so as to form
a compact Hausdorff space $\Omega$ such that ${\cal C}$ is
$C^*$-isomorphic to $C(\Omega)$, and every state on ${\cal C}$ corresponds
to a finitely additive measure on $\Omega$ (with the Borel tribe).
\end{theorem}
 
Bohm asked whether the observed statistics, agreeing with
experiment, can be obtained from a larger, more complicated classical
theory. This is the idea behind the attempts to introduce hidden
variables. This is certainly true of the statistics of any fixed complete
commuting set of observables; for they form an abelian algebra, and so can
be represented by the classical statistics of multiplication operators
on a sample space (the spectrum of the algebra). Obviously
the full non-abelian algebra cannot be a subalgebra of an abelian algebra,
so the way hidden variables are introduced must be more elaborate
than extending the algebra by adding them. However, the deep
result of J. S. Bell shows (if the dimension is 4 or higher) that the
full set of statistics predicted by quantum theory cannot be got
from {\em any} underlying classical theory. In the quantum model
of two spin-half systems, Bell constructs a sum of four correlations
which in a certain state is equal to $2\surd 2$, a factor $\surd 2$
larger than the greatest value allowed in any classical theory.

Let us follow \cite{Landau,Kochen}. Let $P,Q$
be complementary projections, and also let $P^\prime,Q^\prime$ be
complementary projections, while $P$ is compatible with $P^\prime$
and with $Q^\prime$, and $Q$ is compatible with $P^\prime$ and
$Q^\prime$. Define $a=2P-I$, $b=2Q-I$, and similarly for $a^\prime$
and $b^\prime$. For any state $\rho$ define $R$ by
\[R:=\rho(aa^\prime+ab^\prime+bb^\prime-ba^\prime)=\rho(C)\]
where $C=a(a^\prime+b^\prime)+b(b^\prime-a^\prime)$. Then $a^2=b^2=a^{\prime
2}=b^{\prime 2}=1$, so
\begin{equation}
C^2=4+[a,b][a^\prime,b^\prime]=4+16[P,Q][P^\prime,Q^\prime].
\label{Bell}
\end{equation}
Since $\|a\|=\|b\|=\|a^\prime\|=\|b^\prime\|=1$, it follows that
\[\|[a,b][a^\prime,b^\prime]\|\leq 4,\]
so $C^2\leq 8$ and $|R|^2=|\rho(C)|^2\leq\rho(C^2)\leq 8$. So in quantum
theory, $|R|\leq 2\surd 2$.
If there is a joint probability space on which we can describe $a,\ldots,
b^\prime$ by the r. v. $f,\ldots g^\prime$ taking the values
$\pm 1$, and a measure $p$ on it, then $R=E_p[h]$ where
\[h=f(f^\prime+g^\prime)
+g(g^\prime-f^\prime).\]
Then these r. v. commute, so eq.~(\ref{Bell}) becomes $h^2=4$, and
\[|R^2|=E_p[h]^2\leq E_p[h^2]=4.\]
So $|R|\leq 2$, (Bell's inequality). Bell showed that the entangled
states of the Bohm-EPR set-up give a $\rho$ such that $R=2\surd 2$,
violating this. Thus no description by classical probability is possible.

The famous Aspect experiment tested Bell's inequalities.
This involves observing a system (in a pure entangled state) in a long run
of measurements; the correlations singled out by Bell,
between several compatible pairs of spin observables, were measured.
The experiments showed that $R$ was just less than $2\surd 2$, in
agreement with the quantum predictions.


The upshot is that in quantum probability there is no sample space;
we have the $C^*$-algebra ${\cal A}$, and this plays the r\^{o}le of
the space of
bounded functions.

Let us now examine Bohm's claim that there is a hidden
assumption in Bell's proof, that of `locality'. It is now generally
agreed that the term `local', referring to the space-localisation,
is not the best, and that `non-contextual' is a better term; namely
that the choice of random variable $f$ assigned to represent a certain
observable $a$ which is being measured, does not depend on which of the
other observables, $a^\prime$ or $b^\prime$, is being measured at the
same time. This is now called a non-contextual assignment.
Bohm suggested that we should allow a contextual choice of
assignment of random variable, so that the r.v. representing the observable
$a$ when $a^\prime$ is also measured is not the same as the choice of r.v.
for $a$ when it is measured with $b^\prime$.
The two choices will, however, have the
same distribution. It should be said straight away that this idea is
contrary to the practice of probabilists, who would expect there to be
a unique random variable representing an observable. It also goes against
the definition of `element of reality' of {\em EPR} as extended by us
to the random case. The quantum version does not suffer from this unreality,
since the mathematical object assigned to the observable, the Hermitian
matrix, does not depend on the context, i.e. is local in Bohm's language.

Bohm's idea leads to a theory with very few rules.
However there are some restrictions, since the choice must
be done so that {\em all}
statistical measurements of compatible observables (means, correlations,
third moments etc) of the
model can be arranged to give the same answers as the quantum theory.
This is achieved as follows. Let $a,a^\prime$
be compatible, generating a commutative
$C^*$-algebra, ${\cal C}$ and let $\rho$ be a state on the full algebra
${\cal A}$. By restriction, $\rho$ defines a state on ${\cal C}$.
By Gelfand's isomorphism,
we can construct a space $\Omega$, the spectrum
of ${\cal C}$, and a measure $\mu$ on it, such that $a,a^\prime$ can be
represented as multiplication operators on ${\cal C}(\Omega)$, so
they are random variables, $f,g$. The joint probability
distribution of $f,g$ is the same as that of the (diagonal) matrices
$a,a^\prime$ in the state $\rho$.
On the other hand, $\Omega,\mu$ depends on the  set $a,a^\prime$. Let
us record this by denoting this Gelfand representation by $\Omega_{a,
a^\prime},\mu_{a,a^\prime}$. If we
measure $a$ and $b^\prime$, and proceed as Bohm suggests, then we get a
different space $\Omega_{a,b^\prime}$, the spectrum of a different algebra
${\cal C}_{a,b^\prime}$, say. The state $\rho$ leads to a different measure
$\mu_{a,b^\prime}$.
The r.v. assigned to $a$ cannot be $f$ this time;
it must a function on $\Omega_{a,b^\prime}$, a different space;
it has the same distribution in $\mu_{a,b^\prime}$ as the $f$ had in
$\mu_{a,a^\prime}$. In this set-up, there is no obvious definition of
$a^\prime+b^\prime$, as they are not functions on the same space.
This problem does not arise in the quantum formulation:
there is an underlying $C^*$-algebra, in which we can add the operators.

Bohm's suggestion might be said to be an
interpretation of quantum mechanics in terms of classical probability
\cite{Garden}.
However, the construction is not a probability theory in the sense
of Kolmogorov, as there is no single sample space; the theory is
preKolmogorovian, in the tradition of the frequentist school.
One can generalise the frequentist point of view, and specify
that certain collections of observations are
compatible, and others are not; then we can by observation construct
the joint probabilities of each compatible set, and have no need of
the sample space (the space of joint values). A different compatible set
need have no analytic relation to the first, even though it contain
common observables. Bell's inequality need not hold, but then
neither need the quantum version, which is $\surd 2$ times
more generous. It is a feeble theory, not much more that data
collection, and has no predictive power. Mere data give us no
more than mere data.

Another variant of quantum mechanics, a new form of algebra
called `quantum logic', was developed in
\cite{Birkhoff2}. New rules by which propositions can be
manipulated are given. This was worked on later by Jauch and coworkers \cite{Jauch},
culminating in Piron's thesis. This says that the propositions
form a lattice isomorphic to the
lattice of subspaces in a Hilbert space (but not necessarily over the
complex field). Apart from this result, quantum logic has not been very
successful, and it is more productive to keep to
classical logic, but to generalise the concept of probability algebra
from commutative to non-commutative. Another alternative to
quantum probability is stochastic
mechanics, founded but now abandonned by Nelson \cite{Nelson4}
as not being correct physics. Thus Segal's approach is the one we adopt here.
It is well explained in \cite{Emch,Haag,Horuzhy}.

Quantum theory has its version of estimation theory \cite{Holevo,Ohya}.
In finite dimensions, the method of maximum likelihood is to find the
density
matrix $\rho$ that maximises the entropy, subject to given values for
the means, $\{\eta_i\}$ of observables in the subspace of hermitian operators
spanned by a named list $\{X_1,\ldots,X_n\}$ of {\em slow variables}.
So $\eta_i=\mbox{Tr}[\rho X_i]$. It is well known
that the answer is the Gibbs state
\begin{equation}
\rho=Z^{-1}\exp(-H)=Z^{-1}\exp-[\xi^1X_1+\ldots+\xi^nX_n],\hspace{.2in}
Z=\mbox{Tr}[\exp(-H)].
\end{equation}
Again, $\log Z$ is strictly convex, and its Hessian gives a Riemannian
metric on the manifold ${\cal M}$ of all faithful density operators
\cite{Ingarden4,Chentsov,Petz2}. In this case we get the
Kubo-Mori-Bogoliubov metric; in terms of the centred variables
$\hat{X}_i:=X_i-\eta_i$,
the metric is
\begin{equation}
g(\hat{X}_i,\hat{X}_j)=\mbox{Tr}\left[\int_0^1\rho^\lambda\hat{X}_i\rho^
{1-\lambda}\hat{X}_jd\lambda\right]
\end{equation}
This is the closest point on ${\cal M}$ to any state with the given
means, where `distance' is measured by the relative entropy $S(\rho|
\rho^\prime):=\mbox{Tr}\,\rho[\log\rho-\log\rho]$.
Again, the $\xi^j$ are uniquely determined by
the measured means $\eta_i$.

\input new20003
\newpage

