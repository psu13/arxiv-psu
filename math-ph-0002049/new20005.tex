\section{Quantum Processes}
Is friction a classical concept? 

`There is no
friction in quantum systems: the ground state of the atom does not grind
to a halt. The introduction of friction, e. g. the term
$-\gamma\dot{x}$ in Newton's laws, is to account for atomic phenomena
such as radiation of moving charges, in a very crude way. Such effects
are treated exactly in quantum mechanics, and therefore frictional terms
do not appear'. 

The view is still widespread but not universal among
physicists. Friction does not appear in classical mechanics either if
it is not put in.

A quantum process is, in a general way, a Hilbert space ${\cal H}$
and a family of self-adjoint operators $\{A(t)\}_{t\geq 0}$ on ${\cal H}$.
A quantum field used as {\em noise} appeared in
\cite{Senitzky}. Senitzky obtained
the approximate dynamics of a quantum oscillator by reduction
from the dynamics of a larger conservative system. He arrived at the
following quantum Langevin equation with a Gaussian positive-energy
quantum driving term $(\varphi(t),\pi(t))$ (the noise):
\begin{equation}
\frac{dQ(t)}{dt}=\omega P(t)-\gamma Q(t)+\varphi(t)\hspace{.3in}
\frac{dP(t)}{dt}=-\omega Q(t)-\gamma P(t)+\pi(t).
\end{equation}
He noticed that without the `noise', the Heisenberg commutation
relations fade with time: $[Q(t),P(t)]=ie^{-2\gamma t}$; he considered
this to be inconsistent with quantum mechanics. With the noise, the
solutions obey $[Q(t),P(t)]\approx i$ for all time. The noise
was a free quantum field with constant energy spectrum from $0$ to
$\infty$. This does not quite satisfy the requirement that the Heisenberg
cummutation relations should hold for all time. In \cite{RFS5}
we found the general exact solution to this problem.  A special case is
\[\varphi(t)=2^{-1/2}(a(t)+a^*(t)),\hspace{.4in}\pi(t)=i2^{-1/2}
(a(t)-a^*(t)),\]
where
\[a(t)=(2\gamma/\pi)^{1/2}\int_\omega^\infty e^{-ikt}a(k)\,dk,
\hspace{.3in}[a(k),a^*(k^\prime)]=\delta(k-k^\prime).\]
This has a constant energy spectrum from $\omega$ to $\infty$.
The feature of this solution, and Senitzky's approximate solution,
is the relationship between the dissipation $\gamma$ and the correlation
of the quantum noise, which at zero temperature is 
\[ \langle a(s)a^*(t)\rangle=\frac{2\gamma}{\pi}e^{i\omega(t-s)}\frac{1}{
t-s+i\epsilon}.\]
This is called the fluctuation-dissipation theorem.


Lax \cite{Lax} used noise with all frequencies, with two-point function
\[ \langle a(s)a^*(t)\rangle=\frac{\gamma}{\pi}\delta(t-s).\]
This is closer to the classical white noise, in that the increments to
the process are independent, and the field obeys a quantum version of
the Markov property. It was to be used
later by Hudson and Parthasarathy in a rigorous body of theory
\cite{Hudson3,Partha2}.
As physics, it was criticised by Kubo and others, as violating the
{\em KMS} condition, which comes from the
axiom of positive energy \cite{Haag}. The correct treatment (at
non-zero temperature) was obtained by Ford et al., \cite{Ford} by taking
the limit of one oscillator
coupled to a large system of oscillators (or a string \cite{Lewishb}).
This was truly the quantum Langevin equation, in that the noise is added
only to the equation for $P$ and not to $Q$. This can also be obtained
\cite{HLK} as a singular limit of the asymmetric solution given in
\cite{RFS5}. The quantum noises in \cite{Ford,RFS5} are not martingales,
and have not got independent increments. They do fit in to the axiomatic
scheme offered in \cite{Accardi}. In \cite{Ford2},
Ford emphasizes the role played by causality. Instead of eq.~(\ref{Langevin}),
he considers the equation with memory
\begin{equation}
m\stackrel{..}{x}+\int_{-\infty}^t \mu(t-s)\dot{x}(s)\,ds+V^\prime(x)=F(t).
\end{equation}
The fact that the dissipation due to the future must be zero leads us
to consider only those $\mu$ which vanish for negative argument.
Perhaps this is a lesson for
those \cite{ArakiW,Streater2,RFS3,HIK,Hudson3,Partha2} who like to work
on Lax's version. 

The first work to use the words `continuous
tensor products' ({\em CPT}) was \cite{ArakiW}. The notable conclusion was that the
theory can always be embedded in a boson Fock space; the Wiener chaos is an
example of this. We start with a definition of current algebra, or
better, current group. Let $G$ be a Lie group, with Lie algebra ${\cal G}$,
and denote by ${\cal D}(G)$ the set
of $C^\infty$-maps from ${\bf R}^n$ into $G$, being the identity outside
a compact set. We can furnish ${\cal D}(G)$, the current group,
with a group law by pointwise
multiplication: $fg(x):=f(x)g(x)$. This group has a Lie algebra, denoted
${\cal D}({\cal G})$, which is the set of
all $C^\infty$-maps $F:{\bf R}^n\rightarrow {\cal G}$, of compact support,
under the pointwise bracket
\cite{RFS3}
\[ [F(f),G(g)]:=[F,G](fg).\]
The problem is to find representations of the current groups and
the current algebras, by unitary or self-adjoint operators respectively.

Guichardet \cite{Guichardet} proposed a construction for the
tensor product of Banach spaces or algebras, labelled by a continuous index.
The first thing is to define, if possible, the continuous product of
$f(x)$ over $x\in{\bf R}^n$, when $f$ has compact support. He tries
\begin{equation}
\prod_xf(x):=\exp\left(\int\log f(x)\,dx\right).
\label{prod}
\end{equation}
For Hilbert spaces, we wish to define the scalar product between two
fields of vectors $\psi(x)$ and $\phi(x)$. We put $f(x)=\langle\psi(x),
\phi(x)\rangle$ and use eq.~(\ref{prod}), provided that $f(x)=1$ outside
a compact set and we take $\log 1=0$ (the principal branch).
We then need to be
able to extend the scalar product to linear combinations of product vectors.
In \cite{Dubin}, we
give an example of a non-existent Hilbert continuous product, in that the
positivity fails on linear combinations.
Guichardet presents a class of Hilbert spaces for which the construction
works, and writes the Fock representation of the free field in
these terms. 
To explain his  examples, let ${\cal H}$ be a Hilbert space, and
$\Gamma({\cal H})$ the Fock space over ${\cal H}$. We define the map
$\exp{\cal H}\rightarrow\Gamma({\cal H})$ by
\begin{equation}
\exp\phi:=1\oplus\phi\oplus 2^{-1/2}\phi\otimes\phi\oplus
\ldots\oplus(n!)^{-1/2}\otimes^n\phi\ldots
\label{coherent}
\end{equation}
The $\exp\phi\in\Gamma({\cal H})$ is called the coherent state determined by
the one-particle state $\phi$. One shows that they form a total set (their
span is dense) in $\Gamma({\cal H})$; clearly, 
\begin{equation}
\langle\exp\phi,\exp\psi\rangle=\exp\langle\phi,\psi\rangle.
\label{exp}
\end{equation}
In \cite{Guichardet},
the Hilbert spaces ${\cal H}_x$ at each point is itself the Fock space
$\Gamma(H)$ of a Hilbert space $H$, and the family ${\cal I}$ consists
of coherent states at each point. This is a special case of the construction
given below.

The case of current groups was treated in \cite{RFS2,ArakiW}.
We give here a special case when the continuous label is ${\bf R}$,
interpreted as time; we start with $({\cal H},U,\psi)$, where
${\cal H}$ is a Hilbert space, $\psi\in{\cal H}$, and $U$ is a representation
of $G$ on ${\cal H}$ such $\{U(g)\psi,g\in G\}$ has dense span.
The triple $({\cal H},U,\psi)$ is called a cyclic representation of $G$.

We say that $({\cal H}_1,U_1,\psi_1)$ and $({\cal H}_1,U_2,\psi_2)$
are {\em cyclic equivalent} if there exists a unitary isomorphism
$W:{\cal H}_1\rightarrow{\cal H}_2$ such that for all $g\in G$,
\[WU_1(g)W^{-1}=U_2(g);\hspace{.5in}W\psi_1=\psi_2.\]
A cyclic representation gives us a function on the group, analogous to the
characteristic function of a random variable. Indeed, it reduces to
the characteristic function when the group is ${\bf R}$. Thus
\begin{equation}
C(g):=\langle\psi,U(g)\psi\rangle.
\label{characteristic}
\end{equation}
Let Span$\,G$ denote the complex vector space of finite formal sums of
elements of $G$. Then $C$ is continuous and of positive type on Span$\,G$,
which determines $({\cal H}, U,\psi)$ up to cyclic equivalence.
Conversely, a continuous function $C$ of positive type on $G$
determines a cyclic representation $({\cal H},\pi,\psi)$ related to $C$ by
eq.~(\ref{characteristic}). The construction is very similar to the proof
of the {\em GNS} representation. First, we construct the vector space,
Span$\,G$, and furnish it with the scalar product, determined by its values
on the linearly independent elements $g_1,g_2,\ldots$, by
\[\langle g_i,g_j\rangle=C(g_i^{-1}g_j);\]
we complete Span$\,G$ in the norm (or, if a semi-norm with kernel $K$,
we complete the quotient Span$\,G/K$), giving the space ${\cal H}$.
Then we choose $\psi$ to be
the identity of the group. The operator $U(g)$ can be defined on
Span$\,G$ as left multiplication; this is easily shown to be unitary,
and so can be extended to the whole space to get the representation $U$
of $G$.

In an infinite tensor product over a discrete index, von Neumann was able
to end up with a separable Hilbert space only by labelling a special
vector, say $\psi_x$ in each factor ${\cal H}_x$, and then considering
products
$\otimes\phi_x$ of vectors in a subset $\Delta$ that at infinity are close
to $\psi_x$. Only then
does the infinite product $\prod\langle \phi_x,\psi_x\rangle$ converge.
The tensor product then carries the labels
$\{\psi(x),\Delta\}$. Guichardet used a similar idea for the continuous
product. We are less ambitious, in that
we ask for the tensor product of a cyclic representation $({\cal H},
U,\psi)$ of a group. We use the same representation at each point of
the time axis, because we want to get a stationary quantum
process. We then define the function $C:{\cal D}(G)
\rightarrow{\bf C}$ as
\begin{equation}
C(g(\;.\;):=\prod_x\langle\psi,U(g(x))\psi\rangle,
\label{77}
\end{equation}
which is well defined if we choose at each $x$ one branch of the logarithm.
To get a representation of the current group, it is necessary and sufficient
that this be of positive type on the current group, in which case we say that
the {\em CTP} exists.
We also want the function to be extendable to step functions, constant
in an interval $[s,t]$ and the identity outside. For such a $g(\;.\;)$,
we divide an interval $[s,t]$ into an arbitrary number, $N$, of equal
intervals; then $C(g)$ is the product of $N$
equal factors, each a characteristic function on $G$. Thus $C$
has the property
that it has an $N^{\rm th}$ root that is also a characteristic function.
Such $C$ is called {\em infinitely divisible}. By the relation of
characteristic functions to cyclic representations, we are able to
transfer the concept of $\infty$-divisibility to cyclic representations:
\begin{definition}
Let $({\cal H},U,\psi)$ be a cyclic representation of a group $G$. We
say \cite{RFS2} that it is $\infty$-divisible if, for any integer $N>0$,
there is another cyclic representation $({\cal H}^{1/n},U^{1/n},\psi^{1/n})$,
called the $n^{\rm th}$-root, such
that $({\cal H},U,\psi)$ is cyclically equivalent to
\[\left(\otimes{\cal H}^{1/n},\otimes U^{1/n},\otimes\psi^{1/n}\right)\]
where the tensor product is over $N$ factors, and the resulting
representation is restricted to the cyclic subspace spanned by the group
acting on the product vector $\otimes\psi^{1/n}$.
\end{definition}
We see immediately that if for some $n$ the $n^{\rm th}$ root of the
representation exists, then it is unique (up to cyclic equivalence).
For, the characteristic function of two $n^{\rm th}$-roots, $C_1,C_2$ say,
both satisfy $C_i^n=C$, and so their ratio is $\omega_n$, an
$n^{\rm th}$-root of unity. But this violates positivity unless $\omega_n=1$.
The converse also holds: if $C$ is the product of $n$ functions of positive
type, then $C$ itself is of positive type.
In \cite{RFS2} we assumed that $C(g)$ never vanishes; we prove this later.

Following \cite{RFS2} we can now give the criterion for the positivity
of the scalar product in a continuous tensor product $\otimes^{\psi,\Delta}
{\cal H}_x$ of cyclic group representations,
relative to the cyclic vector $\psi$ and the set of states
$\Delta:=\{U(g)\psi:g\in G\}$.
\begin{theorem}

The following are equivalent.
\begin{enumerate}
\item The function $C(g)$
is a continuous function of positive type on $G$ with
$C(e)=1$, and is $\infty$-divisible.
\item There exists an $\infty$-divisible cyclic representation
$({\cal H},U,\psi)$ of $G$ such that $C(g)=\langle\psi,U(g)\psi\rangle$.
\item $\bigotimes^{\psi,\Delta}$ exists.
\item $C(e)=1$ and a branch of $\log C(g)$ is a conditionally positive
function on $G$.
\end{enumerate}
\end{theorem}
In (3) and (4) the branch of the logarithm is determined by which root
of $C$ is of positive type.
Only the item (4) needs explanation. A function $F(g)$ on a group is said to be
conditionally positive if
\[\sum_{ij}\overline{z}_iz_jF(g_i^{-1}g_j)\geq 0\]
for all $n$-tuples $(g_1,\ldots,g_n)$ of group elements and all
complex $n$-tuples\\
$(z_1,\ldots,z_n)$ summing to zero: $\sum_iz_i=0$.

To sketch the proof, if
$C$ is $\infty$-divisible, and $C=e^F$, then $C^s$ is also of
positive type, for all small $s>0$. Then 
\begin{equation}
\sum_{ij}\overline{z}_iz_j(1+sF_{ij}+\ldots)\geq0,
\label{conditional}
\end{equation}
and so if $\sum_iz_i=0$, we get
that $F$ is conditionally positive semidefinite. For the converse,
if $F$ is conditionally positive definite, then $e^F$ is of
positive type for all $s>0$, see \cite{GelfandV}, page 280.

The following result is called an Araki-Woods
embedding theorem \cite{RFS2}, because of the similarity with
\cite{ArakiW}, (but with different hypotheses).
We remark that under the above
conditions $F$ is conditionally positive semidefinite; then the function
\begin{equation}
\langle g,h\rangle:=F(g^{-1}h)-F(g)-F(h^{-1})
\label{condition2}
\end{equation}
is of positive type, and so can be used to define a
semi-definite form on Span$\,G$ by sesquilinearity.

Let ${\cal K}$ be the (separated, completed) Hilbert space formed using this
as scalar product on Span$\,G$. Let $G_0$ be the subgroup of $G$ such that
$U(g)\psi=e^{i\lambda}\psi$ for some real $\lambda$. We see that
$\langle g,h\rangle$ vanishes on Span$\,G_0$, and
defines a scalar product
on $\mbox{Span}\,G/(\mbox{Span}\,G_0)$, (perhaps after
identifying vectors of zero norm with zero).
We then complete this to give a Hilbert space, ${\cal K}$. We see that the
equivalence class of the identity $e\in G$ is the zero vector
in ${\cal K}$. The original
cyclic representation $({\cal H},U,\psi)$ can then be embedded in the Fock
space over ${\cal K}$, as follows: define the map $W$ from ${\cal H}$ to
$\Gamma({\cal K})$ by its action on the total set $U(G)\psi$:
\begin{equation}
W(U(g)\psi)=C(g)\exp[g],\,g\in G.
\end{equation}
One easily sees that this preserves the scalar product, using
(\ref{condition2}). Thus it can be extended by linearity and continuity
to ${\cal H}$. We see that the cyclic vector $\psi$ is mapped to the
`vacuum' vector $\psi_0$ of the Fock space. As for the group
action, we use the fact that $G/G_0$ is a $g$-space, with left multiplication
$\tau_g[h]=[gh]$. This defines an action $\exp\{\tau_g\}$ on the Fock space
as usual, by its actions on the coherent vectors:
\[\exp\{\tau_g\}\exp[h]:=\exp[gh] .\]
Define an operator $U^\prime$ closely related to $\exp\{\tau_g\}$:
\begin{equation}
U^\prime(g)C(h)\exp[h]:=C(gh)\exp[gh]
\label{embed}
\end{equation}
Then by calculation one sees that $({\cal H},U,\psi)$ is cyclically
equivalent to the cyclic subspace of $({\cal K},U^\prime,\psi_0)$; $W$
intertwines $U$ and $U^\prime$ and maps $\psi$ to $\psi_0=\exp[e]$.
From the unitarity of $U^\prime$ we see that
$|C(g)|^2=e^{-\langle[g],[g]\rangle}\neq0$.

The Gaussian measure is $\infty$-divisible, and the representation
of the translation group, $U(\lambda)$,
with Gaussian cyclic vector $\psi(x)=(2\pi)^{-1/4}e^{-x^2/4}$, is
$\infty$-divisible. The corresponding {\em CTP} contains Brownian
motion  \S2; the continuous product $\otimes_0^tU(\lambda)$ is the
exponential martingale. A representation
of the oscillator group is $\infty$-divisible, and the {\em CTP}
of this is the free non-relativistic quantised fields \cite{RFS3}.

H. Araki independently obtained similar results \cite{Araki}. Instead of
$\infty$-divisible cyclic representations of groups, Araki started with
a factorizable representation of current algebra.
He remarked that, putting $[g]=\phi_g$
the map $V(g)\phi_h:=\phi_{gh}-\phi_g$ 
is a unitary representation of $G$; this is proved on the vectors
$\phi_h$, $\phi_k$ by use of (\ref{condition2}). The equation
expresses that the
map $g\mapsto \phi_g\in{\cal K}$ is a {\em one-cocycle} of the group, with
values in ${\cal K}$. We briefly explain this.

So, let $G$ be a group, and let ${\cal K}$ be a Hilbert space on which
$G$ acts by unitary operators $g\mapsto V(g)$.
We shall write the left action $\phi\mapsto V(g)\phi$ as left
multiplication, $\phi\mapsto g\phi$. The right action, which appears in the
general theory of group cohomology, is taken to be trivial: $\phi g=\phi$.
An $n$-{\em cochain} with values in
${\cal K}$ is a map from
$G^n$ into ${\cal K}$, that is, it is a function of $n$ group elements
with values in ${\cal K}$, thus: $\phi(g_1,\ldots,g_n)$. We shall need
only the $0$-cochains, which make up the space $C^0:={\cal K}$ of vectors
independent of $g$, and the 1-cochains, which are vector fields $\phi(g)
\in{\cal K}$ defined on the group. These make up the vector space $C^1$.
We shall also need the  $2$-co-chains, when
${\cal K}={\bf C}$; these are complex-valued functions of two group
elements. We see that the cochains of any degree $k$ form a vector space
$C^k$. Fundamental to any cohomology theory is the coboundary operator,
which is a linear map, $\delta:C^k\rightarrow C^{k+1}$,
so increasing the degree of the cochain. It obeys $\delta^2=0$.
In the case of a group $G$ and a left and right action of $G$ on ${\cal K}$,
$\delta$ is the linear map defined on $C^0$ by
\[(\delta\phi_0)(g)=g\phi_0-\phi_0g.\]
On $C^1$, $\delta$ is the linear map defined by
\[(\delta\phi_1)(g_1,g_2)=g_1\phi_1(g_2)-\phi_1(g_1g_2)_+\phi_1(g_1)g_2.\]
On $C^2$, $\delta$ is the linear map defined by
\[(\delta\phi_2)(g_1,g_2,g_3)=g_1\phi_2(g_2,g_3)-\phi_2(g_1g_2,g_3)+
\phi_2(g_1,g_2g_3)-\phi_2(g_1,g_2)g_3.\]
The vector space of cocycles of degree $k$ in a vector space ${\cal K}$, with
left and right actions $\tau_1,\tau_2$, is denoted $Z^k(G,{\cal K},\tau_1,
\tau_2)$. One checks that $\delta^2=0$. A coboundary of degree $k$
is a vector function
of the form $\delta\psi$, where $\phi$ is a cochain of degree $k-1$.
The coboundaries of degree $k$ form the vector space $B^k(G,V,
\tau_1,\tau_2)$. Since $\delta^2=0$, we see that every coboundary is a
cocycle. If the converse holds, the cohomology group $H^k:=Z^k/B^k$,
is trivial.
One sees that if $\phi$ is a one-cocycle in $C^1(G,
{\cal K},V)$, then $\langle\phi(g_1^{-1}),\phi(g_2)\rangle$ is a two-cocycle
in $C^2(G,{\bf C},I)$.

A $2$-cocycle $\sigma(g,h)$ with values in the unit circle is also
called a multiplier for the group. A multiplier representation of a group
$G$ is a map $g\mapsto U(g),\;g\in G$, such that $U(g)U(h)=\sigma(g,h)U(gh)$
for all $g,h\in G$. Although Wigner's analysis of symmetry in quantum
mechanics leads naturally to multiplier representations, their occurrence
is sometimes called an `anomaly' by physicists.
When the {\em CTP} exists, we can represent the element
$g(\;.\;)$ of the current group by the operator $(\otimes U)_g$, defined on
the product vectors $\otimes_x U(h(x))\psi_x$ by
\begin{equation}
(\otimes U)_g(\otimes_x U(h(x))\psi_x):=\otimes_xU(g(x)h(x))\psi_x,
\label{current}
\end{equation}
The space of the {\em CTP} is then $\Gamma(\int_{\oplus}\exp{\cal K}dx)$
and $\Delta$ consists of coherent states of the form $\exp\phi_{g(x)}$.
So we obtain a local representation of the current algebra. We get a
multiplier when the branch of the logarithm in (\ref{77}) obtained by the
group law differs from the one needed to give a function of positive type on
the group. This gives rise to an anomaly.

Araki showed that if $\phi$ is the cocycle defined by the $\infty$-divisible
representation $U$, then it is necessary that ${\rm Im}\langle
\phi(g_1^{-1}),\phi(g_2)\rangle$ be a coboundary. Conversely, given a
cocycle $\phi$
with this property, it comes from an $\infty$-divisible representation.
He proved that if $G$ is compact, then any cocycle is a coboundary, i. e.
of the form $\phi_g=(V(g)-I)\chi$ for some $\chi\in{\cal K}$. Use of
a coboundary leads to a {\em CTP} of the form assumed by
Guichardet \cite{Guichardet}.
Araki was able to obtain
analogues of the Levy formula (\ref{Levy}) for various groups;
for the group ${\bf R}$ this takes on
a new meaning, as the decomposition of a cocycle into its parts
coming from primitive cocycles, some algebraic and some topological.
The topological cocycles are of the form $(V(g)-I)\chi$; it is not a
coboundary because $\chi$ is not in ${\cal K}$, but lies in a larger
space that admits an extension of $V$; the $V(g)-I$ brings the
vector back into ${\cal K}$. Some groups, e. g. ${\bf R}$, also have
cocycles called algebraic by Araki. For example, in the case $G={\bf R}$,
take ${\cal K}={\bf C}$, and $V(a)=I$ for all $a\in G$. The cocycle is
$\phi(a)=a$. Then $\langle\phi_a,\phi_b\rangle=ab$ is real, and $C(\lambda)
=\exp\{-\frac{1}{2}\lambda^2\}$, the characteristic function of the Gaussian
distribution. The Poisson part of the Levy formula comes from
the coboundaries, and the Levy processes from the topological cocycles.

The question arises, given ${\cal K},V$ and a cocycle $g\mapsto \phi_g$,
can we construct a {\em CTP}? We can construct $({\cal H},
U(g),\psi)$ from $C$, which can be regarded as a function such that
$C(e)=1$ and the map $C(h)\exp\phi_h\mapsto C(gh)\exp\phi_{gh}$ is unitary.
The next big step was by
Parthasarathy and Schmidt \cite{Partha}, who showed that given a cocycle
there is indeed an $\infty$-divisible representation associated with it,
but that it is a multiplier representation, with an $\infty$-divisible
multiplier $\sigma$. The corresponding function $C(g)$ is $\sigma$-positive.
This means that
\begin{equation}
\sum_{ij}\overline{z}_iz_j\sigma(g_i^{-1},g_j)C(g_i^{-1}g_j)\geq0.
\end{equation}
Naturally, this gives to a multiplier representation of the current group
in general, and they found the multiplier; this leads to a tidier theory than
\cite{Araki}, since the condition for the absence of multiplier can be
dropped. Since the physical interpretation of a symmetry group
leads (according to Wigner\cite{Wigner}) to the ambiguity of the
induced unitary representation up to a coboundary, the projective theory
is certainly the right setting. Holevo has presented some similar
concepts at the level of the algebra of observables, and found applications
in quantum theory \cite{Holevo}. Notable in the development was the work of Gelfand, et al.
\cite{Gelfand} who used a cocycle of $SL(2,{\bf R})$ to construct a
factorisable representation of the corresponding current group.
The whole theory is well explained in \cite{Guichardet2,Erven}.

A theory of processes with independent increments
and values in a Lie algebra ${\cal G}$ was developed
in \cite{RFS4}, extended to multiplier representations by Mathon
\cite{Mathon} and to Clifford algebras in \cite{Mathon2}.
Corresponding central limit theorems were proved by Hudson, and Cushen
and Hudson \cite{Cushen,Hudson7}.
A Lie process
can be obtained by differentiation of the corresponding object for a
Lie group. For example, near the identity any group element $g$ lies on a
one-parameter subgroup generated by an $X\in{\cal G}$, and we
write (Exp means the exponential map from ${\cal G}$ to $G$, not
the Fock map)
$g(t)=\mbox{Exp}\,tX,\;g(0)=e,\;g(1)=g$; given a representation $U(g)$ we get
a representation of ${\cal G}$ by $\pi(X)=d/dt[U(g(t)]_{t=0}$. By Stone's
theorem, $X$ is self-sdjoint. However, given a cyclic vector $\psi$  for $U$
it does not follow that $\psi$ is cyclic for $\pi$, because of domain
questions. Let ${\cal E}$ be the universal enveloping algebra of ${\cal G}$.
This is the nonabelian polynomial algebra, modulo the ideal generated by
the commutators $XY-YX-[X,Y]$. Here, $[X,Y]\in{\cal G}$ is the Lie product,
a polynomial of degree 1. A cyclic representation $({\cal H},\pi,\psi)$
is determined
(up to equivalence) by a positive linear functional, or state, on ${\cal E}$:
\[X_1X_2\ldots X_n\mapsto \langle\psi,\pi(X_1)\pi(X_2)\ldots\pi(X_n)\psi\rangle
=W_n(X_1\ldots X_n).\]
These are the noncommutative moments, or Wightman functions; they determine
a representation, by the Wightman reconstruction theorem \cite{SW}.
They are generated by the characteristic function
\begin{equation}
C({\lambda})=\langle\psi,U(\mbox{Exp}\lambda_1X_1)\ldots U(\mbox{Exp}
\lambda_nX_n)\psi\rangle,\;\;\lambda\in{\bf R}^n.
\label{Hegerfeldt}
\end{equation}
Here, $\{X_j\}$ is a basis of the Lie algebra, and any moment out of order
is determined by a derivative of $C$ and use of the commutation relations.
The truncated functions $W_T$ are generated by $\log C$, \cite{Hegerfeldt2}
and are related to
$W$ by a formula similar to eq.~(\ref{cumulants}), relating cumulants to the
moments. Two cyclic representations with the same $W$, or the same $W_T$,
are cyclic equivalent. The cumulants of $\exp U$ (the Fock construction)
are the same as the moments of $U$; this follows from $\exp U(g)\exp U(h)
\psi)=\exp(U(gh)\psi)$ and (\ref{Hegerfeldt}). 

Given two representations $U_1$, $U_2$ of $G$, their tensor product
$U_1\otimes U_2$, restricted to the diagonal subgroup of $G\times G$, gives
the representation $\pi_1\otimes I+I\otimes\pi_2$ of ${\cal G}$. This led to
the use of a coproduct, though it was not recognised as such until
\cite{Schurmann}. Whereas a product on an algebra ${\cal A}$ is a linear
map ${\cal A}\otimes{\cal A}\rightarrow{\cal A}$, a coproduct is a map
${\cal A}\rightarrow{\cal A}\otimes{\cal A}$.
For Lie algebras the coproduct is $X\mapsto X\otimes I
+I\otimes X$. Then we say that a cyclic representation $({\cal H},\pi,\psi)$
is $\infty$-divisible if for each $N$ there is another, $({\cal H}^{1/N},
\pi^{1/n},\psi^{1/N})$ such that $({\cal H},\pi,\psi)$ is cyclically equivalent
to $\pi^{1/N}\otimes I+I\otimes\pi^{1-1/N}$. Starting at $N=2$ this gives
the concept of rational powers of $\pi$.

The differentiation of a {\em CTP} representation $\otimes_t
U_t(g(t))$ of the
current group leads to an {\em ultralocal field} \cite{Arakithesis,Klauder}. These are
such that the truncated Wightman functions have the form
\begin{equation}
W_T(X_1(f_1)\ldots X_n(f_n))=\kappa_n(X_1\ldots X_n)\int f_1(t)\ldots
f_n(t)\,dt.
\label{ultralocal}
\end{equation}
Here, $\{\kappa_n\}$ are the cumulants of $\pi=dU$. The commutative analogue
was analysed in \cite{GelfandV}. For Lie algebras, we found \cite{RFS4}:
\begin{theorem}
The following are equivalent;\\
1) Eq.~(\ref{ultralocal}) defines a representation of ${\cal D}({\cal G})$.\\
2) The $\kappa_n$ are the cumulants of some $\infty$-divisible cyclic
representation of ${\cal G}$.\\
3) The $\kappa_n$ are positive semi-definite on ${\cal E}_1$,
the subalgebra of ${\cal E}$ with identity omitted.
\end{theorem}
We note that (3) is the expression of conditional positivity at the
algebraic level.
Since the cumulants of $\exp U$ are the moments of $U$, we can get a set of
$\kappa_n$ that obey the positivity (3) by using the moments of $\exp U$.
These happen to have a positive extension to ${\cal E}$: any
conditionally positive functional is positive. Th. (5.3) has a cohomological
version, which we outline.

Let ${\cal E}$ be an associative algebra with identity, ${\cal K}$ a linear
space and
$\tau$ a representation of ${\cal E}$ on ${\cal K}$. The $p$-cochain group
$C^p({\cal E},{\cal K},\tau)$ is the linear space of $p$-multilinear maps
$\phi:{\cal E}\times\ldots{\cal E}\rightarrow{\cal K}$. The coboundary
operator $\delta:C^p\rightarrow C^{p+1}$ is given by
\[(\delta\phi)(X_1,\ldots,X_{p+1})=\tau(X_1)\phi(X_2,\ldots,X_{p+1})+
\sum(-1)^j\phi(X_1,\ldots,X_jX_{j+1},\ldots,X_{p+1}).\]
Then $\delta^2=0$ and we define as usual the cocycle group $Z^*:=\mbox{ker}\,
\delta$ and the coboundary group $B^*:=\mbox{Ran}\,\delta$, and the
cohomology as $H^*:=Z^*/B^*$. ($^*$ means for any $p$). We see that
a 1-cocycle is a map $\phi:{\cal E}\rightarrow{\cal K}$ that satisfies
$\phi(XY)=\tau(X)\phi(Y)$, and a 1-coboundary 
is a cocycle of the form $\phi(X)=\tau(X)\phi_0$ for some $\phi_0\in
{\cal K}$.

The states on ${\cal E}_1$ are positive elements of $B^2
({\cal E}_1,{\bf C},0)$. Thus if $({\cal H},\pi,\psi)$ is
$\infty$-divisible, then its cumulants $W_T$ define a state on ${\cal E}_1$, and thus
a scalar product:
$\langle X,Y\rangle:=W_T(X^*Y)$. Here we define $X^*=-X$, since we want $\pi$
to represent the generators $iX$ of one-parameter subgroups by hermitian
operators. Define ${\cal K}$ as the separated prehilbert space obtained from
${\cal E}_1$ as usual. Let $\phi:{\cal E}_1\rightarrow{\cal K}$ be the
embedding obtained from this, and define a *-action $\tau$ of ${\cal G}$
on monomials by
\[\tau(X)\phi(X_1\ldots X_n):=\phi(XX_1\ldots X_n).\]
This states that $\phi$ is a 1-cocycle. We then show that there is a
bijection between the set of $\infty$-divisible cyclic representations
$({\cal H},\pi,\psi)$ of ${\cal G}$ and the triples $(\tau,\phi,\chi)$,
where $\tau$ is a hermitian representation of ${\cal G}$ on a prehilbert
space ${\cal K}$, $\chi$ is a real character, and
$\phi\in Z^1({\cal E}_1,{\cal K},\tau)$ such that
\begin{equation}
\gamma:=\mbox{Im}\langle\phi(X),\phi(Y)\rangle\in B^2({\cal E}_1,{\bf R},0).
\label{araki}
\end{equation}
In this bijection, ${\cal H}$ is embedded in $\Gamma({\cal K)}$, $\psi$ is
mapped to the Fock vacuum, and $\pi$ is related to $\exp\tau$ \cite{RFS4}.
So this is the Araki-Woods embedding theorem in this case.
If (\ref{araki}) fails then we get a projective representation of ${\cal G}$,
with multiplier $\sigma$ related to the cocycle $\gamma$
\cite{Mathon,Erven}. We see that a cocycle for ${\bf R}$ is defined by
a function $\chi\in L^1({\bf R})$ such that $x\chi\in L^2({\bf R})$.
We thus see the origin of the condition near $\alpha=0$ in (\ref{Levy}).


In \cite{Mathon2} we show that for Clifford algebras, the only possible
$\infty$-divisible states are
Gaussian (all cumulants above the second vanish). Here 
the coproduct is that of Chevalley,
$A\mapsto A\otimes I+(-1)^FI\otimes A$ where $F$ is the
degree of $A$, for elements of even or odd degree.

The algebraic theory
was extended to associative algebras (that were not enveloping
algebras of Lie algebras) by Hegerfeldt, who applied it to classify
$\infty$-divisible quantum fields \cite{Hegerfeldt2}.

Goldin et al. have, independently of this work, constructed
representations of a vector form of charge-current algebra, starting
with the Fock space creation-annihilation operators \cite{Goldin}; they
have been able to identify the representations in terms of the general
anlysis of semi-direct products.

Sch\"{u}rmann \cite{Schurmann} introduced the concept of infinite
divisibility for a representation of a Hopf algebra, and obtained 
essentially all the results of \cite{RFS4,Mathon,Mathon2} in this more
general setting.
Stochastic integrals for these processes were also constructed.
For a clear account, see \cite{Meyer}.

 
Voiculecsu developed the algebraic side into a subject called `free
probability' \cite{Voiculescu}, as it lives in full Fock space, without
symmetry or antsymmetry.


Albeverio and Hoegh-Krohn \cite{Albeverio} have constructed
representations of current 
groups, and been able to replace the independence at every point by
a covariance similar to the Nelson free field.
\input new20006
