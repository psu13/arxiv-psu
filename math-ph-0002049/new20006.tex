\section{Quantum Stochastic Semigroups}
These models of non-commutative noise, or quantum noise, are possible
driving random terms for noisy quantum dynamics.
What should we be looking for in a nonequilibrium stochastic quantum
dynamics? From 1970, E. B. Davies made progress in formulating stochastic
quantum dynamics \cite{Davies}.
Suppose that the $C^*$-algebra of observables is ${\cal A}$.
We look at the Fokker-Planck equation in the classical case, and we see that
we might expect a quantum stochastic process to be determined by a
semigroup (in continuous or discrete time) of maps $T_t$
from the state space $\Sigma({\cal A})$ to itself. It must map positive
operators, the density matrices, to positive operators, and preserve the
trace. We also do not want it to map a normal state to one of the
finitely additive ones, so we require a {\em stochastic map} to obey
\begin{enumerate}
\item $T$ maps $\Sigma$ to itself;
\item $T$ is linear;
\item In continuous time, $\|(T_t-I)A\|_1\rightarrow 0$ as $t\rightarrow0$.
\end{enumerate}
We can throw the action onto to algebra, to get the dual action $T^*:
{\cal A}\rightarrow{\cal A}$, by the requirement that for $A\in{\cal A}$,
\[\langle T\rho,A\rangle=\langle\rho,T^*A\rangle\mbox{ for all }\rho\in\Sigma.\]
$T^*$ is automatically normal. We see that if ${\cal A}$ is abelian,
then our conditions reduce to the properties
needed for a classical stochastic process. It is obvious that a
unitary time-evolution gives us a one-parameter family of
stochastic maps, which can be extended to a group by including the
inverses. We can get a large class of stochastic maps by
forming mixtures of unitary groups; thus if $\tau_j$ is a family of
invertible dynamics, then $T=\sum_j\lambda_j\tau_j$ is stochastic if
$\lambda_j\geq0$ and $\sum\lambda_j=1$.
Any stochastic map is non-invertible if
it is not unitary, and so is in this sense dissipative \cite{Davies}, p 25.
In addition, in the quantum case, Kraus \cite{Kraus} has argued that
to get a satisfactory interpretation of the semigroup, $T$ must be
{\em completely positive}. We say that a map $T:{\cal A}\mapsto{\cal A}$
is $n$-positive if $T\otimes I_n$ is positive on the algebra ${\cal A}\otimes{\bf
M}^n$. This is needed, since if our quantum system is described by the
algebra ${\cal A}$, and there is an $n$-state quantum system far away,
then the combined system will be described by ${\cal A}\otimes{\bf M}^n$,
and the dynamics on the combined system could be $T\otimes I_n$. This
must be positivity preserving, or else some state of the combined system
will evolve to give negative probabilities. Since we want to avoid this for
all $n$, we want $T$ to be $n$-positive for all $n=1,2\ldots$. Such a
condition is called complete positivity. It should be said that any
positive map on an abelian algebra is always completely positive, so this
concept only seriously arises in quantum probability.

Kraus showed that a map $T$ is completely positive if and only if $T(A)$
is a sum of maps of the form $S_n^*AS_n$, where the $S_n$ are bounded;
\cite{Davies}, p. 140.


The great result in the subject is the classification of continuous
semi-groups of completely positive maps. In finite dimensions this was
achieved in \cite{Gorini}, and independently, by Lindblad, \cite{Lindblad}
whose
result holds for norm-continuous dynamics on $C^*$ algebras. Their result
is the quantum analogue of the heat equation, i. e. it is a dynamical
equation for the density matrix. For a simple derivation,
see \cite{Landau2}. The result is:
\begin{theorem}
Let $T_t$ be a semigroup of completely positive stochastic maps on ${\bf
M}^n$. Then there exists a Hermitian matrix $H$ and matrices $S_j$
such that the generator of the semigroup has the form
\begin{equation}
Z(A)=i[H,A]-\frac{1}{2}(RA+AR)+\sum_jS_j^*AS_j,\hspace{.3in}\mbox{ where }R=
\sum_jS^*_jS_j.
\label{Lindblad}
\end{equation}
\end{theorem}
This can be thrown onto the density matrices by duality. The first term
$i[H,A]$ is non-dissipative, and is called the hamiltonian term.
The second term is the dissipation.

It is very interesting that the first two terms
of the Heisenberg expansion of the dynamics are of this form. Thus,
\begin{eqnarray*}
\left(e^{iHt}Ae^{-iHt}-A\right)t^{-1}&=&i[H,A]-\frac{1}{2}[H,[H,A]]t+
O(t^2)\\
&=&i[H,A]-\frac{1}{2}(AS^2+S^2A)+SAS\\
& &\mbox{ where }S=Ht^{1/2},
\end{eqnarray*}
up to O(t), so it is of the form eq.~(\ref{Lindblad}) with $R=S^2$.
In the {\em anti-van Hove limit} \cite{Streater} we replace
$S$ by $\lambda H$.

It has been remarked that the commutator $A\mapsto i[H,A]$ is a derivation
of the operator algebra, and so has many of the properties of a derivative.
The double commutator
has many of the properties of the second derivative, including some
positivity, which mimics the positive spectrum of $-\Delta$ and the
positivity improving properties of $e^{\Delta t}$.
Lindblad has analysed continuous semigroups of cp maps, with generator ${\cal
L}$, in terms of the ``dissipation operator'', being minus the coboundary of
$L$:
\begin{equation}
D(A,B):=-\delta{\cal L}(A,B)={\cal L}(AB)-{\cal L}(A)B-A{\cal L}(B).
\end{equation}
He proves that $T_t:=\exp(i{\cal L}t)$ is a continuous semigroup of cp maps
if and only if $D$ is positive in the sense that
\begin{equation}
\sum_{ij}C_i^*D(A_i^*,A_j)C_j\geq0\hspace{.3in}\mbox{ for all }A_i, C_j\in
{\cal A}.
\end{equation}
Note the formal similarity with \cite{Araki,RFS2,RFS4,Partha}. Fannes and
Quaegebeur \cite{Fannes} have defined the concept of $\infty$-divisible
completely positive mappings on groups, in which the function $C(g)$ is
replaced by a cp operator. They prove an Araki-Woods embedding theorem
for such structures.


Recall that for Markov chains, Brownian motion and Euclidean field theory,
we can express the given semigroup as an isometric time-translation,
followed by the conditional expectation onto the initial space.
By using two-sided time, the isometries can be replaced by a unitary group.
The finding of the appropriate unitary group is called the {\em dilation}
of the semi-group. It is not unique, but there is a unique minimal one.
\cite{Stinespring}. It would be nice to interpret the dilated system
as representing the full physics of system plus environment, with a
unitary evolution; the projection onto a subspace represents our loss
of information due to incomplete knowledge. The ambiguity of the dilation
then shows that several different models give the same (crude)
coarse-grained dynamics. However, it will rarely be the case that a dilation
has the good properties, such as positivity of the energy, needed for
this interpretation.

This is illustrated in the quantum case, which in finite dimensions
takes the form \cite{Davies}
\begin{theorem}
Let $T_t$ be a semigroup of cp stochastic maps on ${\bf M}_n$ acting
on ${\cal H}$. Then there exists a Hilbert space ${\cal K}$, a pure state
$\rho$ on ${\cal H}\otimes{\cal K}$ and a one parameter unitary group
$V_t$ on ${\cal H}\otimes{\cal K}$ such that
\[ T_t(A)=E_\rho[V^*_t(A\otimes I)V_t]\]
for all $A\in{\cal M}$ and all $t\in{\bf R}$.
\end{theorem}

This is proved by putting together Theorem (4.2) and \S 7.2 of \cite{Davies}.
Note that the Hilbert space ${\cal K}$ is constructed by adding Wiener
noise, and so is not finite-dimensional. The semi-group has been
dilated to a unitary group on the Wiener space with two-sided time;
the generator of time-evolution is not bounded below, since it has white
spectrum. This does not represent an environment at any finite temperature.
A special case is the dilation of the semigroup given by the anti-van Hove
limit. In that case the process is given by
\begin{equation}
X(t)=(2\pi t\lambda^2)^{-1/2}\int e^{-s^2/(2\lambda^2t)}U(s+t)XU(-s-t)\,ds.
\end{equation}
This has the interpretation as the Heisenberg evolution, but with the time
$t$ slightly uncertain, and getting more uncertain in the future.
This interpretation is only a slight
variation on the methods used in the justification of the microcanonical
state by ergodic theory. There, it is said that the atomic times are so
small that we never measure an observable {\em at} a particular time;
rather, we measure the average over the time $0\leq s\leq t$
of the measurement thus: $\overline{A}=t^{-1}\int_0^t A(s)ds$. Since
$t$ is so large compared with the atomic processes, we take the limit
$t\rightarrow\infty$. This idea is a non-starter for non-equilibrium
statistical mechanics, since if the limit exists it is time-independent.
Instead, we may say that we cannot measure an observable at an {\em exact}
time, but form the weighted average, with Gaussian weight, around
the desired time $t$. The uncertainty in the Gaussian is $\lambda^2 t$,
growing with time. $\lambda$ is the dissipation parameter. In models it
turns out to be the hopping parameter of the atomic system.

Some authors limit the concept of quantum stochastic process to
the case where the possible observed path of measurements themselves
make up a classical process. The grounds for this is that the observations
(in a set of repeated experiments) have actually been seen; these form
the {\em quantum record}; take them to
form a sample space. However, this is not true. The process $X(t)$
at different times might not
commute, so the measurement of $X(t)$ alters the state (by collapse),
and subsequent measurements are not those predicted by $X(t+s)$, $s>0$,
as computed using the given initial state. It needs conditioning
to the new information, and quantum conditional expectations only
commute on abelian subalgebras.
Moreover, one can measure $X(t)$ in one sampling and $Y(t)$ in another,
where $X$ and $Y$ do not commute. No classical model would predict
the statistics of the process; the classical theorist is liable to be hit
by the {\em EPR} paradox in acute form. We regard $X(t)$ as the observable
seen at time $t$ when no measurement has been made in $\{s:0<s<t\}$. So we
cannot agree with the idea that the randomness itself is caused by the
reduction of the wave-function due to continuous measurement; it might
be due to interaction with a large other body, but not one designed to
measure any particular observable.

Davies's dilation of the Lindblad semigroup uses a number
of independent Wiener processes to provide the set-up.
The question arises whether there is a relation between quantum dynamical
semigroups and a class of quantum stochastic differential equations, similar
to the relation between the Fokker-Planck equation (\ref{63}) and the sde
(\ref{sde}). For this, we need a quantum version of Ito's integral.
In 1956, Umegaki defined the concept of conditional expectation in
non-commutative integration theory \cite{Umegaki}. Let ${\cal A}$ be a
von Neumann algebra
with a semi-finite trace, and say an operator $A$ is integrable if
$\mbox{Tr}\,|A|<\infty$. The vector space of integrable operators can
be completed to form the space $L^1({\cal A})$. Segal and Nelson showed
that there
is a closed operator representing an element of the completion.
Let ${\cal A}_t$ be an increasing family
of subalgebras which generate ${\cal A}$ and are right continuous
\cite{Barnett}, such that the trace, restricted to each ${\cal A}_t$
is semi-finite.
Then a conditional expectation relative to the trace
is a linear map $M:L^1({\cal A})\rightarrow L^1({\cal A}_t),\;t\geq 0$,
such that
\[\mbox{Tr}(XA)=\mbox{Tr}(M_t(X)A)\hspace{.5in}\mbox{ for all }A\in{\cal
A}_t,\hspace{.1in}X\in L^1({\cal A}).\]
A martingale is a process $X_t$ of integrable operators such that
\[M_sX_t=X_s\]
for all $0\leq s\leq t$.
This concept can be generalised to a filtration of algebra with specified
state, rather than trace.

Cuculescu \cite{Cuculescu} proved a martingale convergence theorem for
discrete time. Barnett \cite{Barnett} obtained a martingale theorem for
continuous time. This work persuaded us to look for examples of noncommuting
martingales. Soon we found plenty within the theory of continuous
tensor products \cite{Hudson}. Let $({\cal H},U,\psi)$ be an
$\infty$-divisible representation of a Lie group G, and consider
$\otimes_{t=0}^\infty {\cal H}_t$ relative to the vector $\otimes\psi_t$
and the set $\Delta$ of coherent vectors. Here, all factors are the same.
To $g\in G$ we associate the family of unitary operators
\begin{equation}
V_t(g):=\otimes_0^t U(g)\otimes_t^\infty I.
\label{mart}
\end{equation}
We call such an operator {\em simple}, localised in $[0,t]$.
Let ${\cal A}_t$ be the algebra generated by $\{V_s(g)\}$ with
$0\leq s\leq t$ and $g\in G$. Then for $s<t$ define the map $M_s:{\cal A}_t
\rightarrow{\cal A}_s$ by continuous linear extension of $M_s\otimes_{r=0}
^tV_r(g)=\otimes_{r=0}^sV_r(g)$. Then $M_s$ is a conditional expectation,
and relative to $M_s$, the family $V_t$ is a martingale.
Applied to $G={\bf R}$ with $\psi$ a Gaussian state, $V_t$ is the
exponential martingale of Brownian motion.
When $G$ is the oscillator group, the lie algebra is spanned by $P,Q,H$ and
a central element $I$. There is a representation by self-adjoint operators
on $L^2({\bf R})$, with the ground state of the harmonic oscillator as cyclic
vector. This is
infinitely divisible, and the unitary operators in (\ref{mart})
are copies of the exponential martingale $e^{iW_t}$ for the subgroups
generated by $P$ and $Q$, and is the Poisson exponential martingale
for the subgroup generated by $H$ \cite{Wulfsohn}. This became known as the
gauge process \cite{Partha2}.
All these martingales are defined on the total set of coherent states.
Since they are unitary, they can be extended to an everywhere-defined
unitary group, the generators of which are self-adjoint operators.
This is
the main technique of the Hudson-Parthasarathy calculus
\cite{Hudson4,Hudson3,Partha2}

Examples of martingales with trace were given in \cite{Barnett2}.
Consider the Fock Fermi operators $b(f),b*(g)$ with anticommutation relations
$[b(f),b^*(g)]=\langle f,g\rangle$ for $f,g\in L^2({\bf R}_+)$. The algebra
generated by these and the Fock condition $b(f)|0\rangle=0$ is represented on
antisymmetric Fock space over $L^2({\bf R}_+)$ as the $W^*$-algebra
generated by the Fermi field $\psi(f)=b(f)+b^*(\overline{f})$ acting on the
Fock vacuum $|0\rangle$. The Clifford process is the set of operators
\begin{equation}
\Psi(t):=\psi(\xi_{[0,t]}).
\end{equation}
The non-commutative integration theory \cite{Segal0,Segal2,Kunze},
taking the place of measure theory, is that based on the hyperfinite
von Neumann factor of type $II_1$, which is
furnished with a faithful trace $\varphi(A)=\langle 0|A|0\rangle$.
The completion of ${\cal A}$ in the norm $\|A\|=\varphi(A^*A)^{1/2}$
is denoted $L^2({\cal A},\varphi)$.
The projection $M_t$ from $L^2({\cal A},\varphi)$ onto $L^2({\cal A}_t,
\varphi)$
is the same as the projection from $\Gamma(L^2[0,\infty])$ onto $\Gamma
(L^2[0,t])$; it obeys the laws for a conditional expectation, and
$\Psi(t)$ is a martingale.

The increments of $\Psi(t)$ are independent, but anti-commute. Otherwise, all
the properties are analogous to Brownian motion. The isometric time-evolution
analogous to the left shift of the classical theory is that given by the map
$U_s:\Psi(t)\mapsto \Psi(s+t)$. The antisymmetric Fock space over
$L^2({\bf R})$ carries a unitary extension of $U_s$, namely the second
quantisation of translation in ${\bf R}$.
We define an {\em adapted} process $h(t)$ to be a family of operators
such that $h(t)\in{\cal A}_t$; it is {\em simple} if it can be expressed
as
\begin{equation}
h=\sum_{k=1}^n h_{k-1}\chi_{[t_{k-1},t_k)}\mbox{ on }[0,t).
\end{equation}
We then define the stochastic integral of any simple adapted process,
relative to $\Psi$, to be that constructed in the manner of Ito,
with the forward difference in $d\Psi$:
\begin{equation}
\int_0^t f(s)d\Psi(s):=\sum_{k=1}^n h_{k-1}\left(\Psi(t_k)-\Psi(t_{k-1})
\right).
\label{Itoclifford}
\end{equation}
As in Ito's theory, what make it work is an isometry property:
\begin{theorem}
If $h(t)$ is a simple process made up of $L^2$ operators, then $\int_0^t
h(s)d\Psi(s)\in L^2$, and
\[\|\int_0^th(s)d\Psi(s)\|_2^2=\int_0^t\|h(s)\|_2^2ds.\]
\end{theorem}
The proof \cite{Barnett2} is similar to Ito's.
We use this to construct the integral of square-integrable adapted processes,
and some $L^p$ processes, by extension to the completion of the space of
simple adapted processes. The stochastic integral is the quantised field
$\Psi$, smeared with an operator $h$ rather than a test-function.
There is a Doob-Meyer theorem: $M_t^2$ is the sum of a martingale, denoted by
$[M_t,M_t]$ in classical theory, (NOT the commutator!) and an increasing
process of bounded variation, denoted $\langle M_t,M_t\rangle$.
Any stochastic integral is a martingale, and
we show the converse, that any $L^2$ martingale of mean zero is a
stochastic integral.
We also define the stochastic integral $N(t)=\int_0^th(s)
dM(s)$, where $h$ is adapted and square-integrable relative to $\langle M_t,
M_t\rangle$. Here,
$M$ is an $L^2$-martingale. This representation of
$N$ is unique; we then write $h$ as the stochastic derivative: $h=\partial
N/\partial M$. We show that we can change variables in the
integral: the stochastic Radon-Nikodym theorem \cite{Barnett}.

We are able to show \cite{Barnett5} that the quantum sde
\begin{equation}
dX_t=F(X_t,t)dM_t+dM_tG(X_t,t)+H(X_t,t)dt
\end{equation}
has a solution in $L^2({\cal A},\varphi)$ for $F,G,H$ continuous, adapted
and locally uniformly Lipschitz, for any martingale $M_t$ of degree $n$,
and that the solution obeys the Markov property \cite{Barnett6}.

Manipulations of differentials
are similar to the Ito calculus: $(dt)^2=0=(dt)(d\Psi)$; $(d\Psi)^2=dt$.
Pisier and Xu have obtained ``Burkholder-Gundy'' inequalities
within this theory \cite{Pisier}.

The central state $\varphi$ of the Clifford algebra corresponds physically
to an infinite temperature.
For the {\em CCR} and {\em CAR} algebras, we constructed the stochastic
integrals starting with quasifree states with no Fock part, using the
non-central state in place of the trace \cite{Barnett4,Lindsay}. This theory
is somewhat technical (``unreadable'' \cite{Meyer2}).

The general Lindblad semigroup can be dilated \cite{Evans}
using the flow defined by a solution to a quantum stochastic equation in
the sense of Hudson and Parthasarathy \cite{Hudson3,Partha2,Smith}.
It was extended to some unbounded cases by Belavkin \cite{Belavkin}.
We now give a brief account of this, following Frigerio \cite{Frigerio}.

Let $T_t=\exp({\cal L}t)$ be a semigroup of completely positive normal
stochastic maps on the algebra ${\cal B}({\cal H})$.
\begin{theorem}
There exists a Hilbert space ${\cal F}$, a group $\{\alpha_t:t\in{\bf R}\}$
of $^*$-automorphisms of ${\cal B}({\cal H}\otimes{\cal F})$ and a
conditional expectation $E_0$ of ${\cal B}({\cal H}\otimes{\cal F})$
onto ${\cal B}({\cal H})\otimes I_{\cal F}$ such that
\begin{equation}
T_t(X)\otimes I_{\cal F}=E_0[\alpha_t(X\otimes I_{\cal F}0],\hspace{.3in}
X\in{\cal B}({\cal H}),\;\;t\in{\bf R}.
\end{equation}
\end{theorem}
The evolution $\alpha_t$ is a perturbation of the ``free evolution''
$\alpha_t^0$ on ${\cal B}({\cal H})$, of the form
\begin{equation}
\alpha_t(\,.\,)=U(t)\alpha_t^0(\,.\,)U(t)^*,
\end{equation}
where $\{U(t):t\in{\bf R}\}$ satisfies the cocycle condition
\begin{equation}
U(t)\alpha_t^0(U(s))=U(s+t),\hspace{.3in}t,s\in{\bf r},
\end{equation}
is unitary and is the solution of a qsde in the sense of
\cite{Hudson3,Partha2}. We give the details in the simplest case,
eq.~(\ref{Lindblad}) with only one term $S$ in the sum. We take ${\cal F}=
\Gamma(L^2{\bf R})$, with total the set of coherent vectors $\exp\phi:\phi
\in L^2({\bf R})\cap L^1({\bf R})$. We define the {\em annihilation} process,
{\em creation} process and {\em gauge} process on this total set by
\begin{eqnarray}
A(t)\exp\phi&=&(\int_0^t\phi(s)ds)\exp\phi\\
A^*(t)\exp\phi&=&\frac{d}{d\epsilon}\exp\left(\phi+\epsilon\chi_{[0,t]}
\right)|_{\epsilon=0}\\
\Lambda(t)\exp\phi&=&\frac{d}{d\epsilon}\exp\left(e^{\epsilon\chi_{[0,t]}}
\phi\right)|_{\epsilon=0}.
\end{eqnarray}
The conditional expectation $M_t$ is as for the {\em CTP},
$\otimes_{s=0}^t({\cal F}_s)$, based on the Fock vacuum, and
$A(t),A^*(t)$ are the creators and annihilators defined by
the generators $P,Q$ of the Heisenberg
subgroup of the oscillator group; $\Lambda$ is the number operator.

We identify any operator $X$ in ${\cal B}({\cal H})$ with its ampliation
$X\otimes I_{\cal F}$, and any operator $Y$ with domain ${\cal D}
\subseteq{\cal F}$ with the algebraic tensor product $I_{\cal H}\otimes Y$.
A family $U(t)$ is found by solving the qsde
\begin{equation}
dU(t)=U(t)\left[iS^*dA(t)+iS\,dA^*(t)+(iH-S^*S/2)dt\right],
\end{equation}
with the initial condition $U(0)=I$. The structure of the equation is
designed to ensure that the solution, defined on the set of coherent states,
is continuous, unitary, and adapted. The term $S^*S/2$ arises as the Ito
correction, or as due to the Wick ordering \cite{Hudson2}.
To ensure that $\alpha_t$ obeys the group law, the usual free evolution
$\alpha^0$ on ${\cal F}$, the second quantisation of the
translation group on $L^2({\bf R})$, is chosen. It is then proved that
$\alpha^0_{-s}[U^*(s)U(s+t)]$ satisfies the same qsde as $U(t)$, and so,
by uniqueness, must be $U(t)$. So $U$ satisfies the cocycle condition.
On multiplying out, we see that $\alpha_y$ is a group.

The theorem for a semigroup with a finite number of operators $S_j$
follows a similar line.\hspace{\fill}$\Box$\\
There is a fermionic version of this dilation \cite{Applebaum}.

Quantum stochastic calculus has become a mature field of mathematics.
The approach of \cite{Hudson3}, rather than \cite{Barnett2}, has the
disadvantage that the stochastic integrals are defined as operators
only on a dense set. It is not always clear that they have a unique closed
extension.
This is overcome in \cite{Hudson3,Partha2} by limiting the class of
equations to those with unitary solutions.  Another help in the analysis
is by the use of Maassen kernels \cite{Maassen}. Alternatively, \cite{Obata}
one may give a meaning to these objects as maps between test-functions and
distributions, using white-noise analysis.

One problem with this work, and this includes \cite{Barnett}
as well, is that the spectrum of the noise is white, so that random
negative energy is added as well as positive energy.
We saw that positive energy seems to exclude martingales
\cite{Senitzky,RFS5}. In fact, the {\em KMS} condition excludes the existence
of a conditional expectation except in trivial cases. It has been remarked
that it also excludes the Markov property and the regression theorem
\cite{Talkner}. Lindblad has remarked \cite{Lindblad2} that for the
oscillator, the {\em KMS} condition is not compatible with the axioms of
dynamical semigroups. So to model random external forces in a real system,
coupled to a heat-bath, the white noise sde is an approximation, that
might be good if the time-interval is large compared with the memory time.
These ideas are used
to describe quantum systems like lasers, which are subject to external
forces; this was the original intention of Senitzky and Lax. The
modern version is described in \cite{Alicki}. Since external forces
introduce energy and entropy into a system, such models have two
drawbacks:
\begin{enumerate}
\item The first law of thermodynamics is not obeyed.
\item The second law of thermodynamics is not obeyed.
\end{enumerate}
This is the starting point of \cite{AlickiM,Balian,Streater}.
One step of the linear dynamics is given by a bistochastic map
$\rho\mapsto\rho T$, so that entropy increases. We require that $T^*$ maps
any spectral projection of the energy to itself; this will preserve energy.
To reduce the description, we then project the
new state $\rho T$ onto the information manifold ${\cal M}$ defined by
the set of slow variables, to get the state $\rho TQ$.
To preserve mean energy, the energy must be a slow variable. The map $Q$
is nonlinear and is interpreted as the thermalisation of the fast
variables. Thus, after the map $T$, the system itself decides to find the
best estimate $\rho TQ$ to $\rho T$ within ${\cal M}$. The resulting map
gives a nonlinear motion through the manifold, obeying the first and second
laws of thermodynamics. This theory, called {\em statistical dynamics},
is still being explored \cite{RFS1,RFS7,Streater}.
\input qbib




\bibitem{Nelson4} Nelson, E., {\bf Dynamical Theories of Brownian Motion},
Princeton Univ. Press, 1967.
\bibitem{Emch} Emch, G. G., {\bf Algebraic Methods in Statistical Mechanics
and Quantum Field Theory}, Wiley, 1972.
\bibitem{Haag} Haag, R., {\bf Local Quantum Physics}, $2^{\rm nd}$ ed.,
Springer-Verlag, 1996
\bibitem{Horuzhy} Horuzhy, S. S., {\bf Introduction to Algebraic Quantum
Field Theory}, Kluwer Academic, 1990.
\bibitem{Holevo} Holevo, A. S. {\bf Probabilistic and Statistical Aspects
of Quantum Theory}, North Holland, 1982.
\bibitem{Ohya} Ohya, M., and D. Petz, {\bf Quantum Entropy and its Use},
Springer-Verlag, Heidelberg, 1993.
\bibitem{Ingarden4} Ingarden, R. S., H. Janyszek, A. Kossakowski and
T. Kawaguchi, {\em Ibid}, {\bf 37}, 105-111, 1982.
\bibitem{Petz} Petz, D., and G. Toth, Lett. Math. Phys., {\bf 27}, 205-216,
1993.
\bibitem{Hasagawa} Hasagawa, H., and D. Petz, `Noncommutative extension of
information geometry II', pp 109-118 in: {\bf Quantum Commun., Computing and
Measurement}, Ed. Hirota et al., Plenum Press, N. Y. 1997.
\bibitem{Gross} Gross, L., `Abstract Wiener spaces',
{\em Proc. $5^{\rm th}$ Berkeley Symp. in math. stat. and prob. theory},
31-42, 1965-66.
\bibitem{Schwartz} Schwartz, L., {\bf Radon Measures on Arbitrary
Toplological Space and Cylindrical Measures}, Oxford Univ. Press, 1973.
\bibitem{Williams} Williams, D. {\bf Probability with Martingales},
Cambridge University Press, 1991.
\bibitem{Grimmett} Grimmett, G. R., and D. R. Stirzaker, {\bf
Probability and Random Processes}, Oxford, 1982.
\bibitem{Varadhan} Varadhan, S. R. S., {\bf Large Deviations and
Applications}, SIAM, Philadelphia, 1984.
\bibitem{Donsker} Donsker, M. D., and S. R. S. Varadhan, {\em Phys. Rep.},
{\bf 3}, 235-237, 1981.
\bibitem{Lewis} Lewis, J. T. `Large deviation principle in statistical
mechanics'. pp 141-155, LNM {\bf 1325}, Ed. A. Truman and I. M. Davies,
Springer-Verlag, 1988.
\bibitem{Stroock2} Stroock, D. W., {\bf An Introduction to the Theory
of Large Deviations}, Springer-Verlag, 1984.
\bibitem{GelfandV} Gelfand, I. M. and N. Ya. Vilenkin, {\bf Generalised
Functions IV}, Academic Press, 1964.
\bibitem{Stroock} Stroock, D. and S. R. S. Varadhan, {\bf Multidimensional
Diffusion Processes}, Springer-Verlag, 1979.
\bibitem{Williams2} Williams, D., `To begin at the beginning', 1-55 in
{\bf Stochastic Integrals}, Ed. D. Williams, LNM {\bf 851}, Springer-Verlag,
1981.
\bibitem{Nelson4} Nelson, E., {\bf Dynamical Theories of Brownian Motion},
Princeton University Press, 1967.
\bibitem{McShane} McShane, E. J., {\bf Stochastic Calculus and Stochastic
Models}, Academic Press, N. Y., 1974.
\bibitem{Barnett3} Barnett, C., R. F. Streater and I. F. Wilde, `Quantum
stochastic integrals under standing hypotheses', {\em J. of Math.
Anal. and Applications}, {\bf 127}, 181-192, 1987.
\bibitem{Kac} Kac, M., `Some connections between probability theory
and differential and integral equations', {\em Proc. $2^{\rm nd}$ Berkeley
Symp.} J. Neyman (ed.), University of Calif Press, Berkeley, 1951.
\bibitem{Feynman} Feynman, R. P., {\em Rev. Mod. Phys.}, {\bf 20}, 267-,
1948).
\bibitem{Nelson} Nelson, E., `Feynman Integrals and the
Schr\"{o}dinger Equation', {\em Jour. Math. Phys.,} {\bf 5}, 332-343, 1964.
\bibitem{Simon} Simon, B., {\bf Functional Integration and Quantum
Physics}, Academic Press, N. Y., 1979.
\bibitem{Glimm} Glimm, J., and A. M. Jaffe, {\bf Quantum Physics},
Springer-Verlag, Second Ed., 1987.
\bibitem{Glimm2} Glimm, J., and A. M. Jaffe, `Probability applied to
physics', {\em Univ. Arkansas Lect. Notes in Math.}, {\bf 2},
Fayetteville, 1978.
\bibitem{Frohlich} Fr\"{o}hlich, J., `Schwinger functions and their
generating functionals, I', {\em Helv. Phys. Acta}, {\bf 47}, 265-306, 1974.
II, `Markovian and generalized path space measures on ${\cal S}$',
{\em Adv. in Math.}, {\bf 23}, 119-180, 1977.
\bibitem{Guerra} Guerra, F., L. Rosen and B. Simon, `The $P(\varphi)_2$
Euclidean quantum field theory as classical statistical mechanics',
{\em Annals of Math.}, {\bf 101}, 111-259, 1975.
\bibitem{Dyson} Dyson, F. J., `The S-matrix in quantum electrodynamics',
{\em Phys. Rev.}, {\bf 75}, 1736-1755, 1949.
\bibitem{Schwinger} Schwinger, J. {\em Phys. Rev}, {\bf 82}, 664-, 1951.
\bibitem{Jost} Jost, R. {\bf General Theory of Quantized Fields}, Amer.
Math. Soc., Providence, 1965.
\bibitem{SW} Streater, R. F., and A. S. Wightman, {\bf PCT, Spin and Statistics,
and All That}, Benjamin-Cummings, 1964.
\bibitem{Symanzik} Symanzik, K., `Euclidean quantum field theory, pp 152-226
in {\bf Local Quantum Theory}, R. Jost (ed.), Academic Press, 1969.
\bibitem{Minlos} Minlos, R. A. Trudy Mosk. Mat. Obs. {\bf 8}, 497-518, 1959.
\bibitem{Wong} Wong, E., {\em Ann. Math. Stat.}, {\bf 40}, 1625-1634, 1969.
\bibitem{Nelson2} Nelson, E., `The Free Markov Field', {\em J. Functl.
Anal.}, {\bf 12}, 211-227, 1973.
\bibitem{Simon2} Simon, B., {\bf The $P(\varphi)_2$ Euclidean (Quantum)
Field Theory}, Princeton Univ. Press, 1974.
\bibitem{Gross2} Gross, L., `The free Euclidean Proca and electromagnetic
fields', pp 69-82 in: {\bf Functional Integration and its
Applications}, ed. A. M. Arthurs, Oxford, 1975.
\bibitem{Senitzky} Senitzsky, I. R., `Dissipation in Quantum Mechanics:
The harmonic oscillator I,II'. {\em Phys. Rev.}, {\bf 119}, 670-679,
1960; {\em ibid}, {\bf 124}, 642-648, 1961.
\bibitem{RFS5} Streater, R. F., `Damped oscillator with quantum noise',
{\em J. Phys.}, {\bf A15}, 1477-1486, 1982.\bibitem{RFS5} Streater, R. F., `Damped oscillator with quantum noise',
{\em J. Phys.}, {\bf A15}, 1477-1486, 1982.
\bibitem{Lax} Lax, M., {\em Phys. Rev.}, {\bf 145}, 111-129, 1965.
\bibitem{Hudson3} Hudson, R. L., and K. R. Parthasarathy, `Quantum Ito's
formula and stochastic evolutions', {\em Commun. Math. Phys.}, {\bf 93},
301-303, 1984.
\bibitem{Partha2} Parthasarathy, K. {\bf An Introduction to Quantum
Stochastic Calculus}, Birkh\"{a}user, Basel, 1992.
\bibitem{Ford} Ford, G. W., M. Kac and P. Mazur, {\em J. Math. Phys.},
{\bf 6}, 505-515, 1965.
\bibitem{Lewishb} Lewis, J. T., and L. C. Thomas, `How to make a heat
bath', 97-123 in {\bf Functional Integration and its Applications},
Ed. A. M. Arthurs, Clarendon Press, Oxford, 1975.
\bibitem{HLK} Hasagawa, H., J. R. Klauder and M. Lakshmanan, {\em J. of
Phys.}, {\bf A14}, L123-L128, 1985.
\bibitem{Accardi} Accardi, L., A. Frigerio and J. T. Lewis, `Quantum
stochastic processes', {\em Proc. Res. Inst. Math. Sci.}, {\bf 18},
97-133, 1982.
\bibitem{Ford2} Ford, G. W., `Temperature-dependent Lamb shift of a quantum
oscillator', pp 202-206 in {\bf Quantum Probability and Applications II},
LNM {\bf 1136}, Eds. L. Accardi and W. von Waldenfels, Springer-Verlag, 1985.
\bibitem{ArakiW} Araki, H., and E. J. Woods, {\em Proc. R. I. M. S., Kyoto},
{\bf 2}, No.2, 1966.
\bibitem{Streater2} Streater, R. F. {\em Nuovo Cimento,} {\bf 53}, 487-495,
1968.
\bibitem{RFS3} Streater, R. F. {\em Nuovo Cimento} {\bf 53}, 487-495, 1968.
\bibitem{HIK} Hudson, R. L., P. D. F. Ion and K. R. Parthasarathy,
`Time-orthogonal unitary dilations...'{\em Commun. Math. Phys.},
{\bf 83}, 261-280, 1982.
\bibitem{Guichardet} Guichardet, A., {\em Commun. Math. Phys.}, {\bf 5},
262-, 1967.
\bibitem{Dubin} Dubin, D. A. and R. F. Streater, {\em Nuovo Cimento} {\bf 50},
154-157, 1967.
\bibitem{RFS2} Streater. R. F. `Current commutation relations, continuous
tensor products, and infinitely divisible group representations', pp
247-263 in {\bf Local Quantum Theory}, Ed. R. Jost, Academic Press, 1969.
\bibitem{Araki} Araki, H. `Factorizable representations of current algebras'
{\em Proc. R. I. M. S., Kyoto}, {\bf 5}, 361-422, 1970/71.
\bibitem{Partha} Parthasarathy, K., and Schmidt, K., `Positive definite
kernels, continuous tensor products, and and central limit theorems of
probability theory', {\bf Lecture Notes in Maths.}, {\bf 272}, 1972.
\bibitem{Wigner} Wigner, E. P., {\bf Group Theory and Applications to 
Atomic Spectroscopy}, First Ed. (German), Viewig., Brannschweig, 1931.
\bibitem{Gelfand} Gelfand, I. M., M. I. Graev and A. M. Vershik,
`Representations of the group $SL(2,{\bf R})$, where ${\bf R}$ is a ring
of functions.' {\em Uspehi-Mat-Nauk} {\bf 28}, 83-128, 1973.
\bibitem{Guichardet2} Guichardet, A., {\bf Symmetric Hilbert Space and Related
Topics}, LNM {\bf 261}, Springer-Verlag, 1972.
\bibitem{Erven} Erven, J., and B.-J. Falkowski, {\bf Low Order Cohomology
and Applications}, Lecture Notes in Mathematics, 877, Springer-Verlag, 1981.
\bibitem{RFS4} Streater, R. F. `Infinitely divisible representations of Lie
algebras', {\em Wahrschein. ver. Geb.}, {\bf 19}, 67-80, 1971.
\bibitem{Mathon} Mathon, D., `Infinitely divisible projective
representations of the Lie algebras', {\em Proc. Camb. Phil. Soc.}, {\bf 72},
357-368, 1972.
\bibitem{Mathon2} Mathon, D. and R. F. Streater, `Infinitely divisible
representations of Clifford algebras', {\em Zeits Wahr. verw. Geb.},
{\bf 20}, 308-316, 1971.
\bibitem{Cushen} Cushen, C. D., and R. L. Hudson, `A quantum mechanical
central limit theorem', {\em J. Appl. Prob.}, {\bf 8}, 454-469, 1971.
\bibitem{Hudson7} Hudson, R. L., `A quantum-mechanical central limit
theorem for anti-commuting variables', {\em J. Appl. Prob.}, {\bf 10},
502-509, 1973.
\bibitem{Hegerfeldt2} Hegerfeldt, G. C., `Prime field decompositions and
infinitely divisible states on Borchers' tensor algebra', {\em Commun.
Math. Phys.}, {\bf 45}, 137-151, 1975.
\bibitem{Schurmann} Sch\"{u}rmann, M., `Positive and conditionally positive
linear functionals on coalgebras', 475-492 in: {\em Quantum Probability II},
Eds. L. Accardi and W. von Waldenfels, LNM {\bf 1136}, 1985.
`Infinitely divisible states on cocommutative
bialgebras', Proc. {\em probability Measures on Groups IX}, Oberwohlfach,
1988. {\bf White Noise on Bialgebras}, {\em Lecture Notes in Math.},
{\bf 1544}, Springer-Verlag, 1993.
\bibitem{Arakithesis} Araki, H., Ph. D. Thesis, Princeton, 1960 (unpublished).
\bibitem{Klauder} Klauder, J. R., `Fock space revisited', {\em Jour.
Math. Phys.}, {\bf 11}, 609-630, 1970; `Ultralocal scalar field
models', {\em Commun. Math Phys.}, {\bf 18}, 307-318, 1970.
\bibitem{Goldin} Goldin, G. A., R. Menikoff and D. H. Sharp, {\em J. Math.
Phys.}, {\bf 21}, 650-, 1980.
\bibitem{Meyer} Meyer, P.-A., {\bf Quantum Probability for Probabilists}
Lecture Notes in Mathematics {\bf 1538}, Springer-Verlag, 1991.
\bibitem{Voiculescu} Voiculescu, D., `Addition of certain non-commuting
random variables', {\em J. Functl. Anal.}, {\bf 66}, 323-346, 1986.
\bibitem{Albeverio} Albeverio, S. and Hoegh-Krohn, `Some Markov processes
and Markov fields', pp 497-540, in {\bf Stochastic Integrals},
ed. D. Williams, LNM {\bf 851}, Springer-Verlag, 1981.
\bibitem{Davies} Davies, E. B., {\bf Quantum Theory of Open Systems},
Academic Press, 1976.
\bibitem{Kraus} Kraus, K., `General State Changes in Quantum Theory',
{\em Ann. Phys.}, {\bf 64}, 311-335, 1971.
\bibitem{Gorini} Gorini, V., A. Kossakowski and E. C. G. Sudarshan,
`Completely positive dynamical semigroups of $n$-level systems',
{\em J. Math. Phys.}, {\bf 17}, 821-825, 1976.
\bibitem{Lindblad} Lindblad, G., {\em Commun. Math. Phys.}, {\bf 48},
119-130, 1976.
\bibitem{Landau2} Landau, L. J., and R. F. Streater, `On Birkhoff's
theorem...', {\em Linear Algebra and its applications}, {\bf 193},
107-127, 1993.
\bibitem{Fannes} Fannes, M., and J. Quaegebure, `Infinite divisibility
and central limit theorems for completely positive mappings', {\bf Quantum
Probability and Applications}, LNM 1136, Ed. L. Accardi and W. von
Waldenfels, Springer-Verlag, 1985.
\bibitem{Stinespring} Stinespring, W. F., `Positive functions on
$C^*$-algebras', {\em Proc. Amer. Math. Soc.}, {\bf 6}, 242-247, 1955.
\bibitem{Umegaki} Umegaki, H., `Conditional expectation in an operator
algebra', {\em Tohoku Math. J.}, {\bf 8}, 86-100, 1956.
\bibitem{Barnett} Barnett, C., `Supermartingales on semi-finite von
Neumann algebras', {\em J. Lond. Math. Soc.}, {\bf 24}, 175-181, 1981.
\bibitem{Cuculescu} Cuculescu, I., `Martingales on von Neumann algebras',
{\em J. Multivariate Analysis}, {\bf 1}, 17-27, 1971.
\bibitem{Hudson} Hudson, R. L. and R. F. Streater, `Examples of
Quantum Martingales', {\em Phys. Lett.}, {\bf 85A}, 64-67, 1981.
\bibitem{Wulfsohn} Streater, R. F., and A. Wulfsohn, `Continuous tensor
products of Hilbert spaces and generalised random fields', {\em Nuovo Cimento},
{\bf 57}, 330-339, 1968.
\bibitem{Hudson4} Hudson, R. L. and R. F. Streater, `Noncommutative
martingales and
stochastic integrals in Fock space', pp 216-222 in {\bf Stochastic Processes
in Quantum Theory and Statistical Physics}, {\em Lecture Notes in Physics},
{\bf 173}, eds S. Albeverio, Ph. Combe and M. Sirugue-Collin,
Springer-Verlag, 1981.
\bibitem{Barnett2} Barnett, C., R. F. Streater and I. F. Wilde, `The
Ito-Clifford integral', {\em J. Functl. Anal.}, {\bf 48}, 172-212, 1982.
\bibitem{Segal0} Segal. I. E. `A non-commutative extension of abstract
integration', {\em Ann. of Math.}, {\bf 57}, 401-457, and 595-596, 1953.
\bibitem{Segal2} Segal, I. E., `Tensor algebras over Hilbert space II',
{\em Annals of Math.}, {\bf 63}, 160-175, 1956.
\bibitem{Kunze} R. A. Kunze and I. E. Segal, {\bf Integrals and
Operators}, Springer-Verlag, 1978.
\bibitem{Barnett5} Barnett, C., R. F. Streater and I. F. Wilde,
`The Ito-Cilfford integral II' {\em J. Lond. Math. Soc.}, {\bf 27},
373-384, 1983.
\bibitem{Barnett6} Barnett, C., R. F. Streater and I. F. Wilde, `The
Ito-Clifford integral IV' {\em J. Operator Theory}, {\bf 11}, 255-271, 1984.
\bibitem{Pisier} Pisier, G., and Q. Xu, `Non-commutative martingale
inequalities', {\em Commun. Math. Phys.}, {\bf 189}, 667-698, 1997.
\bibitem{Barnett4} Barnett, C., R. F. Streater and I. F. Wilde, `Quasi-free
quantum stochastic integrals for the {\em CAR} and {\em CCR}',
{\em J. Funct. Anal.}, {\bf 52}, 19-47, 1983.
\bibitem{Lindsay} Lindsay, M. and I. F. Wilde, `On non-Fock boson
stochastic integrals', {\em J. Functl. Anal.}, {\bf 65}, 76-82, 1986.
\bibitem{Meyer2} Meyer, P.-A., Private communication, Oberwohlfach, 1985.
\bibitem{Evans} Evans, M. P., and R. L. Hudson, `Multidimensional quantum
diffusions', LNM, {\bf 1303}, 69-88, Springer-Verlag, 1988.
\bibitem{Smith} Vincent-Smith, G. F., `Unitary quantum stochastic evolutions',
{\em Proc. London Math. Soc.} {\bf 63}, 401-425, 1991.
\bibitem{Belavkin} Belavkin, V. P., `Quantum stochastic positive
evolutions', {\em Commun. Math. Phys.}, {\bf 184}, 533-566, 1997. `On stochastic
generators of completely positive cocycles', {\em Russian J. of Math.
Phys.}, {\bf 3}, 523-528, 1995.
\bibitem{Frigerio} Frigerio, A., `Construction of stationary quantum Markov
processes', pp 207-222 in {\bf Quantum Probability and Applications II},  
LNM {\bf 1136}, (eds. L. Accardi and W. von Waldenfels), Springer-Verlag,
1985.
\bibitem{Hudson2} Hudson, R. L. and R. F. Streater, `Ito's formula is
the chain rule with Wick ordering,' {\em Phys. Lett.}, {\bf 86A}, 277-279,
1981.
\bibitem{Applebaum} Applebaum, D. and R. L. Hudson, `Fermion Ito's formula
and stochastic evolutions', {\em Commun. Math Phys.}, {\bf 96},
473-496, 1984.
\bibitem{Maassen} Maassen, H. `Quantum Markov processes on Fock space
described by integral kernels', {\bf Quantum Probability II}, 361-374, 1985,
Lecture Notes in Mathematics 1136, eds. L. Accardi and W. von Waldenfels,
Springer-Verlag.
\bibitem{Obata} Obata, N., 'Wick product of white noise operators and
quantum stochastic differential equations', {\em J. Math. Soc. Japan},
{\bf 51}, 613-641, 1999.
\bibitem{Talkner} Talkner, P. `Failure of the quantum regression
theorem', {\em Ann. of Phys.}, {\bf 167}, 390-436, 1986.
\bibitem{Lindblad2} Lindblad, G. `Brownian Motion of quantum harmonic
oscillators', {\em J. Math. Phys.}, {\bf 39}, 2763-2780, 1998.
\bibitem{Alicki} Alicki, R. and K. Lendl, {\bf Quantum Dynamical Semigroups
and Applications}, Lecture Notes in Physics, {\bf 286}, Springer-Verlag,
1987.
\bibitem{AlickiM} Alicki, R., and J. Messer, `Nonlinear quantum dynamical 
semigroups for many-body open systems', {\em J. Stat. Phys.}, {\bf 32},
299-312, 1983.
\bibitem{Balian} Balian, R., Y. Alhassid and R. Reinhardt,
`Dissipation in many-body systems: a geometric approach based on
information theory', {\em Physics Reports}, {\bf 131}, 1-146, 1986.
\bibitem{Ingarden3} Ingarden, R. S., Y. Sato, K. Sagura and T. Kawaguchi,
{\em Tensor}, {\bf 33}, 347-, 1979.
\bibitem{RFS7} Streater, R. F. `Onsager symmetry in statistical dynamics',
{\em Open Systems and Information Dynamics}, {\bf 6}, 87-100, 1999.

\end{thebibliography}


\end{document}
