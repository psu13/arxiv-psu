\usepackage{latexsym}
%\usepackage[danish]{babel}      % Danske overskrifter
\usepackage[latin1]{inputenc}   % Danske tegn
%\usepackage{array}
\usepackage{amsmath}
\usepackage{epsfig}
\usepackage{graphics}
\usepackage{amsfonts}
\usepackage{pifont}
\usepackage{booktabs}
\usepackage{xspace}
\usepackage{color}
\usepackage{booktabs}
\usepackage{xtab}
\usepackage{eurosym}
\usepackage{makeidx}
\usepackage{fancyhdr}

%\usepackage{palatino}
%\usepackage{bookman}
%\usepackage{chancery}
%\usepackage{utopia}
%\usepackage{newcent}
%\usepackage{times}
\usepackage{charter}
%\renewcommand{\familydefault}{phv}  
%\renewcommand{\familydefault}{pac}  
%\renewcommand{\familydefault}{pad}  
%\renewcommand{\familydefault}{pag}  
%\renewcommand{\familydefault}{pbb}  
%\renewcommand{\familydefault}{pfr}  
\newlength{\cvskip}
\newcommand{\tsc}[1]{\textsc{#1}}
\definecolor{lgrey}{rgb}{0.4,0.4,0.4}            
%%
\newcommand{\tb}[1]{\textbf{#1}\xspace}
\newcommand{\dt}[1]{{\small #1}}
\newcommand{\pskip}{\vskip4mm}
\newcommand{\sskip}{\vspace*{2mm}}
\newcommand{\heading}[1]{
\begin{center}
\textcolor{lgrey}{\LARGE #1}
\end{center}
}
\newcommand{\sheading}[1]{\pskip\noindent{\underline{#1}}\newline}
\newcommand{\entry}[2]{\noindent\parbox[c]{\textwidth}{
#1}\\[1.8\itemsep]}
\newcounter{bibnumber}
\setcounter{bibnumber}{0}
\newcounter{impnumber}
\setcounter{impnumber}{0}
\newcommand{\bibentry}[5]{\refstepcounter{bibnumber}\label{#1}\noindent
\begin{minipage}{1.0cm}
\flushright
[\thebibnumber]\hspace*{3mm}
\end{minipage}\parbox[t]{0.93\textwidth}{
{\bf #2}\\ \textcolor{lgrey}{By} #3\\ 
#4 \textcolor{lgrey}{\hfill #5}
}\sskip\newline}
\newcommand{\impentry}[4]{\refstepcounter{impnumber}\label{#1}\noindent
\begin{minipage}{1.0cm}
\flushright
[\theimpnumber]\hspace*{3mm}
\end{minipage}\parbox[t]{0.93\textwidth}{
{\bf #2}\\ \textcolor{lgrey}{By} #3\\ 
#4
}\sskip\newline}
\newtheorem{pub}{}
\newcommand{\bref}[1]{[\ref{#1}]\xspace} 
\vfuzz2pt % Don't report over-full v-boxes if over-edge is small

\hfuzz8pt % Don't report over-full h-boxes if over-edge is smallish
%
%EXTERNAL GRAPHICS
\usepackage{feynmp}
% Needed to make feynmp work with pdflatex
\DeclareGraphicsRule{*}{mps}{*}{}

\usepackage{cite}
\usepackage{xspace}
\usepackage{booktabs}
\usepackage[small, nooneline, center]{subfig}
\usepackage{amsmath,amssymb,bm}
\usepackage{longtable}
% The following two lines must be uncommented for arXiv
%\usepackage[colorlinks=true, urlcolor=blue, linkcolor=blue, citecolor=blue]{hyperref}
%\hypersetup{pdfauthor={Peter Skands},pdftitle={Introduction to QCD}}

%\DeclareGraphicsRule{*}{mps}{*}{} 
\usepackage{mciteplus}
\definecolor{lightgray}{rgb}{0.85,0.85,0.87}
\definecolor{gray}{rgb}{0.5,0.5,0.5}
\definecolor{darkgray}{rgb}{0.36,0.36,0.36}%Lengths
%\addtolength{\headheight}{-0.5cm}
%\addtolength{\textheight}{0.5cm}
%
\def\lsim{\mathrel{\rlap{\lower4pt\hbox{\hskip1pt$\sim$}}
    \raise1pt\hbox{$<$}}}                % less than or approx. symbol
\def\gsim{\mathrel{\rlap{\lower4pt\hbox{\hskip1pt$\sim$}}
    \raise1pt\hbox{$>$}}}                % greater than or approx. symbol
\newcommand{\dd}[1]{\ensuremath{\mrm{d}#1}\hspace*{0.2em} }
\newcommand{\drm}{\ensuremath{\mathrm{d}}}
\newcommand{\mbf}[1]{\ensuremath{\mathbf{#1}}}
\newcommand{\mrm}[1]{\ensuremath{\mathrm{#1}}}
\newcommand{\mop}[2][]{\ensuremath{\{p\}_{#2}^{#1}}}
\newcommand{\mom}[3][]{\ensuremath{\{#1\}_{#3}^{#2}}}
\newcommand{\obs}{\ensuremath{\mathcal{O}}}
\newcommand{\dobs}{\ensuremath{\dd{\obs}}}
\newcommand{\pt}[1][]{\ensuremath{p_{\perp #1}}}
\newcommand{\pthat}[1][]{\ensuremath{\hat{p}_{\perp #1}}}
\newcommand{\ptmin}[1][]{\pt[\mrm{min}#1]\xspace}
\newcommand{\pdf}[1]{\ensuremath{f_{#1}}}
\newcommand{\xpdf}[1][]{\ensuremath{x\!f_{#1}}}
\newcommand{\pqcd}[2][]{\ensuremath{\sigma^{(#1)}_{#2}}}
\newcommand{\PS}[2][]{\ensuremath{\Phi_{#2}^{#1}}\xspace}
\newcommand{\dPS}[2][]{\ensuremath{\dd{\PS[#1]{#2}}}\xspace}
\newcommand{\sig}{\ensuremath{\sigma}}
\newcommand{\dsig}{\ensuremath{\dd{\sig}}}
\newcommand{\ttt}[1]{\texttt{#1}}
% References
\newcommand{\eqRef}[1]{equation~\eqref{#1}\xspace}
\newcommand{\EqRef}[1]{Equation~\eqref{#1}\xspace}
\newcommand{\eqsRef}[1]{equations~\eqref{#1}\xspace}
\newcommand{\EqsRef}[1]{Equations~\eqref{#1}\xspace}
\newcommand{\secRef}[1]{section~\ref{#1}\xspace}
\newcommand{\SecRef}[1]{Section~\ref{#1}\xspace}
\newcommand{\secsRef}[1]{sections~\ref{#1}\xspace}
\newcommand{\SecsRef}[1]{Sections~\ref{#1}\xspace}
\newcommand{\tabRef}[1]{table~\ref{#1}\xspace}
\newcommand{\TabRef}[1]{Table~\ref{#1}\xspace}
\newcommand{\figRef}[1]{figure~\ref{#1}\xspace}
\newcommand{\FigRef}[1]{Figure~\ref{#1}\xspace}
\newcommand{\tabsRef}[1]{tables~\ref{#1}\xspace}
\newcommand{\TabsRef}[1]{Tables~\ref{#1}\xspace}
\newcommand{\figsRef}[1]{figures~\ref{#1}\xspace}
\newcommand{\FigsRef}[1]{Figures~\ref{#1}\xspace}
% Monte Carlos
\newcommand{\Al}{\tsc{Alpgen}\xspace}
\newcommand{\Ar}{\tsc{Ariadne}\xspace}
\newcommand{\Co}{\tsc{Comphep}\xspace}
\newcommand{\Ca}{\tsc{Calchep}\xspace}
\newcommand{\Fw}{\tsc{Mc@nlo}\xspace}
\newcommand{\Hw}{\tsc{Herwig}\xspace}
\newcommand{\Mcfm}{\tsc{Mcfm}\xspace}
\newcommand{\Mg}{\tsc{Madgraph}\xspace}
\newcommand{\Py}{\tsc{Pythia}\xspace}
\newcommand{\Sh}{\tsc{Sherpa}\xspace}
\newcommand{\Vc}{\tsc{Vincia}\xspace}
\newcommand{\Qgsjet}{\tsc{Qgsjet}\xspace}
\newcommand{\Dpmjet}{\tsc{Dpmjet}\xspace}
\newcommand{\Epos}{\tsc{Epos}\xspace}
\newcommand{\Sibyll}{\tsc{Sibyll}\xspace}
\newcommand{\Phojet}{\tsc{Phojet}\xspace}
\newcommand{\Pw}{\tsc{Powheg}\xspace}
\newcommand{\Herwig}{\Hw}
\newcommand{\Pythia}{\Py}
% Lengths
\newlength{\tabcolsepsave}
% Environments
\newenvironment{loopsnlegs}[1][t]{
\setlength{\tabcolsepsave}{\tabcolsep}
\setlength{\tabcolsep}{0pt}
\begin{tabular}{cc}\parbox[c]{1.1em}{\rotatebox{90}{\small $\ell$ (loops)}}&%
\begin{tabular}[#1]}{
\end{tabular}\\[-1mm]
 & \small $k$ (legs)
\end{tabular}%
\setlength{\tabcolsep}{\tabcolsepsave}
}
\newcommand{\cbox}[2]{%
\begin{minipage}[c]{1.4cm}%
\center%
%\colorbox{gray}
{%
\parbox[c]{1.4cm}{\includegraphics*[width=1.4cm]{#1.pdf}}}%
\end{minipage}%
\hspace*{-1.4cm}%
\begin{minipage}[c]{1.4cm}
\center
#2%
\end{minipage}}
\newcommand{\cyanbox}[1]{\cbox{cbox}{#1}}
\newcommand{\eggbox}[1]{\cbox{eggbox}{#1}}
\newcommand{\gbox}[1]{\cbox{greenbox}{#1}}
\newcommand{\wbox}[1]{\cbox{whitebox}{#1}}
\newcommand{\gwbox}[1]{\cbox{gwbox}{#1}}
\newcommand{\gybox}[1]{\cbox{gybox}{#1}}
\newcommand{\gywbox}[1]{\cbox{gywbox}{#1}}
\newcommand{\ybox}[1]{\cbox{yellowbox}{#1}}
\newcommand{\ywbox}[1]{\cbox{ywbox}{#1}}
\newcommand{\rybox}[1]{\cbox{rybox}{#1}}
\newcommand{\wybox}[1]{\cbox{wybox}{#1}}
\newcommand{\wywbox}[1]{\cbox{wybox2}{#1}}
\newcommand{\wwybox}[1]{\cbox{wybox3}{#1}}

\makeindex
