
\section{Introduction \label{sec:introduction}}

When probed at very short wavelengths, QCD is essentially a theory of
free \index{Partons}`partons' --- quarks and gluons --- which only
scatter off one another through relatively small quantum corrections,
that can be systematically calculated. 
But at longer wavelengths, of order the size of the proton $\sim
1\mathrm{fm} = 10^{-15}\mathrm{m}$,  
we see strongly bound towers of hadron resonances emerge, with string-like
potentials building up if we try to separate their partonic
constituents. Due to our
inability to perform analytic calculations in 
strongly coupled field theories, QCD is therefore 
still only partially solved. Nonetheless,  all its features, across all
distance scales, are believed to be encoded in a single one-line
formula of alluring simplicity; the
\index{QCD!Lagrangian}%
Lagrangian\footnote{Throughout these notes we let it be implicit that
  ``Lagrangian'' really refers to Lagrangian density, ${\cal L}$, the
  four-dimensional space-time integral of which is the action.} of QCD.

The consequence for collider physics is that some parts of QCD can be
calculated in terms of the fundamental parameters of the Lagrangian,
whereas others must be expressed through models or functions whose effective 
parameters are not a priori calculable but which can be constrained
by fits to data. 
However, even in the absence of a
perturbative expansion, there are still several strong theorems which
hold, and which can be used to give relations between seemingly
different processes. (This is, e.g., the reason it makes sense to 
measure the partonic substructure of the proton in $ep$ collisions and
then re-use the same parametrisations for $pp$
collisions.) Thus, in the chapters 
dealing with phenomenological models we shall emphasise that the loss
of a factorised perturbative expansion is not equivalent to a total
loss of predictivity.   

An alternative approach would be to give up on calculating QCD 
and use leptons instead. Formally, this amounts to summing inclusively over
strong-interaction phenomena, when such are present. While such a
strategy might succeed in replacing what we do know about QCD by
``unity'', however, even the most adamant chromophobe would acknowledge
the following basic facts of collider physics for the next decade(s): 
1) At the LHC, the initial states are
hadrons, and hence, at
the very least, well-understood and precise parton distribution
functions (PDFs) will be required; 2) high precision will mandate
 calculations to higher orders in perturbation theory, 
which in turn will involve more QCD; 3) the requirement of lepton
\emph{isolation} makes the very definition of a lepton
 depend implicitly on QCD and 4) 
 the rate of jets that are misreconstructed as leptons in
 the experiment depends explicitly on it. 
And, 5) though many new-physics signals \emph{do} give observable
signals in the lepton sector, this is far from guaranteed, nor is it
exclusive when it occurs. 
 It would therefore be  unwise not to attempt to solve QCD to the best
 of our ability, the better to prepare ourselves for both the largest
 possible discovery reach and the highest attainable subsequent
 precision. 

Furthermore, QCD is the richest gauge theory we have so far
 encountered. Its emergent phenomena, unitarity properties, colour structure, 
 non-perturbative dynamics, quantum vs.\ classical limits, 
interplay between scale-invariant and
 scale-dependent properties, and its wide
 range of phenomenological applications, are still very much topics of
 active investigation, about which we continue to learn.  

In addition, or perhaps as a consequence, the field of QCD is
currently experiencing something of a revolution. On the perturbative
side, new methods to compute scattering amplitudes with very high
particle multiplicities are being developed, together with advanced
techniques for combining such amplitudes with all-orders resummation
frameworks. On the non-perturbative side, the wealth of data on
soft-physics processes from the LHC is
forcing us to reconsider the reliability of the standard fragmentation
models, and heavy-ion collisions are providing new insights into
the collective behavior of hadronic matter. The
study of cosmic rays impinging on the Earth's
atmosphere challenges our ability to extrapolate fragmentation models
from collider energy scales to the region of ultra-high energy cosmic
rays. And finally, dark-matter annihilation processes in space  may produce 
hadrons, whose spectra are sensitive to the modeling 
of fragmentation.

In the following, we shall focus on QCD for mainstream 
collider physics. This includes the basics of SU(3), colour factors, the running
of $\alpha_s$, factorisation, 
hard processes, infrared safety, parton showers and matching, event generators, hadronisation, and the so-called underlying event. 
While not covering everything, hopefully these topics can also serve
at least as stepping stones to more specialised
issues that have been left out, such as twistor-inspired techniques, 
heavy flavours, polarisation, or forward physics, or to topics more tangential to
other fields, such as axions, lattice QCD, or heavy-ion physics.  

\subsection{A First Hint of Colour}
Looking for new physics, as we do now at the LHC, it is instructive to 
consider the story of the discovery of colour. The first hint was
arguably the $\Delta^{++}$ \index{Baryons}baryon, discovered in 
1951~\cite{Brueckner:1952zz}. The title and part of the abstract from this
historical paper are reproduced in \figRef{fig:Delta}.
\begin{figure}[t]
\begin{center}
\begin{tabular}{c}
\colorbox{gray}{\includegraphics*[scale=0.75]{DeltaTitle.pdf}}\\[5mm]
\hspace*{2mm}\begin{minipage}{0.88\textwidth}
\small\sl  ``[...] It is concluded that the apparently anomalous features of the
scattering can be interpreted to be an indication of a resonant
meson-nucleon interaction corresponding to a nucleon isobar with spin
$\frac32$, isotopic spin $\frac32$, and with an excitation energy of
$277\,$MeV.''\\[1mm]
\end{minipage}
\end{tabular}
\caption{The title and part of the abstract of the 1951 paper
  \cite{Brueckner:1952zz} (published in 1952) in which the existence 
  of the $\Delta^{++}$ baryon was deduced, based on data from Sachs and
  Steinberger at Columbia~\cite{Chedester:1951sc}  and from Anderson,
  Fermi, Nagle, et al.~at Chicago~\cite{Fermi:1952zz}. Further studies 
  at Chicago were quickly performed
  in~\cite{Anderson:1952nw,Anderson:1952zza}. See also the memoir by
  Nagle~\cite{nagle1984delta}. 
\label{fig:Delta}}  
\end{center}
\end{figure}
In the context of the \index{Quarks}quark model --- which first
had to be developed, successively joining together the notions of 
spin, isospin, strangeness, and 
the \index{Eightfold way}eightfold way\footnote{In physics, the ``eightfold way''
refers to the classification of the lowest-lying pseudoscalar
\index{Mesons}mesons and 
\index{SU(3)!Of Flavour}%
spin-1/2 \index{Baryons}baryons within \index{Octet}octets in SU(3)-flavour space ($u,d,s$). The
$\Delta^{++}$ is part of a spin-3/2 baryon \index{Decuplet}decuplet, a ``tenfold way'' in this
terminology.} 
--- the \index{Flavour}flavour and spin content of the $\Delta^{++}$
baryon is: 
\begin{equation}
\left\vert \Delta^{++} \right> = \left\vert
\,u_\uparrow\ u_\uparrow\ u_\uparrow \right>~,
\end{equation} 
clearly a highly symmetric configuration. However, since 
the $\Delta^{++}$ is a fermion, it must have an overall
antisymmetric wave function. In 1965, fourteen years after its
discovery, this was finally understood by the introduction of colour
\index{SU(3)}%
\index{SU(3)!Of Colour}%
as a new quantum number associated with the group SU(3)
\cite{Greenberg:1964pe,Han:1965pf}. The $\Delta^{++}$ wave function can now be made
antisymmetric by arranging its three quarks antisymmetrically 
in this new degree of freedom, 
\begin{equation}
\left\vert \Delta^{++} \right> = \epsilon^{ijk} \left\vert
\,u_{i\uparrow}\ u_{j\uparrow}\ u_{k\uparrow}\right>~,
\end{equation} 
hence solving the mystery.

More direct experimental tests of the number of colours were provided first by
measurements of the decay width of $\pi^0\to \gamma\gamma$ decays, which 
is proportional to $N_C^2$, 
and later by the famous ``R'' ratio in
$e^+e^-$ collisions ($R=\sigma(e^+e^-\to q\bar{q})/\sigma(e^+e^-\to
\mu^+\mu^-)$), which is proportional to $N_C$, see
e.g.~\cite{Dissertori:2003pj}. 
Below, in \SecRef{sec:L} we shall see how to
calculate such colour factors. 

\subsection{The Lagrangian of QCD \label{sec:L}}
\index{QCD!Lagrangian}%
Quantum Chromodynamics is based on the gauge group
\index{SU(3)}$\mrm{SU(3)}$, the 
Special Unitary group in 3 (complex) dimensions, whose elements 
are the set of unitary $3\times 3$ matrices with determinant one. 
\index{Fundamental representation}%
\index{SU(3)!Fundamental representation}%
Since there are 9 linearly independent unitary complex
matrices\footnote{A complex $N\times N$ matrix has $2N^2$ degrees of
  freedom, on which unitarity provides $N^2$ constraints.}, one of
which has determinant $-1$, there are a total of 8
independent directions in this matrix space, corresponding to eight
different generators as compared
with the single one of QED. In the context of QCD, we normally
represent this group using the 
so-called \emph{fundamental}, or \emph{defining}, representation, in
which the generators of $\mrm{SU(3)}$ appear as a set of eight traceless and
hermitean matrices, to which we return below.  
We shall refer to indices enumerating
the rows and columns of these matrices  (from 1 to 3) as
\emph{fundamental} indices, and we use the letters $i$,
$j$, $k$, \ldots, to denote them.
\index{Adjoint representation}%
\index{SU(3)!Adjoint representation}%
We refer to indices enumerating the generators (from 1 to 8),
as \emph{adjoint} 
indices\footnote{The dimension of the \emph{adjoint}, or
  \emph{vector}, representation is equal to the number of generators,
  $N^2-1=8$ for $\mrm{SU(3)}$, while the  
\index{Fundamental representation}%
\index{SU(3)!Fundamental representation}%
dimension of the fundamental representation is
  the degree of the group, $N=3$ for $\mrm{SU(3)}$.}, and we use the first
letters of the alphabet ($a$, $b$, $c$, \ldots) to denote them. 
These matrices can operate both on each other (representing
combinations of successive gauge transformations) and on a set of
$3$-vectors, the latter of 
which represent \index{Quarks}quarks in colour 
space; the quarks are \emph{triplets} under $\mrm{SU(3)}$. The matrices can be
thought of as representing gluons in colour 
space (or, more precisely, the gauge transformations carried out by
gluons), hence there are
eight different gluons; the gluons are \emph{octets} under $\mrm{SU(3)}$. 

\index{QCD!Lagrangian}%
The Lagrangian density of QCD is 
\begin{equation}
{\cal L} = \bar{\psi}_q^i(i\gamma^\mu)(D_\mu)_{ij}\psi_q^j - m_q
\bar{\psi}_q^i\psi_{qi} - \frac14 F^a_{\mu\nu}F^{a\mu\nu}~,\label{eq:L}
\end{equation}
where $\psi_q^i$ denotes a quark field with
(fundamental) colour index $i$, 
$\psi_q = ({\textcolor{red}{\psi_{qR}}},{\color{green}\psi_{qG}}, 
{\color{blue}\psi_{qB}})^T$, 
$\gamma^\mu$ is a Dirac matrix that expresses the
vector nature of the strong interaction, with $\mu$ being a Lorentz
vector index, $m_q$ allows for the
possibility of non-zero \index{Quarks}quark masses (induced by the
standard Higgs 
mechanism or similar), $F^a_{\mu\nu}$ is the gluon field strength 
tensor for a gluon\footnote{The definition of the gluon field strength
  tensor will be given below in \eqRef{eq:F}.} with (adjoint) 
colour index $a$ (i.e., $a\in[1,\ldots,8]$), 
and $D_\mu$ is the covariant derivative in QCD,
\begin{equation}
(D_{\mu})_{ij} = \delta_{ij}\partial_\mu - i g_s t_{ij}^a A_\mu^a~,\label{eq:D}
\end{equation}
\index{QCD!Coupling}
with $g_s$ the \index{alphaS@$\alpha_s$}strong coupling (related to
$\alpha_s$ by $g_s^2 = 4\pi 
\alpha_s$; we return to the strong coupling in more detail below), 
$A^a_\mu$  the gluon field with 
colour index $a$, and $t_{ij}^a$ proportional to the hermitean and
traceless \index{Gell-Mann matrices|see{SU(3)}}Gell-Mann matrices of $\mrm{SU(3)}$, 
\index{SU(3)!Generators}%
\begin{equation}
\mbox{\includegraphics*[scale=1.0]{gell-mann}}~.
\end{equation}
These generators are just the $\mrm{SU(3)}$ analogs of the
Pauli matrices in 
$\mrm{SU(2)}$. 
By convention, the constant of proportionality is normally
taken to 
be 
\begin{equation}
t^a_{ij} = \frac12 \lambda^a_{ij}~. \label{eq:t}
\end{equation}
\index{QCD!Coupling}
This choice in turn determines the normalisation of the coupling
$g_s$, via \eqRef{eq:D}, and
fixes the values of the $\mrm{SU(3)}$ \index{Casimirs}Casimirs and structure constants, to which we return below. 

An example of the colour flow for a
quark-gluon interaction in colour 
space is given in \figRef{fig:qg}.
\begin{figure}[t]
\begin{center}
\begin{minipage}[h]{4.6cm}
\begin{center}
$A^1_\mu$\\
\includegraphics*[scale=0.75]{qgv.pdf}\\[-3mm]
$\psi_{q\textcolor{green}{G}}$\hfill$\psi_{q\textcolor{red}{R}}$
\end{center}
\end{minipage}~~~
\parbox{0.4\textwidth}{
$
\begin{array}{ccccc}
\propto & - \frac{i}{2} g_s & \bar{\psi}_{q\color{red}R}  & \lambda^{1} & \psi_{q\color{green}G} 
\\[2mm]
= & -\frac{i}{2}g_s & \left(\begin{array}{ccc} \textcolor{red}{1} & \color{green} 0 &
  \color{blue} 0 
\end{array}\right) & 
\left(\begin{array}{ccc}
0 & 1 & 0  \\
1 & 0 & 0 \\
0 & 0 & 0
\end{array}\right) & 
 \left(\begin{array}{c}
\textcolor{red}{0} \\
\color{green}1 \\
\color{blue}0
\end{array}\right) \end{array}
$}
\caption{Illustration of a 
\index{Quarks}\index{Gluons}$qqg$ vertex in QCD, before
  summing/averaging over colours: a gluon in a state represented by $\lambda^1$
  interacts with quarks in the states $\psi_{qR}$ and
  $\psi_{qG}$. \label{fig:qg}}
\end{center}
\end{figure}
Normally, of course, we sum over all the colour indices, so this
example merely gives a pictorial representation of what one particular
(non-zero) term in the colour sum looks like.


\subsection{Colour Factors}
\index{QCD!Colour factors}
\index{Colour factors}%
\index{Colour-space indices|see{Colour connections}}%
\index{Matrix elements}%
Typically, we do not measure colour in the final state ---
instead we average over all possible incoming colours and sum over all
possible outgoing ones, wherefore QCD scattering amplitudes (squared) in
practice always contain sums over quark fields contracted with
\index{SU(3)!Generators}Gell-Mann matrices. These contractions in turn
produce traces  
which yield the \index{Colour factors}\emph{colour factors} that are associated to each QCD
process, and which basically count the number of ``paths through
colour space'' that the process at hand can take\footnote{The
  convention choice represented by \eqRef{eq:t} introduces a
  ``spurious'' factor of 2 for each power of the coupling $\alpha_s$. 
Although one could in principle absorb that factor into a redefinition
of the coupling, effectively redefining the normalisation of ``unit
colour charge'', the standard definition of $\alpha_s$ is now so
entrenched that alternative choices would be counter-productive, at
least in the context of a pedagogical review.}.

A very simple example of a colour factor is given by the decay process $Z\to
q\bar{q}$. This vertex contains a simple $\delta_{ij}$ in colour
space; the outgoing quark and antiquark must have identical 
(anti-)col\-ours. Squaring the corresponding matrix element and summing over
final-state colours yields a colour factor of
\begin{equation}
e^+e^-\to Z \to q\bar{q}~~~:~~~\sum_{\mrm{colours}}|M|^2 \propto
\delta_{ij}\delta_{ji} = \mrm{Tr}\{\delta\} = N_C = 3~,
\end{equation}
since $i$ and $j$ are quark (i.e., 3-dimensional
fundamental) indices. This factor corresponds directly to the 3 different
``paths through colour space'' that the process at hand can take; the
produced quarks can be red, green, or blue. 

A next-to-simplest example is given by $q\bar{q}\to
\gamma^*/Z\to\ell^+\ell^-$ (usually referred to as the
\index{Drell-Yan}Drell-Yan 
process~\cite{Drell:1970wh}),  
which is just a crossing of the previous one. By crossing
symmetry, the squared matrix element, including the colour factor, is
exactly the same as before, but since the quarks are here incoming, we
must \emph{average} rather than sum over their colours, leading to
\begin{equation}
q\bar{q}\to Z\to e^+e^-~~~:~~~\frac{1}{9}\sum_{\mrm{colours}}|M|^2 \propto \frac19\delta_{ij}\delta_{ji} = \frac19 \mrm{Tr}\{\delta\} = \frac13~,
\end{equation}
where the colour factor now expresses a \emph{suppression} which can
be interpreted as due to the fact that only quarks of matching colours
are able to collide and produce a $Z$ boson. The chance that a quark
and an antiquark picked at random from the colliding hadrons have 
matching colours is $1/N_C$. 
\begin{figure}[t]
\end{figure}

Similarly, $\ell q \to
\ell q$ via $t$-channel photon exchange (usually called Deep
Inelastic Scattering --- \index{DIS}\index{Deep inelastic scattering|see{DIS}}DIS --- with ``deep'' referring to a 
large virtuality of the exchanged photon), constitutes yet another
crossing of the same basic process, 
see \figRef{fig:Zcrossings}. \index{Colour factors}The colour factor in this case 
comes out as unity. 
\begin{figure}[t]
\centering\vspace*{-8mm}
\begin{tabular}{ccc}
\rotatebox{360}{\includegraphics*[scale=0.93]{ee2qq}} \ \ 
& \ \ \includegraphics*[scale=0.93,angle=180,origin=c]{ee2qq}
\ \ & \ \ \includegraphics*[scale=0.9,angle=297,origin=c]{ee2qq}\\
Hadronic $Z$ decay & \index{Drell-Yan}Drell-Yan & \index{DIS}DIS \\[1mm]
$e^-e^+ \to \gamma^*/Z^0 \to q\bar{q}$ &
$q\bar{q} \to \gamma^*/Z^0 \to \ell^+\ell^-$ &
$\ell \bar{q} \stackrel{\gamma^*/Z^*}{\to} \ell \bar{q}$
\\[2mm] 
$\propto N_C$ & $\propto 1/N_C$ & $\propto 1$
\end{tabular}
\caption{Illustration of the three crossings of the interaction of a
  lepton current (black) with a \index{Quarks}quark current (red) 
  via an intermediate photon or
  $Z$ boson, with corresponding colour factors. \label{fig:Zcrossings}}
\end{figure}

To illustrate what happens when we insert (and sum over)
quark-gluon
vertices, such as the one depicted in \figRef{fig:qg}, we take
the process $Z\to3\,$jets. \index{Colour factors}The colour factor for
this process can be 
computed as follows, with the accompanying illustration showing a
corresponding diagram (squared) with explicit colour-space indices on
each vertex:\\
\index{Colour connections}
\begin{equation}
\mbox{
\begin{tabular}{cc}
\parbox{5.2cm}{
$Z \to qg\bar{q}$~~~:~~~\\
\[
\begin{array}{rcl}
\displaystyle\sum_{\mrm{colours}}|M|^2 & \propto & \displaystyle
\delta_{ij}t_{jk}^a t_{k\ell
    }^a\delta_{\ell i} \\
& = & \displaystyle
\mrm{Tr}\{t^at^a\}\\[4mm] & = & \displaystyle
  \frac12\mrm{Tr}\{\delta\} = 4~,
\end{array}
\]}
&
\parbox{8.5cm}{\includegraphics*[scale=0.6]{colFacZ3.pdf}
}
\end{tabular}}
\end{equation}
where the last $\mrm{Tr}\{\delta\} = 8$, since the trace runs over
the 8-dimensional adjoint indices. If we
want to ``count the paths through colour space'', we should leave out
the factor $\frac12$ which comes from the normalisation convention for
the $t$ matrices, \eqRef{eq:t}, hence this process can take 8
different paths through colour space, one for each gluon basis state.

The tedious task of taking traces over $t$
matrices can be greatly alleviated by use of the relations given in
\TabRef{tab:lambda}.  
\index{Traces in SU(3)|see{SU(3)}}%
\index{SU(3)!Trace relations}%
\index{QCD!Trace relations|see{SU(3)}}%
\begin{table}
\begin{center}
\scalebox{1.04}{\begin{tabular}{ccc}
\toprule
\index{SU(3)!Trace relations}Trace Relation & Indices & Occurs in Diagram Squared
\\
\midrule
$\mrm{Tr}\{t^at^b\} = T_R\, \delta^{ab}$ & $a,b\in[1,\ldots,8]$
& \parbox[c]{4cm}{\includegraphics*[scale=0.5]{traces1}}\\
$\sum_a t^a_{ij}t^a_{jk} = C_F\, \delta_{ik}$ &%
\parbox[c]{3cm}{\begin{center}
$a\in[1,\ldots,8]$\\
$i,j,k\in[1,\ldots,3]$\end{center}}
& \parbox[c]{4cm}{\includegraphics*[scale=0.5]{traces2}}\\
$\sum_{c,d} f^{acd} f^{bcd} = C_A\, \delta^{ab}$ & $a,b,c,d\in[1,\ldots,8]$
& \parbox[c]{4cm}{\includegraphics*[scale=0.5]{traces3}}\\
$ t^a_{ij}t^a_{k\ell} = T_R \left(\delta_{jk}\delta_{i\ell}
- \frac{1}{N_C}\delta_{ij}\delta_{k\ell}\right)$ & $i,j,k,\ell\in[1,\ldots,3]$
& \parbox[c]{4cm}{\includegraphics*[scale=0.5]{traces4}}\hspace*{-0.2cm}(Fierz)\\
\bottomrule
\end{tabular}}
\caption{Trace relations for $t$ matrices (convention-independent). 
 More relations
  can be found in \cite[Section 1.2]{Ellis:1991qj} and in 
  \cite[Appendix A.3]{Peskin:1995ev}.
\label{tab:lambda}}
\end{center}
\end{table}
In the standard normalisation convention for the \index{SU(3)}$\mrm{SU(3)}$ generators,
\eqRef{eq:t}, the \index{Casimirs}Casimirs of $\mrm{SU(3)}$ appearing in
\TabRef{tab:lambda} are\footnote{See, e.g., \cite[Appendix
    A.3]{Peskin:1995ev} for how to obtain the Casimirs in other
  normalisation conventions. As an example, choosing $t^a_{ij} = \lambda_{ij}^a/\sqrt{2}$ would yield $T_R=1$, $C_F=T_R(N_C^2-1)/N_C=8/3$, $C_A=3$.} 
\index{Casimirs}\index{TR@$T_R$}\index{CA@$C_A$}\index{CF@$C_F$}
\begin{equation}
T_R = \frac12 \hspace*{2cm} C_F = \frac43 \hspace*{2cm} C_A = N_C = 3~.
\end{equation}
In addition, the gluon self-coupling on the third line in
\TabRef{tab:lambda} involves factors of $f^{abc}$. These
\index{QCD!Structure constants|see{SU(3)}}%
are called the \index{SU(3)!Structure constants}\emph{structure constants} of QCD and they enter via 
the non-Abelian term in the \index{Gluons}gluon field strength tensor appearing in
\eqRef{eq:L}, 
\begin{equation}
F^a_{\mu\nu} = \underbrace{\partial_\mu A_\nu^a - \partial_\nu
  A^a_\mu}_{\mathrm{Abelian}} +
\underbrace{ g_s f^{abc} A_\mu^b A_\nu^c}_{\mathrm{non-Abelian}}~. \label{eq:F}
\end{equation}

\noindent\begin{minipage}[t]{0.46\textwidth}
The structure constants of $\mrm{SU(3)}$ are listed in the table to the
right. They define the \emph{adjoint}, or \emph{vector}, representation of $\mrm{SU(3)}$
and are related to the fundamental-representation generators via the
commutator relations
\begin{equation}
t^at^b - t^bt^a = [t^a,t^b] = i f^{abc} t_c~,
\end{equation} 
or equivalently,
\begin{equation}
if^{abc}~=~2\mrm{Tr}\{t^c[t^a,t^b]\}~.
\end{equation}
Thus, it is a matter of choice whether one prefers to express colour
space on a basis of fundamental-representation $t$ matrices, or via
the structure constants $f$, and one can go back and forth between the
two.
\end{minipage}%
\hfill%
\colorbox{darkgray}{%
\colorbox{lightgray}{%
\begin{minipage}[t]{0.46\textwidth}
\vspace*{3mm}\begin{center}
\textbf{Structure Constants of SU(3)}
\begin{equation}
f_{123} = 1
\end{equation}
\begin{equation}
f_{147} = f_{246} = f_{257} = f_{345} = \frac12
\end{equation}
\begin{equation}
f_{156} = f_{367} = -\frac12
\end{equation}
\begin{equation}
f_{458} = f_{678} = \frac{\sqrt{3}}{2}
\end{equation}
Antisymmetric in all indices\\[3mm]
All other $f_{abc}=0$\vspace*{3mm}\\
\end{center}
\end{minipage}%
}}\vskip1mm

\begin{figure}[t]
\begin{center}
\begin{minipage}[h]{4.6cm}
\begin{center}
$A_\nu^4(k_2)$\\
\includegraphics*[scale=0.75]{ggv.pdf}\\[-3mm]
$A^6_\rho(k_1)$\hfill$A_\mu^2(k_3)$
\end{center}
\end{minipage}~~~
\parbox{0.35\textwidth}{
$
\begin{array}{cccc}
\propto & - g_s \ f^{246} \!\! & \!\! [ (k_3 - k_2)^\rho g^{\mu\nu}  \\ 
& & +(k_2 - k_1)^\mu g^{\nu\rho} \\ 
& &+(k_1 - k_3)^\nu g^{\rho\mu}]
\end{array}
$}\vspace*{1mm}
\caption{Illustration of a \index{Gluons}$ggg$ vertex in QCD, before
  summing/averaging over colours: interaction between gluons in the 
  states $\lambda^2$, $\lambda^4$, and $\lambda^6$ is represented by
  the structure constant $f^{246}$. 
\label{fig:gg}}
\end{center}
\end{figure}
 Expanding the $F_{\mu\nu}F^{\mu\nu}$ term of the
Lagrangian using \eqRef{eq:F}, we see that there is a 3-gluon and a
4-gluon vertex that involve $f^{abc}$, the latter of which has two
powers of $f$ and two powers of the coupling. 

Finally, the last line of \TabRef{tab:lambda} is not really a trace
relation but instead a useful so-called Fierz transformation, which
expresses products of $t$ matrices in terms of Kronecker $\delta$ functions. 
It is often used, for instance, in shower Monte Carlo
applications, to assist in mapping between colour flows in $N_C = 3$,
in which cross sections and splitting probabilities are calculated, 
and those in $N_C\to\infty$ (``leading colour''), used to represent colour flow in
the MC ``event record''.

A \index{Gluons}gluon self-interaction vertex is
illustrated in \figRef{fig:gg}, to be compared with the quark-gluon
one in \figRef{fig:qg}. We remind the reader that gauge boson
self-interactions are a hallmark of non-Abelian theories and that their
presence leads to some of the main differences between QED and
QCD. One should also keep in mind 
that the \index{Colour factors}colour factor for the vertex in \figRef{fig:gg}, \index{CA@$C_A$}$C_A$, 
is roughly twice as large as that for a quark, \index{CF@$C_F$}$C_F$.

\subsection{The Strong Coupling \label{sec:coupling}}
\index{QCD!Coupling}
\index{Jets}
\index{alphaS@$\alpha_s$}To first approximation, QCD is 
\index{QCD!Scale invariance}\emph{scale invariant}. That is, if one
``zooms in'' on a QCD jet, one will find a repeated self-similar 
pattern of jets within jets within jets, reminiscent of
fractals. 
In the context of QCD, this property was originally 
called \index{Lightcone scaling|see{QCD Scale invariance}}light-cone scaling, or 
\index{Bjorken scaling|see{QCD Scale invariance}}Bj{\o}rken scaling. 
This type of scaling is closely related to the class of
angle-preserving symmetries, called \index{Conformal
invariance}\emph{conformal} symmetries. In physics 
today, the terms ``conformal'' and ``scale invariant'' are used 
interchangeably\footnote{Strictly speaking, conformal symmetry is more
restrictive than just scale invariance, but examples of
scale-invariant field theories that are not conformal are rare.}.
Conformal invariance is a mathematical property of several
QCD-``like'' theories which are now being studied (such as $N=4$
supersymmetric relatives of QCD). It is also 
related to the physics of so-called ``unparticles'', though that is a
relation that goes beyond the scope of these lectures.

Regardless of the labelling, 
if the  \index{alphaS@$\alpha_s$}strong coupling did not run (we shall
return to the running 
of the coupling below), Bj{\o}rken scaling would be absolutely true. QCD
would be a theory with a fixed coupling, the same at all scales. 
This simplified picture already captures some of the most important
properties of QCD, as we shall discuss presently.  

\index{QCD!Scale invariance}%
In the limit of exact Bj{\o}rken scaling --- QCD at fixed coupling
--- properties of high-energy interactions are determined 
only by \emph{dimensionless} kinematic quantities, such as scattering
angles (pseudorapidities) and ratios of energy
scales\footnote{Originally, the observed approximate agreement with
this was used as a powerful argument
for pointlike substructure in hadrons; since measurements at different
energies are sensitive to different resolution scales, independence of the absolute
energy scale is indicative of the absence of other fundamental
scales in the problem and hence of pointlike constituents.}.
For applications of QCD to high-energy collider physics, an important
consequence of Bj{\o}rken scaling is thus that the rate of 
\index{Parton showers}%
\index{Bremsstrahlung|see{Parton showers}}
bremsstrahlung
jets, with a given transverse momentum, scales in direct proportion to
the hardness 
of the fundamental partonic scattering process they are produced in
association with. This agrees well with our intuition about accelerated
charges; the harder you ``kick'' them, the harder the radiation they
produce.  

For instance, in the limit of exact scaling, a
measurement of the rate of 10-GeV jets produced in association with an
ordinary $Z$ 
boson could be used as a direct prediction of the rate of 100-GeV jets
that would be 
produced in association with a 900-GeV $Z'$ boson, and so 
forth. Our intuition about how many bremsstrahlung jets a given type of
process is likely to have should therefore be governed first and
foremost by the \emph{ratios} of scales that appear in that particular
process, as has been  highlighted in a number of studies focusing on
the mass and $p_\perp$ scales appearing, e.g., in
Beyond-the-Standard-Model (BSM) 
physics processes
\cite{Plehn:2005cq,Alwall:2008qv,Papaefstathiou:2009hp,Krohn:2011zp}. 
\index{QCD!Scale invariance}Bj{\o}rken scaling 
\index{Scale invariance|see{QCD}}
is also fundamental to the understanding of jet substructure in QCD, see, e.g.,
\cite{Vermilion:2011nm,Altheimer:2012mn}.  

\index{alphaS@$\alpha_s$!Running coupling}%
On top of the underlying scaling behavior, the running coupling will
introduce a dependence on the absolute scale, implying more radiation
at low scales than at high ones. The running is logarithmic with
\index{alphaS@$\alpha_s$!beta function}%
energy, and is governed by the so-called \emph{beta function}, 
\index{alphaS@$\alpha_s$}
\begin{equation}
Q^2 \frac{\partial \alpha_s}{\partial Q^2} = \frac{\partial
  \alpha_s}{\partial \ln Q^2} =
\beta(\alpha_s)~, \label{eq:running}
\end{equation}
where the function driving the energy dependence, the \index{Beta function}{beta
  function}, is defined as
\begin{equation}
\beta(\alpha_s) = -\alpha_s^2(b_0 +
b_1\alpha_s + b_2\alpha_s^2 + \ldots)~,\label{eq:beta}
\end{equation}
with LO (1-loop) and NLO (2-loop) coefficients
\begin{eqnarray}
b_0 & = & \frac{11C_A - 4 T_R n_f}{12\pi}~,\\[3mm]
b_1 & = & \frac{17C_A^2 - 10 T_R C_A n_f - 6 T_R C_F n_f}{24\pi^2} ~=~
\frac{153-19 n_f}{24\pi^2}~.\label{eq:b}
\end{eqnarray}
In the $b_0$ coefficient, the first term is due to
\index{Gluons!Contribution to beta function}gluon loops while the
second is due to \index{Quarks!Contribution to beta function}quark
ones. Similarly, the first 
term of the $b_1$ coefficient arises from double gluon loops,
while the second and third represent mixed quark-gluon ones. 
At higher loop orders, the $b_i$ coefficients depend explicitly on the
renormalisation scheme that is used. A brief discussion can be found in the
PDG review on QCD~\cite{pdg2012}, with more elaborate ones
contained in \cite{Dissertori:2003pj,Ellis:1991qj}. 
Note that, if there are additional coloured particles beyond the
Standard-Model ones, loops involving those particles enter
 at energy scales above the masses of the
new particles, thus modifying the  \index{alphaS@$\alpha_s$}running of the coupling at high scales. 
This is discussed, e.g., for supersymmetric models in
\cite{Martin:1997ns}. For the running of other SM couplings, see
e.g.,~\cite{Langacker:2010zza}. 

\index{alphaS@$\alpha_s$!Running coupling}%
Numerically, the value of the  \index{alphaS@$\alpha_s$}strong coupling is usually specified by
giving its value at the specific 
reference scale $Q^2=M^2_Z$, from which we can obtain its
value at any other scale by solving \eqRef{eq:running}, 
\begin{equation}
\alpha_s(Q^2) = \alpha_s(M_Z^2) \frac{1}{1+b_0
  \alpha_s(M_Z^2)\ln\frac{Q^2}{M_Z^2} + {\cal O}(\alpha_s^2)}~,
\label{eq:alphaq2}
\end{equation}
with relations including the ${\cal O}(\alpha_s^2)$ terms 
available, e.g., in \cite{Ellis:1991qj}. 
Relations between scales 
not involving $M_Z^2$ can obviously be obtained by just replacing $M_Z^2$
by some other scale $Q'^2$ everywhere in \eqRef{eq:alphaq2}. A
comparison of running at one- and two-loop order, in both cases starting from
$\alpha_s(M_Z)=0.12$, is given in \figRef{fig:asRun}.
\begin{figure}[t]
\centering
\includegraphics*[scale=0.45]{vc-alphaS.pdf}
\caption{Illustration of the running of
 $\alpha_s$ at 1- (open 
  circles) and 2-loop
  order (filled circles), 
starting from the same value of $\alpha_s(M_Z)=0.12$. 
\label{fig:asRun}}
\end{figure}
As is evident from the figure, the 2-loop running is somewhat faster
than the 1-loop one.

\index{alphaS@$\alpha_s$!Running coupling}%
As an application, let us prove that the 
logarithmic running of the coupling implies that an intrinsically 
multi-scale problem can be converted to a single-scale one, up to
corrections suppressed by two powers of $\alpha_s$, 
by taking the geometric mean of the scales involved. This follows from
expanding an arbitrary product of individual  \index{alphaS@$\alpha_s$}$\alpha_s$ factors around an
arbitrary scale $\mu$, using \eqRef{eq:alphaq2}, 
\begin{eqnarray}
\alpha_s(\mu_1)\alpha_s(\mu_2)\cdots\alpha_s(\mu_n) & = &
\prod_{i=1}^{n} \alpha_s(\mu) \left(1 +
b_0\,\alpha_s\ln\left(\frac{\mu^2}{\mu_i^2}\right) + {\cal O}(\alpha_s^2)\right)
\nonumber\\[2mm]
& = & \alpha_s^n(\mu) \left(1 + b_0\, \alpha_s \ln \left(
 \frac{\mu^{2n}}{\mu_1^2\mu_2^2\cdots\mu_n^2}\right) +  {\cal
   O}(\alpha_s^2) \right)~,
\end{eqnarray}
whereby the specific single-scale choice $\mu^n =
\mu_1\mu_2\cdots\mu_n$ (the geometric mean) can
be seen to push the difference between the two sides of the equation one order higher
than would be the case for any other combination of scales\footnote{In
  a fixed-order calculation, the individual scales $\mu_i$,
would correspond, e.g., to the $n$ hardest scales appearing in an infrared
safe sequential clustering algorithm applied to the given momentum
configuration.}. 

The appearance of the number of \index{Flavour}flavours, $n_f$, in $b_0$ implies that the
slope of the running depends on the number of contributing
\index{Flavour}flavours. Since full QCD is best approximated by $n_f=3$
below the charm threshold, by $n_f=4$ and $5$ from there to the $b$
and $t$ thresholds, respectively, and then by $n_f=6$ at scales
higher than $m_t$, it is therefore important to be aware that 
the running changes slope across quark \index{Flavour}flavour
thresholds. Likewise, it would change across the threshold for any coloured
new-physics particles that might exist, with a magnitude depending on
the particles' colour and spin quantum numbers.

\index{alphaS@$\alpha_s$!Running coupling}%
\index{alphaS@$\alpha_s$}
The negative overall sign of \eqRef{eq:beta}, combined with the fact
that $b_0 > 0$ (for $n_f \le 16$), leads to the famous
result\footnote{
Perhaps the highest pinnacle of fame for \eqRef{eq:beta} was reached
when the sign of it featured in an episode of the TV series ``Big Bang
Theory''.} 
that the QCD coupling effectively \emph{decreases} with
 energy, called \index{Asymptotic freedom}asymptotic 
freedom, for the discovery of which the \index{Nobel prize}Nobel prize in physics was
awarded to D.~Gross, H.~Politzer, and F.~Wilczek in 2004. An extract
of the prize announcement runs as follows:
\begin{center}
\begin{minipage}{0.84\textwidth}
\sl  What this year's Laureates discovered was something that, at
first sight, seemed completely contradictory. The interpretation of
their mathematical result was that the closer the quarks are to each
other, the \emph{weaker} is the ``colour charge''. When the quarks are
really close to each other, the force is so weak that they behave
almost as free particles\footnote{More correctly, it is the coupling
  rather than the  
  force which becomes weak as the distance decreases. 
  The $1/r^2$ Coulomb singularity of the force is only dampened, not removed, 
  by the diminishing coupling.}. 
This phenomenon is called ``asymptotic
freedom''. The converse is true when the quarks move apart: the force
becomes stronger when the distance increases\footnote{More correctly,
 it is the potential which grows, linearly, while the force becomes
 constant.}. 
\end{minipage}
\end{center}

\index{Running coupling|see{alphaS@$\alpha_s$}}%
\index{alphaS@$\alpha_s$!Running coupling}%
Among the consequences of \index{Asymptotic freedom}asymptotic freedom is that perturbation
theory becomes better behaved at higher absolute energies, due to the
effectively decreasing coupling. Perturbative calculations for our
900-GeV $Z'$ boson from before should therefore be slightly faster
converging than equivalent calculations for the 90-GeV one. 
Furthermore, since the running of  \index{alphaS@$\alpha_s$}$\alpha_s$ explicitly
breaks Bj{\o}rken scaling, we also expect to see small changes in jet
shapes and in jet production ratios as we vary the energy. For
instance, since high-$p_\perp$ jets
start out with a smaller effective coupling, their intrinsic shape
(irrespective of boost effects) is
somewhat narrower than for low-$p_\perp$ jets, an issue which can be
important for jet calibration. Our current understanding of the
running of the QCD coupling is summarised by the plot in
\figRef{fig:alphas}, taken from a recent comprehensive review by S.\ Bethke
\cite{pdg2012,Bethke:2012jm}. A complementary up-to-date overview of
$\alpha_s$ determinations can be found in~\cite{d'Enterria:2015toz}. 

\index{alphaS@$\alpha_s$!Running coupling}%
As a final remark on \index{Asymptotic freedom}asymptotic freedom, note
that the decreasing 
value of the  \index{alphaS@$\alpha_s$}strong coupling with energy must eventually cause it to
become comparable to the electromagnetic and weak ones, at some energy
scale. Beyond that point, which may lie at energies of order
$10^{15}-10^{17}\,$GeV (though it may be lower if as yet undiscovered
particles generate large corrections to the running), 
we do not know  what the further evolution of the combined theory will 
actually look like, or whether it will continue to exhibit
\index{Asymptotic freedom}asymptotic
freedom. 

\index{alphaS@$\alpha_s$}%
\index{alphaS@$\alpha_s$!Running coupling}%
\index{alphaS@$\alpha_s$!LambdaQCD@$\Lambda_{\mathrm{QCD}}$}%
Now consider what happens when we run the coupling in the other
direction, towards smaller energies. 
\begin{figure}[t]
\begin{center}\hspace*{-0.25cm}
\parbox[c]{3.1cm}{\includegraphics*[scale=0.65]{arr-ir.pdf}}
\parbox[c]{8cm}{\includegraphics*[scale=0.5]{asq-2011.pdf}}\hspace*{-1mm}
\parbox[c]{3.1cm}{\includegraphics*[scale=0.65]{arr-uv.pdf}}
\caption{Illustration of the running of $\alpha_s$ in a theoretical
  calculation (band) and in physical processes at
  different characteristic scales, from
  \cite{pdg2012,Bethke:2012jm}. The little kinks at $Q=m_{c}$ and
  $Q=m_b$ are
  caused by discontinuities in the running across the flavour
  thresholds.\label{fig:alphas}}  
\end{center}           
\end{figure}
Taken at face value, the numerical value of the coupling diverges
rapidly at scales below 1 GeV, as illustrated by the curves
disappearing off the left-hand edge of the plot in
\figRef{fig:alphas}. To make this divergence
explicit, one can rewrite
\eqRef{eq:alphaq2} in the following form, 
 \index{alphaS@$\alpha_s$}
\begin{equation}
\alpha_s(Q^2) = \frac{1}{b_0 \ln \frac{Q^2}{\Lambda^2}}~,\label{eq:alphasLam}
\end{equation}
where 
\begin{equation}
\Lambda \sim 200\, \mbox{MeV}
\end{equation}
\index{alphaS@$\alpha_s$!LambdaQCD@$\Lambda_{\mathrm{QCD}}$}%
\index{alphaS@$\alpha_s$!Landau Pole|see{$\Lambda_{\mathrm{QCD}}$}}%
\index{LambdaQCD@$\Lambda_{\mathrm{QCD}}$|see{alphaS@$\alpha_s$}}%
specifies the energy scale at which the perturbative coupling would nominally become
infinite, called the Landau pole. (Note, however, that this only
parametrises the purely \emph{perturbative} result, which is not
reliable at \index{Strong coupling}strong coupling, so \eqRef{eq:alphasLam} should 
not be taken to imply that the physical behavior of full QCD should
exhibit a divergence for $Q\to \Lambda$.) 

\index{alphaS@$\alpha_s$}%
\index{alphaS@$\alpha_s$!Running coupling}%
\index{alphaS@$\alpha_s$!LambdaQCD@$\Lambda_{\mathrm{QCD}}$}%
Finally, one should be aware that there is a multitude of different
ways of defining both $\Lambda$ and $\alpha_s(M_Z)$. At the very
least, the numerical value one obtains depends both on the
renormalisation scheme used (with the dimensional-regularisation-based
``modified minimal subtraction'' scheme, $\overline{\mbox{MS}}$, being the
most common one) and on the perturbative order of the calculations 
used to extract them. As a rule of thumb, fits to experimental data typically yield 
smaller values for $\alpha_s(M_Z)$ the higher the order of the
calculation used to extract it (see, e.g.,
\cite{Bethke:2009jm,Dissertori:2009ik,Bethke:2012jm,pdg2012}), with  $
\alpha_s(M_Z)\vert_{\mrm{LO}} \gsim \alpha_s(M_Z)\vert_{\mrm{NLO}}
\gsim \alpha_s(M_Z)\vert_{\mrm{NNLO}}$. 
Further, since the number of \index{Flavour}flavours changes the slope
of the running, the location of the Landau pole for fixed
$\alpha_s(M_Z)$ depends explicitly on the number of \index{Flavour}flavours used in
the running. Thus each value of $n_f$ is associated with its own
value of $\Lambda$, with the following matching relations across
thresholds guaranteeing continuity of the coupling at one loop,
\index{LambdaQCD@$\Lambda_{\mathrm{QCD}}$|see{$\alpha_s$}}
\index{alphaS@$\alpha_s$!LambdaQCD@$\Lambda_{\mathrm{QCD}}$}%
\begin{eqnarray}
n_f = 5 \leftrightarrow 6 ~~~:~~~~~~\Lambda_6 = \Lambda_5
  \left(\frac{\Lambda_5}{m_t}\right)^{\frac{2}{21}} & & 
\Lambda_5 = \Lambda_6
  \left(\frac{m_t}{\Lambda_6}\right)^{\frac{2}{23}} ~, \\[2mm]
n_f = 4 \leftrightarrow 5 ~~~:~~~~~~\Lambda_5 = \Lambda_4
  \left(\frac{\Lambda_4}{m_b}\right)^{\frac{2}{23}} & & 
\Lambda_4 = \Lambda_5
  \left(\frac{m_b}{\Lambda_5}\right)^{\frac{2}{25}} ~, \\[2mm]
n_f = 3 \leftrightarrow 4 ~~~:~~~~~~\Lambda_4 = \Lambda_3 
  \left(\frac{\Lambda_3}{m_c}\right)^{\frac{2}{25}} & &
\Lambda_3 = \Lambda_4 
  \left(\frac{m_c}{\Lambda_4}\right)^{\frac{2}{27}} ~.
\end{eqnarray}

\index{alphaS@$\alpha_s$}%
\index{alphaS@$\alpha_s$!Running coupling}%
It is sometimes stated that QCD only has a single free
parameter, the  \index{alphaS@$\alpha_s$}strong coupling. 
However, even in the perturbative
region, the beta function depends explicitly on the number of
quark \index{Flavour}flavours, as we have seen, and thereby also on the quark
masses. Furthermore, in the non-perturbative region around or below
$\Lambda_{\mrm{QCD}}$, the value of the 
perturbative coupling, as obtained, e.g., from \eqRef{eq:alphasLam},
gives little or no insight into the behavior of the full theory. 
Instead, universal functions (such as parton densities, form factors,
fragmentation functions, etc), effective theories (such as the
Operator Product Expansion, Chiral Perturbation Theory, or Heavy Quark
Effective Theory), or phenomenological models (such as Regge Theory or
the String and Cluster Hadronisation Models) must be used, which in
turn depend on additional non-perturbative parameters whose relation to, e.g.,
$\alpha_s(M_Z)$, is not a priori known. 

\index{Lattice QCD}
For some of these questions,
such as hadron masses, lattice QCD can furnish important
additional insight, but for multi-scale and/or time-evolution
problems, the applicability of lattice methods is still severely
restricted; the lattice formulation of QCD requires 
  a Wick rotation to
  Euclidean space. The time-coordinate can then be treated on an
  equal footing with the other dimensions, but intrinsically
  Minkowskian problems, such as the time evolution of a system, are
   inaccessible. The limited size of current lattices
  also severely constrain the scale hierarchies that it is possible to
  ``fit'' between the lattice spacing and the lattice size. 

\index{Landau pole|see{$\alpha_s$}}%
\index{QCD!Landau Pole|see{$\alpha_s$}}%
\index{Renormalisation|see{$\alpha_s$}}%
\index{QCD!Renormalisation|see{$\alpha_s$}}%

\subsection{Colour States}
\index{Coherence}%
A final example of the application of the underlying $\mrm{SU(3)}$ group
theory to QCD is given by considering which colour states we can
obtain by combinations of quarks and gluons. The simplest example of
this is the combination of a quark and antiquark. We can form a total
of nine different colour-anticolour combinations, which fall into two
irreducible representations of $\mrm{SU(3)}$:
\begin{equation}
3 \otimes \overline{3} = 8 \oplus 1~.\label{eq:33bar}
\end{equation}
The singlet corresponds to the symmetric wave function 
$\frac{1}{\sqrt{3}}\left(\left|R\bar{R}\right>+\left|G\bar{G}\right>+\left|B\bar{B}\right>\right)$, 
which is invariant under $\mrm{SU(3)}$ transformations (the definition of a
singlet). The other eight linearly independent 
combinations (which can be represented by one for each Gell-Mann
matrix, with the singlet corresponding to the identity matrix) transform
into each other under $\mrm{SU(3)}$. Thus, although we sometimes talk about
colour-singlet states as 
being made up, e.g., of ``red-antired'', that is not quite precise
language. The actual state $\left|R\bar{R}\right>$ is \emph{not} a
pure colour singlet.  Although it does
have a non-zero \emph{projection} onto the singlet wave function
above, it also has non-zero projections onto the two members of
the octet that correspond to the diagonal Gell-Mann
matrices. Intuitively, one can also easily realise this by noting that
an $\mrm{SU(3)}$ rotation of $\left|R\bar{R}\right>$ would in general turn it into a
different state, say $\left|B\bar{B}\right>$, whereas a true colour singlet
would be invariant. 
Finally, we can also realise from \eqRef{eq:33bar} that a random
(colour-uncorrelated) quark-antiquark pair has a $1/N^2=1/9$ 
chance to be in an overall colour-singlet state; otherwise it is in
an octet. 

Similarly, there are also nine possible quark-quark (or
antiquark-antiquark) combinations, six of which are symmetric
under interchange of the two quarks and three of which are antisymmetric:
\index{Sextet}%
\begin{equation}
6 ~=~ \left(\begin{array}{c}
\left|RR\right>\\
\left|GG\right>\\
\left|BB\right>\\
\frac{1}{\sqrt{2}}\left(\left|RG\right> + \left|GR\right>\right)\\
\frac{1}{\sqrt{2}}\left(\left|GB\right> + \left|BG\right>\right)\\
\frac{1}{\sqrt{2}}\left(\left|BR\right> + \left|RB\right>\right)
\end{array}\right)
~~~~~~~~~
\bar{3} = \left(\begin{array}{c}
\frac{1}{\sqrt{2}}\left(\left|RG\right> - \left|GR\right>\right)\\
\frac{1}{\sqrt{2}}\left(\left|GB\right> - \left|BG\right>\right)\\
\frac{1}{\sqrt{2}}\left(\left|BR\right> - \left|RB\right>\right)
\end{array}\right)~.
\end{equation}
The members of the sextet transform into (linear combinations of) 
each other under $\mrm{SU(3)}$ transformations, and similarly for the
members of the antitriplet, hence neither of these can be reduced
further. The breakdown into
irreducible $\mrm{SU(3)}$ multiplets is therefore
\begin{equation}
3 \otimes 3 = 6 \oplus \overline{3}~.
\end{equation}
Thus, an uncorrelated pair of quarks has a $1/3$ chance to add to an overall
anti-triplet state (corresponding to coherent
superpositions like ``red + green = antiblue''\footnote{In the context of
  hadronisation models, 
  this coherent superposition of two quarks in an overall antitriplet
  state is sometimes called a
  \index{Diquarks}``diquark'' (at low $m_{qq}$)
  \index{String junctions}or a ``string junction'' (at high $m_{qq}$), see
  \secRef{sec:stringModel}; it corresponds to the antisymmatric ``red
  + green = antiblue'' combination needed to create a baryon
  wavefunction. }); otherwise it is in an overall 
sextet state. 

Note that the emphasis on
the quark-(anti)quark pair being \emph{uncorrelated} is important;
production processes that correlate the produced partons, like $Z\to q\bar{q}$ or $g\to q\bar{q}$, will
project out specific components (here the singlet and octet,
respectively). 
Note also that, if the quark
and (anti)quark are on opposite sides of the universe (i.e., living in
two different hadrons), the QCD \emph{dynamics} will not care what
overall colour state they 
are in, so for the formation of multi-partonic states in QCD, obviously the
spatial part of the wave functions (causality at the very least) 
will also play a role. Here, we are considering \emph{only} the colour part
of the wave functions. 
Some additional examples are 
\begin{eqnarray}
8\otimes 8 & = & 27 \oplus 10 \oplus \overline{10} \oplus 8 \oplus 8
\oplus 1 ~,\\ 
3 \otimes 8 & = & 15 \oplus 6 \oplus 3~,\\
3 \otimes 6 & = & 10 \oplus 8~,\\
3\otimes3\otimes3 & = & (6 \oplus \overline{3}) \otimes 3 = 10 \oplus 8
\oplus 8 \oplus 1 ~.
\end{eqnarray}
Physically, the 27 in the first line corresponds to a completely
incoherent addition of the colour charges of two gluons;
\index{Decuplet}the decuplets are slightly more coherent (with a lower
total colour charge), the octets
yet more, and the singlet corresponds to the combination of two gluons
that have precisely equal and opposite colour charges, so that their
total charge is zero. 
Further extensions and generalisations of these combination rules can
\index{Young tableaux}be obtained, e.g., using the method of Young
tableaux~\cite{young1901,youngSagan}.  

