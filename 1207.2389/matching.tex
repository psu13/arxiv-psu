\index{Matching}%
\index{LO}%
\section{Matching at LO and NLO\label{sec:matching}}
The essential problem that leads to matrix-element/parton-shower
matching can be illustrated in a very simple way.
\begin{figure}
\begin{center}%
\scalebox{0.70}{\begin{tabular}{l}
\large\bf F @ LO$\times$LL \\[2mm]
\begin{loopsnlegs}[c]{p{0.25cm}|ccccc}
 \small 2&~\ybox{\pqcd[2]{0}} & \ybox{\pqcd[2]{1}} & \ldots &
\\[2mm]
 \small 1&~\ybox{\pqcd[1]{0}} & \ybox{\pqcd[1]{1}}  
   & \ybox{\pqcd[1]{2}} & \ldots \\[2mm]
 \small 0&~\gbox{\pqcd[0]{0}} & \ybox{\pqcd[0]{1}} 
   & \ybox{\pqcd[0]{2}} &\ybox{\pqcd[0]{3}} & \ldots \\
\hline
& \small 0 & \small 1 & \small 2 & \small 3 & \ldots
 \end{loopsnlegs}
\end{tabular}\hspace*{-7mm}\raisebox{0.2cm}{\huge\bf +}\hspace*{-1.5mm}
\begin{tabular}{l}
\large\bf F+1 @ LO$\times$LL\\[2mm]
\begin{loopsnlegs}[c]{p{0.25cm}|ccccc}
 \small 2&~\wbox{\pqcd[2]{0}} & \ywbox{\pqcd[2]{1}} & \ldots & 
\\[2mm]
 \small 1&~\wbox{\pqcd[1]{0}} & \ywbox{\pqcd[1]{1}}  
   & \ywbox{\pqcd[1]{2}} & \ldots \\[2mm]
 \small 0&~\wbox{\pqcd[0]{0}} & \gwbox{\pqcd[0]{1}} 
   & \ywbox{\pqcd[0]{2}} &\ywbox{\pqcd[0]{3}} & \ldots \\
\hline
& \small 0 & \small 1 & \small 2 & \small 3 & \ldots
 \end{loopsnlegs}
\end{tabular}
\hspace*{-7mm}\raisebox{0.2cm}{\huge\bf =}\hspace*{-1.5mm}
\begin{tabular}{l}
\large\bf F \& F+1 @ LO$\times$LL\\[2mm]
\begin{loopsnlegs}[c]{p{0.25cm}|ccccc}
 \small 2&~\ybox{\pqcd[2]{0}} & \rybox{\pqcd[2]{1}} & \ldots & 
\\[2mm]
 \small 1&~\ybox{\pqcd[1]{0}} & \rybox{\pqcd[1]{1}}  
   & \rybox{\pqcd[1]{2}} & \ldots \\[2mm]
 \small 0&~\gbox{\pqcd[0]{0}} & \rybox{\pqcd[0]{1}} 
   & \rybox{\pqcd[0]{2}} &\rybox{\pqcd[0]{3}} & \ldots \\
\hline
& \small 0 & \small 1 & \small 2 & \small 3 & \ldots
 \end{loopsnlegs}
\end{tabular}}
\caption{The double-counting problem caused by naively
  adding cross sections involving matrix elements with different
  numbers of legs.\label{fig:doublecounting}}
\end{center}
\end{figure}
 Assume we have computed the LO cross section for some process, $F$,
 and that we have added an LL shower to it, as in the left-hand pane
 of \figRef{fig:doublecounting}. We know that this only gives us
 an LL description of $F+1$. We now wish to improve this from LL to LO  
 by adding the actual LO matrix element for $F+1$. Since we also want to
 be able to hadronize these events, etc, we again add an LL shower off
 them. However, since the matrix element for $F+1$ is divergent, we must
 restrict it to cover only the phase-space region with at least one
 hard resolved jet, illustrated by the half-shaded boxes in the middle
 pane of \figRef{fig:doublecounting}. 

\index{Matching}%
\index{Double counting}%
\index{Inclusive cross sections}%
\index{Exclusive cross sections}%
Adding these two samples,
 however, we end up counting the LL terms of the inclusive cross
 section for $F+1$ twice, since we are now getting them once from the shower
 off $F$ and once from the matrix element for $F+1$, illustrated by
 the dark shaded (red) areas of the right-hand pane of
 \figRef{fig:doublecounting}. This \emph{double-counting} problem
 would grow worse if we attempted to add more matrix elements, with
 more legs. The cause is very simple. Each such calculation
 corresponds to an \emph{inclusive} cross section, and hence naive
 addition would give
\begin{equation}
\sigma_{\mrm{tot}} =\sigma_{0;\mathrm{incl}} +
  \sigma_{1;\mathrm{incl}} =  \sigma_{0;\mathrm{excl}} +  2\,\sigma_{1;\mathrm{incl}}~.
\end{equation}
Recall the definition of inclusive and exclusive cross
 sections, \eqRef{eq:incexc}: $F$ \emph{inclusive} $=$ $F$ plus
  anything. $F$ \emph{exclusive} $=$ $F$ and only
  $F$. Thus, $\sigma_{F;\mathrm{incl}}=\sum_{\mrm{k}=
    0}^{\infty}\sigma_{F+k;\mrm{excl}}$.

 Instead, we must \emph{match} the coefficients calculated
 by the two parts of the full calculation --- showers and matrix
 elements --- more systematically, for each order in perturbation
 theory, so that the nesting of inclusive and exclusive cross sections
 is respected without overcounting.

Given a parton shower and a matrix-element generator, there are
fundamentally three different ways in which we can consider matching
the two \cite{Giele:2011cb}: slicing, subtraction, and unitarity. The
following subsections will briefly introduce each of these.

\index{Matching}%
\index{Matching!Slicing}%
\index{HERWIG}%
\subsection{Slicing}
 The most commonly encountered matching type is
  currently based on separating (slicing)
  phase space into two regions, one of which is supposed to be
  mainly described by hard matrix elements and the other of which is
  supposed to be described by the shower. This type of 
  approach was first 
  used in \Hw~\cite{Corcella:2000bw}, to
  include matrix-element corrections for one emission beyond the 
  basic hard process \cite{Seymour:1994we,Seymour:1994df}.
  This is illustrated in \figRef{fig:herwig}.
\index{Sudakov factor}
\begin{figure}
\begin{center}%
\scalebox{0.70}{\begin{tabular}{l}
{\large\bf F @ LO$\times$LL-Soft} (\Hw\ Shower)\\[2mm]
\begin{loopsnlegs}[c]{p{0.25cm}|ccccc}
 \small 2&~\ybox{\pqcd[2]{0}} & \wybox{\pqcd[2]{1}} & \ldots &
\\[2mm]
 \small 1&~\ybox{\pqcd[1]{0}} & \wybox{\pqcd[1]{1}}  
   & \wybox{\pqcd[1]{2}} & \ldots \\[2mm]
 \small 0&~\gbox{\pqcd[0]{0}} & \wybox{\pqcd[0]{1}} 
   & \wybox{\pqcd[0]{2}} &\wybox{\pqcd[0]{3}} & \ldots \\
\hline
& \small 0 & \small 1 & \small 2 & \small 3 & \ldots
 \end{loopsnlegs}
\end{tabular}\hspace*{-7mm}\raisebox{0.2cm}{\huge\bf +}\hspace*{-1.5mm}
\begin{tabular}{l}
{\large\bf F+1 @ LO$\times$LL} (\Hw\ Corrections)\\[2mm]
\begin{loopsnlegs}[c]{p{0.25cm}|ccccc}
 \small 2&~\wbox{\pqcd[2]{0}} & \ywbox{\pqcd[2]{1}} & \ldots & 
\\[2mm]
 \small 1&~\wbox{\pqcd[1]{0}} & \ywbox{\pqcd[1]{1}}  
   & \ywbox{\pqcd[1]{2}} & \ldots \\[2mm]
 \small 0&~\wbox{\pqcd[0]{0}} & \gwbox{\pqcd[0]{1}} 
   & \ywbox{\pqcd[0]{2}} &\ywbox{\pqcd[0]{3}} & \ldots \\
\hline
& \small 0 & \small 1 & \small 2 & \small 3 & \ldots
 \end{loopsnlegs}
\end{tabular}
\hspace*{-7mm}\raisebox{0.2cm}{\huge\bf =}\hspace*{-1.5mm}
\begin{tabular}{l}
{\large\bf F @ LO$_1\times$LL} (\Hw\ Matched)\\[2mm]
\begin{loopsnlegs}[c]{p{0.25cm}|ccccc}
 \small 2&~\ybox{\pqcd[2]{0}} & \ybox{\pqcd[2]{1}} & \ldots & 
\\[2mm]
 \small 1&~\ybox{\pqcd[1]{0}} & \ybox{\pqcd[1]{1}}  
   & \ybox{\pqcd[1]{2}} & \ldots \\[2mm]
 \small 0&~\gbox{\pqcd[0]{0}} & \gybox{\pqcd[0]{1}} 
   & \ybox{\pqcd[0]{2}} &\ybox{\pqcd[0]{3}} & \ldots \\
\hline
& \small 0 & \small 1 & \small 2 & \small 3 & \ldots
 \end{loopsnlegs}
\end{tabular}}
\index{HERWIG}%
\caption{\Hw's original matching
 scheme~\cite{Seymour:1994we,Seymour:1994df}, in which the dead zone
 of the \Hw\ shower was used as an 
  effective ``matching scale'' for one emission beyond a basic hard
 process. 
\label{fig:herwig}}
\end{center}
\end{figure}
\index{CKKW|see{Matching}}%
\index{MLM|see{Matching}}%
\index{Matching!CKKW}%
\index{Matching!L-CKKW}%
\index{Matching!MLM}%
  The method has since been generalized by several
  independent groups to include
  arbitrary numbers of additional legs, the most well-known of these
  being the CKKW~\cite{Catani:2001cc},
  CKKW-L~\cite{Lonnblad:2001iq,Lavesson:2005xu}, and
  MLM~\cite{Mangano:2006rw,Mrenna:2003if} approaches. 

Effectively,  the shower approximation is set to zero
  above some scale, either due to the presence of explicit dead zones
  in the shower, as in \Hw, or by vetoing any emissions above a certain
  \emph{matching scale}, as in the (L)-CKKW 
  and MLM approaches. The empty part of phase space can then be filled
  by separate events generated according to 
  higher-multiplicity tree-level matrix elements (MEs). In the
  (L)-CKKW and MLM schemes, this process can be iterated to include
  arbitrary numbers of additional hard legs (the practical limit being
  around 3 or 4, due to computational complexity). 

\index{Matching}%
In order to match smoothly with the shower calculation, the
  higher-multiplicity matrix elements 
  must be associated with Sudakov form factors (representing the
  virtual corrections that would have been generated if a shower had 
  produced the same phase-space configuration), and their $\alpha_s$
  factors must be chosen so that, at least at the matching scale, 
  they become identical to the choices made on the shower
  side~\cite{Cooper:2011gk}.  
  The CKKW and MLM approaches do this by 
  constructing ``fake parton-shower histories'' for the
  higher-multiplicity matrix elements. By applying a sequential jet
  clustering algorithm, a tree-like branching structure can be
  created that at least has the same dominant structure as
  that of a parton shower. Given the fake shower tree, $\alpha_s$
  factors can be chosen for each vertex with argument
  $\alpha_s(p_\perp)$ and Sudakov factors can be
  computed for each internal line in the tree. In the CKKW method,
  these Sudakov factors are estimated analytically, while the MLM and CKKW-L
  methods compute them numerically, from the actual shower
  evolution.
  
  Thus, the matched result is identical
  to the matrix element (ME) in the region above the matching scale, 
  modulo higher-order (Sudakov and $\alpha_s$) corrections. We may
  sketch this as  
\begin{equation}
\mbox{Matched (above matching scale)} =
\color{gray}\overbrace{\color{black}\mbox{Exact}}^{\small\mbox{ME}}~
\ {\color{black}\times} \ \overbrace{\color{black}(1 +
\mathcal{O}(\alpha_s))}^{\small\mbox{corrections}} \label{eq:scalebased1}~,
\end{equation}
where the ``shower-corrections'' include the approximate Sudakov factors
and $\alpha_s$ 
reweighting factors applied to the matrix elements in order to obtain
a smooth transition to the shower-dominated region.

Below the matching scale, the
small difference between the matrix elements and the shower
approximation can be dropped (since their leading singularities are
identical and this region by definition includes no hard jets), 
yielding the pure shower answer in that region,
\begin{eqnarray}
\mbox{Matched (below matching scale)} & = &
\color{gray}\overbrace{\color{black}\mbox{Approximate}}^{\mbox{shower}}
\ {\color{black}+} \ \overbrace{\color{black}(\mbox{Exact} -
\mbox{Approximate})}^{\mbox{correction}}\nonumber\\ 
& = & \mbox{Approximate} \ + \ \mbox{non-singular} \nonumber \\
& \to & \mbox{Approximate}~.
\end{eqnarray}
This type of strategy is illustrated in \figRef{fig:slicing}.
\begin{figure}
\begin{center}%
\scalebox{0.70}{\begin{tabular}{l}
{\large\bf F @ LO$\times$LL-Soft} (excl)\\[2mm]
\begin{loopsnlegs}[c]{p{0.25cm}|ccccc}
 \small 2&~\ybox{\pqcd[2]{0}} &  \ldots &
\\[2mm]
 \small 1&~\ybox{\pqcd[1]{0}} & \wybox{\pqcd[1]{1}}  
   & \ldots \\[2mm]
 \small 0&~\gbox{\pqcd[0]{0}} & \wybox{\pqcd[0]{1}} 
   & \wwybox{\pqcd[0]{2}} & \\
\hline
& \small 0 & \small 1 & \small 2 &
 \end{loopsnlegs}
\end{tabular}\hspace*{-6mm}\raisebox{0.2cm}{\huge\bf +}\hspace*{-2.5mm}
\begin{tabular}{l}
{\large\bf F+1 @ LO$\times$LL-Soft} (excl)\\[2mm]
\begin{loopsnlegs}[c]{p{0.25cm}|ccccc}
 \small 2&~\wbox{\pqcd[2]{0}} & \ldots & 
\\[2mm]
 \small 1&~\wbox{\pqcd[1]{0}} & \ywbox{\pqcd[1]{1}}  
   &  \ldots \\[2mm]
 \small 0&~\wbox{\pqcd[0]{0}} & \gwbox{\pqcd[0]{1}} 
   & \wywbox{\pqcd[0]{2}} &  \\
\hline
& \small 0 & \small 1 & \small 2 & 
 \end{loopsnlegs}
\end{tabular}\hspace*{-6mm}\raisebox{0.2cm}{\huge\bf +}\hspace*{-2.5mm}
\begin{tabular}{l}
{\large\bf F+2 @ LO$\times$LL} (incl)\\[2mm]
\begin{loopsnlegs}[c]{p{0.25cm}|ccccc}
 \small 2&~\wbox{\pqcd[2]{0}} & \ldots & 
\\[2mm]
 \small 1&~\wbox{\pqcd[1]{0}} & \wbox{\pqcd[1]{1}}  
   &\ldots \\[2mm]
 \small 0&~\wbox{\pqcd[0]{0}} & \wbox{\pqcd[0]{1}} 
   & \gwbox{\pqcd[0]{2}} &  \\
\hline
& \small 0 & \small 1 & \small 2 & 
 \end{loopsnlegs}
\end{tabular}
\hspace*{-6mm}\raisebox{0.2cm}{\huge\bf =}\hspace*{-2.5mm}
\begin{tabular}{l}
{\large\bf F @ LO$_2\times$LL} (MLM \& (L)-CKKW)\\[2mm]
\begin{loopsnlegs}[c]{p{0.25cm}|ccccc}
 \small 2&~\ybox{\pqcd[2]{0}} &  \ldots & 
\\[2mm]
 \small 1&~\ybox{\pqcd[1]{0}} & \ybox{\pqcd[1]{1}}  
   & \ldots \\[2mm]
 \small 0&~\gbox{\pqcd[0]{0}} & \gybox{\pqcd[0]{1}} 
   & \gybox{\pqcd[0]{2}} & \\
\hline
& \small 0 & \small 1 & \small 2 &
 \end{loopsnlegs}
\end{tabular}}
\caption{Slicing, with up
  to two additional emissions beyond the basic process. The showers off
  $F$ and $F+1$ are set to zero above a specific ``matching
  scale''. (The number of coefficients 
  shown was reduced a bit in these plots to make them fit in one row.)
\label{fig:slicing}}
\end{center}
\end{figure}

\index{Matching}%
As emphasized above, 
since this strategy is discontinuous across phase space, a main point
here is to ensure that the behavior across the matching scale be as
\index{CKKW}smooth as possible. CKKW showed \cite{Catani:2001cc} that it is
possible to remove any  dependence on the matching scale through  NLL
precision by careful choices of all ingredients in the matching;
technical details of the implementation 
(affecting the 
$\mathcal{O}(\alpha_s)$ terms in eq.~(\ref{eq:scalebased1}))
are important, and the dependence on the unphysical matching scale
may be larger than NLL unless the implementation
matches the theoretical algorithm
precisely~\cite{Lonnblad:2001iq,Lavesson:2005xu,Lavesson:2008ah}. 
Furthermore, since the Sudakov
\index{Matching!L-CKKW}\index{Matching!MLM}factors are generally computed using showers
(MLM, L-CKKW) or a  
\index{Matching!CKKW}shower-like formalism (CKKW), while the real corrections
are computed 
using matrix elements, care must be taken not to (re-)introduce
differences that could break the detailed real-virtual balance that
ensures unitarity among the singular parts, see
e.g.,~\cite{Cooper:2011gk}. 

\index{Matching}%
\index{QCD!Scale invariance}%
\index{Matching scale|see{Matching}}%
\index{Matching!Matching scale}%
It is advisable not to choose the matching scale too low. This is
again  essentially due to the approximate scale invariance of QCD
imploring us to 
write the matching scale as a ratio, rather than as an absolute
number. 
If one uses a
very low matching 
scale, the higher-multiplicity matrix elements will already be quite
singular, leading to very large LO cross sections before matching. 
After matching, these large cross sections are tamed by
the Sudakov factors produced by the matching scheme, and hence the
final cross sections may still look reasonable. But 
the higher-multiplicity matrix elements in
general contain subleading singularity structures, beyond those
accounted for by the shower, and hence the delicate
line of detailed balance that ensures unitarity has most
assuredly been overstepped. We therefore recommend not to take the 
matching scale lower than about an order of magnitude below the
 characteristic scale of the hard process. 

One should also be aware that all strategies of this type are 
quite computing intensive. This is basically due to the fact that 
a separate phase-space generator is required for each of the
$n$-parton correction terms, with each such sample a priori consisting
of weighted events such that a separate unweighting step (often with
quite low efficiency) is needed before an
unweighted sample can be produced. 

\index{Matching}%
\index{Matching!Subtraction}%
\index{Subtraction!In the context of matching}%
\subsection{Subtraction}
Another way of matching two calculations is by subtracting one
  from the other and correcting by the difference, schematically
\begin{equation}
%\mbox{\sl Additive : }~~~
\mbox{Matched} =
\color{gray}\overbrace{\color{black}\mbox{Approximate}}^{\mbox{shower}}
\color{black} \ + \ \color{gray}\overbrace{\color{black}
(\mbox{Exact}-\mbox{Approximate})}^{\mbox{correction}}~. \label{eq:additive}
\end{equation}
\index{NLO}%
\index{Matching!MCatNLO}
This looks very much like the structure of a subtraction-based NLO
fixed-order calculation, \secRef{sec:subtraction}, in which the shower
approximation here plays the role of subtraction terms, and indeed
this is what is used  in 
strategies like \Fw{}
\cite{Frixione:2002ik,Frixione:2003ei,Frixione:2008ym}, illustrated in
\figRef{fig:fw}.
\begin{figure}
\begin{center}%
\scalebox{0.70}{\begin{tabular}{l}
{\large\bf F @ LO$\times$LL}\\[2mm]
\begin{loopsnlegs}[c]{p{0.25cm}|ccccc}
 \small 2&~\ybox{\pqcd[2]{0}} & \ybox{\pqcd[2]{1}} & \ldots &
\\[2mm]
 \small 1&~\ybox{\pqcd[1]{0}} & \ybox{\pqcd[1]{1}}  
   & \ybox{\pqcd[1]{2}} & \ldots \\[2mm]
 \small 0&~\gbox{\pqcd[0]{0}} & \ybox{\pqcd[0]{1}} 
   & \ybox{\pqcd[0]{2}} &\ybox{\pqcd[0]{3}} & \ldots \\
\hline
& \small 0 & \small 1 & \small 2 & \small 3 & \ldots
 \end{loopsnlegs}
\end{tabular}
\hspace*{-7mm}\raisebox{0.2cm}{\huge\bf +}\hspace*{-1.5mm}
\begin{tabular}{l}
{\large\bf (F @ NLO$\times$LL) - (F @ LO$\times$LL)}\\[2mm]
\begin{loopsnlegs}[c]{p{0.25cm}|ccccc}
 \small 2&~\eggbox{\pqcd[2]{0}} & \eggbox{\pqcd[2]{1}} & \ldots &
\\[2mm]
 \small 1&~\cyanbox{\pqcd[1]{0}} & \eggbox{\pqcd[1]{1}}  
   & \eggbox{\pqcd[1]{2}} & \ldots \\[2mm]
 \small 0&~\wbox{\pqcd[0]{0}} & \cyanbox{\pqcd[0]{1}} 
   & \eggbox{\pqcd[0]{2}} &\eggbox{\pqcd[0]{3}} & \ldots \\
\hline
& \small 0 & \small 1 & \small 2 & \small 3 & \ldots
 \end{loopsnlegs}
\end{tabular}
\hspace*{-7mm}\raisebox{0.2cm}{\huge\bf =}\hspace*{-1.5mm}
\begin{tabular}{l}
{\large\bf F @ NLO$\times$LL} (\Fw)\\[2mm]
\begin{loopsnlegs}[c]{p{0.25cm}|ccccc}
 \small 2&~\ybox{\pqcd[2]{0}} & \ybox{\pqcd[2]{1}} & \ldots &
\\[2mm]
 \small 1&~\gbox{\pqcd[1]{0}} & \ybox{\pqcd[1]{1}}  
   & \ybox{\pqcd[1]{2}} & \ldots \\[2mm]
 \small 0&~\gbox{\pqcd[0]{0}} & \gbox{\pqcd[0]{1}} 
   & \ybox{\pqcd[0]{2}} &\ybox{\pqcd[0]{3}} & \ldots \\
\hline
& \small 0 & \small 1 & \small 2 & \small 3 & \ldots
 \end{loopsnlegs}
\end{tabular}}
\caption{\Fw. In the middle
  pane, cyan boxes
  denote non-singular correction terms, while the egg-colored ones 
  denote showers off such corrections, which cannot lead to
  double-counting at the LL level.
\label{fig:fw}}
\end{center}
\end{figure}
In this type of approach, however, negative-weight events will
generally occur, for instance in  
phase-space points where the approximation is larger than the exact
answer. 

Negative weights are not in principle an insurmountable
problem. Histograms can be filled with each event counted 
according to its weight, as usual. However, negative weights do affect
efficiency. Imagine a worst-case scenario in which 1000
positive-weight events have been generated, along with 999 negative-weight
ones (assuming each event weight has the same absolute value, the
closest one can get to an unweighted sample in the presence of
negative weights). 
The statistical precision of the MC answer would be equivalent
to one event, for 2000 generated, i.e., a big loss in convergence
rate. 

\index{Matching!Subtraction}%
\index{NLO}%
\index{MCatNLO|see{Matching}}%
\index{Matching!MCatNLO}%
In practice, generators like MC@NLO ``only'' produce around 10\% or
less events with negative weights, so the convergence rate should not
be severely affected for ordinary applications. Nevertheless,  
the problem of negative weights 
motivated the development of the so-called \Pw\ approach 
\index{POWHEG|see{Matching}}\index{Matching!POWHEG}\cite{Frixione:2007vw}, illustrated in \figRef{fig:powheg}, 
\begin{figure}
\begin{center}%
\scalebox{0.685}{\begin{tabular}{l}
{\large\bf F @ LO$_1\times$LL}\\[2mm]
\begin{loopsnlegs}[c]{p{0.25cm}|ccccc}
 \small 2&~\ybox{\pqcd[2]{0}} & \ybox{\pqcd[2]{1}} & \ldots &
\\[2mm]
 \small 1&~\ybox{\pqcd[1]{0}} & \ybox{\pqcd[1]{1}}  
   & \ybox{\pqcd[1]{2}} & \ldots \\[2mm]
 \small 0&~\gbox{\pqcd[0]{0}} & \gbox{\pqcd[0]{1}} 
   & \ybox{\pqcd[0]{2}} &\ybox{\pqcd[0]{3}} & \ldots \\
\hline
& \small 0 & \small 1 & \small 2 & \small 3 & \ldots
 \end{loopsnlegs}
\end{tabular}
\hspace*{-7mm}\raisebox{0.2cm}{\huge\bf +}\hspace*{-1.5mm}
\begin{tabular}{l}
{\large\bf (F @ NLO$\times$LL) - (F @ LO$_1\times$LL)}\\[2mm]
\begin{loopsnlegs}[c]{p{0.25cm}|ccccc}
 \small 2&~\eggbox{\pqcd[2]{0}} & \eggbox{\pqcd[2]{1}} & \ldots &
\\[2mm]
 \small 1&~\cyanbox{\pqcd[1]{0}} & \eggbox{\pqcd[1]{1}}  
   & \eggbox{\pqcd[1]{2}} & \ldots \\[2mm]
 \small 0&~\wbox{\pqcd[0]{0}} & \wbox{\pqcd[0]{1}} 
   & \wbox{\pqcd[0]{2}} &\wbox{\pqcd[0]{3}} & \ldots \\
\hline
& \small 0 & \small 1 & \small 2 & \small 3 & \ldots
 \end{loopsnlegs}
\end{tabular}
\hspace*{-7mm}\raisebox{0.2cm}{\huge\bf =}\hspace*{-1.5mm}
\begin{tabular}{l}
{\large\bf F @ NLO$\times$LL} (\Pw)\\[2mm]
\begin{loopsnlegs}[c]{p{0.25cm}|ccccc}
 \small 2&~\ybox{\pqcd[2]{0}} & \ybox{\pqcd[2]{1}} & \ldots &
\\[2mm]
 \small 1&~\gbox{\pqcd[1]{0}} & \ybox{\pqcd[1]{1}}  
   & \ybox{\pqcd[1]{2}} & \ldots \\[2mm]
 \small 0&~\gbox{\pqcd[0]{0}} & \gbox{\pqcd[0]{1}} 
   & \ybox{\pqcd[0]{2}} &\ybox{\pqcd[0]{3}} & \ldots \\
\hline
& \small 0 & \small 1 & \small 2 & \small 3 & \ldots
 \end{loopsnlegs}
\end{tabular}}
\caption{\Pw. In the middle
  pane, cyan boxes
  denote non-singular correction terms, while the egg-colored ones 
  denote showers off such corrections, which cannot lead to
  double-counting at the LL level.
\label{fig:powheg}}
\end{center}
\end{figure}
which is
constructed specifically to prevent negative-weight events from
occurring and simultaneously to be more independent of which
parton-shower algorithm it is used with. In the \Pw\ method, 
one effectively modifies the real-emission probability for the first
emission
to agree with
the $F+1$ matrix element (this is covered under unitarity, below). One
is then left with a purely virtual correction, which will typically be
positive, at least for processes for which the NLO cross section is
larger than the LO one. 

\index{NLO}%
\index{Matching}%
The advantage of these methods
is obviously that NLO corrections to the Born level can be
systematically incorporated. However, a systematic way of
extending this strategy beyond the first additional emission is not
available, save for combining them with a slicing-based strategy
\index{MENLOPS|see{Matching}}\index{Matching!MENLOPS}for the additional legs, as in \textsc{Menlops}
\cite{Hamilton:2010wh}, illustrated in \figRef{fig:menlops}. 
\begin{figure}
\begin{center}%
\scalebox{0.685}{\begin{tabular}{l}
{\large\bf F @ NLO$\times$LL-Soft} ($\sim$ \Pw)\\[2mm]
\begin{loopsnlegs}[c]{p{0.25cm}|ccccc}
 \small 2&~\ybox{\pqcd[2]{0}} & \ybox{\pqcd[2]{1}} & \ldots &
\\[2mm]
 \small 1&~\gbox{\pqcd[1]{0}} & \ybox{\pqcd[1]{1}}  
   & \wybox{\pqcd[1]{2}} & \ldots \\[2mm]
 \small 0&~\gbox{\pqcd[0]{0}} & \gbox{\pqcd[0]{1}} 
   & \wybox{\pqcd[0]{2}} &\wwybox{\pqcd[0]{3}} & \ldots \\
\hline
& \small 0 & \small 1 & \small 2 & \small 3 & \ldots
 \end{loopsnlegs}
\end{tabular}
\hspace*{-7mm}\raisebox{0.2cm}{\huge\bf +}\hspace*{-1.5mm}
\begin{tabular}{l}
{\large\bf F +2 @ LO$_n\times$LL} ($\sim$ CKKW for F+2)\\[2mm]
\begin{loopsnlegs}[c]{p{0.25cm}|ccccc}
 \small 2&~\wbox{\pqcd[2]{0}} & \wbox{\pqcd[2]{1}} & \ldots &
\\[2mm]
 \small 1&~\wbox{\pqcd[1]{0}} & \wbox{\pqcd[1]{1}}  
   & \ywbox{\pqcd[1]{2}} & \ldots \\[2mm]
 \small 0&~\wbox{\pqcd[0]{0}} & \wbox{\pqcd[0]{1}} 
   & \gwbox{\pqcd[0]{2}} &\gywbox{\pqcd[0]{3}} & \ldots \\
\hline
& \small 0 & \small 1 & \small 2 & \small 3 & \ldots
 \end{loopsnlegs}
\end{tabular}
\hspace*{-7mm}\raisebox{0.2cm}{\huge\bf =}\hspace*{-1.5mm}
\begin{tabular}{l}
{\large\bf F @ NLO$\times$LL} (\textsc{Menlops})\\[2mm]
\begin{loopsnlegs}[c]{p{0.25cm}|ccccc}
 \small 2&~\ybox{\pqcd[2]{0}} & \ybox{\pqcd[2]{1}} & \ldots &
\\[2mm]
 \small 1&~\gbox{\pqcd[1]{0}} & \ybox{\pqcd[1]{1}}  
   & \ybox{\pqcd[1]{2}} & \ldots \\[2mm]
 \small 0&~\gbox{\pqcd[0]{0}} & \gbox{\pqcd[0]{1}} 
   & \gybox{\pqcd[0]{2}} &\gybox{\pqcd[0]{3}} & \ldots \\
\hline
& \small 0 & \small 1 & \small 2 & \small 3 & \ldots
 \end{loopsnlegs}
\end{tabular}}
\caption{\textsc{Menlops}. Note that each of the \Pw\ and CKKW samples
 are composed 
  of separate sub-samples, as illustrated in \figsRef{fig:slicing} and
  \ref{fig:powheg}. 
\label{fig:menlops}}
\end{center}
\end{figure}
These issues are, however, no more severe than
in ordinary fixed-order NLO 
approaches, and hence they are not viewed as disadvantages if the
point of reference is an NLO computation. 

\index{Matching!Unitarity}%
\index{Matching}%
\subsection{Unitarity}
The oldest, and in my view most attractive, 
approach \cite{Bengtsson:1986et,Bengtsson:1986hr}
  consists of working out the 
 shower approximation to a given fixed order, and correcting
 the shower splitting functions at that order by a multiplicative
 factor given by the ratio of the matrix element
 to the shower approximation, phase-space point by phase-space
 point. We may sketch this as 
\begin{equation}
%\mbox{\sl Multiplicative : }~~~
\mbox{Matched} =
\color{gray}\overbrace{\color{black}\mbox{Approximate}}^{\mbox{shower}}
\ {\color{black}\times} \
\overbrace{\color{black}\frac{\mbox{Exact}}{\mbox{Approximate}}}^{\mbox{correction}}~. \label{eq:multiplicative}
\end{equation}
When these correction factors are inserted back into the
shower evolution, they guarantee that the shower evolution off $n-1$
partons correctly reproduces the $n$-parton matrix elements, 
without the need to generate a separate $n$-parton sample. 
That is, the shower approximation is essentially used as a 
pre-weighted (stratified) all-orders phase-space generator, on which a
more exact answer can subsequently be
imprinted order by order in perturbation theory. Since the shower is 
already optimized for exactly the kind of singular structures that
occur in QCD, very fast computational speeds can therefore be obtained
with this method~\cite{LopezVillarejo:2011ap}.


\index{PYTHIA}%
\index{Matching}%
In the original
approach \cite{Bengtsson:1986et,Bengtsson:1986hr}, used by \Py
\cite{Sjostrand:2006za,Sjostrand:2014zea}, this was only 
worked out for one additional emission beyond the basic hard
process. 
\index{POWHEG}%
In \Pw~\cite{Frixione:2007vw,Alioli:2010xd}, 
it was extended to include also virtual corrections to the Born-level
matrix element. 
\index{VINCIA}%
\index{VINCIA!Matching}%
\index{Matching!VINCIA}%
Finally, in \Vc, it has been 
extended to include arbitrary numbers of emissions at tree
level~\cite{Giele:2011cb,LopezVillarejo:2011ap} and one emission at
loop level~\cite{Hartgring:2013jma}, though that method has so far
only been applied to final-state showers.

\begin{figure}[t]
\begin{center}%
\scalebox{0.70}{
\begin{tabular}{l}
{\large\bf F @ LO$_{1}$+LL} (\Py)\\[2mm]
\begin{loopsnlegs}[c]{p{0.25cm}|ccccc}
 \small 2&~\ybox{\pqcd[2]{0}} & \ybox{\pqcd[2]{1}} & \ldots &
\\[2mm]
 \small 1&~\ybox{\pqcd[1]{0}} & \ybox{\pqcd[1]{1}}  
   & \ybox{\pqcd[1]{2}} & \ldots \\[2mm]
 \small 0&~\gbox{\pqcd[0]{0}} & \gbox{\pqcd[0]{1}} 
   & \ybox{\pqcd[0]{2}} &\ybox{\pqcd[0]{3}} & \ldots \\
\hline
& \small 0 & \small 1 & \small 2 & \small 3 & \ldots
 \end{loopsnlegs}
\end{tabular}\hspace*{1cm}
\begin{tabular}{l}
{\large\bf F @ NLO+LO$_{n}$+LL} (\Vc)\\[2mm]
\begin{loopsnlegs}[c]{p{0.25cm}|ccccc}
 \small 2&~\ybox{\pqcd[2]{0}} & \ybox{\pqcd[2]{1}} & \ldots & 
\\[2mm]
 \small 1&~\gbox{\pqcd[1]{0}} & \ybox{\pqcd[1]{1}}  
   & \ybox{\pqcd[1]{2}} & \ldots \\[2mm]
 \small 0&~\gbox{\pqcd[0]{0}} & \gbox{\pqcd[0]{1}} 
   & \gbox{\pqcd[0]{2}} &\gbox{\pqcd[0]{3}} & \ldots \\
\hline
& \small 0 & \small 1 & \small 2 & \small 3 & \ldots
 \end{loopsnlegs}
\end{tabular}}
\caption{\Py~(left) and \Vc~(right). Unitarity-based. Only one event
 sample is produced by 
  each of these methods, and hence no sub-components are shown. 
\label{fig:match-unitary}}
\end{center}
\end{figure}
An illustration of the perturbative coefficients that can be 
included in each of  these approaches is given in
\figRef{fig:match-unitary}, as usual with green (darker shaded) boxes
representing exact coefficients and yellow (light shaded) boxes
representing logarithmic approximations. 

\index{Matching}%
\index{Sudakov factor}%
Finally, two more
properties unique to this method deserve mention. Firstly, 
since the corrections modify the actual shower evolution kernels, the
corrections are automatically \emph{resummed} in the Sudakov
exponential, which should improve the logarithmic precision once
$k\ge2$ is included, 
and secondly, since the shower is \emph{unitary},  an initially unweighted sample
of $(n-1)$-parton configurations remains unweighted, with 
no need for a separate event-unweighting or event-rejection
step. 


