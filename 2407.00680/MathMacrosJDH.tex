% MathMacrosJDH.tex
%
% This file contains the macros that Joel David Hamkins uses in his
% LaTeX mathematical articles. It is subject to revision.
%
% The following packages are used by some of the macros, and so you might want to
% include them in your main document.
%\usepackage{latexsym,amsfonts,amsmath,amssymb}
%
% The following sets up the main theorem types.
% Theorem numbering increments for all types together.
%
%\newtheorem{theorem}{Theorem}[section]
\newtheorem{theorem}{Theorem}
\newtheorem*{theorem*}{Theorem}
\newtheorem{maintheorem}[theorem]{Main Theorem}
\newtheorem{maintheorems}[theorem]{Main Theorems}
\newtheorem*{maintheorem*}{Main Theorem}
\newtheorem*{maintheorems*}{Main Theorems}
\newtheorem{corollary}[theorem]{Corollary}
\newtheorem*{corollary*}{Corollary}
\newtheorem*{corollaries*}{Corollaries}
\newtheorem{sublemma}{Lemma}[theorem]
\newtheorem{lemma}[theorem]{Lemma}
\newtheorem{keylemma}[theorem]{Key Lemma}
\newtheorem{mainlemma}[theorem]{Main Lemma}
\newtheorem{observation}[theorem]{Observation}
\newtheorem{proposition}[theorem]{Proposition}
\newtheorem{claim}[theorem]{Claim}
\newtheorem{subclaim}[sublemma]{Claim}
\newtheorem{conjecture}[theorem]{Conjecture}
\newtheorem{fact}[theorem]{Fact}
\newtheorem{exercise}{Exercise}[section]
%
\theoremstyle{definition}
\newtheorem{definition}[theorem]{Definition}
\newtheorem{notation}[theorem]{Notation}
\newtheorem*{definition*}{Definition}
\newtheorem{maindefinition}[theorem]{Main Definition}
\newtheorem{subdefinition}[sublemma]{Definition}
\newtheorem{question}[theorem]{Question}
\newtheorem*{question*}{Question}
\newtheorem{questions}[theorem]{Questions}
\newtheorem*{questions*}{Questions}
\newtheorem{mainquestion}[theorem]{Main Question} % with numbering
\newtheorem*{mainquestion*}{Main Question} % without numbering
\newtheorem{openquestion}[theorem]{Open Question} % with numbering
\newtheorem*{openquestion*}{Open Question} % without numbering
%
\theoremstyle{remark}
\newtheorem{example}[theorem]{Example}
\newtheorem{remark}[theorem]{Remark}
%
\newcommand{\QED}{\end{proof}}
\newenvironment{proclamation}{\smallskip\noindent\proclaim}{\par\smallskip}
\newenvironment{points}{\removelastskip\begin{enumerate}\setlength{\parskip}{0pt}}{\end{enumerate}}
\def\proclaim[#1]{{\bf #1}}
\def\BF#1.{{\bf #1.}}
\newenvironment{conversation}{\begin{quote}
   \begin{description}[itemsep=1ex,leftmargin=1cm]}{\end{description}\end{quote}}
\def\says#1:#2\par{\item[#1] #2\par}
%\newcommand{\cal}{\mathcal}
\newcommand{\margin}[1]{\marginpar{\tiny #1}}
%
% macros for certain accented foreign names
%
\newcommand{\Bukovsky}{Bukovsk\`y}
\newcommand{\Dzamonja}{D\v{z}amonja}
\newcommand\Ersov{Er\v sov}
\newcommand\Hrbacek{Hrb\'a\v cek}
\newcommand{\Jonsson}{J\'{o}nsson}
\newcommand{\Lob}{L\"ob}
\newcommand\Vopenka{Vop\v{e}nka}
\newcommand{\Los}{\L o\'s}
\newcommand\Lukasiewicz{\L ukasiewicz}
\newcommand{\Fraisse}{Fra\"\i ss\'e}
\newcommand{\Francois}{Fran\c{c}ois}
\newcommand{\Godel}{G\"odel}
\newcommand{\Goedel}{G\"odel}
\newcommand{\Habic}{Habi\v c}
\newcommand{\Jerabek}{Je\v r\'abek}
\newcommand\Konig{K\"onig}
\newcommand\LHopital{L'Hôpital}
\newcommand{\Lowe}{L\"owe}
\newcommand{\Loewe}{\Lowe}
\newcommand{\Erdos}{Erd\H{o}s}
\newcommand{\Levy}{L\'{e}vy}
\newcommand{\Lowenheim}{L\"owenheim}
\newcommand{\Oystein}{{\O}ystein}
\newcommand{\Sierpinski}{Sierpi\'{n}ski}
\newcommand\Smorynski{Smory\'nski}
\newcommand\Todorcevic{Todor\v{c}evi\'{c}}
\newcommand\Vaananen{Väänänen}
%\newcommand{\Vaananen}{V\"a\"an\"anen}
\newcommand{\Velickovic}{Veli\v ckovi\'c}
%
% macros to name mathematical objects:
%
\newcommand{\A}{{\mathbb A}}
\newcommand{\B}{{\mathbb B}}
\newcommand{\C}{{\mathbb C}}
\newcommand{\D}{{\mathbb C}}
\newcommand{\E}{{\mathbb E}}
\newcommand{\F}{{\mathbb F}}
\newcommand{\G}{{\mathbb G}}
%\newcommand{\H}{{\mathbb B}} already defined
%\newcommand{\I}{{\mathbb B}} already defined
\newcommand{\J}{{\mathbb J}}
\newcommand{\X}{{\mathbb X}}
\newcommand{\N}{{\mathbb N}}
\renewcommand{\P}{{\mathbb P}}
\newcommand{\Q}{{\mathbb Q}}
\newcommand{\U}{{\mathbb U}}
\newcommand{\Z}{{\mathbb Z}}
\newcommand{\R}{{\mathbb R}}
\newcommand{\T}{{\mathbb T}}
\newcommand{\bbS}{{\mathbb S}}% \S already means section symbol
\newcommand{\calM}{{\mathcal M}}
\newcommand{\continuum}{\mathfrak{c}}
\newcommand{\term}{{\!\scriptscriptstyle\rm term}}
\newcommand{\Dterm}{{D_{\!\scriptscriptstyle\rm term}}}
\newcommand{\Ftail}{{\F_{\!\scriptscriptstyle\rm tail}}}
\newcommand{\Fotail}{{\F^0_{\!\scriptscriptstyle\rm tail}}}
\newcommand{\ftail}{{f_{\!\scriptscriptstyle\rm tail}}}
\newcommand{\fotail}{{f^0_{\!\scriptscriptstyle\rm tail}}}
\newcommand{\Gtail}{{G_{\!\scriptscriptstyle\rm tail}}}
\newcommand{\Gotail}{{G^0_{\!\scriptscriptstyle\rm tail}}}
\newcommand{\Goterm}{{G^0_{\!\scriptscriptstyle\rm term}}}
\newcommand{\Htail}{{H_{\!\scriptscriptstyle\rm tail}}}
\newcommand{\Hterm}{{H_{\!\scriptscriptstyle\rm term}}}
\newcommand{\Ptail}{{\P_{\!\scriptscriptstyle\rm tail}}}
\newcommand{\Potail}{{\P^0_{\!\scriptscriptstyle\rm tail}}}
\newcommand{\Pterm}{{\P_{\!\scriptscriptstyle\rm term}}}
\newcommand{\Qterm}{{\Q_{\scriptscriptstyle\rm term}}}
\newcommand{\Gterm}{{G_{\scriptscriptstyle\rm term}}}
\newcommand{\Rtail}{{\R_{\!\scriptscriptstyle\rm tail}}}
\newcommand{\Rterm}{{\R_{\scriptscriptstyle\rm term}}}
\newcommand{\hterm}{{h_{\scriptscriptstyle\rm term}}}
\newcommand{\overbar}[1]{\mkern 3.5mu\overline{\mkern-3.5mu#1\mkern-.5mu}\mkern.5mu}% correct overline for slanted capital letters in math mode
\newcommand{\barin}{\mathrel{\mkern3mu\overline{\mkern-3mu\in\mkern-1.5mu}\mkern1.5mu}}
\newcommand{\Abar}{{\overbar{A}}}
\newcommand{\Cbar}{{\overbar{C}}}
\newcommand{\Dbar}{{\overbar{D}}}
\newcommand{\Fbar}{{\overbar{F}}}
\newcommand{\Gbar}{{\overbar{G}}}
\newcommand{\Mbar}{{\overbar{M}}}
\newcommand{\Nbar}{{\overbar{N}}}
\newcommand{\Vbar}{{\overbar{V}}}
\newcommand{\Wbar}{{\overbar{W}}}
\newcommand{\Vhat}{{\hat{V}}}
\newcommand{\Xbar}{{\overbar{X}}}
\newcommand{\jbar}{{\bar j}}
\newcommand{\Ptilde}{{\tilde\P}}
\newcommand{\Gtilde}{{\tilde G}}
\newcommand{\Qtilde}{{\tilde\Q}}
\newcommand{\Adot}{{\dot A}}
\newcommand{\Bdot}{{\dot B}}
\newcommand{\Cdot}{{\dot C}}
%\newcommand{\Ddot}{{\dot D}}  % this is already defined somehow, but not with my meaning
\newcommand{\Gdot}{{\dot G}}
\newcommand{\Mdot}{{\dot M}}
\newcommand{\Ndot}{{\dot N}}
\newcommand{\Pdot}{{\dot\P}}
\newcommand{\Qdot}{{\dot\Q}}
\newcommand{\qdot}{{\dot q}}
\newcommand{\pdot}{{\dot p}}
\newcommand{\Rdot}{{\dot\R}}
\newcommand{\Tdot}{{\dot T}}
\newcommand{\Xdot}{{\dot X}}
\newcommand{\Ydot}{{\dot Y}}
\newcommand{\mudot}{{\dot\mu}}
\newcommand{\hdot}{{\dot h}}
\newcommand{\rdot}{{\dot r}}
\newcommand{\sdot}{{\dot s}}
\newcommand{\xdot}{{\dot x}}
\newcommand{\I}[1]{\mathop{\hbox{\rm I}_#1}}
\newcommand{\id}{\mathop{\hbox{\small id}}}
\newcommand{\one}{\mathbbm{1}} % requires \usepackage{bbm}
% \newcommand{\zero}{\mathbbm{0}} % doesn't work
%
% Cardinal characteristic numbers:
%
\newcommand\almostdisjointness{\mathfrak{a}}
\newcommand\bounding{\mathfrak{b}}
\newcommand\dominating{\mathfrak{d}}
\newcommand\splitting{\mathfrak{s}}
\newcommand\reaping{\mathfrak{r}}
\newcommand\ultrafilter{\mathfrak{u}}
\newcommand\rr{\mathfrak{rr}}
\newcommand\rrsum{\rr_{\scriptscriptstyle\Sigma}}
\newcommand\non{\mathop{\bf non}}
\newcommand\p{\frak{p}}
\newcommand\jumbling{\frak{j}}
\newcommand\rrcon{\rr_{\mathrm{c}}}
\newcommand\unblocking{\frak{ub}}
%
% Macros for infinite chess:
%
\newcommand{\omegaoneCh}{\omega_1^\Ch}
\newcommand{\omegaoneChi}{\omega_1^{\baselineskip=0pt\vtop to 7pt{\hbox{$\scriptstyle\Ch$}\vskip-1pt\hbox{\,$\scriptscriptstyle\sim$}}}}
\newcommand{\omegaoneChc}{\omega_1^{\Ch,c}}
\newcommand{\omegaoneChthree}{\omega_1^{\Ch_3}}
\newcommand{\omegaoneChthreei}{\omega_1^{\baselineskip=0pt\vtop to 7pt{\hbox{$\scriptstyle\Ch_3$}\vskip-1.5pt\hbox{\,$\scriptscriptstyle\sim$}}}}
\newcommand{\omegaoneChthreec}{\omega_1^{\Ch_3,c}}
%
% macros for mathematical symbols:
%
% dotminus
\makeatletter
\newcommand{\dotminus}{\mathbin{\text{\@dotminus}}}
\newcommand{\@dotminus}{%
  \ooalign{\hidewidth\raise1ex\hbox{.}\hidewidth\cr$\m@th-$\cr}%
}
\makeatother
%
\newcommand{\unaryminus}{\scalebox{0.67}[1.0]{\( - \)}}
\newcommand{\from}{\mathbin{\vbox{\baselineskip=2pt\lineskiplimit=0pt
                         \hbox{.}\hbox{.}\hbox{.}}}}
\newcommand{\surj}{\twoheadrightarrow}
\newcommand{\of}{\subseteq}
\newcommand{\ofnoteq}{\subsetneq}
\newcommand{\ofneq}{\subsetneq}
\newcommand{\fo}{\supseteq}
\newcommand{\sqof}{\sqsubseteq}
\newcommand{\ofsim}{\mathrel{\raisebox{-3pt}{\vbox{\baselineskip=3pt\hbox{$\subset$}\hbox{\tiny$\,\sim$}}}}}
\newcommand{\toward}{\rightharpoonup}
\renewcommand{\Set}[1]{\left\{\,{#1}\,\right\}}
\renewcommand{\set}[1]{\{\,{#1}\,\}}
\newcommand{\singleton}[1]{\left\{{#1}\right\}}
\newcommand{\compose}{\circ}
\newcommand{\curlyelesub}{\Undertilde\prec}
\newcommand{\elesub}{\prec}
\newcommand{\eleequiv}{\equiv}
\newcommand\equinumerous{\simeq}
\newcommand{\muchgt}{\gg}
\newcommand{\muchlt}{\ll}
\newcommand{\inverse}{{-1}}
\newcommand{\jump}{{\!\triangledown}}
\newcommand{\Jump}{{\!\blacktriangledown}}
\newcommand{\ilt}{<_{\infty}}
\newcommand{\ileq}{\leq_{\infty}}
\newcommand{\iequiv}{\equiv_{\infty}}
\newcommand{\leqRK}{\leq_{\hbox{\scriptsize\sc rk}}}
\newcommand{\ltRK}{<_{\hbox{\scriptsize\sc rk}}}
\newcommand{\isoRK}{\iso_{\hbox{\scriptsize\sc rk}}}
\newcommand{\Tequiv}{\equiv_T}
\newcommand{\dom}{\mathop{\rm dom}}
\newcommand{\Ht}{\mathop{\rm ht}}
\newcommand{\tp}{\mathop{\rm tp}}
%\newcommand{\ht}{\mathop{\rm ht}}
\newcommand{\dirlim}{\mathop{\rm dir\,lim}}
\newcommand{\SSy}{\mathop{\rm SSy}}
\newcommand{\ran}{\mathop{\rm ran}}
\newcommand{\rank}{\mathop{\rm rank}}
\newcommand{\supp}{\mathop{\rm supp}}
\newcommand{\add}{\mathop{\rm add}}
\newcommand{\coll}{\mathop{\rm coll}}
\newcommand{\cof}{\mathop{\rm cof}}
\newcommand{\Cof}{\mathop{\rm Cof}}
\newcommand{\Fin}{\mathop{\rm Fin}}
\newcommand{\Add}{\mathop{\rm Add}}
\newcommand{\Aut}{\mathop{\rm Aut}}
\newcommand{\Inn}{\mathop{\rm Inn}}
\newcommand{\Coll}{\mathop{\rm Coll}}
\newcommand{\Ult}{\mathop{\rm Ult}}
\newcommand{\Th}{\mathop{\rm Th}}
\newcommand{\Con}{\mathop{{\rm Con}}}
\newcommand{\Mod}{\mathop{{\rm Mod}}}
\newcommand{\image}{\mathbin{\hbox{\tt\char'42}}}
\newcommand{\plus}{{+}}
\newcommand{\plusplus}{{{+}{+}}}
\newcommand{\plusplusplus}{{{+}{+}{+}}}
\DeclareMathOperator{\sqplus}{\text{\tikz[scale=.6ex/1cm,baseline=-.6ex,line width=.07ex]{\draw (-1,1) -- (-1,-1) -- (1,-1) -- (1,1) (0,-.7) -- (0,.7) (-.7,0) -- (.7,0);}}}
\newcommand{\restrict}{\upharpoonright} % uses amssymb
%\newcommand{\restrict}{\mathbin{\hbox{\msam\char'26}}}
\newcommand{\satisfies}{\models}
\newcommand{\forces}{\Vdash}
\newcommand{\proves}{\vdash}
%\newcommand{\possible}{\mathop{\raisebox{-1pt}{$\Diamond$}}}
%\newcommand{\necessary}{\mathop{\raisebox{-1pt}{$\Box$}}}
%\newcommand{\necessary}{\mathop{\raisebox{3pt}{\framebox[6pt]{}}}}
\newcommand{\necessaryprop}{\necessary_{\hbox{\scriptsize\prop}}}
\newcommand{\possibleprop}{\possible_{\hbox{\scriptsize\,\prop}}}
\newcommand{\necessaryccc}{\necessary_{\hbox{\scriptsize\ccc}}}
\newcommand{\possibleccc}{\possible_{\hbox{\scriptsize\ccc}}}
\newcommand{\necessarycohen}{\necessary_{\hbox{\scriptsize\cohen}}}
\newcommand{\possiblecohen}{\possible_{\hbox{\scriptsize\cohen}}}
\newcommand{\modalscale}{\mathchoice{.72ex/1cm}{.6ex/1cm}{.5ex/1cm}{.4ex/1cm}} % not used?
\DeclareMathOperator{\possible}{\text{\tikz[scale=.6ex/1cm,baseline=-.6ex,rotate=45,line width=.1ex]{\draw (-1,-1) rectangle (1,1);}}}
\DeclareMathOperator{\necessary}{\text{\tikz[scale=.6ex/1cm,baseline=-.6ex,line width=.1ex]{\draw (-1,-1) rectangle (1,1);}}}
\DeclareMathOperator{\downpossible}{\text{\tikz[scale=.6ex/1cm,baseline=-.6ex,rotate=45,line width=.1ex]{
                            \draw (-1,-1) rectangle (1,1); \draw[very thin] (-.6,1) -- (-.6,-.6) -- (1,-.6);}}}
\DeclareMathOperator{\downnecessary}{\text{\tikz[scale=.6ex/1cm,baseline=-.6ex,line width=.1ex]{
                            \draw (-1,-1) rectangle (1,1); \draw[very thin] (-1,-.6) -- (.6,-.6) -- (.6,1);}}}
\DeclareMathOperator{\uppossible}{\text{\tikz[scale=.6ex/1cm,baseline=-.6ex,rotate=45,line width=.1ex]{
                            \draw (-1,-1) rectangle (1,1); \draw[very thin] (-1,.6) -- (.6,.6) -- (.6,-1);}}}
\DeclareMathOperator{\upnecessary}{\text{\tikz[scale=.6ex/1cm,baseline=-.6ex,line width=.1ex]{
                            \draw (-1,-1) rectangle (1,1); \draw[very thin] (-1,.6) -- (.6,.6) -- (.6,-1);}}}
                                                                % old: (-1,0) -- (0,.5) -- (1,0);}}
\DeclareMathOperator{\xpossible}{\text{\tikz[scale=.6ex/1cm,baseline=-.6ex,rotate=45,line width=.1ex]{
                            \draw (-1,-1) rectangle (1,1); \draw[very thin] (-.6,-.6) rectangle (.6,.6);}}}
\DeclareMathOperator{\xnecessary}{\text{\tikz[scale=.6ex/1cm,baseline=-.6ex,line width=.1ex]{
                            \draw (-1,-1) rectangle (1,1); \draw[very thin] (-.6,-.6) rectangle (.6,.6);}}}
\DeclareMathOperator{\Luppossible}{\text{\tikz[scale=.6ex/1cm,baseline=-.6ex,rotate=45,line width=.1ex]{
                            \draw (-1,-1) rectangle (1,1); \draw[very thin,line join=bevel] (.6,1) -- (-1,-1) -- (1,.6);}}}
\DeclareMathOperator{\Lupnecessary}{\text{\tikz[scale=.6ex/1cm,baseline=-.6ex,line width=.1ex]{
                            \draw (-1,-1) rectangle (1,1); \draw[very thin,line join=bevel] (-.25,1) -- (0,-1) -- (.25,1);}}}
\DeclareMathOperator{\soliduppossible}{\text{\tikz[scale=.6ex/1cm,baseline=-.6ex,rotate=45,line width=.1ex]{
                            \draw (-1,-1) rectangle (1,1); \draw[very thin,fill=gray,fill opacity=.25] (-1,-1) rectangle (.6,.6);}}}
\DeclareMathOperator{\solidupnecessary}{\text{\tikz[scale=.6ex/1cm,baseline=-.6ex,line width=.1ex]{
                            \draw (-1,-1) rectangle (1,1); \draw[very thin,fill=gray,fill opacity=.25] (-1,-1) rectangle (.6,.6);}}}
\DeclareMathOperator{\solidxpossible}{\text{\tikz[scale=.6ex/1cm,baseline=-.6ex,rotate=45,line width=.1ex]{
                            \draw (-1,-1) rectangle (1,1); \draw[very thin,fill=gray,fill opacity=.25] (-.6,-.6) rectangle (.6,.6);}}}
\DeclareMathOperator{\solidxnecessary}{\text{\tikz[scale=.6ex/1cm,baseline=-.6ex,line width=.1ex]{
                            \draw (-1,-1) rectangle (1,1); \draw[very thin,fill=gray,fill opacity=.25] (-.6,-.6) rectangle (.6,.6);}}}
\DeclareMathOperator{\consuppossible}{\text{\tikz[scale=.6ex/1cm,baseline=-.6ex,rotate=45,line width=.1ex]{
                            \draw (-1,-1) rectangle (1,1); \draw[very thin] (-1,-1) rectangle (.6,.6);
                            \clip (-1,-1) rectangle (.6,.6); %\draw[fill] (.3,.3) rectangle (.6,.6);
                            \draw[very thin] (-1,-1) -- (.6,.6);}}}
\DeclareMathOperator{\consupnecessary}{\text{\tikz[scale=.6ex/1cm,baseline=-.6ex,line width=.1ex]{
                            \draw (-1,-1) rectangle (1,1); \draw[very thin] (-1,-1) rectangle (.6,.6);
                            \clip (-1,-1) rectangle (.6,.6); %\draw[fill] (.3,.3) rectangle (.6,.6);
                            \draw[very thin] (-1,-1) -- (.6,.6);}}}
\DeclareMathOperator{\consxpossible}{\text{\tikz[scale=.6ex/1cm,baseline=-.6ex,rotate=45,line width=.1ex]{
                            \draw (-1,-1) rectangle (1,1); \draw[very thin] (-.6,-.6) rectangle (.6,.6);
                            \clip (-1,-1) rectangle (.6,.6); %\draw[fill] (.3,.3) rectangle (.6,.6);
                            \draw[very thin] (-.6,-.6) -- (.6,.6);}}}
\DeclareMathOperator{\consxnecessary}{\text{\tikz[scale=.6ex/1cm,baseline=-.6ex,line width=.1ex]{
                            \draw (-1,-1) rectangle (1,1); \draw[very thin] (-.6,-.6) rectangle (.6,.6);
                            \clip (-.6,-.6) rectangle (.6,.6); %\draw[fill] (.3,.3) rectangle (.6,.6);
                            \draw[very thin] (-.6,-.6) -- (.6,.6);}}}
\DeclareMathOperator{\solidconsuppossible}{\text{\tikz[scale=.6ex/1cm,baseline=-.6ex,rotate=45,line width=.1ex]{
                            \draw (-1,-1) rectangle (1,1); \draw[very thin,fill=gray,fill opacity=.25] (-1,-1) rectangle (.6,.6);
                            \draw[very thin] (-1,-1) rectangle (.6,.6);
                            \clip (-1,-1) rectangle (.6,.6); %\draw[fill] (.3,.3) rectangle (.6,.6);
                            \draw[very thin] (-1,-1) -- (.6,.6);}}}
\DeclareMathOperator{\solidconsupnecessary}{\text{\tikz[scale=.6ex/1cm,baseline=-.6ex,line width=.1ex]{
                            \draw (-1,-1) rectangle (1,1); \draw[very thin,fill=gray,fill opacity=.25] (-1,-1) rectangle (.6,.6);
                            \draw[very thin] (-1,-1) rectangle (.6,.6);
                            \clip (-1,-1) rectangle (.6,.6); %\draw[fill] (.3,.3) rectangle (.6,.6);
                            \draw[very thin] (-1,-1) -- (.6,.6);}}}
\DeclareMathOperator{\solidconsxpossible}{\text{\tikz[scale=.6ex/1cm,baseline=-.6ex,rotate=45,line width=.1ex]{
                            \draw (-1,-1) rectangle (1,1); \draw[very thin,fill=gray,fill opacity=.25] (-.6,-.6) rectangle (.6,.6);
                            \draw[very thin] (-.6,-.6) rectangle (.6,.6);
                            \clip (-1,-1) rectangle (.6,.6); %\draw[fill] (.3,.3) rectangle (.6,.6);
                            \draw[very thin] (-.6,-.6) -- (.6,.6);}}}
\DeclareMathOperator{\solidconsxnecessary}{\text{\tikz[scale=.6ex/1cm,baseline=-.6ex,line width=.1ex]{
                            \draw (-1,-1) rectangle (1,1); \draw[very thin,fill=gray,fill opacity=.25] (-.6,-.6) rectangle (.6,.6);
                            \draw[very thin] (-.6,-.6) rectangle (.6,.6);
                            \clip (-1,-1) rectangle (.6,.6); %\draw[fill] (.3,.3) rectangle (.6,.6);
                            \draw[very thin] (-.6,-.6) -- (.6,.6);}}}
\DeclareMathOperator{\ssypossible}{\text{\tikz[scale=.6ex/1cm,baseline=-.6ex,rotate=45,line width=.1ex]{
                            \draw (-1,-1) rectangle (1,1); \draw[very thin] (-1,-1) rectangle (.6,.6);
                            \draw[very thin] (-.6,-.6) rectangle (.2,.2);
                            }}}
\DeclareMathOperator{\ssynecessary}{\text{\tikz[scale=.6ex/1cm,baseline=-.6ex,line width=.1ex]{
                            \draw (-1,-1) rectangle (1,1); \draw[very thin] (-1,-1) rectangle (.6,.6);
                            \draw[very thin] (-.6,-.6) rectangle (.2,.2);
                            }}}
\DeclareMathOperator{\solidssypossible}{\text{\tikz[scale=.6ex/1cm,baseline=-.6ex,rotate=45,line width=.1ex]{
                            \draw (-1,-1) rectangle (1,1); \draw[very thin] (-1,-1) rectangle (.6,.6);
                            \draw[very thin,fill=gray,fill opacity=.25] (-.6,-.6) rectangle (.2,.2);
                            }}}
\DeclareMathOperator{\solidssynecessary}{\text{\tikz[scale=.6ex/1cm,baseline=-.6ex,line width=.1ex]{
                            \draw (-1,-1) rectangle (1,1); \draw[very thin] (-1,-1) rectangle (.6,.6);
                            \draw[very thin,fill=gray,fill opacity=.25] (-.6,-.6) rectangle (.2,.2);
                            }}}
\DeclareMathOperator{\ipossible}{\text{\tikz[scale=.6ex/1cm,baseline=-.6ex,rotate=45,line width=.1ex]{
                            \draw (-1,-1) rectangle (1,1); \draw[very thin] (-1,-1) rectangle (.6,.6);
                            \draw[very thin] (.2,-1) arc (0:90:1.2);
                            }}}
\DeclareMathOperator{\inecessary}{\text{\tikz[scale=.6ex/1cm,baseline=-.6ex,line width=.1ex]{
                            \draw (-1,-1) rectangle (1,1); \draw[very thin] (-1,-1) rectangle (.6,.6);
                            \draw[very thin] (.2,-1) arc (0:90:1.2);
                            }}}
\DeclareMathOperator{\solidipossible}{\text{\tikz[scale=.6ex/1cm,baseline=-.6ex,rotate=45,line width=.1ex]{
                            \draw (-1,-1) rectangle (1,1); \draw[very thin] (-1,-1) rectangle (.6,.6);
                            \draw[very thin,fill=gray,fill opacity=.25] (-1,-1) -- (.2,-1) arc (0:90:1.2) -- cycle;
                            }}}
\DeclareMathOperator{\solidinecessary}{\text{\tikz[scale=.6ex/1cm,baseline=-.6ex,line width=.1ex]{
                            \draw (-1,-1) rectangle (1,1); \draw[very thin] (-1,-1) rectangle (.6,.6);
                            \draw[very thin,fill=gray,fill opacity=.25] (-1,-1) -- (.2,-1) arc (0:90:1.2) -- cycle ;
                            }}}
\DeclareMathOperator{\embedpossible}{\text{\tikz[scale=.6ex/1cm,baseline=-.6ex,rotate=45,line width=.1ex]{\draw (-1,-1) rectangle (1,1);
    \draw (-1,.67) to[looseness=2.2,out=0,in=180] (1,-.67);}}}
\DeclareMathOperator{\embednecessary}{\text{\tikz[scale=.6ex/1cm,baseline=-.6ex,line width=.1ex]{\draw (-1,-1) rectangle (1,1);
    \draw (-1,0) to[looseness=2,out=40,in=220] (1,0);}}}
% not used:
\DeclareMathOperator{\tieuppossible}{\text{\tikz[scale=.6ex/1cm,baseline=-.6ex,rotate=45,line width=.1ex]{
                            \draw (-1,-1) rectangle (1,1); \draw[very thin] (-1,-1) rectangle (.6,.6);
                            \draw[very thin] (-1,-1) grid (.6,.6);}}}
\DeclareMathOperator{\tieupnecessary}{\text{\tikz[scale=.6ex/1cm,baseline=-.6ex,line width=.1ex]{
                            \draw (-1,-1) rectangle (1,1); \draw[very thin] (-1,-1) rectangle (.6,.6);
                            \draw[very thin] (-1,-1) grid (.6,.6);}}}
%
\newcommand\dbrace{\hskip-1.5em\raise3pt\hbox{\rotatebox[origin=c]{-35}{$\left.\strut^{\phantom{|}}\right\}$}}}% useful for tetration
\newcommand\UParrow{\mathrel{\mathchoice{\UParroW}{\UParroW}{\scriptsize\UParroW}{\tiny\UParroW}}}
\newcommand\UParroW{{\setbox0\hbox{$\Uparrow$}\rlap{\hbox to \wd0{\hss$\mid$\hss}}\box0}}
%\newcommand\UPArrow{{\setbox0\hbox{$\Uparrow$}\rlap{\hbox to\wd0{\hss$\mid\ \mid$\hss}}\box0}}
\newcommand{\axiomf}[1]{{\rm #1}}
\newcommand{\theoryf}[1]{{\rm #1}}% {\hbox{$\mathsf{#1}$}}
\newcommand{\Force}{\mathrm{Force}}
\newcommand{\Mantle}{{\mathord{\rm M}}}
\newcommand{\gMantle}{{\mathord{\rm gM}}}  % the generic mantle
\newcommand{\gHOD}{\ensuremath{\mathord{{\rm g}\HOD}}} % the limit HOD
\newcommand{\cross}{\times}
\newcommand{\concat}{\mathbin{{}^\smallfrown}}
\newcommand{\converges}{{\downarrow}}
\newcommand{\diverges}{\uparrow}
\renewcommand{\setminus}{\raise.3ex\hbox{\rotatebox{-20}{$-$}}} % the usual setminus is absurdly huge and vertical
\newcommand{\union}{\cup}
\renewcommand{\emptyset}{\varnothing}
\newcommand{\squnion}{\sqcup}
\newcommand{\Union}{\bigcup}
\newcommand{\sqUnion}{\bigsqcup}
\newcommand{\intersect}{\cap}
\newcommand{\Intersect}{\bigcap}
\newcommand{\Pforces}{\forces_{\P}}
\newcommand{\into}{\hookrightarrow}
\newcommand{\meet}{\wedge}
\newcommand{\join}{\cup}
\newcommand{\trianglelt}{\lhd}
\newcommand{\nottrianglelt}{\ntriangleleft}
\newcommand{\tlt}{\triangle}
\newcommand{\LaverDiamond}{\mathop{\hbox{\line(0,1){10}\line(1,0){8}\line(-4,5){8}}\hskip 1pt}\nolimits}
\newcommand{\LD}{\LaverDiamond}
\newcommand{\LDLD}{\mathop{\hbox{\,\line(0,1){8}\!\line(0,1){10}\line(1,0){8}\line(-4,5){8}}\hskip 1pt}\nolimits}
\newcommand{\LDminus}{\mathop{\LaverDiamond^{\hbox{\!\!-}}}}
\newcommand{\LDwc}{\LD^{\hbox{\!\!\tiny wc}}}
\newcommand{\LDuplift}{\LD^{\hbox{\!\!\tiny uplift}}}
\newcommand{\LDunf}{\LD^{\hbox{\!\!\tiny unf}}}
\newcommand{\LDind}{\LD^{\hbox{\!\!\tiny ind}}}
\newcommand{\LDthetaunf}{\LD^{\hbox{\!\!\tiny $\theta$-unf}}}
\newcommand{\LDsunf}{\LD^{\hbox{\!\!\tiny sunf}}}
\newcommand{\LDthetasunf}{\LD^{\hbox{\!\!\tiny $\theta$-sunf}}}
\newcommand{\LDmeas}{\LD^{\hbox{\!\!\tiny meas}}}
\newcommand{\LDstr}{\LD^{\hbox{\!\!\tiny strong}}}
\newcommand{\LDsuperstrong}{\LD^{\hbox{\!\!\tiny superstrong}}}
\newcommand{\LDram}{\LD^{\hbox{\!\!\tiny Ramsey}}}
\newcommand{\LDstrc}{\LD^{\hbox{\!\!\tiny str compact}}}
\newcommand{\LDsc}{\LD^{\hbox{\!\!\tiny sc}}}
\newcommand{\LDext}{\LD^{\hbox{\!\!\tiny ext}}}
\newcommand{\LDahuge}{\LD^{\hbox{\!\!\tiny ahuge}}}
\newcommand{\LDlambdaahuge}{\LD^{\hbox{\!\!\tiny $\lambda$-ahuge}}}
\newcommand{\LDsahuge}{\LD^{\hbox{\!\!\tiny super ahuge}}}
\newcommand{\LDhuge}{\LD^{\hbox{\!\!\tiny huge}}}
\newcommand{\LDlambdahuge}{\LD^{\hbox{\!\!\tiny $\lambda$-huge}}}
\newcommand{\LDshuge}{\LD^{\hbox{\!\!\tiny superhuge}}}
\newcommand{\LDnhuge}{\LD^{\hbox{\!\!\tiny $n$-huge}}}
\newcommand{\LDlambdanhuge}{\LD^{\hbox{\!\!\tiny $\lambda$ $n$-huge}}}
\newcommand{\LDsnhuge}{\LD^{\hbox{\!\!\tiny super $n$-huge}}}
\newcommand{\LDthetasc}{\LD^{\hbox{\!\!\tiny $\theta$-sc}}}
\newcommand{\LDkappasc}{\LD^{\hbox{\!\!\tiny $\kappa$-sc}}}
\newcommand{\LDthetastr}{\LD^{\hbox{\!\!\tiny $\theta$-strong}}}
\newcommand{\LDthetastrc}{\LD^{\hbox{\!\!\tiny $\theta$-str compact}}}
\newcommand{\LDstar}{\LD^{\hbox{\!\!\tiny$\star$}}}
%\newcommand{\LaverDiamond}{\mathop{\hbox{\line(0,1){8}\line(1,0){6}\line(-3,4){8}}\hskip 1pt}\nolimits}
\newcommand{\smalllt}{\mathrel{\mathchoice{\raise2pt\hbox{$\scriptstyle<$}}{\raise1pt\hbox{$\scriptstyle<$}}{\raise0pt\hbox{$\scriptscriptstyle<$}}{\scriptscriptstyle<}}}
\newcommand{\smallleq}{\mathrel{\mathchoice{\raise2pt\hbox{$\scriptstyle\leq$}}{\raise1pt\hbox{$\scriptstyle\leq$}}{\raise1pt\hbox{$\scriptscriptstyle\leq$}}{\scriptscriptstyle\leq}}}
\newcommand{\lt}{\smalllt}
\newcommand{\ltomega}{{{\smalllt}\omega}}
\newcommand{\leqomega}{{{\smallleq}\omega}}
\newcommand{\ltkappa}{{{\smalllt}\kappa}}
\newcommand{\leqkappa}{{{\smallleq}\kappa}}
\newcommand{\ltalpha}{{{\smalllt}\alpha}}
\newcommand{\leqalpha}{{{\smallleq}\alpha}}
\newcommand{\leqgamma}{{{\smallleq}\gamma}}
\newcommand{\leqlambda}{{{\smallleq}\lambda}}
\newcommand{\ltlambda}{{{\smalllt}\lambda}}
\newcommand{\ltgamma}{{{\smalllt}\gamma}}
\newcommand{\leqeta}{{{\smallleq}\eta}}
\newcommand{\lteta}{{{\smalllt}\eta}}
\newcommand{\leqxi}{{{\smallleq}\xi}}
\newcommand{\ltxi}{{{\smalllt}\xi}}
\newcommand{\leqzeta}{{{\smallleq}\zeta}}
\newcommand{\ltzeta}{{{\smalllt}\zeta}}
\newcommand{\leqtheta}{{{\smallleq}\theta}}
\newcommand{\lttheta}{{{\smalllt}\theta}}
\newcommand{\leqbeta}{{{\smallleq}\beta}}
\newcommand{\leqdelta}{{{\smallleq}\delta}}
\newcommand{\ltdelta}{{{\smalllt}\delta}}
\newcommand{\ltbeta}{{{\smalllt}\beta}}
\newcommand{\leqSigma}{{{\smallleq}\Sigma}}
\newcommand{\ltSigma}{{{\smalllt}\Sigma}}
% closed square root symbol  - search online to fix issue with cube roots etc.
\def\sqsqrt{\mathpalette\DHLhksqrt}
   \def\DHLhksqrt#1#2{%
   \setbox0=\hbox{$#1\sqrt{#2\,}$}\dimen0=\ht0
   \advance\dimen0-0.2\ht0
   \setbox2=\hbox{\vrule height\ht0 depth -\dimen0}%
   {\box0\lower0.4pt\box2}}
%
\newcommand{\Card}[1]{{\left|#1\right|}}
%\newcommand{\card}[1]{{|#1|}}
\newcommand{\boolval}[1]{\mathopen{\lbrack\!\lbrack}\,#1\,\mathclose{\rbrack\!\rbrack}}
\def\[#1]{\mathopen{\lbrack\!\lbrack}#1\mathclose{\rbrack\!\rbrack}}
%\newcommand{\gcode}[1]{{}^\ulcorner\!#1\!{}^\urcorner}
%\newcommand{\gcode}[1]{\ulcorner\!#1\!\urcorner}
% Adapted from Sam Buss's macro for Goedel codes:
\newbox\gnBoxA
\newbox\gnBoxB
\newdimen\gnCornerHgt
\setbox\gnBoxA=\hbox{\tiny$\ulcorner$}
\global\gnCornerHgt=\ht\gnBoxA
\newdimen\gnArgHgt
\def\gcode #1{%
\setbox\gnBoxA=\hbox{$#1$}%
\setbox\gnBoxB=\hbox{$\bar #1$}%
\gnArgHgt=\ht\gnBoxB%
\ifnum     \gnArgHgt<\gnCornerHgt \gnArgHgt=0pt%
\else \advance \gnArgHgt by -\gnCornerHgt%
\fi \raise\gnArgHgt\hbox{\tiny$\ulcorner$} \box\gnBoxA %
\raise\gnArgHgt\hbox{\tiny$\urcorner$}}
%
\newcommand{\UnderTilde}[1]{{\setbox1=\hbox{$#1$}\baselineskip=0pt\vtop{\hbox{$#1$}\hbox to\wd1{\hfil$\sim$\hfil}}}{}}
\newcommand{\Undertilde}[1]{{\setbox1=\hbox{$#1$}\baselineskip=0pt\vtop{\hbox{$#1$}\hbox to\wd1{\hfil$\scriptstyle\sim$\hfil}}}{}}
\newcommand{\undertilde}[1]{{\setbox1=\hbox{$#1$}\baselineskip=0pt\vtop{\hbox{$#1$}\hbox to\wd1{\hfil$\scriptscriptstyle\sim$\hfil}}}{}}
\newcommand{\UnderdTilde}[1]{{\setbox1=\hbox{$#1$}\baselineskip=0pt\vtop{\hbox{$#1$}\hbox to\wd1{\hfil$\approx$\hfil}}}{}}
\newcommand{\Underdtilde}[1]{{\setbox1=\hbox{$#1$}\baselineskip=0pt\vtop{\hbox{$#1$}\hbox to\wd1{\hfil\scriptsize$\approx$\hfil}}}{}}
\newcommand{\underdtilde}[1]{{\baselineskip=0pt\vtop{\hbox{$#1$}\hbox{\hfil$\scriptscriptstyle\approx$\hfil}}}{}}
\newcommand{\varampersand}{{\usefont{OT1}{cmr}{m}{it} \&}}
\newcommand{\fancyampersand}{{\usefont{OT1}{cmr}{m}{it} \&}}
\newcommand{\st}{\mid}
\renewcommand{\th}{{\hbox{\scriptsize th}}}
\renewcommand{\implies}{\mathrel{\rightarrow}}
\newcommand{\Implies}{\mathrel{\Rightarrow}}
\newcommand{\nimplies}{\mathrel{\nrightarrow}}
\renewcommand{\iff}{\mathrel{\leftrightarrow}}
\newcommand{\Iff}{\mathrel{\Longleftrightarrow}}
\newcommand\bottom{\perp}
\newcommand{\minus}{\setminus}
\newcommand{\iso}{\cong}
\def\<#1>{\left\langle#1\right\rangle}
\newcommand{\ot}{\mathop{\rm ot}\nolimits}
\newcommand{\val}{\mathop{\rm val}\nolimits}
\newcommand{\QEDbox}{\fbox{}}
\newcommand{\cp}{\mathop{\rm cp}}
\newcommand{\TC}{\mathop{{\rm TC}}}
\newcommand{\Tr}{\mathop{\rm Tr}\nolimits}
\newcommand{\RO}{\mathop{{\rm RO}}}
\newcommand{\ORD}{\mathord{{\rm Ord}}}
\newcommand{\Ord}{\mathord{{\rm Ord}}}
\newcommand{\No}{\mathord{{\rm No}}} % surreals
\newcommand{\Oz}{\mathord{{\rm Oz}}} % omnific integers
\newcommand{\REG}{\mathord{{\rm REG}}}
\newcommand{\SING}{\mathord{{\rm SING}}}
\newcommand{\Lim}{\mathop{\hbox{\rm Lim}}}
\newcommand{\COF}{\mathop{{\rm COF}}}
\newcommand{\INACC}{\mathop{{\rm INACC}}}
%\newcommand{\CCC}{\mathop{{\rm CCC}}}
\newcommand{\CARD}{\mathop{{\rm CARD}}}
\newcommand{\ETR}{{\rm ETR}}
\newcommand\ETRord{\ETR_{\Ord}}
\newcommand\ACA{{\rm ACA}}
\newcommand\RCA{{\rm RCA}}
\newcommand{\ATR}{{\rm ATR}}
\newcommand{\WO}{\mathord{{\rm WO}}}
\newcommand{\WF}{\mathord{{\rm WF}}}
\newcommand\WKL{\mathord{{\rm WKL}}}
\newcommand{\HC}{\mathop{{\rm HC}}}
\newcommand{\LC}{{\rm LC}}
\newcommand{\ZFC}{{\mathsf{ZFC}}}
\newcommand{\ZF}{{\mathsf{ZF}}}
\newcommand\ZFCfin{\ZFC^{\neg\infty}}
\newcommand\ZFfin{\ZF^{\neg\infty}}
\newcommand{\ZFm}{\ZF^-}
\newcommand{\ZFCm}{\ZFC^-}
\newcommand{\ZFCmm}{\ZFC\text{\tt -}}%{\ZFC{\text{\Large\bf\tt -}}}
\newcommand{\ZFA}{{\rm ZFA}}
\newcommand{\ZFCU}{{\rm ZFCU}}
\newcommand{\ZFU}{{\rm ZFU}}
\newcommand{\ZC}{{\rm ZC}}
\newcommand{\KM}{{\rm KM}}
\newcommand{\GB}{{\rm GB}}
\newcommand{\GBC}{{\rm GBC}}
\newcommand{\GBc}{{\rm GBc}}
\newcommand{\NGBC}{{\rm NGBC}}
\newcommand{\NGB}{{\rm NGB}}
\newcommand{\GBCU}{{\rm GBCU}}
\newcommand{\AAA}{{\rm AAA}}% {\textsc{aaa}}
\newcommand{\CCA}{{\rm CCA}}
\newcommand{\CH}{{\rm CH}}
\newcommand{\Ch}{{\mathfrak{Ch}}}
\newcommand{\KP}{{\rm KP}}
\newcommand{\SH}{{\rm SH}}
\newcommand{\GCH}{{\rm GCH}}
\newcommand{\SCH}{{\rm SCH}}
\newcommand{\AC}{{\rm AC}}
\newcommand{\DC}{{\mathsf{DC}}}
\newcommand{\CC}{{\rm CC}}
\newcommand{\ECC}{{\rm ECC}}
\newcommand{\AD}{{\mathsf{AD}}}
\newcommand{\AFA}{{\rm AFA}}
\newcommand{\BAFA}{{\rm BAFA}}
\newcommand{\FAFA}{{\rm FAFA}}
\newcommand{\SAFA}{{\rm SAFA}}
\newcommand{\AS}{{\rm AS}}
\newcommand{\RR}{{\rm RR}}
\newcommand{\NP}{\mathop{\hbox{\it NP}}\nolimits}
\newcommand{\coNP}{\mathop{\hbox{\rm co-\!\it NP}}\nolimits}
\newcommand{\PD}{{\rm PD}}
\newcommand{\MA}{{\rm MA}}
\newcommand{\GA}{{\rm GA}}
\newcommand{\DFA}{{\rm DFA}}
\newcommand{\bDFA}{\utilde{\DFA}} % requires \usepackage{undertilde}
\newcommand{\LFA}{{\rm LFA}}
\newcommand{\DDG}{{\rm DDG}}
\newcommand{\SDDG}{{\rm SDDG}}
\newcommand{\RA}{{\rm RA}}
\newcommand{\UA}{{\rm UA}}
\newcommand{\GAccc}{{\GA_{\hbox{\scriptsize\rm ccc}}}}
\newcommand{\BA}{{\rm BA}}
\newcommand{\GEA}{{\rm GEA}}
\newcommand{\MM}{{\rm MM}}
\newcommand{\BMM}{{\rm BMM}}
\newcommand{\KH}{{\rm KH}}
\newcommand{\PFA}{{\rm PFA}}
\newcommand{\PFAcproper}{{\rm PFA}(\mathfrak{c}\text{-proper})}
\newcommand{\BPFA}{{\rm BPFA}}
\newcommand{\SPFA}{{\rm SPFA}}
\newcommand{\BSPFA}{{\rm BSPFA}}
\newcommand{\WA}{{\rm WA}}
\newcommand{\MP}{{\rm MP}}
\newcommand{\HOD}{{\rm HOD}}
\newcommand{\OD}{{\rm OD}}
\newcommand{\HOA}{{\rm HOA}}
\newcommand{\HF}{{\rm HF}}
\newcommand{\IMH}{{\rm IMH}}
\newcommand{\MPtilde}{\UnderTilde{\MP}}
\newcommand{\MPccc}{{\MP_{\hbox{\scriptsize\!\rm ccc}}}}
\newcommand{\MPproper}{{\MP_{\hbox{\scriptsize\!\rm proper}}}}
\newcommand{\MPprop}{{\MP_{\hbox{\scriptsize\!\rm prop}}}}
\newcommand{\MPsp}{{\MP_{\hbox{\scriptsize\!\rm semiproper}}}}
\newcommand{\MPsprop}{{\MP_{\scriptsize\!\sprop}}}
\newcommand{\MPstat}{{\MP_{\scriptsize\!{\rm stat}}}}
\newcommand{\MPcard}{{\MP_{\scriptsize\!{\rm card}}}}
\newcommand{\MPcohen}{{\MP_{\scriptsize\!\cohen}}}
\newcommand{\MPcoll}{{\MP_{\scriptsize\!\COLL}}}
\newcommand{\MPdist}{{\MP_{\scriptsize\!{\rm dist}}}}
\newcommand{\MPomegaOne}{{\MP_{\omega_1}}}
\newcommand{\MPcof}{{\MP_{\scriptsize\!{\rm cof}}}}
\newcommand\VP{{\rm VP}}
\newcommand\VS{{\rm VS}}
\newcommand{\prop}{{{\rm prop}}}
\newcommand{\cproper}{{\continuum\text{\rm-proper}}}
\newcommand{\cpluspreserving}{{\continuum^\plus\text{\rm-preserving}}}
\newcommand{\cpluscovering}{{\continuum^\plus\text{\rm-covering}}}
\newcommand{\sizec}{{\text{\rm size }\continuum}}
\newcommand{\card}{{{\rm card}}}
\newcommand{\sprop}{{{\rm sprop}}}
\newcommand{\stat}{{\rm stat}}
\newcommand{\dist}{{\rm dist}}
\newcommand{\ccc}{{{\rm ccc}}}
\newcommand{\cohen}{{{\rm cohen}}}
\newcommand{\COLL}{{{\rm COLL}}}
\newcommand{\PA}{{\rm PA}}
\newcommand{\PRA}{{\rm PRA}}
\newcommand{\TA}{{\rm TA}}
\newcommand{\inacc}{{\rm inacc}}
\newcommand{\omegaCK}{{\omega_1^{\hbox{\tiny\sc CK}}}}
%
% macros for ITTMs:
%
\def\col#1#2#3{\hbox{\vbox{\baselineskip=0pt\parskip=0pt\cell#1\cell#2\cell#3}}}
\newcommand{\cell}[1]{\boxit{\hbox to 17pt{\strut\hfil$#1$\hfil}}}
\newcommand{\head}[2]{\lower2pt\vbox{\hbox{\strut\footnotesize\it\hskip3pt#2}\boxit{\cell#1}}}
\newcommand{\boxit}[1]{\setbox4=\hbox{\kern2pt#1\kern2pt}\hbox{\vrule\vbox{\hrule\kern2pt\box4\kern2pt\hrule}\vrule}}
\newcommand{\Col}[3]{\hbox{\vbox{\baselineskip=0pt\parskip=0pt\cell#1\cell#2\cell#3}}}
\newcommand{\tapenames}{\raise 5pt\vbox to .7in{\hbox to .8in{\it\hfill input: \strut}\vfill\hbox to
.8in{\it\hfill scratch: \strut}\vfill\hbox to .8in{\it\hfill output: \strut}}}
\newcommand{\Head}[4]{\lower2pt\vbox{\hbox to25pt{\strut\footnotesize\it\hfill#4\hfill}\boxit{\Col#1#2#3}}}
\newcommand{\Dots}{\raise 5pt\vbox to .7in{\hbox{\ $\cdots$\strut}\vfill\hbox{\ $\cdots$\strut}\vfill\hbox{\
$\cdots$\strut}}}
%\renewcommand{\dots}{\raise5pt\hbox{\ $\cdots$}}
%
%
%
% macros used for the organization of mathematical articles:
%
\newcommand{\df}{\it} % use italic for definition terms. Idea: also use this to create an index of definitions, if MakeIndex is true.
%
\hyphenation{su-per-com-pact-ness}
\hyphenation{La-ver}%\hyphenation{approxi-ma-tion}
\hyphenation{anti-ci-pat-ing}
