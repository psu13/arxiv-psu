\pdfpagewidth=5in
\pdfpageheight=7in

\hsize=4in
\hoffset=-.5in

\input font_palatino
\baselineskip=13pt
\pdffontexpand\tenrm 30 20 10 autoexpand
\pdfadjustspacing=2
\frenchspacing

When this subject happens to come up, I have a simple rule of thumb that I offer to my fellow scribblers. I call it the Dostoyevsky test: If you have reason to believe that you're another Dostoyevsky, there is no reason to be concerned about the effect what you write might have on the life of some member of your family. Your art is considerably more important than any such consideration. Readers a century or two from now should not be deprived of the prose you fashion out of, say, the circumstances leading to your conclusion that your oldest son simply didn't have the guts to stick to junior-varsity football and thus set a pattern for a life of drifting. You have the right --- the responsibility, really, to future generations of readers --- to mention your mother's private confession to you, in a moment of stress, that she never truly loved your father, even if putting that confession in print causes some awkwardness in their life at the retirement village. If you have reason to believe that you're another Dostoyevsky, you can say anything you need to say. If you don't have reason to believe that you're another Dostoyevsky, you can't.

\end
